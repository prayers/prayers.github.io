\NormalFont\ShortTitle{罗马书}
{\MT 罗马书

\par }\ChapOne{1}{\SH 祝福
\par }{\PP \VerseOne{1}耶稣基督的仆人{\PN{保罗}},奉召为使徒,特派传 神的福音。
\VS{2}这福音是 神从前借众先知在圣经上所应许的,
\VS{3}论到他儿子—我主耶稣基督。按肉体说,是从{\PN{大卫}}后裔生的;
\VS{4}按圣善的灵说,因从死里复活,以大能显明是 神的儿子。
\VS{5}我们从他受了恩惠并使徒的职分,在万国之中叫人为他的名信服真道;
\VS{6}其中也有你们这蒙召属耶稣基督的人。
\VS{7}我写信给你们在{\PN{罗马}}、为 神所爱、奉召作圣徒的众人。愿恩惠、平安从我们的父 神并主耶稣基督归与你们!
\par }{\SH 保罗有意访问罗马
\par }{\PP \VS{8}第一,我靠着耶稣基督,为你们众人感谢我的 神,因你们的信德传遍了天下。
\VS{9}我在他儿子福音上,用心灵所事奉的 神可以见证,我怎样不住地提到你们;
\VS{10}在祷告之间常常恳求,或者照 神的旨意,终能得平坦的道路往你们那里去。
\VS{11}因为我切切地想见你们,要把些属灵的恩赐分给你们,使你们可以坚固;
\VS{12}这样,我在你们中间,因你与我彼此的信心,就可以同得安慰。
\VS{13}弟兄们,我不愿意你们不知道,我屡次定意往你们那里去,要在你们中间得些果子,如同在其余的外邦人中一样;只是到如今仍有阻隔。
\VS{14}无论是{\PN{希腊}}人、化外人、聪明人、愚拙人,我都欠他们的债,
\VS{15}所以情愿尽我的力量,将福音也传给你们在{\PN{罗马}}的人。
\par }{\SH 福音是 神的大能
\par }{\PP \VS{16}我不以福音为耻;这福音本是 神的大能,要救一切相信的,先是{\PN{犹太}}人,后是{\PN{希腊}}人。
\VS{17}因为 神的义正在这福音上显明出来;这义是本于信,以至于信。如{\ADD{经上}}所记:「义人必因信得生。」
\par }{\SH 人类的罪恶
\par }{\PP \VS{18}原来, 神的忿怒从天上显明在一切不虔不义的人身上,就是那些行不义阻挡真理的人。
\VS{19}神的事情,人所能知道的,原显明在人心里,因为 神已经给他们显明。
\VS{20}自从造天地以来, 神的永能和神性是明明可知的,虽是眼不能见,但借着所造之物就可以晓得,叫人无可推诿。
\VS{21}因为,他们虽然知道 神,却不当作 神荣耀他,也不感谢他。他们的思念变为虚妄,无知的心就昏暗了。
\VS{22}自称为聪明,反成了愚拙,
\VS{23}将不能朽坏之 神的荣耀变为偶像,仿佛必朽坏的人和飞禽、走兽、昆虫的样式。
\par }{\PP \VS{24}所以, 神任凭他们逞着心里的情欲行污秽的事,以致彼此玷辱自己的身体。
\VS{25}他们将 神的真实变为虚谎,去敬拜事奉受造之物,不敬奉那造物的主—主乃是可称颂的,直到永远。阿们!
\VS{26}因此, 神任凭他们放纵可羞耻的情欲。他们的女人把顺性的用处变为逆性的用处;
\VS{27}男人也是如此,弃了女人顺性的用处,欲火攻心,彼此贪恋,男和男行可羞耻的事,就在自己身上受这妄为当得的报应。
\VS{28}他们既然故意不认识 神, 神就任凭他们存邪僻的心,行那些不合理的事;
\VS{29}装满了各样不义、邪恶、贪婪、恶毒\FTNT{}{{\FR 1:29: }或译:阴毒};满心是嫉妒、凶杀、争竞、诡诈、毒恨;
\VS{30}又是谗毁的、背后说人的、怨恨 神的\FTNT{}{{\FR 1:30: }或译:被 神所憎恶的}、侮慢人的、狂傲的、自夸的、捏造恶事的、违背父母的、
\VS{31}无知的、背约的、无亲情的、不怜悯人的。
\VS{32}他们虽知道 神判定行这样事的人是当死的,然而他们不但自己去行,还喜欢别人去行。

\par }\Chap{2}{\SH  神的公义判断
\par }{\PP \VerseOne{1}你这论断人的,无论你是谁,也无可推诿。你在什么事上论断人,就在什么事上定自己的罪;因你这论断人的,自己所行却和别人一样。
\VS{2}我们知道这样行的人, 神必照真理审判他。
\VS{3}你这人哪,你论断行这样事的人,自己所行的却和别人一样,你以为能逃脱 神的审判吗?
\VS{4}还是你藐视他丰富的恩慈、宽容、忍耐,不晓得他的恩慈是领你悔改呢?
\VS{5}你竟任着你刚硬不悔改的心,为自己积蓄忿怒,以致 神震怒,显他公义审判的日子来到。
\VS{6}他必照各人的行为报应各人。
\VS{7}凡恒心行善、寻求荣耀、尊贵和不能朽坏{\ADD{之福}}的,就以永生报应他们;
\VS{8}惟有结党、不顺从真理、反顺从不义的,就以忿怒、恼恨报应他们;
\VS{9}将患难、困苦加给一切作恶的人,先是{\PN{犹太}}人,后是{\PN{希腊}}人,
\VS{10}却将荣耀、尊贵、平安加给一切行善的人,先是{\PN{犹太}}人,后是{\PN{希腊}}人。
\VS{11}因为 神不偏待人。
\VS{12}凡没有律法犯了罪的,也必不按律法灭亡;凡在律法以下犯了罪的,也必按律法受审判。(
\VS{13}原来在 神面前,不是听律法的为义,乃是行律法的称义。
\VS{14}没有律法的外邦人若顺着本性行律法上的事,他们虽然没有律法,自己就是自己的律法。
\VS{15}这是显出律法的功用刻在他们心里,他们是非之心同作见证,并且他们的思念互相较量,或以为是,或以为非。)
\VS{16}就在 神借耶稣基督审判人隐秘事的日子,照着我的福音所言。
\par }{\SH 犹太人与律法
\par }{\PP \VS{17}你称为{\PN{犹太}}人,又倚靠律法,且指着 神夸口;
\VS{18}既从律法中受了教训,就晓得 神的旨意,也能分别是非\FTNT{}{{\FR 2:18: }或译:也喜爱那美好的事};
\VS{19}又深信自己是给瞎子领路的,是黑暗中人的光,
\VS{20}是蠢笨人的师傅,是小孩子的先生,在律法上有知识和真理的模范。
\VS{21}你既是教导别人,还不教导自己吗?你讲说人不可偷窃,自己还偷窃吗?
\VS{22}你说人不可奸淫,自己还奸淫吗?你厌恶偶像,自己还偷窃庙中之物吗?
\VS{23}你指着律法夸口,自己倒犯律法、玷辱 神吗?
\VS{24}神的名在外邦人中,因你们受了亵渎,正如{\ADD{经上}}所记的。
\VS{25}你若是行律法的,割礼固然于你有益;若是犯律法的,你的割礼就算不得割礼。
\VS{26}所以那未受割礼的,若遵守律法的条例,他虽然未受割礼,岂不算是有割礼吗?
\VS{27}而且那本来未受割礼的,若能全守律法,岂不是要审判你这有仪文和割礼竟犯律法的人吗?
\VS{28}因为外面作{\PN{犹太}}人的,不是{\ADD{真}}{\PN{犹太}}人;外面肉身的割礼,也不是{\ADD{真}}割礼。
\VS{29}惟有里面作的,才是{\ADD{真}}{\PN{犹太}}人;{\ADD{真}}割礼也是心里的,在乎灵,不在乎仪文。这人的称赞不是从人来的,乃是从 神来的。

\par }\Chap{3}{\PP \VerseOne{1}这样说来,{\PN{犹太}}人有什么长处?割礼有什么益处呢?
\VS{2}凡事大有好处:第一是 神的圣言交托他们。
\VS{3}即便有不信的,这有何妨呢?难道他们的不信就废掉 神的信吗?
\VS{4}断乎不能!不如说, 神是真实的,人都是虚谎的。如{\ADD{经上}}所记:
\par }{\Q 你责备人的时候,显为公义;
\par }{\Q 被人议论的时候,可以得胜。
\par }{\MM \VS{5}我且照着人的常话说,我们的不义若显出 神的义来,我们可以怎么说呢? 神降怒,是他不义吗?
\VS{6}断乎不是!若是这样, 神怎能审判世界呢?
\VS{7}若 神的真实,因我的虚谎越发显出他的荣耀,为什么我还受审判,好像罪人呢?
\VS{8}为什么不说,我们可以作恶以成善呢?这是毁谤我们的人说我们有这话。这等人定罪是该当的。
\par }{\SH 没有义人
\par }{\PP \VS{9}这却怎么样呢?我们比他们强吗?决不是的!因我们已经证明:{\PN{犹太}}人和{\PN{希腊}}人都在罪恶之下。
\VS{10}就如{\ADD{经上}}所记:
\par }{\Q 没有义人,连一个也没有。
\par }{\Q \VS{11}没有明白的;
\par }{\Q 没有寻求 神的;
\par }{\Q \VS{12}都是偏离正路,
\par }{\Q 一同变为无用。
\par }{\Q 没有行善的,连一个也没有。
\par }{\Q \VS{13}他们的喉咙是敞开的坟墓;
\par }{\Q 他们用舌头弄诡诈,
\par }{\Q 嘴唇里有虺蛇的毒气,
\par }{\Q \VS{14}满口是咒骂苦毒。
\par }{\Q \VS{15}杀人流血,
\par }{\Q 他们的脚飞跑,
\par }{\Q \VS{16}所经过的路便行残害暴虐的事。
\par }{\Q \VS{17}平安的路,他们未曾知道;
\par }{\Q \VS{18}他们眼中不怕 神。
\par }{\PP \VS{19}我们晓得律法上的话都是对律法以下之人说的,好塞住各人的口,叫普世的人都伏在 神审判之下。
\VS{20}所以凡有血气的,没有一个因行律法能在 神面前称义,因为律法本是叫人知罪。
\par }{\SH 因信称义
\par }{\PP \VS{21}但如今, 神的义在律法以外已经显明出来,有律法和先知为证:
\VS{22}就是 神的义,因信耶稣基督加给一切相信的人,并没有分别。
\VS{23}因为世人都犯了罪,亏缺了 神的荣耀;
\VS{24}如今却蒙 神的恩典,因基督耶稣的救赎,就白白地称义。
\VS{25}神设立耶稣作挽回祭,是凭着耶稣的血,借着人的信,要显明 神的义;因为他用忍耐的心宽容人先时所犯的罪,
\VS{26}好在今时显明他的义,使人知道他自己为义,也称信耶稣的人为义。
\VS{27}既是这样,哪里能夸口呢?没有可夸的了。用何法没有的呢?是用立功之法吗?不是,乃用信{\ADD{主}}之法。
\VS{28}所以\FTNT{}{{\FR 3:28: }有古卷:因为}我们看定了:人称义是因着信,不在乎遵行律法。
\VS{29}难道 神只作{\PN{犹太}}人的 神吗?不也是作外邦人的 神吗?是的,也作外邦人的 神。
\VS{30}神既是一位,他就要因信称那受割礼的为义,也要因信称那未受割礼的为义。
\VS{31}这样,我们因信废了律法吗?断乎不是!更是坚固律法。

\par }\Chap{4}{\SH 以亚伯拉罕为例
\par }{\PP \VerseOne{1}如此说来,我们的祖宗{\PN{亚伯拉罕}}凭着肉体得了什么呢?
\VS{2}倘若{\PN{亚伯拉罕}}是因行为称义,就有可夸的;只是在 神面前并无可夸。
\VS{3}经上说什么呢?说:「{\PN{亚伯拉罕}}信 神,这就算为他的义。」
\VS{4}做工的得工价,不算恩典,乃是该得的;
\VS{5}惟有不做工的,只信称罪人为义的 神,他的信就算为义。
\VS{6}正如{\PN{大卫}}称那在行为以外蒙 神算为义的人是有福的。
\VS{7}{\ADD{他}}说:
\par }{\Q 得赦免其过、遮盖其罪的,
\par }{\Q 这人是有福的。
\par }{\Q \VS{8}主不算为有罪的,
\par }{\Q 这人是有福的。
\par }{\PP \VS{9}如此看来,这福是单加给那受割礼的人吗?不也是加给那未受割礼的人吗?因我们所说,{\PN{亚伯拉罕}}的信,就算为他的义,
\VS{10}是怎么算的呢?是在他受割礼的时候呢?是在他未受割礼的时候呢?不是在受割礼的时候,乃是在未受割礼的时候。
\VS{11}并且他受了割礼的记号,作他未受割礼的时候因信称义的印证,叫他作一切未受割礼而信之人的父,使他们也算为义;
\VS{12}又作受割礼之人的父,就是那些不但受割礼,并且按我们的祖宗{\PN{亚伯拉罕}}未受割礼而信之踪迹去行的人。
\par }{\SH 应许因信而实现
\par }{\PP \VS{13}因为 {\ADD{神}}应许{\PN{亚伯拉罕}}和他后裔,必得承受世界,不是因律法,乃是因信而得的义。
\VS{14}若是属乎律法的人才得为后嗣,信就归于虚空,应许也就废弃了。
\VS{15}因为律法是惹动忿怒的\FTNT{}{{\FR 4:15: }或译:叫人受刑的};哪里没有律法,那里就没有过犯。
\VS{16}所以{\ADD{人得为后嗣}}是本乎信,因此就属乎恩,叫应许定然归给一切后裔;不但归给那属乎律法的,也归给那效法{\PN{亚伯拉罕}}之信的。
\VS{17}{\PN{亚伯拉罕}}所信的,是那叫死人复活、使无变为有的 神,他在主面前作我们世人的父。如{\ADD{经上}}所记:「我已经立你作多国的父。」
\VS{18}他在无可指望的时候,因信仍有指望,就得以作多国的父,正如先前所说:「你的后裔将要如此。」
\VS{19}他将近百岁的时候,虽然想到自己的身体如同已死,{\PN{撒拉}}的生育已经断绝,他的信心还是不软弱;
\VS{20}并且仰望 神的应许,总没有因不信心里起疑惑,反倒因信心里得坚固,将荣耀归给 神,
\VS{21}且满心相信 神所应许的必能做成。
\VS{22}所以,这就算为他的义。
\VS{23}「算为他义」的这句话不是单为他写的,
\VS{24}也是为我们将来得算为义之人写的,就是我们这信 神使我们的主耶稣从死里复活的人。
\VS{25}耶稣被交给人,是为我们的过犯;复活,是为叫我们称义\FTNT{}{{\FR 4:25: }或译:耶稣是为我们的过犯交付了,是为我们称义复活了}。

\par }\Chap{5}{\SH 因信称义的福
\par }{\PP \VerseOne{1}我们既因信称义,就借着我们的主耶稣基督得与 神相和。
\VS{2}我们又借着他,因信得进入现在所站的这恩典中,并且欢欢喜喜盼望 神的荣耀。
\VS{3}不但如此,就是在患难中也是欢欢喜喜的;因为知道患难生忍耐,
\VS{4}忍耐生老练,老练生盼望;
\VS{5}盼望不至于羞耻,因为所赐给我们的圣灵将 神的爱浇灌在我们心里。
\VS{6}因我们还软弱的时候,基督就按所定的日期为罪人死。
\VS{7}为义人死,是少有的;为仁人死,或者有敢做的。
\VS{8}惟有基督在我们还作罪人的时候为我们死, 神的爱就在此向我们显明了。
\VS{9}现在我们既靠着他的血称义,就更要借着他免去 {\ADD{神的}}忿怒。
\VS{10}因为我们作仇敌的时候,且借着 神儿子的死,得与 神和好;既已和好,就更要因他的生得救了。
\VS{11}不但如此,我们既借着我主耶稣基督得与 神和好,也就借着他以 神为乐。
\par }{\SH 亚当和基督
\par }{\PP \VS{12}这就如罪是从一人入了世界,死又是从罪来的,于是死就临到众人,因为众人都犯了罪。
\VS{13}没有律法之先,罪已经在世上;但没有律法,罪也不算罪。
\VS{14}然而从{\PN{亚当}}到{\PN{摩西}},死就作了王,连那些不与{\PN{亚当}}犯一样罪过的,也在他的权下。{\PN{亚当}}乃是那以后要来之人的预像。
\VS{15}只是过犯不如恩赐,若因一人的过犯,众人都死了,何况 神的恩典,与那因耶稣基督一人恩典中的赏赐,岂不更加倍地临到众人吗?
\VS{16}因一人犯罪就定罪,也不如恩赐,原来审判是由一人而定罪,恩赐乃是由许多过犯而称义。
\VS{17}若因一人的过犯,死就因这一人作了王,何况那些受洪恩又蒙所赐之义的,岂不更要因耶稣基督一人在生命中作王吗?
\VS{18}如此说来,因一次的过犯,众人都被定罪;照样,因一次的义行,众人也就被称义得生命了。
\VS{19}因一人的悖逆,众人成为罪人;照样,因一人的顺从,众人也成为义了。
\VS{20}律法本是外添的,叫过犯显多;只是罪在哪里显多,恩典就更显多了。
\VS{21}就如罪作王叫人死;照样,恩典也借着义作王,叫人因我们的主耶稣基督得永生。

\par }\Chap{6}{\SH 在罪上死,在基督里活
\par }{\PP \VerseOne{1}这样,怎么说呢?我们可以仍在罪中、叫恩典显多吗?
\VS{2}断乎不可!我们在罪上死了的人岂可仍在罪中活着呢?
\VS{3}岂不知我们这受洗归入基督耶稣的人是受洗归入他的死吗?
\VS{4}所以,我们借着洗礼归入死,和他一同埋葬,原是叫我们一举一动有新生的样式,像基督借着父的荣耀从死里复活一样。
\VS{5}我们若在他死的形状上与他联合,也要在他复活的形状上与他联合;
\VS{6}因为知道我们的旧人和他同钉十字架,使罪身灭绝,叫我们不再作罪的奴仆;
\VS{7}因为已死的人是脱离了罪。
\VS{8}我们若是与基督同死,就信必与他同活。
\VS{9}因为知道基督既从死里复活,就不再死,死也不再作他的主了。
\VS{10}他死是向罪死了,只有一次;他活是向 神活着。
\VS{11}这样,你们向罪也当看自己是死的;向 神在基督耶稣里,却当看自己是活的。
\par }{\PP \VS{12}所以,不要容罪在你们必死的身上作王,使你们顺从身子的私欲。
\VS{13}也不要将你们的肢体献给罪作不义的器具;倒要像从死里复活的人,将自己献给 神,并将肢体作义的器具献给 神。
\VS{14}罪必不能作你们的主;因你们不在律法之下,乃在恩典之下。
\par }{\SH 义的奴仆
\par }{\PP \VS{15}这却怎么样呢?我们在恩典之下,不在律法之下,就可以犯罪吗?断乎不可!
\VS{16}岂不晓得你们献上自己作奴仆,顺从谁,就作谁的奴仆吗?或作罪的奴仆,以至于死;或作顺命的奴仆,以至成义。
\VS{17}感谢 神!因为你们从前虽然作罪的奴仆,现今却从心里顺服了所传给你们道理的模范。
\VS{18}你们既从罪里得了释放,就作了义的奴仆。
\VS{19}我因你们肉体的软弱,就照人的常话对你们说。你们从前怎样将肢体献给不洁不法作奴仆,以至于不法;现今也要照样将肢体献给义作奴仆,以至于成圣。
\VS{20}因为你们作罪之奴仆的时候,就不被义约束了。
\VS{21}你们现今所看为羞耻的事,当日有什么果子呢?那些事的结局就是死。
\VS{22}但现今,你们既从罪里得了释放,作了 神的奴仆,就有成圣的果子,那结局就是永生。
\VS{23}因为罪的工价乃是死;惟有 神的恩赐,在我们的主基督耶稣里,乃是永生。

\par }\Chap{7}{\SH 以婚姻关系为例
\par }{\PP \VerseOne{1}弟兄们,我现在对明白律法的人说,你们岂不晓得律法管人是在活着的时候吗?
\VS{2}就如女人有了丈夫,丈夫还活着,就被律法约束;丈夫若死了,就脱离了丈夫的律法。
\VS{3}所以丈夫活着,她若归于别人,便叫淫妇;丈夫若死了,她就脱离了丈夫的律法,虽然归于别人,也不是淫妇。
\VS{4}我的弟兄们,这样说来,你们借着基督的身体,在律法上也是死了,叫你们归于别人,就是归于那从死里复活的,叫我们结果子给 神。
\VS{5}因为我们属肉体的时候,那因律法而生的恶欲就在我们肢体中发动,以致结成死亡的果子。
\VS{6}但我们既然在捆我们的律法上死了,现今就脱离了律法,叫我们服事{\ADD{主}},要按着心灵\FTNT{}{{\FR 7:6: }心灵:或译圣灵}的新样,不按着仪文的旧样。
\par }{\SH 内在的罪
\par }{\PP \VS{7}这样,我们可说什么呢?律法是罪吗?断乎不是!只是非因律法,我就不知何为罪。非律法说「不可起贪心」,我就不知何为贪心。
\VS{8}然而,罪趁着机会,就借着诫命叫诸般的贪心在我里头发动;因为没有律法,罪是死的。
\VS{9}我以前没有律法是活着的;但是诫命来到,罪又活了,我就死了。
\VS{10}那本来叫人活的诫命,反倒叫我死;
\VS{11}因为罪趁着机会,就借着诫命引诱我,并且杀了我。
\VS{12}这样看来,律法是圣洁的,诫命也是圣洁、公义、良善的。
\VS{13}既然如此,那良善的是叫我死吗?断乎不是!{\ADD{叫我死的乃是罪}}。但罪借着那良善的叫我死,就显出真是罪,叫罪因着诫命更显出是恶极了。
\VS{14}我们原晓得律法是属乎灵的,但我是属乎肉体的,是已经卖给罪了。
\VS{15}因为我所做的,我自己不明白;我所愿意的,我并不做;我所恨恶的,我倒去做。
\VS{16}若我所做的,是我所不愿意的,我就应承律法是善的。
\VS{17}既是这样,就不是我做的,乃是住在我里头的罪做的。
\VS{18}我也知道,在我里头,就是我肉体之中,没有良善。因为,立志为善由得我,只是行出来由不得我。
\VS{19}故此,我所愿意的善,我反不做;我所不愿意的恶,我倒去做。
\VS{20}若我去做所不愿意做的,就不是我做的,乃是住在我里头的罪做的。
\VS{21}我觉得有个律,就是我愿意为善的时候,便有恶与我同在。
\VS{22}因为按着我里面的意思\FTNT{}{{\FR 7:22: }原文是人},我是喜欢 神的律;
\VS{23}但我觉得肢体中另有个律和我心中的律交战,把我掳去,叫我附从那肢体中犯罪的律。
\VS{24}我真是苦啊!谁能救我脱离这取死的身体呢?
\VS{25}感谢 神,靠着我们的主耶稣基督{\ADD{就能脱离了}}。这样看来,我以内心顺服 神的律,我肉体却顺服罪的律了。

\par }\Chap{8}{\SH 在圣灵里的生活
\par }{\PP \VerseOne{1}如今,那些在基督耶稣里的就不定罪了。
\VS{2}因为赐生命{\ADD{圣}}灵的律,在基督耶稣里释放了我,使我脱离罪和死的律了。
\VS{3}律法既因肉体软弱,有所不能行的, 神就差遣自己的儿子,成为罪身的形状,作了赎罪祭,在肉体中定了罪案,
\VS{4}使律法的义成就在我们这不随从肉体、只随从{\ADD{圣}}灵的人身上。
\VS{5}因为,随从肉体的人体贴肉体的事;随从{\ADD{圣}}灵的人体贴{\ADD{圣}}灵的事。
\VS{6}体贴肉体的,就是死;体贴{\ADD{圣}}灵的,乃是生命、平安。
\VS{7}原来体贴肉体的,就是与 神为仇;因为不服 神的律法,也是不能服,
\VS{8}而且属肉体的人不能得 神的喜欢。
\VS{9}如果 神的灵住在你们心里,你们就不属肉体,乃属{\ADD{圣}}灵了。人若没有基督的灵,就不是属基督的。
\VS{10}基督若在你们心里,身体就因罪而死,心灵却因义而活。
\VS{11}然而,叫耶稣从死里复活者的灵若住在你们心里,那叫基督耶稣从死里复活的,也必借着住在你们心里的{\ADD{圣}}灵,使你们必死的身体又活过来。
\par }{\PP \VS{12}弟兄们,这样看来,我们并不是欠肉体的债去顺从肉体活着。
\VS{13}你们若顺从肉体活着,必要死;若靠着{\ADD{圣}}灵治死身体的恶行,必要活着。
\VS{14}因为凡被 神的灵引导的,都是 神的儿子。
\VS{15}你们所受的,不是奴仆的心,仍旧害怕;所受的,乃是儿子的心,因此我们呼叫:「阿爸!父!」
\VS{16}圣灵与我们的心同证我们是 神的儿女;
\VS{17}既是儿女,便是后嗣,就是 神的后嗣,和基督同作后嗣。如果我们和他一同受苦,也必和他一同得荣耀。
\par }{\SH 将来的荣耀
\par }{\PP \VS{18}我想,现在的苦楚若比起将来要显于我们的荣耀就不足介意了。
\VS{19}受造之物切望等候 神的众子显出来。
\VS{20}因为受造之物服在虚空之下,不是自己愿意,乃是因那叫他如此的。
\VS{21}但受造之物仍然指望脱离败坏的辖制,得享\FTNT{}{{\FR 8:21: }享:原文是入} 神儿女自由的荣耀。
\VS{22}我们知道,一切受造之物一同叹息,劳苦,直到如今。
\VS{23}不但如此,就是我们这有{\ADD{圣}}灵初结果子的,也是自己心里叹息,等候得着儿子的名分,乃是我们的身体得赎。
\VS{24}我们得救是在乎盼望;只是所见的盼望不是盼望,谁还盼望他所见的呢\FTNT{}{{\FR 8:24: }有古卷:人所看见的何必再盼望呢}?
\VS{25}但我们若盼望那所不见的,就必忍耐等候。
\par }{\PP \VS{26}况且,我们的软弱有{\ADD{圣}}灵帮助;我们本不晓得当怎样祷告,只是{\ADD{圣}}灵亲自用说不出来的叹息替我们祷告。
\VS{27}鉴察人心的,晓得{\ADD{圣}}灵的意思,因为{\ADD{圣}}灵照着 神的{\ADD{旨意}}替圣徒祈求。
\VS{28}我们晓得万事都互相效力,叫爱 神的人得益处,就是按他旨意被召的人。
\VS{29}因为他预先所知道的人,就预先定下效法他儿子的模样,使他儿子在许多弟兄中作长子。
\VS{30}预先所定下的人又召他们来;所召来的人又称他们为义;所称为义的人又叫他们得荣耀。
\par }{\SH  神的爱
\par }{\PP \VS{31}既是这样,还有什么说的呢? 神若帮助我们,谁能敌挡我们呢?
\VS{32}神既不爱惜自己的儿子,为我们众人舍了,岂不也把万物和他一同白白地赐给我们吗?
\VS{33}谁能控告 神所拣选的人呢?有 神称他们为义了\FTNT{}{{\FR 8:33: }或译:是称他们为义的 神吗}。
\VS{34}谁能定他们的罪呢?有基督耶稣已经死了,而且从死里复活,现今在 神的右边,也替我们祈求\FTNT{}{{\FR 8:34: }有基督......或译:是已经死了,而且从死里复活,现今在 神的右边,也替我们祈求的基督耶稣吗}。
\VS{35}谁能使我们与基督的爱隔绝呢?难道是患难吗?是困苦吗?是逼迫吗?是饥饿吗?是赤身露体吗?是危险吗?是刀剑吗?
\VS{36}如{\ADD{经上}}所记:
\par }{\Q 我们为你的缘故终日被杀;
\par }{\Q 人看我们如将宰的羊。
\par }{\PP \VS{37}然而,靠着爱我们的{\ADD{主}},在这一切的事上已经得胜有余了。
\VS{38}因为我深信无论是死,是生,是天使,是掌权的,是有能的,是现在的事,是将来的事,
\VS{39}是高处的,是低处的,是别的受造之物,都不能叫我们与 神的爱隔绝;这爱是在我们的主基督耶稣里的。

\par }\Chap{9}{\SH  神拣选以色列人
\par }{\PP \VerseOne{1}我在基督里说真话,并不谎言,有我良心被圣灵感动,给我作见证:
\VS{2}我是大有忧愁,心里时常伤痛;
\VS{3}为我弟兄,我骨肉之亲,就是自己被咒诅,与基督分离,我也愿意。
\VS{4}他们是{\PN{以色列}}人;那儿子的名分、荣耀、诸约、律法、礼仪、应许都是他们的。
\VS{5}列祖就是他们的祖宗;按肉体说,基督也是从他们出来的。他是在万有之上,永远可称颂的 神。阿们!
\par }{\PP \VS{6}这不是说 神的话落了空。因为从{\PN{以色列}}生的不都是{\PN{以色列}}人,
\VS{7}也不因为是{\PN{亚伯拉罕}}的后裔就都作他的儿女;惟独「从{\PN{以撒}}生的才要称为你的后裔。」
\VS{8}这就是说,肉身所生的儿女不是 神的儿女,惟独那应许的儿女才算是后裔。
\VS{9}因为所应许的话是这样说:「到{\ADD{明年}}这时候我要来,{\PN{撒拉}}必生一个儿子。」
\VS{10}不但如此,还有{\PN{利百加}},既从一个人,就是从我们的祖宗{\PN{以撒}}怀了孕,(
\VS{11}{\ADD{双子}}还没有生下来,善恶还没有做出来,只因要显明 神拣选人的旨意,不在乎人的行为,乃在乎召人的{\ADD{主}}。)
\VS{12}神就对{\PN{利百加}}说:「将来,大的要服事小的。」
\VS{13}正如{\ADD{经上}}所记:{\PN{雅各}}是我所爱的;{\PN{以扫}}是我所恶的。
\par }{\PP \VS{14}这样,我们可说什么呢?难道 神有什么不公平吗?断乎没有!
\VS{15}因他对{\PN{摩西}}说:
\par }{\Q 我要怜悯谁就怜悯谁,
\par }{\Q 要恩待谁就恩待谁。
\par }{\MM \VS{16}据此看来,这不在乎那定意的,也不在乎那奔跑的,只在乎发怜悯的 神。
\VS{17}因为经上有话向法老说:「我将你兴起来,特要在你身上彰显我的权能,并要使我的名传遍天下。」
\VS{18}如此看来, 神要怜悯谁就怜悯谁,要叫谁刚硬就叫谁刚硬。
\par }{\SH  神的忿怒和怜悯
\par }{\PP \VS{19}这样,你必对我说:「他为什么还指责人呢?有谁抗拒他的旨意呢?」
\VS{20}你这个人哪,你是谁,竟敢向 神强嘴呢?受造之物岂能对造他的说:「你为什么这样造我呢?」
\VS{21}窑匠难道没有权柄从一团泥里拿一块做成贵重的器皿,又拿一块做成卑贱的器皿吗?
\VS{22}倘若 神要显明他的忿怒,彰显他的权能,就多多忍耐宽容那可怒、预备遭毁灭的器皿,
\VS{23}又要将他丰盛的荣耀彰显在那蒙怜悯、早预备得荣耀的器皿上。
\VS{24}这{\ADD{器皿}}就是我们被 神所召的,不但是从{\PN{犹太}}人中,也是从外邦人中。这有什么不可呢?
\VS{25}就像 神在{\PN{何西阿}}书上说:
\par }{\Q 那本来不是我子民的,
\par }{\Q 我要称为「我的子民」;
\par }{\Q 本来不是蒙爱的,
\par }{\Q 我要称为「蒙爱的」。
\par }{\Q \VS{26}从前在什么地方对他们说:
\par }{\Q 你们不是我的子民,
\par }{\Q 将来就在那里称他们为「永生 神的儿子」。
\par }{\PP \VS{27}{\PN{以赛亚}}指着{\PN{以色列}}人喊着说:「{\PN{以色列}}人虽多如海沙,得救的不过是剩下的余数;
\VS{28}因为主要在世上施行他的话,叫他的话都成全,速速地完结。」
\VS{29}又如{\PN{以赛亚}}先前说过:
\par }{\Q 若不是万军之主给我们存留余种,
\par }{\Q 我们早已像{\PN{所多玛}}、{\PN{蛾摩拉}}的样子了。
\par }{\SH 以色列人和福音
\par }{\PP \VS{30}这样,我们可说什么呢?那本来不追求义的外邦人反得了义,就是因信而得的义。
\VS{31}但{\PN{以色列}}人追求律法的义,反得不着律法的义。
\VS{32}这是什么缘故呢?是因为他们不凭着信心求,只凭着行为求;他们正跌在那绊脚石上。
\VS{33}就如经上所记:
\par }{\Q 我在{\PN{锡安}}放一块绊脚的石头,跌人的磐石;
\par }{\Q 信靠他的人必不至于羞愧。

\par }\Chap{10}{\PP \VerseOne{1}弟兄们,我心里所愿的,向 神所求的,是要{\PN{以色列}}人得救。
\VS{2}我可以证明,他们向 神有热心,但不是按着真知识;
\VS{3}因为不知道 神的义,想要立自己的义,就不服 神的义了。
\VS{4}律法的总结就是基督,使凡信他的都得着义。
\par }{\SH 求告主名的必要得救
\par }{\PP \VS{5}{\PN{摩西}}写着说:「人若行那出于律法的义,就必因此活着。」
\VS{6}惟有出于信心的义如此说:「你不要心里说:谁要升到天上去呢?(就是要领下基督来;)
\VS{7}谁要下到阴间去呢?(就是要领基督从死里上来。)」
\VS{8}他到底怎么说呢?他说:
\par }{\Q 这道离你不远,
\par }{\Q 正在你口里,在你心里—
\par }{\PP (就是我们所传信{\ADD{主}}的道。)
\VS{9}你若口里认耶稣为主,心里信 神叫他从死里复活,就必得救。
\VS{10}因为,人心里相信就可以称义,口里承认就可以得救。
\VS{11}经上说:「凡信他的人必不至于羞愧。」
\VS{12}{\PN{犹太}}人和{\PN{希腊}}人并没有分别,因为众人同有一位主;他也厚待一切求告他的人。
\VS{13}因为「凡求告主名的就必得救」。
\par }{\PP \VS{14}然而,人未曾信他,怎能求他呢?未曾听见他,怎能信他呢?没有传道的,怎能听见呢?
\VS{15}若没有奉差遣,怎能传道呢?如{\ADD{经}}上所记:「报福音、传喜信的人,他们的脚{\ADD{踪}}何等佳美!」
\VS{16}只是人没有都听从福音,因为{\PN{以赛亚}}说:「主啊,我们所传的有谁信呢?」
\VS{17}可见,信道是从听道来的,听道是从基督的话来的。
\VS{18}但我说,人没有听见吗?诚然听见了。
\par }{\Q 他们的声音传遍天下;
\par }{\Q 他们的言语传到地极。
\par }{\MM \VS{19}我再说,{\PN{以色列}}人不知道吗?先有{\PN{摩西}}说:
\par }{\Q 我要用那不成子民的惹动你们的愤恨;
\par }{\Q 我要用那无知的民触动你们的怒气。
\par }{\MM \VS{20}又有{\PN{以赛亚}}放胆说:
\par }{\Q 没有寻找我的,我叫他们遇见;
\par }{\Q 没有访问我的,我向他们显现。
\par }{\MM \VS{21}至于{\PN{以色列}}人,他说:「我整天伸手招呼那悖逆顶嘴的百姓。」

\par }\Chap{11}{\SH 以色列的余民
\par }{\PP \VerseOne{1}我且说, 神弃绝了他的百姓吗?断乎没有!因为我也是{\PN{以色列}}人,{\PN{亚伯拉罕}}的后裔,属{\PN{便雅悯}}支派的。
\VS{2}神并没有弃绝他预先所知道的百姓。你们岂不晓得经上论到{\PN{以利亚}}是怎么说的呢?他在 神面前怎样控告{\PN{以色列}}人说:
\VS{3}「主啊,他们杀了你的先知,拆了你的祭坛,只剩下我一个人;他们还要寻索我的命。」
\VS{4}神的回话是怎么说的呢?{\ADD{他说}}:「我为自己留下七千人,是未曾向{\PN{巴力}}屈膝的。」
\VS{5}如今也是这样,照着拣选的恩典,还有所留的余数。
\VS{6}既是出于恩典,就不在乎行为;不然,恩典就不是恩典了。
\VS{7}这是怎么样呢?{\PN{以色列}}人所求的,他们没有得着。惟有蒙拣选的人得着了;其余的就成了顽梗不化的。
\VS{8}如{\ADD{经上}}所记:
\par }{\Q  神给他们昏迷的心,
\par }{\Q 眼睛不能看见,
\par }{\Q 耳朵不能听见,
\par }{\Q 直到今日。
\par }{\MM \VS{9}{\PN{大卫}}也说:
\par }{\Q 愿他们的筵席变为网罗,变为机槛,
\par }{\Q 变为绊脚石,作他们的报应。
\par }{\Q \VS{10}愿他们的眼睛昏蒙,不得看见;
\par }{\Q 愿你时常弯下他们的腰。
\par }{\SH 外邦人得救
\par }{\PP \VS{11}我且说,他们失脚是要他们跌倒吗?断乎不是!反倒因他们的过失,救恩便临到外邦人,要激动他们发愤。
\VS{12}若他们的过失为天下的富足,他们的缺乏为外邦人的富足,何况他们的丰满呢?
\par }{\PP \VS{13}我对你们外邦人说这话;因我是外邦人的使徒,所以敬重\FTNT{}{{\FR 11:13: }原文是荣耀}我的职分,
\VS{14}或者可以激动我骨肉之亲发愤,好救他们一些人。
\VS{15}若他们被丢弃,天下就得{\ADD{与 神}}和好,他们被收纳,岂不是死而复生吗?
\VS{16}所献的新面若是圣洁,全团也就圣洁了;树根若是圣洁,树枝也就圣洁了。
\par }{\PP \VS{17}若有几根枝子被折下来,你这野橄榄得接在其中,一同得着橄榄根的肥汁,
\VS{18}你就不可向{\ADD{旧}}枝子夸口;若是夸口,{\ADD{当知道}}不是你托着根,乃是根托着你。
\VS{19}你若说,那枝子被折下来是特为叫我接上。
\VS{20}不错!他们因为不信,所以被折下来;你因为信,所以立得住;你不可自高,反要惧怕。
\VS{21}神既不爱惜原来的枝子,也必不爱惜你。
\VS{22}可见, 神的恩慈和严厉向那跌倒的人是严厉的,向你是有恩慈的,只要你长久在他的恩慈里;不然,你也要被砍下来。
\VS{23}而且他们若不是长久不信,仍要被接上,因为 神能够把他们从新接上。
\VS{24}你是从那天生的野橄榄上砍下来的,尚且逆着性得接在好橄榄上,何况这本树的枝子,要接在本树上呢!
\par }{\SH 以色列人的复兴
\par }{\PP \VS{25}弟兄们,我不愿意你们不知道这奥秘(恐怕你们自以为聪明),就是{\PN{以色列}}人有几分是硬心的,等到外邦人的数目添满了,
\VS{26}于是{\PN{以色列}}全家都要得救。如{\ADD{经上}}所记:
\par }{\Q 必有一位救主从{\PN{锡安}}出来,
\par }{\Q 要消除{\PN{雅各}}家的一切罪恶;
\par }{\Q \VS{27}又{\ADD{说}}:我除去他们罪的时候,
\par }{\Q 这就是我与他们所立的约。
\par }{\MM \VS{28}就着福音说,他们为你们的缘故是仇敌;就着拣选说,他们为列祖的缘故是蒙爱的。
\VS{29}因为 神的恩赐和选召是没有后悔的。
\VS{30}你们从前不顺服 神,如今因他们的不顺服,你们倒蒙了怜恤。
\VS{31}这样,他们也是不顺服,叫他们因着施给你们的怜恤,现在也就蒙怜恤。
\VS{32}因为 神将众人都圈在不顺服之中,特意要怜恤众人。
\par }{\Q \VS{33}深哉, 神丰富的智慧和知识!
\par }{\Q 他的判断何其难测!
\par }{\Q 他的踪迹何其难寻!
\par }{\Q \VS{34}谁知道主的心?
\par }{\Q 谁作过他的谋士呢?
\par }{\Q \VS{35}谁是先给了他,
\par }{\Q 使他后来偿还呢?
\par }{\Q \VS{36}因为万有都是本于他,
\par }{\Q 倚靠他,归于他。
\par }{\Q 愿荣耀归给他,直到永远。阿们!

\par }\Chap{12}{\SH 基督里的新生活
\par }{\PP \VerseOne{1}所以,弟兄们,我以 神的慈悲劝你们,将身体献上,当作活祭,是圣洁的,是 神所喜悦的;你们如此事奉乃是理所当然的。
\VS{2}不要效法这个世界,只要心意更新而变化,叫你们察验何为 神的善良、纯全、可喜悦的旨意。
\par }{\PP \VS{3}我凭着所赐我的恩对你们各人说:不要看自己过于所当看的;要照着 神所分给各人信心的大小,看得合乎中道。
\VS{4}正如我们一个身子上有好些肢体,肢体也不都是一样的用处。
\VS{5}我们这许多人,在基督里成为一身,互相联络作肢体,也是如此。
\VS{6}按我们所得的恩赐,各有不同。或说预言,就当照着信心的程度说{\ADD{预言}};
\VS{7}或作执事,就当{\ADD{专一}}执事;或作教导的,就当{\ADD{专一}}教导;
\VS{8}或作劝化的,就当{\ADD{专一}}劝化;施舍的,就当诚实;治理的,就当殷勤;怜悯人的,就当甘心。
\par }{\SH 基督徒的生活守则
\par }{\PP \VS{9}爱人不可虚假。恶,要厌恶;善,要亲近。
\VS{10}爱弟兄,要彼此亲热;恭敬人,要彼此推让。
\VS{11}殷勤,不可懒惰;要心里火热,常常服事主。
\VS{12}在指望中要喜乐;在患难中要忍耐;祷告要恒切。
\VS{13}圣徒缺乏,要帮补;客,要一味地款待。
\VS{14}逼迫你们的,要给他们祝福;只要祝福,不可咒诅。
\VS{15}与喜乐的人要同乐;与哀哭的人要同哭。
\VS{16}要彼此同心;不要志气高大,倒要俯就卑微的人\FTNT{}{{\FR 12:16: }人:或译事}。不要自以为聪明。
\VS{17}不要以恶报恶;众人以为美的事要留心去做。
\VS{18}若是能行,总要尽力与众人和睦。
\VS{19}亲爱的弟兄,不要自己伸冤,宁可让步,听凭主怒\FTNT{}{{\FR 12:19: }或译:让人发怒};因为{\ADD{经上}}记着:「主说:『伸冤在我,我必报应。』」
\VS{20}所以,「你的仇敌若饿了,就给他吃,若渴了,就给他喝;因为你这样行就是把炭火堆在他的头上。」
\VS{21}你不可为恶所胜,反要以善胜恶。

\par }\Chap{13}{\SH 顺服掌权者
\par }{\PP \VerseOne{1}在上有权柄的,人人当顺服他,因为没有权柄不是出于 神的。凡掌权的都是 神所命的。
\VS{2}所以,抗拒掌权的就是抗拒 神的命;抗拒的必自取刑罚。
\VS{3}作官的原不是叫行善的惧怕,乃是叫作恶的惧怕。你愿意不惧怕掌权的吗?你只要行善,就可得他的称赞;
\VS{4}因为他是 神的用人,是与你有益的。你若作恶,却当惧怕,因为他不是空空地佩剑;他是 神的用人,是伸冤的,刑罚那作恶的。
\VS{5}所以,你们必须顺服,不但是因为刑罚,也是因为良心。
\VS{6}你们纳粮,也为这个缘故;因他们是 神的差役,常常特管这事。
\VS{7}凡人所当得的,就给他。当得粮的,给他纳粮;当得税的,给他上税;当惧怕的,惧怕他;当恭敬的,恭敬他。
\par }{\SH 相爱如兄弟
\par }{\PP \VS{8}凡事都不可亏欠人,惟有彼此相爱要常以为亏欠,因为爱人的就完全了律法。
\VS{9}像那不可奸淫,不可杀人,不可偷盗,不可贪婪,或有别的诫命,都包在爱人如己这一句话之内了。
\VS{10}爱是不加害与人的,所以爱就完全了律法。
\par }{\SH 白昼将近
\par }{\PP \VS{11}再者,你们晓得,现今就是该趁早睡醒的时候;因为我们得救,现今比{\ADD{初}}信的时候更近了。
\VS{12}黑夜已深,白昼将近。我们就当脱去暗昧的行为,带上光明的兵器。
\VS{13}行事为人要端正,好像行在白昼。不可荒宴醉酒;不可好色邪荡;不可争竞嫉妒。
\VS{14}总要披戴主耶稣基督,不要为肉体安排,去{\ADD{放纵}}私欲。

\par }\Chap{14}{\SH 不可论断弟兄
\par }{\PP \VerseOne{1}信心软弱的,你们要接纳,但不要辩论所疑惑的事。
\VS{2}有人信百物都可吃;但那软弱的,只吃蔬菜。
\VS{3}吃的人不可轻看不吃的人;不吃的人不可论断吃的人;因为 神已经收纳他了。
\VS{4}你是谁,竟论断别人的仆人呢?他或站住或跌倒,自有他的主人在;而且他也必要站住,因为主能使他站住。
\VS{5}有人看这日比那日强;有人看日日都是一样。只是各人心里要意见坚定。
\VS{6}守日的人是为主守的。吃的人是为主吃的,因他感谢 神;不吃的人是为主不吃的,也感谢 神。
\par }{\PP \VS{7}我们没有一个人为自己活,也没有一个人为自己死。
\VS{8}我们若活着,是为主而活;若死了,是为主而死。所以,我们或活或死总是主的人。
\VS{9}因此,基督死了,又活了,为要作死人并活人的主。
\VS{10}你这个人,为什么论断弟兄呢?又为什么轻看弟兄呢?因我们都要站在 神的台前。
\VS{11}{\ADD{经上}}写着:
\par }{\Q 主说:我凭着我的永生起誓:
\par }{\Q 万膝必向我跪拜;
\par }{\Q 万口必向我承认。
\par }{\MM \VS{12}这样看来,我们各人必要将自己的事在 神面前说明。
\par }{\SH 不可使弟兄跌倒
\par }{\PP \VS{13}所以,我们不可再彼此论断,宁可定意谁也不给弟兄放下绊脚跌人之物。
\VS{14}我凭着主耶稣确知深信,凡物本来没有不洁净的;惟独人以为不洁净的,在他就不洁净了。
\VS{15}你若因食物叫弟兄忧愁,就不是按着爱人的道理行。基督已经替他死,你不可因你的食物叫他败坏。
\VS{16}不可叫你的善被人毁谤;
\VS{17}因为 神的国不在乎吃喝,只在乎公义、和平,并圣灵中的喜乐。
\VS{18}在这几样上服事基督的,就为 神所喜悦,又为人所称许。
\VS{19}所以,我们务要追求和睦的事与彼此建立{\ADD{德行}}的事。
\VS{20}不可因食物毁坏 神的工程。凡物固然洁净,但有人因食物叫人跌倒,就是他的罪了。
\VS{21}无论是吃肉是喝酒,是什么别的事,叫弟兄跌倒,一概不做才好。
\VS{22}你有信心,就当在 神面前守着。人在自己以为可行的事上能不自责,就有福了。
\VS{23}若有疑心而吃的,就必有罪,因为他吃不是出于信心。凡不出于信心的都是罪。

\par }\Chap{15}{\SH 不求自己的喜悦,要叫邻舍喜悦
\par }{\PP \VerseOne{1}我们坚固的人应该担代不坚固人的软弱,不求自己的喜悦。
\VS{2}我们各人务要叫邻舍喜悦,使他得益处,建立{\ADD{德行}}。
\VS{3}因为基督也不求自己的喜悦,如{\ADD{经上}}所记:「辱骂你人的辱骂都落在我身上。」
\VS{4}从前所写的{\ADD{圣经}}都是为教训我们写的,叫我们因圣经所生的忍耐和安慰可以得着盼望。
\VS{5}但愿赐忍耐安慰的 神叫你们彼此同心,效法基督耶稣,
\VS{6}一心一口荣耀 神—我们主耶稣基督的父!
\par }{\SH  神的福音一视同仁
\par }{\PP \VS{7}所以,你们要彼此接纳,如同基督接纳你们一样,使荣耀归与 神。
\VS{8}我说,基督是为 神真理作了受割礼人的执事,要证实所应许列祖的话,
\VS{9}并叫外邦人因他的怜悯荣耀 神。如{\ADD{经上}}所记:
\par }{\Q 因此,我要在外邦中称赞你,
\par }{\Q 歌颂你的名;
\par }{\MM \VS{10}又说:
\par }{\Q 你们外邦人当与主的百姓一同欢乐;
\par }{\MM \VS{11}又说:
\par }{\Q 外邦啊,你们当赞美主!
\par }{\Q 万民哪,你们都当颂赞他!
\par }{\MM \VS{12}又有{\PN{以赛亚}}说:
\par }{\Q 将来有{\PN{耶西}}的根,
\par }{\Q 就是那兴起来要治理外邦的;
\par }{\Q 外邦人要仰望他。
\par }{\MM \VS{13}但愿使人有盼望的 神,因信将诸般的喜乐、平安充满你们的心,使你们借着圣灵的能力大有盼望!
\par }{\SH 保罗的宣教使命
\par }{\PP \VS{14}弟兄们,我自己也深信你们是满有良善,充足了诸般的知识,也能彼此劝戒。
\VS{15}但我稍微放胆写信给你们,是要提醒你们的记性,特因 神所给我的恩典,
\VS{16}使我为外邦人作基督耶稣的仆役,作 神福音的祭司,叫所献上的外邦人,因着圣灵成为圣洁,可蒙悦纳。
\VS{17}所以论到 神的事,我在基督耶稣里有可夸的。
\VS{18}除了基督借我做的那些事,我什么都不敢提,只提他借我言语作为,用神迹奇事的能力,并圣灵的能力,使外邦人顺服;
\VS{19}甚至我从{\PN{耶路撒冷}},直转到{\PN{以利哩古}},到处传了基督的福音。
\VS{20}我立了志向,不在基督的名被称过的地方传福音,免得建造在别人的根基上。
\VS{21}就如{\ADD{经上}}所记:
\par }{\Q 未曾闻知他信息的,将要看见;
\par }{\Q 未曾听过的,将要明白。
\par }{\SH 保罗计划访问罗马
\par }{\PP \VS{22}我因多次被拦阻,总不得到你们那里去。
\VS{23}但如今,在这里再没有{\ADD{可传的}}地方,而且这好几年,我切心想望到{\PN{西班牙}}去的时候,可以到你们那里,
\VS{24}盼望从你们那里经过,得见你们,先与你们彼此交往,心里稍微满足,然后蒙你们送行。
\VS{25}但现在,我往{\PN{耶路撒冷}}去供给圣徒。
\VS{26}因为{\PN{马其顿}}和{\PN{亚该亚}}人乐意凑出捐项给{\PN{耶路撒冷}}圣徒中的穷人。
\VS{27}这固然是他们乐意的,其实也算是所欠的债;因外邦人既然在他们属灵的好处上有分,就当把养身之物供给他们。
\VS{28}等我办完了这事,把这善果向他们交付明白,我就要路过你们那里,往{\PN{西班牙}}去。
\VS{29}我也晓得,去的时候必带着基督丰盛的恩典而去。
\par }{\PP \VS{30}弟兄们,我借着我们主耶稣基督,又借着{\ADD{圣}}灵的爱,劝你们与我一同竭力,为我祈求 神,
\VS{31}叫我脱离在{\PN{犹太}}不顺从的人,也叫我为{\PN{耶路撒冷}}所办的捐项可蒙圣徒悦纳,
\VS{32}并叫我顺着 神的旨意,欢欢喜喜地到你们那里,与你们同得安息。
\VS{33}愿赐平安的 神常和你们众人同在。阿们!

\par }\Chap{16}{\SH 问安
\par }{\PP \VerseOne{1}我对你们举荐我们的姊妹{\PN{非比}};她是{\PN{坚革哩}}教会中的女执事。
\VS{2}请你们为主接待她,合乎圣徒的体统。她在何事上要你们帮助,你们就帮助她;因她素来帮助许多人,也帮助了我。
\par }{\PP \VS{3}问{\PN{百基拉}}和{\PN{亚居拉}}安。他们在基督耶稣里与我同工,
\VS{4}也为我的命将自己的颈项置之度外。不但我感谢他们,就是外邦的众教会也感谢他们。
\VS{5}又问在他们家中的教会安。问我所亲爱的{\PN{以拜尼土}}安;他在{\PN{亚细亚}}是归基督初结的果子。
\VS{6}又问{\PN{马利亚}}安;她为你们多受劳苦。
\VS{7}又问我亲属与我一同坐监的{\PN{安多尼古}}和{\PN{犹尼亚}}安;他们在使徒中是有名望的,也是比我先在基督里。
\VS{8}又问我在主里面所亲爱的{\PN{暗伯利}}安。
\VS{9}又问在基督里与我们同工的{\PN{耳巴奴}},并我所亲爱的{\PN{士大古}}安。
\VS{10}又问在基督里经过试验的{\PN{亚比利}}安。问{\PN{亚利多布}}家里的人安。
\VS{11}又问我亲属{\PN{希罗天}}安。问{\PN{拿其数}}家在主里的人安。
\VS{12}又问为主劳苦的{\PN{土非拿}}氏和{\PN{土富撒}}氏安。问可亲爱为主多受劳苦的{\PN{彼息}}氏安。
\VS{13}又问在主蒙拣选的{\PN{鲁孚}}和他母亲安;他的母亲就是我的母亲。
\VS{14}又问{\PN{亚逊其土}}、{\PN{弗勒干}}、{\PN{黑米}}、{\PN{八罗巴}}、{\PN{黑马}},并与他们在一处的弟兄们安。
\VS{15}又问{\PN{非罗罗古}}和{\PN{犹利亚}},{\PN{尼利亚}}和他姊妹,同{\PN{阿林巴}}并与他们在一处的众圣徒安。
\VS{16}你们亲嘴问安,彼此务要圣洁。基督的众教会都问你们安。
\par }{\PP \VS{17}弟兄们,那些离间你们、叫你们跌倒、背乎所学之道的人,我劝你们要留意躲避他们。
\VS{18}因为这样的人不服事我们的主基督,只服事自己的肚腹,用花言巧语诱惑那些老实人的心。
\VS{19}你们的顺服已经传于众人,所以我为你们欢喜;但我愿意你们在善上聪明,在恶上愚拙。
\VS{20}赐平安的 神快要将撒但践踏在你们脚下。愿我主耶稣基督的恩常和你们同在!
\par }{\PP \VS{21}与我同工的{\PN{提摩太}},和我的亲属{\PN{路求}}、{\PN{耶孙}}、{\PN{所西巴德}},问你们安。
\VS{22}我这代笔写信的{\PN{德提}},在主里面问你们安。
\VS{23}那接待我、也接待全教会的{\PN{该犹}}问你们安。
\VS{24}城内管银库的{\PN{以拉都}}和兄弟{\PN{括土}}问你们安。
\par }{\SH 颂赞
\par }{\PP \VS{25}惟有 神能照我所传的福音和所讲的耶稣基督,并照永古隐藏不言的奥秘,坚固你们的心。
\VS{26}这奥秘如今显明出来,而且按着永生 神的命,借众先知的书指示万国的民,使他们信服真道。
\VS{27}愿荣耀,因耶稣基督,归与独一全智的 神,直到永远。阿们!
\par }