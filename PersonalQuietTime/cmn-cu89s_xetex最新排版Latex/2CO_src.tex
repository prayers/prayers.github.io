\NormalFont\ShortTitle{哥林多后书}
{\MT 哥林多后书

\par }\ChapOne{1}{\SH 问候
\par }{\PP \VerseOne{1}奉 神旨意作基督耶稣使徒的{\PN{保罗}}和兄弟{\PN{提摩太}},写信给在{\PN{哥林多}} 神的教会,并{\PN{亚该亚}}遍处的众圣徒。
\VS{2}愿恩惠、平安从 神我们的父和主耶稣基督归与你们!
\par }{\SH 保罗受患难后的感恩
\par }{\PP \VS{3}愿颂赞归与我们的主耶稣基督的父 神,就是发慈悲的父,赐各样安慰的 神。
\VS{4}我们在一切患难中,他就安慰我们,叫我们能用 神所赐的安慰去安慰那遭各样患难的人。
\VS{5}我们既多受基督的苦楚,就靠基督多得安慰。
\VS{6}我们受患难呢,是为叫你们得安慰,得拯救;我们得安慰呢,也是为叫你们得安慰;这安慰能叫你们忍受我们所受的那样苦楚。
\VS{7}我们为你们所存的盼望是确定的,因为知道你们既是同受苦楚,也必同得安慰。
\par }{\PP \VS{8}弟兄们,我们不要你们不晓得,我们从前在{\PN{亚细亚}}遭遇苦难,被压太重,力不能胜,甚至连活命的指望都绝了;
\VS{9}自己心里也断定是必死的,叫我们不靠自己,只靠叫死人复活的 神。
\VS{10}他曾救我们脱离那极大的死亡,现在仍要救我们,并且我们指望他将来还要救我们。
\VS{11}你们以祈祷帮助我们,好叫许多人为我们谢恩,就是为我们因许多人所得的恩。
\par }{\SH 保罗延期访问哥林多
\par }{\PP \VS{12}我们所夸的是自己的良心,见证我们凭着 神的圣洁和诚实;在世为人不靠人的聪明,乃靠 神的恩惠,向你们更是这样。
\VS{13}我们现在写给你们的话,并不外乎你们所念的、所认识的,我也盼望你们到底还是要认识;
\VS{14}正如你们已经有几分认识我们,以我们夸口,好像我们在我们主耶稣的日子以你们夸口一样。
\par }{\PP \VS{15}我既然这样深信,就早有意到你们那里去,叫你们再得益处;
\VS{16}也要从你们那里经过,往{\PN{马其顿}}去,再从{\PN{马其顿}}回到你们那里,叫你们给我送行往{\PN{犹太}}去。
\VS{17}我有此意,岂是反复不定吗?我所起的意,岂是从情欲起的,叫我忽是忽非吗?
\VS{18}我指着信实的 神说,我们向你们所传的道,并没有是而又非的。
\VS{19}因为我和{\PN{西拉}}并{\PN{提摩太}},在你们中间所传 神的儿子耶稣基督,总没有是而又非的,在他只有一是。
\VS{20}神的应许,不论有多少,在基督都是是的。所以借着他也都是实在\FTNT{}{{\FR 1:20: }实在:原文是阿们}的,叫 神因我们得荣耀。
\VS{21}那在基督里坚固我们和你们,并且膏我们的就是 神。
\VS{22}他又用印印了我们,并赐{\ADD{圣}}灵在我们心里作凭据\FTNT{}{{\FR 1:22: }原文是质}。
\par }{\PP \VS{23}我呼吁 神给我的心作见证,我没有往{\PN{哥林多}}去是为要宽容你们。
\VS{24}我们并不是辖管你们的信心,乃是帮助你们的快乐,因为你们凭信才站立得住。

\par }\Chap{2}{\PP \VerseOne{1}我自己定了主意再到你们那里去,必须大家没有忧愁。
\VS{2}倘若我叫你们忧愁,除了我叫那忧愁的人以外,谁能叫我快乐呢?
\VS{3}我曾把这事写给你们,恐怕我到的时候,应该叫我快乐的那些人,反倒叫我忧愁。我也深信,你们众人都以我的快乐为自己的快乐。
\VS{4}我先前心里难过痛苦,多多地流泪,写信给你们,不是叫你们忧愁,乃是叫你们知道我格外地疼爱你们。
\par }{\SH 赦免犯过失的人
\par }{\PP \VS{5}若有叫人忧愁的,他不但叫我忧愁,也是叫你们众人有几分忧愁。我说几分,恐怕说得太重。
\VS{6}这样的人受了众人的责罚也就够了,
\VS{7}倒不如赦免他,安慰他,免得他忧愁太过,甚至沉沦了。
\VS{8}所以我劝你们,要向他显出坚定不移的爱心来。
\VS{9}为此我先前也写信给你们,要试验你们,看你们凡事顺从不顺从。
\VS{10}你们赦免谁,我也赦免谁。我若有所赦免的,是在基督面前为你们赦免的,
\VS{11}免得撒但趁着机会胜过我们,因我们并非不晓得他的诡计。
\par }{\SH 保罗的忧愁和安慰
\par }{\PP \VS{12}我从前为基督的福音到了{\PN{特罗亚}},主也给我开了门。
\VS{13}那时,因为没有遇见兄弟{\PN{提多}},我心里不安,便辞别那里的人往{\PN{马其顿}}去了。
\par }{\PP \VS{14}感谢 神!常率领我们在基督里夸胜,并借着我们在各处显扬那因认识基督而有的香气。
\VS{15}因为我们在 神面前,无论在得救的人身上或灭亡的人身上,都有基督馨香之气。
\VS{16}在这等人,就作了死的香气叫他死;在那等人,就作了活的香气叫他活。这事谁能当得起呢?
\VS{17}我们不像那许多人,为利混乱 神的道;乃是由于诚实,由于 神,在 神面前凭着基督讲道。

\par }\Chap{3}{\SH 新约的执事
\par }{\PP \VerseOne{1}我们岂是又举荐自己吗?岂像别人用人的荐信给你们或用你们的荐信给人吗?
\VS{2}你们就是我们的荐信,写在我们的心里,被众人所知道所念诵的。
\VS{3}你们明显是基督的信,借着我们修成的。不是用墨写的,乃是用永生 神的灵写的;不是写在石版上,乃是写在心版上。
\par }{\PP \VS{4}我们因基督,所以在 神面前才有这样的信心。
\VS{5}并不是我们凭自己能承担什么事;我们所能承担的,乃是出于 神。
\VS{6}他叫我们能承当这新约的执事,不是凭着字句,乃是凭着精意;因为那字句是叫人死,精意\FTNT{}{{\FR 3:6: }或译:圣灵}是叫人活。
\VS{7}那用字刻在石头上属死的职事尚且有荣光,甚至{\PN{以色列}}人因{\PN{摩西}}面上的荣光,不能定睛看他的脸;这荣光原是渐渐退去的,
\VS{8}何况那属灵的职事岂不更有荣光吗?
\VS{9}若是定罪的职事有荣光,那称义的职事荣光就越发大了。
\VS{10}那从前有荣光的,因这极大的荣光就算不得有荣光了;
\VS{11}若那废掉的有荣光,这长存的就更有荣光了。
\par }{\PP \VS{12}我们既有这样的盼望,就大胆讲说,
\VS{13}不像{\PN{摩西}}将帕子蒙在脸上,叫{\PN{以色列}}人不能定睛看到那将废者{\ADD{的}}结局。
\VS{14}但他们的心地刚硬,直到今日诵读旧约的时候,这帕子还没有揭去。这帕子在基督里已经废去了。
\VS{15}然而直到今日,每逢诵读{\PN{摩西}}{\ADD{书}}的时候,帕子还在他们心上。
\VS{16}但他们的心几时归向主,帕子就几时除去了。
\VS{17}主就是那灵;主的灵在哪里,那里就得以自由。
\VS{18}我们众人既然敞着脸得以看见主的荣光,好像从镜子里返照,就变成主的形状,荣上加荣,如同从主的灵变成的。

\par }\Chap{4}{\SH 在瓦器里的宝贝
\par }{\PP \VerseOne{1}我们既然蒙怜悯,受了这职分,就不丧胆,
\VS{2}乃将那些暗昧可耻的事弃绝了;不行诡诈,不谬讲 神的道理,只将真理表明出来,好在 神面前把自己荐与各人的良心。
\VS{3}如果我们的福音蒙蔽,就是蒙蔽在灭亡的人身上。
\VS{4}此等不信之人被这世界的神弄瞎了心眼,不叫基督荣耀福音的光照着他们。基督本是 神的像。
\VS{5}我们原不是传自己,乃是传基督耶稣为主,并且自己因耶稣作你们的仆人。
\VS{6}那吩咐光从黑暗里照出来的 神,已经照在我们心里,叫我们得知 神荣耀的光显在耶稣基督的面上。
\par }{\PP \VS{7}我们有这宝贝放在瓦器里,要显明这莫大的能力是出于 神,不是出于我们。
\VS{8}我们四面受敌,却不被困住;心里作难,却不至失望;
\VS{9}遭逼迫,却不被丢弃;打倒了,却不至死亡。
\VS{10}身上常带着耶稣的死,使耶稣的生也显明在我们身上。
\VS{11}因为我们这活着的人是常为耶稣被交于死地,使耶稣的生在我们这必死的身上显明出来。
\VS{12}这样看来,死是在我们身上发动,生却在你们身上发动。
\VS{13}但我们既有信心,正如{\ADD{经上}}记着说:「我因信,所以如此说话。」我们也信,所以也说话。
\VS{14}自己知道那叫主耶稣复活的,也必叫我们与耶稣一同复活,并且叫我们与你们一同站在他面前。
\VS{15}凡事都是为你们,好叫恩惠因人多越发加增,感谢格外显多,以致荣耀归与 神。
\par }{\SH 靠信心生活
\par }{\PP \VS{16}所以,我们不丧胆。外体虽然毁坏,内心却一天新似一天。
\VS{17}我们这至暂至轻的苦楚,要为我们成就极重无比、永远的荣耀。
\VS{18}原来我们不是顾念所见的,乃是顾念所不见的;因为所见的是暂时的,所不见的是永远的。

\par }\Chap{5}{\PP \VerseOne{1}我们原知道,我们这地上的帐棚若拆毁了,必得 神所造,不是人手所造,在天上永存的房屋。
\VS{2}我们在这帐棚里叹息,深想得那从天上来的房屋,好像穿上{\ADD{衣服}};
\VS{3}倘若穿上,被遇见的时候就不至于赤身了。
\VS{4}我们在这帐棚里叹息劳苦,并非愿意脱下这个,乃是愿意穿上那个,好叫这必死的被生命吞灭了。
\VS{5}为此,培植我们的就是 神,他又赐给我们{\ADD{圣}}灵作凭据\FTNT{}{{\FR 5:5: }原文是质}。
\par }{\PP \VS{6}所以,我们时常坦然无惧,并且晓得我们住在身内,便与主相离。
\VS{7}因我们行事为人是凭着信心,不是凭着眼见。
\VS{8}我们坦然无惧,是更愿意离开身体与主同住。
\VS{9}所以,无论是住在身内,离开身外,我们立了志向,要得主的喜悦。
\VS{10}因为我们众人必要在基督台前显露出来,叫各人按着本身所行的,或善或恶受报。
\par }{\SH 劝人与 神和好的职分
\par }{\PP \VS{11}我们既知道主是可畏的,所以劝人。但我们在 神面前是显明的,盼望在你们的良心里也是显明的。
\VS{12}我们不是向你们再举荐自己,乃是叫你们因我们有可夸之处,好对那凭外貌不凭内心夸口的人,有言可答。
\VS{13}我们若果颠狂,是为 神;若果谨守,是为你们。
\VS{14}原来基督的爱激励我们;因我们想,一人既替众人死,众人就都死了;
\VS{15}并且他替众人死,是叫那些活着的人不再为自己活,乃为替他们死而复活的{\ADD{主}}活。
\par }{\PP \VS{16}所以,我们从今以后,不凭着外貌\FTNT{}{{\FR 5:16: }原文是肉体;本节同}认人了。虽然凭着外貌认过基督,如今却不再{\ADD{这样}}认他了。
\VS{17}若有人在基督里,他就是新造的人,旧事已过,都变成新的了。
\VS{18}一切都是出于 神;他借着基督使我们与他和好,又将劝人与他和好的职分赐给我们。
\VS{19}这就是 神在基督里,叫世人与自己和好,不将他们的过犯归到他们身上,并且将这和好的道理托付了我们。
\VS{20}所以,我们作基督的使者,就好像 神借我们劝你们一般。我们替基督求你们与 神和好。
\VS{21}神使那无罪\FTNT{}{{\FR 5:21: }无罪:原文是不知罪}的,替我们成为罪,好叫我们在他里面成为 神的义。

\par }\Chap{6}{\PP \VerseOne{1}我们与 {\ADD{神}}同工的,也劝你们不可徒受他的恩典。
\VS{2}因为他说:「在悦纳的时候,我应允了你;在拯救的日子,我搭救了你。」看哪,现在正是悦纳的时候!现在正是拯救的日子;
\VS{3}我们凡事都不叫人有妨碍,免得这职分被人毁谤;
\VS{4}反倒在各样的事上表明自己是 神的用人,就如在许多的忍耐、患难、穷乏、困苦、
\VS{5}鞭打、监禁、扰乱、勤劳、警醒、不食、
\VS{6}廉洁、知识、恒忍、恩慈、圣灵的感化、无伪的爱心、
\VS{7}真实的道理、 神的大能;仁义的兵器在左在右;
\VS{8}荣耀、羞辱,恶名、美名;似乎是诱惑人的,却是诚实的;
\VS{9}似乎不为人所知,却是人所共知的;似乎要死,却是活着的;似乎受责罚,却是不至丧命的;
\VS{10}似乎忧愁,却是常常快乐的;似乎贫穷,却是叫许多人富足的;似乎一无所有,却是样样都有的。
\par }{\PP \VS{11}{\PN{哥林多}}人哪,我们向你们,口是张开的,心是宽宏的。
\VS{12}你们狭窄,原不在乎我们,是在乎自己的心肠狭窄。
\VS{13}你们也要照样用宽宏的心报答我。我这话正像对自己的孩子说的。
\par }{\SH 永生 神的殿
\par }{\PP \VS{14}你们和不信的原不相配,不要同负一轭。义和不义有什么相交呢?光明和黑暗有什么相通呢?
\VS{15}基督和彼列\FTNT{}{{\FR 6:15: }彼列就是撒但的别名}有什么相和呢?信{\ADD{主}}的和不信{\ADD{主}}的有什么相干呢?
\VS{16}神的殿和偶像有什么相同呢?因为我们是永生 神的殿,就如 神曾说:
\par }{\Q 我要在他们中间居住,
\par }{\Q 在他们中间来往;
\par }{\Q 我要作他们的 神;
\par }{\Q 他们要作我的子民。
\par }{\Q \VS{17}又说:你们务要从他们中间出来,
\par }{\Q 与他们分别;
\par }{\Q 不要沾不洁净的物,我就收纳你们。
\par }{\Q \VS{18}我要作你们的父;
\par }{\Q 你们要作我的儿女。
\par }{\Q 这是全能的主说的。

\par }\Chap{7}{\PP \VerseOne{1}亲爱的{\ADD{弟兄}}啊,我们既有这等应许,就当洁净自己,除去身体、灵魂一切的污秽,敬畏 神,得以成圣。
\par }{\SH 保罗为教会悔改而欢乐
\par }{\PP \VS{2}你们要心地宽大收纳我们。我们未曾亏负谁,未曾败坏谁,未曾占谁的便宜。
\VS{3}我说这话,不是要定你们的罪。我已经说过,你们常在我们心里,情愿与你们同生同死。
\VS{4}我大大地放胆,向你们说话;我因你们多多夸口,满得安慰;我们在一切患难中分外的快乐。
\par }{\PP \VS{5}我们从前就是到了{\PN{马其顿}}的时候,身体也不得安宁,周围遭患难,外有争战,内有惧怕。
\VS{6}但那安慰丧气之人的 神借着{\PN{提多}}来安慰了我们;
\VS{7}不但借着他来,也借着他从你们所得的安慰,安慰了我们;因他把你们的想念、哀恸,和向我的热心,都告诉了我,叫我更加欢喜。
\VS{8}我先前写信叫你们忧愁,我后来虽然懊悔,如今却不懊悔;因我知道,那信叫你们忧愁不过是暂时的。
\VS{9}如今我欢喜,不是因你们忧愁,是因你们从忧愁中生出懊悔来。你们依着 神的意思忧愁,凡事就不至于因我们受亏损了。
\VS{10}因为依着 神的意思忧愁,就生出没有后悔的懊悔来,以致得救;但世俗的忧愁是叫人死。
\VS{11}你看,你们依着 神的意思忧愁,从此就生出何等的殷勤、自诉、自恨、恐惧、想念、热心、责罚\FTNT{}{{\FR 7:11: }或译:自责}。在这一切事上,你们都表明自己是洁净的。
\VS{12}我虽然从前写信给你们,却不是为那亏负人的,也不是为那受人亏负的,乃要在 神面前把你们顾念我们的热心表明出来。
\VS{13}故此,我们得了安慰。
\par }{\PP 并且在安慰之中,因你们众人使{\PN{提多}}心里畅快欢喜,我们就更加欢喜了。
\VS{14}我若对{\PN{提多}}夸奖了你们什么,也觉得没有惭愧;因我对{\PN{提多}}夸奖你们的话成了真的,正如我对你们所说的话也都是真的。
\VS{15}并且{\PN{提多}}想起你们众人的顺服,是怎样恐惧战兢地接待他,他爱你们的心肠就越发热了。
\VS{16}我如今欢喜,能在凡事上为你们放心。

\par }\Chap{8}{\SH 慷慨的捐助
\par }{\PP \VerseOne{1}弟兄们,我把 神赐给{\PN{马其顿}}众教会的恩告诉你们,
\VS{2}就是他们在患难中受大试炼的时候,仍有满足的快乐,在极穷之间还格外显出他们乐捐的厚恩。
\VS{3}我可以证明,他们是按着力量,而且也过了力量,自己甘心乐意地捐助,
\VS{4}再三地求我们,准他们在这供给圣徒的恩情上有分;
\VS{5}并且他们所做的,不但照我们所想望的,更照 神的旨意先把自己献给主,又归附了我们。
\VS{6}因此我劝{\PN{提多}},既然在你们中间开办这慈惠的事,就当办成了。
\VS{7}你们既然在信心、口才、知识、热心,和待我们的爱心上,都格外显出满足来,就当在这慈惠的事上也格外显出满足来。
\par }{\PP \VS{8}我说这话,不是吩咐你们,乃是借着别人的热心试验你们爱心的实在。
\VS{9}你们知道我们主耶稣基督的恩典:他本来富足,却为你们成了贫穷,叫你们因他的贫穷,可以成为富足。
\VS{10}我在这事上把我的意见告诉你们,是与你们有益;因为你们下手办这事,而且起此心意,已经有一年了,
\VS{11}如今就当办成这事。既有愿做的心,也当照你们所有的去办成。
\VS{12}因为人若有愿做的心,必蒙悦纳,乃是照他所有的,并不是照他所无的。
\VS{13}我原不是要别人轻省,你们受累,
\VS{14}乃要均平,就是要你们的富余,现在{\ADD{可以补}}他们的不足,使他们的富余,{\ADD{将来}}也可以{\ADD{补}}你们的不足,这就均平了。
\VS{15}如{\ADD{经上}}所记:
\par }{\Q 多{\ADD{收}}的也没有余;
\par }{\Q 少{\ADD{收}}的也没有缺。
\par }{\SH 提多和他的同工
\par }{\PP \VS{16}多谢 神,感动{\PN{提多}}的心,叫他待你们殷勤,像我一样。
\VS{17}他固然是听了我的劝,但自己更是热心,情愿往你们那里去。
\VS{18}我们还打发一位兄弟和他同去,这人在福音上得了众教会的称赞。
\VS{19}不但这样,他也被众教会挑选,和我们同行,把所托与我们的这捐资送到了,可以荣耀主,又表明我们乐意的心。
\VS{20}这就免得有人因我们收的捐银很多,就挑我们的不是。
\VS{21}我们留心行光明的事,不但在主面前,就在人面前也是这样。
\VS{22}我们又打发一位兄弟同去;这人的热心,我们在许多事上屡次试验过。现在他因为深信你们,就更加热心了。
\VS{23}论到{\PN{提多}},他是我的同伴,一同为你们劳碌的。论到那两位兄弟,他们是众教会的使者,是基督的荣耀。
\VS{24}所以,你们务要在众教会面前显明你们爱心的凭据,并我所夸奖你们的凭据。

\par }\Chap{9}{\SH 帮助圣徒的捐款
\par }{\PP \VerseOne{1}论到供给圣徒的事,我不必写信给你们;
\VS{2}因为我知道你们乐意的心,常对{\PN{马其顿}}人夸奖你们,说{\PN{亚该亚}}人预备好了,已经有一年了;并且你们的热心激动了许多人。
\VS{3}但我打发那几位弟兄去,要叫你们照我的话预备妥当,免得我们在这事上夸奖你们的话落了空。
\VS{4}万一有{\PN{马其顿}}人与我同去,见你们没有预备,就叫我们所确信的,反成了羞愧;你们羞愧,更不用说了。
\VS{5}因此,我想不得不求那几位弟兄先到你们那里去,把从前所应许的捐资预备妥当,就显出你们所捐的是出于乐意,不是出于勉强。
\par }{\PP \VS{6}「少种的少收,多种的多收」,这话是真的。
\VS{7}各人要随本心所酌定的,不要作难,不要勉强,因为捐得乐意的人是 神所喜爱的。
\VS{8}神能将各样的恩惠多多地加给你们,使你们凡事常常充足,能多行各样善事。
\VS{9}如{\ADD{经上}}所记:
\par }{\Q 他施舍{\ADD{钱财}},周济贫穷;
\par }{\Q 他的仁义存到永远。
\par }{\MM \VS{10}那赐种给撒种的,赐粮给人吃的,必多多加给你们种地的种子,又增添你们仁义的果子;
\VS{11}叫你们凡事富足,可以多多施舍,就借着我们使感谢归于 神。
\VS{12}因为办这供给的事,不但补圣徒的缺乏,而且叫许多人越发感谢 神。
\VS{13}他们从这供给的事上得了凭据,知道你们承认基督顺服他的福音,多多地捐钱给他们和众人,便将荣耀归与 神。
\VS{14}他们也因 神极大的恩赐显在你们心里,就切切地想念你们,为你们祈祷。
\VS{15}感谢 神,因他有说不尽的恩赐!

\par }\Chap{10}{\SH 保罗为他的工作辩护
\par }{\PP \VerseOne{1}我—{\PN{保罗}},就是与你们见面的时候是谦卑的,不在你们那里的时候向你们是勇敢的,如今亲自借着基督的温柔、和平劝你们。
\VS{2}有人以为我是凭着血气行事,我也以为必须用勇敢待这等人;求你们不要叫我在你们那里的时候,有这样的勇敢。
\VS{3}因为我们虽然在血气中行事,却不凭着血气争战。
\VS{4}我们争战的兵器本不是属血气的,乃是在 神面前有能力,可以攻破坚固的营垒,
\VS{5}将各样的计谋,各样拦阻人认识 神的那些自高之事,一概攻破了,又将人所有的心意夺回,使他都顺服基督。
\VS{6}并且我已经预备好了,等你们十分顺服的时候,要责罚那一切不顺服的人。
\par }{\PP \VS{7}你们是看眼前的吗?倘若有人自信是属基督的,他要再想想,他如何属基督,我们也是如何属基督的。
\VS{8}主赐给我们权柄,是要造就你们,并不是要败坏你们;我就是为这权柄稍微夸口,也不至于惭愧。
\VS{9}我{\ADD{说这话}},免得你们以为我写信是要威吓你们;
\VS{10}因为有人说:「他的信又沉重又厉害,及至见面,却是气貌不扬,言语粗俗的。」
\VS{11}这等人当想,我们不在那里的时候,信上的言语如何,见面的时候,行事也必如何。
\VS{12}因为我们不敢将自己和那自荐的人同列相比。他们用自己度量自己,用自己比较自己,乃是不通达的。
\VS{13}我们不愿意分外夸口,只要照 神所量给我们的界限搆到你们那里。
\VS{14}我们并非过了自己的界限,好像搆不到你们那里;因为我们早到你们那里,传了基督的福音。
\par }{\SH 当指着主夸口
\par }{\PP \VS{15}我们不仗着别人所劳碌的,分外夸口;但指望你们信心增长的时候,所量给我们的界限,就可以因着你们更加开展,
\VS{16}得以将福音传到你们以外的地方;并不是在别人界限之内,借着他现成的事夸口。
\VS{17}但夸口的,当指着主夸口。
\VS{18}因为蒙悦纳的,不是自己称许的,乃是主所称许的。

\par }\Chap{11}{\SH 保罗和假使徒
\par }{\PP \VerseOne{1}但愿你们宽容我这一点愚妄,其实你们原是宽容我的。
\VS{2}我为你们起的愤恨,原是 神那样的愤恨。因为我曾把你们许配一个丈夫,要把你们如同贞洁的童女,献给基督。
\VS{3}我只怕你们的心或偏于邪,失去那向基督所存纯一清洁的心,就像蛇用诡诈诱惑了{\PN{夏娃}}一样。
\VS{4}假如有人来另传一个耶稣,不是我们所传过的;或者你们另受一个灵,不是你们所受过的;或者另得一个福音,不是你们所得过的;你们容让他也就罢了。
\VS{5}但我想,我一点不在那些最大的使徒以下。
\VS{6}我的言语虽然粗俗,我的知识却不粗俗。这是我们在凡事上向你们众人显明出来的。
\par }{\PP \VS{7}我因为白白传 神的福音给你们,就自居卑微,叫你们高升,这算是我犯罪吗?
\VS{8}我亏负了别的教会,向他们取了工价来给你们效力。
\VS{9}我在你们那里缺乏的时候,并没有累着你们一个人;因我所缺乏的,那从{\PN{马其顿}}来的弟兄们都补足了。我向来凡事谨守,后来也必谨守,总不至于累着你们。
\VS{10}既有基督的诚实在我里面,就无人能在{\PN{亚该亚}}一带地方阻挡我这自夸。
\VS{11}为什么呢?是因我不爱你们吗?这有 神知道。
\par }{\PP \VS{12}我现在所做的,后来还要做,为要断绝那些寻机会人的机会,使他们在所夸的事上也不过与我们一样。
\VS{13}那等人是假使徒,行事诡诈,装作基督使徒的模样。
\VS{14}这也不足为怪,因为连撒但也装作光明的天使。
\VS{15}所以他的差役,若装作仁义的差役,也不算希奇。他们的结局必然照着他们的行为。
\par }{\SH 保罗作使徒所受的患难
\par }{\PP \VS{16}我再说,人不可把我看作愚妄的。纵然如此,也要把我当作愚妄人接纳,叫我可以略略自夸。
\VS{17}我说的话不是奉主命说的,乃是像愚妄人放胆自夸;
\VS{18}既有好些人凭着血气自夸,我也要自夸了。
\VS{19}你们既是精明人,就能甘心忍耐愚妄人。
\VS{20}假若有人强你们作奴仆,或侵吞你们,或掳掠你们,或侮慢你们,或打你们的脸,你们都能忍耐他。
\VS{21}我说这话是羞辱自己,好像我们从前是软弱的。然而,人在何事上勇敢,(我说句愚妄话,)我也勇敢。
\VS{22}他们是{\PN{希伯来}}人吗?我也是。他们是{\PN{以色列}}人吗?我也是。他们是{\PN{亚伯拉罕}}的后裔吗?我也是。
\VS{23}他们是基督的仆人吗?(我说句狂话,)我更是。我比他们多受劳苦,多下监牢,受鞭打是过重的,冒死是屡次有的。
\VS{24}被{\PN{犹太}}人鞭打五次,每次四十减去一下;
\VS{25}被棍打了三次;被石头打了一次;遇着船坏三次,一昼一夜在深海里。
\VS{26}又屡次行远路,遭江河的危险、盗贼的危险、同族的危险、外邦人的危险、城里的危险、旷野的危险、海中的危险、假弟兄的危险。
\VS{27}受劳碌、受困苦,多次不得睡,又饥又渴,多次不得食,受寒冷,赤身露体。
\VS{28}除了这外面的事,还有为众教会挂心的事,天天压在我身上。
\VS{29}有谁软弱,我不软弱呢?有谁跌倒,我不焦急呢?
\par }{\PP \VS{30}我若必须自夸,就夸那关乎我软弱的事便了。
\VS{31}那永远可称颂之主耶稣的父 神知道我不说谎。
\VS{32}在{\PN{大马士革}}的{\PN{亚哩达}}王手下的提督把守{\PN{大马士革}}城,要捉拿我,
\VS{33}我就从窗户中,在筐子里,从城墙上被人缒下去,脱离了他的手。

\par }\Chap{12}{\SH 异象和启示
\par }{\PP \VerseOne{1}我自夸固然无益,但我是不得已的。如今我要说到主的显现和启示。
\VS{2}我认得一个在基督里的人,他前十四年被提到第三层天上去;(或在身内,我不知道;或在身外,我也不知道;只有 神知道。)
\VS{3}我认得这人;(或在身内,或在身外,我都不知道,只有 神知道。)
\VS{4}他被提到乐园里,听见隐秘的言语,是人不可说的。
\VS{5}为这人,我要夸口;但是为我自己,除了{\ADD{我的}}软弱以外,我并不夸口。
\VS{6}我就是愿意夸口也不算狂,因为我必说实话;只是我禁止不说,恐怕有人把我看高了,过于他在我身上所看见所听见的。
\VS{7}又恐怕我因所得的启示甚大,就过于自高,所以有一根刺加在我肉体上,就是撒但的差役要攻击我,免得我过于自高。
\VS{8}为这事,我三次求过主,叫这刺离开我。
\VS{9}他对我说:「我的恩典够你用的,因为我的能力是在人的软弱上显得完全。」所以,我更喜欢夸自己的软弱,好叫基督的能力覆庇我。
\VS{10}我为基督的缘故,就以软弱、凌辱、急难、逼迫、困苦为可喜乐的;因我什么时候软弱,什么时候就刚强了。
\par }{\SH 保罗关心哥林多教会
\par }{\PP \VS{11}我成了愚妄人,是被你们强逼的。我本该被你们称许才是。我虽算不了什么,却没有一件事在那些最大的使徒以下。
\VS{12}我在你们中间,用百般的忍耐,借着神迹、奇事、异能显出使徒的凭据来。
\VS{13}除了我不累着你们这一件事,你们还有什么事不及别的教会呢?这不公之处,求你们饶恕我吧。
\par }{\PP \VS{14}如今,我打算第三次到你们那里去,也必不累着你们;因我所求的是你们,不是你们的{\ADD{财物}}。儿女不该为父母积财,父母该为儿女积财。
\VS{15}我也甘心乐意为你们的灵魂费财费力。难道我越发爱你们,就越发少得你们的爱吗?
\VS{16}罢了,我自己并没有累着你们,{\ADD{你们却有人说}},我是诡诈,用心计牢笼你们。
\VS{17}我所差到你们那里去的人,我借着他们一个人占过你们的便宜吗?
\VS{18}我劝了{\PN{提多}}{\ADD{到你们那里去}},又差那位兄弟与他同去。{\PN{提多}}占过你们的便宜吗?我们行事,不同是一个心灵\FTNT{}{{\FR 12:18: }或译:圣灵}吗?不同是一个脚踪吗?
\par }{\PP \VS{19}你们到如今,还想我们是向你们分诉;我们本是在基督里当 神面前说话。亲爱的{\ADD{弟兄}}啊,一切的事都是为造就你们。
\VS{20}我怕我再来的时候,见你们不合我所想望的,你们见我也不合你们所想望的;又怕有纷争、嫉妒、恼怒、结党、毁谤、谗言、狂傲、混乱的事。
\VS{21}且怕我来的时候,我的 神叫我在你们面前惭愧,又因许多人从前犯罪,行污秽、奸淫、邪荡的事不肯悔改,我就忧愁。

\par }\Chap{13}{\SH 警告和问候
\par }{\PP \VerseOne{1}这是我第三次要到你们那里去。「凭两三个人的口作见证,句句都要定准。」
\VS{2}我从前说过,如今不在你们那里又说,正如我第二次见你们的时候所说的一样,就是对那犯了罪的和其余的人说:「我若再来,必不宽容。」
\VS{3}你们既然寻求基督在我里面说话的凭据,{\ADD{我必不宽容}}。因为,基督在你们身上不是软弱的,在你们里面是有大能的。
\VS{4}他因软弱被钉在十字架上,却因 神的大能仍然活着。我们也是这样同他软弱,但因 神向你们所显的大能,也必与他同活。
\par }{\PP \VS{5}你们总要自己省察有信心没有,也要自己试验。岂不知你们若不是可弃绝的,就有耶稣基督在你们心里吗?
\VS{6}我却盼望你们晓得,我们不是可弃绝的人。
\VS{7}我们求 神,叫你们一件恶事都不做;这不是要显明我们是蒙悦纳的,是要你们行事端正,任凭人看我们是被弃绝的吧!
\VS{8}我们凡事不能敌挡真理,只能扶助真理。
\VS{9}即便我们软弱,你们刚强,我们也欢喜;并且我们所求的,就是你们作完全人。
\VS{10}所以,我不在你们那里的时候,把这话写给你们,好叫我见你们的时候,不用照主所给我的权柄严厉地待你们。这权柄原是为造就人,并不是为败坏人。
\par }{\PP \VS{11}还有末了的话:愿弟兄们都喜乐。要作完全人;要受安慰;要同心合意;要彼此和睦。如此,仁爱和平的 神必常与你们同在。
\VS{12}你们亲嘴问安,彼此务要圣洁。
\VS{13}众圣徒都问你们安。
\par }{\PP \VS{14}愿主耶稣基督的恩惠、 神的慈爱、圣灵的感动{\ADD{常}}与你们众人同在!
\par }