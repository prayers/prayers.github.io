\NormalFont\ShortTitle{加拉太书}
{\MT 加拉太书

\par }\ChapOne{1}{\SH 问候
\par }{\PP \VerseOne{1}作使徒的{\PN{保罗}}(不是由于人,也不是借着人,乃是借着耶稣基督,与叫他从死里复活的父 神)
\VS{2}和一切与我同在的众弟兄,写信给{\PN{加拉太}}的各教会。
\VS{3}愿恩惠、平安从父 神与我们的主耶稣基督归与你们!
\VS{4}基督照我们父 神的旨意,为我们的罪舍己,要救我们脱离这罪恶的世代。
\VS{5}但愿荣耀归于 神,直到永永远远。阿们!
\par }{\SH 没有别的福音
\par }{\PP \VS{6}我希奇你们这么快离开那借着基督之恩召你们的,去从别的福音。
\VS{7}那并不是{\ADD{福音}},不过有些人搅扰你们,要把基督的福音更改了。
\VS{8}但无论是我们,是天上来的使者,若传福音给你们,与我们所传给你们的不同,他就应当被咒诅。
\VS{9}我们已经说了,现在又说,若有人传福音给你们,与你们所领受的不同,他就应当被咒诅。
\par }{\PP \VS{10}我现在是要得人的心呢?还是要得 神的心呢?我岂是讨人的喜欢吗?若仍旧讨人的喜欢,我就不是基督的仆人了。
\par }{\SH 保罗怎样成为使徒
\par }{\PP \VS{11}弟兄们,我告诉你们,我素来所传的福音不是出于人的意思。
\VS{12}因为我不是从人领受的,也不是人教导我的,乃是从耶稣基督启示来的。
\par }{\PP \VS{13}你们听见我从前在{\PN{犹太}}教中所行的事,怎样极力逼迫残害 神的教会。
\VS{14}我又在{\PN{犹太}}教中,比我本国许多同岁的人更有长进,为我祖宗的遗传更加热心。
\VS{15}然而,那把我从母腹里分别出来、又施恩召我的 神,
\VS{16}既然乐意将他儿子启示在我心里,叫我把他传在外邦人中,我就没有与属血气的人商量,
\VS{17}也没有上{\PN{耶路撒冷}}去见那些比我先作使徒的,惟独往{\PN{阿拉伯}}去,后又回到{\PN{大马士革}}。
\par }{\PP \VS{18}过了三年,才上{\PN{耶路撒冷}}去见{\PN{矶法}},和他同住了十五天。
\VS{19}至于别的使徒,除了主的兄弟{\PN{雅各}},我都没有看见。
\VS{20}我写给你们的不是谎话,这是我在 神面前说的。
\VS{21}以后我到了{\PN{叙利亚}}和{\PN{基利家}}境内。
\VS{22}那时,{\PN{犹太}}信基督的各教会都没有见过我的面。
\VS{23}不过听说那从前逼迫我们的,现在传扬他原先所残害的真道。
\VS{24}他们就为我的缘故,归荣耀给 神。

\par }\Chap{2}{\SH 保罗被其他的使徒接纳
\par }{\PP \VerseOne{1}过了十四年,我同{\PN{巴拿巴}}又上{\PN{耶路撒冷}}去,并带着{\PN{提多}}同去。
\VS{2}我是奉启示上去的,把我在外邦人中所传的福音对弟兄们陈说;却是背地里对那有名望之人说的,惟恐我现在,或是从前,徒然奔跑。
\VS{3}但与我同去的{\PN{提多}},虽是{\PN{希腊}}人,也没有勉强他受割礼;
\VS{4}因为有偷着引进来的假弟兄,私下窥探我们在基督耶稣里的自由,要叫我们作奴仆。
\VS{5}我们就是一刻的工夫也没有容让顺服他们,为要叫福音的真理仍存在你们中间。
\VS{6}至于那些有名望的,不论他是何等人,都与我无干。 神不以外貌取人。那些有名望的,并没有加增我什么,
\VS{7}反倒看见了主托我传福音给那未受割礼的人,正如托{\PN{彼得}}传福音给那受割礼的人。(
\VS{8}那感动{\PN{彼得}}、叫他为受割礼之人作使徒的,也感动我,叫我为外邦人作使徒;)
\VS{9}又知道所赐给我的恩典,那称为{\ADD{教会}}柱石的{\PN{雅各}}、{\PN{矶法}}、{\PN{约翰}},就向我和{\PN{巴拿巴}}用右手行相交之礼,叫我们往外邦人那里去,他们往受割礼的人那里去。
\VS{10}只是愿意我们记念穷人;这也是我本来热心去行的。
\par }{\SH 保罗在安提阿责备彼得
\par }{\PP \VS{11}后来,{\PN{矶法}}到了{\PN{安提阿}};因他有可责之处,我就当面抵挡他。
\VS{12}从{\PN{雅各}}那里来的人未到以先,他和外邦人一同吃饭,及至他们来到,他因怕奉割礼的人,就退去与外邦人隔开了。
\VS{13}其余的{\PN{犹太}}人也都随着他装假,甚至连{\PN{巴拿巴}}也随伙装假。
\VS{14}但我一看见他们行的不正,与福音的真理不合,就在众人面前对{\PN{矶法}}说:「你既是{\PN{犹太}}人,若随外邦人行事,不随{\PN{犹太}}人行事,怎么还勉强外邦人随{\PN{犹太}}人呢?」
\par }{\SH 犹太人和外邦人都因信得救
\par }{\PP \VS{15}我们这生来的{\PN{犹太}}人,不是外邦的罪人;
\VS{16}既知道人称义不是因行律法,乃是因信耶稣基督,连我们也信了基督耶稣,使我们因信基督称义,不因行律法称义;因为凡有血气的,没有一人因行律法称义。
\VS{17}我们若求在基督里称义,却仍旧是罪人,难道基督是叫人犯罪的吗?断乎不是!
\VS{18}我素来所拆毁的,若重新建造,这就证明自己是犯罪的人。
\VS{19}我因律法,就向律法死了,叫我可以向 神活着。
\VS{20}我已经与基督同钉十字架,现在活着的不再是我,乃是基督在我里面活着;并且我如今在肉身活着,是因信 神的儿子而活;他是爱我,为我舍己。
\VS{21}我不废掉 神的恩;义若是借着律法得的,基督就是徒然死了。

\par }\Chap{3}{\SH 律法和信心
\par }{\PP \VerseOne{1}无知的{\PN{加拉太}}人哪,耶稣基督钉十字架,已经活画在你们眼前,谁又迷惑了你们呢?
\VS{2}我只要问你们这一件:你们受了{\ADD{圣}}灵,是因行律法呢?是因听信{\ADD{福音}}呢?
\VS{3}你们既靠{\ADD{圣}}灵入门,如今还靠肉身成全吗?你们是这样的无知吗?
\VS{4}你们受苦如此之多,都是徒然的吗?难道果真是徒然的吗?
\VS{5}那赐给你们{\ADD{圣}}灵,又在你们中间行异能的,是因你们行律法呢?是因你们听信{\ADD{福音}}呢?
\VS{6}正如「{\PN{亚伯拉罕}}信 神,这就算为他的义」。
\par }{\PP \VS{7}所以,你们要知道:那以信为本的人,就是{\PN{亚伯拉罕}}的子孙。
\VS{8}并且圣经既然预先看明, 神要叫外邦人因信称义,就早已传福音给{\PN{亚伯拉罕}},{\ADD{说}}:「万国都必因你得福。」
\VS{9}可见那以信为本的人和有信心的{\PN{亚伯拉罕}}一同得福。
\VS{10}凡以行律法为本的,都是被咒诅的;因为{\ADD{经上}}记着:「凡不常照律法书上所记一切之事去行的,就被咒诅。」
\VS{11}没有一个人靠着律法在 神面前称义,这是明显的;因为{\ADD{经上}}说:「义人必因信得生。」
\VS{12}律法原不本乎信,只{\ADD{说}}:「行这些事的,就必因此活着。」
\VS{13}基督既为我们受\FTNT{}{{\FR 3:13: }原文是成}了咒诅,就赎出我们脱离律法的咒诅;因为{\ADD{经上}}记着:「凡挂在木头上都是被咒诅的。」
\VS{14}这便叫{\PN{亚伯拉罕}}的福,因基督耶稣可以临到外邦人,使我们因信得着所应许的{\ADD{圣}}灵。
\par }{\SH 律法和应许
\par }{\PP \VS{15}弟兄们,我且照着人的常话说:虽然是人的文约,若已经立定了,就没有能废弃或加增的。
\VS{16}所应许的原是向{\PN{亚伯拉罕}}和他子孙说的。 神并不是说「众子孙」,指着许多人,乃是说「你那一个子孙」,指着一个人,就是基督。
\VS{17}我是这么说, 神预先所立的约,不能被那四百三十年以后的律法废掉,叫应许归于虚空。
\VS{18}因为承受产业,若本乎律法,就不本乎应许;但 神是凭着应许把产业赐给{\PN{亚伯拉罕}}。
\VS{19}这样说来,律法是为什么有的呢?原是为过犯添上的,等候那蒙应许的子孙来到,并且是借天使经中保之手设立的。
\VS{20}但中保本不是为一面作的; 神却是一位。
\par }{\SH 奴仆和儿子
\par }{\PP \VS{21}这样,律法是与 神的应许反对吗?断乎不是!若曾传一个能叫人得生的律法,义就诚然本乎律法了。
\VS{22}但圣经把众人都圈在罪里,使所应许的{\ADD{福}}因信耶稣基督,归给那信的人。
\par }{\PP \VS{23}但这{\ADD{因}}信{\ADD{得救的理}}还未来以先,我们被看守在律法之下,直圈到那将来的真道显明出来。
\VS{24}这样,律法是我们训蒙的师傅,{\ADD{引我们}}到基督那里,使我们因信称义。
\VS{25}但这{\ADD{因}}信{\ADD{得救的理}}既然来到,我们从此就不在师傅的手下了。
\par }{\PP \VS{26}所以,你们因信基督耶稣都是 神的儿子。
\VS{27}你们受洗归入基督的都是披戴基督了。
\VS{28}并不分{\PN{犹太}}人、{\PN{希腊}}人,自主的、为奴的,或男或女,因为你们在基督耶稣里都成为一了。
\VS{29}你们既属乎基督,就是{\PN{亚伯拉罕}}的后裔,是照着应许承受产业的了。

\par }\Chap{4}{\PP \VerseOne{1}我说那承受产业的,虽然是全业的主人,但为孩童的时候却与奴仆毫无分别,
\VS{2}乃在师傅和管家的手下,直等他父亲预定的时候来到。
\VS{3}我们为孩童的时候,受管于世俗小学之下,也是如此。
\VS{4}及至时候满足, 神就差遣他的儿子,为女子所生,且生在律法以下,
\VS{5}要把律法以下的人赎出来,叫我们得着儿子的名分。
\VS{6}你们既为儿子, 神就差他儿子的灵进入你们\FTNT{}{{\FR 4:6: }原文是我们}的心,呼叫:「阿爸!父!」
\VS{7}可见,从此以后,你不是奴仆,乃是儿子了;既是儿子,就靠着 神为后嗣。
\par }{\SH 保罗关怀加拉太人
\par }{\PP \VS{8}但从前你们不认识 神的时候,是给那些本来不是神的作奴仆;
\VS{9}现在你们既然认识 神,更可说是被 神所认识的,怎么还要归回那懦弱无用的小学,情愿再给他作奴仆呢?
\VS{10}你们谨守日子、月份、节期、年份,
\VS{11}我为你们害怕,惟恐我在你们身上是枉费了工夫。
\par }{\PP \VS{12}弟兄们,我劝你们要像我一样,因为我也像你们一样。你们一点没有亏负我。
\VS{13}你们知道我头一次传福音给你们,是因为身体有疾病。
\VS{14}你们为我身体的缘故受试炼,没有轻看我,也没有厌弃我,反倒接待我,如同 神的使者,如同基督耶稣。
\VS{15}你们当日所夸的福气在哪里呢?那时你们若能行,就是把自己的眼睛剜出来给我,也都情愿。这是我可以给你们作见证的。
\VS{16}如今我将真理告诉你们,就成了你们的仇敌吗?
\VS{17}那些人热心待你们,却不是好意,是要离间你们\FTNT{}{{\FR 4:17: }原文是把你们关在外面},叫你们热心待他们。
\VS{18}在善事上,常用热心待人原是好的,却不单我与你们同在的时候才这样。
\VS{19}我小子啊,我为你们再受生产之苦,直等到基督成形在你们心里。
\VS{20}我巴不得现今在你们那里,改换口气,因我为你们心里作难。
\par }{\SH 夏甲和撒拉的例子
\par }{\PP \VS{21}你们这愿意在律法以下的人,请告诉我,你们岂没有听见律法吗?
\VS{22}因为{\ADD{律法上}}记着,{\PN{亚伯拉罕}}有两个儿子,一个是使女生的,一个是自主之妇人生的。
\VS{23}然而,那使女所生的是按着血气生的;那自主之妇人所生的是凭着应许生的。
\VS{24}这都是比方:那两个妇人就是两约。一约是出于{\PN{西奈山}},生子为奴,乃是{\PN{夏甲}}。
\VS{25}这{\PN{夏甲}}二字是指着{\PN{阿拉伯}}的{\PN{西奈山}},与现在的{\PN{耶路撒冷}}同类,因{\PN{耶路撒冷}}和她的儿女都是为奴的。
\VS{26}但那在上的{\PN{耶路撒冷}}是自主的,她是我们的母。
\VS{27}因为{\ADD{经上}}记着:
\par }{\Q 不怀孕、不生养的,你要欢乐;
\par }{\Q 未曾经过产难的,你要高声欢呼;
\par }{\Q 因为没有丈夫的,比有丈夫的儿女更多。
\par }{\MM \VS{28}弟兄们,我们是凭着应许作儿女,如同{\PN{以撒}}一样。
\VS{29}当时,那按着血气生的逼迫了那按着{\ADD{圣}}灵生的,现在也是这样。
\VS{30}然而经上是怎么说的呢?{\ADD{是说}}:「把使女和她儿子赶出去!因为使女的儿子不可与自主妇人的儿子一同承受产业。」
\VS{31}弟兄们,这样看来,我们不是使女的儿女,乃是自主妇人的儿女了。

\par }\Chap{5}{\PP \VerseOne{1}基督释放了我们,叫我们得以自由。所以要站立得稳,不要再被奴仆的轭挟制。
\par }{\SH 基督徒的自由
\par }{\PP \VS{2}我—{\PN{保罗}}告诉你们,若受割礼,基督就与你们无益了。
\VS{3}我再指着凡受割礼的人确实地说,他是欠着行全律法的债。
\VS{4}你们这要靠律法称义的,是与基督隔绝,从恩典中坠落了。
\VS{5}我们靠着{\ADD{圣}}灵,凭着信心,等候所盼望的义。
\VS{6}原来在基督耶稣里,受割礼不受割礼全无功效,惟独使人生发仁爱的信心才有功效。
\par }{\PP \VS{7}你们向来跑得好,有谁拦阻你们,叫你们不顺从真理呢?
\VS{8}这样的劝导不是出于那召你们的。
\VS{9}一点面酵能使全团都发起来。
\VS{10}我在主里很信你们必不怀别样的心;但搅扰你们的,无论是谁,必担当他的罪名。
\VS{11}弟兄们,我若仍旧传割礼,为什么还受逼迫呢?若是这样,那十字架讨厌的地方就没有了。
\VS{12}恨不得那搅乱你们的人把自己割绝了。
\par }{\PP \VS{13}弟兄们,你们蒙召是要得自由,只是不可将你们的自由当作放纵情欲的机会,总要用爱心互相服事。
\VS{14}因为全律法都包在「爱人如己」这一句话之内了。
\VS{15}你们要谨慎,若相咬相吞,只怕要彼此消灭了。
\par }{\SH 圣灵的果子和肉体的恶行
\par }{\PP \VS{16}我说,你们当顺着{\ADD{圣}}灵而行,就不放纵肉体的情欲了。
\VS{17}因为情欲和{\ADD{圣}}灵相争,{\ADD{圣}}灵和情欲相争,这两个是彼此相敌,使你们不能做所愿意做的。
\VS{18}但你们若被{\ADD{圣}}灵引导,就不在律法以下。
\VS{19}情欲的事都是显而易见的,就如奸淫、污秽、邪荡、
\VS{20}拜偶像、邪术、仇恨、争竞、忌恨、恼怒、结党、纷争、异端、
\VS{21}嫉妒\FTNT{}{{\FR 5:21: }有古卷加:凶杀二字}、醉酒、荒宴等类。我从前告诉你们,现在又告诉你们,行这样事的人必不能承受 神的国。
\par }{\PP \VS{22}圣灵所结的果子,就是仁爱、喜乐、和平、忍耐、恩慈、良善、信实、
\VS{23}温柔、节制。这样的事没有律法禁止。
\VS{24}凡属基督耶稣的人,是已经把肉体连肉体的邪情私欲同钉在十字架上了。
\VS{25}我们若是靠{\ADD{圣}}灵得生,就当靠{\ADD{圣}}灵行事。
\VS{26}不要贪图虚名,彼此惹气,互相嫉妒。

\par }\Chap{6}{\SH 各人的重担要互相担当
\par }{\PP \VerseOne{1}弟兄们,若有人偶然被过犯所胜,你们属灵的人就当用温柔的心把他挽回过来;又当自己小心,恐怕也被引诱。
\VS{2}你们各人的重担要互相担当,如此,就完全了基督的律法。
\VS{3}人若无有,自己还以为有,就是自欺了。
\VS{4}各人应当察验自己的行为;这样,他所夸的就专在自己,不在别人了,
\VS{5}因为各人必担当自己的担子。
\VS{6}在道理上受教的,当把一切需用的供给施教的人。
\VS{7}不要自欺, 神是轻慢不得的。人种的是什么,收的也是什么。
\VS{8}顺着情欲撒种的,必从情欲收败坏;顺着{\ADD{圣}}灵撒种的,必从{\ADD{圣}}灵收永生。
\VS{9}我们行善,不可丧志;若不灰心,到了时候就要收成。
\VS{10}所以,有了机会就当向众人行善,向信徒一家的人更当这样。
\par }{\SH 警告和祝福
\par }{\PP \VS{11}请看我亲手写给你们的字是何等的大呢!
\VS{12}凡希图外貌体面的人都勉强你们受割礼,无非是怕自己为基督的十字架受逼迫。
\VS{13}他们那些受割礼的,连自己也不守律法;他们愿意你们受割礼,不过要借着你们的肉体夸口。
\VS{14}但我断不以别的夸口,只夸我们主耶稣基督的十字架;因这十字架,就我而论,世界已经钉在十字架上;就世界而论,我已经钉在十字架上。
\VS{15}受割礼不受割礼都无关紧要,要紧的就是作新造的人。
\VS{16}凡照此理而行的,愿平安、怜悯加给他们,和 神的{\PN{以色列}}民。
\par }{\PP \VS{17}从今以后,人都不要搅扰我,因为我身上带着耶稣的印记。
\par }{\PP \VS{18}弟兄们,愿我主耶稣基督的恩{\ADD{常}}在你们心里。阿们!
\par }