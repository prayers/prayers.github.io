\NormalFont\ShortTitle{民数记}
{\MT 民数记

\par }\ChapOne{1}{\SH 以色列的第一次人口普查
\par }{\PP \VerseOne{1}{\PN{以色列}}人出{\PN{埃及}}地后,第二年二月初一日,耶和华在{\PN{西奈}}的旷野、会幕中晓谕{\PN{摩西}}说:
\VS{2}「你要按{\PN{以色列}}全会众的家室、宗族、人名的数目计算所有的男丁。
\VS{3}凡{\PN{以色列}}中,从二十岁以外,能出去打仗的,你和{\PN{亚伦}}要照他们的军队数点。
\VS{4}每支派中必有一人作本支派的族长,帮助你们。
\VS{5}他们的名字:属{\PN{吕便}}的,有{\PN{示丢珥}}的儿子{\PN{以利蓿}};
\VS{6}属{\PN{西缅}}的,有{\PN{苏利沙代}}的儿子{\PN{示路蔑}};
\VS{7}属{\PN{犹大}}的,有{\PN{亚米拿达}}的儿子{\PN{拿顺}};
\VS{8}属{\PN{以萨迦}}的,有{\PN{苏押}}的儿子{\PN{拿坦业}};
\VS{9}属{\PN{西布伦}}的,有{\PN{希伦}}的儿子{\PN{以利押}};
\VS{10}{\PN{约瑟}}子孙、属{\PN{以法莲}}的,有{\PN{亚米忽}}的儿子{\PN{以利沙玛}};属{\PN{玛拿西}}的,有{\PN{比大蓿}}的儿子{\PN{迦玛列}};
\VS{11}属{\PN{便雅悯}}的,有{\PN{基多尼}}的儿子{\PN{亚比但}};
\VS{12}属{\PN{但}}的,有{\PN{亚米沙代}}的儿子{\PN{亚希以谢}};
\VS{13}属{\PN{亚设}}的,有{\PN{俄兰}}的儿子{\PN{帕结}};
\VS{14}属{\PN{迦得}}的,有{\PN{丢珥}}的儿子{\PN{以利雅萨}};
\VS{15}属{\PN{拿弗他利}}的,有{\PN{以南}}的儿子{\PN{亚希拉}}。
\VS{16}这都是从会中选召的,各作本支派的首领,都是{\PN{以色列}}军中的统领。」
\par }{\PP \VS{17}于是,{\PN{摩西}}、{\PN{亚伦}}带着这些按名指定的人,
\VS{18}当二月初一日招聚全会众。会众就照他们的家室、宗族、人名的数目,从二十岁以外的,都述说自己的家谱。
\VS{19}耶和华怎样吩咐{\PN{摩西}},他就怎样在{\PN{西奈}}的旷野数点他们。
\par }{\PP \VS{20-21}{\PN{以色列}}的长子,{\PN{吕便}}子孙的后代,照着家室、宗族、人名的数目,从二十岁以外,凡能出去打仗、被数的男丁,共有四万六千五百名。
\par }{\PP \VS{22-23}{\PN{西缅}}子孙的后代,照着家室、宗族、人名的数目,从二十岁以外,凡能出去打仗、被数的男丁,共有五万九千三百名。
\par }{\PP \VS{24-25}{\PN{迦得}}子孙的后代,照着家室、宗族、人名的数目,从二十岁以外,凡能出去打仗、被数的,共有四万五千六百五十名。
\par }{\PP \VS{26-27}{\PN{犹大}}子孙的后代,照着家室、宗族、人名的数目,从二十岁以外,凡能出去打仗、被数的,共有七万四千六百名。
\par }{\PP \VS{28-29}{\PN{以萨迦}}子孙的后代,照着家室、宗族、人名的数目,从二十岁以外,凡能出去打仗、被数的,共有五万四千四百名。
\par }{\PP \VS{30-31}{\PN{西布伦}}子孙的后代,照着家室、宗族、人名的数目,从二十岁以外,凡能出去打仗、被数的,共有五万七千四百名。
\par }{\PP \VS{32-33}{\PN{约瑟}}子孙属{\PN{以法莲}}子孙的后代,照着家室、宗族、人名的数目,从二十岁以外,凡能出去打仗、被数的,共有四万零五百名。
\par }{\PP \VS{34-35}{\PN{玛拿西}}子孙的后代,照着家室、宗族、人名的数目,从二十岁以外,凡能出去打仗、被数的,共有三万二千二百名。
\par }{\PP \VS{36-37}{\PN{便雅悯}}子孙的后代,照着家室、宗族、人名的数目,从二十岁以外,凡能出去打仗、被数的,共有三万五千四百名。
\par }{\PP \VS{38-39}{\PN{但}}子孙的后代,照着家室、宗族、人名的数目,从二十岁以外,凡能出去打仗、被数的,共有六万二千七百名。
\par }{\PP \VS{40-41}{\PN{亚设}}子孙的后代,照着家室、宗族、人名的数目,从二十岁以外,凡能出去打仗、被数的,共有四万一千五百名。
\par }{\PP \VS{42-43}{\PN{拿弗他利}}子孙的后代,照着家室、宗族、人名的数目,从二十岁以外,凡能出去打仗、被数的,共有五万三千四百名。
\par }{\PP \VS{44}这些就是被数点的,是{\PN{摩西}}、{\PN{亚伦}},和{\PN{以色列}}中十二个首领所数点的;这十二个人各作各宗族的代表。
\VS{45-46}这样,凡{\PN{以色列}}人中被数的,照着宗族,从二十岁以外,能出去打仗、被数的,共有六十万零三千五百五十名。
\par }{\PP \VS{47}{\PN{利未}}人却没有按着支派数在其中,
\VS{48}因为耶和华晓谕{\PN{摩西}}说:
\VS{49}「惟独{\PN{利未}}支派你不可数点,也不可在{\PN{以色列}}人中计算他们的总数。
\VS{50}只要派{\PN{利未}}人管法{\ADD{柜}}的帐幕和其中的器具,并属乎帐幕的;他们要抬\FTNT{}{{\FR 1:50: }或译:搬运}帐幕和其中的器具,并要办理帐幕的事,在帐幕的四围安营。
\VS{51}帐幕将往前行的时候,{\PN{利未}}人要拆卸;将支搭的时候,{\PN{利未}}人要竖起。近前来的外人必被治死。
\VS{52}{\PN{以色列}}人支搭帐棚,要照他们的军队,各归本营,各归本纛。
\VS{53}但{\PN{利未}}人要在法{\ADD{柜}}帐幕的四围安营,免得忿怒临到{\PN{以色列}}会众;{\PN{利未}}人并要谨守法{\ADD{柜}}的帐幕。」
\VS{54}{\PN{以色列}}人就这样行。凡耶和华所吩咐{\PN{摩西}}的,他们就照样行了。

\par }\Chap{2}{\SH 各支派在营地的位置
\par }{\PP \VerseOne{1}耶和华晓谕{\PN{摩西}}、{\PN{亚伦}}说:
\VS{2}「{\PN{以色列}}人要各归自己的纛下,在本族的旗号那里,对着会幕的四围安营。
\VS{3}在东边,向日出之地,照着军队安营的是{\PN{犹大}}营的纛。有{\PN{亚米拿达}}的儿子{\PN{拿顺}}作{\PN{犹大}}人的首领。
\VS{4}他军队被数的,共有七万四千六百名。
\VS{5}挨着他安营的是{\PN{以萨迦}}支派。有{\PN{苏押}}的儿子{\PN{拿坦业}}作{\PN{以萨迦}}人的首领。
\VS{6}他军队被数的,共有五万四千四百名。
\VS{7}又有{\PN{西布伦}}支派。{\PN{希伦}}的儿子{\PN{以利押}}作{\PN{西布伦}}人的首领。
\VS{8}他军队被数的,共有五万七千四百名。
\VS{9}凡属{\PN{犹大}}营、按着军队被数的,共有十八万六千四百名,要作第一队往前行。
\par }{\PP \VS{10}「在南边,按着军队是{\PN{吕便}}营的纛。有{\PN{示丢珥}}的儿子{\PN{以利蓿}}作{\PN{吕便}}人的首领。
\VS{11}他军队被数的,共有四万六千五百名。
\VS{12}挨着他安营的是{\PN{西缅}}支派。{\PN{苏利沙代}}的儿子{\PN{示路蔑}}作{\PN{西缅}}人的首领。
\VS{13}他军队被数的,共有五万九千三百名。
\VS{14}又有{\PN{迦得}}支派。{\PN{丢珥}}的儿子{\PN{以利雅萨}}作{\PN{迦得}}人的首领。
\VS{15}他军队被数的,共有四万五千六百五十名,
\VS{16}凡属{\PN{吕便}}营、按着军队被数的,共有十五万一千四百五十名,要作第二队往前行。
\par }{\PP \VS{17}「随后,会幕要往前行,有{\PN{利未}}营在诸营中间。他们怎样安营就怎样往前行,各按本位,各归本纛。
\par }{\PP \VS{18}「在西边,按着军队是{\PN{以法莲}}营的纛。{\PN{亚米忽}}的儿子{\PN{以利沙玛}}作{\PN{以法莲}}人的首领。
\VS{19}他军队被数的,共有四万零五百名。
\VS{20}挨着他的是{\PN{玛拿西}}支派。{\PN{比大蓿}}的儿子{\PN{迦玛列}}作{\PN{玛拿西}}人的首领。
\VS{21}他军队被数的,共有三万二千二百名。
\VS{22}又有{\PN{便雅悯}}支派。{\PN{基多尼}}的儿子{\PN{亚比但}}作{\PN{便雅悯}}人的首领。
\VS{23}他军队被数的,共有三万五千四百名。
\VS{24}凡属{\PN{以法莲}}营、按着军队被数的,共有十万零八千一百名,要作第三队往前行。
\par }{\PP \VS{25}「在北边,按着军队是{\PN{但}}营的纛。{\PN{亚米沙代}}的儿子{\PN{亚希以谢}}作{\PN{但}}人的首领。
\VS{26}他军队被数的,共有六万二千七百名。
\VS{27}挨着他安营的是{\PN{亚设}}支派。{\PN{俄兰}}的儿子{\PN{帕结}}作{\PN{亚设}}人的首领。
\VS{28}他军队被数的,共有四万一千五百名。
\VS{29}又有{\PN{拿弗他利}}支派。{\PN{以南}}的儿子{\PN{亚希拉}}作{\PN{拿弗他利}}人的首领。
\VS{30}他军队被数的,共有五万三千四百名。
\VS{31}凡{\PN{但}}营被数的,共有十五万七千六百名,要归本纛作末队往前行。」
\par }{\PP \VS{32}这些{\PN{以色列}}人,照他们的宗族,按他们的军队,在诸营中被数的,共有六十万零三千五百五十名。
\VS{33}惟独{\PN{利未}}人没有数在{\PN{以色列}}人中,是照耶和华所吩咐{\PN{摩西}}的。
\par }{\PP \VS{34}{\PN{以色列}}人就这样行,各人照他们的家室、宗族归于本纛,安营起行,都是照耶和华所吩咐{\PN{摩西}}的。

\par }\Chap{3}{\SH 亚伦的儿子们
\par }{\PP \VerseOne{1}耶和华在{\PN{西奈山}}晓谕{\PN{摩西}}的日子,{\PN{亚伦}}和{\PN{摩西}}的后代如下:
\VS{2}{\PN{亚伦}}的儿子,长子名叫{\PN{拿答}},还有{\PN{亚比户}}、{\PN{以利亚撒}}、{\PN{以他玛}}。
\VS{3}这是{\PN{亚伦}}儿子的名字,都是受膏的祭司,是{\PN{摩西}}叫他们承接圣职供祭司职分的。
\VS{4}{\PN{拿答}}、{\PN{亚比户}}在{\PN{西奈}}的旷野向耶和华献凡火的时候就死在耶和华面前了。他们也没有儿子。{\PN{以利亚撒}}、{\PN{以他玛}}在他们的父亲{\PN{亚伦}}面前供祭司的职分。
\par }{\SH 利未人被派作祭司
\par }{\PP \VS{5}耶和华晓谕{\PN{摩西}}说:
\VS{6}「你使{\PN{利未}}支派近前来,站在祭司{\PN{亚伦}}面前好服事他,
\VS{7}替他和会众在会幕前守所吩咐的,办理帐幕的事。
\VS{8}又要看守会幕的器具,并守所吩咐{\PN{以色列}}人的,办理帐幕的事。
\VS{9}你要将{\PN{利未}}人给{\PN{亚伦}}和他的儿子,因为他们是从{\PN{以色列}}人中选出来给他的。
\VS{10}你要嘱咐{\PN{亚伦}}和他的儿子谨守自己祭司的职任。近前来的外人必被治死。」
\par }{\PP \VS{11}耶和华晓谕{\PN{摩西}}说:
\VS{12}「我从{\PN{以色列}}人中拣选了{\PN{利未}}人,代替{\PN{以色列}}人一切头生的;{\PN{利未}}人要归我。
\VS{13}因为凡头生的是我的;我在{\PN{埃及}}地击杀一切头生的那日就把{\PN{以色列}}中一切头生的,连人带牲畜都分别为圣归我;他们定要属我。我是耶和华。」
\par }{\SH 利未人男丁的统计
\par }{\PP \VS{14}耶和华在{\PN{西奈}}的旷野晓谕{\PN{摩西}}说:
\VS{15}「你要照{\PN{利未}}人的宗族、家室数点他们。凡一个月以外的男子都要数点。」
\VS{16}于是{\PN{摩西}}照耶和华所吩咐的数点他们。
\VS{17}{\PN{利未}}众子的名字是{\PN{革顺}}、{\PN{哥辖}}、{\PN{米拉利}}。
\VS{18}{\PN{革顺}}的儿子,按着家室,是{\PN{立尼}}、{\PN{示每}}。
\VS{19}{\PN{哥辖}}的儿子,按着家室,是{\PN{暗兰}}、{\PN{以斯哈}}、{\PN{希伯伦}}、{\PN{乌薛}}。
\VS{20}{\PN{米拉利}}的儿子,按着家室,是{\PN{抹利}}、{\PN{母示}}。这些按着宗族是{\PN{利未}}人的家室。
\par }{\PP \VS{21}属{\PN{革顺}}的,有{\PN{立尼}}族、{\PN{示每}}族。这是{\PN{革顺}}的二族。
\VS{22}其中被数、从一个月以外所有的男子共有七千五百名。
\VS{23}这{\PN{革顺}}的二族要在帐幕后西边安营。
\VS{24}{\PN{拉伊勒}}的儿子{\PN{以利雅萨}}作{\PN{革顺}}人宗族的首领。
\VS{25}{\PN{革顺}}的子孙在会幕中所要看守的,就是帐幕和罩棚,并罩棚的盖与会幕的门帘,
\VS{26}院子的帷子和门帘(院子是围帐幕和坛的),并一切使用的绳子。
\par }{\PP \VS{27}属{\PN{哥辖}}的,有{\PN{暗兰}}族、{\PN{以斯哈}}族、{\PN{希伯伦}}族、{\PN{乌薛}}族。这是{\PN{哥辖}}的诸族。
\VS{28}按所有男子的数目,从一个月以外看守圣所的,共有八千六百名。
\VS{29}{\PN{哥辖}}儿子的诸族要在帐幕的南边安营。
\VS{30}{\PN{乌薛}}的儿子{\PN{以利撒反}}作{\PN{哥辖}}宗族家室的首领。
\VS{31}他们所要看守的是{\ADD{约}}柜、桌子、灯台、两座坛与圣所内使用的器皿,并帘子和一切使用之物。
\VS{32}祭司{\PN{亚伦}}的儿子{\PN{以利亚撒}}作{\PN{利未}}人众首领的领袖,要监察那些看守圣所的人。
\par }{\PP \VS{33}属{\PN{米拉利}}的,有{\PN{抹利}}族、{\PN{母示}}族。这是{\PN{米拉利}}的二族。
\VS{34}他们被数的,按所有男子的数目,从一个月以外的,共有六千二百名。
\VS{35}{\PN{亚比亥}}的儿子{\PN{苏列}}作{\PN{米拉利}}二宗族的首领。他们要在帐幕的北边安营。
\VS{36}{\PN{米拉利}}子孙的职分是看守帐幕的板、闩、柱子、带卯的座,和帐幕一切所使用的器具,
\VS{37}院子四围的柱子、带卯的座、橛子,和绳子。
\par }{\PP \VS{38}在帐幕前东边,向日出之地安营的是{\PN{摩西}}、{\PN{亚伦}},和{\PN{亚伦}}的儿子。他们看守圣所,替{\PN{以色列}}人守{\ADD{耶和华}}所吩咐的。近前来的外人必被治死。
\VS{39}凡被数的{\PN{利未}}人,就是{\PN{摩西}}、{\PN{亚伦}}照耶和华吩咐所数的,按着家室,从一个月以外的男子,共有二万二千名。
\par }{\SH 利未人代替长子的地位
\par }{\PP \VS{40}耶和华对{\PN{摩西}}说:「你要从{\PN{以色列}}人中数点一个月以外、凡头生的男子,把他们的名字记下。
\VS{41}我是耶和华。你要拣选{\PN{利未}}人归我,代替{\PN{以色列}}人所有头生的,也取{\PN{利未}}人的牲畜代替{\PN{以色列}}所有头生的牲畜。」
\VS{42}{\PN{摩西}}就照耶和华所吩咐的把{\PN{以色列}}人头生的都数点了。
\VS{43}按人名的数目,从一个月以外、凡头生的男子,共有二万二千二百七十三名。
\par }{\PP \VS{44}耶和华晓谕{\PN{摩西}}说:
\VS{45}「你拣选{\PN{利未}}人代替{\PN{以色列}}人所有头生的,也取{\PN{利未}}人的牲畜代替{\PN{以色列}}人的牲畜。{\PN{利未}}人要归我;我是耶和华。
\VS{46}{\PN{以色列}}人中头生的男子比{\PN{利未}}人多二百七十三个,必当将他们赎出来。
\VS{47}你要按人丁,照圣所的平,每人取赎银五舍客勒(一舍客勒是二十季拉),
\VS{48}把那多余之人的赎银交给{\PN{亚伦}}和他的儿子。」
\VS{49}于是{\PN{摩西}}从那被{\PN{利未}}人所赎以外的人取了赎银。
\VS{50}从{\PN{以色列}}人头生的所取之银,按圣所的平,有一千三百六十五{\ADD{舍客勒}}。
\VS{51}{\PN{摩西}}照耶和华的话把这赎银给{\PN{亚伦}}和他的儿子,正如耶和华所吩咐的。

\par }\Chap{4}{\SH 哥辖子孙的职责
\par }{\PP \VerseOne{1}耶和华晓谕{\PN{摩西}}、{\PN{亚伦}}说:
\VS{2}「你从{\PN{利未}}人中,将{\PN{哥辖}}子孙的总数,照他们的家室、宗族,
\VS{3}从三十岁直到五十岁,凡前来任职、在会幕里办事的,全都计算。
\VS{4}{\PN{哥辖}}子孙在会幕{\ADD{搬运}}至圣之物,所办的事乃是这样:
\VS{5}起营的时候,{\PN{亚伦}}和他儿子要进去摘下遮掩{\ADD{柜}}的幔子,用以蒙盖法柜,
\VS{6}又用海狗皮盖在上头,再蒙上纯蓝色的毯子,把杠穿上。
\VS{7}又用蓝色毯子铺在陈设饼的桌子上,将盘子、调羹、奠{\ADD{酒}}的爵,和杯摆在上头。桌子上也必有常{\ADD{设的}}饼。
\VS{8}在其上又要蒙朱红色的毯子,再蒙上海狗皮,把杠穿上。
\VS{9}要拿蓝色毯子,把灯台和灯台上所用的灯盏、剪子、蜡花盘,并一切盛油的器皿,全都遮盖。
\VS{10}又要把灯台和灯台的一切器具包在海狗皮里,放在抬架上。
\VS{11}在金坛上要铺蓝色毯子,蒙上海狗皮,把杠穿上。
\VS{12}又要把圣所用的一切器具包在蓝色毯子里,用海狗皮蒙上,放在抬架上。
\VS{13}要收去坛上的灰,把紫色毯子铺在坛上;
\VS{14}又要把所用的一切器具,就是火鼎、肉锸子、铲子、盘子,一切属坛的器具都摆在坛上,又蒙上海狗皮,把杠穿上。
\VS{15}将要起营的时候,{\PN{亚伦}}和他儿子把圣所和圣所的一切器具遮盖完了,{\PN{哥辖}}的子孙就要来抬,只是不可摸圣物,免得他们死亡。会幕里这些物件是{\PN{哥辖}}子孙所当抬的。
\par }{\PP \VS{16}「祭司{\PN{亚伦}}的儿子{\PN{以利亚撒}}所要看守的是点灯的油与香料,并当献的素祭和膏油,也要看守全帐幕与其中所有的,并圣所和圣所的器具。」
\par }{\PP \VS{17}耶和华晓谕{\PN{摩西}}、{\PN{亚伦}}说:
\VS{18}「你们不可将{\PN{哥辖}}人的支派从{\PN{利未}}人中剪除。
\VS{19}他们挨近至圣物的时候,{\PN{亚伦}}和他儿子要进去派他们各人所当办的,所当抬的。这样待他们,好使他们活着,不致死亡。
\VS{20}只是他们连片时不可进去观看圣所,免得他们死亡。」
\par }{\SH 革顺子孙的职责
\par }{\PP \VS{21}耶和华晓谕{\PN{摩西}}说:
\VS{22}「你要将{\PN{革顺}}子孙的总数,照着宗族、家室,
\VS{23}从三十岁直到五十岁,凡前来任职、在会幕里办事的,全都数点。
\VS{24}{\PN{革顺}}人各族所办的事、所抬的物乃是这样:
\VS{25}他们要抬帐幕的幔子和会幕,并会幕的盖与其上的海狗皮,和会幕的门帘,
\VS{26}院子的帷子和门帘(院子是围帐幕和坛的)、绳子,并所用的器具,不论是做什么用的,他们都要经理。
\VS{27}{\PN{革顺}}的子孙在一切抬物办事之上都要凭{\PN{亚伦}}和他儿子的吩咐;他们所当抬的,要派他们看守。
\VS{28}这是{\PN{革顺}}子孙的各族在会幕里所办的事;他们所看守的,必在祭司{\PN{亚伦}}儿子{\PN{以他玛}}的手下。」
\par }{\SH 米拉利子孙的职责
\par }{\PP \VS{29}「至于{\PN{米拉利}}的子孙,你要照着家室、宗族把他们数点。
\VS{30}从三十岁直到五十岁,凡前来任职、在会幕里办事的,你都要数点。
\VS{31}他们办理会幕的事,就是抬帐幕的板、闩、柱子,和带卯的座,
\VS{32}院子四围的柱子和其上带卯的座、橛子、绳子,并一切使用的器具。他们所抬的器具,你们要按名指定。
\VS{33}这是{\PN{米拉利}}子孙各族在会幕里所办的事,都在祭司{\PN{亚伦}}儿子{\PN{以他玛}}的手下。」
\par }{\SH 利未支派男丁的统计
\par }{\PP \VS{34}{\PN{摩西}}、{\PN{亚伦}}与会众的诸首领将{\PN{哥辖}}的子孙,照着家室、宗族,
\VS{35}从三十岁直到五十岁,凡前来任职、在会幕里办事的,都数点了。
\VS{36}被数的共有二千七百五十名。
\VS{37}这是{\PN{哥辖}}各族中被数的,是在会幕里办事的,就是{\PN{摩西}}、{\PN{亚伦}}照耶和华借{\PN{摩西}}所吩咐数点的。
\par }{\PP \VS{38}{\PN{革顺}}子孙被数的,照着家室、宗族,
\VS{39-40}从三十岁直到五十岁,凡前来任职、在会幕里办事的,共有二千六百三十名。
\VS{41}这是{\PN{革顺}}子孙各族中被数的,是在会幕里办事的,就是{\PN{摩西}}、{\PN{亚伦}}照耶和华借{\PN{摩西}}所吩咐数点的。
\par }{\PP \VS{42}{\PN{米拉利}}子孙中各族被数的,照着家室、宗族,
\VS{43-44}从三十岁直到五十岁,凡前来任职、在会幕里办事的,共有三千二百名。
\VS{45}这是{\PN{米拉利}}子孙各族中被数的,就是{\PN{摩西}}、{\PN{亚伦}}照耶和华借{\PN{摩西}}所吩咐数点的。
\par }{\PP \VS{46}凡被数的{\PN{利未}}人,就是{\PN{摩西}}、{\PN{亚伦}}并{\PN{以色列}}众首领,照着家室、宗族所数点的,
\VS{47-48}从三十岁直到五十岁,凡前来任职、在会幕里做抬物之工的,共有八千五百八十名。
\VS{49}{\PN{摩西}}按他们所办的事、所抬的物,凭耶和华的吩咐数点他们;他们这样被{\PN{摩西}}数点,正如耶和华所吩咐他的。

\par }\Chap{5}{\SH 不洁的人
\par }{\PP \VerseOne{1}耶和华晓谕{\PN{摩西}}说:
\VS{2}「你吩咐{\PN{以色列}}人,使一切长大麻风的,患漏症的,并因死尸不洁净的,都出营外去。
\VS{3}无论男女都要使他们出到营外,免得污秽他们的营;这营是我所住的。」
\VS{4}{\PN{以色列}}人就这样行,使他们出到营外。耶和华怎样吩咐{\PN{摩西}},{\PN{以色列}}人就怎样行了。
\par }{\SH 赔偿亏负
\par }{\PP \VS{5}耶和华对{\PN{摩西}}说:
\VS{6}「你晓谕{\PN{以色列}}人说:无论男女,若犯了人所常犯的罪,以致干犯耶和华,那人就有了罪。
\VS{7}他要承认所犯的罪,将所亏负人的,如数赔还,另外加上五分之一,也归与所亏负的人。
\VS{8}那人若没有亲属可受所赔还的,那所赔还的就要归与服事耶和华的祭司;至于那为他赎罪的公羊是在外。
\VS{9}{\PN{以色列}}人一切的圣物中,所奉给祭司的举祭都要归与祭司。
\VS{10}各人所分别为圣的物,无论是什么,都要归给祭司。」
\par }{\SH 疑妻不贞的案件
\par }{\PP \VS{11}耶和华对{\PN{摩西}}说:
\VS{12}「你晓谕{\PN{以色列}}人说:人的妻若有邪行,得罪她丈夫,
\VS{13}有人与她行淫,事情严密,瞒过她丈夫,而且她被玷污,没有作见证的人,当她行淫的时候也没有被捉住,
\VS{14}她丈夫生了疑恨的心,疑恨她,她是被玷污,或是她丈夫生了疑恨的心,疑恨她,她并没有被玷污,
\VS{15}这人就要将妻送到祭司那里,又为她带着大麦面伊法十分之一作供物,不可浇上油,也不可加上乳香;因为这是疑恨的素祭,是思念的素祭,使人思念罪孽。
\par }{\PP \VS{16}「祭司要使那妇人近前来,站在耶和华面前。
\VS{17}祭司要把圣水盛在瓦器里,又从帐幕的地上取点尘土,放在水中。
\VS{18}祭司要叫那妇人蓬头散发,站在耶和华面前,把思念的素祭,就是疑恨的素祭,放在她手中。祭司手里拿着致咒诅的苦水,
\VS{19}要叫妇人起誓,对她说:『若没有人与你行淫,也未曾背着丈夫做污秽的事,你就免受这致咒诅苦水的灾。
\VS{20}你若背着丈夫行了污秽的事,在你丈夫以外有人与你行淫,
\VS{21}(祭司叫妇人发咒起誓,)愿耶和华叫你大腿消瘦,肚腹发胀,使你在你民中被人咒诅,成了誓语;
\VS{22}并且这致咒诅的水入你的肠中,要叫你的肚腹发胀,大腿消瘦。』妇人要回答说:『阿们,阿们。』
\par }{\PP \VS{23}「祭司要写这咒诅的话,将所写的字抹在苦水里,
\VS{24}又叫妇人喝这致咒诅的苦水;这水要进入她里面{\ADD{变}}苦了。
\VS{25}祭司要从妇人的手中取那疑恨的素祭,在耶和华面前摇一摇,拿到坛前;
\VS{26}又要从素祭中取出一把,作为这事的纪念,烧在坛上,然后叫妇人喝这水。
\VS{27}叫她喝了以后,她若被玷污,得罪了丈夫,这致咒诅的水必进入她里面{\ADD{变}}苦了,她的肚腹就要发胀,大腿就要消瘦,那妇人便要在他民中被人咒诅。
\VS{28}若妇人没有被玷污,却是清洁的,就要免受这灾,且要怀孕。
\par }{\PP \VS{29}「妻子背着丈夫行了污秽的事,
\VS{30}或是人生了疑恨的心,疑恨他的妻,就有这疑恨的条例。那时他要叫妇人站在耶和华面前,祭司要在她身上照这条例而行。
\VS{31}男人就为无罪,妇人必担当自己的罪孽。」

\par }\Chap{6}{\SH 拿细耳人的条例
\par }{\PP \VerseOne{1}耶和华对{\PN{摩西}}说:
\VS{2}「你晓谕{\PN{以色列}}人说:无论男女许了特别的愿,就是拿细耳人的愿\FTNT{}{{\FR 6:2: }拿细耳就是归主的意思;下同},要离俗归耶和华。
\VS{3}他就要远离清酒浓酒,也不可喝什么清酒浓酒做的醋;不可喝什么葡萄汁,也不可吃鲜葡萄和干葡萄。
\VS{4}在一切离俗的日子,凡葡萄树上结的,自核至皮所做的物,都不可吃。
\par }{\PP \VS{5}「在他一切许愿离俗的日子,不可用剃头刀剃头,要由发绺长长了。他要圣洁,直到离俗归耶和华的日子满了。
\VS{6}在他离俗归耶和华的一切日子,不可挨近死尸。
\VS{7}他的父母或是弟兄姊妹死了的时候,他不可因他们使自己不洁净,因为那离俗归 神的{\ADD{凭据}}是在他头上。
\VS{8}在他一切离俗的日子是归耶和华为圣。
\par }{\PP \VS{9}「若在他旁边忽然有人死了,以致沾染了他离俗的头,他要在第七日,得洁净的时候,剃头。
\VS{10}第八日,他要把两只斑鸠或两只雏鸽带到会幕门口,交给祭司。
\VS{11}祭司要献一只作赎罪祭,一只作燔祭,为他赎那因死尸而有的罪,并要当日使他的头成为圣洁。
\VS{12}他要另选离俗归耶和华的日子,又要牵一只一岁的公羊羔来作赎愆祭;但先前的日子要归徒然,因为他在离俗之间被玷污了。
\par }{\PP \VS{13}「拿细耳人满了离俗的日子乃有这条例:人要领他到会幕门口,
\VS{14}他要将供物奉给耶和华,就是一只没有残疾、一岁的公羊羔作燔祭,一只没有残疾、一岁的母羊羔作赎罪祭,和一只没有残疾的公绵羊作平安祭,
\VS{15}并一筐子无酵调油的细面饼,与抹油的无酵薄饼,并同献的素祭和奠祭。
\VS{16}祭司要在耶和华面前献那人的赎罪祭和燔祭;
\VS{17}也要把那只公羊和那筐无酵饼献给耶和华作平安祭,又要将同献的素祭和奠祭献上。
\VS{18}拿细耳人要在会幕门口剃离俗的头,把离俗头上的发放在平安祭下的火上。
\VS{19}他剃了以后,祭司就要取那已煮的公羊一条前腿,又从筐子里取一个无酵饼和一个无酵薄饼,都放在他手上。
\VS{20}祭司要拿这些作为摇祭,在耶和华面前摇一摇;这与所摇的胸、所举的腿同为圣物,归给祭司。然后拿细耳人可以喝酒。
\par }{\PP \VS{21}「许愿的拿细耳人为离俗所献的供物,和他以外所能得的献给耶和华,就有这条例。他怎样许愿就当照离俗的条例行。」
\par }{\SH 祭司的祝福
\par }{\PP \VS{22}耶和华晓谕{\PN{摩西}}说:
\VS{23}「你告诉{\PN{亚伦}}和他儿子说:你们要这样为{\PN{以色列}}人祝福,说:
\VS{24}『愿耶和华赐福给你,保护你。
\VS{25}愿耶和华使他的脸光照你,赐恩给你。
\VS{26}愿耶和华向你仰脸,赐你平安。』
\VS{27}他们要如此奉我的名为{\PN{以色列}}人祝福;我也要赐福给他们。」

\par }\Chap{7}{\SH 各族长奉献祭物
\par }{\PP \VerseOne{1}{\PN{摩西}}立完了帐幕,就把帐幕用膏抹了,使它成圣,又把其中的器具和坛,并坛上的器具,都抹了,使它成圣。
\VS{2}当天,{\PN{以色列}}的众首领,就是各族的族长,都来奉献。他们是各支派的首领,管理那些被数的人。
\VS{3}他们把自己的供物送到耶和华面前,就是六辆篷子车和十二只公牛。每两个首领奉献一辆车,每首领奉献一只牛。他们把这些都奉到帐幕前。
\VS{4}耶和华晓谕{\PN{摩西}}说:
\VS{5}「你要收下这些,好作会幕的使用,都要照{\PN{利未}}人所办的事交给他们。」
\VS{6}于是{\PN{摩西}}收了车和牛,交给{\PN{利未}}人,
\VS{7}把两辆车,四只牛,照{\PN{革顺}}子孙所办的事交给他们,
\VS{8}又把四辆车,八只牛,照{\PN{米拉利}}子孙所办的事交给他们;他们都在祭司{\PN{亚伦}}的儿子{\PN{以他玛}}手下。
\VS{9}但车与牛都没有交给{\PN{哥辖}}子孙;因为他们办的是圣所的事,在肩头上抬圣物。
\VS{10}用膏抹坛的日子,首领都来行奉献坛的礼,众首领就在坛前献供物。
\VS{11}耶和华对{\PN{摩西}}说:「众首领为行奉献坛的礼,要每天一个首领来献供物。」
\par }{\PP \VS{12}头一日献供物的是{\PN{犹大}}支派的{\PN{亚米拿达}}的儿子{\PN{拿顺}}。
\VS{13}他的供物是:一个银盘子,重一百三十{\ADD{舍客勒}},一个银碗,重七十舍客勒,都是按圣所的平,也都盛满了调油的细面作素祭;
\VS{14}一个金盂,重十{\ADD{舍客勒}},盛满了香;
\VS{15}一只公牛犊,一只公绵羊,一只一岁的公羊羔作燔祭;
\VS{16}一只公山羊作赎罪祭;
\VS{17}两只公牛,五只公绵羊,五只公山羊,五只一岁的公羊羔作平安祭。这是{\PN{亚米拿达}}儿子{\PN{拿顺}}的供物。
\par }{\PP \VS{18}第二日来献的是{\PN{以萨迦}}{\ADD{子孙}}的首领、{\PN{苏押}}的儿子{\PN{拿坦业}}。
\VS{19}他献为供物的是:一个银盘子,重一百三十{\ADD{舍客勒}},一个银碗,重七十{\ADD{舍客勒}},都是按圣所的平,也都盛满了调油的细面作素祭;
\VS{20}一个金盂,重十{\ADD{舍客勒}},盛满了香;
\VS{21}一只公牛犊,一只公绵羊,一只一岁的公羊羔作燔祭;
\VS{22}一只公山羊作赎罪祭;
\VS{23}两只公牛,五只公绵羊,五只公山羊,五只一岁的公羊羔作平安祭。这是{\PN{苏押}}儿子{\PN{拿坦业}}的供物。
\par }{\PP \VS{24}第三日{\ADD{来献的}}是{\PN{西布伦}}子孙的首领、{\PN{希伦}}的儿子{\PN{以利押}}。
\VS{25}他的供物是:一个银盘子,重一百三十{\ADD{舍客勒}},一个银碗,重七十舍客勒,都是按圣所的平,也都盛满了调油的细面作素祭;
\VS{26}一个金盂,重十{\ADD{舍客勒}},盛满了香;
\VS{27}一只公牛犊,一只公绵羊,一只一岁的公羊羔作燔祭;
\VS{28}一只公山羊作赎罪祭;
\VS{29}两只公牛,五只公绵羊,五只公山羊,五只一岁的公羊羔作平安祭。这是{\PN{希伦}}儿子{\PN{以利押}}的供物。
\par }{\PP \VS{30}第四日{\ADD{来献的}}是{\PN{吕便}}子孙的首领、{\PN{示丢珥}}的儿子{\PN{以利蓿}}。
\VS{31}他的供物是:一个银盘子,重一百三十{\ADD{舍客勒}},一个银碗,重七十{\ADD{舍客勒}},都是按圣所的平,也都盛满了调油的细面作素祭;
\VS{32}一个金盂,重十{\ADD{舍客勒}},盛满了香;
\VS{33}一只公牛犊,一只公绵羊,一只一岁的公羊羔作燔祭;
\VS{34}一只公山羊作赎罪祭;
\VS{35}两只公牛,五只公绵羊,五只公山羊,五只一岁的公羊羔作平安祭。这是{\PN{示丢珥}}的儿子{\PN{以利蓿}}的供物。
\par }{\PP \VS{36}第五日{\ADD{来献的}}是{\PN{西缅}}子孙的首领、{\PN{苏利沙代}}的儿子{\PN{示路蔑}}。
\VS{37}他的供物是:一个银盘子,重一百三十{\ADD{舍客勒}},一个银碗,重七十舍客勒,都是按圣所的平,也都盛满了调油的细面作素祭;
\VS{38}一个金盂,重十{\ADD{舍客勒}},盛满了香;
\VS{39}一只公牛犊,一只公绵羊,一只一岁的公羊羔作燔祭;
\VS{40}一只公山羊作赎罪祭;
\VS{41}两只公牛,五只公绵羊,五只公山羊,五只一岁的公羊羔作平安祭。这是{\PN{苏利沙代}}儿子{\PN{示路蔑}}的供物。
\par }{\PP \VS{42}第六日{\ADD{来献的}}是{\PN{迦得}}子孙的首领、{\PN{丢珥}}的儿子{\PN{以利雅萨}}。
\VS{43}他的供物是:一个银盘子,重一百三十{\ADD{舍客勒}},一个银碗,重七十舍客勒,都是按圣所的平,也都盛满了调油的细面作素祭;
\VS{44}一个金盂,重十{\ADD{舍客勒}},盛满了香;
\VS{45}一只公牛犊,一只公绵羊,一只一岁的公羊羔作燔祭;
\VS{46}一只公山羊作赎罪祭;
\VS{47}两只公牛,五只公绵羊,五只公山羊,五只一岁的公羊羔作平安祭。这是{\PN{丢珥}}的儿子{\PN{以利雅萨}}的供物。
\par }{\PP \VS{48}第七日{\ADD{来献的}}是{\PN{以法莲}}子孙的首领、{\PN{亚米忽}}的儿子{\PN{以利沙玛}}。
\VS{49}他的供物是:一个银盘子,重一百三十{\ADD{舍客勒}},一个银碗,重七十舍客勒,都是按圣所的平,也都盛满了调油的细面作素祭;
\VS{50}一个金盂,重十{\ADD{舍客勒}},盛满了香;
\VS{51}一只公牛犊,一只公绵羊,一只一岁的公羊羔作燔祭;
\VS{52}一只公山羊作赎罪祭;
\VS{53}两只公牛,五只公绵羊,五只公山羊,五只一岁的公羊羔作平安祭。这是{\PN{亚米忽}}儿子{\PN{以利沙玛}}的供物。
\par }{\PP \VS{54}第八日{\ADD{来献的}}是{\PN{玛拿西}}子孙的首领、{\PN{比大蓿}}的儿子{\PN{迦玛列}}。
\VS{55}他的供物是:一个银盘子,重一百三十{\ADD{舍客勒}},一个银碗,重七十舍客勒,都是按圣所的平,也都盛满了调油的细面作素祭;
\VS{56}一个金盂,重十{\ADD{舍客勒}},盛满了香;
\VS{57}一只公牛犊,一只公绵羊,一只一岁的公羊羔作燔祭;
\VS{58}一只公山羊作赎罪祭;
\VS{59}两只公牛,五只公绵羊,五只公山羊,五只一岁的公羊羔作平安祭。这是{\PN{比大蓿}}儿子{\PN{迦玛列}}的供物。
\par }{\PP \VS{60}第九日{\ADD{来献的}}是{\PN{便雅悯}}子孙的首领、{\PN{基多尼}}的儿子{\PN{亚比但}}。
\VS{61}他的供物是:一个银盘子,重一百三十{\ADD{舍客勒}},一个银碗,重七十舍客勒,都是按圣所的平,也都盛满了调油的细面作素祭;
\VS{62}一个金盂,重十{\ADD{舍客勒}},盛满了香;
\VS{63}一只公牛犊,一只公绵羊,一只一岁的公羊羔作燔祭;
\VS{64}一只公山羊作赎罪祭;
\VS{65}两只公牛,五只公绵羊,五只公山羊,五只一岁的公羊羔作平安祭。这是{\PN{基多尼}}儿子{\PN{亚比但}}的供物。
\par }{\PP \VS{66}第十日{\ADD{来献的}}是{\PN{但}}子孙的首领、{\PN{亚米沙代}}的儿子{\PN{亚希以谢}}。
\VS{67}他的供物是:一个银盘子,重一百三十{\ADD{舍客勒}},一个银碗,重七十舍客勒,都是按圣所的平,也都盛满了调油的细面作素祭;
\VS{68}一个金盂,重十{\ADD{舍客勒}},盛满了香;
\VS{69}一只公牛犊,一只公绵羊,一只一岁的公羊羔作燔祭;
\VS{70}一只公山羊作赎罪祭;
\VS{71}两只公牛,五只公绵羊,五只公山羊,五只一岁的公羊羔作平安祭。这是{\PN{亚米沙代}}儿子{\PN{亚希以谢}}的供物。
\par }{\PP \VS{72}第十一日{\ADD{来献的}}是{\PN{亚设}}子孙的首领、{\PN{俄兰}}的儿子{\PN{帕结}}。
\VS{73}他的供物是:一个银盘子,重一百三十{\ADD{舍客勒}},一个银碗,重七十舍客勒,都是按圣所的平,也都盛满了调油的细面作素祭;
\VS{74}一个金盂,重十{\ADD{舍客勒}},盛满了香;
\VS{75}一只公牛犊,一只公绵羊,一只一岁的公羊羔作燔祭;
\VS{76}一只公山羊作赎罪祭;
\VS{77}两只公牛,五只公绵羊,五只公山羊,五只一岁的公羊羔作平安祭。这是{\PN{俄兰}}儿子{\PN{帕结}}的供物。
\par }{\PP \VS{78}第十二日{\ADD{来献的}}是{\PN{拿弗他利}}子孙的首领、{\PN{以南}}儿子{\PN{亚希拉}}。
\VS{79}他的供物是:一个银盘子,重一百三十{\ADD{舍客勒}},一个银碗,重七十舍客勒,都是按圣所的平,也都盛满了调油的细面作素祭;
\VS{80}一个金盂,重十{\ADD{舍客勒}},盛满了香;
\VS{81}一只公牛犊,一只公绵羊,一只一岁的公羊羔作燔祭;
\VS{82}一只公山羊作赎罪祭;
\VS{83}两只公牛,五只公绵羊,五只公山羊,五只一岁的公羊羔作平安祭。这是{\PN{以南}}儿子{\PN{亚希拉}}的供物。
\par }{\PP \VS{84}用膏抹坛的日子,{\PN{以色列}}的众首领为行献坛之礼所献的是:银盘子十二个,银碗十二个,金盂十二个;
\VS{85}每盘子{\ADD{重}}一百三十{\ADD{舍客勒}},每碗重七十{\ADD{舍客勒}}。一切器皿的银子,按圣所的平,共有二千四百{\ADD{舍客勒}}。
\VS{86}十二个金盂盛满了香,按圣所的平,每盂重十{\ADD{舍客勒}},所有的金子共一百二十{\ADD{舍客勒}}。
\VS{87}作燔祭的,共有公牛十二只,公羊十二只,一岁的公羊羔十二只,并同献的素祭作赎罪祭的公山羊十二只;
\VS{88}作平安祭的,共有公牛二十四只,公绵羊六十只,公山羊六十只,一岁的公羊羔六十只。这就是用膏抹坛之后,为行奉献坛之礼所献的。
\par }{\PP \VS{89}{\PN{摩西}}进会幕要与耶和华说话的时候,听见法柜的施恩座以上、二基路伯中间有与他说话的声音,就是耶和华与他说话。

\par }\Chap{8}{\SH 安置七盏灯
\par }{\PP \VerseOne{1}耶和华晓谕{\PN{摩西}}说:
\VS{2}「你告诉{\PN{亚伦}}说:点灯的时候,七盏灯都要向灯台前面发光。」
\VS{3}{\PN{亚伦}}便这样行。他点灯台上的灯,{\ADD{使灯}}向前{\ADD{发光}},是照耶和华所吩咐{\PN{摩西}}的。
\VS{4}这灯台的做法是用金子锤出来的,连座带花都是锤出来的。{\PN{摩西}}制造灯台,是照耶和华所指示的样式。
\par }{\SH 为利未人行洁净与奉献礼
\par }{\PP \VS{5}耶和华晓谕{\PN{摩西}}说:
\VS{6}「你从{\PN{以色列}}人中选出{\PN{利未}}人来,洁净他们。
\VS{7}洁净他们当这样行:用除罪水弹在他们身上,又叫他们用剃头刀刮全身,洗衣服,洁净自己。
\VS{8}然后叫他们取一只公牛犊,并同献的素祭,就是调油的细面;你要另取一只公牛犊作赎罪祭。
\VS{9}将{\PN{利未}}人奉到会幕前,招聚{\PN{以色列}}全会众。
\VS{10}将{\PN{利未}}人奉到耶和华面前,{\PN{以色列}}人要按手在他们{\ADD{头}}上。
\VS{11}{\PN{亚伦}}也将他们奉到耶和华面前,为{\PN{以色列}}人当作摇祭,使他们好办耶和华的事。
\VS{12}{\PN{利未}}人要按手在那两只牛的头上;你要将一只作赎罪祭,一只作燔祭,献给耶和华,为{\PN{利未}}人赎罪。
\VS{13}你也要使{\PN{利未}}人站在{\PN{亚伦}}和他儿子面前,将他们当作摇祭奉给耶和华。
\par }{\PP \VS{14}「这样,你从{\PN{以色列}}人中将{\PN{利未}}人分别出来,{\PN{利未}}人便要归我。
\VS{15}此后{\PN{利未}}人要进去办会幕的事,你要洁净他们,将他们当作摇祭奉上;
\VS{16}因为他们是从{\PN{以色列}}人中全然给我的,我拣选他们归我,是代替{\PN{以色列}}人中一切头生的。
\VS{17}{\PN{以色列}}人中一切头生的,连人带牲畜,都是我的。我在{\PN{埃及}}地击杀一切头生的那天,将他们分别为圣归我。
\VS{18}我拣选{\PN{利未}}人代替{\PN{以色列}}人中一切头生的。
\VS{19}我从{\PN{以色列}}人中将{\PN{利未}}人当作赏赐给{\PN{亚伦}}和他的儿子,在会幕中办{\PN{以色列}}人的事,又为{\PN{以色列}}人赎罪,免得他们挨近圣所,有灾殃临到他们中间。」
\par }{\PP \VS{20}{\PN{摩西}}、{\PN{亚伦}},并{\PN{以色列}}全会众便向{\PN{利未}}人如此行。凡耶和华指着{\PN{利未}}人所吩咐{\PN{摩西}}的,{\PN{以色列}}人就向他们这样行。
\VS{21}于是{\PN{利未}}人洁净自己,除了罪,洗了衣服;{\PN{亚伦}}将他们当作摇祭奉到耶和华面前,又为他们赎罪,洁净他们。
\VS{22}然后{\PN{利未}}人进去,在{\PN{亚伦}}和他儿子面前,在会幕中办事。耶和华指着{\PN{利未}}人怎样吩咐{\PN{摩西}},{\PN{以色列}}人就怎样向他们行了。
\par }{\PP \VS{23}耶和华晓谕{\PN{摩西}}说:
\VS{24}「{\PN{利未}}人是这样:从二十五岁以外,他们要前来任职,办会幕的事。
\VS{25}到了五十岁要停工退任,不再办事,
\VS{26}只要在会幕里,和他们的弟兄一同伺候,谨守所吩咐的,不再办事了。至于所吩咐{\PN{利未}}人的,你要这样向他们行。」

\par }\Chap{9}{\SH 第二个逾越节
\par }{\PP \VerseOne{1}{\PN{以色列}}人出{\PN{埃及}}地以后,第二年正月,耶和华在{\PN{西奈}}的旷野吩咐{\PN{摩西}}说:
\VS{2}「{\PN{以色列}}人应当在所定的日期守逾越节,
\VS{3}就是本月十四日黄昏的时候,你们要在所定的日期守这节,要按这节的律例典章而守。」
\VS{4}于是{\PN{摩西}}吩咐{\PN{以色列}}人守逾越节。
\VS{5}他们就在{\PN{西奈}}的旷野,正月十四日黄昏的时候,守逾越节。凡耶和华所吩咐{\PN{摩西}}的,{\PN{以色列}}人都照样行了。
\VS{6}有几个人因死尸而不洁净,不能在那日守逾越节。当日他们到{\PN{摩西}}、{\PN{亚伦}}面前,
\VS{7}说:「我们虽因死尸而不洁净,为何被阻止、不得同{\PN{以色列}}人在所定的日期献耶和华的供物呢?」
\VS{8}{\PN{摩西}}对他们说:「你们暂且等候,我可以去听耶和华指着你们是怎样吩咐的。」
\par }{\PP \VS{9}耶和华对{\PN{摩西}}说:
\VS{10}「你晓谕{\PN{以色列}}人说:你们和你们后代中,若有人因死尸而不洁净,或在远方行路,还要向耶和华守逾越节。
\VS{11}他们要在二月十四日黄昏的时候,守逾越节。要用无酵饼与苦菜,和逾越节{\ADD{的羊羔}}同吃。
\VS{12}一点不可留到早晨;羊羔的骨头一根也不可折断。他们要照逾越节的一切律例而守。
\VS{13}那洁净而不行路的人若推辞不守逾越节,那人要从民中剪除;因为他在所定的日期不献耶和华的供物,应该担当他的罪。
\VS{14}若有外人寄居在你们中间,愿意向耶和华守逾越节,他要照逾越节的律例典章行,不管是寄居的是本地人,同归一例。」
\par }{\SH 云彩遮盖帐幕
\par }{\R (出40·34—38)
\par }{\PP \VS{15}立起帐幕的那日,有云彩遮盖帐幕,就是法{\ADD{柜}}的帐幕;从晚上到早晨,云彩在其上,形状如火。
\VS{16}常是这样,云彩遮盖帐幕,夜间形状如火。
\VS{17}云彩几时从帐幕收上去,{\PN{以色列}}人就几时起行;云彩在哪里停住,{\PN{以色列}}人就在那里安营。
\VS{18}{\PN{以色列}}人遵耶和华的吩咐起行,也遵耶和华的吩咐安营。云彩在帐幕上停住几时,他们就住营几时。
\VS{19}云彩在帐幕上停留许多日子,{\PN{以色列}}人就守耶和华所吩咐的不起行。
\VS{20}有时云彩在帐幕上几天,他们就照耶和华的吩咐住营,也照耶和华的吩咐起行。
\VS{21}有时从晚上到早晨,有这云彩{\ADD{在帐幕上}};早晨云彩收上去,他们就起行。有时昼夜{\ADD{云彩停在帐幕上}},收上去的时候,他们就起行。
\VS{22}云彩停留在帐幕上,无论是两天,是一月,是一年,{\PN{以色列}}人就住营不起行;但云彩收上去,他们就起行。
\VS{23}他们遵耶和华的吩咐安营,也遵耶和华的吩咐起行。他们守耶和华所吩咐的,都是凭耶和华吩咐{\PN{摩西}}的。

\par }\Chap{10}{\SH 制银号
\par }{\PP \VerseOne{1}耶和华晓谕{\PN{摩西}}说:
\VS{2}「你要用银子做两枝号,都要锤出来的,用以招聚会众,并叫众营起行。
\par }{\PP \VS{3}吹这号的时候,全会众要到你那里,聚集在会幕门口。
\VS{4}若单吹一枝,众首领,就是{\PN{以色列}}军中的统领,要聚集到你那里。
\VS{5}吹出大声的时候,东边安的营都要起行。
\VS{6}二次吹出大声的时候,南边安的营都要起行。他们将起行,必吹出大声。
\VS{7}但招聚会众的时候,你们要吹号,却不要吹出大声。
\VS{8}{\PN{亚伦}}子孙作祭司的要吹这号;这要作你们世世代代永远的定例。
\VS{9}你们在自己的地,与欺压你们的敌人打仗,就要用号吹出大声,便在耶和华—你们的 神面前得蒙纪念,也蒙拯救脱离仇敌。
\VS{10}在你们快乐的日子和节期,并月朔,献燔祭和平安祭,也要吹号,这都要在你们的 神面前作为纪念。我是耶和华—你们的 神。」
\par }{\SH 离开西奈山
\par }{\PP \VS{11}第二年二月二十日,云彩从法{\ADD{柜}}的帐幕收上去。
\VS{12}{\PN{以色列}}人就按站往前行,离开{\PN{西奈}}的旷野,云彩停住在{\PN{巴兰}}的旷野。
\VS{13}这是他们照耶和华借{\PN{摩西}}所吩咐的,初次往前行。
\VS{14}按着军队首先往前行的是{\PN{犹大}}营的纛。统领军队的是{\PN{亚米拿达}}的儿子{\PN{拿顺}}。
\VS{15}统领{\PN{以萨迦}}支派军队的是{\PN{苏押}}的儿子{\PN{拿坦业}}。
\VS{16}统领{\PN{西布伦}}支派军队的是{\PN{希伦}}的儿子{\PN{以利押}}。
\par }{\PP \VS{17}帐幕拆卸,{\PN{革顺}}的子孙和{\PN{米拉利}}的子孙就抬着帐幕先往前行。
\VS{18}按着军队往前行的是{\PN{吕便}}营的纛。统领军队的是{\PN{示丢珥}}的儿子{\PN{以利蓿}}。
\VS{19}统领{\PN{西缅}}支派军队的是{\PN{苏利沙代}}的儿子{\PN{示路蔑}}。
\VS{20}统领{\PN{迦得}}支派军队的是{\PN{丢珥}}的儿子{\PN{以利雅萨}}。
\par }{\PP \VS{21}{\PN{哥辖}}人抬着圣物先往前行。他们未到以前,{\ADD{抬帐幕的}}已经把帐幕支好。
\VS{22}按着军队往前行的是{\PN{以法莲}}营的纛,统领军队的是{\PN{亚米忽}}的儿子{\PN{以利沙玛}}。
\VS{23}统领{\PN{玛拿西}}支派军队的是{\PN{比大蓿}}的儿子{\PN{迦玛列}}。
\VS{24}统领{\PN{便雅悯}}支派军队的是{\PN{基多尼}}的儿子{\PN{亚比但}}。
\par }{\PP \VS{25}在诸营末后的是{\PN{但}}营的纛,按着军队往前行。统领军队的是{\PN{亚米沙代}}的儿子{\PN{亚希以谢}}。
\VS{26}统领{\PN{亚设}}支派军队的是{\PN{俄兰}}的儿子{\PN{帕结}}。
\VS{27}统领{\PN{拿弗他利}}支派军队的是{\PN{以南}}的儿子{\PN{亚希拉}}。
\VS{28}{\PN{以色列}}人按着军队往前行,就是这样。
\par }{\PP \VS{29}{\PN{摩西}}对他岳父\FTNT{}{{\FR 10:29: }或译:内兄}—{\PN{米甸}}人{\PN{流珥}}的儿子{\PN{何巴}}—说:「我们要行路,往耶和华所应许之地去;他曾说:『我要将这地赐给你们。』现在求你和我们同去,我们必厚待你,因为耶和华指着{\PN{以色列}}人已经应许给好处。」
\VS{30}{\PN{何巴}}回答说:「我不去;我要回本地本族那里去。」
\VS{31}{\PN{摩西}}说:「求你不要离开我们;因为你知道我们要在旷野安营,你可以当作我们的眼目。
\VS{32}你若和我们同去,将来耶和华有什么好处待我们,我们也必以什么好处待你。」
\par }{\SH 众民和约柜往前行
\par }{\PP \VS{33}{\PN{以色列}}人离开耶和华的山,往前行了三天的路程;耶和华的约柜在前头行了三天的路程,为他们寻找安歇的地方。
\VS{34}他们拔营往前行,日间有耶和华的云彩在他们以上。
\par }{\PP \VS{35}约柜往前行的时候,{\PN{摩西}}就说:「耶和华啊,求你兴起!愿你的仇敌四散!愿恨你的人从你面前逃跑!」
\VS{36}约柜停住的时候,他就说:「耶和华啊,求你回到{\PN{以色列}}的千万人中!」

\par }\Chap{11}{\SH 在他备拉发怨言
\par }{\PP \VerseOne{1}众百姓发怨言,他们的恶语达到耶和华的耳中。耶和华听见了就怒气发作,使火在他们中间焚烧,直烧到营的边界。
\VS{2}百姓向{\PN{摩西}}哀求,{\PN{摩西}}祈求耶和华,火就熄了。
\VS{3}那地方便叫做{\PN{他备拉}},因为耶和华的火烧在他们中间。
\par }{\SH 摩西选立七十个长老
\par }{\PP \VS{4}他们中间的闲杂人大起贪欲的心;{\PN{以色列}}人又哭号说:「谁给我们肉吃呢?
\VS{5}我们记得,在{\PN{埃及}}的时候不花钱就吃鱼,也记得有黄瓜、西瓜、韭菜、葱、蒜。
\VS{6}现在我们的心血枯竭了,除这吗哪以外,在我们眼前并没有别的东西。」
\par }{\PP \VS{7}这吗哪仿佛芫荽子,又好像珍珠。
\VS{8}百姓周围行走,把吗哪收起来,或用磨推,或用臼捣,煮在锅中,又做成饼,滋味好像新油。
\VS{9}夜间露水降在营中,吗哪也随着降下。)
\par }{\PP \VS{10}{\PN{摩西}}听见百姓各在各家的帐棚门口哭号。耶和华的怒气便大发作,{\PN{摩西}}就不喜悦。
\VS{11}{\PN{摩西}}对耶和华说:「你为何苦待仆人?我为何不在你眼前蒙恩,竟把这管理百姓的重任加在我身上呢?
\VS{12}这百姓岂是我怀的胎,岂是我生下来的呢?你竟对我说:『把他们抱在怀里,如养育之父抱吃奶的孩子,直抱到你起誓应许给他们祖宗的地去。』
\VS{13}我从哪里得肉给这百姓吃呢?他们都向我哭号说:『你给我们肉吃吧!』
\VS{14}管理这百姓的责任太重了,我独自担当不起。
\VS{15}你这样待我,我若在你眼前蒙恩,求你立时将我杀了,不叫我见自己的苦情。」
\par }{\PP \VS{16}耶和华对{\PN{摩西}}说:「你从{\PN{以色列}}的长老中招聚七十个人,就是你所知道作百姓的长老和官长的,到我这里来,领他们到会幕前,使他们和你一同站立。
\VS{17}我要在那里降临,与你说话,也要把降于你身上的灵分赐他们,他们就和你同当这管百姓的重任,免得你独自担当。
\VS{18}又要对百姓说:『你们应当自洁,预备明天吃肉,因为你们哭号说:谁给我们肉吃!我们在{\PN{埃及}}很好。这声音达到了耶和华的耳中,所以他必给你们肉吃。
\VS{19}你们不止吃一天、两天、五天、十天、二十天,
\VS{20}要吃一个整月,甚至肉从你们鼻孔里喷出来,使你们厌恶了,因为你们厌弃住在你们中间的耶和华,在他面前哭号说:我们为何出了{\PN{埃及}}呢!』」
\VS{21}{\PN{摩西}}{\ADD{对耶和华}}说:「这与我同住的百姓、步行的男人有六十万,你还说:『我要把肉给他们,使他们可以吃一个整月。』
\VS{22}难道给他们宰了羊群牛群,或是把海中所有的鱼都聚了来,就够他们吃吗?」
\VS{23}耶和华对{\PN{摩西}}说:「耶和华的膀臂岂是缩短了吗?现在要看我的话向你应验不应验。」
\par }{\PP \VS{24}{\PN{摩西}}出去,将耶和华的话告诉百姓,又招聚百姓的长老中七十个人来,使他们站在{\ADD{会}}幕的四围。
\VS{25}耶和华在云中降临,对{\PN{摩西}}说话,把降与他身上的灵分赐那七十个长老。灵停在他们身上的时候,他们就受感说话,以后却没有再说。
\par }{\PP \VS{26}但有两个人仍在营里,一个名叫{\PN{伊利达}},一个名叫{\PN{米达}}。他们本是在那些被录的人中,却没有到{\ADD{会}}幕那里去。灵停在他们身上,他们就在营里说预言。
\VS{27}有个少年人跑来告诉{\PN{摩西}}说:「{\PN{伊利达}}、{\PN{米达}}在营里说预言。」
\VS{28}{\PN{摩西}}的帮手,{\PN{嫩}}的儿子{\PN{约书亚}},就是{\PN{摩西}}所拣选的一个人,说:「请我主{\PN{摩西}}禁止他们。」
\VS{29}{\PN{摩西}}对他说:「你为我的缘故嫉妒人吗?惟愿耶和华的百姓都受感说话!愿耶和华把他的灵降在他们身上!」
\VS{30}于是,{\PN{摩西}}和{\PN{以色列}}的长老都回到营里去。
\par }{\SH  神赐鹌鹑为食物
\par }{\PP \VS{31}有风从耶和华那里刮起,把鹌鹑由海面刮来,飞散在营边和营的四围;这边约有一天的路程,那边约有一天的路程,离地面约有二肘。
\VS{32}百姓起来,终日终夜,并次日一整天,捕取鹌鹑;至少的也取了十贺梅珥,为自己摆列在营的四围。
\VS{33}肉在他们牙齿之间尚未嚼烂,耶和华的怒气就向他们发作,用最重的灾殃击杀了他们。
\VS{34}那地方便叫做{\PN{基博罗·哈他瓦}}\FTNT{}{{\FR 11:34: }就是贪欲之人的坟墓},因为他们在那里葬埋那起贪欲之心的人。
\VS{35}百姓从{\PN{基博罗·哈他瓦}}走到{\PN{哈洗录}},就住在{\PN{哈洗录}}。

\par }\Chap{12}{\SH 米利暗受罚
\par }{\PP \VerseOne{1}{\PN{摩西}}娶了{\PN{古实}}女子为妻。{\PN{米利暗}}和{\PN{亚伦}}因他所娶的{\PN{古实}}女子就毁谤他,说:
\VS{2}「难道耶和华单与{\PN{摩西}}说话,不也与我们说话吗?」这话耶和华听见了。
\VS{3}{\PN{摩西}}为人极其谦和,胜过世上的众人。
\VS{4}耶和华忽然对{\PN{摩西}}、{\PN{亚伦}}、{\PN{米利暗}}说:「你们三个人都出来,到会幕这里。」他们三个人就出来了。
\VS{5}耶和华在云柱中降临,站在{\ADD{会}}幕门口,召{\PN{亚伦}}和{\PN{米利暗}},二人就出来了。
\VS{6}耶和华说:「你们且听我的话:你们中间若有先知,我—耶和华必在异象中向他显现,在梦中与他说话。
\VS{7}我的仆人{\PN{摩西}}不是这样;他是在我全家尽忠的。
\VS{8}我要与他面对面说话,乃是明说,不用谜语,并且他必见我的形象。你们毁谤我的仆人{\PN{摩西}},为何不惧怕呢?」
\par }{\PP \VS{9}耶和华就向他们二人发怒而去。
\VS{10}云彩从{\ADD{会}}幕上挪开了,不料,{\PN{米利暗}}长了大麻风,有雪{\ADD{那样}}白。{\PN{亚伦}}一看{\PN{米利}}暗长了大麻风,
\VS{11}就对{\PN{摩西}}说:「我主啊,求你不要因我们愚昧犯罪,便将这罪加在我们身上。
\VS{12}求你不要使她像那出母腹、肉已半烂的死胎。」
\VS{13}于是{\PN{摩西}}哀求耶和华说:「 神啊,求你医治她!」
\VS{14}耶和华对{\PN{摩西}}说:「她父亲若吐唾沫在她脸上,她岂不蒙羞七天吗?现在要把她在营外关锁七天,然后才可以领她进来。」
\VS{15}于是{\PN{米利暗}}关锁在营外七天。百姓没有行路,直等到把{\PN{米利暗}}领进来。
\VS{16}以后百姓从{\PN{哈洗录}}起行,在{\PN{巴兰}}的旷野安营。

\par }\Chap{13}{\SH 十二个探子
\par }{\R (申1·19—33)
\par }{\PP \VerseOne{1}耶和华晓谕{\PN{摩西}}说:
\VS{2}「你打发人去窥探我所赐给{\PN{以色列}}人的{\PN{迦南}}地,他们每支派中要打发一个人,都要作首领的。」
\VS{3}{\PN{摩西}}就照耶和华的吩咐从{\PN{巴兰}}的旷野打发他们去;他们都是{\PN{以色列}}人的族长。
\VS{4}他们的名字:属{\PN{吕便}}支派的有{\PN{撒刻}}的儿子{\PN{沙母亚}}。
\VS{5}属{\PN{西缅}}支派的有{\PN{何利}}的儿子{\PN{沙法}}。
\VS{6}属{\PN{犹大}}支派的有{\PN{耶孚尼}}的儿子{\PN{迦勒}}。
\VS{7}属{\PN{以萨迦}}支派的有{\PN{约色}}的儿子{\PN{以迦}}。
\VS{8}属{\PN{以法莲}}支派的有{\PN{嫩}}的儿子{\PN{何希阿}}。
\VS{9}属{\PN{便雅悯}}支派的有{\PN{拉孚}}的儿子{\PN{帕提}}。
\VS{10}属{\PN{西布伦}}支派的有{\PN{梭底}}的儿子{\PN{迦叠}}。
\VS{11}{\PN{约瑟}}的子孙,属{\PN{玛拿西}}支派的有{\PN{稣西}}的儿子{\PN{迦底}}。
\VS{12}属{\PN{但}}支派的有{\PN{基玛利}}的儿子{\PN{亚米利}}。
\VS{13}属{\PN{亚设}}支派的有{\PN{米迦勒}}的儿子{\PN{西帖}}。
\VS{14}属{\PN{拿弗他利}}支派的有{\PN{缚西}}的儿子{\PN{拿比}}。
\VS{15}属{\PN{迦得}}支派的有{\PN{玛基}}的儿子{\PN{臼利}}。
\VS{16}这就是{\PN{摩西}}所打发、窥探那地之人的名字。{\PN{摩西}}就称{\PN{嫩}}的儿子{\PN{何希阿}}为{\PN{约书亚}}。
\par }{\PP \VS{17}{\PN{摩西}}打发他们去窥探{\PN{迦南}}地,说:「你们从南地上山地去,
\VS{18}看那地如何,其中所住的民是强是弱,是多是少,
\VS{19}所住之地是好是歹,所住之处是营盘是坚城。
\VS{20}又看那地土是肥美是瘠薄,其中有树木没有。你们要放开胆量,把那地的果子带些来。」(那时正是葡萄初熟的时候。)
\par }{\PP \VS{21}他们上去窥探那地,从{\PN{寻}}的旷野到{\PN{利合}},直到{\PN{哈马口}}。
\VS{22}他们从南地上去,到了{\PN{希伯
}};在那里有{\PN{亚衲}}族人{\PN{亚希幔}}、{\PN{示筛}}、{\PN{挞买}}。(原来{\PN{希伯
}}城被建造比{\PN{埃及}}的{\PN{锁安}}城早七年。)
\VS{23}他们到了{\PN{以实各谷}},从那里砍了葡萄树的一枝,上头有一挂葡萄,两个人用杠抬着,又{\ADD{带了}}些石榴和无花果来。(
\VS{24}因为{\PN{以色列}}人从那里砍来的那挂葡萄,所以那地方叫做{\PN{以实各谷}}。)
\par }{\PP \VS{25}过了四十天,他们窥探那地才回来,
\VS{26}到了{\PN{巴兰}}旷野的{\PN{加低斯}},见{\PN{摩西}}、{\PN{亚伦}},并{\PN{以色列}}的全会众,回报{\PN{摩西}}、{\PN{亚伦}},并全会众,又把那地的果子给他们看;
\VS{27}又告诉{\PN{摩西}}说:「我们到了你所打发我们去的那地,果然是流奶与蜜之地;这就是那地的果子。
\VS{28}然而住那地的民强壮,城邑也坚固宽大,并且我们在那里看见了{\PN{亚衲}}族的人。
\VS{29}{\PN{亚玛力}}人住在南地;{\PN{赫}}人、{\PN{耶布斯}}人、{\PN{亚摩利}}人住在山地;{\PN{迦南}}人住在海边并{\PN{约旦河}}旁。」
\par }{\PP \VS{30}{\PN{迦勒}}在{\PN{摩西}}面前安抚百姓,说:「我们立刻上去得那地吧!我们足能得胜。」
\VS{31}但那些和他同去的人说:「我们不能上去攻击那民,因为他们比我们强壮。」
\VS{32}探子{\ADD{中有}}人论到所窥探之地,向{\PN{以色列}}人报恶信,说:「我们所窥探、经过之地是吞吃居民之地,我们在那里所看见的人民都身量高大。
\VS{33}我们在那里看见{\PN{亚衲}}族人,就是伟人;他们是伟人的后裔。据我们看,自己就如蚱蜢一样;据他们看,我们也是如此。」

\par }\Chap{14}{\SH 民众埋怨
\par }{\PP \VerseOne{1}当下,全会众大声喧嚷;那夜百姓都哭号。
\VS{2}{\PN{以色列}}众人向{\PN{摩西}}、{\PN{亚伦}}发怨言;全会众对他们说:「巴不得我们早死在{\PN{埃及}}地,或是死在这旷野。
\VS{3}耶和华为什么把我们领到那地,使我们倒在刀下呢?我们的妻子和孩子必被掳掠。我们回{\PN{埃及}}去岂不好吗?」
\VS{4}众人彼此说:「我们不如立一个首领回{\PN{埃及}}去吧!」
\par }{\PP \VS{5}{\PN{摩西}}、{\PN{亚伦}}就俯伏在{\PN{以色列}}全会众面前。
\VS{6}窥探地的人中,{\PN{嫩}}的儿子{\PN{约书亚}}和{\PN{耶孚尼}}的儿子{\PN{迦勒}}撕裂衣服,
\VS{7}对{\PN{以色列}}全会众说:「我们所窥探、经过之地是极美之地。
\VS{8}耶和华若喜悦我们,就必将我们领进那地,把地赐给我们;那地原是流奶与蜜之地。
\VS{9}但你们不可背叛耶和华,也不要怕那地的居民;因为他们是我们的食物,并且荫庇他们的已经离开他们。有耶和华与我们同在,不要怕他们!」
\VS{10}但全会众说:「拿石头打死他们{\ADD{二人}}。」忽然,耶和华的荣光在会幕中向{\PN{以色列}}众人显现。
\par }{\SH 摩西为民众祷告
\par }{\PP \VS{11}耶和华对{\PN{摩西}}说:「这百姓藐视我要到几时呢?我在他们中间行了这一切神迹,他们还不信我要到几时呢?
\VS{12}我要用瘟疫击杀他们,使他们不得承受{\ADD{那地}},叫你的后裔成为大国,比他们强胜。」
\par }{\PP \VS{13}{\PN{摩西}}对耶和华说:「{\PN{埃及}}人必听见这事;因为你曾施展大能,将这百姓从他们中间领上来。
\VS{14}{\PN{埃及}}人要将这事传给{\PN{迦南}}地的居民;那民已经听见你—耶和华是在这百姓中间;因为你面对面被人看见,有你的云彩停在他们以上。你日间在云柱中,夜间在火柱中,在他们前面行。
\VS{15}如今你若把这百姓杀了,如杀一人,那些听见你名声的列邦必议论说:
\VS{16}『耶和华因为不能把这百姓领进他向他们起誓应许之地,所以在旷野把他们杀了。』
\VS{17}现在求主大显能力,照你所说过的话说:
\VS{18}『耶和华不轻易发怒,并有丰盛的慈爱,赦免罪孽和过犯;万不以{\ADD{有罪的}}为无罪,必追讨他的罪,自父及子,直到三、四代。』
\VS{19}求你照你的大慈爱赦免这百姓的罪孽,好像你从{\PN{埃及}}到如今常赦免他们一样。」
\par }{\PP \VS{20}耶和华说:「我照着你的话赦免了他们。
\VS{21}然我指着我的永生起誓,遍地要被我的荣耀充满。
\VS{22}这些人虽看见我的荣耀和我在{\PN{埃及}}与旷野所行的神迹,仍然试探我这十次,不听从我的话,
\VS{23}他们断不得看见我向他们的祖宗所起誓应许之地。凡藐视我的,一个也不得看见;
\VS{24}惟独我的仆人{\PN{迦勒}},因他另有一个心志,专一跟从我,我就把他领进他所去过的那地;他的后裔也必得那地为业。
\VS{25}{\PN{亚玛力}}人和{\PN{迦南}}人住在谷中,明天你们要转回,从{\PN{红海}}的路往旷野去。」
\par }{\SH 耶和华惩罚埋怨的民众
\par }{\PP \VS{26}耶和华对{\PN{摩西}}、{\PN{亚伦}}说:
\VS{27}「这恶会众向我发怨言,{\ADD{我忍耐}}他们要到几时呢?{\PN{以色列}}人向我所发的怨言,我都听见了。
\VS{28}你们告诉他们,耶和华说:『我指着我的永生起誓,我必要照你们达到我耳中的话待你们。
\VS{29}你们的尸首必倒在这旷野,并且你们中间凡被数点、从二十岁以外、向我发怨言的,
\VS{30}必不得进我起誓应许叫你们住的那地;惟有{\PN{耶孚尼}}的儿子{\PN{迦勒}}和{\PN{嫩}}的儿子{\PN{约书亚}}才能进去。
\VS{31}但你们的{\ADD{妇人}}孩子,就是你们所说、要被掳掠的,我必把他们领进去,他们就得知你们所厌弃的那地。
\VS{32}至于你们,你们的尸首必倒在这旷野;
\VS{33}你们的儿女必在旷野飘流四十年,担当你们淫行的罪,直到你们的尸首在旷野消灭。
\VS{34}按你们窥探那地的四十日,一年顶一日,你们要担当罪孽四十年,就知道我与你们疏远了,
\VS{35}我—耶和华说过,我总要这样待这一切聚集敌我的恶会众;他们必在这旷野消灭,在这里死亡。』」
\par }{\PP \VS{36}{\PN{摩西}}所打发、窥探那地的人回来,报那地的恶信,叫全会众向{\PN{摩西}}发怨言,
\VS{37}这些报恶信的人都遭瘟疫,死在耶和华面前。
\VS{38}其中惟有{\PN{嫩}}的儿子{\PN{约书亚}}和{\PN{耶孚尼}}的儿子{\PN{迦勒}}仍然存活。
\par }{\SH 首次进攻迦南
\par }{\R (申1·41—46)
\par }{\PP \VS{39}{\PN{摩西}}将这些话告诉{\PN{以色列}}众人,他们就甚悲哀。
\VS{40}清早起来,上山顶去,说:「我们在这里,我们有罪了;情愿上耶和华所应许的地方去。」
\VS{41}{\PN{摩西}}说:「你们为何违背耶和华的命令呢?这事不能顺利了。
\VS{42}不要上去;因为耶和华不在你们中间,恐怕你们被仇敌杀败了。
\VS{43}{\PN{亚玛力}}人和{\PN{迦南}}人都在你们面前,你们必倒在刀下;因你们退回不跟从耶和华,所以他必不与你们同在。」
\VS{44}他们却擅敢上山顶去,然而耶和华的约柜和{\PN{摩西}}没有出营。
\VS{45}于是{\PN{亚玛力}}人和住在那山上的{\PN{迦南}}人都下来击打他们,把他们杀退了,直到{\PN{何珥玛}}。

\par }\Chap{15}{\SH 献祭的条例
\par }{\PP \VerseOne{1}耶和华对{\PN{摩西}}说:
\VS{2}「你晓谕{\PN{以色列}}人说:你们到了我所赐给你们居住的地,
\VS{3}若愿意从牛群羊群中取牛羊作火祭,献给耶和华,无论是燔祭是{\ADD{平安}}祭,为要还特许的愿,或是作甘心祭,或是逢你们节期献的,都要奉给耶和华为馨香{\ADD{之祭}}。
\VS{4}那献供物的就要将细面{\ADD{伊法}}十分之一,并油一欣四分之一,调和作素祭,献给耶和华。
\VS{5}无论是燔祭是{\ADD{平安}}祭,你要为每只绵羊羔,一同预备奠祭的酒一欣四分之一。
\VS{6}为公绵羊预备细面{\ADD{伊法}}十分之二,并油一欣三分之一,调和作素祭,
\VS{7}又用酒一欣三分之一作奠祭,献给耶和华为馨香{\ADD{之祭}}。
\VS{8}你预备公牛作燔祭,或是作平安祭,为要还特许的愿,或是作平安祭,献给耶和华,
\VS{9}就要把细面{\ADD{伊法}}十分之三,并油半欣,调和作素祭,和公牛一同献上,
\VS{10}又用酒半欣作奠祭,献给耶和华为馨香的火祭。
\par }{\PP \VS{11}「献公牛、公绵羊、绵羊羔、山羊羔,每只都要这样办理。
\VS{12}照你们所预备的数目,按着只数都要这样办理。
\VS{13}凡本地人将馨香的火祭献给耶和华,都要这样办理。
\VS{14}若有外人和你们同居,或有人世世代代住在你们中间,愿意将馨香的火祭献给耶和华,你们怎样办理,他也要照样办理。
\VS{15}至于会众,你们和同居的外人都归一例,作为你们世世代代永远的定例,在耶和华面前,你们怎样,寄居的也要怎样。
\VS{16}你们并与你们同居的外人当有一样的条例,一样的典章。」
\par }{\PP \VS{17}耶和华对{\PN{摩西}}说:
\VS{18}「你晓谕{\PN{以色列}}人说:你们到了我所领你们进去的那地,
\VS{19}吃那地的粮食,就要把举祭献给耶和华。
\VS{20}你们要用初熟的麦子磨面,做饼当举祭奉献;你们举上,好像举禾场的举祭一样。
\VS{21}你们世世代代要用初熟的麦子磨面,当举祭献给耶和华。
\par }{\PP \VS{22}「你们有错误的时候,不守耶和华所晓谕{\PN{摩西}}的这一切命令,
\VS{23}就是耶和华借{\PN{摩西}}一切所吩咐你们的,自那日以至你们的世世代代,
\VS{24}若有误行,是会众所不知道的,后来全会众就要将一只公牛犊作燔祭,并照典章把素祭和奠祭一同献给耶和华为馨香{\ADD{之祭}},又献一只公山羊作赎罪祭。
\VS{25}祭司要为{\PN{以色列}}全会众赎罪,他们就必蒙赦免,因为这是错误。他们又因自己的错误,把供物,就是向耶和华献的火祭和赎罪祭,一并奉到耶和华面前。
\VS{26}{\PN{以色列}}全会众和寄居在他们中间的外人就必蒙赦免,因为这罪是百姓误犯的。
\par }{\PP \VS{27}「若有一个人误犯了罪,他就要献一岁的母山羊作赎罪祭。
\VS{28}那误行的人犯罪的时候,祭司要在耶和华面前为他赎罪,他就必蒙赦免。
\VS{29}{\PN{以色列}}中的本地人和寄居在他们中间的外人,若误行了什么事,必归一样的条例。
\VS{30}但那擅敢行事的,无论是本地人是寄居的,他亵渎了耶和华,必从民中剪除。
\VS{31}因他藐视耶和华的言语,违背耶和华的命令,那人总要剪除;他的罪孽要归到他身上。」
\par }{\SH 违反安息日的人
\par }{\PP \VS{32}{\PN{以色列}}人在旷野的时候,遇见一个人在安息日捡柴。
\VS{33}遇见他捡柴的人,就把他带到{\PN{摩西}}、{\PN{亚伦}}并全会众那里,
\VS{34}将他收在监内;因为当怎样办他,还没有指明。
\VS{35}耶和华吩咐{\PN{摩西}}说:「总要把那人治死;全会众要在营外用石头把他打死。」
\VS{36}于是全会众将他带到营外,用石头打死他,是照耶和华所吩咐{\PN{摩西}}的。
\par }{\SH 有关衣服 子的条例
\par }{\PP \VS{37}耶和华晓谕{\PN{摩西}}说:
\VS{38}「你吩咐{\PN{以色列}}人,叫他们世世代代在衣服边上做 子,又在底边的 子上钉一根蓝细带子。
\VS{39}你们佩带这 子,好叫你们看见就记念遵行耶和华一切的命令,不随从自己的心意、眼目行邪淫,像你们素常一样;
\VS{40}使你们记念遵行我一切的命令,成为圣洁,归与你们的 神。
\par }{\PP \VS{41}「我是耶和华—你们的 神,曾把你们从{\PN{埃及}}地领出来,要作你们的 神。我是耶和华—你们的 神。」

\par }\Chap{16}{\SH 可拉、大坍、亚比兰的叛变
\par }{\PP \VerseOne{1}{\PN{利未}}的曾孙、{\PN{哥辖}}的孙子、{\PN{以斯哈}}的儿子{\PN{可拉}},和{\PN{吕便}}子孙中{\PN{以利押}}的儿子{\PN{大坍}}、{\PN{亚比兰}},与{\PN{比勒}}的儿子{\PN{安}},
\VS{2}并{\PN{以色列}}会中的二百五十个首领,就是有名望选入会中的人,在{\PN{摩西}}面前一同起来,
\VS{3}聚集攻击{\PN{摩西}}、{\PN{亚伦}},说:「你们擅自专权!全会众个个既是圣洁,耶和华也在他们中间,你们为什么自高,超过耶和华的会众呢?」
\VS{4}{\PN{摩西}}听见这话就俯伏在地,
\VS{5}对{\PN{可拉}}和他一党的人说:「到了早晨,耶和华必指示谁是属他的,谁是圣洁的,就叫谁亲近他;他所拣选的是谁,必叫谁亲近他。
\VS{6}{\PN{可拉}}啊,你们要这样行,你和你的一党要拿香炉来。
\VS{7}明日,在耶和华面前,把火盛在炉中,把香放在其上。耶和华拣选谁,谁就为圣洁。你们这{\PN{利未}}的子孙擅自专权了!」
\VS{8}{\PN{摩西}}又对{\PN{可拉}}说:「{\PN{利未}}的子孙哪,你们听我说!
\VS{9}{\PN{以色列}}的 神从{\PN{以色列}}会中将你们分别出来,使你们亲近他,办耶和华帐幕的事,并站在会众面前替他们当差。
\VS{10}耶和华又使你和你一切弟兄—{\PN{利未}}的子孙—一同亲近他,这岂为小事?你们还要求祭司的职任吗?
\VS{11}你和你一党的人聚集是要攻击耶和华。{\PN{亚伦}}算什么,你们竟向他发怨言呢?」
\par }{\PP \VS{12}{\PN{摩西}}打发人去召{\PN{以利押}}的儿子{\PN{大坍}}、{\PN{亚比兰}}。他们说:「我们不上去!
\VS{13}你将我们从流奶与蜜之地领上来,要在旷野杀我们,这岂为小事?你还要自立为王辖管我们吗?
\VS{14}并且你没有将我们领到流奶与蜜之地,也没有把田地和葡萄园给我们为业。难道你要剜这些人的眼睛吗?我们不上去!」
\par }{\PP \VS{15}{\PN{摩西}}就甚发怒,对耶和华说:「求你不要享受他们的供物。我并没有夺过他们一匹驴,也没有害过他们一个人。」
\VS{16}{\PN{摩西}}对{\PN{可拉}}说:「明天,你和你一党的人,并{\PN{亚伦}},都要站在耶和华面前;
\VS{17}各人要拿一个香炉,共二百五十个,把香放在上面,到耶和华面前。你和{\PN{亚伦}}也各拿自己的香炉。」
\VS{18}于是他们各人拿一个香炉,盛上火,加上香,同{\PN{摩西}}、{\PN{亚伦}}站在会幕门前。
\VS{19}{\PN{可拉}}招聚全会众到会幕门前,要攻击{\PN{摩西}}、{\PN{亚伦}};耶和华的荣光就向全会众显现。
\par }{\PP \VS{20}耶和华晓谕{\PN{摩西}}、{\PN{亚伦}}说:
\VS{21}「你们离开这会众,我好在转眼之间把他们灭绝。」
\VS{22}{\PN{摩西}}、{\PN{亚伦}}就俯伏在地,说:「 神,万人之灵的 神啊,一人犯罪,你就要向全会众发怒吗?」
\VS{23}耶和华晓谕{\PN{摩西}}说:
\VS{24}「你吩咐会众说:『你们离开{\PN{可拉}}、{\PN{大坍}}、{\PN{亚比兰}}帐棚的四围。』」
\par }{\PP \VS{25}{\PN{摩西}}起来,往{\PN{大坍}}、{\PN{亚比兰}}那里去;{\PN{以色列}}的长老也随着他去。
\VS{26}他吩咐会众说:「你们离开这恶人的帐棚吧,他们的物件,什么都不可摸,恐怕你们陷在他们的罪中,与他们一同消灭。」
\VS{27}于是众人离开{\PN{可拉}}、{\PN{大坍}}、{\PN{亚比兰}}帐棚的四围。{\PN{大坍}}、{\PN{亚比兰}}带着妻子、儿女、小孩子,都出来,站在自己的帐棚门口。
\VS{28}{\PN{摩西}}说:「我行的这一切事本不是凭我自己心意行的,乃是耶和华打发我行的,必有证据使你们知道。
\VS{29}这些人死若与世人无异,或是他们所遭的与世人相同,就不是耶和华打发我来的。
\VS{30}倘若耶和华创作一件新事,使地开口,把他们和一切属他们的都吞下去,叫他们活活地坠落阴间,你们就明白这些人是藐视耶和华了。」
\par }{\PP \VS{31}{\PN{摩西}}刚说完了这一切话,他们脚下的地就开了口,
\VS{32}把他们和他们的家眷,并一切属{\PN{可拉}}的人丁、财物,都吞下去。
\VS{33}这样,他们和一切属他们的,都活活地坠落阴间;地口在他们上头照旧合闭,他们就从会中灭亡。
\VS{34}在他们四围的{\PN{以色列}}众人听他们呼号,就都逃跑,说:「恐怕地也把我们吞下去。」
\VS{35}又有火从耶和华那里出来,烧灭了那献香的二百五十个人。
\par }{\SH 香炉
\par }{\PP \VS{36}耶和华晓谕{\PN{摩西}}说:
\VS{37}「你吩咐祭司{\PN{亚伦}}的儿子{\PN{以利亚撒}}从火中捡起那些香炉来,把火撒在别处,因为那些香炉是圣的。
\VS{38}把那些犯罪、自害己命之人的香炉,叫人锤成片子,用以包坛。那些香炉本是他们在耶和华面前献过的,所以是圣的,并且可以给{\PN{以色列}}人作记号。」
\VS{39}于是祭司{\PN{以利亚撒}}将被烧之人所献的铜香炉拿来,人就锤出来,用以包坛,
\VS{40}给{\PN{以色列}}人作纪念,使{\PN{亚伦}}后裔之外的人不得近前来在耶和华面前烧香,免得他遭{\PN{可拉}}和他一党所遭的。这乃是照耶和华借着{\PN{摩西}}所吩咐的。
\par }{\SH 亚伦救了人民
\par }{\PP \VS{41}第二天,{\PN{以色列}}全会众都向{\PN{摩西}}、{\PN{亚伦}}发怨言说:「你们杀了耶和华的百姓了。」
\VS{42}会众聚集攻击{\PN{摩西}}、{\PN{亚伦}}的时候,向会幕观看,不料,有云彩遮盖了,耶和华的荣光显现。
\VS{43}{\PN{摩西}}、{\PN{亚伦}}就来到会幕前。
\VS{44}耶和华吩咐{\PN{摩西}}说:
\VS{45}「你们离开这会众,我好在转眼之间把他们灭绝。」他们二人就俯伏于地。
\VS{46}{\PN{摩西}}对{\PN{亚伦}}说:「拿你的香炉,把坛上的火盛在其中,又加上香,快快带到会众那里,为他们赎罪;因为有忿怒从耶和华那里出来,瘟疫已经发作了。」
\VS{47}{\PN{亚伦}}照着{\PN{摩西}}所说的拿来,跑到会中,不料,瘟疫在百姓中已经发作了。他就加上香,为百姓赎罪。
\VS{48}他站在活人死人中间,瘟疫就止住了。
\VS{49}除了因{\PN{可拉}}事情死的以外,遭瘟疫死的,共有一万四千七百人。
\VS{50}{\PN{亚伦}}回到会幕门口,到{\PN{摩西}}那里,瘟疫已经止住了。

\par }\Chap{17}{\SH 亚伦的杖
\par }{\PP \VerseOne{1}耶和华对{\PN{摩西}}说:
\VS{2}「你晓谕{\PN{以色列}}人,从他们手下取杖,每支派一根;从他们所有的首领,按着支派,共取十二根。你要将各人的名字写在各人的杖上,
\VS{3}并要将{\PN{亚伦}}的名字写在{\PN{利未}}的杖上,因为各族长必有一根杖。
\VS{4}你要把这些杖存在会幕内法{\ADD{柜}}前,就是我与你们相会之处。
\VS{5}后来我所拣选的那人,他的杖必发芽。这样,我必使{\PN{以色列}}人向你们所发的怨言止息,不再达到我耳中。」
\VS{6}于是{\PN{摩西}}晓谕{\PN{以色列}}人,他们的首领就把杖交给他,按着支派,每首领一根,共有十二根;{\PN{亚伦}}的杖也在其中。
\VS{7}{\PN{摩西}}就把杖存在法{\ADD{柜}}的帐幕内,在耶和华面前。
\par }{\PP \VS{8}第二天,{\PN{摩西}}进法{\ADD{柜}}的帐幕去。谁知{\PN{利未}}族{\PN{亚伦}}的杖已经发了芽,生了花苞,开了花,结了熟杏。
\VS{9}{\PN{摩西}}就把所有的杖从耶和华面前拿出来,给{\PN{以色列}}众人看;他们看见了,各首领就把自己的杖拿去。
\VS{10}耶和华吩咐{\PN{摩西}}说:「把{\PN{亚伦}}的杖还放在法{\ADD{柜}}前,给这些背叛之子留作记号。这样,你就使他们向我发的怨言止息,免得他们死亡。」
\VS{11}{\PN{摩西}}就这样行。耶和华怎样吩咐他,他就怎样行了。
\par }{\PP \VS{12}{\PN{以色列}}人对{\PN{摩西}}说:「我们死啦!我们灭亡啦!都灭亡啦!
\VS{13}凡挨近耶和华帐幕的是必死的。我们都要死亡吗?」

\par }\Chap{18}{\SH 祭司和利未人的职责
\par }{\PP \VerseOne{1}耶和华对{\PN{亚伦}}说:「你和你的儿子,并你本族的人,要一同担当干犯圣所的罪孽。你和你的儿子也要一同担当干犯祭司职任的罪孽。
\VS{2}你要带你弟兄{\PN{利未}}人,就是你祖宗支派的人前来,使他们与你联合,服事你,只是你和你的儿子,要一同在法{\ADD{柜}}的帐幕前{\ADD{供职}}。
\VS{3}他们要守所吩咐你的,并守全帐幕,只是不可挨近圣所的器具和坛,免得他们和你们都死亡。
\VS{4}他们要与你联合,也要看守会幕,办理帐幕一切的事,只是外人不可挨近你们。
\VS{5}你们要看守圣所和坛,免得忿怒再临到{\PN{以色列}}人。
\VS{6}我已将你们的弟兄{\PN{利未}}人从{\PN{以色列}}人中拣选出来归耶和华,是给你们为赏赐的,为要办理会幕的事。
\VS{7}你和你的儿子要为一切属坛和幔子内的事一同守祭司的职任。你们要这样供职;我将祭司的职任给你们当作赏赐事奉我。凡挨近的外人必被治死。」
\par }{\SH 祭司当得之物
\par }{\PP \VS{8}耶和华晓谕{\PN{亚伦}}说:「我已将归我的举祭,就是{\PN{以色列}}人一切分别为圣的物,交给你经管;因你受过膏,把这些都赐给你和你的子孙,当作永得的分。
\VS{9}{\PN{以色列}}人归给我至圣的供物,就是一切的素祭、赎罪祭、赎愆祭,其中所有{\ADD{存留}}不经火的,都为至圣之物,要归给你和你的子孙。
\VS{10}你要拿这些当至圣物吃;凡男丁都可以吃。你当以此物为圣。
\VS{11}{\PN{以色列}}人所献的举祭并摇祭都是你的;我已赐给你和你的儿女,当作永得的分;凡在你家中的洁净人都可以吃。
\VS{12}凡油中、新酒中、五谷中至好的,就是{\PN{以色列}}人所献给耶和华初熟之物,我都赐给你。
\VS{13}凡从他们地上所带来给耶和华初熟之物也都要归与你。你家中的洁净人都可以吃。
\VS{14}{\PN{以色列}}中一切永献的都必归与你。
\VS{15}他们所有奉给耶和华的,连人带牲畜,凡头生的,都要归给你;只是人头生的,总要赎出来;不洁净牲畜头生的,也要赎出来。
\VS{16}其中在一月之外所当赎的,要照你所估定的价,按圣所的平,用银子五舍客勒赎出来(一舍客勒是二十季拉)。
\VS{17}只是头生的牛,或是头生的绵羊和山羊,必不可赎,都是圣的,要把它的血洒在坛上,把它的脂油焚烧,当作馨香的火祭献给耶和华。
\VS{18}它的肉必归你,像被摇的胸、{\ADD{被举的}}右腿归你一样。
\VS{19}凡{\PN{以色列}}人所献给耶和华圣物中的举祭,我都赐给你和你的儿女,当作永得的分。这是给你和你的后裔、在耶和华面前作为永远的盐约\FTNT{}{{\FR 18:19: }盐就是不废坏的意思}。」
\VS{20}耶和华对{\PN{亚伦}}说:「你在{\PN{以色列}}人的境内不可有产业,在他们中间也不可有分。我就是你的分,是你的产业。」
\par }{\SH 利未人的分
\par }{\PP \VS{21}「凡{\PN{以色列}}中出产的十分之一,我已赐给{\PN{利未}}的子孙为业;因他们所办的是会幕的事,所以赐给他们为酬他们的劳。
\VS{22}从今以后,{\PN{以色列}}人不可挨近会幕,免得他们担罪而死。
\VS{23}惟独{\PN{利未}}人要办会幕的事,担当罪孽;这要作你们世世代代永远的定例。他们在{\PN{以色列}}人中不可有产业;
\VS{24}因为{\PN{以色列}}人中出产的十分之一,就是献给耶和华为举祭的,我已赐给{\PN{利未}}人为业。所以我对他们说:『在{\PN{以色列}}人中不可有产业。』」
\par }{\SH 利未人的十分之一奉献
\par }{\PP \VS{25}耶和华吩咐{\PN{摩西}}说:
\VS{26}「你晓谕{\PN{利未}}人说:你们从{\PN{以色列}}人中所取的十分之一,就是我给你们为业的,要再从那十分之一中取十分之一作为举祭献给耶和华,
\VS{27}这举祭要算为你们场上的谷,又如满酒榨的酒。
\VS{28}这样,你们从{\PN{以色列}}人中所得的十分之一也要作举祭献给耶和华,从这十分之一中,将所献给耶和华的举祭归给祭司{\PN{亚伦}}。
\VS{29}奉给你们的一切礼物,要从其中将至好的,就是分别为圣的,献给耶和华为举祭。
\VS{30}所以你要对{\PN{利未}}人说:你们从其中将至好的举起,这就算为你们场上的粮,又如酒榨的酒。
\VS{31}你们和你们家属随处可以吃;这原是你们的赏赐,是酬你们在会幕里办事的劳。
\VS{32}你们从其中将至好的举起,就不至因这物担罪。你们不可亵渎{\PN{以色列}}人的圣物,免得死亡。」

\par }\Chap{19}{\SH 红母牛的灰
\par }{\PP \VerseOne{1}耶和华晓谕{\PN{摩西}}、{\PN{亚伦}}说:
\VS{2}「耶和华命定律法中的一条律例乃是这样说:你要吩咐{\PN{以色列}}人,把一只没有残疾、未曾负轭、纯红的母牛牵到你这里来,
\VS{3}交给祭司{\PN{以利亚撒}};他必牵到营外,人就把牛宰在他面前。
\VS{4}祭司{\PN{以利亚撒}}要用指头蘸这牛的血,向会幕前面弹七次。
\VS{5}人要在他眼前把这母牛焚烧;牛的皮、肉、血、粪都要焚烧。
\VS{6}祭司要把香柏木、牛膝草、朱红色{\ADD{线}}都丢在烧牛的火中。
\VS{7}祭司必不洁净到晚上,要洗衣服,用水洗身,然后可以进营。
\VS{8}烧牛的人必不洁净到晚上,也要洗衣服,用水洗身。
\VS{9}必有一个洁净的人收起母牛的灰,存在营外洁净的地方,为{\PN{以色列}}会众调做除污秽的水。这本是除罪的。
\VS{10}收起母牛灰的人必不洁净到晚上,要洗衣服。这要给{\PN{以色列}}人和寄居在他们中间的外人作为永远的定例。」
\par }{\SH 接触尸体不洁的条例
\par }{\PP \VS{11}「摸了人死尸的,就必七天不洁净。
\VS{12}那人到第三天要用这除污秽的水洁净自己,第七天就洁净了。他若在第三天不洁净自己,第七天就不洁净了。
\VS{13}凡摸了人死尸、不洁净自己的,就玷污了耶和华的帐幕,这人必从{\PN{以色列}}中剪除;因为那除污秽的水没有洒在他身上,他就为不洁净,污秽还在他身上。
\par }{\PP \VS{14}「人死在帐棚里的条例乃是这样:凡进那帐棚的,和一切在帐棚里的,都必七天不洁净。
\VS{15}凡敞口的器皿,就是没有扎上盖的,也是不洁净。
\VS{16}无论何人在田野里摸了被刀杀的,或是尸首,或是人的骨头,或是坟墓,就要七天不洁净。
\VS{17}要为这不洁净的人拿些烧成的除罪灰放在器皿里,倒上活水。
\VS{18}必当有一个洁净的人拿牛膝草蘸在这水中,把水洒在帐棚上,和一切器皿并帐棚内的众人身上,又洒在摸了骨头,或摸了被杀的,或摸了自死的,或摸了坟墓的那人身上。
\VS{19}第三天和第七天,洁净的人要洒水在不洁净的人身上,第七天就使他成为洁净。那人要洗衣服,用水洗澡,到晚上就洁净了。
\par }{\PP \VS{20}「但那污秽而不洁净自己的,要将他从会中剪除,因为他玷污了耶和华的圣所。除污秽的水没有洒在他身上,他是不洁净的。
\VS{21}这要给你们作为永远的定例。并且那洒除污秽水的人要洗衣服。凡摸除污秽水的,必不洁净到晚上。
\VS{22}不洁净人所摸的一切物就不洁净;摸了这物的人必不洁净到晚上。」

\par }\Chap{20}{\SH 在加低斯发生的事件
\par }{\R (出17·1—7)
\par }{\PP \VerseOne{1}正月间,{\PN{以色列}}全会众到了{\PN{寻}}的旷野,就住在{\PN{加低斯}}。{\PN{米利暗}}死在那里,就葬在那里。
\par }{\PP \VS{2}会众没有水喝,就聚集攻击{\PN{摩西}}、{\PN{亚伦}}。
\VS{3}百姓向{\PN{摩西}}争闹说:「我们的弟兄曾死在耶和华面前,我们恨不得与他们同死。
\VS{4}你们为何把耶和华的会众领到这旷野、使我们和牲畜都死在这里呢?
\VS{5}你们为何逼着我们出{\PN{埃及}}、领我们到这坏地方呢?这地方不好撒种,也没有无花果树、葡萄树、石榴树,又没有水喝。」
\VS{6}{\PN{摩西}}、{\PN{亚伦}}离开会众,到会幕门口,俯伏在地;耶和华的荣光向他们显现。
\VS{7}耶和华晓谕{\PN{摩西}}说:
\VS{8}「你拿着杖去,和你的哥哥{\PN{亚伦}}招聚会众,在他们眼前吩咐磐石发出水来,水就从磐石流出,给会众和他们的牲畜喝。」
\VS{9}于是{\PN{摩西}}照耶和华所吩咐的,从耶和华面前取了杖去。
\par }{\PP \VS{10}{\PN{摩西}}、{\PN{亚伦}}就招聚会众到磐石前。{\PN{摩西}}说:「你们这些背叛的人听我说:我为你们使水从这磐石中流出来吗?」
\VS{11}{\PN{摩西}}举手,用杖击打磐石两下,就有许多水流出来,会众和他们的牲畜都喝了。
\VS{12}耶和华对{\PN{摩西}}、{\PN{亚伦}}说:「因为你们不信我,不在{\PN{以色列}}人眼前尊我为圣,所以你们必不得领这会众进我所赐给他们的地去。」
\VS{13}这水名叫{\PN{米利巴}}水\FTNT{}{{\FR 20:13: }米利巴就是争闹的意思},是因{\PN{以色列}}人向耶和华争闹,耶和华就在他们面前显为圣。
\par }{\SH 以东王不容以色列人通行
\par }{\PP \VS{14}{\PN{摩西}}从{\PN{加低斯}}差遣使者去见{\PN{以东}}王,说:「你的弟兄{\PN{以色列}}人这样说:『我们所遭遇的一切艰难,
\VS{15}就是我们的列祖下到{\PN{埃及}},我们在{\PN{埃及}}久住;{\PN{埃及}}人恶待我们的列祖和我们,
\VS{16}我们哀求耶和华的时候,他听了我们的声音,差遣使者把我们从{\PN{埃及}}领出来。这事你都知道。如今,我们在你边界上的城{\PN{加低斯}}。
\VS{17}求你容我们从你的地经过。我们不走田间和葡萄园,也不喝井里的水,只走大道\FTNT{}{{\FR 20:17: }原文是王道},不偏左右,直到过了你的境界。』」
\VS{18}{\PN{以东}}{\ADD{王}}说:「你不可从我的{\ADD{地}}经过,免得我带刀出去攻击你。」
\VS{19}{\PN{以色列}}人说:「我们要走大道上去;我们和牲畜若喝你的水,必给你价值。不求别的,只求你容我们步行过去。」
\VS{20}{\PN{以东}}{\ADD{王}}说:「你们不可经过!」就率领许多人出来,要用强硬的手攻击{\PN{以色列}}人。
\VS{21}这样,{\PN{以东}}{\ADD{王}}不肯容{\PN{以色列}}人从他的境界过去。于是他们转去,离开他。
\par }{\SH 亚伦逝世
\par }{\PP \VS{22}{\PN{以色列}}全会众从{\PN{加低斯}}起行,到了{\PN{何珥山}}。
\VS{23}耶和华在附近{\PN{以东}}边界的{\PN{何珥山}}上晓谕{\PN{摩西}}、{\PN{亚伦}}说:
\VS{24}「{\PN{亚伦}}要归到他列祖\FTNT{}{{\FR 20:24: }原文是本民}那里。他必不得入我所赐给{\PN{以色列}}人的地;因为在{\PN{米利巴}}水,你们违背了我的命。
\VS{25}你带{\PN{亚伦}}和他的儿子{\PN{以利亚撒}}上{\PN{何珥山}},
\VS{26}把{\PN{亚伦}}的{\ADD{圣}}衣脱下来,给他的儿子{\PN{以利亚撒}}穿上;{\PN{亚伦}}必死在那里,归{\ADD{他列祖}}。」
\VS{27}{\PN{摩西}}就照耶和华所吩咐的行。三人当着会众的眼前上了{\PN{何珥山}}。
\VS{28}{\PN{摩西}}把{\PN{亚伦}}的{\ADD{圣}}衣脱下来,给他的儿子{\PN{以利亚撒}}穿上,{\PN{亚伦}}就死在山顶那里。于是{\PN{摩西}}和{\PN{以利亚撒}}下了山。
\VS{29}全会众,就是{\PN{以色列}}全家,见{\PN{亚伦}}已经死了,便都为{\PN{亚伦}}哀哭了三十天。

\par }\Chap{21}{\SH 战胜迦南人
\par }{\PP \VerseOne{1}住南地的{\PN{迦南}}人{\PN{亚拉得}}王,听说{\PN{以色列}}人从{\PN{亚他林}}路来,就和{\PN{以色列}}人争战,掳了他们几个人。
\VS{2}{\PN{以色列}}人向耶和华发愿说:「你若将这民交付我手,我就把他们的城邑尽行毁灭。」
\VS{3}耶和华应允了{\PN{以色列}}人,把{\PN{迦南}}人交付他们,他们就把{\PN{迦南}}人和{\PN{迦南}}人的城邑尽行毁灭。那地方的名便叫{\PN{何珥玛}}\FTNT{}{{\FR 21:3: }就是毁灭的意思}。
\par }{\SH 铜制的蛇
\par }{\PP \VS{4}他们从{\PN{何珥山}}起行,往{\PN{红海}}那条路走,要绕过{\PN{以东}}地。百姓因这路难行,心中甚是烦躁,
\VS{5}就怨  神和{\PN{摩西}}说:「你们为什么把我们从{\PN{埃及}}领出来、使我们死在旷野呢?这里没有粮,没有水,我们的心厌恶这淡薄的食物。」
\VS{6}于是耶和华使火蛇进入百姓中间,蛇就咬他们。{\PN{以色列}}人中死了许多。
\VS{7}百姓到{\PN{摩西}}那里,说:「我们怨 耶和华和你,有罪了。求你祷告耶和华,叫这些蛇离开我们。」于是{\PN{摩西}}为百姓祷告。
\VS{8}耶和华对{\PN{摩西}}说:「你制造一条火蛇,挂在杆子上;凡被咬的,一望这蛇,就必得活。」
\VS{9}{\PN{摩西}}便制造一条铜蛇,挂在杆子上;凡被蛇咬的,一望这铜蛇就活了。
\par }{\SH 从何珥山到摩押谷
\par }{\PP \VS{10}{\PN{以色列}}人起行,安营在{\PN{阿伯}}。
\VS{11}又从{\PN{阿伯}}起行,安营在{\PN{以耶·亚巴琳}},与{\PN{摩押}}相对的旷野,向日出之地。
\VS{12}从那里起行,安营在{\PN{撒烈谷}}。
\VS{13}从那里起行,安营在{\PN{亚嫩河}}那边。这{\PN{亚嫩河}}是在旷野,从{\PN{亚摩利}}的境界流出来的;原来{\PN{亚嫩河}}是{\PN{摩押}}的边界,在{\PN{摩押}}和{\PN{亚摩利}}人搭界的地方。
\VS{14}所以耶和华的战记上说:「{\PN{苏法}}的{\PN{哇哈伯}}与{\PN{亚嫩河}}的谷,
\VS{15}并向{\PN{亚珥}}城众谷的下坡,是靠近{\PN{摩押}}的境界。」
\par }{\PP \VS{16}{\PN{以色列}}人从那里{\ADD{起行}},到了{\PN{比珥}}\FTNT{}{{\FR 21:16: }就是井的意思}。从前耶和华吩咐{\PN{摩西}}说:「招聚百姓,我好给他们水喝」,说的就是这井。
\VS{17}当时,{\PN{以色列}}人唱歌说:
\par }{\Q 井啊,涌上水来!
\par }{\Q 你们要向这井歌唱。
\par }{\Q \VS{18}这井是首领和民中的尊贵人
\par }{\Q 用圭用杖所挖所掘的。
\par }{\MM {\PN{以色列}}人从旷野往{\PN{玛他拿}}去,
\VS{19}从{\PN{玛他拿}}到{\PN{拿哈列}},从{\PN{拿哈列}}到{\PN{巴末}},
\VS{20}从{\PN{巴末}}到{\PN{摩押}}地的谷,又到那下望旷野之{\PN{毗斯迦}}的山顶。
\par }{\SH 战胜亚摩利王和巴珊王
\par }{\R (申2·26—3·11)
\par }{\PP \VS{21}{\PN{以色列}}人差遣使者去见{\PN{亚摩利}}人的王{\PN{西宏}},说:
\VS{22}「求你容我们从你的地经过;我们不偏入田间和葡萄园,也不喝井里的水,只走大道\FTNT{}{{\FR 21:22: }原文是王道},直到过了你的境界。」
\VS{23}{\PN{西宏}}不容{\PN{以色列}}人从他的境界经过,就招聚他的众民出到旷野,要攻击{\PN{以色列}}人,到了{\PN{雅杂}}与{\PN{以色列}}人争战。
\VS{24}{\PN{以色列}}人用刀杀了他,得了他的地,从{\PN{亚嫩河}}到{\PN{雅博河}},直到{\PN{亚扪}}人的{\ADD{境界}},因为{\PN{亚扪}}人的境界多有坚垒。
\VS{25}{\PN{以色列}}人夺取这一切的城邑,也住{\PN{亚摩利}}人的城邑,就是{\PN{希实本}}与{\PN{希实本}}的一切乡村。
\VS{26}这{\PN{希实本}}是{\PN{亚摩利}}王{\PN{西宏}}的{\ADD{京}}城;{\PN{西宏}}曾与{\PN{摩押}}的先王争战,从他手中夺取了全地,直到{\PN{亚嫩河}}。
\VS{27}所以那些作诗歌的说:
\par }{\Q 你们来到{\PN{希实本}};
\par }{\Q 愿{\PN{西宏}}的城被修造,被建立。
\par }{\Q \VS{28}因为有火从{\PN{希实本}}发出,
\par }{\Q 有火焰出于{\PN{西宏}}的城,
\par }{\Q 烧尽{\PN{摩押}}的{\PN{亚珥}}和{\PN{亚嫩河}}邱坛的祭司
\FTNT{}{{\FR 21:28: }祭司原文是主}。
\par }{\Q \VS{29}{\PN{摩押}}啊,你有祸了!
\par }{\Q {\PN{基抹}}的民哪,你们灭亡了!
\par }{\Q {\PN{基抹}}的男子逃奔,
\par }{\Q 女子被掳,交付{\PN{亚摩利}}的王{\PN{西宏}}。
\par }{\Q \VS{30}我们射了他们;
\par }{\Q {\PN{希实本}}直到{\PN{底本}}尽皆毁灭。
\par }{\Q 我们使地变成荒场,直到{\PN{挪法}};
\par }{\Q 这{\PN{挪法}}直{\ADD{延}}到{\PN{米底巴}}。
\par }{\PP \VS{31}这样,{\PN{以色列}}人就住在{\PN{亚摩利}}人之地。
\VS{32}{\PN{摩西}}打发人去窥探{\PN{雅谢}},{\PN{以色列}}人就占了{\PN{雅谢}}的镇市,赶出那里的{\PN{亚摩利}}人。
\VS{33}{\PN{以色列}}人转回,向{\PN{巴珊}}去。{\PN{巴珊}}王{\PN{噩}}和他的众民都出来,在{\PN{以得来}}与他们交战。
\VS{34}耶和华对{\PN{摩西}}说:「不要怕他!因我已将他和他的众民,并他的地,都交在你手中;你要待他像从前待住{\PN{希实本}}的{\PN{亚摩利}}王{\PN{西宏}}一般。」
\VS{35}于是他们杀了他和他的众子,并他的众民,没有留下一个,就得了他的地。

\par }\Chap{22}{\SH 摩押王召巴兰
\par }{\PP \VerseOne{1}{\PN{以色列}}人起行,在{\PN{摩押}}平原、{\PN{约旦河}}东,对着{\PN{耶利哥}}安营。
\par }{\PP \VS{2}{\PN{以色列}}人向{\PN{亚摩利}}人所行的一切事,{\PN{西拨}}的儿子{\PN{巴勒}}都看见了。
\VS{3}{\PN{摩押}}人因{\PN{以色列}}民甚多,就大大惧怕,心内忧急,
\VS{4}对{\PN{米甸}}的长老说:「现在这众人要把我们四围所有的一概舔尽,就如牛舔尽田间的草一般。」那时{\PN{西拨}}的儿子{\PN{巴勒}}作{\PN{摩押}}王。
\VS{5}他差遣使者往大河边的{\PN{毗夺}}去,到{\PN{比珥}}的儿子{\PN{巴兰}}本乡那里,召{\PN{巴兰}}来,说:「有一宗民从{\PN{埃及}}出来,遮满地面,与我对居。
\VS{6}这民比我强盛,现在求你来为我咒诅他们,或者我能得胜,攻打他们,赶出此地。因为我知道,你为谁祝福,谁就得福;你咒诅谁,谁就受咒诅。」
\par }{\PP \VS{7}{\PN{摩押}}的长老和{\PN{米甸}}的长老手里拿着卦金,到了{\PN{巴兰}}那里,将{\PN{巴勒}}的话都告诉了他。
\VS{8}{\PN{巴兰}}说:「你们今夜在这里住宿,我必照耶和华所晓谕我的回报你们。」{\PN{摩押}}的使臣就在{\PN{巴兰}}那里住下了。
\VS{9}神临到{\PN{巴兰}}那里,说:「在你这里的人都是谁?」
\VS{10}{\PN{巴兰}}回答说:「是{\PN{摩押}}王{\PN{西拨}}的儿子{\PN{巴勒}}打发人到我这里来,{\ADD{说}}:
\VS{11}『从{\PN{埃及}}出来的民遮满地面,你来为我咒诅他们,或者我能与他们争战,把他们赶出去。』」
\VS{12}神对{\PN{巴兰}}说:「你不可同他们去,也不可咒诅那民,因为那民是蒙福的。」
\VS{13}{\PN{巴兰}}早晨起来,对{\PN{巴勒}}的使臣说:「你们回本地去吧,因为耶和华不容我和你们同去。」
\VS{14}{\PN{摩押}}的使臣就起来,回{\PN{巴勒}}那里去,说:「{\PN{巴兰}}不肯和我们同来。」
\par }{\PP \VS{15}{\PN{巴勒}}又差遣使臣,比先前的又多又尊贵。
\VS{16}他们到了{\PN{巴兰}}那里,对他说:「{\PN{西拨}}的儿子{\PN{巴勒}}这样说:『求你不容什么事拦阻你不到我这里来,
\VS{17}因为我必使你得极大的尊荣。你向我要什么,我就给你什么;只求你来为我咒诅这民。』」
\VS{18}{\PN{巴兰}}回答{\PN{巴勒}}的臣仆说:「{\PN{巴勒}}就是将他满屋的金银给我,我行大事小事也不得越过耶和华—我 神的命。
\VS{19}现在我请你们今夜在这里住宿,等我得知耶和华还要对我说什么。」
\VS{20}当夜, 神临到{\PN{巴兰}}那里,说:「这些人若来召你,你就起来同他们去,你只要遵行我对你所说的话。」
\par }{\SH 巴兰和他的驴
\par }{\PP \VS{21}{\PN{巴兰}}早晨起来,备上驴,和{\PN{摩押}}的使臣一同去了。
\VS{22}神因他去就发了怒;耶和华的使者站在路上敌挡他。他骑着驴,有两个仆人跟随他。
\VS{23}驴看见耶和华的使者站在路上,手里有拔出来的刀,就从路上跨进田间,{\PN{巴兰}}便打驴,要叫它回转上路。
\VS{24}耶和华的使者就站在葡萄园的窄路上;这边有墙,那边也有墙。
\VS{25}驴看见耶和华的使者,就贴靠墙,将{\PN{巴兰}}的脚挤伤了;{\PN{巴兰}}又打驴。
\VS{26}耶和华的使者又往前去,站在狭窄之处,左右都没有转折的地方。
\VS{27}驴看见耶和华的使者,就卧在{\PN{巴兰}}底下,{\PN{巴兰}}发怒,用杖打驴。
\VS{28}耶和华叫驴开口,对{\PN{巴兰}}说:「我向你行了什么,你竟打我这三次呢?」
\VS{29}{\PN{巴兰}}对驴说:「因为你戏弄我,我恨不能手中有刀,把你杀了。」
\VS{30}驴对{\PN{巴兰}}说:「我不是你从小时直到今日所骑的驴吗?我素常向你这样行过吗?」{\PN{巴兰}}说:「没有。」
\par }{\PP \VS{31}当时,耶和华使{\PN{巴兰}}的眼目明亮,他就看见耶和华的使者站在路上,手里有拔出来的刀,{\PN{巴兰}}便低头俯伏在地。
\VS{32}耶和华的使者对他说:「你为何这三次打你的驴呢?我出来敌挡你,因你所行的,在我面前偏僻。
\VS{33}驴看见我就三次从我面前偏过去;驴若没有偏过去,我早把你杀了,留它存活。」
\VS{34}{\PN{巴兰}}对耶和华的使者说:「我有罪了。我不知道你站在路上阻挡我;你若不喜欢我去,我就转回。」
\VS{35}耶和华的使者对{\PN{巴兰}}说:「你同这些人去吧!你只要说我对你说的话。」于是{\PN{巴兰}}同着{\PN{巴勒}}的使臣去了。
\par }{\SH 巴勒欢迎巴兰
\par }{\PP \VS{36}{\PN{巴勒}}听见{\PN{巴兰}}来了,就往{\PN{摩押}}{\ADD{京}}城去迎接他;这城是在边界上,在{\PN{亚嫩河}}旁。
\VS{37}{\PN{巴勒}}对{\PN{巴兰}}说:「我不是急急地打发人到你那里去召你吗?你为何不到我这里来呢?我岂不能使你得尊荣吗?」
\VS{38}{\PN{巴兰}}说:「我已经到你这里来了!现在我岂能擅自说什么呢? 神将什么话传给我,我就说什么。」
\VS{39}{\PN{巴兰}}和{\PN{巴勒}}同行,来到{\PN{基列·胡琐}}。
\VS{40}{\PN{巴勒}}宰了\FTNT{}{{\FR 22:40: }原文是献}牛羊,送给{\PN{巴兰}}和陪伴的使臣。
\par }{\SH 巴兰首次预言
\par }{\PP \VS{41}到了早晨,{\PN{巴勒}}领{\PN{巴兰}}到{\PN{巴力}}的高处;{\PN{巴兰}}从那里观看{\PN{以色列}}{\ADD{营}}的边界。

\par }\Chap{23}{\PP \VerseOne{1}{\PN{巴兰}}对{\PN{巴勒}}说:「你在这里给我筑七座坛,为我预备七只公牛,七只公羊。」
\VS{2}{\PN{巴勒}}照{\PN{巴兰}}的话行了。{\PN{巴勒}}和{\PN{巴兰}}在每座坛上献一只公牛,一只公羊。
\VS{3}{\PN{巴兰}}对{\PN{巴勒}}说:「你站在你的燔祭旁边,我且往前去,或者耶和华来迎见我。他指示我什么,我必告诉你。」于是{\PN{巴兰}}上一净光的高处。
\VS{4}神迎见{\PN{巴兰}};{\PN{巴兰}}说:「我预备了七座坛,在每座坛上献了一只公牛,一只公羊。」
\VS{5}耶和华将话传给{\PN{巴兰}},又说:「你回到{\PN{巴勒}}那里,要如此如此说。」
\VS{6}他就回到{\PN{巴勒}}那里,见他同{\PN{摩押}}的使臣都站在燔祭旁边。
\VS{7}{\PN{巴兰}}便题起诗歌说:
\par }{\Q {\PN{巴勒}}引我出{\PN{亚兰}},
\par }{\Q {\PN{摩押}}王引我出东山,{\ADD{说}}:
\par }{\Q 来啊,为我咒诅{\PN{雅各}};
\par }{\Q 来啊,怒骂{\PN{以色列}}。
\par }{\Q \VS{8}神没有咒诅的,我焉能咒诅?
\par }{\Q 耶和华没有怒骂的,我焉能怒骂?
\par }{\Q \VS{9}我从高峰看他,从小山望他;
\par }{\Q 这是独居的民,不列在万民中。
\par }{\Q \VS{10}谁能数点{\PN{雅各}}的尘土?
\par }{\Q 谁能计算{\PN{以色列}}的四分之一?
\par }{\Q 我愿如义人之死而死;
\par }{\Q 我愿如义人之终而终。
\par }{\PP \VS{11}{\PN{巴勒}}对{\PN{巴兰}}说:「你向我做的是什么事呢?我领你来咒诅我的仇敌,不料,你竟为他们祝福。」
\VS{12}他回答说:「耶和华传给我的话,我能不谨慎传说吗?」
\par }{\SH 巴兰第二次预言
\par }{\PP \VS{13}{\PN{巴勒}}说:「求你同我往别处去,在那里可以看见他们;你不能全看见,只能看见他们边界上的人。在那里要为我咒诅他们。」
\VS{14}于是领{\PN{巴兰}}到了{\PN{琐腓田}},上了{\PN{毗斯迦山}}顶,筑了七座坛;每座坛上献一只公牛,一只公羊。
\VS{15}{\PN{巴兰}}对{\PN{巴勒}}说:「你站在这燔祭旁边,等我往那边去迎见{\ADD{耶和华}}。」
\VS{16}耶和华临到{\PN{巴兰}}那里,将话传给他;又说:「你回到{\PN{巴勒}}那里,要如此如此说。」
\VS{17}他就回到{\PN{巴勒}}那里,见他站在燔祭旁边;{\PN{摩押}}的使臣也和他在一处。{\PN{巴勒}}问他说:「耶和华说了什么话呢?」
\VS{18}{\PN{巴兰}}就题诗歌说:
\par }{\Q {\PN{巴勒}},你起来听;
\par }{\Q {\PN{西拨}}的儿子,你听我言。
\par }{\Q \VS{19}神非人,必不致说谎,
\par }{\Q 也非人子,必不致后悔。
\par }{\Q 他说话岂不照着行呢?
\par }{\Q 他发言岂不要成就呢?
\par }{\Q \VS{20}我奉{\ADD{命}}祝福;
\par }{\Q  神也曾赐福,此事我不能翻转。
\par }{\Q \VS{21}他未见{\PN{雅各}}中有罪孽,
\par }{\Q 也未见{\PN{以色列}}中有奸恶。
\par }{\Q 耶和华—他的 神和他同在;
\par }{\Q 有欢呼王的声音在他们中间。
\par }{\Q \VS{22}神领他们出{\PN{埃及}};
\par }{\Q 他们似乎有野牛之力。
\par }{\Q \VS{23}断没有法术可以害{\PN{雅各}},
\par }{\Q 也没有占卜可以害{\PN{以色列}}。
\par }{\Q 现在必有人论及{\PN{雅各}},就是论及{\PN{以色列}}说:
\par }{\Q  神为他行了何等的大事!
\par }{\Q \VS{24}这民起来,仿佛母狮,
\par }{\Q 挺身,好像公狮,
\par }{\Q 未曾吃野食,
\par }{\Q 未曾喝被伤者之血,决不躺卧。
\par }{\PP \VS{25}{\PN{巴勒}}对{\PN{巴兰}}说:「你一点不要咒诅他们,也不要为他们祝福。」
\VS{26}{\PN{巴兰}}回答{\PN{巴勒}}说:「我岂不是告诉你说『凡耶和华所说的,我必须遵行』吗?」
\par }{\SH 巴兰第三次预言
\par }{\PP \VS{27}{\PN{巴勒}}对{\PN{巴兰}}说:「来吧,我领你往别处去,或者 神喜欢你在那里为我咒诅他们。」
\VS{28}{\PN{巴勒}}就领{\PN{巴兰}}到那下望旷野的{\PN{毗珥}}{\ADD{
{\PN{山}}}}顶上。
\VS{29}{\PN{巴兰}}对{\PN{巴勒}}说:「你在这里为我筑七座坛,又在这里为我预备七只公牛,七只公羊。」
\VS{30}{\PN{巴勒}}就照{\PN{巴兰}}的话行,在每座坛上献一只公牛,一只公羊。

\par }\Chap{24}{\PP \VerseOne{1}{\PN{巴兰}}见耶和华喜欢赐福与{\PN{以色列}},就不像前两次去求法术,却面向旷野。
\VS{2}{\PN{巴兰}}举目,看见{\PN{以色列}}人照着支派居住。 神的灵就临到他身上,
\VS{3}他便题起诗歌说:
\par }{\Q {\PN{比珥}}的儿子{\PN{巴兰}}说,
\par }{\Q 眼目闭住\FTNT{}{{\FR 24:3: }或译:睁开}的人说,
\par }{\Q \VS{4}得听 神的言语,
\par }{\Q 得见全能者的异象,
\par }{\Q 眼目睁开而仆倒的人说:
\par }{\Q \VS{5}{\PN{雅各}}啊,你的帐棚何等华美!
\par }{\Q {\PN{以色列}}啊,你的帐幕何其华丽!
\par }{\Q \VS{6}如接连的山谷,
\par }{\Q 如河旁的园子,
\par }{\Q 如耶和华所栽的沉香树,
\par }{\Q 如水边的香柏木。
\par }{\Q \VS{7}水要从他的桶里流出;
\par }{\Q 种子要撒在多水之处。
\par }{\Q 他的王必超过{\PN{亚甲}};
\par }{\Q 他的国必要振兴。
\par }{\Q \VS{8}神领他出{\PN{埃及}};
\par }{\Q 他似乎有野牛之力。
\par }{\Q 他要吞吃敌国,
\par }{\Q 折断他们的骨头,
\par }{\Q 用箭射透他们。
\par }{\Q \VS{9}他蹲如公狮,
\par }{\Q 卧如母狮,谁敢惹他?
\par }{\Q 凡给你祝福的,愿他蒙福;
\par }{\Q 凡咒诅你的,愿他受咒诅。
\par }{\PP \VS{10}{\PN{巴勒}}向{\PN{巴兰}}生气,就拍起手来,对{\PN{巴兰}}说:「我召你来为我咒诅仇敌,不料,你这三次竟为他们祝福。
\VS{11}如今你快回本地去吧!我想使你得大尊荣,耶和华却阻止你不得尊荣。」
\VS{12}{\PN{巴兰}}对{\PN{巴勒}}说:「我岂不是对你所差遣到我那里的使者说:
\VS{13}『{\PN{巴勒}}就是将他满屋的金银给我,我也不得越过耶和华的命,凭自己的心意行好行歹。耶和华说什么,我就要说什么』?」
\par }{\SH 巴兰末次预言
\par }{\PP \VS{14}「现在我要回本族去。你来,我告诉你这民日后要怎样待你的民。」
\VS{15}他就题起诗歌说:
\par }{\Q {\PN{比珥}}的儿子{\PN{巴兰}}说:
\par }{\Q 眼目闭住\FTNT{}{{\FR 24:15: }或译:睁开}的人说,
\par }{\Q \VS{16}得听 神的言语,
\par }{\Q 明白至高者的意旨,
\par }{\Q 看见全能者的异象,
\par }{\Q 眼目睁开而仆倒的人说:
\par }{\Q \VS{17}我看他却不在现时;
\par }{\Q 我望他却不在近日。
\par }{\Q 有星要出于{\PN{雅各}},有杖要兴于{\PN{以色列}},
\par }{\Q 必打破{\PN{摩押}}的四角,毁坏扰乱之子。
\par }{\Q \VS{18}他必得{\PN{以东}}为基业,
\par }{\Q 又得仇敌之地{\PN{西珥}}为产业;
\par }{\Q {\PN{以色列}}必行事勇敢。
\par }{\Q \VS{19}有一位出于{\PN{雅各}}的,必掌大权;
\par }{\Q 他要除灭城中的余民。
\par }{\Q \VS{20}{\PN{巴兰}}观看{\PN{亚玛力}},就题起诗歌说:
\par }{\Q {\PN{亚玛力}}原为诸国之首,
\par }{\Q 但他终必沉沦。
\par }{\Q \VS{21}{\PN{巴兰}}观看{\PN{基尼}}人,就题起诗歌说:
\par }{\Q 你的住处本是坚固;
\par }{\Q 你的窝巢做在岩穴中。
\par }{\Q \VS{22}然而{\PN{基尼}}必至衰微,
\par }{\Q 直到{\PN{亚述}}把你掳去。
\par }{\Q \VS{23}{\PN{巴兰}}又题起诗歌说:
\par }{\Q 哀哉! 神行这事,谁能得活?
\par }{\Q \VS{24}必有人乘船从{\PN{基提}}界{\ADD{而来}},
\par }{\Q 苦害{\PN{亚述}},苦害{\PN{希伯}};
\par }{\Q 他也必至沉沦。
\par }{\Q \VS{25}于是{\PN{巴兰}}起来,回他本地去;{\PN{巴勒}}也回去了。

\par }\Chap{25}{\SH 以色列人在毗珥
\par }{\PP \VerseOne{1}{\PN{以色列}}人住在{\PN{什亭}},百姓与{\PN{摩押}}女子行起淫乱。
\VS{2}因为这女子叫百姓来,一同给她们的神献祭,百姓就吃{\ADD{她们的祭物}},跪拜她们的神。
\VS{3}{\PN{以色列}}人与{\PN{巴力·毗珥}}连合,耶和华的怒气就向{\PN{以色列}}人发作。
\VS{4}耶和华吩咐{\PN{摩西}}说:「将百姓中所有的族长在我面前对着日头悬挂,使我向{\PN{以色列}}人所发的怒气可以消了。」
\VS{5}于是{\PN{摩西}}吩咐{\PN{以色列}}的审判官说:「凡属你们的人,有与{\PN{巴力·毗珥}}连合的,你们各人要把他们杀了。」
\par }{\PP \VS{6}{\PN{摩西}}和{\PN{以色列}}全会众正在会幕门前哭泣的时候,谁知,有{\PN{以色列}}中的一个人,当他们眼前,带着一个{\PN{米甸}}女人到他弟兄那里去。
\VS{7}祭司{\PN{亚伦}}的孙子,{\PN{以利亚撒}}的儿子{\PN{非尼哈}}看见了,就从会中起来,手里拿着枪,
\VS{8}跟随那{\PN{以色列}}人进亭子里去,便将{\PN{以色列}}人和那女人由腹中刺透。这样,在{\PN{以色列}}人中瘟疫就止息了。
\VS{9}那时遭瘟疫死的,有二万四千人。
\par }{\PP \VS{10}耶和华晓谕{\PN{摩西}}说:
\VS{11}「祭司{\PN{亚伦}}的孙子,{\PN{以利亚撒}}的儿子{\PN{非尼哈}},使我向{\PN{以色列}}人所发的怒消了;因他在他们中间,以我的忌邪为心,使我不在忌邪中把他们除灭。
\VS{12}因此,你要说:『我将我平安的约赐给他。
\VS{13}这约要给他和他的后裔,作为永远当祭司职任的约;因他为 神有忌邪的心,为{\PN{以色列}}人赎罪。』」
\par }{\PP \VS{14}那与{\PN{米甸}}女人一同被杀的{\PN{以色列}}人,名叫{\PN{心利}},是{\PN{撒路}}的儿子,是{\PN{西缅}}一个宗族的首领。
\VS{15}那被杀的{\PN{米甸}}女人,名叫{\PN{哥斯比}},是{\PN{苏珥}}的女儿;这{\PN{苏珥}}是{\PN{米甸}}一个宗族的首领。
\par }{\PP \VS{16}耶和华晓谕{\PN{摩西}}说:
\VS{17}「你要扰害{\PN{米甸}}人,击杀他们;
\VS{18}因为他们用诡计扰害你们,在{\PN{毗珥}}的事上和他们的姊妹、{\PN{米甸}}首领的女儿{\PN{哥斯比}}的事上,用这诡计诱惑了你们;这{\PN{哥斯比}},当瘟疫流行的日子,因{\PN{毗珥}}的事被杀了。」

\par }\Chap{26}{\SH 第二次的人口普查
\par }{\PP \VerseOne{1}瘟疫之后,耶和华晓谕{\PN{摩西}}和祭司{\PN{亚伦}}的儿子{\PN{以利亚撒}}说:
\VS{2}「你们要将{\PN{以色列}}全会众,按他们的宗族,凡{\PN{以色列}}中从二十岁以外、能出去打仗的,计算总数。」
\VS{3}{\PN{摩西}}和祭司{\PN{以利亚撒}}在{\PN{摩押}}平原与{\PN{耶利哥}}相对的{\PN{约旦河}}边向{\PN{以色列}}人说:
\VS{4}「将你们中间从二十岁以外的{\ADD{计算总数}}」;是照耶和华吩咐出{\PN{埃及}}地的{\PN{摩西}}和{\PN{以色列}}人的话。
\par }{\PP \VS{5}{\PN{以色列}}的长子是{\PN{吕便}}。{\PN{吕便}}的众子:属{\PN{哈诺}}的,有{\PN{哈诺}}族;属{\PN{法路}}的,有{\PN{法路}}族;
\VS{6}属{\PN{希斯伦}}的,有{\PN{希斯伦}}族;属{\PN{迦米}}的,有{\PN{迦米}}族;
\VS{7}这就是{\PN{吕便}}的各族;其中被数的,共有四万三千七百三十名。
\VS{8}{\PN{法路}}的儿子是{\PN{以利押}}。
\VS{9}{\PN{以利押}}的众子是{\PN{尼母利}}、{\PN{大坍}}、{\PN{亚比兰}}。这{\PN{大坍}}、{\PN{亚比兰}},就是从会中选召的,与{\PN{可拉}}一党同向耶和华争闹的时候也向{\PN{摩西}}、{\PN{亚伦}}争闹;
\VS{10}地便开口吞了他们,和{\PN{可拉}}、{\PN{可拉}}的党类一同死亡。那时火烧灭了二百五十个人;他们就作了警戒。
\VS{11}然而{\PN{可拉}}的众子没有死亡。
\par }{\PP \VS{12}按着家族,{\PN{西缅}}的众子:属{\PN{尼母利}}的,有{\PN{尼母利}}族;属{\PN{雅悯}}的,有{\PN{雅悯}}族;属{\PN{雅斤}}的,有{\PN{雅斤}}族;
\VS{13}属{\PN{谢拉}}的,有{\PN{谢拉}}族;属{\PN{扫罗}}的,有{\PN{扫罗}}族。
\VS{14}这就是{\PN{西缅}}的各族,共有二万二千二百名。
\par }{\PP \VS{15}按着家族,{\PN{迦得}}的众子:属{\PN{洗分}}的,有{\PN{洗分}}族;属{\PN{哈基}}的,有{\PN{哈基}}族;属{\PN{书尼}}的,有{\PN{书尼}}族;
\VS{16}属{\PN{阿斯尼}}的,有{\PN{阿斯尼}}族;属{\PN{以利}}的,有{\PN{以利}}族;
\VS{17}属{\PN{亚律}}的,有{\PN{亚律}}族;属{\PN{亚列利}}的,有{\PN{亚列利}}族。
\VS{18}这就是{\PN{迦得}}子孙的各族,照他们中间被数的,共有四万零五百名。
\par }{\PP \VS{19}{\PN{犹大}}的儿子是{\PN{珥}}和{\PN{俄南}}。这{\PN{珥}}和{\PN{俄南}}死在{\PN{迦南}}地。
\VS{20}按着家族,{\PN{犹大}}{\ADD{其余}}的众子:属{\PN{示拉}}的,有{\PN{示拉}}族;属{\PN{法勒斯}}的,有{\PN{法勒斯}}族;属{\PN{谢拉}}的有{\PN{谢拉}}族。
\VS{21}{\PN{法勒斯}}的儿子:属{\PN{希斯
}}的,有{\PN{希斯
}}族;属{\PN{哈母勒}}的,有{\PN{哈母勒}}族。
\VS{22}这就是{\PN{犹大}}的各族,照他们中间被数的,共有七万六千五百名。
\par }{\PP \VS{23}按着家族,{\PN{以萨迦}}的众子:属{\PN{陀拉}}的,有{\PN{陀拉}}族;属{\PN{普瓦}}的,有{\PN{普瓦}}族;
\VS{24}属{\PN{雅述}}的,有{\PN{雅述}}族;属{\PN{伸
}}的,有{\PN{伸
}}族。
\VS{25}这就是{\PN{以萨迦}}的各族,照他们中间被数的,共有六万四千三百名。
\VS{26}按着家族,{\PN{西布伦}}的众子:属{\PN{西烈}}的,有{\PN{西烈}}族;属{\PN{以伦}}的,有{\PN{以伦}}族;属{\PN{雅利}}的,有{\PN{雅利}}族。
\VS{27}这就是{\PN{西布伦}}的各族,照他们中间被数的,共有六万零五百名。
\par }{\PP \VS{28}按着家族,{\PN{约瑟}}的儿子有{\PN{玛拿西}}、{\PN{以法莲}}。
\VS{29}{\PN{玛拿西}}的众子:属{\PN{玛吉}}的,有{\PN{玛吉}}族;{\PN{玛吉}}生{\PN{基列}};属{\PN{基列}}的,有{\PN{基列}}族。
\VS{30}{\PN{基列}}的众子:属{\PN{伊以谢}}的,有{\PN{伊以谢}}族;属{\PN{希勒}}的,有{\PN{希勒}}族;
\VS{31}属{\PN{亚斯烈}}的,有{\PN{亚斯烈}}族;属{\PN{示剑}}的,有{\PN{示剑}}族;
\VS{32}属{\PN{示米大}}的,有{\PN{示米大}}族;属{\PN{希弗}}的,有{\PN{希弗}}族。
\VS{33}{\PN{希弗}}的儿子:{\PN{西罗非哈}}没儿子,只有女儿。{\PN{西罗非哈}}女儿的名字就是{\PN{玛拉}}、{\PN{挪阿}}、{\PN{曷拉}}、{\PN{密迦}}、{\PN{得撒}}。
\VS{34}这就是{\PN{玛拿西}}的各族;他们中间被数的,共有五万二千七百名。
\par }{\PP \VS{35}按着家族,{\PN{以法莲}}的众子:属{\PN{书提拉}}的,有{\PN{书提拉}}族;属{\PN{比结}}的,有{\PN{比结}}族;属{\PN{他罕}}的,有{\PN{他罕}}族。
\VS{36}{\PN{书提拉}}的众子:属{\PN{以兰}}的,有{\PN{以兰}}族。
\VS{37}这就是{\PN{以法莲}}子孙的各族,照他们中间被数的,共有三万二千五百名。按着家族,这都是{\PN{约瑟}}的子孙。
\par }{\PP \VS{38}按着家族,{\PN{便雅悯}}的众子:属{\PN{比拉}}的,有{\PN{比拉}}族;属{\PN{亚实别}}的,有{\PN{亚实别}}族;属{\PN{亚希兰}}的,有{\PN{亚希兰}}族;
\VS{39}属{\PN{书反}}的,有{\PN{书反}}族;属{\PN{户反}}的,有{\PN{户反}}族。
\VS{40}{\PN{比拉}}的众子是{\PN{亚勒}}、{\PN{乃幔}}。{\ADD{属
{\PN{亚勒}} 的}},有{\PN{亚勒}}族;属{\PN{乃幔}}的,有{\PN{乃幔}}族。
\VS{41}按着家族,这就是{\PN{便雅悯}}的子孙,其中被数的,共有四万五千六百名。
\par }{\PP \VS{42}按着家族,{\PN{但}}的众子:属{\PN{书含}}的,有{\PN{书含}}族。按着家族,这就是{\PN{但}}的各族。
\VS{43}照其中被数的,{\PN{书含}}所有的各族,共有六万四千四百名。
\par }{\PP \VS{44}按着家族,{\PN{亚设}}的众子:属{\PN{音拿}}的,有{\PN{音拿}}族;属{\PN{亦施韦}}的,有{\PN{亦施韦}}族;属{\PN{比利亚}}的,有{\PN{比利亚}}族。
\VS{45}{\PN{比利亚}}的众子:属{\PN{希别}}的,有{\PN{希别}}族;属{\PN{玛结}}的,有{\PN{玛结}}族。
\VS{46}{\PN{亚设}}的女儿名叫{\PN{西拉}}。
\VS{47}这就是{\PN{亚设}}子孙的各族,照他们中间被数的,共有五万三千四百名。
\par }{\PP \VS{48}按着家族,{\PN{拿弗他利}}的众子:属{\PN{雅薛}}的,有{\PN{雅薛}}族;属{\PN{沽尼}}的,有{\PN{沽尼}}族;
\VS{49}属{\PN{耶色}}的,有{\PN{耶色}}族;属{\PN{示冷}}的,有{\PN{示冷}}族。
\VS{50}按着家族,这就是{\PN{拿弗他利}}的各族;他们中间被数的,共有四万五千四百名。
\par }{\PP \VS{51}{\PN{以色列}}人中被数的,共有六十万零一千七百三十名。
\par }{\PP \VS{52}耶和华晓谕{\PN{摩西}}说:
\VS{53}「你要按着人名的数目将地分给这些人为业。
\VS{54}人多的,你要把产业多分给他们;人少的,你要把产业少分给他们;要照被数的人数,把产业分给各人。
\VS{55}虽是这样,还要拈阄分地。他们要按着祖宗各支派的名字承受为业。
\VS{56}要按着所拈的阄,看人数多,人数少,把产业分给他们。」
\par }{\PP \VS{57}{\PN{利未}}人,按着他们的各族被数的:属{\PN{革顺}}的,有{\PN{革顺}}族;属{\PN{哥辖}}的,有{\PN{哥辖}}族;属{\PN{米拉利}}的,有{\PN{米拉利}}族。
\VS{58}{\PN{利未}}的各族有{\PN{立尼}}族、{\PN{希伯伦}}族、{\PN{玛利}}族、{\PN{母示}}族、{\PN{可拉}}族。{\PN{哥辖}}生{\PN{暗兰}}。
\VS{59}{\PN{暗兰}}的妻名叫{\PN{约基别}},是{\PN{利未}}女子,生在{\PN{埃及}}。她给{\PN{暗兰}}生了{\PN{亚伦}}、{\PN{摩西}},并他们的姊姊{\PN{米利暗}}。
\VS{60}{\PN{亚伦}}生{\PN{拿答}}、{\PN{亚比户}}、{\PN{以利亚撒}}、{\PN{以他玛}}。
\VS{61}{\PN{拿答}}、{\PN{亚比户}}在耶和华面前献凡火的时候就死了。
\VS{62}{\PN{利未}}人中,凡一个月以外、被数的男丁,共有二万三千。他们本来没有数在{\PN{以色列}}人中;因为在{\PN{以色列}}人中,没有分给他们产业。
\par }{\PP \VS{63}这些就是被{\PN{摩西}}和祭司{\PN{以利亚撒}}所数的;他们在{\PN{摩押}}平原与{\PN{耶利哥}}相对的{\PN{约旦河}}边数点{\PN{以色列}}人。
\VS{64}但被数的人中,没有一个是{\PN{摩西}}和祭司{\PN{亚伦}}从前在{\PN{西奈}}的旷野所数的{\PN{以色列}}人,
\VS{65}因为耶和华论到他们说:「他们必要死在旷野。」所以,除了{\PN{耶孚尼}}的儿子{\PN{迦勒}}和{\PN{嫩}}的儿子{\PN{约书亚}}以外,连一个人也没有存留。

\par }\Chap{27}{\SH 西罗非哈的女儿们
\par }{\PP \VerseOne{1}属{\PN{约瑟}}的儿子{\PN{玛拿西}}的各族,有{\PN{玛拿西}}的玄孙,{\PN{玛吉}}的曾孙,{\PN{基列}}的孙子,{\PN{希弗}}的儿子{\PN{西罗非哈}}的女儿,名叫{\PN{玛拉}}、{\PN{挪阿}}、{\PN{曷拉}}、{\PN{密迦}}、{\PN{得撒}}。她们前来,
\VS{2}站在会幕门口,在{\PN{摩西}}和祭司{\PN{以利亚撒}},并众首领与全会众面前,说:
\VS{3}「我们的父亲死在旷野。他不与{\PN{可拉}}同党聚集攻击耶和华,是在自己罪中死的;他也没有儿子。
\VS{4}为什么因我们的父亲没有儿子就把他的名从他族中除掉呢?求你们在我们父亲的弟兄中分给我们产业。」
\par }{\PP \VS{5}于是,{\PN{摩西}}将她们的案件呈到耶和华面前。
\VS{6}耶和华晓谕{\PN{摩西}}说:
\VS{7}「{\PN{西罗非哈}}的女儿说得有理。你定要在她们父亲的弟兄中,把地分给她们为业;要将她们父亲的产业归给她们。
\VS{8}你也要晓谕{\PN{以色列}}人说:『人若死了没有儿子,就要把他的产业归给他的女儿。
\VS{9}他若没有女儿,就要把他的产业给他的弟兄。
\VS{10}他若没有弟兄,就要把他的产业给他父亲的弟兄。
\VS{11}他父亲若没有弟兄,就要把他的产业给他族中最近的亲属,他便要得为业。』这要作{\PN{以色列}}人的律例典章,是照耶和华吩咐{\PN{摩西}}的。」
\par }{\SH 约书亚被选为摩西的继承人
\par }{\R (申31·1—8)
\par }{\PP \VS{12}耶和华对{\PN{摩西}}说:「你上这{\PN{亚巴琳山}},观看我所赐给{\PN{以色列}}人的地。
\VS{13}看了以后,你也必归到你列祖\FTNT{}{{\FR 27:13: }原文是本民}那里,像你哥哥{\PN{亚伦}}一样。
\VS{14}因为你们在{\PN{寻}}的旷野,当会众争闹的时候,违背了我的命,没有在{\ADD{涌}}水之地、会众眼前尊我为圣。」(这水就是{\PN{寻}}的旷野{\PN{加低斯米利巴}}水。)
\VS{15}{\PN{摩西}}对耶和华说:
\VS{16}「愿耶和华万人之灵的 神,立一个人治理会众,
\VS{17}可以在他们面前出入,也可以引导他们,免得耶和华的会众如同没有牧人的羊群一般。」
\VS{18}耶和华对{\PN{摩西}}说:「{\PN{嫩}}的儿子{\PN{约书亚}}是心中有{\ADD{圣}}灵的;你将他领来,按手在他{\ADD{头}}上,
\VS{19}使他站在祭司{\PN{以利亚撒}}和全会众面前,嘱咐他,
\VS{20}又将你的尊荣给他几分,使{\PN{以色列}}全会众都听从他。
\VS{21}他要站在祭司{\PN{以利亚撒}}面前;{\PN{以利亚撒}}要凭乌陵的判断,在耶和华面前为他求问。他和{\PN{以色列}}全会众都要遵{\PN{以利亚撒}}的命出入。」
\VS{22}于是{\PN{摩西}}照耶和华所吩咐的将{\PN{约书亚}}领来,使他站在祭司{\PN{以利亚撒}}和全会众面前,
\VS{23}按手在他{\ADD{头}}上,嘱咐他,是照耶和华借{\PN{摩西}}所说的话。

\par }\Chap{28}{\SH 每天例常的祭
\par }{\R (出29·38—46)
\par }{\PP \VerseOne{1}耶和华晓谕{\PN{摩西}}说:
\VS{2}「你要吩咐{\PN{以色列}}人说:『献给我的供物,就是献给我作馨香火祭的食物,你们要按日期献给我』;
\VS{3}又要对他们说:你们要献给耶和华的火祭,就是没有残疾、一岁的公羊羔,每日两只,作为常献的燔祭。
\VS{4}早晨要献一只,黄昏的时候要献一只;
\VS{5}又用细面伊法十分之一,并捣成的油一欣四分之一,调和作为素祭。
\VS{6}这是{\PN{西奈山}}所命定为常献的燔祭,是献给耶和华为馨香的火祭。
\VS{7}为这一只羊羔,要同献奠祭的酒一欣四分之一。在圣所中,你要将醇酒奉给耶和华为奠祭。
\VS{8}晚上,你要献那一只羊羔,必照早晨的素祭和同献的奠祭献上,作为馨香的火祭,献给耶和华。」
\par }{\SH 安息日献的祭
\par }{\PP \VS{9}「当安息日,要献两只没有残疾、一岁的公羊羔,并用调油的细面{\ADD{伊法}}十分之二为素祭,又将同献的奠祭献上。
\VS{10}这是每安息日献的燔祭;那常献的燔祭和同献的奠祭在外。」
\par }{\SH 月初献的祭
\par }{\PP \VS{11}「每月朔,你们要将两只公牛犊,一只公绵羊,七只没有残疾、一岁的公羊羔,献给耶和华为燔祭。
\VS{12}每只公牛要用调油的细面{\ADD{伊法}}十分之三作为素祭;那只公羊也用调油的细面{\ADD{伊法}}十分之二作为素祭;
\VS{13}每只羊羔要用调油的细面{\ADD{伊法}}十分之一作为素祭和馨香的燔祭,是献给耶和华的火祭。
\VS{14}一只公牛要奠酒半欣,一只公羊要奠酒一欣三分之一,一只羊羔也奠酒一欣四分之一。这是每月的燔祭,一年之中要月月如此。
\VS{15}又要将一只公山羊为赎罪祭,献给耶和华;要献在常献的燔祭和同献的奠祭以外。」
\par }{\SH 除酵节献的祭
\par }{\R (利23·5—14)
\par }{\PP \VS{16}「正月十四日是耶和华的逾越节。
\VS{17}这月十五日是节期,要吃无酵饼七日。
\VS{18}第一日当有圣会;什么劳碌的工都不可做。
\VS{19}当将公牛犊两只,公绵羊一只,一岁的公羊羔七只,都要没有残疾的,用火献给耶和华为燔祭。
\VS{20}同献的素祭用调油的细面;为一只公牛要献{\ADD{伊法}}十分之三;为一只公羊要献{\ADD{伊法}}十分之二;
\VS{21}为那七只羊羔,每只要献{\ADD{伊法}}十分之一。
\VS{22}并献一只公山羊作赎罪祭,为你们赎罪。
\VS{23}你们献这些,要在早晨常献的燔祭以外。
\VS{24}一连七日,每日要照这例把馨香火祭的食物献给耶和华,是在常献的燔祭和同献的奠祭以外。
\VS{25}第七日当有圣会,什么劳碌的工都不可做。」
\par }{\SH 七七收获节献的祭
\par }{\R (利23·15—22)
\par }{\PP \VS{26}「七七{\ADD{节}}庄稼初熟,你们献新素祭给耶和华的日子,当有圣会;什么劳碌的工都不可做。
\VS{27}只要将公牛犊两只,公绵羊一只,一岁的公羊羔七只,作为馨香的燔祭献给耶和华。
\VS{28}同献的素祭用调油的细面;为每只公牛要献{\ADD{伊法}}十分之三;为一只公羊要献{\ADD{伊法}}十分之二;
\VS{29}为那七只羊羔,每只要献{\ADD{伊法}}十分之一。
\VS{30}并献一只公山羊为你们赎罪。
\VS{31}这些,你们要献在常献的燔祭和同献的素祭并同献的奠祭以外,都要没有残疾的。」

\par }\Chap{29}{\SH 新年献的祭
\par }{\R (利23·23—25)
\par }{\PP \VerseOne{1}「七月初一日,你们当有圣会;什么劳碌的工都不可做,是你们当守为吹角的日子。
\VS{2}你们要将公牛犊一只,公绵羊一只,没有残疾、一岁的公羊羔七只,作为馨香的燔祭献给耶和华。
\VS{3}同献的素祭用调油的细面;为一只公牛要献{\ADD{伊法}}十分之三;为一只公羊要献{\ADD{伊法}}十分之二;
\VS{4}为那七只羊羔,每只要献{\ADD{伊法}}十分之一。
\VS{5}又献一只公山羊作赎罪祭,为你们赎罪。
\VS{6}这些是在月朔的燔祭和同献的素祭,并常献的燔祭与同献的素祭,以及照例同献的奠祭以外,都作为馨香的火祭献给耶和华。」
\par }{\SH 赎罪日献的祭
\par }{\R (利23·26—32)
\par }{\PP \VS{7}「七月初十日,你们当有圣会;要刻苦己心,什么工都不可做。
\VS{8}只要将公牛犊一只,公绵羊一只,一岁的公羊羔七只,都要没有残疾的,作为馨香的燔祭献给耶和华。
\VS{9}同献的素祭用调油的细面:为一只公牛要献{\ADD{伊法}}十分之三;为一只公羊要献{\ADD{伊法}}十分之二;
\VS{10}为那七只羊羔,每只要献{\ADD{伊法}}十分之一。
\VS{11}又献一只公山羊为赎罪祭。这是在赎罪祭和常献的燔祭,与同献的素祭并同献的奠祭以外。」
\par }{\SH 住棚节献的祭
\par }{\R (利23·33—44)
\par }{\PP \VS{12}「七月十五日,你们当有圣会;什么劳碌的工都不可做,要向耶和华守节七日。
\VS{13}又要将公牛犊十三只,公绵羊两只,一岁的公羊羔十四只,都要没有残疾的,用火献给耶和华为馨香的燔祭。
\VS{14}同献的素祭用调油的细面;为那十三只公牛,每只要献{\ADD{伊法}}十分之三;为那两只公羊,每只要献{\ADD{伊法}}十分之二;
\VS{15}为那十四只羊羔,每只要献{\ADD{伊法}}十分之一。
\VS{16}并献一只公山羊为赎罪祭,这是在常献的燔祭和同献的素祭并同献的奠祭以外。
\par }{\PP \VS{17}「第二日{\ADD{要献}}公牛犊十二只,公绵羊两只,没有残疾、一岁的公羊羔十四只;
\VS{18}并为公牛、公羊,和羊羔,按数照例,献同献的素祭和同献的奠祭。
\VS{19}又要献一只公山羊为赎罪祭。这是在常献的燔祭和同献的素祭并同献的奠祭以外。
\par }{\PP \VS{20}「第三日{\ADD{要献}}公牛十一只,公羊两只,没有残疾、一岁的公羊羔十四只;
\VS{21}并为公牛、公羊,和羊羔,按数照例,献同献的素祭和同献的奠祭。
\VS{22}又要献一只公山羊为赎罪祭。这是在常献的燔祭和同献的素祭并同献的奠祭以外。
\par }{\PP \VS{23}「第四日{\ADD{要献}}公牛十只,公羊两只,没有残疾、一岁的公羊羔十四只;
\VS{24}并为公牛、公羊,和羊羔,按数照例,献同献的素祭和同献的奠祭。
\VS{25}又要献一只公山羊为赎罪祭。这是在常献的燔祭和同献的素祭并同献的奠祭以外。
\par }{\PP \VS{26}「第五日{\ADD{要献}}公牛九只,公羊两只,没有残疾、一岁的公羊羔十四只;
\VS{27}并为公牛、公羊,和羊羔,按数照例,献同献的素祭和同献的奠祭。
\VS{28}又要献一只公山羊为赎罪祭。这是在常献的燔祭和同献的素祭并同献的奠祭以外。
\par }{\PP \VS{29}「第六日{\ADD{要献}}公牛八只,公羊两只,没有残疾、一岁的公羊羔十四只;
\VS{30}并为公牛、公羊,和羊羔,按数照例,献同献的素祭和同献的奠祭。
\VS{31}又要献一只公山羊为赎罪祭。这是在常献的燔祭和同献的素祭并同献的奠祭以外。
\par }{\PP \VS{32}「第七日{\ADD{要献}}公牛七只,公羊两只,没有残疾、一岁的公羊羔十四只;
\VS{33}并为公牛、公羊,和羊羔,按数照例,献同献的素祭和同献的奠祭。
\VS{34}又要献一只公山羊为赎罪祭。这是在常献的燔祭和同献的素祭并同献的奠祭以外。
\par }{\PP \VS{35}「第八日你们当有严肃会;什么劳碌的工都不可做;
\VS{36}只要将公牛一只,公羊一只,没有残疾、一岁的公羊羔七只作火祭,献给耶和华为馨香的燔祭;
\VS{37}并为公牛、公羊,和羊羔,按数照例,献同献的素祭和同献的奠祭。
\VS{38}又要献一只公山羊为赎罪祭。这是在常献的燔祭和同献的素祭并同献的奠祭以外。
\par }{\PP \VS{39}「这些祭要在你们的节期献给耶和华,都在所许的愿并甘心所献的以外,作为你们的燔祭、素祭、奠祭,和平安祭。」
\par }{\PP \VS{40}于是,{\PN{摩西}}照耶和华所吩咐他的一切话告诉{\PN{以色列}}人。

\par }\Chap{30}{\SH 许愿的条例
\par }{\PP \VerseOne{1}{\PN{摩西}}晓谕{\PN{以色列}}各支派的首领说:「耶和华所吩咐的乃是这样:
\VS{2}人若向耶和华许愿或起誓,要约束自己,就不可食言,必要按口中所出的一切话行。
\VS{3}女子年幼、还在父家的时候,若向耶和华许愿,要约束自己,
\VS{4}她父亲也听见她所许的愿并约束自己的话,却向她默默不言,她所许的愿并约束自己的话就都要为定。
\VS{5}但她父亲听见的日子若不应承她所许的愿和约束自己的话,就都不得为定;耶和华也必赦免她,因为她父亲不应承。
\VS{6}她若出了嫁,有愿在身,或是口中出了约束自己的冒失话,
\VS{7}她丈夫听见的日子,却向她默默不言,她所许的愿并约束自己的话就都要为定。
\VS{8}但她丈夫听见的日子,若不应承,就算废了她所许的愿和她出口约束自己的冒失话;耶和华也必赦免她。
\VS{9}寡妇或是被休的妇人所许的愿,就是她约束自己的话,都要为定。
\VS{10}她若在丈夫家里许了愿或起了誓,约束自己,
\VS{11}丈夫听见,却向她默默不言,也没有不应承,她所许的愿并约束自己的话就都要为定。
\VS{12}丈夫听见的日子,若把这两样全废了,妇人口中所许的愿或是约束自己的话就都不得为定,{\ADD{因}}她丈夫已经把这两样废了;耶和华也必赦免她。
\VS{13}凡她所许的愿和刻苦约束自己所起的誓,她丈夫可以坚定,也可以废去。
\VS{14}倘若她丈夫天天向她默默不言,就算是坚定她所许的愿和约束自己的话;因丈夫听见的日子向她默默不言,就使这两样坚定。
\VS{15}但她丈夫听见以后,若使这两样全废了,就要担当妇人的罪孽。」
\par }{\PP \VS{16}这是丈夫待妻子,父亲待女儿,女儿年幼、还在父家,耶和华所吩咐{\PN{摩西}}的律例。

\par }\Chap{31}{\SH 跟米甸人打仗
\par }{\PP \VerseOne{1}耶和华吩咐{\PN{摩西}}说:
\VS{2}「你要在{\PN{米甸}}人身上报{\PN{以色列}}人的仇,后来要归到你列祖\FTNT{}{{\FR 31:2: }原文是本民}那里。」
\VS{3}{\PN{摩西}}吩咐百姓说:「要从你们中间叫人带兵器出去攻击{\PN{米甸}},好在{\PN{米甸}}人身上为耶和华报仇。
\VS{4}从{\PN{以色列}}众支派中,每支派要打发一千人去打仗。」
\VS{5}于是从{\PN{以色列}}千万人中,每支派交出一千人,共一万二千人,带着兵器预备打仗。
\VS{6}{\PN{摩西}}就打发每支派的一千人去打仗,并打发祭司{\PN{以利亚撒}}的儿子{\PN{非尼哈}}同去;{\PN{非尼哈}}手里拿着圣所的器皿和吹大声的号筒。
\VS{7}他们就照耶和华所吩咐{\PN{摩西}}的,与{\PN{米甸}}人打仗,杀了所有的男丁。
\VS{8}在所杀的人中,杀了{\PN{米甸}}的五王,就是{\PN{以未}}、{\PN{利金}}、{\PN{苏珥}}、{\PN{户珥}}、{\PN{利巴}},又用刀杀了{\PN{比珥}}的儿子{\PN{巴兰}}。
\VS{9}{\PN{以色列}}人掳了{\PN{米甸}}人的妇女孩子,并将他们的牲畜、羊群,和所有的财物都夺了来,当作掳物,
\VS{10}又用火焚烧他们所住的城邑和所有的营寨,
\VS{11}把一切所夺的、所掳的,连人带牲畜都带了去,
\VS{12}将所掳的人,所夺的牲畜、财物,都带到{\PN{摩押}}平原,在{\PN{约旦河}}边与{\PN{耶利哥}}相对的营盘,交给{\PN{摩西}}和祭司{\PN{以利亚撒}},并{\PN{以色列}}的会众。
\par }{\SH 打仗回来
\par }{\PP \VS{13}{\PN{摩西}}和祭司{\PN{以利亚撒}},并会众一切的首领,都出到营外迎接他们。
\VS{14}{\PN{摩西}}向打仗回来的军长,就是千夫长、百夫长,发怒,
\VS{15}对他们说:「你们要存留这一切妇女的活命吗?
\VS{16}这些妇女因{\PN{巴兰}}的计谋,叫{\PN{以色列}}人在{\PN{毗珥}}的事上得罪耶和华,以致耶和华的会众遭遇瘟疫。
\VS{17}所以,你们要把一切的男孩和所有已嫁的女子都杀了。
\VS{18}但女孩子中,凡没有出嫁的,你们都可以存留她的活命。
\VS{19}你们要在营外驻扎七日;凡杀了人的,和一切摸了被杀的,并你们所掳来的人口,第三日,第七日,都要洁净自己,
\VS{20}也要因一切的衣服、皮物、山羊{\ADD{毛}}织的物,和各样的木器,洁净自己。」
\par }{\PP \VS{21}祭司{\PN{以利亚撒}}对打仗回来的兵丁说:「耶和华所吩咐{\PN{摩西}}律法中的条例乃是这样:
\VS{22}金、银、铜、铁、锡、铅,
\VS{23}凡能见火的,你们要叫它经火就为洁净,然而还要用除污秽的水洁净它;凡不能见火的,你们要叫它过水。
\VS{24}第七日,你们要洗衣服,就为洁净,然后可以进营。」
\par }{\SH 分战利品
\par }{\PP \VS{25}耶和华晓谕{\PN{摩西}}说:
\VS{26}「你和祭司{\PN{以利亚撒}},并会众的各族长,要计算所掳来的人口和牲畜的总数。
\VS{27}把所掳来的分作两半:一半归与出去打仗的精兵,一半归与全会众。
\VS{28}又要从出去打仗所得的人口、牛、驴、羊群中,每五百取一,作为贡物奉给耶和华。
\VS{29}从他们一半之中,要取出来交给祭司{\PN{以利亚撒}},作为耶和华的举祭。
\VS{30}从{\PN{以色列}}人的一半之中,就是从人口、牛、驴、羊群、各样牲畜中,每五十取一,交给看守耶和华帐幕的{\PN{利未}}人。」
\VS{31}于是{\PN{摩西}}和祭司{\PN{以利亚撒}}照耶和华所吩咐{\PN{摩西}}的行了。
\par }{\PP \VS{32}除了兵丁所夺的财物以外,所掳来的:羊六十七万五千只;
\VS{33}牛七万二千只;
\VS{34}驴六万一千匹;
\VS{35}女人共三万二千口,都是没有出嫁的。
\VS{36}出去打仗之人的分,就是他们所得的那一半,共计羊三十三万七千五百只,
\VS{37}从其中归耶和华为贡物的,有六百七十五只;
\VS{38}牛三万六千只,从其中归耶和华为贡物的,有七十二只;
\VS{39}驴三万零五百匹,从其中归耶和华为贡物的,有六十一匹;
\VS{40}人一万六千口,从其中归耶和华的,有三十二口。
\VS{41}{\PN{摩西}}把贡物,就是归与耶和华的举祭,交给祭司{\PN{以利亚撒}},是照耶和华所吩咐{\PN{摩西}}的。
\par }{\PP \VS{42}{\PN{以色列}}人所得的那一半,就是{\PN{摩西}}从打仗的人取来分给他们的。(
\VS{43}会众的那一半有:羊三十三万七千五百只;
\VS{44}牛三万六千只;
\VS{45}驴三万零五百匹;
\VS{46}人一万六千口。)
\VS{47}无论是人口是牲畜,{\PN{摩西}}每五十取一,交给看守耶和华帐幕的{\PN{利未}}人,是照耶和华所吩咐{\PN{摩西}}的。
\par }{\PP \VS{48}带领千军的各军长,就是千夫长、百夫长,都近前来见{\PN{摩西}},
\VS{49}对他说:「仆人权下的兵已经计算总数,并不短少一人。
\VS{50}如今我们将各人所得的金器,就是脚链子、镯子、打印的戒指、耳环、手钏,都送来为耶和华的供物,好在耶和华面前为我们的生命赎罪。」
\VS{51}{\PN{摩西}}和祭司{\PN{以利亚撒}}就收了他们的金子,都是打成的器皿。
\VS{52}千夫长、百夫长所献给耶和华为举祭的金子共有一万六千七百五十舍客勒。
\VS{53}各兵丁都为自己夺了财物。
\VS{54}{\PN{摩西}}和祭司{\PN{以利亚撒}}收了千夫长、百夫长的金子,就带进会幕,在耶和华面前作为{\PN{以色列}}人的纪念。

\par }\Chap{32}{\SH 约旦河东的两支派
\par }{\R (申3·12—22)
\par }{\PP \VerseOne{1}{\PN{吕便}}子孙和{\PN{迦得}}子孙的牲畜极其众多;他们看见{\PN{雅谢}}地和{\PN{基列}}地是可牧放牲畜之地,
\VS{2}就来见{\PN{摩西}}和祭司{\PN{以利亚撒}},并会众的首领,说:
\VS{3}「{\PN{亚大录}}、{\PN{底本}}、{\PN{雅谢}}、{\PN{宁拉}}、{\PN{希实本}}、{\PN{以利亚利}}、{\PN{示班}}、{\PN{尼波}}、{\PN{比稳}},
\VS{4}就是耶和华在{\PN{以色列}}会众前面所攻取之地,是可牧放牲畜之地,你仆人也有牲畜」;
\VS{5}又说:「我们若在你眼前蒙恩,求你把这地给我们为业,不要领我们过{\PN{约旦河}}。」
\par }{\PP \VS{6}{\PN{摩西}}对{\PN{迦得}}子孙和{\PN{吕便}}子孙说:「难道你们的弟兄去打仗,你们竟坐在这里吗?
\VS{7}你们为何使{\PN{以色列}}人灰心丧胆、不过去进入耶和华所赐给他们的那地呢?
\VS{8}我先前从{\PN{加低斯·巴尼亚}}打发你们先祖去窥探那地,他们也是这样行。
\VS{9}他们上{\PN{以实各谷}},去窥探那地回来的时候,使{\PN{以色列}}人灰心丧胆,不进入耶和华所赐给他们的地。
\VS{10}当日,耶和华的怒气发作,就起誓说:
\VS{11}『凡从{\PN{埃及}}上来、二十岁以外的人断不得看见我对{\PN{亚伯拉罕}}、{\PN{以撒}}、{\PN{雅各}}起誓应许之地,因为他们没有专心跟从我。
\VS{12}惟有{\PN{基尼洗}}族{\PN{耶孚尼}}的儿子{\PN{迦勒}}和{\PN{嫩}}的儿子{\PN{约书亚}}可以看见,因为他们专心跟从我。』
\VS{13}耶和华的怒气向{\PN{以色列}}人发作,使他们在旷野飘流四十年,等到在耶和华眼前行恶的那一代人都消灭了。
\VS{14}谁知,你们起来接续先祖,增添罪人的数目,使耶和华向{\PN{以色列}}大发烈怒。
\VS{15}你们若退后不跟从他,他还要把{\PN{以色列}}人撇在旷野,便是你们使这众民灭亡。」
\par }{\PP \VS{16}两支派的人挨近{\PN{摩西}},说:「我们要在这里为牲畜垒圈,为{\ADD{妇人}}孩子造城。
\VS{17}我们自己要带兵器行在{\PN{以色列}}人的前头,好把他们领到他们的地方;但我们的{\ADD{妇人}}孩子,因这地居民的缘故,要住在坚固的城内。
\VS{18}我们不回家,直等到{\PN{以色列}}人各承受自己的产业。
\VS{19}我们不和他们在{\PN{约旦河}}那边一带之地同受产业,因为我们的产业是坐落在{\PN{约旦河}}东边这里。」
\VS{20}{\PN{摩西}}对他们说:「你们若这样行,在耶和华面前带着兵器出去打仗,
\VS{21}所有带兵器的人都要在耶和华面前过{\PN{约旦河}},等他赶出他的仇敌,
\VS{22}那地被耶和华制伏了,然后你们可以回来,向耶和华和{\PN{以色列}}才为无罪,这地也必在耶和华面前归你们为业。
\VS{23}倘若你们不这样行,就得罪耶和华,要知道你们的罪必追上你们。
\VS{24}如今你们口中所出的,只管去行,为你们的{\ADD{妇人}}孩子造城,为你们的羊群垒圈。」
\VS{25}{\PN{迦得}}子孙和{\PN{吕便}}子孙对{\PN{摩西}}说:「仆人要照我主所吩咐的去行。
\VS{26}我们的妻子、孩子、羊群,和所有的牲畜都要留在{\PN{基列}}的各城。
\VS{27}但你的仆人,凡带兵器的,都要照我主所说的话,在耶和华面前过去打仗。」
\par }{\PP \VS{28}于是,{\PN{摩西}}为他们嘱咐祭司{\PN{以利亚撒}}和{\PN{嫩}}的儿子{\PN{约书亚}},并{\PN{以色列}}众支派的族长,说:
\VS{29}「{\PN{迦得}}子孙和{\PN{吕便}}子孙,凡带兵器在耶和华面前去打仗的,若与你们一同过{\PN{约旦河}},那地被你们制伏了,你们就要把{\PN{基列}}地给他们为业。
\VS{30}倘若他们不带兵器和你们一同过去,就要在{\PN{迦南}}地你们中间得产业。」
\VS{31}{\PN{迦得}}子孙和{\PN{吕便}}子孙回答说:「耶和华怎样吩咐仆人,仆人就怎样行。
\VS{32}我们要带兵器,在耶和华面前过去,进入{\PN{迦南}}地,只是{\PN{约旦河}}这边、我们所得为业之地{\ADD{仍归}}我们。」
\par }{\PP \VS{33}{\PN{摩西}}将{\PN{亚摩利}}王{\PN{西宏}}的国和{\PN{巴珊}}王{\PN{噩}}的国,连那地和周围的城邑,都给了{\PN{迦得}}子孙和{\PN{吕便}}子孙,并{\PN{约瑟}}的儿子{\PN{玛拿西}}半个支派。
\VS{34}{\PN{迦得}}子孙建造{\PN{底本}}、{\PN{亚他录}}、{\PN{亚罗珥}}、
\VS{35}{\PN{亚他录·朔反}}、{\PN{雅谢}}、{\PN{约比哈}}、
\VS{36}{\PN{伯·宁拉}}、{\PN{伯·哈兰}},都是坚固城。他们又垒羊圈。
\VS{37}{\PN{吕便}}子孙建造{\PN{希实本}}、{\PN{以利亚利}}、{\PN{基列亭}}、
\VS{38}{\PN{尼波}}、{\PN{巴力·免}}、{\PN{西比玛}}({\PN{尼波}}、{\PN{巴力·免}},名字是改了的),又给他们所建造的城另起别名。
\VS{39}{\PN{玛拿西}}的儿子{\PN{玛吉}},他的子孙往{\PN{基列}}去,占了那地,赶出那里的{\PN{亚摩利}}人。
\VS{40}{\PN{摩西}}将{\PN{基列}}赐给{\PN{玛拿西}}的儿子{\PN{玛吉}},他{\ADD{子孙}}就住在那里。
\VS{41}{\PN{玛拿西}}的子孙{\PN{睚珥}}去占了{\PN{基列}}的村庄,就称这些村庄为{\PN{哈倭特·睚珥}}。
\VS{42}{\PN{挪巴}}去占了{\PN{基纳}}和{\PN{基纳}}的乡村,就按自己的名称{\PN{基纳}}为{\PN{挪巴}}。

\par }\Chap{33}{\SH 从埃及到摩押
\par }{\PP \VerseOne{1}{\PN{以色列}}人按着军队,在{\PN{摩西}}、{\PN{亚伦}}的手下出{\PN{埃及}}地所行的路程\FTNT{}{{\FR 33:1: }或译:站口;下同}记在下面。
\VS{2}{\PN{摩西}}遵着耶和华的吩咐记载他们所行的路程,其路程乃是这样:
\VS{3}正月十五日,就是逾越节的次日,{\PN{以色列}}人从{\PN{兰塞}}起行,在一切{\PN{埃及}}人眼前昂然无惧地出去。
\VS{4}那时,{\PN{埃及}}人正葬埋他们的长子,就是耶和华在他们中间所击杀的;耶和华也败坏他们的神。
\par }{\PP \VS{5}{\PN{以色列}}人从{\PN{兰塞}}起行,安营在{\PN{疏割}}。
\VS{6}从{\PN{疏割}}起行,安营在旷野边的{\PN{以倘}}。
\VS{7}从{\PN{以倘}}起行,转到{\PN{比哈·希录}},是在{\PN{巴力·洗分}}对面,就在{\PN{密夺}}安营。
\VS{8}从{\PN{比哈·希录}}对面起行,经过海中到了{\ADD{
{\PN{书珥}}}}旷野,又在{\PN{伊坦}}的旷野走了三天的路程,就安营在{\PN{玛拉}}。
\VS{9}从{\PN{玛拉}}起行,来到{\PN{以琳}}({\PN{以琳}}有十二股水泉,七十棵棕树),就在那里安营。
\VS{10}从{\PN{以琳}}起行,安营在{\PN{红海}}边。
\VS{11}从{\PN{红海}}边起行,安营在{\PN{汛}}的旷野。
\VS{12}从{\PN{汛}}的旷野起行,安营在{\PN{脱加}}。
\VS{13}从{\PN{脱加}}起行,安营在{\PN{亚录}}。
\VS{14}从{\PN{亚录}}起行,安营在{\PN{利非订}};在那里,百姓没有水喝。
\VS{15}从{\PN{利非订}}起行,安营在{\PN{西奈}}的旷野。
\VS{16}从{\PN{西奈}}的旷野起行,安营在{\PN{基博罗·哈他瓦}}。
\VS{17}从{\PN{基博罗·哈他瓦}}起行,安营在{\PN{哈洗录}}。
\VS{18}从{\PN{哈洗录}}起行,安营在{\PN{利提玛}}。
\VS{19}从{\PN{利提玛}}起行,安营在{\PN{临门·帕烈}}。
\VS{20}从{\PN{临门·帕烈}}起行,安营在{\PN{立拿}}。
\VS{21}从{\PN{立拿}}起行,安营在{\PN{勒撒}}。
\VS{22}从{\PN{勒撒}}起行,安营在{\PN{基希拉他}}。
\VS{23}从{\PN{基希拉他}}起行,安营在{\PN{沙斐山}}。
\VS{24}从{\PN{沙斐山}}起行,安营在{\PN{哈拉大}}。
\VS{25}从{\PN{哈拉大}}起行,安营在{\PN{玛吉希录}}。
\VS{26}从{\PN{玛吉希录}}起行,安营在{\PN{他哈}}。
\VS{27}从{\PN{他哈}}起行,安营在{\PN{他拉}}。
\VS{28}从{\PN{他拉}}起行,安营在{\PN{密加}}。
\VS{29}从{\PN{密加}}起行,安营在{\PN{哈摩拿}}。
\VS{30}从{\PN{哈摩拿}}起行,安营在{\PN{摩西录}}。
\VS{31}从{\PN{摩西录}}起行,安营在{\PN{比尼·亚干}}。
\VS{32}从{\PN{比尼·亚干}}起行,安营在{\PN{曷·哈及甲}}。
\VS{33}从{\PN{曷·哈及甲}}起行,安营在{\PN{约巴他}}。
\VS{34}从{\PN{约巴他}}起行,安营在{\PN{阿博拿}}。
\VS{35}从{\PN{阿博拿}}起行,安营在{\PN{以旬·迦别}}。
\VS{36}从{\PN{以旬·迦别}}起行,安营在{\PN{寻}}的旷野,就是{\PN{加低斯}}。
\VS{37}从{\PN{加低斯}}起行,安营在{\PN{何珥山}},{\PN{以东}}地的边界。
\par }{\PP \VS{38}{\PN{以色列}}人出了{\PN{埃及}}地后四十年,五月初一日,祭司{\PN{亚伦}}遵着耶和华的吩咐上{\PN{何珥山}},就死在那里。
\VS{39}{\PN{亚伦}}死在{\PN{何珥山}}的时候年一百二十三岁。
\par }{\PP \VS{40}住在{\PN{迦南}}南地的{\PN{迦南}}人{\PN{亚拉得}}王听说{\PN{以色列}}人来了。
\par }{\PP \VS{41}{\PN{以色列}}人从{\PN{何珥山}}起行,安营在{\PN{撒摩拿}}。
\VS{42}从{\PN{撒摩拿}}起行,安营在{\PN{普嫩}}。
\VS{43}从{\PN{普嫩}}起行,安营在{\PN{阿伯}}。
\VS{44}从{\PN{阿伯}}起行,安营在{\PN{以耶·亚巴琳}},{\PN{摩押}}的边界。
\VS{45}从{\PN{以耶·亚巴琳}}起行,安营在{\PN{底本·迦得}}。
\VS{46}从{\PN{底本·迦得}}起行,安营在{\PN{亚门·低比拉太音}}。
\VS{47}从{\PN{亚门·低比拉太音}}起行,安营在{\PN{尼波}}对面的{\PN{亚巴琳山}}里。
\VS{48}从{\PN{亚巴琳山}}起行,安营在{\PN{摩押}}平原—{\PN{约旦河}}边、{\PN{耶利哥}}对面。
\VS{49}他们在{\PN{摩押}}平原沿{\PN{约旦河}}边安营,从{\PN{伯·耶施末}}直到{\PN{亚伯·什亭}}。
\par }{\SH 过约旦河前的指示
\par }{\PP \VS{50}耶和华在{\PN{摩押}}平原—{\PN{约旦河}}边、{\PN{耶利哥}}对面晓谕{\PN{摩西}}说:
\VS{51}「你吩咐{\PN{以色列}}人说:你们过{\PN{约旦河}}进{\PN{迦南}}地的时候,
\VS{52}就要从你们面前赶出那里所有的居民,毁灭他们一切錾成的{\ADD{石}}像和他们一切铸成的偶像,又拆毁他们一切的邱坛。
\VS{53}你们要夺那地,住在其中,因我把那地赐给你们为业。
\VS{54}你们要按家室拈阄,承受那地;人多的,要把产业多分给他们;人少的,要把产业少分给他们。拈出何地给何人,就要归何人。你们要按宗族的支派承受。
\VS{55}倘若你们不赶出那地的居民,所容留的居民就必作你们眼中的刺,肋下的荆棘,也必在你们所住的地上扰害你们。
\VS{56}而且我素常有意怎样待他们,也必照样待你们。」

\par }\Chap{34}{\SH 迦南地的境界
\par }{\PP \VerseOne{1}耶和华晓谕{\PN{摩西}}说:
\VS{2}「你吩咐{\PN{以色列}}人说:你们到了{\PN{迦南}}地,就是归你们为业的{\PN{迦南}}四境之地,
\VS{3}南角要从{\PN{寻}}的旷野,贴着{\PN{以东}}的边界;南界要从{\PN{盐海}}东头起,
\VS{4}绕到{\PN{亚克拉滨}}坡的南边,接连到{\PN{寻}},直通到{\PN{加低斯·巴尼亚}}的南边,又通到{\PN{哈萨·亚达}},接连到{\PN{押们}},
\VS{5}从{\PN{押们}}转到{\PN{埃及}}小河,直通到海为止。
\par }{\PP \VS{6}「西边要以大海为界;这就是你们的西界。
\par }{\PP \VS{7}「北界要从大海起,划到{\PN{何珥山}},
\VS{8}从{\PN{何珥山}}划到{\PN{哈马口}},通到{\PN{西达达}},
\VS{9}又通到{\PN{西斐
}},直到{\PN{哈萨·以难}}。这要作你们的北界。
\par }{\PP \VS{10}「你们要从{\PN{哈萨·以难}}划到{\PN{示番}}为东界。
\VS{11}这界要从{\PN{示番}}下到{\PN{亚延}}东边的{\PN{利比拉}},又要达到{\PN{基尼烈湖}}的东边。
\VS{12}这界要下到{\PN{约旦河}},通到{\PN{盐海}}为止。这四围的边界以内,要作你们的地。」
\par }{\PP \VS{13}{\PN{摩西}}吩咐{\PN{以色列}}人说:「这地就是耶和华吩咐拈阄给九个半支派承受为业的;
\VS{14}因为{\PN{吕便}}支派和{\PN{迦得}}支派按着宗族受了产业,{\PN{玛拿西}}半个支派也受了产业。
\VS{15}这两个半支派已经在{\PN{耶利哥}}对面、{\PN{约旦河}}东、向日出之地受了产业。」
\par }{\SH 各支派首领负责分地
\par }{\PP \VS{16}耶和华晓谕{\PN{摩西}}说:
\VS{17}「要给你们分地为业之人的名字是祭司{\PN{以利亚撒}}和{\PN{嫩}}的儿子{\PN{约书亚}}。
\VS{18}又要从每支派中选一个首领帮助他们。
\VS{19}这些人的名字:{\PN{犹大}}支派有{\PN{耶孚尼}}的儿子{\PN{迦勒}}。
\VS{20}{\PN{西缅}}支派有{\PN{亚米忽}}的儿子{\PN{示母利}}。
\VS{21}{\PN{便雅悯}}支派有{\PN{基斯伦}}的儿子{\PN{以利达}}。
\VS{22}{\PN{但}}支派有一个首领,{\PN{约利}}的儿子{\PN{布基}}。
\VS{23}{\PN{约瑟}}的子孙{\PN{玛拿西}}支派有一个首领,{\PN{以弗}}的儿子{\PN{汉聂}}。
\VS{24}{\PN{以法莲}}支派有一个首领,{\PN{拾弗但}}的儿子{\PN{基母利}}。
\VS{25}{\PN{西布伦}}支派有一个首领,{\PN{帕纳}}的儿子{\PN{以利撒番}}。
\VS{26}{\PN{以萨迦}}支派有一个首领,{\PN{阿散}}的儿子{\PN{帕铁}}。
\VS{27}{\PN{亚设}}支派有一个首领,{\PN{示罗米}}的儿子{\PN{亚希忽}}。
\VS{28}{\PN{拿弗他利}}支派有一个首领,{\PN{亚米忽}}的儿子{\PN{比大黑}}。」
\par }{\PP \VS{29}这些人就是耶和华所吩咐、在{\PN{迦南}}地把产业分给{\PN{以色列}}人的。

\par }\Chap{35}{\SH 分给利未人的城邑
\par }{\PP \VerseOne{1}耶和华在{\PN{摩押}}平原—{\PN{约旦河}}边、{\PN{耶利哥}}对面晓谕{\PN{摩西}}说:
\VS{2}「你吩咐{\PN{以色列}}人,要从所得为业的地中把些城给{\PN{利未}}人居住,也要把这城四围的郊野给{\PN{利未}}人。
\VS{3}这城邑要归他们居住,城邑的郊野可以牧养他们的牛羊和各样的牲畜,又可以安置他们的财物。
\VS{4}你们给{\PN{利未}}人的郊野,要从城根起,四围往外量一千肘。
\VS{5}另外东量二千肘,南量二千肘,西量二千肘,北量二千肘,为边界,城在当中;这要归他们作城邑的郊野。
\VS{6}你们给{\PN{利未}}人的城邑,其中当有六座逃城,使误杀人的可以逃到那里。此外还要给他们四十二座城。
\VS{7}你们要给{\PN{利未}}人的城,共有四十八座,连城带郊野{\ADD{都要给他们}}。
\VS{8}{\PN{以色列}}人所得的地业从中要把些城邑给{\ADD{
{\PN{利未}} 人}};人多的就多给,人少的就少给;各{\ADD{支派}}要按所承受为业之地把城邑给{\PN{利未}}人。」
\par }{\SH 逃城
\par }{\R (申19·1—13;书20·1—9)
\par }{\PP \VS{9}耶和华晓谕{\PN{摩西}}说:
\VS{10}「你吩咐{\PN{以色列}}人说:你们过{\PN{约旦河}},进了{\PN{迦南}}地,
\VS{11}就要分出几座城,为你们作逃城,使误杀人的可以逃到那里。
\VS{12}这些城可以作逃避报仇人的城,使误杀人的不至于死,等他站在会众面前听审判。
\VS{13}你们所分出来的城,要作六座逃城。
\VS{14}在{\PN{约旦河}}东要分出三座城,在{\PN{迦南}}地也要分出三座城,都作逃城。
\VS{15}这六座城要给{\PN{以色列}}人和他们中间的外人,并寄居的,作为逃城,使误杀人的都可以逃到那里。
\par }{\PP \VS{16}「倘若人用铁器打人,以致打死,他就是故杀人的;故杀人的必被治死。
\VS{17}若用可以打死人的石头打死了人,他就是故杀人的;故杀人的必被治死。
\VS{18}若用可以打死人的木器打死了人,他就是故杀人的;故杀人的必被治死。
\VS{19}报血仇的必亲自杀那故杀人的,一遇见就杀他。
\VS{20}人若因怨恨把人推倒,或是埋伏往人身上扔物,以致于死,
\VS{21}或是因仇恨用手打人,以致于死,那打人的必被治死。他是故杀人的;报血仇的一遇见就杀他。
\par }{\PP \VS{22}「倘若人没有仇恨,忽然将人推倒,或是没有埋伏把物扔在人身上,
\VS{23}或是没有看见的时候用可以打死人的石头扔在人身上,以致于死,本来与他无仇,也无意害他。
\VS{24}会众就要照典章,在打死人的和报血仇的中间审判。
\VS{25}会众要救这误杀人的脱离报血仇人的手,也要使他归入逃城。他要住在其中,直等到受圣膏的大祭司死了。
\VS{26}但误杀人的,无论什么时候,若出了逃城的境外,
\VS{27}报血仇的在逃城境外遇见他,将他杀了,报血仇的就没有流血之罪。
\VS{28}因为误杀人的该住在逃城里,等到大祭司死了。大祭司死了以后,误杀人的才可以回到他所得为业之地。
\VS{29}这在你们一切的住处,要作你们世世代代的律例典章。
\par }{\PP \VS{30}「无论谁故杀人,要凭几个见证人的口把那故杀人的杀了,只是不可凭一个见证的口叫人死。
\VS{31}故杀人、犯死罪的,你们不可收赎价代替他的命;他必被治死。
\VS{32}那逃到逃城的人,你们不可为他收赎价,使他在大祭司未死以先再来住在本地。
\VS{33}这样,你们就不污秽所住之地,因为血是污秽地的;若有在地上流人血的,非流那杀人者的血,那地就不得洁净\FTNT{}{{\FR 35:33: }原文是赎}。
\VS{34}你们不可玷污所住之地,就是我住在其中之地,因为我—耶和华住在{\PN{以色列}}人中间。」

\par }\Chap{36}{\SH 承业女子结婚的条例
\par }{\PP \VerseOne{1}{\PN{约瑟}}的后裔,{\PN{玛拿西}}的孙子,{\PN{玛吉}}的儿子{\PN{基列}},他子孙中的诸族长来到{\PN{摩西}}和作首领的{\PN{以色列}}人族长面前,说:
\VS{2}「耶和华曾吩咐我主拈阄分地给{\PN{以色列}}人为业,我主也受了耶和华的吩咐将我们兄弟{\PN{西罗非哈}}的产业分给他的众女儿。
\VS{3}她们若嫁{\PN{以色列}}{\ADD{别}}支派的人,就必将我们祖宗所遗留的产业加在她们丈夫支派的产业中。这样,我们拈阄所得的产业就要减少了。
\VS{4}到了{\PN{以色列}}人的禧年,这女儿的产业就必加在她们丈夫支派的产业上。这样,我们祖宗支派的产业就减少了。」
\par }{\PP \VS{5}{\PN{摩西}}照耶和华的话吩咐{\PN{以色列}}人说:「{\PN{约瑟}}支派的人说得有理。
\VS{6}论到{\PN{西罗非哈}}的众女儿,耶和华这样吩咐说:她们可以随意嫁人,只是要嫁同宗支派的人。
\VS{7}这样,{\PN{以色列}}人的产业就不从这支派归到那支派,因为{\PN{以色列}}人要各守各祖宗支派的产业。
\VS{8}凡在{\PN{以色列}}支派中得了产业的女子必作同宗支派人的妻,好叫{\PN{以色列}}人各自承受他祖宗的产业。
\VS{9}这样,他们的产业就不从这支派归到那支派,因为{\PN{以色列}}支派的人要各守各的产业。」
\par }{\PP \VS{10}耶和华怎样吩咐{\PN{摩西}},{\PN{西罗非哈}}的众女儿就怎样行。
\VS{11}{\PN{西罗非哈}}的女儿{\PN{玛拉}}、{\PN{得撒}}、{\PN{曷拉}}、{\PN{密迦}}、{\PN{挪阿}}都嫁了他们伯叔的儿子。
\VS{12}她们嫁入{\PN{约瑟}}儿子、{\PN{玛拿西}}子孙的族中;她们的产业仍留在同宗支派中。
\par }{\PP \VS{13}这是耶和华在{\PN{摩押}}平原—{\PN{约旦河}}边、{\PN{耶利哥}}对面—借着{\PN{摩西}}所吩咐{\PN{以色列}}人的命令典章。
\par }