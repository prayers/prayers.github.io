\NormalFont\ShortTitle{利未记}
{\MT 利未记

\par }\ChapOne{1}{\SH 燔祭的条例
\par }{\PP \VerseOne{1}耶和华从会幕中呼叫{\PN{摩西}},对他说:
\VS{2}「你晓谕{\PN{以色列}}人说:你们中间若有人献供物给耶和华,要从牛群羊群中献牲畜为供物。
\par }{\PP \VS{3}「他的供物若以牛为燔祭,就要在会幕门口献一只没有残疾的公牛,可以在耶和华面前蒙悦纳。
\VS{4}他要按手在燔祭牲的头上,燔祭便蒙悦纳,为他赎罪。
\VS{5}他要在耶和华面前宰公牛;{\PN{亚伦}}子孙作祭司的,要奉上血,把血洒在会幕门口、坛的周围。
\VS{6}那人要剥去燔祭牲的皮,把燔祭牲切成块子。
\VS{7}祭司{\PN{亚伦}}的子孙要把火放在坛上,把柴摆在火上。
\VS{8}{\PN{亚伦}}子孙作祭司的,要把肉块和头并脂油摆在坛上火的柴上。
\VS{9}但燔祭的脏腑与腿要用水洗。祭司就要把一切全烧在坛上,当作燔祭,献与耶和华为馨香的火祭。
\par }{\PP \VS{10}「人的供物若以绵羊或山羊为燔祭,就要献上没有残疾的公羊。
\VS{11}要把羊宰于坛的北边,在耶和华面前;{\PN{亚伦}}子孙作祭司的,要把羊血洒在坛的周围。
\VS{12}要把燔祭牲切成块子,连头和脂油,祭司就要摆在坛上火的柴上;
\VS{13}但脏腑与腿要用水洗,祭司就要全然奉献,烧在坛上。这是燔祭,是献与耶和华为馨香的火祭。
\par }{\PP \VS{14}「人奉给耶和华的供物,若以鸟为燔祭,就要献斑鸠或是雏鸽为供物。
\VS{15}祭司要把鸟拿到坛前,揪下头来,把鸟烧在坛上;鸟的血要流在坛的旁边;
\VS{16}又要把鸟的嗉子和脏物\FTNT{}{{\FR 1:16: }脏物:或译翎毛}除掉,丢在坛的东边{\ADD{倒}}灰的地方。
\VS{17}要拿着鸟的两个翅膀,把鸟撕开,只是不可撕断;祭司要在坛上、在火的柴上焚烧。这是燔祭,是献与耶和华为馨香的火祭。」

\par }\Chap{2}{\SH 素祭的条例
\par }{\PP \VerseOne{1}「若有人献素祭为供物给耶和华,要用细面浇上油,加上乳香,
\VS{2}带到{\PN{亚伦}}子孙作祭司的那里;祭司就要从细面中取出一把来,并取些油和所有的乳香,然后要把所取的这些作为纪念,烧在坛上,是献与耶和华为馨香的火祭。
\VS{3}素祭所剩的要归给{\PN{亚伦}}和他的子孙;这是献与耶和华的火祭中为至圣的。
\par }{\PP \VS{4}「若用炉中烤的物为素祭,就要用调油的无酵细面饼,或是抹油的无酵薄饼。
\VS{5}若用铁鏊上做的物为素祭,就要用调油的无酵细面,
\VS{6}分成块子,浇上油;这是素祭。
\VS{7}若用煎盘做的物为素祭,就要用油与细面做成。
\VS{8}要把这些东西做的素祭带到耶和华面前,并奉给祭司,带到坛前。
\VS{9}祭司要从素祭中取出作为纪念的,烧在坛上,是献与耶和华为馨香的火祭。
\VS{10}素祭所剩的要归给{\PN{亚伦}}和他的子孙。这是献与耶和华的火祭中为至圣的。
\par }{\PP \VS{11}「凡献给耶和华的素祭都不可有酵;因为你们不可烧一点酵、一点蜜当作火祭献给耶和华。
\VS{12}这些物要献给耶和华作为初熟的供物,只是不可在坛上献为馨香的祭。
\VS{13}凡献为素祭的供物都要用盐调和,在素祭上不可缺了你 神立约的盐。一切的供物都要配盐而献。
\VS{14}若向耶和华献初熟之物为素祭,要献上烘了的禾穗子,就是轧了的新穗子,当作初熟之物的素祭。
\VS{15}并要抹上油,加上乳香;这是素祭。
\VS{16}祭司要把其中作为纪念的,就是一些轧了的禾穗子和一些油,并所有的乳香,都焚烧,是向耶和华献的火祭。」

\par }\Chap{3}{\SH 平安祭的条例
\par }{\PP \VerseOne{1}「人献供物为平安祭\FTNT{}{{\FR 3:1: }平安:或译酬恩;下同},若是从牛群中献,无论是公的是母的,必用没有残疾的献在耶和华面前。
\VS{2}他要按手在供物的头上,宰于会幕门口。{\PN{亚伦}}子孙作祭司的,要把血洒在坛的周围。
\VS{3}从平安祭中,将火祭献给耶和华,也要把盖脏的脂油和脏上所有的脂油,
\VS{4}并两个腰子和腰子上的脂油,就是靠腰两旁的脂油,与肝上的网子和腰子,一概取下。
\VS{5}{\PN{亚伦}}的子孙要把这些烧在坛的燔祭上,就是在火的柴上,是献与耶和华为馨香的火祭。
\par }{\PP \VS{6}「人向耶和华献供物为平安祭,若是从羊群中献,无论是公的是母的,必用没有残疾的。
\VS{7}若献一只羊羔为供物,必在耶和华面前献上,
\VS{8}并要按手在供物的头上,宰于会幕前。{\PN{亚伦}}的子孙要把血洒在坛的周围。
\VS{9}从平安祭中,将火祭献给耶和华,其中的脂油和整肥尾巴都要在靠近脊骨处取下,并要把盖脏的脂油和脏上所有的脂油,
\VS{10}两个腰子和腰子上的脂油,就是靠腰两旁的脂油,并肝上的网子和腰子,一概取下。
\VS{11}祭司要在坛上焚烧,是献给耶和华为食物的火祭。
\par }{\PP \VS{12}「人的供物若是山羊,必在耶和华面前献上。
\VS{13}要按手在山羊头上,宰于会幕前。{\PN{亚伦}}的子孙要把血洒在坛的周围,
\VS{14-15}又把盖脏的脂油和脏上所有的脂油,两个腰子和腰子上的脂油,就是靠腰两旁的脂油,并肝上的网子和腰子,一概取下,献给耶和华为火祭。
\VS{16}祭司要在坛上焚烧,作为馨香火祭的食物。脂油都是耶和华的。
\VS{17}在你们一切的住处,脂油和血都不可吃;这要成为你们世世代代永远的定例。」

\par }\Chap{4}{\SH 为误犯的罪献祭的条例
\par }{\PP \VerseOne{1}耶和华对{\PN{摩西}}说:
\VS{2}「你晓谕{\PN{以色列}}人说:若有人在耶和华所吩咐不可行的什么事上误犯了一件,
\VS{3}或是受膏的祭司犯罪,使百姓陷在罪里,就当为他所犯的罪把没有残疾的公牛犊献给耶和华为赎罪祭。
\VS{4}他要牵公牛到会幕门口,在耶和华面前按手在牛的头上,把牛宰于耶和华面前。
\VS{5}受膏的祭司要取些公牛的血带到会幕,
\VS{6}把指头蘸于血中,在耶和华面前对着圣所的幔子弹血七次,
\VS{7}又要把些血抹在会幕内、耶和华面前香坛的四角上,再把公牛所有的血倒在会幕门口、燔祭坛的脚那里。
\VS{8}要把赎罪祭公牛所有的脂油,乃是盖脏的脂油和脏上所有的脂油,
\VS{9}并两个腰子和腰子上的脂油,就是靠腰两旁的脂油,与肝上的网子和腰子,一概取下,
\VS{10}与平安祭公牛上所取的一样;祭司要把这些烧在燔祭的坛上。
\VS{11}公牛的皮和所有的肉,并头、腿、脏、腑、粪,
\VS{12}就是全公牛,要搬到营外洁净之地、倒灰之所,用火烧在柴上。
\par }{\PP \VS{13}「{\PN{以色列}}全会众若行了耶和华所吩咐不可行的什么事,误犯了罪,是隐而未现、会众看不出来的,
\VS{14}会众一知道所犯的罪就要献一只公牛犊为赎罪祭,牵到会幕前。
\VS{15}会中的长老就要在耶和华面前按手在牛的头上,将牛在耶和华面前宰了。
\VS{16}受膏的祭司要取些公牛的血带到会幕,
\VS{17}把指头蘸于血中,在耶和华面前对着幔子弹血七次,
\VS{18}又要把些血抹在会幕内、耶和华面前坛的四角上,再把所有的血倒在会幕门口、燔祭坛的脚那里。
\VS{19}把牛所有的脂油都取下,烧在坛上;
\VS{20}收拾这牛,与那赎罪祭的牛一样。祭司要为他们赎罪,他们必蒙赦免。
\VS{21}他要把牛搬到营外烧了,像烧头一个牛一样;这是会众的赎罪祭。
\par }{\PP \VS{22}「官长若行了耶和华—他 神所吩咐不可行的什么事,误犯了罪,
\VS{23}所犯的罪自己知道了,就要牵一只没有残疾的公山羊为供物,
\VS{24}按手在羊的头上,宰于耶和华面前、宰燔祭牲的地方;这是赎罪祭。
\VS{25}祭司要用指头蘸些赎罪祭牲的血,抹在燔祭坛的四角上,把血倒在燔祭坛的脚那里。
\VS{26}所有的脂油,祭司都要烧在坛上,正如平安祭的脂油一样。至于他的罪,祭司要为他赎了,他必蒙赦免。
\par }{\PP \VS{27}「民中若有人行了耶和华所吩咐不可行的什么事,误犯了罪,
\VS{28}所犯的罪自己知道了,就要为所犯的罪牵一只没有残疾的母山羊为供物,
\VS{29}按手在赎罪祭牲的头上,在那宰燔祭牲的地方宰了。
\VS{30}祭司要用指头蘸些羊的血,抹在燔祭坛的四角上,所有的血都要倒在坛的脚那里,
\VS{31}又要把羊所有的脂油都取下,正如取平安祭牲的脂油一样。祭司要在坛上焚烧,在耶和华面前作为馨香{\ADD{的祭}},为他赎罪,他必蒙赦免。
\par }{\PP \VS{32}「人若牵一只绵羊羔为赎罪祭的供物,必要牵一只没有残疾的母羊,
\VS{33}按手在赎罪祭牲的头上,在那宰燔祭牲的地方宰了作赎罪祭。
\VS{34}祭司要用指头蘸些赎罪祭牲的血,抹在燔祭坛的四角上,所有的血都要倒在坛的脚那里,
\VS{35}又要把所有的脂油都取下,正如取平安祭羊羔的脂油一样。祭司要按献给耶和华火祭的条例,烧在坛上。至于所犯的罪,祭司要为他赎了,他必蒙赦免。」

\par }\Chap{5}{\SH 赎愆祭的条例
\par }{\PP \VerseOne{1}「若有人听见发誓的声音\FTNT{}{{\FR 5:1: }或译:若有人听见叫人发誓的声音},他本是见证,却不把所看见的、所知道的说出来,这就是罪;他要担当他的罪孽。
\VS{2}或是有人摸了不洁的物,无论是不洁的死兽,是不洁的死畜,是不洁的死虫,他却不知道,因此成了不洁,就有了罪。
\VS{3}或是他摸了别人的污秽,无论是染了什么污秽,他却不知道,一知道了就有了罪。
\VS{4}或是有人嘴里冒失发誓,要行恶,要行善,无论人在什么事上冒失发誓,他却不知道,一知道了就要在这其中的一件上有了罪。
\VS{5}他有了罪的时候,就要承认所犯的罪,
\VS{6}并要因所犯的罪,把他的赎愆祭牲—就是羊群中的母羊,或是一只羊羔,或是一只山羊—牵到耶和华面前为赎罪祭。至于他的罪,祭司要为他赎了。
\par }{\PP \VS{7}「他的力量若不够献一只羊羔,就要因所犯的罪,把两只斑鸠或是两只雏鸽带到耶和华面前为赎愆祭:一只作赎罪祭,一只作燔祭。
\VS{8}把这些带到祭司那里,祭司就要先把那赎罪祭献上,从鸟的颈项上揪下头来,只是不可把鸟撕断,
\VS{9}也把些赎罪祭牲的血弹在坛的旁边,剩下的血要流在坛的脚那里;这是赎罪祭。
\VS{10}他要照例献第二只为燔祭。至于他所犯的罪,祭司要为他赎了,他必蒙赦免。
\par }{\PP \VS{11}「他的力量若不够献两只斑鸠或是两只雏鸽,就要因所犯的罪带供物来,就是细面伊法十分之一为赎罪祭;不可加上油,也不可加上乳香,因为是赎罪祭。
\VS{12}他要把供物带到祭司那里,祭司要取出自己的一把来作为纪念,按献给耶和华火祭的条例烧在坛上;这是赎罪祭。
\VS{13}至于他在这几件事中所犯的罪,祭司要为他赎了,他必蒙赦免。{\ADD{剩下的面}}都归与祭司,和素祭一样。」
\par }{\PP \VS{14}耶和华晓谕{\PN{摩西}}说:
\VS{15}「人若在耶和华的圣物上误犯了罪,有了过犯,就要照你所估的,按圣所的舍客勒拿银子,将赎愆祭牲—就是羊群中一只没有残疾的公绵羊—牵到耶和华面前为赎愆祭;
\VS{16}并且他因在圣物上的差错要偿还,另外加五分之一,都给祭司。祭司要用赎愆祭的公绵羊为他赎罪,他必蒙赦免。
\par }{\PP \VS{17}「若有人犯罪,行了耶和华所吩咐不可行的什么事,他虽然不知道,还是有了罪,就要担当他的罪孽;
\VS{18}也要照你所估定的价,从羊群中牵一只没有残疾的公绵羊来,给祭司作赎愆祭。至于他误行的那错事,祭司要为他赎罪,他必蒙赦免。
\VS{19}这是赎愆祭,因他在耶和华面前实在有了罪。」

\par }\Chap{6}{\PP \VerseOne{1}耶和华晓谕{\PN{摩西}}说:
\VS{2}「若有人犯罪,干犯耶和华,在邻舍交付他的物上,或是在交易上行了诡诈,或是抢夺人的财物,或是欺压邻舍,
\VS{3}或是在捡了遗失的物上行了诡诈,说谎起誓,在这一切的事上犯了什么罪;
\VS{4}他既犯了罪,有了过犯,就要归还他所抢夺的,或是因欺压所得的,或是人交付他的,或是人遗失他所捡的物,
\VS{5}或是他因什么物起了假誓,就要如数归还,另外加上五分之一,在查出他有罪的日子要交还本主。
\VS{6}也要照你所估定的价,把赎愆祭牲—就是羊群中一只没有残疾的公绵羊—牵到耶和华面前,给祭司为赎愆祭。
\VS{7}祭司要在耶和华面前为他赎罪;他无论行了什么事,使他有了罪,都必蒙赦免。」
\par }{\SH 祭司献燔祭的责任
\par }{\PP \VS{8}耶和华晓谕{\PN{摩西}}说:
\VS{9}「你要吩咐{\PN{亚伦}}和他的子孙说,燔祭的条例乃是这样:燔祭要放在坛的柴上,从晚上到天亮,坛上的火要常常烧着。
\VS{10}祭司要穿上细麻布衣服,又要把细麻布裤子穿在身上,把坛上所烧的燔祭灰收起来,倒在坛的旁边;
\VS{11}随后要脱去这衣服,穿上别的衣服,把灰拿到营外洁净之处。
\VS{12}坛上的火要在其上常常烧着,不可熄灭。祭司要每日早晨在上面烧柴,并要把燔祭摆在坛上,在其上烧平安祭牲的脂油。
\VS{13}在坛上必有常常烧着的火,不可熄灭。」
\par }{\SH 祭司献素祭的责任
\par }{\PP \VS{14}「素祭的条例乃是这样:{\PN{亚伦}}的子孙要在坛前把这祭献在耶和华面前。
\VS{15}祭司要从其中—就是从素祭的细面中—取出自己的一把,又要取些油和素祭上所有的乳香,烧在坛上,奉给耶和华为馨香素祭的纪念。
\VS{16}所剩下的,{\PN{亚伦}}和他子孙要吃,必在圣处不带酵而吃,要在会幕的院子里吃。
\VS{17}{\ADD{烤的时候}}不可搀酵。这是从所献给我的火祭中赐给他们的分,是至圣的,和赎罪祭并赎愆祭一样。
\VS{18}凡献给耶和华的火祭,{\PN{亚伦}}子孙中的男丁都要吃这一分,直到万代,作他们永得的分。摸这些祭物的,都要成为圣。」
\par }{\PP \VS{19}耶和华晓谕{\PN{摩西}}说:
\VS{20}「当{\PN{亚伦}}受膏的日子,他和他子孙所要献给耶和华的供物,就是细面伊法十分之一,为常献的素祭:早晨一半,晚上一半。
\VS{21}要在铁鏊上用油调和做成,调匀了,你就拿进来;烤好了分成块子,献给耶和华为馨香的素祭。
\VS{22}{\PN{亚伦}}的子孙中,接续他为受膏的祭司,要把这素祭献上,要全烧给耶和华。这是永远的定例。
\VS{23}祭司的素祭都要烧了,却不可吃。」
\par }{\SH 祭司献赎罪祭的责任
\par }{\PP \VS{24}耶和华晓谕{\PN{摩西}}说:
\VS{25}「你对{\PN{亚伦}}和他的子孙说,赎罪祭的条例乃是这样:要在耶和华面前、宰燔祭牲的地方宰赎罪祭牲;这是至圣的。
\VS{26}为赎罪献这祭的祭司要吃,要在圣处,就是在会幕的院子里吃。
\VS{27}凡摸这祭肉的要成为圣;这祭牲的血若弹在什么衣服上,所弹的那一件要在圣处洗净。
\VS{28}惟有煮祭物的瓦器要打碎;若是煮在铜器里,这铜器要擦磨,在水中涮净。
\VS{29}凡祭司中的男丁都可以吃;这是至圣的。
\VS{30}凡赎罪祭,若将血带进会幕在圣所赎罪,那肉都不可吃,必用火焚烧。」

\par }\Chap{7}{\SH 祭司献赎愆祭的责任
\par }{\PP \VerseOne{1}「赎愆祭的条例乃是如此:这祭是至圣的。
\VS{2}人在那里宰燔祭牲,也要在那里宰赎愆祭牲;其血,祭司要洒在坛的周围。
\VS{3}又要将肥尾巴和盖脏的脂油,
\VS{4}两个腰子和腰子上的脂油,就是靠腰两旁的脂油,并肝上的网子和腰子,一概取下。
\VS{5}祭司要在坛上焚烧,为献给耶和华的火祭,是赎愆祭。
\VS{6}祭司中的男丁都可以吃这祭物;要在圣处吃,是至圣的。
\VS{7}赎罪祭怎样,赎愆祭也是怎样,两个祭是一个条例。献赎愆祭赎罪的祭司要得这祭物。
\VS{8}献燔祭的祭司,无论为谁奉献,要亲自得他所献那燔祭牲的皮。
\VS{9}凡在炉中烤的素祭和煎盘中做的,并铁鏊上做的,都要归那献祭的祭司。
\VS{10}凡素祭,无论是油调和的是干的,都要归{\PN{亚伦}}的子孙,大家均分。」
\par }{\SH 祭司献平安祭的责任
\par }{\PP \VS{11}「人献与耶和华平安祭的条例乃是这样:
\VS{12}他若为感谢献上,就要用调油的无酵饼和抹油的无酵薄饼,并用油调匀细面做的饼,与感谢祭一同献上。
\VS{13}要用有酵的饼和为感谢献的平安祭,与供物一同献上。
\VS{14}从各样的供物中,他要把一个饼献给耶和华为举祭,是要归给洒平安祭牲血的祭司。
\VS{15}为感谢献平安祭牲的肉,要在献的日子吃,一点不可留到早晨。
\VS{16}若所献的是为还愿,或是甘心献的,必在献祭的日子吃,所剩下的第二天也可以吃。
\VS{17}但所剩下的祭肉,到第三天要用火焚烧;
\VS{18}第三天若吃了平安祭的肉,这祭必不蒙悦纳,人所献的也不算为祭,反为可憎嫌的,吃这祭肉的,就必担当他的罪孽。
\par }{\PP \VS{19}「挨了污秽物的肉就不可吃,要用火焚烧。至于平安祭的肉,凡洁净的人都要吃;
\VS{20}只是献与耶和华平安祭的肉,人若不洁净而吃了,这人必从民中剪除。
\VS{21}有人摸了什么不洁净的物,或是人的不洁净,或是不洁净的牲畜,或是不洁可憎之物,吃了献与耶和华平安祭的肉,这人必从民中剪除。」
\par }{\PP \VS{22}耶和华对{\PN{摩西}}说:
\VS{23}「你晓谕{\PN{以色列}}人说:牛的脂油、绵羊的脂油、山羊的脂油,你们都不可吃。
\VS{24}自死的和被野兽撕裂的,那脂油可以做别的使用,只是你们万不可吃。
\VS{25}无论何人吃了献给耶和华当火祭牲畜的脂油,那人必从民中剪除。
\VS{26}在你们一切的住处,无论是雀鸟的血是野兽的血,你们都不可吃。
\VS{27}无论是谁吃血,那人必从民中剪除。」
\par }{\PP \VS{28}耶和华对{\PN{摩西}}说:
\VS{29}「你晓谕{\PN{以色列}}人说:献平安祭给耶和华的,要从平安祭中取些来奉给耶和华。
\VS{30}他亲手献给耶和华的火祭,就是脂油和胸,要带来,好把胸在耶和华面前作摇祭,摇一摇。
\VS{31}祭司要把脂油在坛上焚烧,但胸要归{\PN{亚伦}}和他的子孙。
\VS{32}你们要从平安祭中把右腿作举祭,奉给祭司。
\VS{33}{\PN{亚伦}}子孙中,献平安祭牲血和脂油的,要得这右腿为分;
\VS{34}因为我从{\PN{以色列}}人的平安祭中,取了这摇的胸和举的腿给祭司{\PN{亚伦}}和他子孙,作他们从{\PN{以色列}}人中所永得的分。」
\VS{35}这是从耶和华火祭中,作{\PN{亚伦}}受膏的分和他子孙受膏的分,正在{\PN{摩西}}\FTNT{}{{\FR 7:35: }原文是他}叫他们前来给耶和华供祭司职分的日子,
\VS{36}就是在{\PN{摩西}}\FTNT{}{{\FR 7:36: }原文是他}膏他们的日子,耶和华吩咐{\PN{以色列}}人给他们的。这是他们世世代代永得的分。
\par }{\PP \VS{37}这就是燔祭、素祭、赎罪祭、赎愆祭,和平安祭的条例,并承接圣职的礼,
\VS{38}都是耶和华在{\PN{西奈山}}所吩咐{\PN{摩西}}的,就是他在{\PN{西奈}}旷野吩咐{\PN{以色列}}人献供物给耶和华之日所说的。

\par }\Chap{8}{\SH 立亚伦和他的儿子们作祭司
\par }{\R (出29·1—37)
\par }{\PP \VerseOne{1}耶和华晓谕{\PN{摩西}}说:
\VS{2}「你将{\PN{亚伦}}和他儿子一同带来,并将圣衣、膏油,与赎罪祭的一只公牛、两只公绵羊、一筐无酵饼都带来,
\VS{3}又招聚会众到会幕门口。」
\VS{4}{\PN{摩西}}就照耶和华所吩咐的行了;于是会众聚集在会幕门口。
\VS{5}{\PN{摩西}}告诉会众说:「这就是耶和华所吩咐当行的事。」
\VS{6}{\PN{摩西}}带了{\PN{亚伦}}和他儿子来,用水洗了他们。
\VS{7}给{\PN{亚伦}}穿上内袍,束上腰带,穿上外袍,又加上以弗得,用其上巧工织的带子把以弗得系在他身上,
\VS{8}又给他戴上胸牌,把乌陵和土明放在胸牌内,
\VS{9}把冠冕戴在他头上,在冠冕的前面钉上金牌,就是圣冠,都是照耶和华所吩咐{\PN{摩西}}的。
\par }{\PP \VS{10}{\PN{摩西}}用膏油抹帐幕和其中所有的,使它成圣;
\VS{11}又用膏油在坛上弹了七次,又抹了坛和坛的一切器皿,并洗濯盆和盆座,使它成圣;
\VS{12}又把膏油倒在{\PN{亚伦}}的头上膏他,使他成圣。
\VS{13}{\PN{摩西}}带了{\PN{亚伦}}的儿子来,给他们穿上内袍,束上腰带,包上裹头巾,都是照耶和华所吩咐{\PN{摩西}}的。
\par }{\PP \VS{14}他牵了赎罪祭的公牛来,{\PN{亚伦}}和他儿子按手在赎罪祭公牛的头上,
\VS{15}就宰了公牛。{\PN{摩西}}用指头蘸血,抹在坛上四角的周围,使坛洁净,把血倒在坛的脚那里,使坛成圣,坛就洁净了;
\VS{16}又取脏上所有的脂油和肝上的网子,并两个腰子与腰子上的脂油,都烧在坛上;
\VS{17}惟有公牛,连皮带肉并粪,用火烧在营外,都是照耶和华所吩咐{\PN{摩西}}的。
\par }{\PP \VS{18}他奉上燔祭的公绵羊;{\PN{亚伦}}和他儿子按手在羊的头上,
\VS{19}就宰了公羊。{\PN{摩西}}把血洒在坛的周围,
\VS{20}把羊切成块子,把头和肉块并脂油都烧了。
\VS{21}用水洗了脏腑和腿,就把全羊烧在坛上为馨香的燔祭,是献给耶和华的火祭,都是照耶和华所吩咐{\PN{摩西}}的。
\par }{\PP \VS{22}他又奉上第二只公绵羊,就是承接圣职之礼的羊;{\PN{亚伦}}和他儿子按手在羊的头上,
\VS{23}就宰了羊。{\PN{摩西}}把些血抹在{\PN{亚伦}}的右耳垂上和右手的大拇指上,并右脚的大拇指上,
\VS{24}又带了{\PN{亚伦}}的儿子来,把些血抹在他们的右耳垂上和右手的大拇指上,并右脚的大拇指上,又把血洒在坛的周围。
\VS{25}取脂油和肥尾巴,并脏上一切的脂油与肝上的网子,两个腰子和腰子上的脂油,并右腿,
\VS{26}再从耶和华面前、盛无酵饼的筐子里取出一个无酵饼,一个油饼,一个薄饼,都放在脂油和右腿上,
\VS{27}把这一切放在{\PN{亚伦}}的手上和他儿子的手上作摇祭,在耶和华面前摇一摇。
\VS{28}{\PN{摩西}}从他们的手上拿下来,烧在坛上的燔祭上,都是为承接圣职献给耶和华馨香的火祭。
\VS{29}{\PN{摩西}}拿羊的胸作为摇祭,在耶和华面前摇一摇,是承接圣职之礼,归{\PN{摩西}}的分,都是照耶和华所吩咐{\PN{摩西}}的。
\par }{\PP \VS{30}{\PN{摩西}}取点膏油和坛上的血,弹在{\PN{亚伦}}和他的衣服上,并他儿子和他儿子的衣服上,使他和他们的衣服一同成圣。
\par }{\PP \VS{31}{\PN{摩西}}对{\PN{亚伦}}和他儿子说:「把肉煮在会幕门口,在那里吃,又吃承接圣职筐子里的饼,按我所吩咐的说\FTNT{}{{\FR 8:31: }或译:按所吩咐我的说}:『这是{\PN{亚伦}}和他儿子要吃的。』
\VS{32}剩下的肉和饼,你们要用火焚烧。
\VS{33}你们七天不可出会幕的门,等到你们承接圣职的日子满了,因为主叫你们七天承接圣职。
\VS{34}像今天所行的都是耶和华吩咐行的,为你们赎罪。
\VS{35}七天你们要昼夜住在会幕门口,遵守耶和华的吩咐,免得你们死亡,因为所吩咐我的就是这样。」
\VS{36}于是{\PN{亚伦}}和他儿子行了耶和华借着{\PN{摩西}}所吩咐的一切事。

\par }\Chap{9}{\SH 亚伦献祭
\par }{\PP \VerseOne{1}到了第八天,{\PN{摩西}}召了{\PN{亚伦}}和他儿子,并{\PN{以色列}}的众长老来,
\VS{2}对{\PN{亚伦}}说:「你当取牛群中的一只公牛犊作赎罪祭,一只公绵羊作燔祭,都要没有残疾的,献在耶和华面前。
\VS{3}你也要对{\PN{以色列}}人说:『你们当取一只公山羊作赎罪祭,又取一只牛犊和一只绵羊羔,都要一岁、没有残疾的,作燔祭,
\VS{4}又取一只公牛,一只公绵羊作平安祭,献在耶和华面前,并取调油的素祭,因为今天耶和华要向你们显现。』」
\VS{5}于是他们把{\PN{摩西}}所吩咐的,带到会幕前;全会众都近前来,站在耶和华面前。
\VS{6}{\PN{摩西}}说:「这是耶和华吩咐你们所当行的;耶和华的荣光就要向你们显现。」
\VS{7}{\PN{摩西}}对{\PN{亚伦}}说:「你就近坛前,献你的赎罪祭和燔祭,为自己与百姓赎罪,又献上百姓的供物,为他们赎罪,都照耶和华所吩咐的。」
\par }{\PP \VS{8}于是,{\PN{亚伦}}就近坛前,宰了为自己作赎罪祭的牛犊。
\VS{9}{\PN{亚伦}}的儿子把血奉给他,他就把指头蘸在血中,抹在坛的四角上,又把血倒在坛脚那里。
\VS{10}惟有赎罪祭的脂油和腰子,并肝上取的网子,都烧在坛上,是照耶和华所吩咐{\PN{摩西}}的;
\VS{11}又用火将肉和皮烧在营外。
\par }{\PP \VS{12}{\PN{亚伦}}宰了燔祭牲,他儿子把血递给他,他就洒在坛的周围,
\VS{13}又把燔祭一块一块地、连头递给他,他都烧在坛上;
\VS{14}又洗了脏腑和腿,烧在坛上的燔祭上。
\par }{\PP \VS{15}他奉上百姓的供物,把那给百姓作赎罪祭的公山羊宰了,为罪献上,和先献的一样;
\VS{16}也奉上燔祭,照例而献。
\VS{17}他又奉上素祭,从其中取一满把,烧在坛上;这是在早晨的燔祭以外。
\VS{18}{\PN{亚伦}}宰了那给百姓作平安祭的公牛和公绵羊。他儿子把血递给他,他就洒在坛的周围;
\VS{19}又把公牛和公绵羊的脂油、肥尾巴,并盖{\ADD{脏}}的脂油与腰子,和肝上的网子,都递给他;
\VS{20}把脂油放在胸上,他就把脂油烧在坛上。
\VS{21}胸和右腿,{\PN{亚伦}}当作摇祭,在耶和华面前摇一摇,都是照{\PN{摩西}}所吩咐的。
\par }{\PP \VS{22}{\PN{亚伦}}向百姓举手,为他们祝福。他献了赎罪祭、燔祭、平安祭就下来了。
\VS{23}{\PN{摩西}}、{\PN{亚伦}}进入会幕,又出来为百姓祝福,耶和华的荣光就向众民显现。
\VS{24}有火从耶和华面前出来,在坛上烧尽燔祭和脂油;众民一见,就都欢呼,俯伏在地。

\par }\Chap{10}{\SH 拿答和亚比户犯罪
\par }{\PP \VerseOne{1}{\PN{亚伦}}的儿子{\PN{拿答}}、{\PN{亚比户}}各拿自己的香炉,盛上火,加上香,在耶和华面前献上凡火,是耶和华没有吩咐他们的,
\VS{2}就有火从耶和华面前出来,把他们烧灭,他们就死在耶和华面前。
\VS{3}于是{\PN{摩西}}对{\PN{亚伦}}说:「这就是耶和华所说:『我在亲近我的人中要显为圣;在众民面前,我要得荣耀。』」{\PN{亚伦}}就默默不言。
\par }{\PP \VS{4}{\PN{摩西}}召了{\PN{亚伦}}叔父{\PN{乌薛}}的儿子{\PN{米沙利}}、{\PN{以利撒反}}来,对他们说:「上前来,把你们的亲属从圣所前抬到营外。」
\VS{5}于是二人上前来,把他们穿着袍子抬到营外,是照{\PN{摩西}}所吩咐的。
\VS{6}{\PN{摩西}}对{\PN{亚伦}}和他儿子{\PN{以利亚撒}}、{\PN{以他玛}}说:「不可蓬头散发,也不可撕裂衣裳,免得你们死亡,又免得耶和华向会众发怒;只要你们的弟兄{\PN{以色列}}全家为耶和华所发的火哀哭。
\VS{7}你们也不可出会幕的门,恐怕你们死亡,因为耶和华的膏油在你们的身上。」他们就照{\PN{摩西}}的话行了。
\par }{\SH 有关祭司的条例
\par }{\PP \VS{8}耶和华晓谕{\PN{亚伦}}说:
\VS{9}「你和你儿子进会幕的时候,清酒、浓酒都不可喝,免得你们死亡;这要作你们世世代代永远的定例。
\VS{10}使你们可以将圣的、俗的,洁净的、不洁净的,分别出来;
\VS{11}又使你们可以将耶和华借{\PN{摩西}}晓谕{\PN{以色列}}人的一切律例教训他们。」
\par }{\PP \VS{12}{\PN{摩西}}对{\PN{亚伦}}和他剩下的儿子{\PN{以利亚撒}}、{\PN{以他玛}}说:「你们献给耶和华火祭中所剩的素祭,要在坛旁不带酵而吃,因为是至圣的。
\VS{13}你们要在圣处吃;因为在献给耶和华的火祭中,这是你的分和你儿子的分;所吩咐我的本是这样。
\VS{14}所摇的胸,所举的腿,你们要在洁净地方吃。你和你的儿女都要同吃;因为这些是从{\PN{以色列}}人平安祭中给你,当你的分和你儿子的分。
\VS{15}所举的腿,所摇的胸,他们要与火祭的脂油一同带来当摇祭,在耶和华面前摇一摇;这要归你和你儿子,当作永得的分,都是照耶和华所吩咐的。」
\par }{\PP \VS{16}当下{\PN{摩西}}急切地寻找作赎罪祭的公山羊,谁知已经焚烧了,便向{\PN{亚伦}}剩下的儿子{\PN{以利亚撒}}、{\PN{以他玛}}发怒,说:
\VS{17}「这赎罪祭既是至圣的,主又给了你们,为要你们担当会众的罪孽,在耶和华面前为他们赎罪,你们为何没有在圣所吃呢?
\VS{18}看哪,这祭牲的血并没有拿到圣所里去,你们本当照我所吩咐的,在圣所里吃这祭肉。」
\VS{19}{\PN{亚伦}}对{\PN{摩西}}说:「今天他们在耶和华面前献上赎罪祭和燔祭,我又遇见这样的灾,若今天吃了赎罪祭,耶和华岂能看为美呢?」
\VS{20}{\PN{摩西}}听见{\ADD{这话}},便以为美。

\par }\Chap{11}{\SH 有关食物的条例
\par }{\R (申14·3—21)
\par }{\PP \VerseOne{1}耶和华对{\PN{摩西}}、{\PN{亚伦}}说:
\VS{2}「你们晓谕{\PN{以色列}}人说,在地上一切走兽中可吃的乃是这些:
\VS{3}凡蹄分两瓣、倒嚼的走兽,你们都可以吃。
\VS{4}但那倒嚼或分蹄之中不可吃的乃是:骆驼—因为倒嚼不分蹄,就与你们不洁净;
\VS{5}沙番—因为倒嚼不分蹄,就与你们不洁净;
\VS{6}兔子—因为倒嚼不分蹄,就与你们不洁净;
\VS{7}猪—因为蹄分两瓣,却不倒嚼,就与你们不洁净。
\VS{8}这些兽的肉,你们不可吃;死的,你们不可摸,都与你们不洁净。
\par }{\PP \VS{9}「水中可吃的乃是这些:凡在水里、海里、河里、有翅有鳞的,都可以吃。
\VS{10}凡在海里、河里,并一切水里游动的活物,无翅无鳞的,你们都当以为可憎。
\VS{11}这些无翅无鳞、以为可憎的,你们不可吃它的肉;死的也当以为可憎。
\VS{12}凡水里无翅无鳞的,你们都当以为可憎。
\par }{\PP \VS{13}「雀鸟中你们当以为可憎、不可吃的乃是:雕、狗头雕、红头雕、
\VS{14}鹞鹰、小鹰与其类;
\VS{15}乌鸦与其类;
\VS{16}鸵鸟、夜鹰、鱼鹰、鹰与其类;
\VS{17}鸮鸟、鸬鹚、猫头鹰、
\VS{18}角鸱、鹈鹕、秃雕、
\VS{19}鹳、鹭鸶与其类;戴 与蝙蝠。
\par }{\PP \VS{20}「凡有翅膀用四足爬行的物,你们都当以为可憎。
\VS{21}只是有翅膀用四足爬行的物中,有足有腿,在地上蹦跳的,你们还可以吃。
\VS{22}其中有蝗虫、蚂蚱、蟋蟀与其类;蚱蜢与其类;这些你们都可以吃。
\VS{23}但是有翅膀有四足的爬物,你们都当以为可憎。
\par }{\PP \VS{24}「这些都能使你们不洁净。凡摸了死的,必不洁净到晚上。
\VS{25}凡拿了死的,必不洁净到晚上,并要洗衣服。
\VS{26}凡走兽分蹄不成两瓣、也不倒嚼的,是与你们不洁净;凡摸了的就不洁净。
\VS{27}凡四足的走兽,用掌行走的,是与你们不洁净;摸其尸的,必不洁净到晚上。
\VS{28}拿其尸的,必不洁净到晚上,并要洗衣服。这些是与你们不洁净的。
\par }{\PP \VS{29}「地上爬物与你们不洁净的乃是这些:鼬鼠、鼫鼠、蜥蜴与其类;
\VS{30}壁虎、龙子、守宫、蛇医、蝘蜓。
\VS{31}这些爬物都是与你们不洁净的。在它死了以后,凡摸了的,必不洁净到晚上。
\VS{32}其中死了的,掉在什么东西上,这东西就不洁净,无论是木器、衣服、皮子、口袋,不拘是做什么工用的器皿,须要放在水中,必不洁净到晚上,到晚上才洁净了。
\VS{33}若有死了掉在瓦器里的,其中不拘有什么,就不洁净,你们要把这瓦器打破了。
\VS{34}其中一切可吃的食物,沾水的就不洁净,并且那样器皿中一切可喝的,也必不洁净。
\VS{35}其中已死的,若有一点掉在什么物件上,那物件就不洁净,不拘是炉子,是锅台,就要打碎,都不洁净,也必与你们不洁净。
\VS{36}但是泉源或是聚水的池子仍是洁净;惟挨了那死的,就不洁净。
\VS{37}若是死的,有一点掉在要种的子粒上,子粒仍是洁净;
\VS{38}若水已经浇在子粒上,那死的有一点掉在上头,这子粒就与你们不洁净。
\par }{\PP \VS{39}「你们可吃的走兽若是死了,有人摸它,必不洁净到晚上;
\VS{40}有人吃那死了的走兽,必不洁净到晚上,并要洗衣服;拿了死走兽的,必不洁净到晚上,并要洗衣服。
\par }{\PP \VS{41}「凡地上的爬物是可憎的,都不可吃。
\VS{42}凡用肚子行走的和用四足行走的,或是有许多足的,就是一切爬在地上的,你们都不可吃,因为是可憎的。
\VS{43}你们不可因什么爬物使自己成为可憎的,也不可因这些使自己不洁净,以致染了污秽。
\VS{44}我是耶和华—你们的 神;所以你们要成为圣洁,因为我是圣洁的。你们也不可在地上的爬物污秽自己。
\VS{45}我是把你们从{\PN{埃及}}地领出来的耶和华,要作你们的 神;所以你们要圣洁,因为我是圣洁的。」
\par }{\PP \VS{46}这是走兽、飞鸟,和水中游动的活物,并地上爬物的条例。
\VS{47}要把洁净的和不洁净的,可吃的与不可吃的活物,都分别出来。

\par }\Chap{12}{\SH 产妇洁净礼的条例
\par }{\PP \VerseOne{1}耶和华对{\PN{摩西}}说:
\VS{2}「你晓谕{\PN{以色列}}人说:若有妇人怀孕生男孩,她就不洁净七天,像在月经污秽的日子不洁净一样。
\VS{3}第八天,要给婴孩行割礼。
\VS{4}妇人在产血不洁之中,要家居三十三天。她洁净的日子未满,不可摸圣物,也不可进入圣所。
\VS{5}她若生女孩,就不洁净两个七天,像污秽{\ADD{的时候}}一样,要在产血不洁之中,家居六十六天。
\par }{\PP \VS{6}「满了洁净的日子,无论是为男孩是为女孩,她要把一岁的羊羔为燔祭,一只雏鸽或是一只斑鸠为赎罪祭,带到会幕门口交给祭司。
\VS{7}祭司要献在耶和华面前,为她赎罪,她的血源就洁净了。这条例是为生育的妇人,无论是生男生女。
\VS{8}她的力量若不够献一只羊羔,她就要取两只斑鸠或是两只雏鸽,一只为燔祭,一只为赎罪祭。祭司要为她赎罪,她就洁净了。」

\par }\Chap{13}{\SH 有关皮肤病的条例
\par }{\PP \VerseOne{1}耶和华晓谕{\PN{摩西}}、{\PN{亚伦}}说:
\VS{2}「人的肉皮上若长了疖子,或长了癣,或长了火斑,在他肉皮上成了大麻风的灾病,就要将他带到祭司{\PN{亚伦}}或{\PN{亚伦}}作祭司的一个子孙面前。
\VS{3}祭司要察看肉皮上的灾病,若灾病处的毛已经变白,灾病的现象深于肉上的皮,这便是大麻风的灾病。祭司要察看他,定他为不洁净。
\VS{4}若火斑在他肉皮上是白的,现象不深于皮,其上的毛也没有变白,祭司就要将有灾病的人关锁七天。
\VS{5}第七天,祭司要察看他,若看灾病止住了,没有在皮上发散,祭司还要将他关锁七天。
\VS{6}第七天,祭司要再察看他,若灾病发暗,而且没有在皮上发散,祭司要定他为洁净,原来是癣;那人就要洗衣服,得为洁净。
\VS{7}但他为得洁净,将身体给祭司察看以后,癣若在皮上发散开了,他要再将身体给祭司察看。
\VS{8}祭司要察看,癣若在皮上发散,就要定他为不洁净,是大麻风。
\par }{\PP \VS{9}「人有了大麻风的灾病,就要将他带到祭司面前。
\VS{10}祭司要察看,皮上若长了白疖,使毛变白,在长白疖之处有了红瘀肉,
\VS{11}这是肉皮上的旧大麻风,祭司要定他为不洁净,不用将他关锁,因为他是不洁净了。
\VS{12}大麻风若在皮上四外发散,长满了患灾病人的皮,据祭司察看,从头到脚无处不有,
\VS{13}祭司就要察看,全身的肉若长满了大麻风,就要定那患灾病的为洁净;全身都变为白,他乃洁净了。
\VS{14}但红肉几时显在他的身上就几时不洁净。
\VS{15}祭司一看那红肉就定他为不洁净。红肉本是不洁净,是大麻风。
\VS{16}红肉若复原,又变白了,他就要来见祭司。
\VS{17}祭司要察看,灾病处若变白了,祭司就要定那患灾病的为洁净,他乃洁净了。
\par }{\PP \VS{18}「人若在皮肉上长疮,却治好了,
\VS{19}在长疮之处又起了白疖,或是白中带红的火斑,就要给祭司察看。
\VS{20}祭司要察看,若现象洼于皮,其上的毛也变白了,就要定他为不洁净,是大麻风的灾病发在疮中。
\VS{21}祭司若察看,其上没有白毛,也没有洼于皮,乃是发暗,就要将他关锁七天。
\VS{22}若在皮上发散开了,祭司就要定他为不洁净,是灾病。
\VS{23}火斑若在原处止住,没有发散,便是疮的痕迹,祭司就要定他为洁净。
\par }{\PP \VS{24}「人的皮肉上若起了火毒,火毒的瘀{\ADD{肉}}成了火斑,或是白中带红的,或是全白的,
\VS{25}祭司就要察看,火斑中的毛若变白了,现象又深于皮,是大麻风在火毒中发出,就要定他为不洁净,是大麻风的灾病。
\VS{26}但是祭司察看,在火斑中若没有白毛,也没有洼于皮,乃是发暗,就要将他关锁七天。
\VS{27}到第七天,祭司要察看他,火斑若在皮上发散开了,就要定他为不洁净,是大麻风的灾病。
\VS{28}火斑若在原处止住,没有在皮上发散,乃是发暗,是起的火毒,祭司要定他为洁净,不过是火毒的痕迹。
\par }{\PP \VS{29}「无论男女,若在头上有灾病,或是男人胡须上有灾病,
\VS{30}祭司就要察看;这灾病现象若深于皮,其间有细黄毛,就要定他为不洁净,这是头疥,是头上或是胡须上的大麻风。
\VS{31}祭司若察看头疥的灾病,现象不深于皮,其间也没有黑毛,就要将长头疥灾病的关锁七天。
\VS{32}第七天,祭司要察看灾病,若头疥没有发散,其间也没有黄毛,头疥的现象不深于皮,
\VS{33}那人就要剃去须发,但他不可剃头疥之处。祭司要将那长头疥的,再关锁七天。
\VS{34}第七天,祭司要察看头疥,头疥若没有在皮上发散,现象也不深于皮,就要定他为洁净,他要洗衣服,便成为洁净。
\VS{35}但他得洁净以后,头疥若在皮上发散开了,
\VS{36}祭司就要察看他。头疥若在皮上发散,就不必找那黄毛,他是不洁净了。
\VS{37}祭司若看头疥已经止住,其间也长了黑毛,头疥已然痊愈,那人是洁净了,就要定他为洁净。
\par }{\PP \VS{38}「无论男女,皮肉上若起了火斑,就是白火斑,
\VS{39}祭司就要察看,他们肉皮上的火斑若白中带黑,这是皮上发出的白癣,那人是洁净了。
\par }{\PP \VS{40}「人头上的发若掉了,他不过是头秃,还是洁净。
\VS{41}他顶前若掉了头发,他不过是顶门秃,还是洁净。
\VS{42}头秃处或是顶门秃处若有白中带红的灾病,这就是大麻风发在他头秃处或是顶门秃处,
\VS{43}祭司就要察看,他起的那灾病若在头秃处或是顶门秃处有白中带红的,像肉皮上大麻风的现象,
\VS{44}那人就是长大麻风,不洁净的,祭司总要定他为不洁净,他的灾病是在头上。
\par }{\PP \VS{45}「身上有长大麻风灾病的,他的衣服要撕裂,也要蓬头散发,蒙着上唇,喊叫说:『不洁净了!不洁净了!』
\VS{46}灾病在他身上的日子,他便是不洁净;他既是不洁净,就要独居营外。」
\par }{\SH 衣物发霉的条例
\par }{\PP \VS{47}「染了大麻风灾病的衣服,无论是羊毛衣服、是麻布衣服,
\VS{48}无论是在经上、在纬上,是麻布的、是羊毛的,是在皮子上,或在皮子做的什么物件上,
\VS{49}或在衣服上、皮子上,经上、纬上,或在皮子做的什么物件上,这灾病若是发绿,或是发红,是大麻风的灾病,要给祭司察看。
\VS{50}祭司就要察看那灾病,把染了灾病的物件关锁七天。
\VS{51}第七天,他要察看那灾病,灾病或在衣服上,经上、纬上,皮子上,若发散,这皮子无论当作何用,这灾病是蚕食的大麻风,都是不洁净了。
\VS{52}那染了灾病的衣服,或是经上、纬上,羊毛上,麻衣上,或是皮子做的什么物件上,他都要焚烧;因为这是蚕食的大麻风,必在火中焚烧。
\par }{\PP \VS{53}「祭司要察看,若灾病在衣服上,经上、纬上,或是皮子做的什么物件上,没有发散,
\VS{54}祭司就要吩咐他们,把染了灾病的物件洗了,再关锁七天。
\VS{55}洗过以后,祭司要察看,那物件若没有变色,灾病也没有消散,那物件就不洁净,是透重的灾病,无论正面反面,都要在火中焚烧。
\VS{56}洗过以后,祭司要察看,若见那灾病发暗,他就要把那灾病从衣服上,皮子上,经上、纬上,都撕去。
\VS{57}若仍现在衣服上,或是经上、纬上、皮子做的什么物件上,这就是灾病又发了,必用火焚烧那染灾病的物件。
\VS{58}所洗的衣服,或是经,或是纬,或是皮子做的什么物件,若灾病离开了,要再洗,就洁净了。」
\par }{\PP \VS{59}这就是大麻风灾病的条例,无论是在羊毛衣服上,麻布衣服上,经上、纬上,和皮子做的什么物件上,可以定为洁净或是不洁净。

\par }\Chap{14}{\SH 皮肤病后洁净的条例
\par }{\PP \VerseOne{1}耶和华晓谕{\PN{摩西}}说:「
\VS{2}长大麻风得洁净的日子,其例乃是这样:要带他去见祭司;
\VS{3}祭司要出到营外察看,若见他的大麻风痊愈了,
\VS{4}就要吩咐人为那求洁净的拿两只洁净的活鸟和香柏木、朱红色{\ADD{线}},并牛膝草来。
\VS{5}祭司要吩咐用瓦器盛活水,把一只鸟宰在上面。
\VS{6}至于那只活鸟,祭司要把它和香柏木、朱红色{\ADD{线}}并牛膝草一同蘸于宰在活水上的鸟血中,
\VS{7}用以在那长大麻风求洁净的人身上洒七次,就定他为洁净,又把活鸟放在田野里。
\VS{8}求洁净的人当洗衣服,剃去毛发,用水洗澡,就洁净了;然后可以进营,只是要在自己的帐棚外居住七天。
\VS{9}第七天,再把头上所有的头发与胡须、眉毛,并全身的毛,都剃了;又要洗衣服,用水洗身,就洁净了。
\par }{\PP \VS{10}「第八天,他要取两只没有残疾的公羊羔和一只没有残疾、一岁的母羊羔,又要把调油的细面{\ADD{伊法}}十分之三为素祭,并油一罗革,一同取来。
\VS{11}行洁净之礼的祭司要将那求洁净的人和这些东西安置在会幕门口、耶和华面前。
\VS{12}祭司要取一只公羊羔献为赎愆祭,和那一罗革油一同作摇祭,在耶和华面前摇一摇;
\VS{13}把公羊羔宰于圣地,就是宰赎罪祭牲和燔祭牲之地。赎愆祭要归祭司,与赎罪祭一样,是至圣的。
\VS{14}祭司要取些赎愆祭牲的血,抹在求洁净人的右耳垂上和右手的大拇指上,并右脚的大拇指上。
\VS{15}祭司要从那一罗革油中取些倒在自己的左手掌里,
\VS{16}把右手的一个指头蘸在左手的油里,在耶和华面前用指头弹七次。
\VS{17}将手里所剩的油抹在那求洁净人的右耳垂上和右手的大拇指上,并右脚的大拇指上,就是抹在赎愆祭牲的血上。
\VS{18}祭司手里所剩的油要抹在那求洁净人的头上,在耶和华面前为他赎罪。
\VS{19}祭司要献赎罪祭,为那本不洁净、求洁净的人赎罪;然后要宰燔祭牲,
\VS{20}把燔祭和素祭献在坛上,为他赎罪,他就洁净了。
\par }{\PP \VS{21}「他若贫穷不能预备够数,就要取一只公羊羔作赎愆祭,可以摇一摇,为他赎罪;也要把调油的细面{\ADD{伊法}}十分之一为素祭,和油一罗革一同取来;
\VS{22}又照他的力量取两只斑鸠或是两只雏鸽,一只作赎罪祭,一只作燔祭。
\VS{23}第八天,要为洁净,把这些带到会幕门口、耶和华面前,交给祭司。
\VS{24}祭司要把赎愆祭的羊羔和那一罗革油一同作摇祭,在耶和华面前摇一摇。
\VS{25}要宰了赎愆祭的羊羔,取些赎愆祭牲的血,抹在那求洁净人的右耳垂上和右手的大拇指上,并右脚的大拇指上。
\VS{26}祭司要把些油倒在自己的左手掌里,
\VS{27}把左手里的油,在耶和华面前,用右手的一个指头弹七次,
\VS{28}又把手里的油抹些在那求洁净人的右耳垂上和右手的大拇指上,并右脚的大拇指上,就是抹赎愆祭之血的原处。
\VS{29}祭司手里所剩的油要抹在那求洁净人的头上,在耶和华面前为他赎罪。
\VS{30}那人又要照他的力量献上一只斑鸠或是一只雏鸽,
\VS{31}就是他所能办的,一只为赎罪祭,一只为燔祭,与素祭一同献上;祭司要在耶和华面前为他赎罪。
\VS{32}这是那有大麻风灾病的人、不能将{\ADD{关乎}}得洁净之物预备够数的条例。」
\par }{\SH 处理房屋发霉的条例
\par }{\PP \VS{33}耶和华晓谕{\PN{摩西}}、{\PN{亚伦}}说:
\VS{34}「你们到了我赐给你们为业的{\PN{迦南}}地,我若使你们所得为业之地的房屋中有大麻风的灾病,
\VS{35}房主就要去告诉祭司说:『据我看,房屋中似乎有灾病。』
\VS{36}祭司还没有进去察看灾病以前,就要吩咐人把房子腾空,免得房子里所有的都成了不洁净;然后祭司要进去察看房子。
\VS{37}他要察看那灾病,灾病若在房子的墙上有发绿或发红的凹斑纹,现象洼于墙,
\VS{38}祭司就要出到房门外,把房子封锁七天。
\VS{39}第七天,祭司要再去察看,灾病若在房子的墙上发散,
\VS{40}就要吩咐人把那有灾病的石头挖出来,扔在城外不洁净之处;
\VS{41}也要叫人刮房内的四围,所刮掉的灰泥要倒在城外不洁净之处;
\VS{42}又要用别的石头代替那挖出来的石头,要另用灰泥墁房子。
\par }{\PP \VS{43}「他挖出石头,刮了房子,墁了以后,灾病若在房子里又发现,
\VS{44}祭司就要进去察看,灾病若在房子里发散,这就是房内蚕食的大麻风,是不洁净。
\VS{45}他就要拆毁房子,把石头、木头、灰泥都搬到城外不洁净之处。
\VS{46}在房子封锁的时候,进去的人必不洁净到晚上;
\VS{47}在房子里躺着的必洗衣服;在房子里吃饭的也必洗衣服。
\par }{\PP \VS{48}「房子墁了以后,祭司若进去察看,见灾病在房内没有发散,就要定房子为洁净,因为灾病已经消除。
\VS{49}要为洁净房子取两只鸟和香柏木、朱红色{\ADD{线}}并牛膝草,
\VS{50}用瓦器盛活水,把一只鸟宰在上面,
\VS{51}把香柏木、牛膝草、朱红色{\ADD{线}},并那活鸟,都蘸在被宰的鸟血中与活水中,用以洒房子七次。
\VS{52}要用鸟血、活水、活鸟、香柏木、牛膝草,并朱红色{\ADD{线}},洁净那房子。
\VS{53}但要把活鸟放在城外田野里。这样洁净房子\FTNT{}{{\FR 14:53: }原文是为房子赎罪},房子就洁净了。」
\par }{\PP \VS{54}这是为各类大麻风的灾病和头疥,
\VS{55}并衣服与房子的大麻风,
\VS{56}以及疖子、癣、火斑所立的条例,
\VS{57}指明何时为洁净,何时为不洁净。这是大麻风的条例。

\par }\Chap{15}{\SH 漏症患者洁净的条例
\par }{\PP \VerseOne{1}耶和华对{\PN{摩西}}、{\PN{亚伦}}说:
\VS{2}「你们晓谕{\PN{以色列}}人说:人若身患漏症,他因这漏症就不洁净了。
\VS{3}他患漏症,无论是下流的,是止住的,都是不洁净。
\VS{4}他所躺的床都为不洁净,所坐的物也为不洁净。
\VS{5}凡摸那床的,必不洁净到晚上,并要洗衣服,用水洗澡。
\VS{6}那坐患漏症人所坐之物的,必不洁净到晚上,并要洗衣服,用水洗澡。
\VS{7}那摸患漏症人身体的,必不洁净到晚上,并要洗衣服,用水洗澡。
\VS{8}若患漏症人吐在洁净的人身上,那人必不洁净到晚上,并要洗衣服,用水洗澡。
\VS{9}患漏症人所骑的鞍子也为不洁净。
\VS{10}凡摸了他身下之物的,必不洁净到晚上;拿了那物的,必不洁净到晚上,并要洗衣服,用水洗澡。
\VS{11}患漏症的人没有用水涮手,无论摸了谁,谁必不洁净到晚上,并要洗衣服,用水洗澡。
\VS{12}患漏症人所摸的瓦器就必打破;所摸的一切木器也必用水涮洗。
\par }{\PP \VS{13}「患漏症的人痊愈了,就要为洁净自己计算七天,也必洗衣服,用活水洗身,就洁净了。
\VS{14}第八天,要取两只斑鸠或是两只雏鸽,来到会幕门口、耶和华面前,把鸟交给祭司。
\VS{15}祭司要献上一只为赎罪祭,一只为燔祭;因那人患的漏症,祭司要在耶和华面前为他赎罪。
\par }{\PP \VS{16}「人若{\ADD{梦}}遗,他必不洁净到晚上,并要用水洗全身。
\VS{17}无论是衣服是皮子,被精所染,必不洁净到晚上,并要用水洗。
\VS{18}若男女交合,两个人必不洁净到晚上,并要用水洗澡。
\par }{\PP \VS{19}「女人行经,必污秽七天;凡摸她的,必不洁净到晚上。
\VS{20}女人在污秽之中,凡她所躺的物件都为不洁净,所坐的物件也都不洁净。
\VS{21}凡摸她床的,必不洁净到晚上,并要洗衣服,用水洗澡。
\VS{22}凡摸她所坐什么物件的,必不洁净到晚上,并要洗衣服,用水洗澡。
\VS{23}在女人的床上,或在她坐的物上,若有别的物件,人一摸了,必不洁净到晚上。
\VS{24}男人若与那女人同房,染了她的污秽,就要七天不洁净;所躺的床也为不洁净。
\par }{\PP \VS{25}「女人若在经期以外患多日的血漏,或是经期过长,有了漏症,她就因这漏症不洁净,与她在经期不洁净一样。
\VS{26}她在患漏症的日子所躺的床、所坐的物都要看为不洁净,与她月经的时候一样。
\VS{27}凡摸这些物件的,就为不洁净,必不洁净到晚上,并要洗衣服,用水洗澡。
\VS{28}女人的漏症若好了,就要计算七天,然后才为洁净。
\VS{29}第八天,要取两只斑鸠或是两只雏鸽,带到会幕门口给祭司。
\VS{30}祭司要献一只为赎罪祭,一只为燔祭;因那人血漏不洁,祭司要在耶和华面前为她赎罪。
\par }{\PP \VS{31}「你们要这样使{\PN{以色列}}人与他们的污秽隔绝,免得他们玷污我的帐幕,就因自己的污秽死亡。」
\par }{\PP \VS{32}这是患漏症和{\ADD{梦}}遗而不洁净的,
\VS{33}并有月经病的和患漏症的,无论男女,并人与不洁净女人同房的条例。

\par }\Chap{16}{\SH 赎罪日
\par }{\PP \VerseOne{1}{\PN{亚伦}}的两个儿子近到耶和华面前死了。死了之后,耶和华晓谕{\PN{摩西}}说:
\VS{2}「要告诉你哥哥{\PN{亚伦}},不可随时进圣所的幔子内、到柜上的施恩座前,免得他死亡,因为我要从云中显现在施恩座上。
\VS{3}{\PN{亚伦}}进圣所,要带一只公牛犊为赎罪祭,一只公绵羊为燔祭。
\VS{4}要穿上细麻布圣内袍,把细麻布裤子穿在身上,腰束细麻布带子,头戴细麻布冠冕;这都是圣服。他要用水洗身,然后穿戴。
\VS{5}要从{\PN{以色列}}会众取两只公山羊为赎罪祭,一只公绵羊为燔祭。
\par }{\PP \VS{6}「{\PN{亚伦}}要把赎罪祭的公牛奉上,为自己和本家赎罪;
\VS{7}也要把两只公山羊安置在会幕门口、耶和华面前,
\VS{8}为那两只羊拈阄,一阄归与耶和华,一阄归与阿撒泻勒。
\VS{9}{\PN{亚伦}}要把那拈阄归与耶和华的羊献为赎罪祭,
\VS{10}但那拈阄归与阿撒泻勒的羊要活着安置在耶和华面前,用以赎罪,打发{\ADD{人送}}到旷野去,归与阿撒泻勒。
\par }{\PP \VS{11}「{\PN{亚伦}}要把赎罪祭的公牛牵来宰了,为自己和本家赎罪;
\VS{12}拿香炉,从耶和华面前的坛上盛满火炭,又拿一捧捣细的香料,都带入幔子内,
\VS{13}在耶和华面前,把香放在火上,使香的烟云遮掩法{\ADD{柜}}上的施恩座,免得他死亡;
\VS{14}也要取些公牛的血,用指头弹在施恩座的东面,又在施恩座的前面弹血七次。
\par }{\PP \VS{15}「随后他要宰那为百姓作赎罪祭的公山羊,把羊的血带入幔子内,弹在施恩座的上面和前面,好像弹公牛的血一样。
\VS{16}他因{\PN{以色列}}人诸般的污秽、过犯,就是他们一切的罪愆,当这样在圣所行赎罪之礼,并因会幕在他们污秽之中,也要照样而行。
\VS{17}他进圣所赎罪的时候,会幕里不可有人,直等到他为自己和本家并{\PN{以色列}}全会众赎了罪出来。
\VS{18}他出来,要到耶和华面前的坛那里,在坛上行赎罪之礼,又要取些公牛的血和公山羊的血,抹在坛上四角的周围;
\VS{19}也要用指头把血弹在坛上七次,洁净了坛,从坛上除掉{\PN{以色列}}人诸般的污秽,使坛成圣。」
\par }{\SH 代罪羊
\par }{\PP \VS{20}「{\PN{亚伦}}为圣所和会幕并坛献完了赎罪祭,就要把那只活着的公山羊奉上。
\VS{21}两手按在羊头上,承认{\PN{以色列}}人诸般的罪孽过犯,就是他们一切的罪愆,把这罪都归在羊的头上,借着所派之人的手,送到旷野去。
\VS{22}要把这羊放在旷野,这羊要担当他们一切的罪孽,带到无人之地。
\par }{\PP \VS{23}「{\PN{亚伦}}要进会幕,把他进圣所时所穿的细麻布衣服脱下,放在那里,
\VS{24}又要在圣处用水洗身,穿上衣服,出来,把自己的燔祭和百姓的燔祭献上,为自己和百姓赎罪。
\VS{25}赎罪祭牲的脂油要在坛上焚烧。
\VS{26}那放羊归与阿撒泻勒的人要洗衣服,用水洗身,然后进营。
\VS{27}作赎罪祭的公牛和公山羊的血既带入圣所赎罪,这牛羊就要搬到营外,将皮、肉、粪用火焚烧。
\VS{28}焚烧的人要洗衣服,用水洗身,然后进营。」
\par }{\SH 守赎罪日
\par }{\PP \VS{29}「每逢七月初十日,你们要刻苦己心,无论是本地人,是寄居在你们中间的外人,什么工都不可做;这要作你们永远的定例。
\VS{30}因在这日要为你们赎罪,使你们洁净。你们要在耶和华面前得以洁净,脱尽一切的罪愆。
\VS{31}这日你们要守为圣安息日,要刻苦己心;这为永远的定例。
\VS{32}那受膏、接续他父亲承接圣职的祭司要穿上细麻布的圣衣,行赎罪之礼。
\VS{33}他要在至圣所和会幕与坛行赎罪之礼,并要为众祭司和会众的百姓赎罪。
\VS{34}这要作你们永远的定例—就是因{\PN{以色列}}人一切的罪,要一年一次为他们赎罪。」
\par }{\PP 于是,{\PN{亚伦}}照耶和华所吩咐{\PN{摩西}}的行了。

\par }\Chap{17}{\SH 血的神圣
\par }{\PP \VerseOne{1}耶和华对{\PN{摩西}}说:
\VS{2}「你晓谕{\PN{亚伦}}和他儿子并{\PN{以色列}}众人说,耶和华所吩咐的乃是这样:
\VS{3}凡{\PN{以色列}}家中的人宰公牛,或是绵羊羔,或是山羊,不拘宰于营内营外,
\VS{4}若未曾牵到会幕门口、耶和华的帐幕前献给耶和华为供物,流血的罪必归到那人身上。他流了血,要从民中剪除。
\VS{5}这是为要使{\PN{以色列}}人把他们在田野里所献的祭带到会幕门口、耶和华面前,交给祭司,献与耶和华为平安祭。
\VS{6}祭司要把血洒在会幕门口、耶和华的坛上,把脂油焚烧,献给耶和华为馨香的祭。
\VS{7}他们不可再献祭给他们行邪淫所随从的鬼魔\FTNT{}{{\FR 17:7: }原文是公山羊};这要作他们世世代代永远的定例。
\par }{\PP \VS{8}「你要晓谕他们说:凡{\PN{以色列}}家中的人,或是寄居在他们中间的外人,献燔祭或是{\ADD{平安}}祭,
\VS{9}若不带到会幕门口献给耶和华,那人必从民中剪除。
\par }{\PP \VS{10}「凡{\PN{以色列}}家中的人,或是寄居在他们中间的外人,若吃什么血,我必向那吃血的人变脸,把他从民中剪除。
\VS{11}因为活物的生命是在血中。我把这血赐给你们,可以在坛上为你们的生命赎罪;因血里有生命,所以能赎罪。
\VS{12}因此,我对{\PN{以色列}}人说:你们都不可吃血;寄居在你们中间的外人也不可吃血。
\VS{13}凡{\PN{以色列}}人,或是寄居在他们中间的外人,若打猎得了可吃的禽兽,必放出它的血来,用土掩盖。
\par }{\PP \VS{14}「论到一切活物的生命,就在血中。所以我对{\PN{以色列}}人说:无论什么活物的血,你们都不可吃,因为一切活物的血就是他的生命。凡吃了血的,必被剪除。
\VS{15}凡吃自死的,或是被野兽撕裂的,无论是本地人,是寄居的,必不洁净到晚上,都要洗衣服,用水洗身,到了晚上才为洁净。
\VS{16}但他若不洗衣服,也不洗身,就必担当他的罪孽。」

\par }\Chap{18}{\SH 有关淫乱的禁令
\par }{\PP \VerseOne{1}耶和华对{\PN{摩西}}说:
\VS{2}「你晓谕{\PN{以色列}}人说:我是耶和华—你们的 神。
\VS{3}你们从前住的{\PN{埃及}}地,那里人的行为,你们不可效法,我要领你们到的{\PN{迦南}}地,那里人的行为也不可效法,也不可照他们的恶俗行。
\VS{4}你们要遵我的典章,守我的律例,按此而行。我是耶和华—你们的 神。
\VS{5}所以,你们要守我的律例典章;人若遵行,就必因此活着。我是耶和华。
\par }{\PP \VS{6}「你们都不可露骨肉之亲的下体,亲近他们。我是耶和华。
\VS{7}不可露你母亲的下体,羞辱了你父亲。她是你的母亲,不可露她的下体。
\VS{8}不可露你继母的下体;这本是你父亲的下体。
\VS{9}你的姊妹,不拘是异母同父的,是异父同母的,无论是生在家生在外的,都不可露她们的下体。
\VS{10}不可露你孙女或是外孙女的下体,露了她们的下体就是露了自己的下体。
\VS{11}你继母从你父亲生的女儿本是你的妹妹,不可露她的下体。
\VS{12}不可露你姑母的下体;她是你父亲的骨肉之亲。
\VS{13}不可露你姨母的下体;她是你母亲的骨肉之亲。
\VS{14}不可亲近你伯叔之妻,羞辱了你伯叔;她是你的伯叔母。
\VS{15}不可露你儿妇的下体;她是你儿子的妻,不可露她的下体。
\VS{16}不可露你弟兄妻子的下体;这本是你弟兄的下体。
\VS{17}不可露了妇人的下体,又露她女儿的下体,也不可娶她孙女或是外孙女,露她们的下体;她们是骨肉之亲,这本是大恶。
\VS{18}你妻还在的时候,不可另娶她的姊妹作对头,露她的下体。
\par }{\PP \VS{19}「女人行经不洁净的时候,不可露她的下体,与她亲近。
\VS{20}不可与邻舍的妻行淫,玷污自己。
\VS{21}不可使你的儿女经火归与{\PN{摩洛}},也不可亵渎你 神的名。我是耶和华。
\VS{22}不可与男人苟合,像与女人一样;这本是可憎恶的。
\VS{23}不可与兽淫合,玷污自己。女人也不可站在兽前,与它淫合;这本是逆性的事。
\par }{\PP \VS{24}「在这一切的事上,你们都不可玷污自己;因为我在你们面前所逐出的列邦,在这一切的事上玷污了自己;
\VS{25}连地也玷污了,所以我追讨那地的罪孽,那地也吐出它的居民。
\VS{26}故此,你们要守我的律例典章。这一切可憎恶的事,无论是本地人,是寄居在你们中间的外人,都不可行,(
\VS{27}在你们以先居住那地的人行了这一切可憎恶的事,地就玷污了,)
\VS{28}免得你们玷污那地的时候,地就把你们吐出,像吐出在你们以先的国民一样。
\VS{29}无论什么人,行了其中可憎的一件事,必从民中剪除。
\VS{30}所以,你们要守我所吩咐的,免得你们随从那些可憎的恶俗,就是在你们以先{\ADD{的人}}所常行的,以致玷污了自己。我是耶和华—你们的 神。」

\par }\Chap{19}{\SH 圣洁和公正的法例
\par }{\PP \VerseOne{1}耶和华对{\PN{摩西}}说:
\VS{2}「你晓谕{\PN{以色列}}全会众说:你们要圣洁,因为我耶和华—你们的 神是圣洁的。
\VS{3}你们各人都当孝敬父母,也要守我的安息日。我是耶和华—你们的 神。
\VS{4}你们不可偏向虚无的神,也不可为自己铸造神{\ADD{像}}。我是耶和华—你们的 神。
\VS{5}你们献平安祭给耶和华的时候,要献得可蒙悦纳。
\VS{6}这祭物要在献的那一天和第二天吃,若有剩到第三天的,就必用火焚烧。
\VS{7}第三天若再吃,这就为可憎恶的,必不蒙悦纳。
\VS{8}凡吃的人必担当他的罪孽;因为他亵渎了耶和华的圣物,那人必从民中剪除。
\par }{\PP \VS{9}「在你们的地收割庄稼,不可割尽田角,也不可拾取所遗落的。
\VS{10}不可摘尽葡萄园的果子,也不可拾取葡萄园所掉的果子;要留给穷人和寄居的。我是耶和华—你们的 神。
\par }{\PP \VS{11}「你们不可偷盗,不可欺骗,也不可彼此说谎。
\VS{12}不可指着我的名起假誓,亵渎你 神的名。我是耶和华。
\par }{\PP \VS{13}「不可欺压你的邻舍,也不可抢夺他的物。雇工人的工价,不可在你那里过夜,留到早晨。
\VS{14}不可咒骂聋子,也不可将绊脚石放在瞎子面前,只要敬畏你的 神。我是耶和华。
\par }{\PP \VS{15}「你们施行审判,不可行不义;不可偏护穷人,也不可重看有势力的人,只要按着公义审判你的邻舍。
\VS{16}不可在民中往来搬弄是非,也不可与邻舍为敌,置之于死\FTNT{}{{\FR 19:16: }原文是流他的血}。我是耶和华。
\par }{\PP \VS{17}「不可心里恨你的弟兄;总要指摘你的邻舍,免得因他担罪。
\VS{18}不可报仇,也不可埋怨你本国的子民,却要爱人如己。我是耶和华。
\par }{\PP \VS{19}「你们要守我的律例。不可叫你的牲畜与异类配合;不可用两样搀杂的种种你的地,也不可用两样搀杂的料做衣服穿在身上。
\par }{\PP \VS{20}「婢女许配了丈夫,还没有被赎、得释放,人若与她行淫,二人要受刑罚,却不把他们治死,因为婢女还没有得自由。
\VS{21}那人要把赎愆祭,就是一只公绵羊牵到会幕门口、耶和华面前。
\VS{22}祭司要用赎愆祭的羊在耶和华面前赎他所犯的罪,他的罪就必蒙赦免。
\par }{\PP \VS{23}「你们到了{\PN{迦南}}地,栽种各样结果子的树木,就要以所结的果子如未受割礼的一样。三年之久,你们要以这些果子,如未受割礼的,是不可吃的。
\VS{24}但第四年所结的果子全要成为圣,用以赞美耶和华。
\VS{25}第五年,你们要吃那树上的果子,好叫树给你们结果子更多。我是耶和华—你们的 神。
\par }{\PP \VS{26}「你们不可吃带血的物;不可用法术,也不可观兆。
\VS{27}头的周围\FTNT{}{{\FR 19:27: }或译:两鬓}不可剃,胡须的周围也不可损坏。
\VS{28}不可为死人用刀划身,也不可在身上刺花纹。我是耶和华。
\par }{\PP \VS{29}「不可辱没你的女儿,使她为娼妓,恐怕地上{\ADD{的人}}专向淫乱,地就满了大恶。
\VS{30}你们要守我的安息日,敬我的圣所。我是耶和华。
\par }{\PP \VS{31}「不可偏向那些交鬼的和行巫术的;不可求问他们,以致被他们玷污了。我是耶和华—你们的 神。
\par }{\PP \VS{32}「在白发的人面前,你要站起来;也要尊敬老人,又要敬畏你的 神。我是耶和华。
\par }{\PP \VS{33}「若有外人在你们国中和你同居,就不可欺负他。
\VS{34}和你们同居的外人,你们要看他如本地人一样,并要爱他如己,因为你们在{\PN{埃及}}地也作过寄居的。我是耶和华—你们的 神。
\par }{\PP \VS{35}「你们施行审判,不可行不义;在尺、秤、升、斗上也是如此。
\VS{36}要用公道天平、公道法码、公道升斗、公道秤。我是耶和华—你们的 神,曾把你们从{\PN{埃及}}地领出来的。
\VS{37}你们要谨守遵行我一切的律例典章。我是耶和华。」

\par }\Chap{20}{\SH 处罚悖逆的人
\par }{\PP \VerseOne{1}耶和华对{\PN{摩西}}说:
\VS{2}「你还要晓谕{\PN{以色列}}人说:凡{\PN{以色列}}人,或是在{\PN{以色列}}中寄居的外人,把自己的儿女献给{\PN{摩洛}}的,总要治死他;本地人要用石头把他打死。
\VS{3}我也要向那人变脸,把他从民中剪除;因为他把儿女献给{\PN{摩洛}},玷污我的圣所,亵渎我的圣名。
\VS{4}那人把儿女献给{\PN{摩洛}},本地人若佯为不见,不把他治死,
\VS{5}我就要向这人和他的家变脸,把他和一切随他与{\PN{摩洛}}行邪淫的人都从民中剪除。
\par }{\PP \VS{6}「人偏向交鬼的和行巫术的,随他们行邪淫,我要向那人变脸,把他从民中剪除。
\VS{7}所以你们要自洁成圣,因为我是耶和华—你们的 神。
\VS{8}你们要谨守遵行我的律例;我是叫你们成圣的耶和华。
\VS{9}凡咒骂父母的,总要治死他;他咒骂了父母,他的罪\FTNT{}{{\FR 20:9: }原文是血;本章同}要归到他身上。
\par }{\PP \VS{10}「与邻舍之妻行淫的,奸夫淫妇都必治死。
\VS{11}与继母行淫的,就是羞辱了他父亲,总要把他们二人治死,罪要归到他们身上。
\VS{12}与儿妇同房的,总要把他们二人治死;他们行了逆伦的事,罪要归到他们身上。
\VS{13}人若与男人苟合,像与女人一样,他们二人行了可憎的事,总要把他们治死,罪要归到他们身上。
\VS{14}人若娶妻,并娶其母,便是大恶,要把这三人用火焚烧,使你们中间免去大恶。
\VS{15}人若与兽淫合,总要治死他,也要杀那兽。
\VS{16}女人若与兽亲近,与它淫合,你要杀那女人和那兽,总要把他们治死,罪要归到他们身上。
\par }{\PP \VS{17}「人若娶他的姊妹,无论是异母同父的,是异父同母的,彼此见了下体,这是可耻的事;他们必在本民的眼前被剪除。他露了姊妹的下体,必担当自己的罪孽。
\VS{18}妇人有月经,若与她同房,露了她的下体,就是露了妇人的血源,妇人也露了自己的血源,二人必从民中剪除。
\VS{19}不可露姨母或是姑母的下体,这是露了骨肉之亲的下体;二人必担当自己的罪孽。
\VS{20}人若与伯叔之妻同房,就羞辱了他的伯叔;二人要担当自己的罪,必无子女而死。
\VS{21}人若娶弟兄之妻,这本是污秽的事,羞辱了他的弟兄;二人必无子女。
\par }{\PP \VS{22}「所以,你们要谨守遵行我一切的律例典章,免得我领你们去住的那地把你们吐出。
\VS{23}我在你们面前所逐出的国民,你们不可随从他们的风俗;因为他们行了这一切的事,所以我厌恶他们。
\VS{24}但我对你们说过,你们要承受他们的地,就是我要赐给你们为业、流奶与蜜之地。我是耶和华—你们的 神,使你们与万民有分别的。
\VS{25}所以,你们要把洁净和不洁净的禽兽分别出来;不可因我给你们分为不洁净的禽兽,或是滋生在地上的活物,使自己成为可憎恶的。
\VS{26}你们要归我为圣,因为我—耶和华是圣的,并叫你们与万民有分别,使你们作我的民。
\par }{\PP \VS{27}「无论男女,是交鬼的或行巫术的,总要治死他们。人必用石头把他们打死,罪要归到他们身上。」

\par }\Chap{21}{\SH 祭司圣洁的条例
\par }{\PP \VerseOne{1}耶和华对{\PN{摩西}}说:「你告诉{\PN{亚伦}}子孙作祭司的说:祭司不可为民中的死人沾染自己,
\VS{2}除非为他骨肉之亲的父母、儿女、弟兄,
\VS{3}和未曾出嫁、作处女的姊妹,才可以沾染自己。
\VS{4}祭司既在民中为首,就不可从俗沾染自己。
\VS{5}不可使头光秃;不可剃除胡须的周围,也不可用刀划身。
\VS{6}要归 神为圣,不可亵渎 神的名;因为耶和华的火祭,就是 神的食物,是他们献的,所以他们要成为圣。
\VS{7}不可娶妓女或被污的女人为妻,也不可娶被休的妇人为妻,因为祭司是归 神为圣。
\VS{8}所以你要使他成圣,因为他奉献你 神的食物;你要以他为圣,因为我—使你们成圣的耶和华—是圣的。
\VS{9}祭司的女儿若行淫辱没自己,就辱没了父亲,必用火将她焚烧。
\par }{\PP \VS{10}「在弟兄中作大祭司、头上倒了膏油、又承接圣职、穿了{\ADD{圣}}衣的,不可蓬头散发,也不可撕裂衣服。
\VS{11}不可挨近死尸,也不可为父母沾染自己。
\VS{12}不可出圣所,也不可亵渎 神的圣所,因为 神膏油的冠冕在他头上。我是耶和华。
\VS{13}他要娶处女为妻。
\VS{14}寡妇或是被休的妇人,或是被污为妓的女人,都不可娶;只可娶本民中的处女为妻。
\VS{15}不可在民中辱没他的儿女,因为我是叫他成圣的耶和华。」
\par }{\PP \VS{16}耶和华对{\PN{摩西}}说:
\VS{17}「你告诉{\PN{亚伦}}说:你世世代代的后裔,凡有残疾的,都不可近前来献他 神的食物。
\VS{18}因为凡有残疾的,无论是瞎眼的、瘸腿的、塌鼻子的、肢体有余的、
\VS{19}折脚折手的、
\VS{20}驼背的、矮矬的、眼睛有毛病的、长癣的、长疥的,或是损坏肾子的,都不可近前来。
\VS{21}祭司{\PN{亚伦}}的后裔,凡有残疾的,都不可近前来,将火祭献给耶和华。他有残疾,不可近前来献 神的食物。
\VS{22}神的食物,无论是圣的,至圣的,他都可以吃。
\VS{23}但不可进到幔子前,也不可就近坛前;因为他有残疾,免得亵渎我的圣所。我是叫他成圣的耶和华。」
\par }{\PP \VS{24}于是,{\PN{摩西}}晓谕{\PN{亚伦}}和{\PN{亚伦}}的子孙,并{\PN{以色列}}众人。

\par }\Chap{22}{\SH 圣物的圣洁
\par }{\PP \VerseOne{1}耶和华对{\PN{摩西}}说:
\VS{2}「你吩咐{\PN{亚伦}}和他子孙说:要远离{\PN{以色列}}人所分别为圣、归给我的圣物,免得亵渎我的圣名。我是耶和华。
\VS{3}你要对他们说:你们世世代代的后裔,凡身上有污秽、亲近{\PN{以色列}}人所分别为圣、归耶和华圣物的,那人必在我面前剪除。我是耶和华。
\VS{4}{\PN{亚伦}}的后裔,凡长大麻风的,或是有漏症的,不可吃圣物,直等他洁净了。无论谁摸那因死尸不洁净的物\FTNT{}{{\FR 22:4: }或译:人},或是遗精的人,
\VS{5}或是摸什么使他不洁净的爬物,或是摸那使他不洁净的人(不拘那人有什么不洁净),
\VS{6}摸了这些人、物的,必不洁净到晚上;若不用水洗身,就不可吃圣物。
\VS{7}日落的时候,他就洁净了,然后可以吃圣物,因为这是他的食物。
\VS{8}自死的或是被野兽撕裂的,他不可吃,因此污秽自己。我是耶和华。
\VS{9}所以他们要守我所吩咐的,免得轻忽了,因此担罪而死。我是叫他们成圣的耶和华。
\par }{\PP \VS{10}「凡外人不可吃圣物;寄居在祭司家的,或是雇工人,都不可吃圣物;
\VS{11}倘若祭司买人,是他的钱买的,那人就可以吃圣物;生在他家的人也可以吃。
\VS{12}祭司的女儿若嫁外人,就不可吃举祭的圣物。
\VS{13}但祭司的女儿若是寡妇,或是被休的,没有孩子,又归回父家,与她青年一样,就可以吃她父亲的食物;只是外人不可吃。
\VS{14}若有人误吃了圣物,要照圣物的原数加上五分之一交给祭司。
\VS{15}祭司不可亵渎{\PN{以色列}}人所献给耶和华的圣物,
\VS{16}免得他们在吃圣物上自取罪孽,因为我是叫他们成圣的耶和华。」
\par }{\PP \VS{17}耶和华对{\PN{摩西}}说:
\VS{18}「你晓谕{\PN{亚伦}}和他子孙,并{\PN{以色列}}众人说:{\PN{以色列}}家中的人,或在{\PN{以色列}}中寄居的,凡献供物,无论是所许的愿,是甘心献的,就是献给耶和华作燔祭的,
\VS{19}要将没有残疾的公牛,或是绵羊,或是山羊{\ADD{献上}},如此方蒙悦纳。
\VS{20}凡有残疾的,你们不可献上,因为这不蒙悦纳。
\VS{21}凡从牛群或是羊群中,将平安祭献给耶和华,为要还特许的愿,或是作甘心献的,所献的必纯全无残疾的才蒙悦纳。
\VS{22}瞎眼的、折伤的、残废的、有瘤子的、长癣的、长疥的都不可献给耶和华,也不可在坛上作为火祭献给耶和华。
\VS{23}无论是公牛是绵羊羔,若肢体有余的,或是缺少的,只可作甘心祭献上;用以还愿,却不蒙悦纳。
\VS{24}肾子损伤的,或是压碎的,或是破裂的,或是骟了的,不可献给耶和华,在你们的地上也不可{\ADD{这样}}行。
\VS{25}这类的物,你们从外人的手,一样也不可接受作你们 神的食物献上;因为这些都有损坏,有残疾,不蒙悦纳。」
\par }{\PP \VS{26}耶和华晓谕{\PN{摩西}}说:
\VS{27}「才生的公牛,或是绵羊或是山羊,七天当跟着母;从第八天以后,可以当供物蒙悦纳,作为耶和华的火祭。
\VS{28}无论是母牛是母羊,不可同日宰母和子。
\VS{29}你们献感谢祭给耶和华,要献得可蒙悦纳。
\VS{30}要当天吃,一点不可留到早晨。我是耶和华。
\par }{\PP \VS{31}「你们要谨守遵行我的诫命。我是耶和华。
\VS{32}你们不可亵渎我的圣名;我在{\PN{以色列}}人中,却要被尊为圣。我是叫你们成圣的耶和华,
\VS{33}把你们从{\PN{埃及}}地领出来,作你们的 神。我是耶和华。」

\par }\Chap{23}{\SH 节期
\par }{\PP \VerseOne{1}耶和华对{\PN{摩西}}说:
\VS{2}「你晓谕{\PN{以色列}}人说:耶和华的节期,你们要宣告为圣会的节期。
\VS{3}六日要做工,第七日是圣安息日,当有圣会;你们什么工都不可做。这是在你们一切的住处向耶和华守的安息日。
\VS{4}耶和华的节期,就是你们到了日期要宣告为圣会的,乃是这些。」
\par }{\SH 逾越节和无酵节
\par }{\R (民28·16—25)
\par }{\PP \VS{5}「正月十四日,黄昏的时候,是耶和华的逾越节。
\VS{6}这月十五日是向耶和华守的无酵节;你们要吃无酵饼七日。
\VS{7}第一日当有圣会,什么劳碌的工都不可做;
\VS{8}要将火祭献给耶和华七日。第七日是圣会,什么劳碌的工都不可做。」
\par }{\PP \VS{9}耶和华对{\PN{摩西}}说:
\VS{10}「你晓谕{\PN{以色列}}人说:你们到了我赐给你们的地,收割庄稼的时候,要将初熟的庄稼一捆带给祭司。
\VS{11}他要把这一捆在耶和华面前摇一摇,使你们得蒙悦纳。祭司要在安息日的次日把这捆摇一摇。
\VS{12}摇这捆的日子,你们要把一岁、没有残疾的公绵羊羔献给耶和华为燔祭。
\VS{13}同献的素祭,就是调油的细面{\ADD{伊法}}十分之二,作为馨香的火祭,献给耶和华。同献的奠祭,要酒一欣四分之一。
\VS{14}无论是饼,是烘的子粒,是新穗子,你们都不可吃,直等到把你们献给 神的供物带来的那一天才可以吃。这在你们一切的住处作为世世代代永远的定例。」
\par }{\SH 五旬收获节
\par }{\R (民28·26—31)
\par }{\PP \VS{15}「你们要从安息日的次日,献禾捆为摇祭的那日算起,要满了七个安息日。
\VS{16}到第七个安息日的次日,共计五十天,又要将新素祭献给耶和华。
\VS{17}要从你们的住处取出细面{\ADD{伊法}}十分之二,加酵,烤成两个摇祭的饼,当作初熟之物献给耶和华。
\VS{18}又要将一岁、没有残疾的羊羔七只、公牛犊一只、公绵羊两只,和饼一同奉上。这些与同献的素祭和奠祭要作为燔祭献给耶和华,就是作馨香的火祭献给耶和华。
\VS{19}你们要献一只公山羊为赎罪祭,两只一岁的公绵羊羔为平安祭。
\VS{20}祭司要把这些和初熟{\ADD{麦子做}}的饼一同作摇祭,在耶和华面前摇一摇;这是献与耶和华为圣物归给祭司的。
\VS{21}当这日,你们要宣告圣会;什么劳碌的工都不可做。这在你们一切的住处作为世世代代永远的定例。
\par }{\PP \VS{22}「在你们的地收割庄稼,不可割尽田角,也不可拾取所遗落的;要留给穷人和寄居的。我是耶和华—你们的 神。」
\par }{\SH 新年
\par }{\R (民29·1—6)
\par }{\PP \VS{23}耶和华对{\PN{摩西}}说:
\VS{24}「你晓谕{\PN{以色列}}人说:七月初一,你们要守为圣安息日,要吹角作纪念,当有圣会。
\VS{25}什么劳碌的工都不可做;要将火祭献给耶和华。」
\par }{\SH 赎罪日
\par }{\R (民29·7—11)
\par }{\PP \VS{26}耶和华晓谕{\PN{摩西}}说:
\VS{27}「七月初十是赎罪日;你们要守为圣会,并要刻苦己心,也要将火祭献给耶和华。
\VS{28}当这日,什么工都不可做;因为是赎罪日,要在耶和华—你们的 神面前赎罪。
\VS{29}当这日,凡不刻苦己心的,必从民中剪除。
\VS{30}凡这日做什么工的,我必将他从民中除灭。
\VS{31}你们什么工都不可做。这在你们一切的住处作为世世代代永远的定例。
\VS{32}你们要守这日为圣安息日,并要刻苦己心。从这月初九日晚上到{\ADD{次日}}晚上,要守为安息日。」
\par }{\SH 住棚节
\par }{\R (民29·12—40)
\par }{\PP \VS{33}耶和华对{\PN{摩西}}说:
\VS{34}「你晓谕{\PN{以色列}}人说:这七月十五日是住棚节,要在耶和华面前守这节七日。
\VS{35}第一日当有圣会,什么劳碌的工都不可做。
\VS{36}七日内要将火祭献给耶和华。第八日当守圣会,要将火祭献给耶和华。这是严肃会,什么劳碌的工都不可做。
\par }{\PP \VS{37}「这是耶和华的节期,就是你们要宣告为圣会的节期;要将火祭、燔祭、素祭、祭物,并奠祭,各归各日,献给耶和华。
\VS{38}这是在耶和华的安息日以外,又在你们的供物和所许的愿,并甘心献给耶和华的以外。
\par }{\PP \VS{39}「你们收藏了地的出产,就从七月十五日起,要守耶和华的节七日。第一日为圣安息;第八日也为圣安息。
\VS{40}第一日要拿美好树上的果子和棕树上的枝子,与茂密树的枝条并河旁的柳枝,在耶和华—你们的 神面前欢乐七日。
\VS{41}每年七月间,要向耶和华守这节七日。这为你们世世代代永远的定例。
\VS{42}你们要住在棚里七日;凡{\PN{以色列}}家的人都要住在棚里,
\VS{43}好叫你们世世代代知道,我领{\PN{以色列}}人出{\PN{埃及}}地的时候曾使他们住在棚里。我是耶和华—你们的 神。」
\par }{\PP \VS{44}于是,{\PN{摩西}}将耶和华的节期传给{\PN{以色列}}人。

\par }\Chap{24}{\SH 燃灯的条例
\par }{\R (出27·20—21)
\par }{\PP \VerseOne{1}耶和华晓谕{\PN{摩西}}说:
\VS{2}「要吩咐{\PN{以色列}}人,把那为点灯捣成的清橄榄油拿来给你,使灯常常点着。
\VS{3}在会幕中法{\ADD{柜}}的幔子外,{\PN{亚伦}}从晚上到早晨必在耶和华面前经理这灯。这要作你们世世代代永远的定例。
\VS{4}他要在耶和华面前常收拾精{\ADD{金}}灯台上的灯。」
\par }{\SH 陈设饼
\par }{\PP \VS{5}「你要取细面,烤成十二个饼,每饼用面{\ADD{伊法}}十分之二。
\VS{6}要把饼摆列两行\FTNT{}{{\FR 24:6: }或译:摞;下同},每行六个,在耶和华面前精{\ADD{金}}的桌子上;
\VS{7}又要把净乳香放在每行饼上,作为纪念,就是作为火祭献给耶和华。
\VS{8}每安息日要常摆在耶和华面前;这为{\PN{以色列}}人作永远的约。
\VS{9}这饼是要给{\PN{亚伦}}和他子孙的;他们要在圣处吃,为永远的定例,因为在献给耶和华的火祭中是至圣的。」
\par }{\SH 公正的判例
\par }{\PP \VS{10}有一个{\PN{以色列}}妇人的儿子,他父亲是{\PN{埃及}}人,{\ADD{一日}}闲游在{\PN{以色列}}人中。这{\PN{以色列}}妇人的儿子和一个{\PN{以色列}}人在营里争斗。
\VS{11}这{\PN{以色列}}妇人的儿子亵渎了{\ADD{圣}}名,并且咒诅,就有人把他送到{\PN{摩西}}那里。(他母亲名叫{\PN{示罗密}},是{\PN{但}}支派{\PN{底伯利}}的女儿。)
\VS{12}他们把那人收在监里,要得耶和华所指示的话。
\VS{13}耶和华晓谕{\PN{摩西}}说:
\VS{14}「把那咒诅{\ADD{圣名}}的人带到营外。叫听见的人都放手在他头上;全会众就要用石头打死他。
\VS{15}你要晓谕{\PN{以色列}}人说:凡咒诅 神的,必担当他的罪。
\VS{16}那亵渎耶和华名的,必被治死;全会众总要用石头打死他。不管是寄居的是本地人,他亵渎{\ADD{耶和华}}名的时候,必被治死。
\VS{17}打死人的,必被治死;
\VS{18}打死牲畜的,必赔上牲畜,以命偿命。
\VS{19}人若使他邻舍的身体有残疾,他怎样行,也要照样向他行:
\VS{20}以伤还伤,以眼还眼,以牙还牙。他怎样叫人的身体有残疾,也要照样向他行。
\VS{21}打死牲畜的,必赔上牲畜;打死人的,必被治死。
\VS{22}不管是寄居的是本地人,同归一例。我是耶和华—你们的 神。」
\par }{\PP \VS{23}于是,{\PN{摩西}}晓谕{\PN{以色列}}人,他们就把那咒诅{\ADD{圣名}}的人带到营外,用石头打死。{\PN{以色列}}人就照耶和华所吩咐{\PN{摩西}}的行了。

\par }\Chap{25}{\SH 安息年
\par }{\R (申15·1—11)
\par }{\PP \VerseOne{1}耶和华在{\PN{西奈山}}对{\PN{摩西}}说:
\VS{2}「你晓谕{\PN{以色列}}人说:你们到了我所赐你们那地的时候,地就要向耶和华守安息。
\VS{3}六年要耕种田地,也要修理葡萄园,收藏地的出产。
\VS{4}第七年,地要守圣安息,就是向耶和华守的安息,不可耕种田地,也不可修理葡萄园。
\VS{5}遗落自长的庄稼不可收割;没有修理的葡萄树也不可摘取葡萄。这年,地要守圣安息。
\VS{6}地在安息年所出的,要给你和你的仆人、婢女、雇工人,并寄居的外人当食物。
\VS{7}这年的土产也要给你的牲畜和你地上的走兽当食物。」
\par }{\SH 禧年
\par }{\PP \VS{8}「你要计算七个安息年,就是七七年。这便为你成了七个安息年,共是四十九年。
\VS{9}当年七月初十日,你要大发角声;这日就是赎罪日,要在遍地发出角声。
\VS{10}第五十年,你们要当作圣年,在遍地给一切的居民宣告自由。这年必为你们的禧年,各人要归自己的产业,各归本家。
\VS{11}第五十年要作为你们的禧年。这年不可耕种;地中自长的,不可收割;没有修理的葡萄树也不可摘取{\ADD{葡萄}}。
\VS{12}因为这是禧年,你们要当作圣年,吃地中{\ADD{自出}}的土产。
\par }{\PP \VS{13}「这禧年,你们各人要归自己的地业。
\VS{14}你若卖什么给邻舍,或是从邻舍的手中买什么,彼此不可亏负。
\VS{15}你要按禧年以后的年数向邻舍买;他也要按年数的收成卖给你。
\VS{16}年岁若多,要照数加添价值;年岁若少,要照数减去价值,因为他照收成的数目卖给你。
\VS{17}你们彼此不可亏负,只要敬畏你们的 神,因为我是耶和华—你们的 神。」
\par }{\SH 安息年的问题
\par }{\PP \VS{18}「我的律例,你们要遵行,我的典章,你们要谨守,就可以在那地上安然居住。
\VS{19}地必出土产,你们就要吃饱,在那地上安然居住。
\VS{20}你们若说:『这第七年我们不耕种,也不收藏土产,吃什么呢?』
\VS{21}我必在第六年将我所命的福赐给你们,地便生三年的土产。
\VS{22}第八年,你们要耕种,也要吃陈粮,等到第九年出产收来的时候,你们还吃陈粮。」
\par }{\SH 赎房地产的条例
\par }{\PP \VS{23}「地不可永卖,因为地是我的;你们在我面前是客旅,是寄居的。
\VS{24}在你们所得为业的全地,也要准人将地赎回。
\VS{25}你的弟兄\FTNT{}{{\FR 25:25: }弟兄是指本国人说;下同}若渐渐穷乏,卖了几分地业,他至近的亲属就要来把弟兄所卖的赎回。
\VS{26}若没有能给他赎回的,他自己渐渐富足,能够赎回,
\VS{27}就要算出卖地的年数,把余剩年数的价值还那买主,自己便归回自己的地业。
\VS{28}倘若不能为自己得回所卖的,仍要存在买主的手里直到禧年;到了禧年,地业要出{\ADD{买主的手}},自己便归回自己的地业。
\par }{\PP \VS{29}「人若卖城内的住宅,卖了以后,一年之内可以赎回;在一整年,必有赎回的权柄。
\VS{30}若在一整年之内不赎回,这城内的房屋就定准永归买主,世世代代为业;在禧年也不得出{\ADD{买主的手}}。
\VS{31}但房屋在无城墙的村庄里,要看如乡下的田地一样,可以赎回;到了禧年,都要出{\ADD{买主的手}}。
\VS{32}然而{\PN{利未}}人所得为业的城邑,其中的房屋,{\PN{利未}}人可以随时赎回。
\VS{33}若是一个{\PN{利未}}人不将所卖的房屋赎回,是在所得为业的城内,到了禧年就要出{\ADD{买主的手}},因为{\PN{利未}}人城邑的房屋是他们在{\PN{以色列}}人中的产业。
\VS{34}只是他们各城郊野之地不可卖,因为是他们永远的产业。」
\par }{\SH 借贷给穷人的条例
\par }{\PP \VS{35}「你的弟兄在你那里若渐渐贫穷,手中缺乏,你就要帮补他,使他与你同住,像外人和寄居的一样。
\VS{36}不可向他取利,也不可向他多要;只要敬畏你的 神,使你的弟兄与你同住。
\VS{37}你借钱给他,不可向他取利;借粮给他,也不可向他多要。
\VS{38}我是耶和华—你们的 神,曾领你们从{\PN{埃及}}地出来,为要把{\PN{迦南}}地赐给你们,要作你们的 神。」
\par }{\SH 赎回奴仆的条例
\par }{\PP \VS{39}「你的弟兄若在你那里渐渐穷乏,将自己卖给你,不可叫他像奴仆服事你。
\VS{40}他要在你那里像雇工人和寄居的一样,要服事你直到禧年。
\VS{41}到了禧年,他和他儿女要离开你,一同出去归回本家,到他祖宗的地业那里去。
\VS{42}因为他们是我的仆人,是我从{\PN{埃及}}地领出来的,不可卖为奴仆。
\VS{43}不可严严地辖管他,只要敬畏你的 神。
\VS{44}至于你的奴仆、婢女,可以从你四围的国中买。
\VS{45}并且那寄居在你们中间的外人和他们的家属,在你们地上所生的,你们也可以从其中买人;他们要作你们的产业。
\VS{46}你们要将他们遗留给你们的子孙为产业,要永远从他们中间拣出奴仆;只是你们的弟兄{\PN{以色列}}人,你们不可严严地辖管。
\par }{\PP \VS{47}「住在你那里的外人,或是寄居的,若渐渐富足,你的弟兄却渐渐穷乏,将自己卖给那外人,或是寄居的,或是外人的宗族,
\VS{48}卖了以后,可以将他赎回。无论是他的弟兄,
\VS{49}或伯叔、伯叔的儿子,本家的近支,都可以赎他。他自己若渐渐富足,也可以自赎。
\VS{50}他要和买主计算,从卖自己的那年起,算到禧年;所卖的价值照着年数多少,好像工人每年的工价。
\VS{51}若缺少的年数多,就要按着年数从买价中偿还他的赎价。
\VS{52}若到禧年只缺少几年,就要按着年数和买主计算,偿还他的赎价。
\VS{53}他和买主同住,要像每年雇的工人,买主不可严严地辖管他。
\VS{54}他若不这样被赎,到了禧年,要和他的儿女一同出去。
\VS{55}因为{\PN{以色列}}人都是我的仆人,是我从{\PN{埃及}}地领出来的。我是耶和华—你们的 神。」

\par }\Chap{26}{\SH 遵行诫命者的福祉
\par }{\R (申7·12—24;28·1—14)
\par }{\PP \VerseOne{1}「你们不可做什么虚无的神{\ADD{像}},不可立雕刻的偶像或是柱像,也不可在你们的地上安什么錾成的石像,向它跪拜,因为我是耶和华—你们的 神。
\VS{2}你们要守我的安息日,敬我的圣所。我是耶和华。
\par }{\PP \VS{3}「你们若遵行我的律例,谨守我的诫命,
\VS{4}我就给你们降下时雨,叫地生出土产,田野的树木结果子。
\VS{5}你们打粮食要打到摘葡萄的时候,摘葡萄要摘到撒种的时候;并且要吃得饱足,在你们的地上安然居住。
\VS{6}我要赐平安在{\ADD{你们的}}地上;你们躺卧,无人惊吓。我要叫恶兽从{\ADD{你们的}}地上息灭;刀剑也必不经过你们的地。
\VS{7}你们要追赶仇敌,他们必倒在你们刀下。
\VS{8}你们五个人要追赶一百人,一百人要追赶一万人;仇敌必倒在你们刀下。
\VS{9}我要眷顾你们,使你们生养众多,也要与你们坚定所立的约。
\VS{10}你们要吃陈粮,又因新粮挪开陈粮。
\VS{11}我要在你们中间立我的帐幕;我的心也不厌恶你们。
\VS{12}我要在你们中间行走;我要作你们的 神,你们要作我的子民。
\VS{13}我是耶和华—你们的 神,曾将你们从{\PN{埃及}}地领出来,使你们不作{\PN{埃及}}人的奴仆;我也折断你们所负的轭,叫你们挺身而走。」
\par }{\SH 违反诫命者受惩罚
\par }{\R (申28·15—68)
\par }{\PP \VS{14}「你们若不听从我,不遵行我的诫命,
\VS{15}厌弃我的律例,厌恶我的典章,不遵行我一切的诫命,背弃我的约,
\VS{16}我待你们就要这样:我必命定惊惶,叫眼目干瘪、精神消耗的痨病热病辖制你们。你们也要白白地撒种,因为仇敌要吃你们所种的。
\VS{17}我要向你们变脸,你们就要败在仇敌面前。恨恶你们的,必辖管你们;无人追赶,你们却要逃跑。
\VS{18}你们因这些事若还不听从我,我就要为你们的罪加七倍惩罚你们。
\VS{19}我必断绝你们因势力而有的骄傲,又要使{\ADD{覆}}你们的天如铁,{\ADD{载}}你们的地如铜。
\VS{20}你们要白白地劳力;因为你们的地不出土产,其上的树木也不结果子。
\par }{\PP \VS{21}「你们行事若与我反对,不肯听从我,我就要按你们的罪加七倍降灾与你们。
\VS{22}我也要打发野地的走兽到你们中间,抢吃你们的儿女,吞灭你们的牲畜,使你们的人数减少,道路荒凉。
\par }{\PP \VS{23}「你们因这些事若仍不改正归我,行事与我反对,
\VS{24}我就要行事与你们反对,因你们的罪击打你们七次。
\VS{25}我又要使刀剑临到你们,报复你们背约的仇;聚集你们在各城内,降瘟疫在你们中间,也必将你们交在仇敌的手中。
\VS{26}我要折断你们的杖,就是断绝你们的粮。那时,必有十个女人在一个炉子给你们烤饼,按分量秤给你们;你们要吃,也吃不饱。
\par }{\PP \VS{27}「你们因这一切的事若不听从我,却行事与我反对,
\VS{28}我就要发烈怒,行事与你们反对,又因你们的罪惩罚你们七次。
\VS{29}并且你们要吃儿子的肉,也要吃女儿的肉。
\VS{30}我又要毁坏你们的邱坛,砍下你们的日像,把你们的尸首扔在你们偶像的身上;我的心也必厌恶你们。
\VS{31}我要使你们的城邑变为荒凉,使你们的众圣所成为荒场;我也不闻你们馨香的香气。
\VS{32}我要使地成为荒场,住在其上的仇敌就因此诧异。
\VS{33}我要把你们散在列邦中;我也要拔刀追赶你们。你们的地要成为荒场;你们的城邑要变为荒凉。
\par }{\PP \VS{34}「你们在仇敌之地居住的时候,你们的地荒凉,要享受众安息;正在那时候,地要歇息,享受安息。
\VS{35}地多时为荒场,就要多时歇息;地这样歇息,是你们住在其上的安息年所不能得的。
\VS{36}至于你们剩下的人,我要使他们在仇敌之地心惊胆怯。叶子被风吹的响声,要追赶他们;他们要逃避,像人逃避刀剑,无人追赶,却要跌倒。
\VS{37}无人追赶,他们要彼此撞跌,像在刀剑之前。你们在仇敌面前也必站立不住。
\VS{38}你们要在列邦中灭亡;仇敌之地要吞吃你们。
\VS{39}你们剩下的人必因自己的罪孽和祖宗的罪孽在仇敌之地消灭。
\par }{\PP \VS{40}「他们要承认自己的罪和他们祖宗的罪,就是干犯我的那罪,并且承认自己行事与我反对,
\VS{41}我所以行事与他们反对,把他们带到仇敌之地。那时,他们未受割礼的心若谦卑了,他们也服了罪孽的刑罚,
\VS{42}我就要记念我与{\PN{雅各}}所立的约,与{\PN{以撒}}所立的约,与{\PN{亚伯拉罕}}所立的约,并要记念这地。
\VS{43}他们离开这地,地在荒废无人的时候就要享受安息。并且他们要服罪孽的刑罚;因为他们厌弃了我的典章,心中厌恶了我的律例。
\VS{44}虽是这样,他们在仇敌之地,我却不厌弃他们,也不厌恶他们,将他们尽行灭绝,也不背弃我与他们所立的约,因为我是耶和华—他们的 神。
\VS{45}我却要为他们的缘故记念我与他们先祖所立的约。他们的先祖是我在列邦人眼前、从{\PN{埃及}}地领出来的,为要作他们的 神。我是耶和华。」
\VS{46}这些律例、典章,和法度是耶和华与{\PN{以色列}}人在{\PN{西奈山}}借着{\PN{摩西}}立的。

\par }\Chap{27}{\SH 许愿的条例
\par }{\PP \VerseOne{1}耶和华对{\PN{摩西}}说:
\VS{2}「你晓谕{\PN{以色列}}人说:人还特许的愿,被许的人要按你所估的价值归给耶和华。
\VS{3}你估定的,从二十岁到六十岁的男人,要按圣所的平,估定价银五十舍客勒。
\VS{4}若是女人,你要估定三十舍客勒。
\VS{5}若是从五岁到二十岁,男子你要估定二十舍客勒,女子估定十舍客勒。
\VS{6}若是从一月到五岁,男子你要估定五舍客勒,女子估定三舍客勒。
\VS{7}若是从六十岁以上,男人你要估定十五舍客勒,女人估定十舍客勒。
\VS{8}他若贫穷,不能照你所估定的价,就要把他带到祭司面前,祭司要按许愿人的力量估定他的价。
\par }{\PP \VS{9}「{\ADD{所许的}}若是牲畜,就是人献给耶和华为供物的,凡这一类献给耶和华的,都要成为圣。
\VS{10}人不可改换,也不可更换,或是好的换坏的,或是坏的换好的。若以牲畜更换牲畜,所许的与所换的都要成为圣。
\VS{11}若牲畜不洁净,是不可献给耶和华为供物的,就要把牲畜安置在祭司面前。
\VS{12}祭司就要估定价值;牲畜是好是坏,祭司怎样估定,就要以怎样为是。
\VS{13}他若一定要赎回,就要在你所估定的价值以外加上五分之一。
\par }{\PP \VS{14}「人将房屋分别为圣,归给耶和华,祭司就要估定价值。房屋是好是坏,祭司怎样估定,就要以怎样为定。
\VS{15}将房屋分别为圣的人,若要赎回房屋,就必在你所估定的价值以外加上五分之一,房屋仍旧归他。
\par }{\PP \VS{16}「人若将承受为业的几分地分别为圣,归给耶和华,你要按这地撒种多少估定价值,若撒大麦一贺梅珥,{\ADD{要估}}价五十舍客勒。
\VS{17}他若从禧年将地分别为圣,就要以你所估定的价为定。
\VS{18}倘若他在禧年以后将地分别为圣,祭司就要按着未到禧年所剩的年数推算价值,也要从你所估的减去价值。
\VS{19}将地分别为圣的人若定要把地赎回,他便要在你所估的价值以外加上五分之一,地就准定归他。
\VS{20}他若不赎回那地,或是将地卖给别人,就再不能赎了。
\VS{21}但到了禧年,那地从{\ADD{买主手下}}出来的时候,就要归耶和华为圣,和永献的地一样,要归祭司为业。
\VS{22}他若将所买的一块地,不是承受为业的,分别为圣归给耶和华,
\VS{23}祭司就要将你所估的价值给他推算到禧年。当日,他要以你所估的价银为圣,归给耶和华。
\VS{24}到了禧年,那地要归卖主,就是那承受为业的原主。
\VS{25}凡你所估定的价银都要按着圣所的平:二十季拉为一舍客勒。
\par }{\PP \VS{26}「惟独牲畜中头生的,无论是牛是羊,既归耶和华,谁也不可再分别为圣,因为这是耶和华的。
\VS{27}若是不洁净的牲畜生的,就要按你所估定的价值加上五分之一赎回;若不赎回,就要按你所估定的价值卖了。
\par }{\PP \VS{28}「但一切永献的,就是人从他所有永献给耶和华的,无论是人,是牲畜,是他承受为业的地,都不可卖,也不可赎。凡永献的是归给耶和华为至圣。
\VS{29}凡从人中当灭的都不可赎,必被治死。」
\par }{\PP \VS{30}「地上所有的,无论是地上的种子是树上的果子,十分之一是耶和华的,是归给耶和华为圣的。
\VS{31}人若要赎这十分之一的什么物,就要加上五分之一。
\VS{32}凡牛群羊群中,一切从杖下经过的,每第十只要归给耶和华为圣。
\VS{33}不可问是好是坏,也不可更换;若定要更换,所更换的与本来的牲畜都要成为圣,不可赎回。」
\par }{\PP \VS{34}这就是耶和华在{\PN{西奈山}}为{\PN{以色列}}人所吩咐{\PN{摩西}}的命令。
\par }