\NormalFont\ShortTitle{马可福音}
{\MT 马可福音

\par }\ChapOne{1}{\SH 施洗约翰传道
\par }{\R (太3·1—12;路3·1—18;约1·19—28)
\par }{\PP \VerseOne{1}神的儿子,耶稣基督福音的起头。
\VS{2}正如先知{\PN{以赛亚}}\FTNT{}{{\FR 1:2: }有古卷没有以赛亚三个字}书上记着说:
\par }{\Q 看哪,我要差遣我的使者在你前面,
\par }{\Q 预备道路。
\par }{\Q \VS{3}在旷野有人声喊着说:
\par }{\Q 预备主的道,
\par }{\Q 修直他的路。
\par }{\PP \VS{4}{\ADD{照这话}},{\PN{约翰}}来了,在旷野施洗,传悔改的洗礼,使罪得赦。
\VS{5}{\PN{犹太}}全地和{\PN{耶路撒冷}}的人都出去到{\PN{约翰}}那里,承认他们的罪,在{\PN{约旦河}}里受他的洗。
\VS{6}{\PN{约翰}}穿骆驼毛{\ADD{的衣服}},腰束皮带,吃的是蝗虫、野蜜。
\VS{7}他传道说:「有一位在我以后来的,能力比我更大,我就是弯腰给他解鞋带也是不配的。
\VS{8}我是用水给你们施洗,他却要用圣灵给你们施洗。」
\par }{\SH 耶稣受洗
\par }{\R (太3·13—17;路3·21—22)
\par }{\PP \VS{9}那时,耶稣从{\PN{加利利}}的{\PN{拿撒勒}}来,在{\PN{约旦河}}里受了
{\PN{约翰}}的洗。
\VS{10}他从水里一上来,就看见天裂开了,圣灵仿佛鸽子,降在他身上。
\VS{11}又有声音从天上来,{\ADD{说}}:「你是我的爱子,我喜悦你。」
\par }{\SH 耶稣受试探
\par }{\R (太4·1—11;路4·1—13)
\par }{\PP \VS{12}圣灵就把耶稣催到旷野里去。
\VS{13}他在旷野四十天,受撒但的试探,并与野兽同在一处,且有天使来伺候他。
\par }{\SH 开始在加利利传道
\par }{\R (太4·12—17;路4·14—15)
\par }{\PP \VS{14}{\PN{约翰}}下监以后,耶稣来到{\PN{加利利}},宣传 神的福音,
\VS{15}说:「日期满了, 神的国近了。你们当悔改,信福音!」
\par }{\SH 呼召四个渔夫
\par }{\R (太4·18—22;路5·1—11)
\par }{\PP \VS{16}耶稣顺着{\PN{加利利}}的海边走,看见{\PN{西门}}和{\PN{西门}}的兄弟{\PN{安得烈}}在海里撒网;他们本是打鱼的。
\VS{17}耶稣对他们说:「来跟从我,我要叫你们得人如得鱼一样。」
\VS{18}他们就立刻舍了网,跟从了他。
\VS{19}耶稣稍往前走,又见{\PN{西庇太}}的儿子{\PN{雅各}}和{\PN{雅各}}的兄弟{\PN{约翰}}在船上补网。
\VS{20}耶稣随即招呼他们,他们就把父亲
{\PN{西庇太}}和雇工人留在船上,跟从耶稣去了。
\par }{\SH 一个污鬼附身的人
\par }{\R (路4·31—37)
\par }{\PP \VS{21}到了{\PN{迦百农}},耶稣就在安息日进了会堂教训人。
\VS{22}众人很希奇他的教训;因为他教训他们,正像有权柄的人,不像文士。
\VS{23}在会堂里,有一个人被污鬼附着。他喊叫说:
\VS{24}「{\PN{拿撒勒}}人耶稣,我们与你有什么相干?你来灭我们吗?我知道你是谁,乃是 神的圣者。」
\VS{25}耶稣责备他说:「不要作声!从这人身上出来吧。」
\VS{26}污鬼叫那人抽了一阵风,大声喊叫,就出来了。
\VS{27}众人都惊讶,以致彼此对问说:「这是什么事?是个新道理啊!他用权柄吩咐污鬼,连污鬼也听从了他。」
\VS{28}耶稣的名声就传遍了{\PN{加利利}}的四方。
\par }{\SH 治好许多病人
\par }{\R (太8·14—17;路4·38—41)
\par }{\PP \VS{29}他们一出会堂,就同着{\PN{雅各}}、{\PN{约翰}},进了{\PN{西门}}和{\PN{安得烈}}的家。
\VS{30}{\PN{西门}}的岳母正害热病躺着,就有人告诉耶稣。
\VS{31}耶稣进前拉着她的手,扶她起来,热就退了,她就服事他们。
\VS{32}天晚日落的时候,有人带着一切害病的,和被鬼附的,来到耶稣跟前。
\VS{33}合城的人都聚集在门前。
\VS{34}耶稣治好了许多害各样病的人,又赶出许多鬼,不许鬼说话,因为鬼认识他。
\par }{\SH 在加利利各会堂传道
\par }{\R (路4·42—44)
\par }{\PP \VS{35}次日早晨,天未亮的时候,耶稣起来,到旷野地方去,在那里祷告。
\VS{36}{\PN{西门}}和同伴追了他去,
\VS{37}遇见了就对他说:「众人都找你。」
\VS{38}耶稣对他们说:「我们可以往别处去,到邻近的乡村,我也好在那里传道,因为我是为这事出来的。」
\VS{39}于是在{\PN{加利利}}全地,进了会堂,传道,赶鬼。
\par }{\SH 洁净长大麻风的人
\par }{\R (太8·1—4;路5·12—16)
\par }{\PP \VS{40}有一个长大麻风的来求耶稣,向他跪下,说:「你若肯,必能叫我洁净了。」
\VS{41}耶稣动了慈心,就伸手摸他,说:「我肯,你洁净了吧!」
\VS{42}大麻风即时离开他,他就洁净了。
\VS{43}耶稣严严地嘱咐他,就打发他走,
\VS{44}对他说:「你要谨慎,什么话都不可告诉人,只要去把身体给祭司察看,又因为你洁净了,献上{\PN{摩西}}所吩咐的礼物,对众人作证据。」
\VS{45}那人出去,倒说许多的话,把这件事传扬开了,叫耶稣以后不得再明明地进城,只好在外边旷野地方。人从各处都就了他来。

\par }\Chap{2}{\SH 治好瘫痪病人
\par }{\R (太9·1—8;路5·17—26)
\par }{\PP \VerseOne{1}过了些日子,耶稣又进了{\PN{迦百农}}。人听见他在房子里,
\VS{2}就有许多人聚集,甚至连门前都没有空地;耶稣就对他们讲道。
\VS{3}有人带着一个瘫子来见耶稣,是用四个人抬来的;
\VS{4}因为人多,不得近前,就把耶稣所在的房子,拆了房顶,既拆通了,就把瘫子连所躺卧的褥子都缒下来。
\VS{5}耶稣见他们的信心,就对瘫子说:「小子,你的罪赦了。」
\VS{6}有几个文士坐在那里,心里议论,说:
\VS{7}「这个人为什么这样说呢?他说僭妄的话了。除了 神以外,谁能赦罪呢?」
\VS{8}耶稣心中知道他们心里这样议论,就说:「你们心里为什么这样议论呢?
\VS{9}或对瘫子说『你的罪赦了』,或说『起来!拿你的褥子行走』,哪一样容易呢?
\VS{10}但要叫你们知道,人子在地上有赦罪的权柄。」就对瘫子说:
\VS{11}「我吩咐你,起来!拿你的褥子回家去吧。」
\VS{12}那人就起来,立刻拿着褥子,当众人面前出去了,以致众人都惊奇,归荣耀与 神,说:「我们从来没有见过这样的事!」
\par }{\SH 呼召利未
\par }{\R (太9·9—13;路5·27—32)
\par }{\PP \VS{13}耶稣又出到海边去,众人都就了他来,他便教训他们。
\VS{14}耶稣经过的时候,看见{\PN{亚勒腓}}的儿子{\PN{利未}}坐在税关上,就对他说:「你跟从我来。」他就起来,跟从了耶稣。
\VS{15}耶稣在{\PN{利未}}家里坐席的时候,有好些税吏和罪人与耶稣并门徒一同坐席;因为这样的人多,他们也跟随耶稣。
\VS{16}法利赛人中的文士\FTNT{}{{\FR 2:16: }有古卷:文士和法利赛人}看见耶稣和罪人并税吏一同吃饭,就对他门徒说:「他和税吏并罪人一同吃喝吗?」
\VS{17}耶稣听见,就对他们说:「康健的人用不着医生,有病的人才用得着。我来本不是召义人,乃是召罪人。」
\par }{\SH 禁食的问题
\par }{\R (太9·14—17;路5·33—39)
\par }{\PP \VS{18}当下,{\PN{约翰}}的门徒和法利赛人禁食。他们来问耶稣说:「{\PN{约翰}}的门徒和法利赛人的门徒禁食,你的门徒倒不禁食,这是为什么呢?」
\VS{19}耶稣对他们说:「新郎和陪伴之人同在的时候,陪伴之人岂能禁食呢?新郎还同在,他们不能禁食。
\VS{20}但日子将到,新郎要离开他们,那日他们就要禁食。
\VS{21}没有人把新布缝在旧衣服上,恐怕所补上的新布带坏了旧衣服,破的就更大了。
\VS{22}也没有人把新酒装在旧皮袋里,恐怕酒把皮袋裂开,酒和皮袋就都坏了;惟把新酒装在新皮袋里。」
\par }{\SH 安息日的问题
\par }{\R (太12·1—8;路6·1—5)
\par }{\PP \VS{23}耶稣当安息日从麦地经过。他门徒行路的时候,掐了麦穗。
\VS{24}法利赛人对耶稣说:「看哪,他们在安息日为什么做不可做的事呢?」
\VS{25}耶稣对他们说:「{\ADD{经上记着}}{\PN{大卫}}和跟从他的人缺乏饥饿之时所做的事,你们没有念过吗?
\VS{26}他当{\PN{亚比亚他}}作大祭司的时候,怎么进了 神的殿,吃了陈设饼,又给跟从他的人吃。这饼除了祭司以外,人都不可吃。」
\VS{27}又对他们说:「安息日是为人设立的,人不是为安息日设立的。
\VS{28}所以,人子也是安息日的主。」

\par }\Chap{3}{\SH 治好枯干了一只手的人
\par }{\R (太12·9—14;路6·6—11)
\par }{\PP \VerseOne{1}耶稣又进了会堂,在那里有一个人枯干了一只手。
\VS{2}众人窥探耶稣,在安息日医治不医治,意思是要控告耶稣。
\VS{3}耶稣对那枯干一只手的人说:「起来,站在当中。」
\VS{4}又问众人说:「在安息日行善行恶,救命害命,哪样是可以的呢?」他们都不作声。
\VS{5}耶稣怒目周围看他们,忧愁他们的心刚硬,就对那人说:「伸出手来!」他把手一伸,手就复了原。
\VS{6}法利赛人出去,同{\PN{希律}}一党的人商议怎样可以除灭耶稣。
\par }{\SH 海边的群众
\par }{\PP \VS{7}耶稣和门徒退到海边去,有许多人从{\PN{加利利}}跟随他。
\VS{8}还有许多人听见他所做的大事,就从{\PN{犹太}}、{\PN{耶路撒冷}}、{\PN{以土买}}、{\PN{约旦河}}外,并{\PN{泰尔}}、{\PN{西顿}}的四方来到他那里。
\VS{9}他因为人多,就吩咐门徒叫一只小船伺候着,免得众人拥挤他。
\VS{10}他治好了许多人,所以凡有灾病的,都挤进来要摸他。
\VS{11}污鬼无论何时看见他,就俯伏在他面前,喊着说:「你是 神的儿子。」
\VS{12}耶稣再三地嘱咐他们,不要把他显露出来。
\par }{\SH 设立十二门徒
\par }{\R (太10·1—4;路6·12—16)
\par }{\PP \VS{13}耶稣上了山,随自己的意思叫人来;他们便来到他那里。
\VS{14}他就设立十二个人,要他们常和自己同在,也要差他们去传道,
\VS{15}并给他们权柄赶鬼。
\VS{16}{\ADD{这十二个人}}有{\PN{西门}}(耶稣又给他起名叫{\PN{彼得}}),
\VS{17}还有{\PN{西庇太}}的儿子{\PN{雅各}}和{\PN{雅各}}的兄弟{\PN{约翰}}(又给这两个人起名叫{\PN{半尼其}},就是雷子的意思),
\VS{18}又有{\PN{安得烈}}、{\PN{腓力}}、{\PN{巴多罗买}}、{\PN{马太}}、{\PN{多马}}、{\PN{亚勒腓}}的儿子{\PN{雅各}},和{\PN{达太}},并奋锐党的{\PN{西门}},
\VS{19}还有卖耶稣的{\PN{加略}}人{\PN{犹大}}。
\par }{\SH 耶稣和别西卜
\par }{\R (太12·22—32;路11·14—23;12·10)
\par }{\PP \VS{20}耶稣进了一个屋子,众人又聚集,甚至他连饭也顾不得吃。
\VS{21}耶稣的亲属听见,就出来要拉住他,因为他们说他癫狂了。
\VS{22}从{\PN{耶路撒冷}}下来的文士说:「他是被别西卜附着」;又说:「他是靠着鬼王赶鬼。」
\VS{23}耶稣叫他们来,用比喻对他们说:「撒但怎能赶出撒但呢?
\VS{24}若一国自相纷争,那国就站立不住;
\VS{25}若一家自相纷争,那家就站立不住。
\VS{26}若撒但自相攻打纷争,他就站立不住,必要灭亡。
\VS{27}没有人能进壮士家里,抢夺他的家具;必先捆住那壮士,才可以抢夺他的家。
\VS{28}我实在告诉你们,世人一切的罪和一切亵渎的话都可得赦免;
\VS{29}凡亵渎圣灵的,却永不得赦免,乃要担当永远的罪。」
\VS{30}{\ADD{这话是}}因为他们说:「他是被污鬼附着的。」
\par }{\SH 耶稣的母亲和兄弟们
\par }{\R (太12·46—50;路8·19—21)
\par }{\PP \VS{31}当下,耶稣的母亲和弟兄来,站在外边,打发人去叫他。
\VS{32}有许多人在耶稣周围坐着,他们就告诉他说:「看哪,你母亲和你弟兄在外边找你。」
\VS{33}耶稣回答说:「谁是我的母亲?谁是我的弟兄?」
\VS{34}就四面观看那周围坐着的人,说:「看哪,我的母亲,我的弟兄。
\VS{35}凡遵行 神旨意的人就是我的弟兄姊妹和母亲了。」

\par }\Chap{4}{\SH 撒种的比喻
\par }{\R (太13·1—9;路8·4—8)
\par }{\PP \VerseOne{1}耶稣又在海边教训人。有许多人到他那里聚集,他只得上船坐下。船在海里,众人都靠近海,站在岸上。
\VS{2}耶稣就用比喻教训他们许多道理。在教训之间,对他们说:
\VS{3}「你们听啊!有一个撒种的出去撒种。
\VS{4}撒的时候,有落在路旁的,飞鸟来吃尽了;
\VS{5}有落在土浅石头地上的,土既不深,发苗最快,
\VS{6}日头出来一晒,因为没有根,就枯干了;
\VS{7}有落在荆棘里的,荆棘长起来,把它挤住了,就不结实;
\VS{8}又有落在好土里的,就发生长大,结实有三十倍的,有六十倍的,有一百倍的」;
\VS{9}又说:「有耳可听的,就应当听!」
\par }{\SH 用比喻的目的
\par }{\R (太13·10—17;路8·9—10)
\par }{\PP \VS{10}无人的时候,跟随耶稣的人和十二个门徒问他这比喻的意思。
\VS{11}耶稣对他们说:「 神国的奥秘只叫你们知道,若是对外人讲,凡事就用比喻,
\VS{12}叫他们
\par }{\Q 看是看见,却不晓得;
\par }{\Q 听是听见,却不明白;
\par }{\Q 恐怕他们回转过来,就得赦免。」
\par }{\SH 解明撒种的比喻
\par }{\R (太13·18—23;路8·11—15)
\par }{\PP \VS{13}又对他们说:「你们不明白这比喻吗?这样怎能明白一切的比喻呢?
\VS{14}撒种之人所撒的就是道。
\VS{15}那撒在路旁的,就是人听了道,撒但立刻来,把撒在他心里的道夺了去。
\VS{16}那撒在石头地上的,就是人听了道,立刻欢喜领受,
\VS{17}但他心里没有根,不过是暂时的,及至为道遭了患难,或是受了逼迫,立刻就跌倒了。
\VS{18}还有那撒在荆棘里的,就是人听了道,
\VS{19}后来有世上的思虑、钱财的迷惑,和别样的私欲进来,把道挤住了,就不能结实。
\VS{20}那撒在好地上的,就是人听道,又领受,并且结实,有三十倍的,有六十倍的,有一百倍的。」
\par }{\SH 斗底下的灯
\par }{\R (路8·16—18)
\par }{\PP \VS{21}耶稣又对他们说:「人拿灯来,岂是要放在斗底下,床底下,不放在灯台上吗?
\VS{22}因为掩藏的事,没有不显出来的;隐瞒的事,没有不露出来的。
\VS{23}有耳可听的,就应当听!」
\VS{24}又说:「你们所听的要留心。你们用什么量器量给人,也必用什么量器量给你们,并且要多给你们。
\VS{25}因为有的,还要给他;没有的,连他所有的也要夺去。」
\par }{\SH 种子长大的比喻
\par }{\PP \VS{26}又说:「 神的国如同人把种撒在地上。
\VS{27}黑夜睡觉,白日起来,这种就发芽渐长,那人却不晓得如何这样。
\VS{28}地生五谷是出于自然的:先发苗,后长穗,再后穗上结成饱满的子粒;
\VS{29}谷既熟了,就用镰刀去割,因为收成的时候到了。」
\par }{\SH 芥菜种的比喻
\par }{\R (太13·31—32;路13·18—19)
\par }{\PP \VS{30}又说:「 神的国,我们可用什么比较呢?可用什么比喻表明呢?
\VS{31}好像一粒芥菜种,种在地里的时候,虽比地上的百种都小,
\VS{32}但种上以后,就长起来,比各样的菜都大,又长出大枝来,甚至天上的飞鸟可以宿在它的荫下。」
\par }{\SH 耶稣用比喻讲道
\par }{\R (太13·34)
\par }{\PP \VS{33}耶稣用许多这样的比喻,照他们所能听的,对他们讲道。
\VS{34}若不用比喻,就不对他们讲;没有人的时候,就把一切的道讲给门徒听。
\par }{\SH 平静风和海
\par }{\R (太8·23—27;路8·22—25)
\par }{\PP \VS{35}当那天晚上,耶稣对门徒说:「我们渡到那边去吧。」
\VS{36}门徒离开众人,耶稣仍在船上,他们就把他一同带去;也有别的船和他同行。
\VS{37}忽然起了暴风,波浪打入船内,甚至船要满了水。
\VS{38}耶稣在船尾上,枕着枕头睡觉。门徒叫醒了他,说:「夫子!我们丧命,你不顾吗?」
\VS{39}耶稣醒了,斥责风,向海说:「住了吧!静了吧!」风就止住,大大地平静了。
\VS{40}耶稣对他们说:「为什么胆怯?你们还没有信心吗?」
\VS{41}他们就大大地惧怕,彼此说:「这到底是谁,连风和海也听从他了。」

\par }\Chap{5}{\SH 治好格拉森被鬼附的人
\par }{\R (太8·28—34;路8·26—39)
\par }{\PP \VerseOne{1}他们来到海那边{\PN{格拉森}}人的地方。
\VS{2}耶稣一下船,就有一个被污鬼附着的人从坟茔里出来迎着他。
\VS{3}那人常住在坟茔里,没有人能捆住他,就是用铁链也不能;
\VS{4}因为人屡次用脚镣和铁链捆锁他,铁链竟被他挣断了,脚镣也被他弄碎了;总没有人能制伏他。
\VS{5}他昼夜常在坟茔里和山中喊叫,又用石头砍自己。
\VS{6}他远远地看见耶稣,就跑过去拜他,
\VS{7}大声呼叫说:「至高 神的儿子耶稣,我与你有什么相干?我指着 神恳求你,不要叫我受苦!」
\VS{8}是因耶稣曾吩咐他说:「污鬼啊,从这人身上出来吧!」
\VS{9}耶稣问他说:「你名叫什么?」回答说:「我名叫『群』,因为我们多的缘故」;
\VS{10}就再三地求耶稣,不要叫他们离开那地方。
\par }{\PP \VS{11}在那里山坡上,有一大群猪吃食;
\VS{12}鬼就央求耶稣说:「求你打发我们往猪群里,附着猪去。」
\VS{13}耶稣准了他们,污鬼就出来,进入猪里去。于是那群猪闯下山崖,投在海里,淹死了。{\ADD{猪的数目}}约有二千。
\VS{14}放猪的就逃跑了,去告诉城里和乡下的人。众人就来,要看是什么事。
\VS{15}他们来到耶稣那里,看见那被鬼附着的人,就是从前被群鬼所附的,坐着,穿上衣服,心里明白过来,他们就害怕。
\VS{16}看见这事的,便将鬼附之人所遇见的和那群猪的事都告诉了众人;
\VS{17}众人就央求耶稣离开他们的境界。
\VS{18}耶稣上船的时候,那从前被鬼附着的人恳求和耶稣同在。
\VS{19}耶稣不许,却对他说:「你回家去,到你的亲属那里,将主为你所做的是何等大的事,是怎样怜悯你,都告诉他们。」
\VS{20}那人就走了,在{\PN{低加坡里}}传扬耶稣为他做了何等大的事,众人就都希奇。
\par }{\SH 睚鲁的女儿和血漏的女人
\par }{\R (太9·18—26;路8·40—56)
\par }{\PP \VS{21}耶稣坐船又渡到那边去,就有许多人到他那里聚集;他正在海边上。
\VS{22}有一个管会堂的人,名叫{\PN{睚鲁}},来见耶稣,就俯伏在他脚前,
\VS{23}再三地求他,说:「我的小女儿快要死了,{\ADD{求你}}去按手在她身上,使她痊愈,得以活了。」
\VS{24}耶稣就和他同去。
\par }{\PP 有许多人跟随拥挤他。
\VS{25}有一个女人,患了十二年的血漏,
\VS{26}在好些医生手里受了许多的苦,又花尽了她所有的,一点也不见好,病势反倒更重了。
\VS{27}她听见耶稣的事,就从后头来,杂在众人中间,摸耶稣的衣裳,
\VS{28}意思说:「我只摸他的衣裳,就必痊愈。」
\VS{29}于是她血漏的源头立刻干了;她便觉得身上的灾病好了。
\VS{30}耶稣顿时心里觉得有能力从自己身上出去,就在众人中间转过来,说:「谁摸我的衣裳?」
\VS{31}门徒对他说:「你看众人拥挤你,还说『谁摸我』吗?」
\VS{32}耶稣周围观看,要见做这事的女人。
\VS{33}那女人知道在自己身上所成的事,就恐惧战兢,来俯伏在耶稣跟前,将实情全告诉他。
\VS{34}耶稣对她说:「女儿,你的信救了你,平平安安地回去吧!你的灾病痊愈了。」
\par }{\PP \VS{35}还说话的时候,有人从管会堂的家里来,说:「你的女儿死了,何必还劳动先生呢?」
\VS{36}耶稣听见所说的话,就对管会堂的说:「不要怕,只要信!」
\VS{37}于是带着{\PN{彼得}}、{\PN{雅各}},和{\PN{雅各}}的兄弟{\PN{约翰}}同去,不许别人跟随他。
\VS{38}他们来到管会堂的家里;耶稣看见那里乱嚷,并有人大大地哭泣哀号,
\VS{39}进到里面,就对他们说:「为什么乱嚷哭泣呢?孩子不是死了,是睡着了。」
\VS{40}他们就嗤笑耶稣。耶稣把他们都撵出去,就带着孩子的父母和跟随的人进了孩子所在的地方,
\VS{41}就拉着孩子的手,对她说:「大利大,古米!」(翻出来就是说:「闺女,我吩咐你起来!」)
\VS{42}那闺女立时起来走。他们就大大地惊奇;闺女已经十二岁了。
\VS{43}耶稣切切地嘱咐他们,不要叫人知道这事,又吩咐给她东西吃。

\par }\Chap{6}{\SH 拿撒勒人厌弃耶稣
\par }{\R (太13·53—58;路4·16—30)
\par }{\PP \VerseOne{1}耶稣离开那里,来到自己的家乡;门徒也跟从他。
\VS{2}到了安息日,他在会堂里教训人。众人听见,就甚希奇,说:「这人从哪里有这些事呢?所赐给他的是什么智慧?他手所做的是何等的异能呢?
\VS{3}这不是那木匠吗?不是{\PN{马利亚}}的儿子{\PN{雅各}}、{\PN{约西}}、{\PN{犹大}}、{\PN{西门}}的长兄吗?他妹妹们不也是在我们这里吗?」他们就厌弃他\FTNT{}{{\FR 6:3: }厌弃他:原文是因他跌倒}。
\VS{4}耶稣对他们说:「大凡先知,除了本地、亲属、本家之外,没有不被人尊敬的。」
\VS{5}耶稣就在那里不得行什么异能,不过按手在几个病人身上,治好他们。
\VS{6}他也诧异他们不信,就往周围乡村教训人去了。
\par }{\SH 耶稣差遣十二门徒
\par }{\R (太10·5—15;路9·1—6)
\par }{\PP \VS{7}耶稣叫了十二个门徒来,差遣他们两个两个地出去,也赐给他们权柄,制伏污鬼;
\VS{8}并且嘱咐他们:「行路的时候不要带食物和口袋,腰袋里也不要带钱,除了拐杖以外,什么都不要带;
\VS{9}只要穿鞋,也不要穿两件褂子」;
\VS{10}又对他们说:「你们无论到何处,进了人的家,就住在那里,直到离开那地方。
\VS{11}何处的人不接待你们,不听你们,你们离开那里的时候,就把脚上的尘土跺下去,对他们作见证。」
\VS{12}门徒就出去传道,叫人悔改,
\VS{13}又赶出许多的鬼,用油抹了许多病人,治好他们。
\par }{\SH 施洗约翰的死
\par }{\R (太14·1—12;路9·7—9)
\par }{\PP \VS{14}耶稣的名声传扬出来。{\PN{希律}}王听见了,就说:「施洗的{\PN{约翰}}从死里复活了,所以这些异能由他里面发出来。」
\VS{15}但别人说:「是{\PN{以利亚}}。」又有人说:「是先知,{\ADD{正}}像先知中的一位。」
\VS{16}{\PN{希律}}听见却说:「是我所斩的{\PN{约翰}},他复活了。」
\VS{17}先是{\PN{希律}}为他兄弟{\PN{腓力}}的妻子{\PN{希罗底}}的缘故,差人去拿住{\PN{约翰}},锁在监里,因为{\PN{希律}}已经娶了那妇人。
\VS{18}{\PN{约翰}}曾对{\PN{希律}}说:「你娶你兄弟的妻子是不合理的。」
\VS{19}于是{\PN{希罗底}}怀恨他,想要杀他,只是不能;
\VS{20}因为{\PN{希律}}知道{\PN{约翰}}是义人,是圣人,所以敬畏他,保护他,听他讲论,就多照着行\FTNT{}{{\FR 6:20: }有古卷:游移不定},并且乐意听他。
\VS{21}有一天,恰巧是{\PN{希律}}的生日,{\PN{希律}}摆设筵席,请了大臣和千夫长,并{\PN{加利利}}作首领的。
\VS{22}{\PN{希罗底}}的女儿进来跳舞,使{\PN{希律}}和同席的人都欢喜。王就对女子说:「你随意向我求什么,我必给你。」
\VS{23}又对她起誓说:「随你向我求什么,就是我国的一半,我也必给你。」
\VS{24}她就出去对她母亲说:「我可以求什么呢?」她母亲说:「施洗{\PN{约翰}}的头。」
\VS{25}她就急忙进去见王,求他说:「我愿王立时把施洗{\PN{约翰}}的头放在盘子里给我。」
\VS{26}王就甚忧愁;但因他所起的誓,又因同席的人,就不肯推辞,
\VS{27}随即差一个护卫兵,吩咐拿{\PN{约翰}}的头来。护卫兵就去,在监里斩了{\PN{约翰}},
\VS{28}把头放在盘子里,拿来给女子,女子就给她母亲。
\VS{29}{\PN{约翰}}的门徒听见了,就来把他的尸首领去,葬在坟墓里。
\par }{\SH 耶稣给五千人吃饱
\par }{\R (太14·13—21;路9·10—17;约6·1—14)
\par }{\PP \VS{30}使徒聚集到耶稣那里,将一切所做的事、所传的道全告诉他。
\VS{31}他就说:「你们来,同我暗暗地到旷野地方去歇一歇。」这是因为来往的人多,他们连吃饭也没有工夫。
\VS{32}他们就坐船,暗暗地往旷野地方去。
\VS{33}{\ADD{众人}}看见他们去,有许多认识他们的,就从各城步行,一同跑到那里,比他们先赶到了。
\VS{34}耶稣出来,见有许多的人,就怜悯他们,因为他们如同羊没有牧人一般,于是开口教训他们许多道理。
\VS{35}天已经晚了,门徒进前来,说:「这是野地,天已经晚了,
\VS{36}请叫众人散开,他们好往四面乡村里去,自己买什么吃。」
\VS{37}耶稣回答说:「你们给他们吃吧。」门徒说:「我们可以去买二十两银子的饼给他们吃吗?」
\VS{38}耶稣说:「你们有多少饼,可以去看看。」他们知道了,就说:「五个饼,两条鱼。」
\VS{39}耶稣吩咐他们,叫众人一帮一帮地坐在青草地上。
\VS{40}众人就一排一排地坐下,有一百一排的,有五十一排的。
\VS{41}耶稣拿着这五个饼,两条鱼,望着天祝福,擘开饼,递给门徒,摆在众人面前,也把那两条鱼分给众人。
\VS{42}他们都吃,并且吃饱了。
\VS{43}门徒就把碎饼碎鱼收拾起来,装满了十二个篮子。
\VS{44}吃饼的男人共有五千。
\par }{\SH 耶稣在海上行走
\par }{\R (太14·22—33;约6·16—21)
\par }{\PP \VS{45}耶稣随即催门徒上船,先渡到那边{\PN{伯赛大}}去,等他叫众人散开。
\VS{46}他既辞别了他们,就往山上去祷告。
\VS{47}到了晚上,船在海中,耶稣独自在岸上;
\VS{48}看见门徒因风不顺,摇橹甚苦。夜里约有四更天,就在海面上走,往他们那里去,意思要走过他们去。
\VS{49}但门徒看见他在海面上走,以为是鬼怪,就喊叫起来;
\VS{50}因为他们都看见了他,且甚惊慌。耶稣连忙对他们说:「你们放心!是我,不要怕!」
\VS{51}于是到他们那里,上了船,风就住了;他们心里十分惊奇。
\VS{52}这是因为他们不明白那分饼的事,心里还是愚顽。
\par }{\SH 治好革尼撒勒的病人
\par }{\R (太14·34—36)
\par }{\PP \VS{53}既渡过去,来到{\PN{革尼撒勒}}地方,就靠了岸,
\VS{54}一下船,众人认得是耶稣,
\VS{55}就跑遍那一带地方,听见他在何处,便将有病的人用褥子抬到那里。
\VS{56}凡耶稣所到的地方,或村中,或城里,或乡间,他们都将病人放在街市上,求耶稣只容他们摸他的衣裳 子;凡摸着的人就都好了。

\par }\Chap{7}{\SH 古人的传统
\par }{\R (太15·1—20)
\par }{\PP \VerseOne{1}有法利赛人和几个文士从{\PN{耶路撒冷}}来,到耶稣那里聚集。
\VS{2}他们曾看见他的门徒中有人用俗手,就是没有洗的手,吃饭。(
\VS{3}原来法利赛人和{\PN{犹太}}人都拘守古人的遗传,若不仔细洗手就不吃饭;
\VS{4}从市上来,若不洗浴也不吃饭;还有好些别的规矩,他们历代拘守,就是洗杯、罐、铜器{\ADD{等物}}。)
\VS{5}法利赛人和文士问他说:「你的门徒为什么不照古人的遗传,用俗手吃饭呢?」
\VS{6}耶稣说:「{\PN{以赛亚}}指着你们假冒为善之人所说的预言是不错的。如经上说:
\par }{\Q 这百姓用嘴唇尊敬我,
\par }{\Q 心却远离我。
\par }{\Q \VS{7}他们将人的吩咐当作道理教导人,
\par }{\Q 所以拜我也是枉然。
\par }{\MM \VS{8}你们是离弃 神的诫命,拘守人的遗传」;
\VS{9}又说:「你们诚然是废弃 神的诫命,要守自己的遗传。
\VS{10}{\PN{摩西}}说:『当孝敬父母』;又说:『咒骂父母的,必治死他。』
\VS{11}你们倒说:『人若对父母说:我所当奉给你的,已经作了各耳板』(各耳板就是供献的意思),
\VS{12}以后你们就不容他再奉养父母。
\VS{13}这就是你们承接遗传,废了 神的道。你们还做许多这样的事。」
\par }{\PP \VS{14}耶稣又叫众人来,对他们说:「你们都要听我的话,也要明白。
\VS{15}从外面进去的不能污秽人,惟有从里面出来的乃能污秽人。」\FTNT{}{{\FR 7:15: }有古卷加:16有耳可听的,就应当听!}
\VS{17}耶稣离开众人,进了屋子,门徒就问他这比喻的意思。
\VS{18}耶稣对他们说:「你们也是这样不明白吗?岂不晓得凡从外面进入的,不能污秽人,
\VS{19}因为不是入他的心,乃是入他的肚腹,又落到茅厕里({\ADD{这是说}},各样的食物都是洁净的)」;
\VS{20}又说:「从人里面出来的,那才能污秽人;
\VS{21}因为从里面,就是从人心里,发出恶念、苟合、
\VS{22}偷盗、凶杀、奸淫、贪婪、邪恶、诡诈、淫荡、嫉妒、谤 、骄傲、狂妄。
\VS{23}这一切的恶都是从里面出来,且能污秽人。」
\par }{\SH 一个妇人的信心
\par }{\R (太15·21—28)
\par }{\PP \VS{24}耶稣从那里起身,往{\PN{泰尔}}、{\PN{西顿}}的境内去,进了一家,不愿意人知道,却隐藏不住。
\VS{25}当下,有一个妇人,她的小女儿被污鬼附着,听见耶稣的事,就来俯伏在他脚前。
\VS{26}这妇人是{\PN{希腊}}人,属{\PN{叙利腓尼基}}族。她求耶稣赶出那鬼离开她的女儿。
\VS{27}耶稣对她说:「让儿女们先吃饱,不好拿儿女的饼丢给狗吃。」
\VS{28}妇人回答说:「主啊,不错;但是狗在桌子底下也吃孩子们的碎渣儿。」
\VS{29}耶稣对她说:「因这句话,你回去吧;鬼已经离开你的女儿了。」
\VS{30}她就回家去,见小孩子躺在床上,鬼已经出去了。
\par }{\SH 耶稣治好耳聋舌结的人
\par }{\PP \VS{31}耶稣又离了{\PN{泰尔}}的境界,经过{\PN{西顿}},就从{\PN{低加坡里}}境内来到{\PN{加利利}}
{\PN{海}}。
\VS{32}有人带着一个耳聋舌结的人来见耶稣,求他按手在他身上。
\VS{33}耶稣领他离开众人,到一边去,就用指头探他的耳朵,吐唾沫抹他的舌头,
\VS{34}望天叹息,对他说:「以法大!」就是说:「开了吧!」
\VS{35}他的耳朵就开了,舌结也解了,说话也清楚了。
\VS{36}耶稣嘱咐他们不要告诉人;但他越发嘱咐,他们越发传扬开了。
\VS{37}众人分外希奇,说:「他所做的事都好,他连聋子也叫他们听见,哑巴也叫他们说话。」

\par }\Chap{8}{\SH 耶稣给四千人吃饱
\par }{\R (太15·32—39)
\par }{\PP \VerseOne{1}那时,又有许多人聚集,并没有什么吃的。耶稣叫门徒来,说:
\VS{2}「我怜悯这众人;因为他们同我在这里已经三天,也没有吃的了。
\VS{3}我若打发他们饿着回家,就必在路上困乏,因为其中有从远处来的。」
\VS{4}门徒回答说:「在这野地,从哪里能得饼,叫这些人吃饱呢?」
\VS{5}耶稣问他们说:「你们有多少饼?」他们说:「七个。」
\VS{6}他吩咐众人坐在地上,就拿着这七个饼祝谢了,擘开,递给门徒,叫他们摆开,门徒就摆在众人面前。
\VS{7}又有几条小鱼;耶稣祝了福,就吩咐也摆在众人面前。
\VS{8}众人都吃,并且吃饱了,收拾剩下的零碎,有七筐子。
\VS{9}人数约有四千。耶稣打发他们走了,
\VS{10}随即同门徒上船,来到{\PN{大玛努他}}境内。
\par }{\SH 求主显个神迹
\par }{\R (太16·1—4)
\par }{\PP \VS{11}法利赛人出来盘问耶稣,求他从天上显个神迹给他们看,想要试探他。
\VS{12}耶稣心里深深地叹息,说:「这世代为什么求神迹呢?我实在告诉你们,没有神迹给这世代看。」
\VS{13}他就离开他们,又上{\ADD{船}}往海那边去了。
\par }{\SH 防备法利赛人和希律的酵
\par }{\R (太16·5—12)
\par }{\PP \VS{14}门徒忘了带饼;在船上除了一个饼,没有别的食物。
\VS{15}耶稣嘱咐他们说:「你们要谨慎,防备法利赛人的酵和{\PN{希律}}的酵。」
\VS{16}他们彼此议论说:「{\ADD{这是因为}}我们没有饼吧。」
\VS{17}耶稣看出来,就说:「你们为什么因为没有饼就议论呢?你们还不省悟,还不明白吗?你们的心还是愚顽吗?
\VS{18}你们有眼睛,看不见吗?有耳朵,听不见吗?也不记得吗?
\VS{19}我擘开那五个饼分给五千人,你们收拾的零碎装满了多少篮子呢?」他们说:「十二个。」
\VS{20}「{\ADD{又擘开}}那七个饼分给四千人,你们收拾的零碎装满了多少筐子呢?」他们说:「七个。」
\VS{21}耶稣说:「你们还是不明白吗?」
\par }{\SH 治好伯赛大的瞎子
\par }{\PP \VS{22}他们来到{\PN{伯赛大}},有人带一个瞎子来,求耶稣摸他。
\VS{23}耶稣拉着瞎子的手,领他到村外,就吐唾沫在他眼睛上,按手在他身上,问他说:「你看见什么了?」
\VS{24}他就抬头一看,说:「我看见人了;他们好像树木,并且行走。」
\VS{25}随后又按手在他眼睛上,他定睛一看,就复了原,样样都看得清楚了。
\VS{26}耶稣打发他回家,说:「连这村子你也不要进去。」
\par }{\SH 彼得认耶稣为基督
\par }{\R (太16·13—20;路9·18—21)
\par }{\PP \VS{27}耶稣和门徒出去,往{\PN{凯撒利亚·腓立比}}村庄去;在路上问门徒说:「人说我是谁?」
\VS{28}他们说:「{\ADD{有人说}}是施洗的{\PN{约翰}};有人说是{\PN{以利亚}};又有人说是先知里的一位。」
\VS{29}又问他们说:「你们说我是谁?」{\PN{彼得}}回答说:「你是基督。」
\VS{30}耶稣就禁戒他们,不要告诉人。
\par }{\SH 耶稣预言受难和复活
\par }{\R (太16·21—28;路9·22—27)
\par }{\PP \VS{31}从此,他教训他们说:「人子必须受许多的苦,被长老、祭司长,和文士弃绝,并且被杀,过三天复活。」
\VS{32}耶稣明明地说这话,{\PN{彼得}}就拉着他,劝他。
\VS{33}耶稣转过来,看着门徒,就责备{\PN{彼得}}说:「撒但,退我后边去吧!因为你不体贴 神的意思,只体贴人的意思。」
\VS{34}于是叫众人和门徒来,对他们说:「若有人要跟从我,就当舍己,背起他的十字架来跟从我。
\VS{35}因为,凡要救自己生命\FTNT{}{{\FR 8:35: }或译:灵魂;下同}的,必丧掉生命;凡为我和福音丧掉生命的,必救了生命。
\VS{36}人就是赚得全世界,赔上自己的生命,有什么益处呢?
\VS{37}人还能拿什么换生命呢?
\VS{38}凡在这淫乱罪恶的世代,把我和我的道当作可耻的,人子在他父的荣耀里,同圣天使降临的时候,也要把那人当作可耻的。」

\par }\Chap{9}{\PP \VerseOne{1}耶稣又对他们说:「我实在告诉你们,站在这里的,有人在没尝死味以前,必要看见 神的国大有能力临到。」
\par }{\SH 耶稣改变形象
\par }{\R (太17·1—13;路9·28—36)
\par }{\PP \VS{2}过了六天,耶稣带着{\PN{彼得}}、{\PN{雅各}}、{\PN{约翰}}暗暗地上了高山,就在他们面前变了形象,
\VS{3}衣服放光,极其洁白,地上漂布的,没有一个能漂得那样白。
\VS{4}{\ADD{忽然}},有{\PN{以利亚}}同{\PN{摩西}}向他们显现,并且和耶稣说话。
\VS{5}{\PN{彼得}}对耶稣说:「拉比\FTNT{}{{\FR 9:5: }就是夫子},我们在这里真好!可以搭三座棚,一座为你,一座为{\PN{摩西}},一座为{\PN{以利亚}}。」
\VS{6}{\PN{彼得}}不知道说什么才好,因为他们甚是惧怕。
\VS{7}有一朵云彩来遮盖他们;也有声音从云彩里出来,说:「这是我的爱子,你们要听他。」
\VS{8}门徒忽然周围一看,不再见一人,只见耶稣同他们在那里。
\VS{9}下山的时候,耶稣嘱咐他们说:「人子还没有从死里复活,你们不要将所看见的告诉人。」
\VS{10}门徒将这话存记在心,彼此议论「从死里复活」是什么意思。
\VS{11}他们就问耶稣说:「文士为什么说{\PN{以利亚}}必须先来?」
\VS{12}耶稣说:「{\PN{以利亚}}固然先来复兴万事;{\ADD{经上}}不是指着人子说,他要受许多的苦被人轻慢呢?
\VS{13}我告诉你们,{\PN{以利亚}}已经来了,他们也任意待他,正如{\ADD{经上}}所指着他的话。」
\par }{\SH 治好被污鬼附身的孩子
\par }{\R (太17·14—21;路9·37—43)
\par }{\PP \VS{14}耶稣到了门徒那里,看见有许多人围着他们,又有文士和他们辩论。
\VS{15}众人一见耶稣,都甚希奇,就跑上去问他的安。
\VS{16}耶稣问他们说:「你们和他们辩论的是什么?」
\VS{17}众人中间有一个人回答说:「夫子,我带了我的儿子到你这里来,他被哑巴鬼附着。
\VS{18}无论在哪里,鬼捉弄他,把他摔倒,他就口中流沫,咬牙切齿,身体枯干。我请过你的门徒把鬼赶出去,他们却是不能。」
\VS{19}耶稣说:「嗳!不信的世代啊,我在你们这里要到几时呢?我忍耐你们要到几时呢?把他带到我这里来吧。」
\VS{20}他们就带了他来。他一见耶稣,鬼便叫他重重地抽风,倒在地上,翻来覆去,口中流沫。
\VS{21}耶稣问他父亲说:「他得这病有多少日子呢?」回答说:「从小的时候。
\VS{22}鬼屡次把他扔在火里、水里,要灭他。你若能做什么,求你怜悯我们,帮助我们。」
\VS{23}耶稣对他说:「你若能{\ADD{信}},在信的人,凡事都能。」
\VS{24}孩子的父亲立时喊着说\FTNT{}{{\FR 9:24: }有古卷:立时流泪地喊着说}:「我信!但我信不足,求主帮助。」
\VS{25}耶稣看见众人都跑上来,就斥责那污鬼,说:「你这聋哑的鬼,我吩咐你从他里头出来,再不要进去!」
\VS{26}那鬼喊叫,使孩子大大地抽了一阵风,就出来了。{\ADD{孩子}}好像死了一般,以致众人多半说:「他是死了。」
\VS{27}但耶稣拉着他的手,扶他起来,他就站起来了。
\VS{28}耶稣进了屋子,门徒就暗暗地问他说:「我们为什么不能赶出他去呢?」
\VS{29}耶稣说:「非用祷告\FTNT{}{{\FR 9:29: }有古卷加:禁食二字},这一类的{\ADD{鬼}}总不能出来\FTNT{}{{\FR 9:29: }或译:不能赶他出来}。」
\par }{\SH 耶稣第二次预言受难和复活
\par }{\R (太17·22—23;路9·43—45)
\par }{\PP \VS{30}他们离开那地方,经过{\PN{加利利}};耶稣不愿意人知道。
\VS{31}于是教训门徒,说:「人子将要被交在人手里,他们要杀害他;被杀以后,过三天他要复活。」
\VS{32}门徒却不明白这话,又不敢问他。
\par }{\SH 谁最伟大
\par }{\R (太18·1—5;路9·46—48)
\par }{\PP \VS{33}他们来到{\PN{迦百农}}。耶稣在屋里问门徒说:「你们在路上议论的是什么?」
\VS{34}门徒不作声,因为他们在路上彼此争论谁为大。
\VS{35}耶稣坐下,叫十二个门徒来,说:「若有人愿意作首先的,他必作众人末后的,作众人的用人。」
\VS{36}于是领过一个小孩子来,叫他站在门徒中间,又抱起他来,对他们说:
\VS{37}「凡为我名接待一个像这小孩子的,就是接待我;凡接待我的,不是接待我,乃是接待那差我来的。」
\par }{\SH 不敌挡我们就是帮助我们
\par }{\R (路9·49—50)
\par }{\PP \VS{38}{\PN{约翰}}对耶稣说:「夫子,我们看见一个人奉你的名赶鬼,我们就禁止他,因为他不跟从我们。」
\VS{39}耶稣说:「不要禁止他;因为没有人奉我名行异能,反倒轻易毁谤我。
\VS{40}不敌挡我们的,就是帮助我们的。
\VS{41}凡因你们是属基督,给你们一杯水喝的,我实在告诉你们,他不能不得赏赐。」
\par }{\SH 罪的诱惑
\par }{\R (太18·6—9;路17·1—2)
\par }{\PP \VS{42}「凡使这信我的一个小子跌倒的,倒不如把大磨石拴在这人的颈项上,扔在海里。
\VS{43}倘若你一只手叫你跌倒,就把它砍下来;
\VS{44}你缺了肢体进入{\ADD{永}}生,强如有两{\ADD{只}}手落到地狱,入那不灭的火里去。
\VS{45}倘若你一只脚叫你跌倒,就把它砍下来;
\VS{46}你瘸腿进入{\ADD{永}}生,强如有两只脚被丢在地狱里。
\VS{47}倘若你一只眼叫你跌倒,就去掉它;你只有一只眼进入 神的国,强如有两只眼被丢在地狱里。
\VS{48}在那里,虫是不死的,火是不灭的。
\VS{49}因为必用火当盐腌各人。\FTNT{}{{\FR 9:49: }有古卷加:凡祭物必用盐腌。}
\VS{50}盐本是好的,若失了味,可用什么叫它再咸呢?你们里头应当有盐,彼此和睦。」

\par }\Chap{10}{\SH 休妻的问题
\par }{\R (太19·1—12;路16·18)
\par }{\PP \VerseOne{1}耶稣从那里起身,来到{\PN{犹太}}的境界并{\PN{约旦河}}外。众人又聚集到他那里,他又照常教训他们。
\VS{2}有法利赛人来问他说:「人休妻可以不可以?」意思要试探他。
\VS{3}耶稣回答说:「{\PN{摩西}}吩咐你们的是什么?」
\VS{4}他们说:「{\PN{摩西}}许人写了休书便可以休妻。」
\VS{5}耶稣说:「{\PN{摩西}}因为你们的心硬,所以写这条例给你们;
\VS{6}但从起初创造的时候, 神造人是造男造女。
\VS{7}因此,人要离开父母,与妻子连合,二人成为一体。
\VS{8}既然如此,夫妻不再是两个人,乃是一体的了。
\VS{9}所以, 神配合的,人不可分开。」
\VS{10}到了屋里,门徒就问他这事。
\VS{11}耶稣对他们说:「凡休妻另娶的,就是犯奸淫,辜负他的妻子;
\VS{12}妻子若离弃丈夫另嫁,也是犯奸淫了。」
\par }{\SH 耶稣为小孩祝福
\par }{\R (太19·13—15;路18·15—17)
\par }{\PP \VS{13}有人带着小孩子来见耶稣,要耶稣摸他们,门徒便责备那些人。
\VS{14}耶稣看见就恼怒,对门徒说:「让小孩子到我这里来,不要禁止他们;因为在 神国的,正是这样的人。
\VS{15}我实在告诉你们,凡要承受 神国的,若不像小孩子,断不能进去。」
\VS{16}于是抱着小孩子,给他们按手,为他们祝福。
\par }{\SH 财主寻求永生之道
\par }{\R (太19·16—30;路18·18—30)
\par }{\PP \VS{17}耶稣出来行路的时候,有一个人跑来,跪在他面前,问他说:「良善的夫子,我当做什么事才可以承受永生?」
\VS{18}耶稣对他说:「你为什么称我是良善的?除了 神一位之外,再没有良善的。
\VS{19}诫命你是晓得的:不可杀人;不可奸淫;不可偷盗;不可作假见证;不可亏负人;当孝敬父母。」
\VS{20}他对耶稣说:「夫子,这一切我从小都遵守了。」
\VS{21}耶稣看着他,就爱他,对他说:「你还缺少一件:去变卖你所有的,分给穷人,就必有财宝在天上;你还要来跟从我。」
\VS{22}他听见这话,脸上就变了色,忧忧愁愁地走了,因为他的产业很多。
\par }{\PP \VS{23}耶稣周围一看,对门徒说:「有钱财的人进 神的国是何等地难哪!」
\VS{24}门徒希奇他的话。耶稣又对他们说:「小子,倚靠钱财的人进 神的国是何等地难哪!
\VS{25}骆驼穿过针的眼,比财主进 神的国还容易呢。」
\VS{26}门徒就分外希奇,对他说:「这样谁能得救呢?」
\VS{27}耶稣看着他们,说:「在人是不能,在 神却不然,因为 神凡事都能。」
\VS{28}{\PN{彼得}}就对他说:「看哪,我们已经撇下所有的跟从你了。」
\VS{29}耶稣说:「我实在告诉你们,人为我和福音撇下房屋,或是弟兄、姊妹、父母、儿女、田地,
\VS{30}没有不在今世得百倍的,就是房屋、弟兄、姊妹、母亲、儿女、田地,并且要受逼迫,在来世必得永生。
\VS{31}然而,有许多在前的,将要在后,在后的,将要在前。」
\par }{\SH 耶稣第三次预言受难和复活
\par }{\R (太20·17—19;路18·31—34)
\par }{\PP \VS{32}他们行路上{\PN{耶路撒冷}}去。耶稣在前头走,门徒就希奇,跟从的人也害怕。耶稣又叫过十二个门徒来,把自己将要遭遇的事告诉他们说:
\VS{33}「看哪,我们上{\PN{耶路撒冷}}去,人子将要被交给祭司长和文士,他们要定他死罪,交给外邦人。
\VS{34}他们要戏弄他,吐唾沫在他脸上,鞭打他,杀害他。过了三天,他要复活。」
\par }{\SH 雅各和约翰的要求
\par }{\R (太20·20—28)
\par }{\PP \VS{35}{\PN{西庇太}}的儿子{\PN{雅各}}、{\PN{约翰}}进前来,对耶稣说:「夫子,我们无论求你什么,愿你给我们做。」
\VS{36}耶稣说:「要我给你们做什么?」
\VS{37}他们说:「赐我们在你的荣耀里,一个坐在你右边,一个坐在你左边。」
\VS{38}耶稣说:「你们不知道所求的是什么。我所喝的杯,你们能喝吗?我所受的洗,你们能受吗?」
\VS{39}他们说:「我们能。」耶稣说:「我所喝的杯,你们也要喝;我所受的洗,你们也要受;
\VS{40}只是坐在我的左右,不是我可以赐的,乃是为谁预备的,就赐给谁。」
\VS{41}那十个门徒听见,就恼怒{\PN{雅各}}、{\PN{约翰}}。
\VS{42}耶稣叫他们来,对他们说:「你们知道,外邦人有尊为君王的,治理他们,有大臣操权管束他们。
\VS{43}只是在你们中间,不是这样。你们中间,谁愿为大,就必作你们的用人;
\VS{44}在你们中间,谁愿为首,就必作众人的仆人。
\VS{45}因为人子来,并不是要受人的服事,乃是要服事人,并且要舍命作多人的赎价。」
\par }{\SH 治好瞎子巴底买
\par }{\R (太20·29—34;路18·35—43)
\par }{\PP \VS{46}到了{\PN{耶利哥}};耶稣同门徒并许多人出{\PN{耶利哥}}的时候,有一个讨饭的瞎子,是{\PN{底买}}的儿子{\PN{巴底买}},坐在路旁。
\VS{47}他听见是{\PN{拿撒勒}}的耶稣,就喊着说:「{\PN{大卫}}的子孙耶稣啊!可怜我吧!」
\VS{48}有许多人责备他,不许他作声。他却越发大声喊着说:「{\PN{大卫}}的子孙哪,可怜我吧!」
\VS{49}耶稣就站住,说:「叫过他来。」他们就叫那瞎子,对他说:「放心,起来!他叫你啦。」
\VS{50}瞎子就丢下衣服,跳起来,走到耶稣那里。
\VS{51}耶稣说:「要我为你做什么?」瞎子说:「拉波尼\FTNT{}{{\FR 10:51: }就是夫子},我要能看见。」
\VS{52}耶稣说:「你去吧!你的信救了你了。」瞎子立刻看见了,就在路上跟随耶稣。

\par }\Chap{11}{\SH 光荣地进耶路撒冷
\par }{\R (太21·1—11;路19·28—40;约12·12—19)
\par }{\PP \VerseOne{1}耶稣和门徒将近{\PN{耶路撒冷}},到了{\PN{伯法其}}和{\PN{伯大尼}},在{\PN{橄榄山}}那里;耶稣就打发两个门徒,
\VS{2}对他们说:「你们往对面村子里去,一进去的时候,必看见一匹驴驹拴在那里,是从来没有人骑过的,可以解开,牵来。
\VS{3}若有人对你们说:『为什么做这事?』你们就说:『主要用它。』那人必立时让你们牵来。」
\VS{4}他们去了,便看见一匹驴驹拴在门外街道上,就把它解开。
\VS{5}在那里站着的人,有几个说:「你们解驴驹做什么?」
\VS{6}门徒照着耶稣所说的回答,那些人就任凭他们牵去了。
\VS{7}他们把驴驹牵到耶稣那里,把自己的衣服搭在上面,耶稣就骑上。
\VS{8}有许多人把衣服铺在路上,也有人把田间的树枝砍下来,铺在路上。
\VS{9}前行后随的人都喊着说:
\par }{\Q 和散那\FTNT{}{{\FR 11:9: }和散那:原有求救的意思,在此乃是称颂的话}!
\par }{\Q 奉主名来的是应当称颂的!
\par }{\Q \VS{10}那将要来的我祖{\PN{大卫}}之国是应当称颂的!
\par }{\Q 高高在上和散那!
\par }{\PP \VS{11}耶稣进了{\PN{耶路撒冷}},入了{\ADD{圣}}殿,周围看了各样物件。天色已晚,就和十二个门徒出城,往{\PN{伯大尼}}去了。
\par }{\SH 咒诅无花果树
\par }{\R (太21·18—19)
\par }{\PP \VS{12}第二天,他们从{\PN{伯大尼}}出来,耶稣饿了。
\VS{13}远远地看见一棵无花果树,树上有叶子,就往那里去,或者在树上可以找着什么。到了树下,竟找不着什么,不过有叶子,因为不是收无花果的时候。
\VS{14}耶稣就对树说:「从今以后,永没有人吃你的果子。」他的门徒也听见了。
\par }{\SH 洁净圣殿
\par }{\R (太21·12—17;路19·45—48;约2·13—22)
\par }{\PP \VS{15}他们来到{\PN{耶路撒冷}}。耶稣进入{\ADD{圣}}殿,赶出殿里做买卖的人,推倒兑换银钱之人的桌子和卖鸽子之人的凳子;
\VS{16}也不许人拿着器具从殿里经过;
\VS{17}便教训他们说:「{\ADD{经上}}不是记着说:
\par }{\Q 我的殿必称为万国祷告的殿吗?
\par }{\Q 你们倒使它成为贼窝了。」
\par }{\PP \VS{18}祭司长和文士听见这话,就想法子要除灭耶稣,却又怕他,因为众人都希奇他的教训。
\VS{19}每天晚上,耶稣出城去。
\par }{\SH 从无花果树得教训
\par }{\R (太21·20—22)
\par }{\PP \VS{20}早晨,他们从那里经过,看见无花果树连根都枯干了。
\VS{21}{\PN{彼得}}想起耶稣的话来,就对他说:「拉比,请看!你所咒诅的无花果树已经枯干了。」
\VS{22}耶稣回答说:「你们当信服 神。
\VS{23}我实在告诉你们,无论何人对这座山说:『你挪开此地,投在海里!』他若心里不疑惑,只信他所说的必成,就必给他成了。
\VS{24}所以我告诉你们,凡你们祷告祈求的,无论是什么,只要信是得着的,就必得着。
\VS{25}你们站着祷告的时候,若想起有人得罪你们,就当饶恕他,好叫你们在天上的父也饶恕你们的过犯。
\VS{26}你们若不饶恕人,你们在天上的父也不饶恕你们的过犯。\FTNT{}{{\FR 11:26: }有古卷无此节}」
\par }{\SH 质问耶稣的权柄
\par }{\R (太21·23—27;路20·1—8)
\par }{\PP \VS{27}他们又来到{\PN{耶路撒冷}}。耶稣在殿里行走的时候,祭司长和文士并长老进前来,
\VS{28}问他说:「你仗着什么权柄做这些事?给你这权柄的是谁呢?」
\VS{29}耶稣对他们说:「我要问你们一句话,你们回答我,我就告诉你们我仗着什么权柄做这些事。
\VS{30}{\PN{约翰}}的洗礼是从天上来的?是从人间来的呢?你们可以回答我。」
\VS{31}他们彼此商议说:「我们若说『从天上来』,他必说:『这样,你们为什么不信他呢?』
\VS{32}若说『从人间来』,却又怕百姓,因为众人真以{\PN{约翰}}为先知。」
\VS{33}于是回答耶稣说:「我们不知道。」耶稣说:「我也不告诉你们我仗着什么权柄做这些事。」

\par }\Chap{12}{\SH 凶恶园户的比喻
\par }{\R (太21·33—46;路20·9—19)
\par }{\PP \VerseOne{1}耶稣就用比喻对他们说:「有人栽了一个葡萄园,周围圈上篱笆,挖了一个压酒池,盖了一座楼,租给园户,就往外国去了。
\VS{2}到了时候,打发一个仆人到园户那里,要从园户收葡萄园的果子。
\VS{3}园户拿住他,打了他,叫他空手回去。
\VS{4}再打发一个仆人到他们那里。他们打伤他的头,并且凌辱他。
\VS{5}又打发一个仆人去,他们就杀了他。后又打发好些仆人去,有被他们打的,有被他们杀的。
\VS{6}园主还有一位是他的爱子,末后又打发他去,{\ADD{意思}}说:『他们必尊敬我的儿子。』
\VS{7}不料,那些园户彼此说:『这是承受产业的。来吧,我们杀他,产业就归我们了!』
\VS{8}于是拿住他,杀了他,把他丢在园外。
\VS{9}这样,葡萄园的主人要怎么办呢?他要来除灭那些园户,将葡萄园转给别人。
\VS{10}{\ADD{经上写着说}}:
\par }{\Q 匠人所弃的石头
\par }{\Q 已作了房角的头块石头。
\par }{\Q \VS{11}这是主所做的,
\par }{\Q 在我们眼中看为希奇。
\par }{\MM 这经你们没有念过吗?」
\VS{12}他们看出这比喻是指着他们说的,就想要捉拿他,只是惧怕百姓,于是离开他走了。
\par }{\SH 纳税给凯撒的问题
\par }{\R (太22·15—22;路20·19—26)
\par }{\PP \VS{13}后来,他们打发几个法利赛人和几个{\PN{希律}}党的人到耶稣那里,要就着他的话陷害他。
\VS{14}他们来了,就对他说:「夫子,我们知道你是诚实的,什么人你都不徇情面;因为你不看人的外貌,乃是诚诚实实传 神的道。纳税给凯撒可以不可以?
\VS{15}我们该纳不该纳?」耶稣知道他们的假意,就对他们说:「你们为什么试探我?拿一个银钱来给我看!」
\VS{16}他们就拿了来。耶稣说:「这像和这号是谁的?」他们说:「是凯撒的。」
\VS{17}耶稣说:「凯撒的物当归给凯撒, 神的物当归给 神。」他们就很希奇他。
\par }{\SH 复活的问题
\par }{\R (太22·23—33;路20·27—40)
\par }{\PP \VS{18}撒都该人常说没有复活的事。他们来问耶稣说:
\VS{19}「夫子,{\PN{摩西}}为我们写着说:『人若死了,撇下妻子,没有孩子,他兄弟当娶他的妻,为哥哥生子立后。』
\VS{20}有弟兄七人,第一个娶了妻,死了,没有留下孩子。
\VS{21}第二个娶了她,也死了,没有留下孩子。第三个也是这样。
\VS{22}那七个人都没有留下孩子;末了,那妇人也死了。
\VS{23}当复活的时候,她是哪一个的妻子呢?因为他们七个人都娶过她。」
\VS{24}耶稣说:「你们所以错了,岂不是因为不明白圣经,不晓得 神的大能吗?
\VS{25}人从死里复活,也不娶也不嫁,乃像天上的使者一样。
\VS{26}论到死人复活,你们没有念过{\PN{摩西}}的书荆棘篇上{\ADD{所载的}}吗? 神对{\PN{摩西}}说:『我是{\PN{亚伯拉罕}}的 神,{\PN{以撒}}的 神,{\PN{雅各}}的 神。』
\VS{27}神不是死人的 神,乃是活人的 神。你们是大错了。」
\par }{\SH 最大的诫命
\par }{\R (太22·34—40;路10·25—28)
\par }{\PP \VS{28}有一个文士来,听见他们辩论,晓得耶稣回答的好,就问他说:「诫命中哪是第一{\ADD{要紧的}}呢?」
\VS{29}耶稣回答说:「第一{\ADD{要紧的}}就是说:『{\PN{以色列}}啊,你要听,主—我们 神是独一的主。
\VS{30}你要尽心、尽性、尽意、尽力爱主—你的 神。』
\VS{31}其次就是说:『要爱人如己。』再没有比这两条诫命更大的了。」
\VS{32}那文士对耶稣说:「夫子说, 神是一位,实在不错;除了他以外,再没有别的 神;
\VS{33}并且尽心、尽智、尽力爱他,又爱人如己,就比一切燔祭和各样祭祀好的多。」
\VS{34}耶稣见他回答的有智慧,就对他说:「你离 神的国不远了。」从此以后,没有人敢再问他什么。
\par }{\SH 大卫子孙的问题
\par }{\R (太22·41—46;路20·41—44)
\par }{\PP \VS{35}耶稣在殿里教训人,就问他们说:「文士怎么说基督是{\PN{大卫}}的子孙呢?
\VS{36}{\PN{大卫}}被圣灵感动,说:
\par }{\Q 主对我主说:
\par }{\Q 你坐在我的右边,
\par }{\Q 等我使你仇敌作你的脚凳。
\par }{\MM \VS{37}{\PN{大卫}}既自己称他为主,他怎么又是{\PN{大卫}}的子孙呢?」众人都喜欢听他。
\par }{\SH 谴责文士
\par }{\R (太23·1—36;路20·45—47)
\par }{\PP \VS{38}耶稣在教训之间,说:「你们要防备文士;他们好穿长衣游行,喜爱人在街市上问他们的安,
\VS{39}又喜爱会堂里的高位,筵席上的首座。
\VS{40}他们侵吞寡妇的家产,假意作很长的祷告。这些人要受更重的刑罚!」
\par }{\SH 寡妇的奉献
\par }{\R (路21·1—4)
\par }{\PP \VS{41}耶稣对银库坐着,看众人怎样投钱入库。有好些财主往里投了若干的钱。
\VS{42}有一个穷寡妇来,往里投了两个小钱,就是一个大钱。
\VS{43}耶稣叫门徒来,说:「我实在告诉你们,这穷寡妇投入库里的,比众人所投的更多。
\VS{44}因为,他们都是自己有余,拿出来投在里头;但这寡妇是自己不足,把她一切养生的都投上了。」

\par }\Chap{13}{\SH 预言圣殿被毁
\par }{\R (太24·1—2;路21·5—6)
\par }{\PP \VerseOne{1}耶稣从殿里出来的时候,有一个门徒对他说:「夫子,请看,这是何等的石头!何等的殿宇!」
\VS{2}耶稣对他说:「你看见这大殿宇吗?将来在这里没有一块石头留在石头上,不被拆毁了。」
\par }{\SH 灾难的起头
\par }{\R (太24·3—14;路21·7—19)
\par }{\PP \VS{3}耶稣在{\PN{橄榄山}}上对{\ADD{圣}}殿而坐。{\PN{彼得}}、{\PN{雅各}}、{\PN{约翰}},和{\PN{安得烈}}暗暗地问他说:
\VS{4}「请告诉我们,什么时候有这些事呢?这一切事将成的时候有什么预兆呢?」
\VS{5}耶稣说:「你们要谨慎,免得有人迷惑你们。
\VS{6}将来有好些人冒我的名来,说:『我是{\ADD{基督}}』,并且要迷惑许多人。
\VS{7}你们听见打仗和打仗的风声,不要惊慌。{\ADD{这些事}}是必须有的,只是末期还没有到。
\VS{8}民要攻打民,国要攻打国;多处必有地震、饥荒。这都是灾难\FTNT{}{{\FR 13:8: }灾难:原文是生产之难}的起头。
\VS{9}但你们要谨慎;因为人要把你们交给公会,并且你们在会堂里要受鞭打,又为我的缘故站在诸侯与君王面前,对他们作见证。
\VS{10}然而,福音必须先传给万民。
\VS{11}人把你们拉去交官的时候,不要预先思虑说什么;到那时候,赐给你们什么话,你们就说什么;因为说话的不是你们,乃是圣灵。
\VS{12}弟兄要把弟兄,父亲要把儿子,送到死地;儿女要起来与父母为敌,害死他们;
\VS{13}并且你们要为我的名被众人恨恶。惟有忍耐到底的,必然得救。」
\par }{\SH 大灾难
\par }{\R (太24·15—28;路21·20—24)
\par }{\PP \VS{14}「你们看见那行毁坏可憎的,站在不当站的地方(读{\ADD{这经}}的人须要会意)。那时,在{\PN{犹太}}的,应当逃到山上;
\VS{15}在房上的,不要下来,也不要进去拿家里的东西;
\VS{16}在田里的,也不要回去取衣裳。
\VS{17}当那些日子,怀孕的和奶孩子的有祸了!
\VS{18}你们应当祈求,叫这些事不在冬天临到。
\VS{19}因为在那些日子必有灾难,自从 神创造万物直到如今,并没有这样的灾难,后来也必没有。
\VS{20}若不是主减少那日子,凡有血气的,总没有一个得救的;只是为主的选民,他将那日子减少了。
\VS{21}那时若有人对你们说:『看哪,基督在这里』,或说:『基督在那里』,你们不要信!
\VS{22}因为假基督、假先知将要起来,显神迹奇事,倘若能行,就把选民迷惑了。
\VS{23}你们要谨慎。看哪,凡事我都预先告诉你们了。」
\par }{\SH 人子的降临
\par }{\R (太24·29—31;路21·25—28)
\par }{\Q \VS{24}「在那些日子,那灾难以后,
\par }{\Q 日头要变黑了,
\par }{\Q 月亮也不放光,
\par }{\Q \VS{25}众星要从天上坠落,
\par }{\Q 天势都要震动。
\par }{\MM \VS{26}那时,他们\FTNT{}{{\FR 13:26: }马太二十四章三十节是地上的万族}要看见人子有大能力、大荣耀,驾云降临。
\VS{27}他要差遣天使,把他的选民,从四方\FTNT{}{{\FR 13:27: }方:原文是风},从地极直到天边,都招聚了来。」
\par }{\SH 从无花果树学个比方
\par }{\R (太24·32—35;路21·29—33)
\par }{\PP \VS{28}「你们可以从无花果树学个比方:当树枝发嫩长叶的时候,你们就知道夏天近了。
\VS{29}这样,你们几时看见这些事成就,也该知道人子\FTNT{}{{\FR 13:29: }人子:或译 神的国}近了,正在门口了。
\VS{30}我实在告诉你们,这世代还没有过去,这些事都要成就。
\VS{31}天地要废去,我的话却不能废去。」
\par }{\SH 那日那时没有人知道
\par }{\R (太24·36—44)
\par }{\PP \VS{32}「但那日子,那时辰,没有人知道,连天上的使者也不知道,子也不知道,惟有父知道。
\VS{33}你们要谨慎,警醒祈祷,因为你们不晓得那日期几时来到。
\VS{34}这事正如一个人离开本家,寄居外邦,把权柄交给仆人,分派各人当做的工,又吩咐看门的警醒。
\VS{35}所以,你们要警醒;因为你们不知道家主什么时候来,或晚上,或半夜,或鸡叫,或早晨;
\VS{36}恐怕他忽然来到,看见你们睡着了。
\VS{37}我对你们所说的话,也是对众人说:要警醒!」

\par }\Chap{14}{\SH 杀害耶稣的阴谋
\par }{\R (太26·1—5;路22·1—2;约11·45—53)
\par }{\PP \VerseOne{1}过两天是逾越{\ADD{节}},又是除酵节,祭司长和文士想法子怎么用诡计捉拿耶稣,杀他。
\VS{2}只是说:「当节的日子不可,恐怕百姓生乱。」
\par }{\SH 在伯大尼受膏
\par }{\R (太26·6—13;约12·1—8)
\par }{\PP \VS{3}耶稣在{\PN{伯大尼}}长大麻风的{\PN{西门}}家里坐席的时候,有一个女人拿着一玉瓶至贵的真哪哒香膏来,打破玉瓶,把膏浇在耶稣的头上。
\VS{4}有几个人心中很不喜悦,说:「何用这样枉费香膏呢?
\VS{5}这香膏可以卖三十多两银子周济穷人。」他们就向那女人生气。
\VS{6}耶稣说:「由她吧!为什么难为她呢?她在我身上做的是一件美事。
\VS{7}因为常有穷人和你们同在,要向他们行善随时都可以;只是你们不常有我。
\VS{8}她所做的,是尽她所能的;她是为我安葬的事把香膏预先浇在我身上。
\VS{9}我实在告诉你们,普天之下,无论在什么地方传这福音,也要述说这女人所做的,以为记念。」
\par }{\SH 犹大出卖耶稣
\par }{\R (太26·14—16;路22·3—6)
\par }{\PP \VS{10}十二门徒之中,有一个{\PN{加略}}人{\PN{犹大}}去见祭司长,要把耶稣交给他们。
\VS{11}他们听见就欢喜,又应许给他银子;他就寻思如何得便把耶稣交给他们。
\par }{\SH 和门徒同度逾越节
\par }{\R (太26·17—25;路22·7—13,21—23;约13·21—30)
\par }{\PP \VS{12}除酵节的第一天,就是宰逾越{\ADD{羊羔}}的那一天,门徒对耶稣说:「你吃逾越{\ADD{节的筵席}}要我们往哪里去预备呢?」
\VS{13}耶稣就打发两个门徒,对他们说:「你们进城去,必有人拿着一瓶水迎面而来,你们就跟着他。
\VS{14}他进哪家去,你们就对那家的主人说:『夫子说:客房在哪里?我与门徒好在那里吃逾越{\ADD{节的筵席}}。』
\VS{15}他必指给你们摆设整齐的一间大楼,你们就在那里为我们预备。」
\VS{16}门徒出去,进了城,所遇见的正如耶稣所说的。他们就预备了逾越{\ADD{节的筵席}}。
\VS{17}到了晚上,耶稣和十二个门徒都来了。
\VS{18}他们坐席正吃的时候,耶稣说:「我实在告诉你们,你们中间有一个与我同吃的人要卖我了。」
\VS{19}他们就忧愁起来,一个一个地问他说:「是我吗?」
\VS{20}耶稣对他们说:「是十二个门徒中同我蘸手在盘子里的那个人。
\VS{21}人子必要去世,正如{\ADD{经上}}指着他所写的;但卖人子的人有祸了!那人不生在世上倒好。」
\par }{\SH 设立主的晚餐
\par }{\R (太26·26—30;路22·14—20;林前11·23—25)
\par }{\PP \VS{22}他们吃的时候,耶稣拿起饼来,祝了福,就擘开,递给他们,说:「你们拿着吃,这是我的身体」;
\VS{23}又拿起杯来,祝谢了,递给他们;他们都喝了。
\VS{24}耶稣说:「这是我立约的血,为多人流出来的。
\VS{25}我实在告诉你们,我不再喝这葡萄汁,直到我在 神的国里喝新的那日子。」
\VS{26}他们唱了诗,就出来,往{\PN{橄榄山}}去。
\par }{\SH 预言彼得不认主
\par }{\R (太26·31—35;路22·31—34;约13·36—38)
\par }{\PP \VS{27}耶稣对他们说:「你们都要跌倒了,因为{\ADD{经上}}记着说:
\par }{\Q 我要击打牧人,
\par }{\Q 羊就分散了。
\par }{\MM \VS{28}但我复活以后,要在你们以先往{\PN{加利利}}去。」
\VS{29}{\PN{彼得}}说:「众人虽然跌倒,我总不能。」
\VS{30}耶稣对他说:「我实在告诉你,就在今天夜里,鸡叫两遍以先,你要三次不认我。」
\VS{31}{\PN{彼得}}却极力地说:「我就是必须和你同死,也总不能不认你。」众门徒都是这样说。
\par }{\SH 在客西马尼祷告
\par }{\R (太26·36—46;路22·39—46)
\par }{\PP \VS{32}他们来到一个地方,名叫{\PN{客西马尼}}。耶稣对门徒说:「你们坐在这里,等我祷告。」
\VS{33}于是带着{\PN{彼得}}、{\PN{雅各}}、{\PN{约翰}}同去,就惊恐起来,极其难过,
\VS{34}对他们说:「我心里甚是忧伤,几乎要死;你们在这里等候,警醒。」
\VS{35}他就稍往前走,俯伏在地,祷告说:「倘若可行,便叫那时候过去。」
\VS{36}他说:「阿爸!父啊!在你凡事都能;求你将这杯撤去。然而,不要从我的意思,只要从你的意思。」
\VS{37}耶稣回来,见他们睡着了,就对{\PN{彼得}}说:「{\PN{西门}},你睡觉吗?不能警醒片时吗?
\VS{38}总要警醒祷告,免得入了迷惑。你们心灵固然愿意,肉体却软弱了。」
\VS{39}耶稣又去祷告,说的话还是与先前一样,
\VS{40}又来见他们睡着了,因为他们的眼睛甚是困倦;他们也不知道怎么回答。
\VS{41}第三次来,对他们说:「现在你们仍然睡觉安歇吧\FTNT{}{{\FR 14:41: }吧:或译吗?}!够了,时候到了。看哪,人子被卖在罪人手里了。
\VS{42}起来!我们走吧。看哪,那卖我的人近了!」
\par }{\SH 耶稣被捕
\par }{\R (太26·47—56;路22·47—53;约18·3—12)
\par }{\PP \VS{43}说话之间,忽然那十二个门徒里的{\PN{犹大}}来了,并有许多人带着刀棒,从祭司长和文士并长老那里与他同来。
\VS{44}卖耶稣的人曾给他们一个暗号,说:「我与谁亲嘴,谁就是他。你们把他拿住,牢牢靠靠地带去。」
\VS{45}{\PN{犹大}}来了,随即到耶稣跟前,说:「拉比」,便与他亲嘴。
\VS{46}他们就下手拿住他。
\VS{47}旁边站着的人,有一个拔出刀来,将大祭司的仆人砍了一刀,削掉了他一个耳朵。
\VS{48}耶稣对他们说:「你们带着刀棒出来拿我,如同拿强盗吗?
\VS{49}我天天教训人,同你们在殿里,你们并没有拿我。但{\ADD{这事成就}},为要应验经上的话。」
\VS{50}门徒都离开他,逃走了。
\par }{\SH 逃走的少年人
\par }{\PP \VS{51}有一个少年人,赤身披着一块麻布,跟随耶稣,众人就捉拿他。
\VS{52}他却丢了麻布,赤身逃走了。
\par }{\SH 耶稣在公会里受审
\par }{\R (太26·57—68;路22·54—55,63—71;约18·13—14,19—24)
\par }{\PP \VS{53}他们把耶稣带到大祭司那里,又有众祭司长和长老并文士都来和大祭司一同聚集。
\VS{54}{\PN{彼得}}远远地跟着耶稣,一直进入大祭司的院里,和差役一同坐在火光里烤火。
\VS{55}祭司长和全公会寻找见证控告耶稣,要治死他,却寻不着。
\VS{56}因为有好些人作假见证告他,只是他们的见证各不相合。
\VS{57}又有几个人站起来作假见证告他,说:
\VS{58}「我们听见他说:『我要拆毁这人手所造的殿,三日内就另造一座不是人手所造的。』」
\VS{59}他们就是这么作见证,也是各不相合。
\VS{60}大祭司起来站在中间,问耶稣说:「你什么都不回答吗?这些人作见证告你的是什么呢?」
\VS{61}耶稣却不言语,一句也不回答。大祭司又问他说:「你是那当称颂者的儿子基督不是?」
\VS{62}耶稣说:「我是。你们必看见人子坐在那权能者的右边,驾着天上的云降临。」
\VS{63}大祭司就撕开衣服,说:「我们何必再用见证人呢?
\VS{64}你们已经听见他这僭妄的话了。你们的意见如何?」他们都定他该死的罪。
\VS{65}就有人吐唾沫在他脸上,又蒙着他的脸,用拳头打他,对他说:「你说预言吧!」差役接过他来,用手掌打他。
\par }{\SH 彼得三次不认主
\par }{\R (太26·69—75;路22·56—62;约18·15—18,25—27)
\par }{\PP \VS{66}{\PN{彼得}}在下边院子里;来了大祭司的一个使女,
\VS{67}见{\PN{彼得}}烤火,就看着他,说:「你素来也是同{\PN{拿撒勒}}人耶稣一伙的。」
\VS{68}{\PN{彼得}}却不承认,说:「我不知道,也不明白你说的是什么。」于是出来,到了前院,鸡就叫了。
\VS{69}那使女看见他,又对旁边站着的人说:「这也是他们一党的。」
\VS{70}{\PN{彼得}}又不承认。过了不多的时候,旁边站着的人又对{\PN{彼得}}说:「你真是他们一党的!因为你是{\PN{加利利}}人。」
\VS{71}{\PN{彼得}}就发咒起誓地说:「我不认得你们说的这个人。」
\VS{72}立时鸡叫了第二遍。{\PN{彼得}}想起耶稣对他所说的话:「鸡叫两遍以先,你要三次不认我。」思想起来,就哭了。

\par }\Chap{15}{\SH 耶稣在彼拉多面前受审
\par }{\R (太27·1—2,11—14;路23·1—5;约18·28—38)
\par }{\PP \VerseOne{1}一到早晨,祭司长和长老、文士、全公会的人大家商议,就把耶稣捆绑,解去交给{\PN{彼拉多}}。
\VS{2}{\PN{彼拉多}}问他说:「你是{\PN{犹太}}人的王吗?」耶稣回答说:「你说的是。」
\VS{3}祭司长告他许多的事。
\VS{4}{\PN{彼拉多}}又问他说:「你看,他们告你这么多的事,你什么都不回答吗?」
\VS{5}耶稣仍不回答,以致{\PN{彼拉多}}觉得希奇。
\par }{\SH 耶稣被判死刑
\par }{\R (太27·15—26;路23·13—25;约18·39—19·16)
\par }{\PP \VS{6}每逢这节期,巡抚照众人所求的,释放一个囚犯给他们。
\VS{7}有一个人名叫{\PN{巴拉巴}},和作乱的人一同捆绑。他们作乱的时候,曾杀过人。
\VS{8}众人上去求巡抚,照常例给他们办。
\VS{9}{\PN{彼拉多}}说:「你们要我释放{\PN{犹太}}人的王给你们吗?」
\VS{10}他原晓得,祭司长是因为嫉妒才把耶稣解了来。
\VS{11}只是祭司长挑唆众人,宁可释放{\PN{巴拉巴}}给他们。
\VS{12}{\PN{彼拉多}}又说:「那么样,你们所称为{\PN{犹太}}人的王,我怎么办他呢?」
\VS{13}他们又喊着说:「把他钉十字架!」
\VS{14}{\PN{彼拉多}}说:「为什么呢?他做了什么恶事呢?」他们便极力地喊着说:「把他钉十字架!」
\VS{15}{\PN{彼拉多}}要叫众人喜悦,就释放{\PN{巴拉巴}}给他们,将耶稣鞭打了,交给人钉十字架。
\par }{\SH 兵丁戏弄耶稣
\par }{\R (太27·27—31;约19·2—3)
\par }{\PP \VS{16}兵丁把耶稣带进衙门院里,叫齐了全营的兵。
\VS{17}他们给他穿上紫袍,又用荆棘编做冠冕给他戴上,
\VS{18}就庆贺他说:「恭喜,{\PN{犹太}}人的王啊!」
\VS{19}又拿一根苇子打他的头,吐唾沫在他脸上,屈膝拜他。
\VS{20}戏弄完了,就给他脱了紫袍,仍穿上他自己的衣服,带他出去,要钉十字架。
\par }{\SH 耶稣被钉十字架
\par }{\R (太27·32—44;路23·26—43;约19·17—27)
\par }{\PP \VS{21}有一个{\PN{古利奈}}人{\PN{西门}},就是{\PN{亚历山大}}和{\PN{鲁孚}}的父亲,从乡下来,经过那地方,他们就勉强他同去,好背着耶稣的十字架。
\VS{22}他们带耶稣到了{\PN{各各他}}地方({\PN{各各他}}翻出来就是髑髅地),
\VS{23}拿没药调和的酒给耶稣,他却不受。
\VS{24}于是将他钉在十字架上,拈阄分他的衣服,看是谁得什么。
\VS{25}钉他在十字架上是巳初的时候。
\VS{26}在上面有他的罪状,写的是:「{\PN{犹太}}人的王。」
\VS{27}他们又把两个强盗和他同钉十字架,一个在右边,一个在左边。\FTNT{}{{\FR 15:27: }有古卷加:28这就应了经上的话说:他被列在罪犯之中。}
\VS{29}从那里经过的人辱骂他,摇着头说:「咳!你这拆毁圣殿、三日又建造起来的,
\VS{30}可以救自己,从十字架上下来吧!」
\VS{31}祭司长和文士也是这样戏弄他,彼此说:「他救了别人,不能救自己。
\VS{32}{\PN{以色列}}的王基督,现在可以从十字架上下来,叫我们看见,就信了。」那和他同钉的人也是讥诮他。
\par }{\SH 耶稣的死
\par }{\R (太27·45—56;路23·44—49;约19·28—30)
\par }{\PP \VS{33}从午正到申初,遍地都黑暗了。
\VS{34}申初的时候,耶稣大声喊着说:「以罗伊!以罗伊!拉马撒巴各大尼?」(翻出来就是:我的 神!我的 神!为什么离弃我?)
\VS{35}旁边站着的人,有的听见就说:「看哪,他叫{\PN{以利亚}}呢!」
\VS{36}有一个人跑去,把海绒蘸满了醋,绑在苇子上,送给他喝,说:「且等着,看{\PN{以利亚}}来不来把他取下。」
\VS{37}耶稣大声喊叫,气就断了。
\VS{38}殿里的幔子从上到下裂为两半。
\VS{39}对面站着的百夫长看见耶稣这样喊叫\FTNT{}{{\FR 15:39: }有古卷没有喊叫二字}断气,就说:「这人真是 神的儿子!」
\VS{40}还有些妇女远远地观看;内中有{\PN{抹大拉}}的{\PN{马利亚}},又有小{\PN{雅各}}和{\PN{约西}}的母亲{\PN{马利亚}},并有{\PN{撒罗米}},
\VS{41}就是耶稣在
{\PN{加利利}}的时候,跟随他、服事他的那些人,还有同耶稣上{\PN{耶路撒冷}}的好些妇女
{\ADD{在那里观看}}。
\par }{\SH 耶稣的安葬
\par }{\R (太27·57—61;路23·50—56;约19·38—42)
\par }{\PP \VS{42}到了晚上,因为这是预备日,就是安息日的前一日,
\VS{43}有{\PN{亚利马太}}的{\PN{约瑟}}前来,他是尊贵的议士,也是等候 神国的。他放胆进去见{\PN{彼拉多}},求耶稣的身体;
\VS{44}{\PN{彼拉多}}诧异耶稣已经死了,便叫百夫长来,问他耶稣死了久不久。
\VS{45}既从百夫长得知实情,就把耶稣的尸首赐给{\PN{约瑟}}。
\VS{46}{\PN{约瑟}}买了细麻布,把耶稣取下来,用细麻布裹好,安放在磐石中凿出来的坟墓里,又滚过一块石头来挡住墓门。
\VS{47}{\PN{抹大拉}}的{\PN{马利亚}}和{\PN{约西}}的母亲{\PN{马利亚}}都看见安放他的地方。

\par }\Chap{16}{\SH 耶稣复活
\par }{\R (太28·1—8;路24·1—12;约20·1—10)
\par }{\PP \VerseOne{1}过了安息日,{\PN{抹大拉}}的{\PN{马利亚}}和{\PN{雅各}}的母亲{\PN{马利亚}}并{\PN{撒罗米}},买了香膏要去膏耶稣的身体。
\VS{2}七日的第一日清早,出太阳的时候,她们来到坟墓那里,
\VS{3}彼此说:「谁给我们把石头从墓门滚开呢?」
\VS{4}那石头原来很大,她们抬头一看,却见石头已经滚开了。
\VS{5}她们进了坟墓,看见一个少年人坐在右边,穿着白袍,就甚惊恐。
\VS{6}那少年人对她们说:「不要惊恐!你们寻找那钉十字架的{\PN{拿撒勒}}人耶稣,他已经复活了,不在这里。请看安放他的地方。
\VS{7}你们可以去告诉他的门徒和{\PN{彼得}},说:『他在你们以先往{\PN{加利利}}去。在那里你们要见他,正如他从前所告诉你们的。』」
\VS{8}她们就出来,从坟墓那里逃跑,又发抖又惊奇,什么也不告诉人,因为她们害怕。
\par }{\SH 向抹大拉的马利亚显现
\par }{\R (太28·9—10;约20·11—18)
\par }{\PP \VS{9}在七日的第一日清早,耶稣复活了,就先向{\PN{抹大拉}}的{\PN{马利亚}}显现(耶稣从她身上曾赶出七个鬼)。
\VS{10}她去告诉那向来跟随耶稣的人;那时他们正哀恸哭泣。
\VS{11}他们听见耶稣活了,被{\PN{马利亚}}看见,却是不信。
\par }{\SH 向两个门徒显现
\par }{\R (路24·13—35)
\par }{\PP \VS{12}这事以后,门徒中间有两个人往乡下去。走路的时候,耶稣变了形象,向他们显现。
\VS{13}他们就去告诉其余的门徒;其余的门徒也是不信。
\par }{\SH 门徒奉差遣
\par }{\R (太28·16—20;路24·36—49;约20·19—23;徒1·6—8)
\par }{\PP \VS{14}后来,十一个门徒坐席的时候,耶稣向他们显现,责备他们不信,心里刚硬,因为他们不信那些在他复活以后看见他的人。
\VS{15}他又对他们说:「你们往普天下去,传福音给万民\FTNT{}{{\FR 16:15: }万民:原文是凡受造的}听。
\VS{16}信而受洗的,必然得救;不信的,必被定罪。
\VS{17}信的人必有神迹随着他们,就是奉我的名赶鬼;说新方言;
\VS{18}手能拿蛇;若喝了什么毒物,也必不受害;手按病人,病人就必好了。」
\par }{\SH 耶稣升天
\par }{\R (路24·50—53;徒1·9—11)
\par }{\PP \VS{19}主耶稣和他们说完了话,后来被接到天上,坐在 神的右边。
\VS{20}门徒出去,到处宣传{\ADD{福音}}。主和他们同工,用神迹随着,证实所传的道。阿们!
\par }