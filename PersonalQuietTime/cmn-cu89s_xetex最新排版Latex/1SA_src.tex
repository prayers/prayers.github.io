\NormalFont\ShortTitle{撒母耳记上}
{\MT 撒母耳记上

\par }\ChapOne{1}{\SH 以利加拿和他的家眷在示罗
\par }{\PP \VerseOne{1}{\PN{以法莲}}山地的{\PN{拉玛·琐非}}有一个{\PN{以法莲}}人,名叫{\PN{以利加拿}},是{\PN{苏弗}}的玄孙,{\PN{托户}}的曾孙,{\PN{以利户}}的孙子,{\PN{耶罗罕}}的儿子。
\VS{2}他有两个妻:一名{\PN{哈拿}},一名{\PN{毗尼拿}}。{\PN{毗尼拿}}有儿女,{\PN{哈拿}}没有儿女。
\par }{\PP \VS{3}这人每年从本城上到{\PN{示罗}},敬拜祭祀万军之耶和华;在那里有{\PN{以利}}的两个儿子{\PN{何弗尼}}、{\PN{非尼哈}}当耶和华的祭司。
\VS{4}{\PN{以利加拿}}每逢献祭的日子,将{\ADD{祭肉}}分给他的妻{\PN{毗尼拿}}和{\PN{毗尼拿}}所生的儿女;
\VS{5}给{\PN{哈拿}}的却是双分,因为他爱{\PN{哈拿}}。无奈耶和华不使{\PN{哈拿}}生育。
\VS{6}{\PN{毗尼拿}}见耶和华不使{\PN{哈拿}}生育,就作她的对头,大大激动她,要使她生气。
\VS{7}每年上到耶和华殿的时候,{\PN{以利加拿}}都以双分给{\PN{哈拿}};{\PN{毗尼拿}}仍是激动她,以致她哭泣不吃饭。
\VS{8}她丈夫{\PN{以利加拿}}对她说:「{\PN{哈拿}}啊,你为何哭泣,不吃饭,心里愁闷呢?有我不比十个儿子还好吗?」
\par }{\SH 哈拿和以利
\par }{\PP \VS{9}他们在{\PN{示罗}}吃喝完了,{\PN{哈拿}}就站起来。祭司{\PN{以利}}在耶和华殿的门框旁边,坐在自己的位上。
\VS{10}{\PN{哈拿}}心里愁苦,就痛痛哭泣,祈祷耶和华,
\VS{11}许愿说:「万军之耶和华啊,你若垂顾婢女的苦情,眷念不忘婢女,赐我一个儿子,我必使他终身归与耶和华,不用剃头刀剃他的头。」
\par }{\PP \VS{12}{\PN{哈拿}}在耶和华面前不住地祈祷,{\PN{以利}}定睛看她的嘴。(
\VS{13}原来{\PN{哈拿}}心中默祷,只动嘴唇,不出声音,因此{\PN{以利}}以为她喝醉了。)
\VS{14}{\PN{以利}}对她说:「你要醉到几时呢?你不应该喝酒。」
\VS{15}{\PN{哈拿}}回答说:「主啊,不是这样。我是心里愁苦的妇人,清酒浓酒都没有喝,但在耶和华面前倾心吐意。
\VS{16}不要将婢女看作不正经的女子。我因被人激动,愁苦太多,所以祈求到如今。」
\VS{17}{\PN{以利}}说:「你可以平平安安地回去。愿{\PN{以色列}}的 神允准你向他所求的!」
\VS{18}{\PN{哈拿}}说:「愿婢女在你眼前蒙恩。」于是妇人走去吃饭,面上再不带愁容了。
\par }{\SH 撒母耳出生和奉献
\par }{\PP \VS{19}次日清早,他们起来,在耶和华面前敬拜,就回{\PN{拉玛}}。到了家里,{\PN{以利加拿}}和妻{\PN{哈拿}}同房,耶和华顾念{\PN{哈拿}},
\VS{20}{\PN{哈拿}}就怀孕。日期满足,生了一个儿子,给他起名叫{\PN{撒母耳}},{\ADD{说}}:「这是我从耶和华那里求来的。」
\par }{\PP \VS{21}{\PN{以利加拿}}和他全家都上{\PN{示罗}}去,要向耶和华献年祭,并还所许的愿。
\VS{22}{\PN{哈拿}}却没有上去,对丈夫说:「等孩子断了奶,我便带他上去朝见耶和华,使他永远住在那里。」
\VS{23}她丈夫{\PN{以利加拿}}说:「就随你的意行吧!可以等儿子断了奶。但愿耶和华应验他的话。」于是妇人在家里乳养儿子,直到断了奶;
\VS{24}既断了奶,就把孩子带上{\PN{示罗}},到了耶和华的殿;又带了三只公牛,一伊法细面,一皮袋酒。(那时,孩子还小。)
\VS{25}宰了一只公牛,就领孩子到{\PN{以利}}面前。
\VS{26}妇人说:「主啊,我敢在你面前起誓,从前在你这里站着祈求耶和华的那妇人,就是我。
\VS{27}我祈求为要得这孩子;耶和华已将我所求的赐给我了。
\VS{28}所以,我将这孩子归与耶和华,使他终身归与耶和华。」
\par }{\Q 于是在那里敬拜耶和华。

\par }\Chap{2}{\SH 哈拿的祷告
\par }{\PP \VerseOne{1}{\PN{哈拿}}祷告说:
\par }{\Q 我的心因耶和华快乐;
\par }{\Q 我的角因耶和华高举。
\par }{\Q 我的口向仇敌张开;
\par }{\Q 我因耶和华的救恩欢欣。
\par }{\BB \par }{\Q \VS{2}只有耶和华为圣;
\par }{\Q 除他以外没有可比的,
\par }{\Q 也没有磐石像我们的 神。
\par }{\Q \VS{3}人不要夸口说骄傲的话,
\par }{\Q 也不要出狂妄的言语;
\par }{\Q 因耶和华是大有智识的 神,
\par }{\Q 人的行为被他衡量。
\par }{\Q \VS{4}勇士的弓都已折断;
\par }{\Q 跌倒的人以力量束腰。
\par }{\Q \VS{5}素来饱足的,反作用人求食;
\par }{\Q 饥饿的,再不饥饿。
\par }{\Q 不生育的,生了七个儿子;
\par }{\Q 多有儿女的,反倒衰微。
\par }{\Q \VS{6}耶和华使人死,也使人活,
\par }{\Q 使人下阴间,也使人往上升。
\par }{\Q \VS{7}他使人贫穷,也使人富足,
\par }{\Q 使人卑微,也使人高贵。
\par }{\Q \VS{8}他从灰尘里抬举贫寒人,
\par }{\Q 从粪堆中提拔穷乏人,
\par }{\Q 使他们与王子同坐,
\par }{\Q 得着荣耀的座位。
\par }{\Q 地的柱子属于耶和华;
\par }{\Q 他将世界立在其上。
\par }{\BB \par }{\Q \VS{9}他必保护圣民的脚步,
\par }{\Q 使恶人在黑暗中寂然不动;
\par }{\Q 人都不能靠力量得胜。
\par }{\Q \VS{10}与耶和华争竞的,必被打碎;
\par }{\Q 耶和华必从天上以雷攻击他,
\par }{\Q 必审判地极的人,
\par }{\Q 将力量赐与所立的王,
\par }{\Q 高举受膏者的角。
\par }{\PP \VS{11}{\PN{以利加拿}}往{\PN{拉玛}}回家去了。那孩子在祭司{\PN{以利}}面前事奉耶和华。
\par }{\SH 以利的两个儿子
\par }{\PP \VS{12}{\PN{以利}}的两个儿子是恶人,不认识耶和华。
\VS{13}这二祭司待百姓是这样的规矩:凡有人献祭,正煮肉的时候,祭司的仆人就来,手拿三齿的叉子,
\VS{14}将叉子往罐里,或鼎里,或釜里,或锅里一插,插上来的肉,祭司都取了去。凡上到{\PN{示罗}}的{\PN{以色列}}人,他们都是这样看待。
\VS{15}又在未烧脂油以前,祭司的仆人就来对献祭的人说:「将肉给祭司,叫他烤吧。他不要煮过的,要生的。」
\VS{16}献祭的人若说:「必须先烧脂油,然后你可以随意取肉。」仆人就说:「你立时给我,不然我便抢去。」
\VS{17}如此,这二少年人的罪在耶和华面前甚重了,因为他们藐视耶和华的祭物\FTNT{}{{\FR 2:17: }或译:他们使人厌弃给耶和华献祭}。
\par }{\SH 撒母耳在示罗
\par }{\PP \VS{18}那时,{\PN{撒母耳}}还是孩子,穿着细麻布的以弗得,侍立在耶和华面前。
\VS{19}他母亲每年为他做一件小外袍,同着丈夫上来献年祭的时候带来给他。
\VS{20}{\PN{以利}}为{\PN{以利加拿}}和他的妻祝福,说:「愿耶和华由这妇人再赐你后裔,代替你从耶和华求来的孩子。」他们就回本乡去了。
\par }{\PP \VS{21}耶和华眷顾{\PN{哈拿}},她就怀孕生了三个儿子,两个女儿。那孩子{\PN{撒母耳}}在耶和华面前渐渐长大。
\par }{\SH 以利和他的儿子
\par }{\PP \VS{22}{\PN{以利}}年甚老迈,听见他两个儿子待{\PN{以色列}}众人的事,又听见他们与会幕门前伺候的妇人苟合,
\VS{23}他就对他们说:「你们为何行这样的事呢?我从这众百姓听见你们的恶行。
\VS{24}我儿啊,不可这样!我听见你们的风声不好,你们使耶和华的百姓犯了罪。
\VS{25}人若得罪人,有士师审判他;人若得罪耶和华,谁能为他祈求呢?」然而他们还是不听父亲的话,因为耶和华想要杀他们。
\par }{\PP \VS{26}孩子{\PN{撒母耳}}渐渐长大,耶和华与人越发喜爱他。
\par }{\SH 预言以利家遭祸
\par }{\PP \VS{27}有神人来见{\PN{以利}},对他说:「耶和华如此说:『你祖父在{\PN{埃及}}法老家{\ADD{作奴仆}}的时候,我不是向他们显现吗?
\VS{28}在{\PN{以色列}}众支派中,我不是拣选人作我的祭司,使他烧香,在我坛上献祭,在我面前穿以弗得,又将{\PN{以色列}}人所献的火祭都赐给你父家吗?
\VS{29}我所吩咐献在我居所的祭物,你们为何践踏?尊重你的儿子过于尊重我,将我民{\PN{以色列}}所献美好的祭物肥己呢?』
\VS{30}因此,耶和华—{\PN{以色列}}的 神说:『我曾说,你和你父家必永远行在我面前;现在我却说,决不容你们这样行。因为尊重我的,我必重看他;藐视我的,他必被轻视。
\VS{31}日子必到,我要折断你的膀臂和你父家的膀臂,使你家中没有一个老年人。
\VS{32}在 {\ADD{神}}使{\PN{以色列}}人享福的时候,你必看见我居所的败落。在你家中必永远没有一个老年人。
\VS{33}我必不从我坛前灭尽你家中的人;那未灭的必使你眼目干瘪、心中忧伤。你家中所生的人都必死在中年。
\VS{34}你的两个儿子{\PN{何弗尼}}、{\PN{非尼哈}}所遭遇的事可作你的证据:他们二人必一日同死。
\VS{35}我要为自己立一个忠心的祭司;他必照我的心意而行。我要为他建立坚固的家;他必永远行在我的受膏者面前。
\VS{36}你家所剩下的人都必来叩拜他,求块银子,求个饼,说:求你赐我祭司的职分,好叫我得点饼吃。』」

\par }\Chap{3}{\SH 耶和华向撒母耳显现
\par }{\PP \VerseOne{1}童子{\PN{撒母耳}}在{\PN{以利}}面前事奉耶和华。当那些日子,耶和华的言语稀少,不常有默示。
\VS{2}一日,{\PN{以利}}睡卧在自己的地方;他眼目昏花,看不分明。
\VS{3}神的灯在 神耶和华殿内{\ADD{约}}柜那里,还没有熄灭,{\PN{撒母耳}}已经睡了。
\VS{4}耶和华呼唤{\PN{撒母耳}}。{\PN{撒母耳}}说:「我在这里!」
\VS{5}就跑到{\PN{以利}}那里,说:「你呼唤我?我在这里。」{\PN{以利}}回答说:「我没有呼唤你,你去睡吧。」他就去睡了。
\VS{6}耶和华又呼唤{\PN{撒母耳}}。{\PN{撒母耳}}起来,到{\PN{以利}}那里,说:「你呼唤我?我在这里。」{\PN{以利}}回答说:「我的儿,我没有呼唤你,你去睡吧。」
\VS{7}那时{\PN{撒母耳}}还未认识耶和华,也未得耶和华的默示。
\VS{8}耶和华第三次呼唤{\PN{撒母耳}}。{\PN{撒母耳}}起来,到{\PN{以利}}那里,说:「你又呼唤我?我在这里。」{\PN{以利}}才明白是耶和华呼唤童子。
\VS{9}因此{\PN{以利}}对{\PN{撒母耳}}说:「你仍去睡吧;若再呼唤你,你就说:『耶和华啊,请说,仆人敬听!』」{\PN{撒母耳}}就去,仍睡在原处。
\par }{\PP \VS{10}耶和华又来站着,像前三次呼唤说:「{\PN{撒母耳}}啊!{\PN{撒母耳}}啊!」{\PN{撒母耳}}回答说:「请说,仆人敬听!」
\VS{11}耶和华对{\PN{撒母耳}}说:「我在{\PN{以色列}}中必行一件事,叫听见的人都必耳鸣。
\VS{12}我指着{\PN{以利}}家所说的话,到了时候,我必始终应验在{\PN{以利}}身上。
\VS{13}我曾告诉他必永远降罚与他的家,因他知道儿子作孽,自招咒诅,却不禁止他们。
\VS{14}所以我向{\PN{以利}}家起誓说:『{\PN{以利}}家的罪孽,虽献祭奉礼物,永不能得赎去。』」
\par }{\PP \VS{15}{\PN{撒母耳}}睡到天亮,就开了耶和华的殿门,不敢将默示告诉{\PN{以利}}。
\VS{16}{\PN{以利}}呼唤{\PN{撒母耳}}说:「我儿{\PN{撒母耳}}啊!」{\PN{撒母耳}}回答说:「我在这里!」
\VS{17}{\PN{以利}}说:「{\ADD{耶和华}}对你说什么,你不要向我隐瞒;你若将 神对你所说的隐瞒一句,愿他重重地降罚与你。」
\VS{18}{\PN{撒母耳}}就把一切话都告诉了{\PN{以利}},并没有隐瞒。{\PN{以利}}说:「这是出于耶和华,愿他凭自己的意旨而行。」
\par }{\PP \VS{19}{\PN{撒母耳}}长大了,耶和华与他同在,使他所说的话一句都不落空。
\VS{20}从{\PN{但}}到{\PN{别是巴}}所有的{\PN{以色列}}人都知道耶和华立{\PN{撒母耳}}为先知。
\VS{21}耶和华又在{\PN{示罗}}显现;因为耶和华将自己的话默示{\PN{撒母耳}},{\PN{撒母耳}}就把这话传遍{\PN{以色列}}地。

\par }\Chap{4}{\SH 约柜被掳
\par }{\PP \VerseOne{1}{\PN{以色列}}人出去与{\PN{非利士}}人打仗,安营在{\PN{以便以谢}};{\PN{非利士}}人安营在{\PN{亚弗}}。
\VS{2}{\PN{非利士}}人向{\PN{以色列}}人摆阵。两军交战的时候,{\PN{以色列}}人败在{\PN{非利士}}人面前;{\PN{非利士}}人在战场上杀了他们的军兵约有四千人。
\VS{3}百姓回到营里,{\PN{以色列}}的长老说:「耶和华今日为何使我们败在{\PN{非利士}}人面前呢?我们不如将耶和华的约柜从{\PN{示罗}}抬到我们这里来,好在我们中间救我们脱离敌人的手。」
\VS{4}于是百姓打发人到{\PN{示罗}},从那里将坐在二基路伯上万军之耶和华的约柜抬来。{\PN{以利}}的两个儿子{\PN{何弗尼}}、{\PN{非尼哈}}与 神的约柜同来。
\par }{\PP \VS{5}耶和华的约柜到了营中,{\PN{以色列}}众人就大声欢呼,地便震动。
\VS{6}{\PN{非利士}}人听见欢呼的声音,就说:「在{\PN{希伯来}}人营里大声欢呼,是什么缘故呢?」随后就知道耶和华的{\ADD{约}}柜到了营中。
\VS{7}{\PN{非利士}}人就惧怕起来,说:「有神到了他们营中」;又说:「我们有祸了!向来不曾有这样的事。
\VS{8}我们有祸了!谁能救我们脱离这些大能之神的手呢?从前在旷野用各样灾殃击打{\PN{埃及}}人的,就是这些神。
\VS{9}{\PN{非利士}}人哪,你们要刚强,要作大丈夫,免得作{\PN{希伯来}}人的奴仆,如同他们作你们的奴仆一样。你们要作大丈夫,与他们争战。」
\VS{10}{\PN{非利士}}人和{\PN{以色列}}人打仗,{\PN{以色列}}人败了,各向各家奔逃,被杀的人甚多,{\PN{以色列}}的步兵仆倒了三万。
\VS{11}神的{\ADD{约}}柜被掳去,{\PN{以利}}的两个儿子{\PN{何弗尼}}、{\PN{非尼哈}}也都被杀了。
\par }{\SH 以利的死
\par }{\PP \VS{12}当日,有一个{\PN{便雅悯}}人从阵上逃跑,衣服撕裂,头蒙灰尘,来到{\PN{示罗}}。
\VS{13}到了的时候,{\PN{以利}}正在道旁坐在自己的位上观望,为 神的{\ADD{约}}柜心里担忧。那人进城报信,合城的人就都呼喊起来。
\VS{14}{\PN{以利}}听见呼喊的声音就问说:「这喧嚷是什么缘故呢?」那人急忙来报信给{\PN{以利}}。
\VS{15}那时{\PN{以利}}九十八岁了,眼目发直,不能看见。
\VS{16}那人对{\PN{以利}}说:「我是从阵上来的,今日我从阵上逃回。」{\PN{以利}}说:「我儿,事情怎样?」
\VS{17}报信的回答说:「{\PN{以色列}}人在{\PN{非利士}}人面前逃跑,民中被杀的甚多!你的两个儿子{\PN{何弗尼}}、{\PN{非尼哈}}也都死了,并且 神的{\ADD{约}}柜被掳去。」
\VS{18}他一提 神的{\ADD{约}}柜,{\PN{以利}}就从他的位上往后跌倒,在门旁折断颈项而死;因为他年纪老迈,身体沉重。{\PN{以利}}作{\PN{以色列}}的士师四十年。
\par }{\SH 非尼哈遗孀的死
\par }{\PP \VS{19}{\PN{以利}}的儿妇、{\PN{非尼哈}}的妻怀孕将到产期,她听见 神的{\ADD{约}}柜被掳去,公公和丈夫都死了,就猛然疼痛,曲身生产;
\VS{20}将要死的时候,旁边站着的妇人们对她说:「不要怕!你生了男孩子了。」她却不回答,也不放在心上。
\VS{21}她给孩子起名叫{\PN{以迦博}},说:「荣耀离开{\PN{以色列}}了!」这是因 神的{\ADD{约}}柜被掳去,又因她公公和丈夫都死了。
\VS{22}她又说:「荣耀离开{\PN{以色列}},因为 神的约柜被掳去了。」

\par }\Chap{5}{\SH 约柜在非利士人当中
\par }{\PP \VerseOne{1}{\PN{非利士}}人将 神的{\ADD{约}}柜从{\PN{以便以谢}}抬到{\PN{亚实突}}。
\VS{2}{\PN{非利士}}人将 神的{\ADD{约}}柜抬进{\PN{大衮}}庙,放在{\PN{大衮}}的旁边。
\VS{3}次日清早,{\PN{亚实突}}人起来,见{\PN{大衮}}仆倒在耶和华的{\ADD{约}}柜前,脸伏于地,就把{\PN{大衮}}仍立在原处。
\VS{4}又次日清早起来,见{\PN{大衮}}仆倒在耶和华的{\ADD{约}}柜前,脸伏于地,并且{\PN{大衮}}的头和两手都在门槛上折断,只剩下{\PN{大衮}}的{\ADD{残体}}。(
\VS{5}因此,{\PN{大衮}}的祭司和一切进{\PN{亚实突}}、{\PN{大衮}}庙的人都不踏{\PN{大衮}}庙的门槛,直到今日。)
\par }{\PP \VS{6}耶和华的手重重加在{\PN{亚实突}}人身上,败坏他们,使他们生痔疮。{\PN{亚实突}}和{\PN{亚实突}}的四境都是如此。
\VS{7}{\PN{亚实突}}人见这光景,就说:「{\PN{以色列}} 神的{\ADD{约}}柜不可留在我们这里,因为他的手重重加在我们和我们神{\PN{大衮}}的身上」;
\VS{8}就打发人去请{\PN{非利士}}的众首领来聚集,问他们说:「我们向{\PN{以色列}} 神的{\ADD{约}}柜应当怎样行呢?」他们回答说:「可以将{\PN{以色列}} 神的{\ADD{约}}柜运到{\PN{迦特}}去。」于是将{\PN{以色列}} 神的{\ADD{约}}柜运到那里去。
\VS{9}运到之后,耶和华的手攻击那城,使那城的人大大惊慌,无论大小都生痔疮。
\VS{10}他们就把 神的{\ADD{约}}柜送到{\PN{以革伦}}。 神的{\ADD{约}}柜到了,{\PN{以革伦}}人就喊嚷起来说:「他们将{\PN{以色列}} 神的{\ADD{约}}柜运到我们这里,要害我们和我们的众民!」
\VS{11}于是打发人去请{\PN{非利士}}的众首领来,说:「愿你们将{\PN{以色列}} 神的{\ADD{约}}柜送回原处,免得害了我们和我们的众民!」原来 神的手重重攻击那城,城中的人有因惊慌而死的;
\VS{12}未曾死的人都生了痔疮。合城呼号,声音上达于天。

\par }\Chap{6}{\SH 送回约柜
\par }{\PP \VerseOne{1}耶和华的{\ADD{约}}柜在{\PN{非利士}}人之地七个月。
\VS{2}{\PN{非利士}}人将祭司和占卜的聚了来,问他们说:「我们向耶和华的{\ADD{约}}柜应当怎样行?请指示我们用何法将{\ADD{约}}柜送回原处。」
\VS{3}他们说:「若要将{\PN{以色列}} 神的{\ADD{约}}柜送回去,不可空空地送去,必要给他献赔罪的礼物,然后你们可得痊愈,并知道他的手为何不离开你们。」
\VS{4}{\PN{非利士}}人说:「应当用什么献为赔罪的礼物呢?」他们回答说:「当照{\PN{非利士}}首领的数目,用五个金痔疮,五个金老鼠,因为在你们众人和你们首领的身上都是一样的灾。
\VS{5}所以当制造你们痔疮的像和毁坏你们田地老鼠的像,并要归荣耀给{\PN{以色列}}的 神,或者他向你们和你们的神,并你们的田地,把手放轻些。
\VS{6}你们为何硬着心像{\PN{埃及}}人和法老一样呢? 神在{\PN{埃及}}人中间行奇事,{\PN{埃及}}人岂不释放{\PN{以色列}}人,他们就去了吗?
\VS{7}现在你们应当造一辆新车,将两只未曾负轭有乳的母牛套在车上,使牛犊回家去,离开母牛。
\VS{8}把耶和华的{\ADD{约}}柜放在车上,将所献赔罪的金物装在匣子里,放在柜旁,将柜送去。
\VS{9}你们要看看:车若直行{\PN{以色列}}的境界到{\PN{伯·示麦}}去,这大灾就是耶和华降在我们身上的;若不然,便可以知道不是他的手击打我们,是我们偶然遇见的。」
\par }{\PP \VS{10}{\PN{非利士}}人就这样行:将两只有乳的母牛套在车上,将牛犊关在家里,
\VS{11}把耶和华的{\ADD{约}}柜和装金老鼠并金痔疮像的匣子都放在车上。
\VS{12}牛直行大道,往{\PN{伯·示麦}}去,一面走一面叫,不偏左右。{\PN{非利士}}的首领跟在后面,直到{\PN{伯·示麦}}的境界。
\VS{13}{\PN{伯·示麦}}人正在平原收割麦子,举目看见{\ADD{约}}柜,就欢喜了。
\VS{14}车到了{\PN{伯·示麦}}人{\PN{约书亚}}的田间,就站住了。在那里有一块大磐石,他们把车劈了,将两只母牛献给耶和华为燔祭。
\VS{15}{\PN{利未}}人将耶和华的{\ADD{约}}柜和装金物的匣子拿下来,放在大磐石上。当日{\PN{伯·示麦}}人将燔祭和{\ADD{平安}}祭献给耶和华。
\VS{16}{\PN{非利士}}人的五个首领看见,当日就回{\PN{以革伦}}去了。
\par }{\PP \VS{17}{\PN{非利士}}人献给耶和华作赔罪的金痔疮像,就是这些:一个是为{\PN{亚实突}},一个是为{\PN{迦萨}},一个是为{\PN{亚实基伦}},一个是为{\PN{迦特}},一个是为{\PN{以革伦}}。
\VS{18}金老鼠的数目是照{\PN{非利士}}五个首领的城邑,就是坚固的城邑和乡村,以及大磐石。这磐石是放耶和华{\ADD{约}}柜的,到今日还在{\PN{伯·示麦}}人{\PN{约书亚}}的田间。
\par }{\SH 约柜在基列·耶琳
\par }{\PP \VS{19}耶和华因{\PN{伯·示麦}}人擅观他的{\ADD{约}}柜,就击杀了他们七十人;那时有五万人在那里\FTNT{}{{\FR 6:19: }原文是七十人加五万人}。百姓因耶和华大大击杀他们,就哀哭了。
\VS{20}{\PN{伯·示麦}}人说:「谁能在耶和华这圣洁的 神面前侍立呢?这{\ADD{约}}柜可以从我们这里送到谁那里去呢?」
\VS{21}于是打发人去见{\PN{基列·耶琳}}的居民,说:「{\PN{非利士}}人将耶和华的{\ADD{约}}柜送回来了,你们下来将{\ADD{约}}柜接到你们那里去吧!」

\par }\Chap{7}{\PP \VerseOne{1}{\PN{基列·耶琳}}人就下来,将耶和华的{\ADD{约}}柜接上去,放在山上{\PN{亚比拿达}}的家中,分派他儿子{\PN{以利亚撒}}看守耶和华的{\ADD{约}}柜。
\VS{2}{\ADD{约}}柜在{\PN{基列·耶琳}}许久。过了二十年,{\PN{以色列}}全家都倾向耶和华。
\par }{\SH 撒母耳统治以色列
\par }{\PP \VS{3}{\PN{撒母耳}}对{\PN{以色列}}全家说:「你们若一心归顺耶和华,就要把外邦的神和{\PN{亚斯她录}}从你们中间除掉,专心归向耶和华,单单地事奉他。他必救你们脱离{\PN{非利士}}人的手。」
\VS{4}{\PN{以色列}}人就除掉诸{\PN{巴力}}和{\PN{亚斯她录}},单单地事奉耶和华。
\par }{\PP \VS{5}{\PN{撒母耳}}说:「要使{\PN{以色列}}众人聚集在{\PN{米斯巴}},我好为你们祷告耶和华。」
\VS{6}他们就聚集在{\PN{米斯巴}},打水浇在耶和华面前,当日禁食,说:「我们得罪了耶和华。」于是{\PN{撒母耳}}在{\PN{米斯巴}}审判{\PN{以色列}}人。
\VS{7}{\PN{非利士}}人听见{\PN{以色列}}人聚集在{\PN{米斯巴}},{\PN{非利士}}的首领就上来要攻击{\PN{以色列}}人。{\PN{以色列}}人听见,就惧怕{\PN{非利士}}人。
\VS{8}{\PN{以色列}}人对{\PN{撒母耳}}说:「愿你不住地为我们呼求耶和华—我们的 神,救我们脱离{\PN{非利士}}人的手。」
\VS{9}{\PN{撒母耳}}就把一只吃奶的羊羔献与耶和华作全牲的燔祭,为{\PN{以色列}}人呼求耶和华;耶和华就应允他。
\VS{10}{\PN{撒母耳}}正献燔祭的时候,{\PN{非利士}}人前来要与{\PN{以色列}}人争战。当日,耶和华大发雷声,惊乱{\PN{非利士}}人,他们就败在{\PN{以色列}}人面前。
\VS{11}{\PN{以色列}}人从{\PN{米斯巴}}出来,追赶{\PN{非利士}}人,击杀他们,直到{\PN{伯·甲}}的下边。
\par }{\PP \VS{12}{\PN{撒母耳}}将一块石头立在{\PN{米斯巴}}和{\PN{善}}的中间,给石头起名叫{\PN{以便以谢}},说:「到如今耶和华都帮助我们。」
\VS{13}从此,{\PN{非利士}}人就被制伏,不敢再入{\PN{以色列}}人的境内。{\PN{撒母耳}}作士师的时候,耶和华的手攻击{\PN{非利士}}人。
\VS{14}{\PN{非利士}}人所取{\PN{以色列}}人的城邑,从{\PN{以革伦}}直到{\PN{迦特}},都归{\PN{以色列}}人了。属这些城的四境,{\PN{以色列}}人也从{\PN{非利士}}人手下收回。那时{\PN{以色列}}人与{\PN{亚摩利}}人和好。
\par }{\PP \VS{15}{\PN{撒母耳}}平生作{\PN{以色列}}的士师。
\VS{16}他每年巡行到{\PN{伯特利}}、{\PN{吉甲}}、{\PN{米斯巴}},在这几处审判{\PN{以色列}}人。
\VS{17}随后回到{\PN{拉玛}},因为他的家在那里;也在那里审判{\PN{以色列}}人,且为耶和华筑了一座坛。

\par }\Chap{8}{\SH 以色列人要求立王
\par }{\PP \VerseOne{1}{\PN{撒母耳}}年纪老迈,就立他儿子作{\PN{以色列}}的士师。
\VS{2}长子名叫{\PN{约珥}},次子名叫{\PN{亚比亚}};他们在{\PN{别是巴}}作士师。
\VS{3}他儿子不行他的道,贪图财利,收受贿赂,屈枉正直。
\par }{\PP \VS{4}{\PN{以色列}}的长老都聚集,来到{\PN{拉玛}}见{\PN{撒母耳}},
\VS{5}对他说:「你年纪老迈了,你儿子不行你的道。现在求你为我们立一个王治理我们,像列国一样。」
\VS{6}{\PN{撒母耳}}不喜悦他们说「立一个王治理我们」,他就祷告耶和华。
\VS{7}耶和华对{\PN{撒母耳}}说:「百姓向你说的一切话,你只管依从;因为他们不是厌弃你,乃是厌弃我,不要我作他们的王。
\VS{8}自从我领他们出{\PN{埃及}}到如今,他们常常离弃我,事奉别神。现在他们向你所行的,是照他们素来所行的。
\VS{9}故此你要依从他们的话,只是当警戒他们,告诉他们将来那王怎样管辖他们。」
\par }{\PP \VS{10}{\PN{撒母耳}}将耶和华的话都传给求他立王的百姓,说:
\VS{11}「管辖你们的王必这样行:他必派你们的儿子为他赶车、跟马,奔走在车前;
\VS{12}又派他们作千夫长、五十夫长,为他耕种田地,收割庄稼,打造军器和车上的器械;
\VS{13}必取你们的女儿为他制造香膏,做饭烤饼;
\VS{14}也必取你们最好的田地、葡萄园、橄榄园赐给他的臣仆。
\VS{15}你们的粮食和葡萄园所出的,他必取十分之一给他的太监和臣仆;
\VS{16}又必取你们的仆人婢女,健壮的少年人和你们的驴,供他的差役。
\VS{17}你们的羊群他必取十分之一,你们也必作他的仆人。
\VS{18}那时你们必因所选的王哀求耶和华,耶和华却不应允你们。」
\par }{\PP \VS{19}百姓竟不肯听{\PN{撒母耳}}的话,说:「不然!我们定要一个王治理我们,
\VS{20}使我们像列国一样,有王治理我们,统领我们,为我们争战。」
\VS{21}{\PN{撒母耳}}听见百姓这一切话,就将这话陈明在耶和华面前。
\VS{22}耶和华对{\PN{撒母耳}}说:「你只管依从他们的话,为他们立王。」{\PN{撒母耳}}对{\PN{以色列}}人说:「你们各归各城去吧!」

\par }\Chap{9}{\SH 扫罗见撒母耳
\par }{\PP \VerseOne{1}有一个{\PN{便雅悯}}人,名叫{\PN{基士}},是{\PN{便雅悯}}人{\PN{亚斐亚}}的玄孙,{\PN{比歌拉}}的曾孙,{\PN{洗罗}}的孙子,{\PN{亚别}}的儿子,是个大能的勇士\FTNT{}{{\FR 9:1: }或译:大财主}。
\VS{2}他有一个儿子,名叫{\PN{扫罗}},又健壮、又俊美,在{\PN{以色列}}人中没有一个能比他的;身体比众民高过一头。
\par }{\PP \VS{3}{\PN{扫罗}}的父亲{\PN{基士}}丢了几头驴,他就吩咐儿子{\PN{扫罗}}说:「你带一个仆人去寻找驴。」
\VS{4}{\PN{扫罗}}就走过{\PN{以法莲}}山地,又过{\PN{沙利沙}}地,都没有找着;又过{\PN{沙琳}}地,驴也不在那里;又过{\PN{便雅悯}}地,还没有找着。
\par }{\PP \VS{5}到了{\PN{苏弗}}地,{\PN{扫罗}}对跟随他的仆人说:「我们不如回去,恐怕我父亲不为驴挂心,反为我们担忧。」
\VS{6}仆人说:「这城里有一位神人,是众人所尊重的,凡他所说的全都应验。我们不如往他那里去,或者他能将我们当走的路指示我们。」
\VS{7}{\PN{扫罗}}对仆人说:「我们若去,有什么可以送那人呢?我们囊中的食物都吃尽了,也没有礼物可以送那神人,我们还有什么没有?」
\VS{8}仆人回答{\PN{扫罗}}说:「我手里有银子一舍客勒的四分之一,可以送那神人,请他指示我们当走的路。」(
\VS{9}从前{\PN{以色列}}中,若有人去问 神,就说:「我们问先见去吧!」现在称为「先知」的,从前称为「先见」。)
\VS{10}{\PN{扫罗}}对仆人说:「你说的是,我们可以去。」于是他们往神人所住的城里去了。
\par }{\PP \VS{11}他们上坡要进城,就遇见几个少年女子出来打水,问她们说:「先见在这里没有?」
\VS{12}女子回答说:「在这里,他在你们前面。快去吧!他今日正到城里,因为今日百姓要在邱坛献祭。
\VS{13}在他还没有上邱坛吃祭物之先,你们一进城必遇见他;因他未到,百姓不能吃,必等他先祝祭,然后请的客才吃。现在你们上去,这时候必遇见他。」
\VS{14}二人就上去;将进城的时候,{\PN{撒母耳}}正迎着他们来,要上邱坛去。
\par }{\PP \VS{15}{\PN{扫罗}}未到的前一日,耶和华已经指示{\PN{撒母耳}}说:
\VS{16}「明日这时候,我必使一个人从{\PN{便雅悯}}地到你这里来,你要膏他作我民{\PN{以色列}}的君。他必救我民脱离{\PN{非利士}}人的手;因我民的哀声上达于我,我就眷顾他们。」
\VS{17}{\PN{撒母耳}}看见{\PN{扫罗}}的时候,耶和华对他说:「看哪,这人就是我对你所说的,他必治理我的民。」
\VS{18}{\PN{扫罗}}在城门里走到{\PN{撒母耳}}跟前,说:「请告诉我,先见的寓所在哪里?」
\VS{19}{\PN{撒母耳}}回答说:「我就是先见。你在我前面上邱坛去,因为你们今日必与我同席;明日早晨我送你去,将你心里的事都告诉你。
\VS{20}至于你前三日所丢的那几头驴,你心里不必挂念,已经找着了。{\PN{以色列}}众人所仰慕的是谁呢?不是仰慕你和你父的全家吗?」
\VS{21}{\PN{扫罗}}说:「我不是{\PN{以色列}}支派中至小的{\PN{便雅悯}}人吗?我家不是{\PN{便雅悯}}支派中至小的家吗?你为何对我说这样的话呢?」
\par }{\PP \VS{22}{\PN{撒母耳}}领{\PN{扫罗}}和他仆人进了客堂,使他们在请来的客中坐首位;客约有三十个人。
\VS{23}{\PN{撒母耳}}对厨役说:「我交给你收存的那一分{\ADD{祭肉}}现在可以拿来。」
\VS{24}厨役就把收存的腿拿来,摆在{\PN{扫罗}}面前,{\PN{撒母耳}}说:「这是所留下的,放在你面前。吃吧!因我请百姓的时候,特意为你存留这肉到此时。」
\par }{\PP 当日,{\PN{扫罗}}就与{\PN{撒母耳}}同席。
\par }{\SH 撒母耳膏扫罗作王
\par }{\PP \VS{25}众人从邱坛下来进城,{\PN{撒母耳}}和{\PN{扫罗}}在房顶上说话。
\VS{26}次日清早起来,黎明的时候,{\PN{扫罗}}在房顶上。{\PN{撒母耳}}呼叫他说:「起来吧,我好送你回去。」{\PN{扫罗}}就起来,和{\PN{撒母耳}}一同出去。
\VS{27}二人下到城角,{\PN{撒母耳}}对{\PN{扫罗}}说:「要吩咐仆人先走(仆人就先走了);你且站在这里,等我将 神的话传与你听。」

\par }\Chap{10}{\PP \VerseOne{1}{\PN{撒母耳}}拿瓶膏油倒在{\PN{扫罗}}的头上,与他亲嘴,说:「这不是耶和华膏你作他产业的君吗?
\VS{2}你今日与我离别之后,在{\PN{便雅悯}}境内的{\PN{泄撒}},靠近{\PN{拉结}}的坟墓,要遇见两个人。他们必对你说:『你去找的那几头驴已经找着了。现在你父亲不为驴挂心,反为你担忧,说:我为儿子怎么才好呢?』
\VS{3}你从那里往前行,到了{\PN{他泊}}的橡树那里,必遇见三个往{\PN{伯特利}}去拜 神的人:一个带着三只山羊羔,一个带着三个饼,一个带着一皮袋酒。
\VS{4}他们必问你安,给你两个饼,你就从他们手中接过来。
\VS{5}此后你到 神的山,在那里有{\PN{非利士}}人的防兵。你到了城的时候,必遇见一班先知从邱坛下来,前面有鼓瑟的、击鼓的、吹笛的、弹琴的,他们都受感说话。
\VS{6}耶和华的灵必大大感动你,你就与他们一同受感说话;你要变为新人。
\VS{7}这兆头临到你,你就可以趁时而做,因为 神与你同在。
\VS{8}你当在我以先下到{\PN{吉甲}},我也必下到那里献燔祭和平安祭。你要等候七日,等我到了那里,指示你当行的事。」
\par }{\PP \VS{9}{\PN{扫罗}}转身离别{\PN{撒母耳}}, 神就赐他一个新心。当日这一切兆头都应验了。
\VS{10}{\PN{扫罗}}到了那山,有一班先知遇见他, 神的灵大大感动他,他就在先知中受感说话。
\VS{11}素来认识{\PN{扫罗}}的,看见他和先知一同受感说话,就彼此说:「{\PN{基士}}的儿子遇见什么了?{\PN{扫罗}}也列在先知中吗?」
\VS{12}那地方有一个人说:「这些人的父亲是谁呢?」此后有句俗语说:「{\PN{扫罗}}也列在先知中吗?」
\VS{13}{\PN{扫罗}}受感说话已毕,就上邱坛去了。
\par }{\PP \VS{14}{\PN{扫罗}}的叔叔问{\PN{扫罗}}和他仆人说:「你们往哪里去了?」回答说:「找驴去了。我们见没有驴,就到了{\PN{撒母耳}}那里。」
\VS{15}{\PN{扫罗}}的叔叔说:「请将{\PN{撒母耳}}向你们所说的话告诉我。」
\VS{16}{\PN{扫罗}}对他叔叔说:「他明明地告诉我们驴已经找着了。」至于{\PN{撒母耳}}所说的国事,{\PN{扫罗}}却没有告诉叔叔。
\par }{\SH 抽签得扫罗为王
\par }{\PP \VS{17}{\PN{撒母耳}}将百姓招聚到{\PN{米斯巴}}耶和华那里,
\VS{18}对他们说:「耶和华—{\PN{以色列}}的 神如此说:『我领你们{\PN{以色列}}人出{\PN{埃及}},救你们脱离{\PN{埃及}}人的手,又救你们脱离欺压你们各国之人的手。』
\VS{19}你们今日却厌弃了救你们脱离一切灾难的 神,说:『求你立一个王治理我们。』现在你们应当按着支派、宗族都站在耶和华面前。」
\par }{\PP \VS{20}于是,{\PN{撒母耳}}使{\PN{以色列}}众支派近前来掣签,就掣出{\PN{便雅悯}}支派来;
\VS{21}又使{\PN{便雅悯}}支派按着宗族近前来,就掣出{\PN{玛特利}}族,从其中又掣出{\PN{基士}}的儿子{\PN{扫罗}}。众人寻找他却寻不着,
\VS{22}就问耶和华说:「那人到这里来了没有?」耶和华说:「他藏在器具中了。」
\VS{23}众人就跑去从那里领出他来。他站在百姓中间,身体比众民高过一头。
\VS{24}{\PN{撒母耳}}对众民说:「你们看耶和华所拣选的人,众民中有可比他的吗?」众民就大声欢呼说:「愿王万岁!」
\par }{\PP \VS{25}{\PN{撒母耳}}将国法对百姓说明,又记在书上,放在耶和华面前,然后遣散众民,各回各家去了。
\VS{26}{\PN{扫罗}}往{\PN{基比亚}}回家去,有 神感动的一群人跟随他。
\VS{27}但有些匪徒说:「这人怎能救我们呢?」就藐视他,没有送他礼物;{\PN{扫罗}}却不理会。

\par }\Chap{11}{\SH 扫罗击败亚扪人
\par }{\PP \VerseOne{1}{\PN{亚扪}}人{\ADD{的王}}{\PN{拿辖}}上来,对着{\PN{基列·雅比}}安营。{\PN{雅比}}众人对{\PN{拿辖}}说:「你与我们立约,我们就服事你。」
\VS{2}{\PN{亚扪}}人{\PN{拿辖}}说:「你们若由我剜出你们各人的右眼,以此凌辱{\PN{以色列}}众人,我就与你们立约。」
\VS{3}{\PN{雅比}}的长老对他说:「求你宽容我们七日,等我们打发人往{\PN{以色列}}的全境去;若没有人救我们,我们就出来归顺你。」
\VS{4}使者到了{\PN{扫罗}}住的{\PN{基比亚}},将这话说给百姓听,百姓就都放声而哭。
\par }{\PP \VS{5}{\PN{扫罗}}正从田间赶牛回来,问说:「百姓为什么哭呢?」众人将{\PN{雅比}}人的话告诉他。
\VS{6}{\PN{扫罗}}听见这话,就被 神的灵大大感动,甚是发怒。
\VS{7}他将一对牛切成块子,托付使者传送{\PN{以色列}}的全境,说:「凡不出来跟随{\PN{扫罗}}和{\PN{撒母耳}}的,也必这样切开他的牛。」于是耶和华使百姓惧怕,他们就都出来,如同一人。
\VS{8}{\PN{扫罗}}在{\PN{比色}}数点他们:{\PN{以色列}}人有三十万,{\PN{犹大}}人有三万。
\VS{9}众人对那使者说:「你们要回复{\PN{基列·雅比}}人说,明日太阳近午的时候,你们必得解救。」使者回去告诉{\PN{雅比}}人,他们就欢喜了。
\VS{10}于是{\PN{雅比}}人对{\PN{亚扪}}人说:「明日我们出来归顺你们,你们可以随意待我们。」
\VS{11}第二日,{\PN{扫罗}}将百姓分为三队,在晨更的时候入了{\PN{亚扪}}人的营,击杀他们直到太阳近午,剩下的人都逃散,没有二人同在一处的。
\par }{\PP \VS{12}百姓对{\PN{撒母耳}}说:「那说『{\PN{扫罗}}岂能管理我们』的是谁呢?可以将他交出来,我们好杀死他。」
\VS{13}{\PN{扫罗}}说:「今日耶和华在{\PN{以色列}}中施行拯救,所以不可杀人。」
\VS{14}{\PN{撒母耳}}对百姓说:「我们要往{\PN{吉甲}}去,在那里立国。」
\VS{15}众百姓就到了{\PN{吉甲}}那里,在耶和华面前立{\PN{扫罗}}为王,又在耶和华面前献平安祭。{\PN{扫罗}}和{\PN{以色列}}众人大大欢喜。

\par }\Chap{12}{\SH 撒母耳临别赠言
\par }{\PP \VerseOne{1}{\PN{撒母耳}}对{\PN{以色列}}众人说:「你们向我所求的,我已应允了,为你们立了一个王;
\VS{2}现在有这王在你们前面行。我已年老发白,我的儿子都在你们这里。我从幼年直到今日都在你们前面行。
\VS{3}我在这里,你们要在耶和华和他的受膏者面前给我作见证。我夺过谁的牛,抢过谁的驴,欺负过谁,虐待过谁,从谁手里受过贿赂因而眼瞎呢?若有,我必偿还。」
\VS{4}众人说:「你未曾欺负我们,虐待我们,也未曾从谁手里受过什么。」
\VS{5}{\PN{撒母耳}}对他们说:「你们在我手里没有找着什么,有耶和华和他的受膏者今日为证。」他们说:「愿他为证。」
\par }{\PP \VS{6}{\PN{撒母耳}}对百姓说:「从前立{\PN{摩西}}、{\PN{亚伦}},又领你们列祖出{\PN{埃及}}地的是耶和华。
\VS{7}现在你们要站住,等我在耶和华面前对你们讲论耶和华向你们和你们列祖所行一切公义的事。
\VS{8}从前{\PN{雅各}}到了{\PN{埃及}},后来你们列祖呼求耶和华,耶和华就差遣{\PN{摩西}}、{\PN{亚伦}}领你们列祖出{\PN{埃及}},使他们在这地方居住。
\VS{9}他们却忘记耶和华—他们的 神,他就把他们付与{\PN{夏琐}}将军{\PN{西西拉}}的手里,和{\PN{非利士}}人并{\PN{摩押}}王的手里。于是这些人常来攻击他们。
\VS{10}他们就呼求耶和华说:『我们离弃耶和华,事奉{\PN{巴力}}和{\PN{亚斯她录}},是有罪了。现在求你救我们脱离仇敌的手,我们必事奉你。』
\VS{11}耶和华就差遣{\PN{耶路·巴力}}、{\PN{比但}}、{\PN{耶弗他}}、{\PN{撒母耳}}救你们脱离四围仇敌的手,你们才安然居住。
\VS{12}你们见{\PN{亚扪}}人的王{\PN{拿辖}}来攻击你们,就对我说:『我们定要一个王治理我们。』其实耶和华—你们的 神是你们的王。
\VS{13}现在,你们所求所选的王在这里。看哪,耶和华已经为你们立王了。
\VS{14}你们若敬畏耶和华,事奉他,听从他的话,不违背他的命令,你们和治理你们的王也都顺从耶和华—你们的 神就好了。
\VS{15}倘若不听从耶和华的话,违背他的命令,耶和华的手必攻击你们,像从前攻击你们列祖一样。
\VS{16}现在你们要站住,看耶和华在你们眼前要行一件大事。
\VS{17}这不是割麦子的时候吗?我求告耶和华,他必打雷降雨,使你们又知道又看出,你们求立王的事是在耶和华面前犯大罪了。」
\VS{18}于是{\PN{撒母耳}}求告耶和华,耶和华就在这日打雷降雨,众民便甚惧怕耶和华和{\PN{撒母耳}}。
\par }{\PP \VS{19}众民对{\PN{撒母耳}}说:「求你为仆人们祷告耶和华—你的 神,免得我们死亡,因为我们求立王的事正是罪上加罪了。」
\VS{20}{\PN{撒母耳}}对百姓说:「不要惧怕!你们虽然行了这恶,却不要偏离耶和华,只要尽心事奉他。
\VS{21}若偏离{\ADD{耶和华}}去顺从那不能救人的虚神是无益的。
\VS{22}耶和华既喜悦选你们作他的子民,就必因他的大名不撇弃你们。
\VS{23}至于我,断不停止为你们祷告,以致得罪耶和华。我必以善道正路指教你们。
\VS{24}只要你们敬畏耶和华,诚诚实实地尽心事奉他,想念他向你们所行的事何等大。
\VS{25}你们若仍然作恶,你们和你们的王必一同灭亡。」

\par }\Chap{13}{\SH 与非利士人争战
\par }{\PP \VerseOne{1}{\PN{扫罗}}登基年{\ADD{四十岁}};作{\PN{以色列}}王二年的时候,
\VS{2}就从{\PN{以色列}}中拣选了三千人:二千跟随{\PN{扫罗}}在{\PN{密抹}}和{\PN{伯特利}}山,一千跟随{\PN{约拿单}}在{\PN{便雅悯}}的{\PN{基比亚}};其余的人{\PN{扫罗}}都打发各回各家去了。
\VS{3}{\PN{约拿单}}攻击在{\PN{迦巴}}的{\PN{非利士}}人的防营,{\PN{非利士}}人听见了。{\PN{扫罗}}就在遍地吹角,意思说,要使{\PN{希伯来}}人听见。
\VS{4}{\PN{以色列}}众人听见{\PN{扫罗}}攻击{\PN{非利士}}人的防营,又听见{\PN{以色列}}人为{\PN{非利士}}人所憎恶,就跟随{\PN{扫罗}}聚集在{\PN{吉甲}}。
\par }{\PP \VS{5}{\PN{非利士}}人聚集,要与{\PN{以色列}}人争战,有车三万辆,马兵六千,步兵像海边的沙那样多,就上来在{\PN{伯·亚文}}东边的{\PN{密抹}}安营。
\VS{6}{\PN{以色列}}百姓见自己危急窘迫,就藏在山洞、丛林、石穴、隐密处,和坑中。
\VS{7}有些{\PN{希伯来}}人过了{\PN{约旦河}},逃到{\PN{迦得}}和{\PN{基列}}地。{\PN{扫罗}}还是在{\PN{吉甲}},百姓都战战兢兢地跟随他。
\par }{\PP \VS{8}{\PN{扫罗}}照着{\PN{撒母耳}}所定的日期等了七日。{\PN{撒母耳}}还没有来到{\PN{吉甲}},百姓也离开{\PN{扫罗}}散去了。
\VS{9}{\PN{扫罗}}说:「把燔祭和平安祭带到我这里来。」{\PN{扫罗}}就献上燔祭。
\VS{10}刚献完燔祭,{\PN{撒母耳}}就到了。{\PN{扫罗}}出去迎接他,要问他好。
\VS{11}{\PN{撒母耳}}说:「你做的是什么事呢?」{\PN{扫罗}}说:「因为我见百姓离开我散去,你也不照所定的日期来到,而且{\PN{非利士}}人聚集在{\PN{密抹}}。
\VS{12}所以我心里说:恐怕我没有祷告耶和华。{\PN{非利士}}人下到{\PN{吉甲}}攻击我,我就勉强献上燔祭。」
\VS{13}{\PN{撒母耳}}对{\PN{扫罗}}说:「你做了糊涂事了,没有遵守耶和华—你 神所吩咐你的命令。若遵守,耶和华必在{\PN{以色列}}中坚立你的王位,直到永远。
\VS{14}现在你的王位必不长久。耶和华已经寻着一个合他心意的人,立他作百姓的君,因为你没有遵守耶和华所吩咐你的。」
\VS{15}{\PN{撒母耳}}就起来,从{\PN{吉甲}}上到{\PN{便雅悯}}的{\PN{基比亚}}。
{\PN{扫罗}}数点跟随他的,约有六百人。
\par }{\PP \VS{16}{\PN{扫罗}}和他儿子{\PN{约拿单}},并跟随他们的人,都住在{\PN{便雅悯}}的{\PN{迦巴}};但{\PN{非利士}}人安营在{\PN{密抹}}。
\VS{17}有掠兵从{\PN{非利士}}营中出来,分为三队:一队往{\PN{俄弗拉}}向{\PN{书亚}}地去,
\VS{18}一队往{\PN{伯·和
}}去,一队往{\PN{洗波音谷}}对面的地境向旷野去。
\par }{\PP \VS{19}那时,{\PN{以色列}}全地没有一个铁匠;因为{\PN{非利士}}人说,恐怕{\PN{希伯来}}人制造刀枪。
\VS{20}{\PN{以色列}}人要磨锄、犁、斧、铲,就下到{\PN{非利士}}人那里去磨。
\VS{21}但有锉可以锉铲、犁、三齿叉、斧子,并赶牛锥。
\VS{22}所以到了争战的日子,跟随{\PN{扫罗}}和{\PN{约拿单}}的人没有一个手里有刀有枪的,惟独{\PN{扫罗}}和他儿子{\PN{约拿单}}有。
\VS{23}{\PN{非利士}}人的一队防兵到了{\PN{密抹}}的隘口。

\par }\Chap{14}{\SH 约拿单的勇敢战绩
\par }{\PP \VerseOne{1}有一日,{\PN{扫罗}}的儿子{\PN{约拿单}}对拿他兵器的少年人说:「我们不如过到那边,到{\PN{非利士}}人的防营那里去。」但他没有告诉父亲。
\VS{2}{\PN{扫罗}}在{\PN{基比亚}}的尽边,坐在{\PN{米矶
}}的石榴树下,跟随他的约有六百人。
\VS{3}在那里有{\PN{亚希突}}的儿子{\PN{亚希亚}},穿着以弗得。({\PN{亚希突}}是{\PN{以迦博}}的哥哥,{\PN{非尼哈}}的儿子,{\PN{以利}}的孙子。{\PN{以利}}从前在{\PN{示罗}}作耶和华的祭司。){\PN{约拿单}}去了,百姓却不知道。
\VS{4}{\PN{约拿单}}要从隘口过到{\PN{非利士}}防营那里去。这隘口两边各有一个山峰:一名{\PN{播薛}},一名{\PN{西尼}};
\VS{5}一峰向北,与{\PN{密抹}}相对,一峰向南,与{\PN{迦巴}}相对。
\par }{\PP \VS{6}{\PN{约拿单}}对拿兵器的少年人说:「我们不如过到未受割礼人的防营那里去,或者耶和华为我们施展能力;因为耶和华使人得胜,不在乎人多人少。」
\VS{7}拿兵器的对他说:「随你的心意行吧。你可以上去,我必跟随你,与你同心。」
\VS{8}{\PN{约拿单}}说:「我们要过到那些人那里去,使他们看见我们。
\VS{9}他们若对我们说:『你们站住,等我们到你们那里去』,我们就站住,不上他们那里去。
\VS{10}他们若说:『你们上到我们这里来』,这话就是我们的证据;我们便上去,因为耶和华将他们交在我们手里了。」
\VS{11}二人就使{\PN{非利士}}的防兵看见。{\PN{非利士}}人说:「{\PN{希伯来}}人从所藏的洞穴里出来了!」
\VS{12}防兵对{\PN{约拿单}}和拿兵器的人说:「你们上到这里来,我们有一件事指示你们。」{\PN{约拿单}}就对拿兵器的人说:「你跟随我上去,因为耶和华将他们交在{\PN{以色列}}人手里了。」
\VS{13}{\PN{约拿单}}就爬上去,拿兵器的人跟随他。{\PN{约拿单}}杀倒{\PN{非利士}}人,拿兵器的人也随着杀他们。
\VS{14}{\PN{约拿单}}和拿兵器的人起头所杀的约有二十人,都在一亩地的半犁沟之内。
\VS{15}于是在营中、在田野、在众民内都有战兢,防兵和掠兵也都战兢,地也震动,战兢之势甚大。
\par }{\SH 非利士人败退
\par }{\PP \VS{16}在{\PN{便雅悯}}的{\PN{基比亚}},{\PN{扫罗}}的守望兵看见{\PN{非利士}}的军众溃散,四围乱窜。
\VS{17}{\PN{扫罗}}就对跟随他的民说:「你们查点查点,看从我们这里出去的是谁?」他们一查点,就知道{\PN{约拿单}}和拿兵器的人没有在这里。
\VS{18}那时 神的{\ADD{约}}柜在{\PN{以色列}}人那里。{\PN{扫罗}}对{\PN{亚希亚}}说:「你将 神的{\ADD{约}}柜运了来。」
\par }{\PP \VS{19}{\PN{扫罗}}正与祭司说话的时候,{\PN{非利士}}营中的喧嚷越发大了;{\PN{扫罗}}就对祭司说:「停手吧!」
\VS{20}{\PN{扫罗}}和跟随他的人都聚集,来到战场,看见{\PN{非利士}}人用刀互相击杀,大大惶乱。
\VS{21}从前由四方来跟随{\PN{非利士}}军的{\PN{希伯来}}人现在也转过来,帮助跟随{\PN{扫罗}}和{\PN{约拿单}}的{\PN{以色列}}人了。
\VS{22}那藏在{\PN{以法莲}}山地的{\PN{以色列}}人听说{\PN{非利士}}人逃跑,就出来紧紧地追杀他们。
\VS{23}那日,耶和华使{\PN{以色列}}人得胜,一直战到{\PN{伯·亚文}}。
\par }{\SH 战后发生的事件
\par }{\PP \VS{24}{\PN{扫罗}}叫百姓起誓说,凡不等到晚上向敌人报完了仇吃什么的,必受咒诅。因此这日百姓没有吃什么,就极其困惫。
\VS{25}众民进入树林,见有蜜在地上。
\VS{26}他们进了树林,见有蜜流下来,却没有人敢用手{\ADD{取蜜}}入口,因为他们怕那誓言。
\VS{27}{\PN{约拿单}}没有听见他父亲叫百姓起誓,所以伸手中的杖,用杖头蘸在蜂房里,转手送入口内,眼睛就明亮了。
\VS{28}百姓中有一人对他说:「你父亲曾叫百姓严严地起誓说,今日吃什么的,必受咒诅;因此百姓就疲乏了。」
\VS{29}{\PN{约拿单}}说:「我父亲连累你们了。你看,我尝了这一点蜜,眼睛就明亮了。
\VS{30}今日百姓若任意吃了从仇敌所夺的物,击杀的{\PN{非利士}}人岂不更多吗?」
\par }{\PP \VS{31}这日,{\PN{以色列}}人击杀{\PN{非利士}}人,从{\PN{密抹}}直到{\PN{亚雅
}}。百姓甚是疲乏,
\VS{32}就急忙将所夺的牛羊和牛犊宰于地上,肉还带血就吃了。
\VS{33}有人告诉{\PN{扫罗}}说:「百姓吃带血的肉,得罪耶和华了。」{\PN{扫罗}}说:「你们有罪了,今日要将大石头滚到我这里来。」
\VS{34}{\PN{扫罗}}又说:「你们散在百姓中,对他们说,你们各人将牛羊牵到我这里来宰了吃,不可吃带血的肉得罪耶和华。」这夜,百姓就把牛羊牵到那里宰了。
\VS{35}{\PN{扫罗}}为耶和华筑了一座坛,这是他初次为耶和华筑的坛。
\par }{\PP \VS{36}{\PN{扫罗}}说:「我们不如夜里下去追赶{\PN{非利士}}人,抢掠他们,直到天亮,不留他们一人。」众民说:「你看怎样好就去行吧!」祭司说:「我们先当亲近 神。」
\VS{37}{\PN{扫罗}}求问 神说:「我下去追赶{\PN{非利士}}人可以不可以?你将他们交在{\PN{以色列}}人手里不交?」这日 神没有回答他。
\VS{38}{\PN{扫罗}}说:「你们百姓中的长老都上这里来,查明今日是谁犯了罪。
\VS{39}我指着救{\PN{以色列}}—永生的耶和华起誓,就是我儿子{\PN{约拿单}}犯了罪,他也必死。」但百姓中无一人回答他。
\VS{40}{\PN{扫罗}}就对{\PN{以色列}}众人说:「你们站在一边,我与我儿子{\PN{约拿单}}也站在一边。」百姓对{\PN{扫罗}}说:「你看怎样好就去行吧!」
\VS{41}{\PN{扫罗}}祷告耶和华—{\PN{以色列}}的 神说:「求你指示实情。」于是{\ADD{掣签}}掣出{\PN{扫罗}}和{\PN{约拿单}}来;百姓尽都无事。
\VS{42}{\PN{扫罗}}说:「你们再{\ADD{掣签}},看是我,是我儿子{\PN{约拿单}}」,就掣出{\PN{约拿单}}来。
\par }{\PP \VS{43}{\PN{扫罗}}对{\PN{约拿单}}说:「你告诉我,你做了什么事?」{\PN{约拿单}}说:「我实在以手里的杖,用杖头蘸了一点蜜尝了一尝。这样我就死吗\FTNT{}{{\FR 14:43: }或译:吧!}?」
\VS{44}{\PN{扫罗}}说:「{\PN{约拿单}}哪,你定要死!若不然,愿 神重重地降罚与我。」
\VS{45}百姓对{\PN{扫罗}}说:「{\PN{约拿单}}在{\PN{以色列}}人中这样大行拯救,岂可使他死呢?断乎不可!我们指着永生的耶和华起誓,连他的一根头发也不可落地,因为他今日与 神一同做事。」于是百姓救{\PN{约拿单}}免了死亡。
\VS{46}{\PN{扫罗}}回去,不追赶{\PN{非利士}}人;{\PN{非利士}}人也回本地去了。
\par }{\SH 扫罗的王权和家眷
\par }{\PP \VS{47}{\PN{扫罗}}执掌{\PN{以色列}}的国权,常常攻击他四围的一切仇敌,就是{\PN{摩押}}人、{\PN{亚扪}}人、{\PN{以东}}人,和{\PN{琐巴}}诸王,并{\PN{非利士}}人。他无论往何处去,都打败仇敌。
\VS{48}{\PN{扫罗}}奋勇攻击{\PN{亚玛力}}人,救了{\PN{以色列}}人脱离抢掠他们之人的手。
\par }{\PP \VS{49}{\PN{扫罗}}的儿子是{\PN{约拿单}}、{\PN{亦施韦}}、{\PN{麦基舒亚}}。他的两个女儿:长女名{\PN{米拉}},次女名{\PN{米甲}}。
\VS{50}{\PN{扫罗}}的妻名叫{\PN{亚希暖}},是{\PN{亚希玛斯}}的女儿。{\PN{扫罗}}的元帅名叫{\PN{押尼珥}},是{\PN{尼珥}}的儿子;{\PN{尼珥}}是{\PN{扫罗}}的叔叔。
\VS{51}{\PN{扫罗}}的父亲{\PN{基士}},{\PN{押尼珥}}的父亲{\PN{尼珥}},都是{\PN{亚别}}的儿子。
\par }{\PP \VS{52}{\PN{扫罗}}平生常与{\PN{非利士}}人大大争战。{\PN{扫罗}}遇见有能力的人或勇士,都招募了来跟随他。

\par }\Chap{15}{\SH 与亚玛力人争战
\par }{\PP \VerseOne{1}{\PN{撒母耳}}对{\PN{扫罗}}说:「耶和华差遣我膏你为王,治理他的百姓{\PN{以色列}};所以你当听从耶和华的话。
\VS{2}万军之耶和华如此说:『{\PN{以色列}}人出{\PN{埃及}}的时候,在路上{\PN{亚玛力}}人怎样待他们,怎样抵挡他们,我都没忘。
\VS{3}现在你要去击打{\PN{亚玛力}}人,灭尽他们所有的,不可怜惜他们,将男女、孩童、吃奶的,并牛、羊、骆驼,和驴尽行杀死。』」
\par }{\PP \VS{4}于是{\PN{扫罗}}招聚百姓在{\PN{提拉因}},数点他们,共有步兵二十万,另有{\PN{犹大}}人一万。
\VS{5}{\PN{扫罗}}到了{\PN{亚玛力}}的{\ADD{京}}城,在谷中设下埋伏。
\VS{6}{\PN{扫罗}}对{\PN{基尼}}人说:「你们离开{\PN{亚玛力}}人下去吧,恐怕我将你们和{\PN{亚玛力}}人一同杀灭;因为{\PN{以色列}}人出{\PN{埃及}}的时候,你们曾恩待他们。」于是{\PN{基尼}}人离开{\PN{亚玛力}}人去了。
\VS{7}{\PN{扫罗}}击打{\PN{亚玛力}}人,从{\PN{哈腓拉}}直到{\PN{埃及}}前的{\PN{书珥}},
\VS{8}生擒了{\PN{亚玛力}}王{\PN{亚甲}},用刀杀尽{\PN{亚玛力}}的众民。
\VS{9}{\PN{扫罗}}和百姓却怜惜{\PN{亚甲}},也爱惜上好的牛、羊、牛犊、羊羔,并一切美物,不肯灭绝。凡下贱瘦弱的,尽都杀了。
\par }{\SH 扫罗被废
\par }{\PP \VS{10}耶和华的话临到{\PN{撒母耳}}说:
\VS{11}「我立{\PN{扫罗}}为王,我后悔了;因为他转去不跟从我,不遵守我的命令。」{\PN{撒母耳}}便甚忧愁,终夜哀求耶和华。
\VS{12}{\PN{撒母耳}}清早起来,迎接{\PN{扫罗}}。有人告诉{\PN{撒母耳}}说:「{\PN{扫罗}}到了{\PN{迦密}},在那里立了纪念碑,又转身下到{\PN{吉甲}}。」
\VS{13}{\PN{撒母耳}}到了{\PN{扫罗}}那里,{\PN{扫罗}}对他说:「愿耶和华赐福与你,耶和华的命令我已遵守了。」
\VS{14}{\PN{撒母耳}}说:「我耳中听见有羊叫、牛鸣,是从哪里来的呢?」
\VS{15}{\PN{扫罗}}说:「这是百姓从{\PN{亚玛力}}人那里带来的;因为他们爱惜上好的牛羊,要献与耶和华—你的 神;其余的,我们都灭尽了。」
\VS{16}{\PN{撒母耳}}对{\PN{扫罗}}说:「你住口吧!等我将耶和华昨夜向我所说的话告诉你。」{\PN{扫罗}}说:「请讲。」
\par }{\PP \VS{17}{\PN{撒母耳}}对{\PN{扫罗}}说:「从前你虽然以自己为小,岂不是被立为{\PN{以色列}}支派的元首吗?耶和华膏你作{\PN{以色列}}的王。
\VS{18}耶和华差遣你,吩咐你说,你去击打那些犯罪的{\PN{亚玛力}}人,将他们灭绝净尽。
\VS{19}你为何没有听从耶和华的命令,急忙掳掠财物,行耶和华眼中看为恶的事呢?」
\VS{20}{\PN{扫罗}}对{\PN{撒母耳}}说:「我实在听从了耶和华的命令,行了耶和华所差遣我行的路,擒了{\PN{亚玛力}}王{\PN{亚甲}}来,灭尽了{\PN{亚玛力}}人。
\VS{21}百姓却在所当灭的物中,取了最好的牛羊,要在{\PN{吉甲}}献与耶和华—你的 神。」
\VS{22}{\PN{撒母耳}}说:
\par }{\Q 耶和华喜悦燔祭和{\ADD{平安}}祭,
\par }{\Q 岂如喜悦人听从他的话呢?
\par }{\Q 听命胜于献祭;
\par }{\Q 顺从胜于公羊的脂油。
\par }{\Q \VS{23}悖逆的罪与行邪术的罪相等;
\par }{\Q 顽梗的罪与拜虚神和偶像的罪相同。
\par }{\Q 你既厌弃耶和华的命令,
\par }{\Q 耶和华也厌弃你作王。
\par }{\SH 扫罗认罪
\par }{\PP \VS{24}{\PN{扫罗}}对{\PN{撒母耳}}说:「我有罪了,我因惧怕百姓,听从他们的话,就违背了耶和华的命令和你的言语。
\VS{25}现在求你赦免我的罪,同我回去,我好敬拜耶和华。」
\VS{26}{\PN{撒母耳}}对{\PN{扫罗}}说:「我不同你回去;因为你厌弃耶和华的命令,耶和华也厌弃你作{\PN{以色列}}的王。」
\VS{27}{\PN{撒母耳}}转身要走,{\PN{扫罗}}就扯住他外袍的衣襟,衣襟就撕断了。
\VS{28}{\PN{撒母耳}}对他说:「如此,今日耶和华使{\PN{以色列}}国与你断绝,将这国赐与比你更好的人。
\VS{29}{\PN{以色列}}的大能者必不致说谎,也不致后悔;因为他迥非世人,决不后悔。」
\VS{30}{\PN{扫罗}}说:「我有罪了,虽然如此,求你在我百姓的长老和{\PN{以色列}}人面前抬举我,同我回去,我好敬拜耶和华—你的 神。」
\VS{31}于是{\PN{撒母耳}}转身跟随{\PN{扫罗}}回去,{\PN{扫罗}}就敬拜耶和华。
\par }{\PP \VS{32}{\PN{撒母耳}}说:「要把{\PN{亚玛力}}王{\PN{亚甲}}带到我这里来。」{\PN{亚甲}}就欢欢喜喜地来到他面前,心里说,死亡的苦难必定过去了。
\VS{33}{\PN{撒母耳}}说:「你既用刀使妇人丧子,这样,你母亲在妇人中也必丧子。」于是,{\PN{撒母耳}}在{\PN{吉甲}}耶和华面前将{\PN{亚甲}}杀死。
\par }{\PP \VS{34}{\PN{撒母耳}}回了{\PN{拉玛}}。{\PN{扫罗}}上他所住的{\PN{基比亚}},回自己的家去了。
\VS{35}{\PN{撒母耳}}直到死的日子,再没有见{\PN{扫罗}};但{\PN{撒母耳}}为{\PN{扫罗}}悲伤,是因耶和华后悔立他为{\PN{以色列}}的王。

\par }\Chap{16}{\SH 大卫被膏立为王
\par }{\PP \VerseOne{1}耶和华对{\PN{撒母耳}}说:「我既厌弃{\PN{扫罗}}作{\PN{以色列}}的王,你为他悲伤要到几时呢?你将膏油盛满了角,我差遣你往{\PN{伯利恒}}人{\PN{耶西}}那里去;因为我在他众子之内,预定一个作王的。」
\VS{2}{\PN{撒母耳}}说:「我怎能去呢?{\PN{扫罗}}若听见,必要杀我。」耶和华说:「你可以带一只牛犊去,就说:『我来是要向耶和华献祭。』
\VS{3}你要请{\PN{耶西}}来吃祭肉,我就指示你所当行的事。我所指给你的人,你要膏他。」
\VS{4}{\PN{撒母耳}}就照耶和华的话去行。到了{\PN{伯利恒}},那城里的长老都战战兢兢地出来迎接他,问他说:「你是为平安来的吗?」
\VS{5}他说:「为平安来的,我是给耶和华献祭。你们当自洁,来与我同吃祭肉。」{\PN{撒母耳}}就使{\PN{耶西}}和他众子自洁,请他们来吃祭肉。
\par }{\PP \VS{6}他们来的时候,{\PN{撒母耳}}看见{\PN{以}}
{\PN{利押}},就心里说,耶和华的受膏者必定在他面前。
\VS{7}耶和华却对{\PN{撒母耳}}说:「不要看他的外貌和他身材高大,我不拣选他。因为,{\ADD{耶和华}}不像人看人:人是看外貌;耶和华是看内心。」
\VS{8}{\PN{耶西}}叫{\PN{亚比拿达}}从{\PN{撒母耳}}面前经过,{\PN{撒母耳}}说:「耶和华也不拣选他。」
\VS{9}{\PN{耶西}}又叫{\PN{沙玛}}从{\PN{撒母耳}}面前经过,{\PN{撒母耳}}说:「耶和华也不拣选他。」
\VS{10}{\PN{耶西}}叫他七个儿子都从{\PN{撒母耳}}面前经过,{\PN{撒母耳}}说:「这都不是耶和华所拣选的。」
\VS{11}{\PN{撒母耳}}对{\PN{耶西}}说:「你的儿子都在这里吗?」他回答说:「还有个小的,现在放羊。」{\PN{撒母耳}}对{\PN{耶西}}说:「你打发人去叫他来;他若不来,我们必不坐席。」
\VS{12}{\PN{耶西}}就打发人去叫了他来。他面色光红,双目清秀,容貌俊美。耶和华说:「这就是他,你起来膏他。」
\VS{13}{\PN{撒母耳}}就用角里的膏油,在他诸兄中膏了他。从这日起,耶和华的灵就大大感动{\PN{大卫}}。{\PN{撒母耳}}起身回{\PN{拉玛}}去了。
\par }{\SH 大卫在扫罗的宫里
\par }{\PP \VS{14}耶和华的灵离开{\PN{扫罗}},有恶魔从耶和华那里来扰乱他。
\VS{15}{\PN{扫罗}}的臣仆对他说:「现在有恶魔从 神那里来扰乱你。
\VS{16}我们的主可以吩咐面前的臣仆,找一个善于弹琴的来,等 神那里来的恶魔临到你身上的时候,使他用手弹琴,你就好了。」
\VS{17}{\PN{扫罗}}对臣仆说:「你们可以为我找一个善于弹琴的,带到我这里来。」
\VS{18}其中有一个少年人说:「我曾见{\PN{伯利恒}}人{\PN{耶西}}的一个儿子善于弹琴,是大有勇敢的战士,说话合宜,容貌俊美,耶和华也与他同在。」
\VS{19}于是{\PN{扫罗}}差遣使者去见{\PN{耶西}},说:「请你打发你放羊的儿子{\PN{大卫}}到我这里来。」
\VS{20}{\PN{耶西}}就把几个饼和一皮袋酒,并一只山羊羔,都驮在驴上,交给他儿子{\PN{大卫}},送与{\PN{扫罗}}。
\VS{21}{\PN{大卫}}到了{\PN{扫罗}}那里,就侍立在{\PN{扫罗}}面前。{\PN{扫罗}}甚喜爱他,他就作了{\PN{扫罗}}拿兵器的人。
\VS{22}{\PN{扫罗}}差遣人去见{\PN{耶西}},说:「求你容{\PN{大卫}}侍立在我面前,因为他在我眼前蒙了恩。」
\VS{23}从 神那里来的恶魔临到{\PN{扫罗}}身上的时候,{\PN{大卫}}就拿琴,用手而弹,{\PN{扫罗}}便舒畅爽快,恶魔离了他。

\par }\Chap{17}{\SH 歌利亚向以色列军挑战
\par }{\PP \VerseOne{1}{\PN{非利士}}人招聚他们的军旅,要来争战;聚集在属{\PN{犹大}}的{\PN{梭哥}},安营在{\PN{梭哥}}和{\PN{亚西加}}中间的{\PN{以弗·大悯}}。
\VS{2}{\PN{扫罗}}和{\PN{以色列}}人也聚集,在{\PN{以拉谷}}安营,摆列队伍,要与{\PN{非利士}}人打仗。
\VS{3}{\PN{非利士}}人站在这边山上,{\PN{以色列}}人站在那边山上,当中有谷。
\VS{4}从{\PN{非利士}}营中出来一个讨战的人,名叫{\PN{歌利亚}},是{\PN{迦特}}人,身高六肘零一虎口;
\VS{5}头戴铜盔,身穿铠甲,甲重五千舍客勒;
\VS{6}腿上有铜护膝,两肩之中背负铜戟;
\VS{7}枪杆粗如织布的机轴,铁枪头重六百舍客勒。有一个拿盾牌的人在他前面走。
\VS{8}{\PN{歌利亚}}对着{\PN{以色列}}的军队站立,呼叫说:「你们出来摆列队伍做什么呢?我不是{\PN{非利士}}人吗?你们不是{\PN{扫罗}}的仆人吗?可以从你们中间拣选一人,使他下到我这里来。
\VS{9}他若能与我战斗,将我杀死,我们就作你们的仆人;我若胜了他,将他杀死,你们就作我们的仆人,服事我们。」
\VS{10}那{\PN{非利士}}人又说:「我今日向{\PN{以色列}}人的军队骂阵。你们叫一个人出来,与我战斗。」
\VS{11}{\PN{扫罗}}和{\PN{以色列}}众人听见{\PN{非利士}}人的这些话,就惊惶,极其害怕。
\par }{\SH 大卫在扫罗的营中
\par }{\PP \VS{12}{\PN{大卫}}是{\PN{犹大}}{\PN{伯利恒}}的{\PN{以法他}}人{\PN{耶西}}的儿子。{\PN{耶西}}有八个儿子。当{\PN{扫罗}}的时候,{\PN{耶西}}已经老迈。
\VS{13}{\PN{耶西}}的三个大儿子跟随{\PN{扫罗}}出征。这出征的三个儿子:长子名叫{\PN{以利押}},次子名叫{\PN{亚比拿达}},三子名叫{\PN{沙玛}}。
\VS{14}{\PN{大卫}}是最小的;那三个大儿子跟随{\PN{扫罗}}。
\VS{15}{\PN{大卫}}有时离开{\PN{扫罗}},回{\PN{伯利恒}}放他父亲的羊。
\VS{16}那{\PN{非利士}}人早晚都出来站着,如此四十日。
\par }{\PP \VS{17}一日,{\PN{耶西}}对他儿子{\PN{大卫}}说:「你拿一伊法烘了的穗子和十个饼,速速地送到营里去,交给你哥哥们;
\VS{18}再拿这十块奶饼,送给他们的千夫长,且问你哥哥们好,向他们要一封信来。」
\VS{19}{\PN{扫罗}}与{\PN{大卫}}的三个哥哥和{\PN{以色列}}众人,在{\PN{以拉谷}}与{\PN{非利士}}人打仗。
\VS{20}{\PN{大卫}}早晨起来,将羊交托一个看守的人,照着他父亲所吩咐的话,带着食物去了。到了辎重营,军兵刚出到战场,呐喊要战。
\VS{21}{\PN{以色列}}人和{\PN{非利士}}人都摆列队伍,彼此相对。
\VS{22}{\PN{大卫}}把他带来的食物留在看守物件人的手下,跑到战场,问他哥哥们安。
\VS{23}与他们说话的时候,那讨战的,就是属{\PN{迦特}}的{\PN{非利士}}人{\PN{歌利亚}},从{\PN{非利士}}队中出来,说从前所说的话;{\PN{大卫}}都听见了。
\par }{\PP \VS{24}{\PN{以色列}}众人看见那人,就逃跑,极其害怕。
\VS{25}{\PN{以色列}}人彼此说:「这上来的人你看见了吗?他上来是要向{\PN{以色列}}人骂阵。若有能杀他的,王必赏赐他大财,将自己的女儿给他为妻,并在{\PN{以色列}}人中免他父家纳粮当差。」
\VS{26}{\PN{大卫}}问站在旁边的人说:「有人杀这{\PN{非利士}}人,除掉{\PN{以色列}}人的耻辱,怎样待他呢?这未受割礼的{\PN{非利士}}人是谁呢?竟敢向永生 神的军队骂阵吗?」
\VS{27}百姓照先前的话回答他说:「有人能杀这{\PN{非利士}}人,必如此如此待他。」
\par }{\PP \VS{28}{\PN{大卫}}的长兄{\PN{以利押}}听见{\PN{大卫}}与他们所说的话,就向他发怒,说:「你下来做什么呢?在旷野的那几只羊,你交托了谁呢?我知道你的骄傲和你心里的恶意,你下来特为要看争战!」
\VS{29}{\PN{大卫}}说:「我做了什么呢?我来岂没有缘故吗?」
\VS{30}{\PN{大卫}}就离开他转向别人,照先前的话而问;百姓仍照先前的话回答他。
\par }{\PP \VS{31}有人听见{\PN{大卫}}所说的话,就告诉了{\PN{扫罗}};{\PN{扫罗}}便打发人叫他来。
\VS{32}{\PN{大卫}}对{\PN{扫罗}}说:「人都不必因那{\PN{非利士}}人胆怯。你的仆人要去与那{\PN{非利士}}人战斗。」
\VS{33}{\PN{扫罗}}对{\PN{大卫}}说:「你不能去与那{\PN{非利士}}人战斗;因为你年纪太轻,他自幼就作战士。」
\VS{34}{\PN{大卫}}对{\PN{扫罗}}说:「你仆人为父亲放羊,有时来了狮子,有时来了熊,从群中衔一只羊羔去。
\VS{35}我就追赶它,击打它,将羊羔从它口中救出来。它起来要害我,我就揪着它的胡子,将它打死。
\VS{36}你仆人曾打死狮子和熊,这未受割礼的{\PN{非利士}}人向永生 神的军队骂阵,也必像狮子和熊一般。」
\VS{37}{\PN{大卫}}又说:「耶和华救我脱离狮子和熊的爪,也必救我脱离这{\PN{非利士}}人的手。」{\PN{扫罗}}对{\PN{大卫}}说:「你可以去吧!耶和华必与你同在。」
\VS{38}{\PN{扫罗}}就把自己的战衣给{\PN{大卫}}穿上,将铜盔给他戴上,又给他穿上铠甲。
\VS{39}{\PN{大卫}}把刀跨在战衣外,试试能走不能走;因为素来没有穿惯,就对{\PN{扫罗}}说:「我穿戴这些不能走,因为素来没有穿惯。」于是摘脱了。
\VS{40}他手中拿杖,又在溪中挑选了五块光滑石子,放在袋里,就是牧人带的囊里;手中拿着甩石的机弦,就去迎那{\PN{非利士}}人。
\par }{\SH 大卫击杀歌利亚
\par }{\PP \VS{41}{\PN{非利士}}人也渐渐地迎着{\PN{大卫}}来,拿盾牌的走在前头。
\VS{42}{\PN{非利士}}人观看,见了{\PN{大卫}},就藐视他;因为他年轻,面色光红,容貌俊美。
\VS{43}{\PN{非利士}}人对{\PN{大卫}}说:「你拿杖到我这里来,我岂是狗呢?」{\PN{非利士}}人就指着自己的神咒诅{\PN{大卫}}。
\VS{44}{\PN{非利士}}人又对{\PN{大卫}}说:「来吧!我将你的肉给空中的飞鸟、田野的走兽吃。」
\VS{45}{\PN{大卫}}对{\PN{非利士}}人说:「你来攻击我,是靠着刀枪和{\ADD{铜}}戟;我来攻击你,是靠着万军之耶和华的名,就是你所怒骂带领{\PN{以色列}}军队的 神。
\VS{46}今日耶和华必将你交在我手里。我必杀你,斩你的头,又将{\PN{非利士}}军兵的尸首给空中的飞鸟、地上的野兽吃,使普天下的人都知道{\PN{以色列}}中有 神;
\VS{47}又使这众人知道耶和华使人得胜,不是用刀用枪,因为争战的胜败全在乎耶和华。他必将你们交在我们手里。」
\par }{\PP \VS{48}{\PN{非利士}}人起身,迎着{\PN{大卫}}前来。{\PN{大卫}}急忙迎着{\PN{非利士}}人,往战场跑去。
\VS{49}{\PN{大卫}}用手从囊中掏出一块石子来,用机弦甩去,打中{\PN{非利士}}人的额,石子进入额内,他就仆倒,面伏于地。
\par }{\PP \VS{50}这样,{\PN{大卫}}用机弦甩石,胜了那{\PN{非利士}}人,打死他;{\PN{大卫}}手中却没有刀。
\VS{51}{\PN{大卫}}跑去,站在{\PN{非利士}}人身旁,将他的刀从鞘中拔出来,杀死他,割了他的头。{\PN{非利士}}众人看见他们讨战的勇士死了,就都逃跑。
\VS{52}{\PN{以色列}}人和{\PN{犹大}}人便起身呐喊,追赶{\PN{非利士}}人,直到{\PN{迦特}}\FTNT{}{{\FR 17:52: }或译:该}和{\PN{以革伦}}的城门。被杀的{\PN{非利士}}人倒在{\PN{沙拉音}}的路上,直到{\PN{迦特}}和{\PN{以革伦}}。
\VS{53}{\PN{以色列}}人追赶{\PN{非利士}}人回来,就夺了他们的营盘。
\VS{54}{\PN{大卫}}将那{\PN{非利士}}人的头拿到{\PN{耶路撒冷}},却将他军装放在自己的帐棚里。
\par }{\SH 大卫觐见扫罗
\par }{\PP \VS{55}{\PN{扫罗}}看见{\PN{大卫}}去攻击{\PN{非利士}}人,就问元帅{\PN{押尼珥}}说:「{\PN{押尼珥}}啊,那少年人是谁的儿子?」{\PN{押尼珥}}说:「我敢在王面前起誓,我不知道。」
\VS{56}王说:「你可以问问那幼年人是谁的儿子。」
\VS{57}{\PN{大卫}}打死{\PN{非利士}}人回来,{\PN{押尼珥}}领他到{\PN{扫罗}}面前,他手中拿着{\PN{非利士}}人的头。
\VS{58}{\PN{扫罗}}问他说:「少年人哪,你是谁的儿子?」{\PN{大卫}}说:「我是你仆人{\PN{伯利恒}}人{\PN{耶西}}的儿子。」

\par }\Chap{18}{\PP \VerseOne{1}{\PN{大卫}}对{\PN{扫罗}}说完了话,{\PN{约拿单}}的心与{\PN{大卫}}的心深相契合。{\PN{约拿单}}爱{\PN{大卫}},如同爱自己的性命。
\VS{2}那日{\PN{扫罗}}留住{\PN{大卫}},不容他再回父家。
\VS{3}{\PN{约拿单}}爱{\PN{大卫}}如同爱自己的性命,就与他结盟。
\VS{4}{\PN{约拿单}}从身上脱下外袍,给了{\PN{大卫}},又将战衣、刀、弓、腰带都给了他。
\VS{5}{\PN{扫罗}}无论差遣{\PN{大卫}}往何处去,他都做事精明。{\PN{扫罗}}就立他作战士长,众百姓和{\PN{扫罗}}的臣仆无不喜悦。
\par }{\SH 扫罗妒忌大卫
\par }{\PP \VS{6}{\PN{大卫}}打死了那{\PN{非利士}}人,同众人回来的时候,妇女们从{\PN{以色列}}各城里出来,欢欢喜喜,打鼓击磬,歌唱跳舞,迎接{\PN{扫罗}}王。
\VS{7}众妇女舞蹈唱和,说:「{\PN{扫罗}}杀死千千,{\PN{大卫}}杀死万万。」
\VS{8}{\PN{扫罗}}甚发怒,不喜悦这话,就说:「将万万归{\PN{大卫}},千千归我,只剩下王位没有给他了。」
\VS{9}从这日起,{\PN{扫罗}}就怒视{\PN{大卫}}。
\par }{\PP \VS{10}次日,从 神那里来的恶魔大大降在{\PN{扫罗}}身上,他就在家中胡言乱语。{\PN{大卫}}照常弹琴,{\PN{扫罗}}手里拿着枪。
\VS{11}{\PN{扫罗}}把枪一抡,心里说,我要将{\PN{大卫}}刺透,钉在墙上。{\PN{大卫}}躲避他两次。
\par }{\PP \VS{12}{\PN{扫罗}}惧怕{\PN{大卫}};因为耶和华离开自己,与{\PN{大卫}}同在。
\VS{13}所以{\PN{扫罗}}使{\PN{大卫}}离开自己,立他为千夫长,他就领兵出入。
\VS{14}{\PN{大卫}}做事无不精明,耶和华也与他同在。
\VS{15}{\PN{扫罗}}见{\PN{大卫}}做事精明,就甚怕他。
\VS{16}但{\PN{以色列}}和{\PN{犹大}}众人都爱{\PN{大卫}},因为他领他们出入。
\par }{\SH 大卫娶扫罗的女儿
\par }{\PP \VS{17}{\PN{扫罗}}对{\PN{大卫}}说:「我将大女儿{\PN{米拉}}给你为妻,只要你为我奋勇,为耶和华争战。」{\PN{扫罗}}心里说:「我不好亲手害他,要借{\PN{非利士}}人的手害他。」
\VS{18}{\PN{大卫}}对{\PN{扫罗}}说:「我是谁,我是什么出身,我父家在{\PN{以色列}}中是何等的家,岂敢作王的女婿呢?」
\VS{19}{\PN{扫罗}}的女儿{\PN{米拉}}到了当给{\PN{大卫}}的时候,{\PN{扫罗}}却给了{\PN{米何拉}}人{\PN{亚得列}}为妻。
\par }{\PP \VS{20}{\PN{扫罗}}的次女{\PN{米甲}}爱{\PN{大卫}}。有人告诉{\PN{扫罗}},{\PN{扫罗}}就喜悦。
\VS{21}{\PN{扫罗}}心里说:「我将这女儿给{\PN{大卫}},作他的网罗,好借{\PN{非利士}}人的手害他。」所以{\PN{扫罗}}对{\PN{大卫}}说:「你今日可以第二次作我的女婿。」
\VS{22}{\PN{扫罗}}吩咐臣仆{\ADD{说}}:「你们暗中对{\PN{大卫}}说:『王喜悦你,王的臣仆也都喜爱你,所以你当作王的女婿。』」
\VS{23}{\PN{扫罗}}的臣仆就照这话说给{\PN{大卫}}听。{\PN{大卫}}说:「你们以为作王的女婿是一件小事吗?我是贫穷卑微的人。」
\VS{24}{\PN{扫罗}}的臣仆回奏说,{\PN{大卫}}所说的如此如此。
\VS{25}{\PN{扫罗}}说:「你们要对{\PN{大卫}}这样说:『王不要什么聘礼,只要一百{\PN{非利士}}人的阳皮,好在王的仇敌身上报仇。』」{\PN{扫罗}}的意思要使{\PN{大卫}}丧在{\PN{非利士}}人的手里。
\VS{26}{\PN{扫罗}}的臣仆将这话告诉{\PN{大卫}},{\PN{大卫}}就欢喜作王的女婿。日期还没有到,
\VS{27}{\PN{大卫}}和跟随他的人起身前往,杀了二百{\PN{非利士}}人,将阳皮满数交给王,为要作王的女婿。于是{\PN{扫罗}}将女儿{\PN{米甲}}给{\PN{大卫}}为妻。
\VS{28}{\PN{扫罗}}见耶和华与{\PN{大卫}}同在,又知道女儿{\PN{米甲}}爱{\PN{大卫}},
\VS{29}就更怕{\PN{大卫}},常作{\PN{大卫}}的仇敌。
\par }{\PP \VS{30}每逢{\PN{非利士}}军长出来{\ADD{打仗}},{\PN{大卫}}比{\PN{扫罗}}的臣仆做事精明,因此他的名被人尊重。

\par }\Chap{19}{\SH 扫罗逼迫大卫
\par }{\PP \VerseOne{1}{\PN{扫罗}}对他儿子{\PN{约拿单}}和众臣仆说,要杀{\PN{大卫}};{\PN{扫罗}}的儿子{\PN{约拿单}}却甚喜爱{\PN{大卫}}。
\VS{2}{\PN{约拿单}}告诉{\PN{大卫}}说:「我父{\PN{扫罗}}想要杀你,所以明日早晨你要小心,到一个僻静地方藏身。
\VS{3}我就出到你所藏的田里,站在我父亲旁边与他谈论。我看他情形怎样,我必告诉你。」
\VS{4}{\PN{约拿单}}向他父亲{\PN{扫罗}}替{\PN{大卫}}说好话,说:「王不可得罪王的仆人{\PN{大卫}};因为他未曾得罪你,他所行的都与你大有益处。
\VS{5}他拚命杀那{\PN{非利士}}人,耶和华为{\PN{以色列}}众人大行拯救;那时你看见,甚是欢喜,现在为何无故要杀{\PN{大卫}},流无辜人的血,自己取罪呢?」
\VS{6}{\PN{扫罗}}听了{\PN{约拿单}}的话,就指着永生的耶和华起誓说:「我必不杀他。」
\VS{7}{\PN{约拿单}}叫{\PN{大卫}}来,把这一切事告诉他,带他去见{\PN{扫罗}}。他就仍然侍立在{\PN{扫罗}}面前。
\par }{\PP \VS{8}此后又有争战的事。{\PN{大卫}}出去与{\PN{非利士}}人打仗,大大杀败他们,他们就在他面前逃跑。
\VS{9}从耶和华那里来的恶魔又降在{\PN{扫罗}}身上({\PN{扫罗}}手里拿枪坐在屋里),{\PN{大卫}}就用手弹琴。
\VS{10}{\PN{扫罗}}用枪想要刺透{\PN{大卫}},钉在墙上,他却躲开,{\PN{扫罗}}的枪刺入墙内。当夜{\PN{大卫}}逃走,躲避了。
\par }{\PP \VS{11}{\PN{扫罗}}打发人到{\PN{大卫}}的房屋那里窥探他,要等到天亮杀他。{\PN{大卫}}的妻{\PN{米甲}}对他说:「你今夜若不逃命,明日你要被杀。」
\VS{12}于是{\PN{米甲}}将{\PN{大卫}}从窗户里缒下去,{\PN{大卫}}就逃走,躲避了。
\VS{13}{\PN{米甲}}把家中的神像放在床上,头枕在山羊毛装的枕头上,用被遮盖。
\VS{14}{\PN{扫罗}}打发人去捉拿{\PN{大卫}},{\PN{米甲}}说:「他病了。」
\VS{15}{\PN{扫罗}}又打发人去看{\PN{大卫}},说:「当连床将他抬来,我好杀他。」
\VS{16}使者进去,看见床上有神像,头枕在山羊毛装的枕头上。
\VS{17}{\PN{扫罗}}对{\PN{米甲}}说:「你为什么这样欺哄我,放我仇敌逃走呢?」{\PN{米甲}}回答说:「他对我说:『你放我走,不然我要杀你。』」
\par }{\PP \VS{18}{\PN{大卫}}逃避,来到{\PN{拉玛}}见{\PN{撒母耳}},将{\PN{扫罗}}向他所行的事述说了一遍。他和{\PN{撒母耳}}就往{\PN{拿约}}去居住。
\VS{19}有人告诉{\PN{扫罗}},说{\PN{大卫}}在{\PN{拉玛}}的{\PN{拿约}}。
\VS{20}{\PN{扫罗}}打发人去捉拿{\PN{大卫}}。去的人见有一班先知都受感说话,{\PN{撒母耳}}站在其中监管他们;打发去的人也受 神的灵感动说话。
\VS{21}有人将这事告诉{\PN{扫罗}},他又打发人去,他们也受感说话。{\PN{扫罗}}第三次打发人去,他们也受感说话。
\VS{22}然后{\PN{扫罗}}自己往{\PN{拉玛}}去,到了{\PN{西沽}}的大井,问人说:「{\PN{撒母耳}}和{\PN{大卫}}在哪里呢?」有人说:「在{\PN{拉玛}}的{\PN{拿约}}。」
\VS{23}他就往{\PN{拉玛}}的{\PN{拿约}}去。 神的灵也感动他,一面走一面说话,直到{\PN{拉玛}}的{\PN{拿约}}。
\VS{24}他就脱了衣服,在{\PN{撒母耳}}面前受感说话,一昼一夜露体躺卧。因此有句俗语说:「{\PN{扫罗}}也列在先知中吗?」

\par }\Chap{20}{\SH 约拿单帮助大卫
\par }{\PP \VerseOne{1}{\PN{大卫}}从{\PN{拉玛}}的{\PN{拿约}}逃跑,来到{\PN{约拿单}}那里,对他说:「我做了什么?有什么罪孽呢?在你父亲面前犯了什么罪,他竟寻索我的性命呢?」
\VS{2}{\PN{约拿单}}回答说:「断然不是!你必不致死。我父做事,无论大小,没有不叫我知道的。怎么独有这事隐瞒我呢?决不如此。」
\VS{3}{\PN{大卫}}又起誓说:「你父亲准知我在你眼前蒙恩。他心里说,不如不叫{\PN{约拿单}}知道,恐怕他愁烦。我指着永生的耶和华,又敢在你面前起誓,我离死不过一步。」
\VS{4}{\PN{约拿单}}对{\PN{大卫}}说:「你心里所求的,我必为你成就。」
\VS{5}{\PN{大卫}}对{\PN{约拿单}}说:「明日是初一,我当与王同席,求你容我去藏在田野,直到第三日晚上。
\VS{6}你父亲若见我不在席上,你就说:『{\PN{大卫}}切求我许他回本城{\PN{伯利恒}}去,因为他全家在那里献年祭。』
\VS{7}你父亲若说好,仆人就平安了;他若发怒,你就知道他决意要害我。
\VS{8}求你施恩与仆人,因你在耶和华面前曾与仆人结盟。我若有罪,不如你自己杀我,何必将我交给你父亲呢?」
\VS{9}{\PN{约拿单}}说:「断无此事!我若知道我父亲决意害你,我岂不告诉你呢?」
\VS{10}{\PN{大卫}}对{\PN{约拿单}}说:「你父亲若用厉言回答你,谁来告诉我呢?」
\VS{11}{\PN{约拿单}}对{\PN{大卫}}说:「你我且往田野去。」二人就往田野去了。
\par }{\PP \VS{12}{\PN{约拿单}}对{\PN{大卫}}说:「愿耶和华—{\PN{以色列}}的 神{\ADD{为证}}。明日约在这时候,或第三日,我探我父亲的意思,若向你有好意,我岂不打发人告诉你吗?
\VS{13}我父亲若有意害你,我不告诉你使你平平安安地走,愿耶和华重重地降罚与我。愿耶和华与你同在,如同从前与我父亲同在一样。
\VS{14}你要照耶和华的慈爱恩待我,不但我活着的时候免我死亡,
\VS{15}就是我死后,耶和华从地上剪除你仇敌的时候,你也永不可向我家绝了恩惠。」
\VS{16}于是{\PN{约拿单}}与{\PN{大卫}}家结盟,{\ADD{说}}:「愿耶和华借{\PN{大卫}}的仇敌追讨{\ADD{背约}}的罪。」
\VS{17}{\PN{约拿单}}因爱{\PN{大卫}}如同爱自己的性命,就使他再起誓。
\par }{\PP \VS{18}{\PN{约拿单}}对他说:「明日是初一,你的座位空设,人必理会你不在那里。
\VS{19}你等三日,就要速速下去,到你从前遇事所藏的地方,在{\PN{以色}}磐石那里等候。
\VS{20}我要向磐石旁边射三箭,如同射箭靶一样。
\VS{21}我要打发童子,{\ADD{说}}:『去把箭找来。』我若对童子说:『箭在后头,把箭拿来』,你就可以回来;我指着永生的耶和华起誓,你必平安无事。
\VS{22}我若对童子说:『箭在前头』,你就要去,因为是耶和华打发你去的。
\VS{23}至于你我今日所说的话,有耶和华在你我中间{\ADD{为证}},直到永远。」
\par }{\PP \VS{24}{\PN{大卫}}就去藏在田野。到了初一日,王坐席要吃饭。
\VS{25}王照常坐在靠墙的位上,{\PN{约拿单}}侍立,{\PN{押尼珥}}坐在{\PN{扫罗}}旁边,{\PN{大卫}}的座位空设。
\par }{\PP \VS{26}然而这日{\PN{扫罗}}没有说什么,他想{\PN{大卫}}遇事,偶染不洁,他必定是不洁。
\VS{27}初二日{\PN{大卫}}的座位还空设。{\PN{扫罗}}问他儿子{\PN{约拿单}}说:「{\PN{耶西}}的儿子为何昨日、今日没有来吃饭呢?」
\VS{28}{\PN{约拿单}}回答{\PN{扫罗}}说:「{\PN{大卫}}切求我容他往{\PN{伯利恒}}去。
\VS{29}他说:『求你容我去,因为我家在城里有献祭的事;我长兄吩咐我去。如今我若在你眼前蒙恩,求你容我去见我的弟兄』;所以{\PN{大卫}}没有赴王的席。」
\par }{\PP \VS{30}{\PN{扫罗}}向{\PN{约拿单}}发怒,对他说:「你这顽梗背逆之妇人所生的,我岂不知道你喜悦{\PN{耶西}}的儿子,自取羞辱,以致你母亲露体蒙羞吗?
\VS{31}{\PN{耶西}}的儿子若在世间活着,你和你的国位必站立不住。现在你要打发人去,将他捉拿交给我;他是该死的。」
\VS{32}{\PN{约拿单}}对父亲{\PN{扫罗}}说:「他为什么该死呢?他做了什么呢?」
\par }{\PP \VS{33}{\PN{扫罗}}向{\PN{约拿单}}抡枪要刺他,{\PN{约拿单}}就知道他父亲决意要杀{\PN{大卫}}。
\VS{34}于是{\PN{约拿单}}气忿忿地从席上起来,在这初二日没有吃饭。他因见父亲羞辱{\PN{大卫}},就为{\PN{大卫}}愁烦。
\VS{35}次日早晨,{\PN{约拿单}}按着与{\PN{大卫}}约会的时候出到田野,有一个童子跟随。
\VS{36}{\PN{约拿单}}对童子说:「你跑去,把我所射的箭找来。」童子跑去,{\PN{约拿单}}就把箭射在童子前头。
\VS{37}童子到了{\PN{约拿单}}落箭之地,{\PN{约拿单}}呼叫童子说:「箭不是在你前头吗?」
\VS{38}{\PN{约拿单}}又呼叫童子说:「速速地去,不要迟延!」童子就拾起箭来,回到主人那里。
\VS{39}童子却不知道这是什么意思,只有{\PN{约拿单}}和{\PN{大卫}}知道。
\VS{40}{\PN{约拿单}}将弓箭交给童子,吩咐说:「你拿到城里去。」
\VS{41}童子一去,{\PN{大卫}}就从{\ADD{磐石}}的南边出来,俯伏在地,拜了三拜;二人亲嘴,彼此哭泣,{\PN{大卫}}哭得更恸。
\VS{42}{\PN{约拿单}}对{\PN{大卫}}说:「我们二人曾指着耶和华的名起誓说:『愿耶和华在你我中间,并你我后裔中间{\ADD{为证}},直到永远。』如今你平平安安地去吧!」{\PN{大卫}}就起身走了;{\PN{约拿单}}也回城里去了。

\par }\Chap{21}{\SH 大卫逃离扫罗
\par }{\PP \VerseOne{1}{\PN{大卫}}到了{\PN{挪伯}}祭司{\PN{亚希米勒}}那里,{\PN{亚希米勒}}战战兢兢地出来迎接他,问他说:「你为什么独自来,没有人跟随呢?」
\VS{2}{\PN{大卫}}回答祭司{\PN{亚希米勒}}说:「王吩咐我一件事说:『我差遣你委托你的这件事,不要使人知道。』故此我已派定少年人在某处等候我。
\VS{3}现在你手下有什么?求你给我五个饼或是别样的食物。」
\VS{4}祭司对{\PN{大卫}}说:「我手下没有寻常的饼,只有圣饼;若少年人没有亲近妇人才可以给。」
\VS{5}{\PN{大卫}}对祭司说:「实在约有三日我们没有亲近妇人;我出来的时候,虽是寻常行路,少年人的器皿还是洁净的;何况今日不更是洁净吗?」
\VS{6}祭司就拿圣{\ADD{饼}}给他;因为在那里没有别样饼,只有更换新饼,从耶和华面前撤下来的陈设饼。
\par }{\PP (
\VS{7}当日有{\PN{扫罗}}的一个臣子留在耶和华面前。他名叫{\PN{多益}},是{\PN{以东}}人,作{\PN{扫罗}}的司牧长。)
\par }{\PP \VS{8}{\PN{大卫}}问{\PN{亚希米勒}}说:「你手下有枪有刀没有?因为王的事甚急,连刀剑器械我都没有带。」
\VS{9}祭司说:「你在{\PN{以拉谷}}杀{\PN{非利士}}人{\PN{歌利亚}}的那刀在这里,裹在布中,放在以弗得后边,你要就可以拿去;除此以外,再没有别的。」{\PN{大卫}}说:「这刀没有可比的!求你给我。」
\par }{\PP \VS{10}那日{\PN{大卫}}起来,躲避{\PN{扫罗}},逃到{\PN{迦特}}王{\PN{亚吉}}那里。
\VS{11}{\PN{亚吉}}的臣仆对{\PN{亚吉}}说:「这不是{\ADD{
{\PN{以色列}}}}国王{\PN{大卫}}吗?那里的妇女跳舞唱和,不是指着他说『{\PN{扫罗}}杀死千千,{\PN{大卫}}杀死万万』吗?」
\VS{12}{\PN{大卫}}将这话放在心里,甚惧怕{\PN{迦特}}王{\PN{亚吉}},
\VS{13}就在众人面前改变了寻常的举动,在他们手下假装疯癫,在城门的门扇上胡写乱画,使唾沫流在胡子上。
\VS{14}{\PN{亚吉}}对臣仆说:「你们看,这人是疯子。为什么带他到我这里来呢?
\VS{15}我岂缺少疯子,你们带这人来在我面前疯癫吗?这人岂可进我的家呢?」

\par }\Chap{22}{\SH 残杀祭司
\par }{\PP \VerseOne{1}{\PN{大卫}}就离开那里,逃到{\PN{亚杜兰}}洞。他的弟兄和他父亲的全家听见了,就都下到他那里。
\VS{2}凡受窘迫的、欠债的、心里苦恼的都聚集到{\PN{大卫}}那里;{\PN{大卫}}就作他们的头目,跟随他的约有四百人。
\par }{\PP \VS{3}{\PN{大卫}}从那里往{\PN{摩押}}的{\PN{米斯巴}}去,对{\PN{摩押}}王说:「求你容我父母搬来,住在你们这里,等我知道 神要为我怎样行。」
\VS{4}{\PN{大卫}}领他父母到{\PN{摩押}}王面前。{\PN{大卫}}住山寨多少日子,他父母也住{\PN{摩押}}王那里多少日子。
\VS{5}先知{\PN{迦得}}对{\PN{大卫}}说:「你不要住在山寨,要往{\PN{犹大}}地去。」{\PN{大卫}}就离开那里,进入{\PN{哈列}}的树林。
\par }{\PP \VS{6}{\PN{扫罗}}在{\PN{基比亚}}的{\PN{拉玛}},坐在垂丝柳树下,手里拿着枪,众臣仆侍立在左右。{\PN{扫罗}}听见{\PN{大卫}}和跟随他的人在何处,
\VS{7}就对左右侍立的臣仆说:「{\PN{便雅悯}}人哪,你们要听我的话!{\PN{耶西}}的儿子能将田地和葡萄园赐给你们各人吗?能立你们各人作千夫长百夫长吗?
\VS{8}你们竟都结党害我!我的儿子与{\PN{耶西}}的儿子结盟的时候,无人告诉我;我的儿子挑唆我的臣子谋害我,就如今日的光景,也无人告诉我,为我忧虑。」
\VS{9}那时{\PN{以东}}人{\PN{多益}}站在{\PN{扫罗}}的臣仆中,对他说:「我曾看见{\PN{耶西}}的儿子到了{\PN{挪伯}},{\PN{亚希突}}的儿子{\PN{亚希米勒}}那里。
\VS{10}{\PN{亚希米勒}}为他求问耶和华,又给他食物,并给他{\ADD{杀}}{\PN{非利士}}人{\PN{歌利亚}}的刀。」
\par }{\PP \VS{11}王就打发人将祭司{\PN{亚希突}}的儿子{\PN{亚希米勒}}和他父亲的全家,就是住{\PN{挪伯}}的祭司都召了来;他们就来见王。
\VS{12}{\PN{扫罗}}说:「{\PN{亚希突}}的儿子,要听我的话!」他回答说:「主啊,我在这里。」
\VS{13}{\PN{扫罗}}对他说:「你为什么与{\PN{耶西}}的儿子结党害我,将食物和刀给他,又为他求问 神,使他起来谋害我,就如今日的光景?」
\VS{14}{\PN{亚希米勒}}回答王说:「王的臣仆中有谁比{\PN{大卫}}忠心呢?他是王的女婿,又是王的参谋,并且在王家中是尊贵的。
\VS{15}我岂是从今日才为他求问 神呢?断不是这样!王不要将罪归我和我父的全家;因为这事,无论大小,仆人都不知道。」
\VS{16}王说:「{\PN{亚希米勒}}啊,你和你父的全家都是该死的!」
\VS{17}王就吩咐左右的侍卫说:「你们去杀耶和华的祭司;因为他们帮助{\PN{大卫}},又知道{\PN{大卫}}逃跑,竟没有告诉我。」{\PN{扫罗}}的臣子却不肯伸手杀耶和华的祭司。
\VS{18}王吩咐{\PN{多益}}说:「你去杀祭司吧!」{\PN{以东}}人{\PN{多益}}就去杀祭司,那日杀了穿细麻布以弗得的八十五人;
\VS{19}又用刀将祭司城{\PN{挪伯}}中的男女、孩童、吃奶的,和牛、羊、驴尽都杀灭。
\par }{\PP \VS{20}{\PN{亚希突}}的儿子{\PN{亚希米勒}}有一个儿子,名叫{\PN{亚比亚他}},逃到{\PN{大卫}}那里。
\VS{21}{\PN{亚比亚他}}将{\PN{扫罗}}杀耶和华祭司的事告诉{\PN{大卫}}。
\VS{22}{\PN{大卫}}对{\PN{亚比亚他}}说:「那日我见{\PN{以东}}人{\PN{多益}}在那里,就知道他必告诉{\PN{扫罗}}。你父的全家{\ADD{丧}}命,都是因我的缘故。
\VS{23}你可以住在我这里,不要惧怕。因为寻索你命的就是寻索我的命;你在我这里可得保全。」

\par }\Chap{23}{\SH 大卫救基伊拉的居民
\par }{\PP \VerseOne{1}有人告诉{\PN{大卫}}说:「{\PN{非利士}}人攻击{\PN{基伊拉}},抢夺禾场。」
\VS{2}所以{\PN{大卫}}求问耶和华说:「我去攻打那些{\PN{非利士}}人可以不可以?」耶和华对{\PN{大卫}}说:「你可以去攻打{\PN{非利士}}人,拯救{\PN{基伊拉}}。」
\VS{3}跟随{\PN{大卫}}的人对他说:「我们在{\PN{犹大}}地这里尚且惧怕,何况往{\PN{基伊拉}}去攻打{\PN{非利士}}人的军旅呢?」
\VS{4}{\PN{大卫}}又求问耶和华。耶和华回答说:「你起身下{\PN{基伊拉}}去,我必将{\PN{非利士}}人交在你手里。」
\VS{5}{\PN{大卫}}和跟随他的人往{\PN{基伊拉}}去,与{\PN{非利士}}人打仗,大大杀败他们,又夺获他们的牲畜。这样,{\PN{大卫}}救了{\PN{基伊拉}}的居民。
\par }{\PP \VS{6}{\PN{亚希米勒}}的儿子{\PN{亚比亚他}}逃到{\PN{基伊拉}}见{\PN{大卫}}的时候,手里拿着以弗得。
\VS{7}有人告诉{\PN{扫罗}}说:「{\PN{大卫}}到了{\PN{基伊拉}}。」{\PN{扫罗}}说:「他进了有门有闩的城,困闭在里头;这是 神将他交在我手里了。」
\VS{8}于是{\PN{扫罗}}招聚众民,要下去攻打{\PN{基伊拉}}城,围困{\PN{大卫}}和跟随他的人。
\VS{9}{\PN{大卫}}知道{\PN{扫罗}}设计谋害他,就对祭司{\PN{亚比亚他}}说:「将以弗得拿过来。」
\VS{10}{\PN{大卫}}祷告说:「耶和华—{\PN{以色列}}的 神啊,你仆人听真了{\PN{扫罗}}要往{\PN{基伊拉}}来,为我的缘故灭城。
\VS{11}{\PN{基伊拉}}人将我交在{\PN{扫罗}}手里不交?{\PN{扫罗}}照着你仆人所听的话下来不下来?耶和华—{\PN{以色列}}的 神啊,求你指示仆人!」耶和华说:「{\PN{扫罗}}必下来。」
\VS{12}{\PN{大卫}}又说:「{\PN{基伊拉}}人将我和跟随我的人交在{\PN{扫罗}}手里不交?」耶和华说:「必交出来。」
\VS{13}{\PN{大卫}}和跟随他的约有六百人,就起身出了{\PN{基伊拉}},往他们所能往的地方去。有人告诉{\PN{扫罗}},{\PN{大卫}}离开{\PN{基伊拉}}逃走;于是{\PN{扫罗}}不出来了。
\par }{\SH 大卫在山地
\par }{\PP \VS{14}{\PN{大卫}}住在旷野的山寨里,常在{\PN{西弗}}旷野的山地。{\PN{扫罗}}天天寻索{\PN{大卫}}, 神却不将{\PN{大卫}}交在他手里。
\par }{\PP \VS{15}{\PN{大卫}}知道{\PN{扫罗}}出来寻索他的命。那时,他住在{\PN{西弗}}旷野的树林里;
\VS{16}{\PN{扫罗}}的儿子{\PN{约拿单}}起身,往那树林里去见{\PN{大卫}},使他倚靠 神得以坚固,
\VS{17}对他说:「不要惧怕!我父{\PN{扫罗}}的手必不加害于你;你必作{\PN{以色列}}的王,我也作你的宰相。这事我父{\PN{扫罗}}知道了。」
\VS{18}于是二人在耶和华面前立约。{\PN{大卫}}仍住在树林里,{\PN{约拿单}}回家去了。
\par }{\PP \VS{19}{\PN{西弗}}人上到{\PN{基比亚}}见{\PN{扫罗}},说:「{\PN{大卫}}不是在我们那里的树林里山寨中、旷野南边的{\PN{哈基拉山}}藏着吗?
\VS{20}王啊,请你随你的心愿下来,我们必亲自将他交在王的手里。」
\VS{21}{\PN{扫罗}}说:「愿耶和华赐福与你们,因你们顾恤我。
\VS{22}请你们回去,再确实查明他的住处和行踪,是谁看见他在那里,因为我听见人说他甚狡猾。
\VS{23}所以要看准他藏匿的地方,回来据实地告诉我,我就与你们同去。他若在{\PN{犹大}}的境内,我必从千门万户中搜出他来。」
\VS{24}{\PN{西弗}}人就起身,在{\PN{扫罗}}以先往{\PN{西弗}}去。
{\PN{大卫}}和跟随他的人却在{\PN{玛云}}旷野南边的{\PN{亚拉巴}}。
\par }{\PP \VS{25}{\PN{扫罗}}和跟随他的人去寻找{\PN{大卫}};有人告诉{\PN{大卫}},他就下到磐石,住在{\PN{玛云}}的旷野。{\PN{扫罗}}听见,便在{\PN{玛云}}的旷野追赶{\PN{大卫}}。
\VS{26}{\PN{扫罗}}在山这边走,{\PN{大卫}}和跟随他的人在山那边走。{\PN{大卫}}急忙躲避{\PN{扫罗}};因为{\PN{扫罗}}和跟随他的人,四面围住{\PN{大卫}}和跟随他的人,要拿获他们。
\VS{27}忽有使者来报告{\PN{扫罗}}说:「{\PN{非利士}}人犯境抢掠,请王快快回去!」
\VS{28}于是{\PN{扫罗}}不追赶{\PN{大卫}},回去攻打{\PN{非利士}}人。因此那地方名叫{\PN{西拉·哈玛希罗结}}。
\VS{29}{\PN{大卫}}从那里上去,住在{\PN{隐·基底}}的山寨里。

\par }\Chap{24}{\SH 大卫饶扫罗的命
\par }{\PP \VerseOne{1}{\PN{扫罗}}追赶{\PN{非利士}}人回来,有人告诉他说:「{\PN{大卫}}在{\PN{隐·基底}}的旷野。」
\VS{2}{\PN{扫罗}}就从{\PN{以色列}}人中挑选三千精兵,率领他们往野羊的磐石去,寻索{\PN{大卫}}和跟随他的人。
\VS{3}到了路旁的羊圈,在那里有洞,{\PN{扫罗}}进去大解。{\PN{大卫}}和跟随他的人正藏在洞里的深处。
\VS{4}跟随的人对{\PN{大卫}}说:「耶和华曾应许你说:『我要将你的仇敌交在你手里,你可以任意待他。』如今时候到了!」{\PN{大卫}}就起来,悄悄地割下{\PN{扫罗}}外袍的衣襟。
\VS{5}随后{\PN{大卫}}心中自责,因为割下{\PN{扫罗}}的衣襟;
\VS{6}对跟随他的人说:「我的主乃是耶和华的受膏者,我在耶和华面前万不敢伸手害他,因他是耶和华的受膏者。」
\VS{7}{\PN{大卫}}用这话拦住跟随他的人,不容他们起来害{\PN{扫罗}}。{\PN{扫罗}}起来,从洞里出去行路。
\par }{\PP \VS{8}随后{\PN{大卫}}也起来,从洞里出去,呼叫{\PN{扫罗}}说:「我主,我王!」{\PN{扫罗}}回头观看,{\PN{大卫}}就屈身、脸伏于地下拜。
\VS{9}{\PN{大卫}}对{\PN{扫罗}}说:「你为何听信人的谗言,说{\PN{大卫}}想要害你呢?
\VS{10}今日你亲眼看见在洞中,耶和华将你交在我手里;有人叫我杀你,我却爱惜你,说:『我不敢伸手害我的主,因为他是耶和华的受膏者。』
\VS{11}我父啊,看看你外袍的衣襟在我手中。我割下你的衣襟,没有杀你;你由此可以知道我没有恶意叛逆你。你虽然猎取我的命,我却没有得罪你。
\VS{12}愿耶和华在你我中间判断是非,在你身上为我伸冤,我却不亲手加害于你。
\VS{13}古人有句俗语说:『恶事出于恶人。』我却不亲手加害于你。
\VS{14}{\PN{以色列}}王出来要寻找谁呢?追赶谁呢?不过追赶一条死狗,一个虼蚤就是了。
\VS{15}愿耶和华在你我中间施行审判,断定是非,并且鉴察,为我伸冤,救我脱离你的手。」
\par }{\PP \VS{16}{\PN{大卫}}向{\PN{扫罗}}说完这话,{\PN{扫罗}}说:「我儿{\PN{大卫}},这是你的声音吗?」就放声大哭,
\VS{17}对{\PN{大卫}}说:「你比我公义;因为你以善待我,我却以恶待你。
\VS{18}你今日显明是以善待我;因为耶和华将我交在你手里,你却没有杀我。
\VS{19}人若遇见仇敌,岂肯放他平安无事地去呢?愿耶和华因你今日向我所行的,以善报你。
\VS{20}我也知道你必要作王,{\PN{以色列}}的国必坚立在你手里。
\VS{21}现在你要指着耶和华向我起誓,不剪除我的后裔,在我父家不灭没我的名。」
\VS{22}于是{\PN{大卫}}向{\PN{扫罗}}起誓,{\PN{扫罗}}就回家去;{\PN{大卫}}和跟随他的人上山寨去了。

\par }\Chap{25}{\SH 撒母耳去世
\par }{\PP \VerseOne{1}{\PN{撒母耳}}死了,{\PN{以色列}}众人聚集,为他哀哭,将他葬在{\PN{拉玛}}—他自己的坟墓\FTNT{}{{\FR 25:1: }原文是房屋}里。
\par }{\SH 大卫和亚比该
\par }{\PP {\PN{大卫}}起身,下到{\PN{巴兰}}的旷野。
\VS{2}在{\PN{玛云}}有一个人,他的产业在{\PN{迦密}},是一个大富户,有三千绵羊,一千山羊;他正在{\PN{迦密}}剪羊毛。
\VS{3}那人名叫{\PN{拿八}},是{\PN{迦勒}}族的人;他的妻名叫{\PN{亚比该}},是聪明俊美的妇人。{\PN{拿八}}为人刚愎凶恶。
\VS{4}{\PN{大卫}}在旷野听见说{\PN{拿八}}剪羊毛,
\VS{5}{\PN{大卫}}就打发十个仆人,吩咐他们说:「你们上{\PN{迦密}}去见{\PN{拿八}},提我的名问他安。
\VS{6}要对那富户如此说:『愿你平安,愿你家平安,愿你一切所有的都平安。
\VS{7}现在我听说有人为你剪羊毛,你的牧人在{\PN{迦密}}的时候和我们在一处,我们没有欺负他们,他们也未曾失落什么。
\VS{8}可以问你的仆人,他们必告诉你。所以愿我的仆人在你眼前蒙恩,因为是在好日子来的。求你随手取点赐与仆人和你儿子{\PN{大卫}}。』」
\par }{\PP \VS{9}{\PN{大卫}}的仆人到了,将这话提{\PN{大卫}}的名都告诉了{\PN{拿八}},就住了口。
\VS{10}{\PN{拿八}}回答{\PN{大卫}}的仆人说:「{\PN{大卫}}是谁?{\PN{耶西}}的儿子是谁?近来悖逆主人奔逃的仆人甚多,
\VS{11}我岂可将饮食和为我剪羊毛人所宰的肉给我不知道从哪里来的人呢?」
\VS{12}{\PN{大卫}}的仆人就转身从原路回去,照这话告诉{\PN{大卫}}。
\VS{13}{\PN{大卫}}向跟随他的人说:「你们各人都要带上刀!」众人就都带上刀,{\PN{大卫}}也带上刀。跟随{\PN{大卫}}上去的约有四百人,留下二百人看守器具。
\par }{\PP \VS{14}有{\PN{拿八}}的一个仆人告诉{\PN{拿八}}的妻{\PN{亚比该}}说:「{\PN{大卫}}从旷野打发使者来问我主人的安,主人却辱骂他们。
\VS{15}但是那些人待我们甚好;我们在田野与他们来往的时候,没有受他们的欺负,也未曾失落什么。
\VS{16}我们在他们那里牧羊的时候,他们昼夜作我们的保障。
\VS{17}所以你当筹划,看怎样行才好;不然,祸患定要临到我主人和他全家。他性情凶暴,无人敢与他说话。」
\par }{\PP \VS{18}{\PN{亚比该}}急忙将二百饼,两皮袋酒,五只收拾好了的羊,五细亚烘好了的穗子,一百葡萄饼,二百无花果饼,都驮在驴上,
\VS{19}对仆人说:「你们前头走,我随着你们去。」这事她却没有告诉丈夫{\PN{拿八}}。
\VS{20}{\PN{亚比该}}骑着驴,正下山坡,见{\PN{大卫}}和跟随他的人从对面下来,{\PN{亚比该}}就迎接他们。
\VS{21}{\PN{大卫}}曾说:「我在旷野为那人看守所有的,以致他一样不失落,实在是徒然了!他向我以恶报善。
\VS{22}凡属{\PN{拿八}}的男丁,我若留一个到明日早晨,愿 神重重降罚与我!」
\par }{\PP \VS{23}{\PN{亚比该}}见{\PN{大卫}},便急忙下驴,在{\PN{大卫}}面前脸伏于地叩拜,
\VS{24}俯伏在{\PN{大卫}}的脚前,说:「我主啊,愿这罪归我!求你容婢女向你进言,更求你听婢女的话。
\VS{25}我主不要理这坏人{\PN{拿八}},他的性情与他的名相称;他名叫{\PN{拿八}}\FTNT{}{{\FR 25:25: }就是愚顽的意思},他为人果然愚顽。但我主所打发的仆人,婢女并没有看见。
\VS{26}我主啊,耶和华既然阻止你亲手报仇,取流血的罪,所以我指着永生的耶和华、又敢在你面前起誓说:『愿你的仇敌和谋害你的人都像{\PN{拿八}}一样。』
\VS{27}如今求你将婢女送来的礼物给跟随你的仆人。
\VS{28}求你饶恕婢女的罪过。耶和华必为我主建立坚固的家,因我主为耶和华争战;并且在你平生的日子查不出有什么过来。
\VS{29}虽有人起来追逼你,寻索你的性命,你的性命却在耶和华—你的 神那里蒙保护,如包裹宝器一样;你仇敌的性命,耶和华必抛去,如用机弦甩石一样。
\VS{30-31}我主现在若不亲手报仇流无辜人的血,到了耶和华照所应许你的话赐福与你,立你作{\PN{以色列}}的王,那时我主必不至心里不安,觉得良心有亏。耶和华赐福与我主的时候,求你记念婢女。」
\par }{\PP \VS{32}{\PN{大卫}}对{\PN{亚比该}}说:「耶和华—{\PN{以色列}}的 神是应当称颂的,因为他今日使你来迎接我。
\VS{33}你和你的见识也当称赞;因为你今日拦阻我亲手报仇、流人的血。
\VS{34}我指着阻止我加害于你的耶和华—{\PN{以色列}}永生的 神起誓,你若不速速地来迎接我,到明日早晨,凡属{\PN{拿八}}的男丁必定不留一个。」
\VS{35}{\PN{大卫}}受了{\PN{亚比该}}送来的礼物,就对她说:「我听了你的话,准了你的情面,你可以平平安安地回家吧!」
\par }{\PP \VS{36}{\PN{亚比该}}到{\PN{拿八}}那里,见他在家里设摆筵席,如同王的筵席;{\PN{拿八}}快乐大醉。{\PN{亚比该}}无论大小事都没有告诉他,就等到次日早晨。
\VS{37}到了早晨,{\PN{拿八}}醒了酒,他的妻将这些事都告诉他,他就魂不附体,身僵如石头一般。
\VS{38}过了十天,耶和华击打{\PN{拿八}},他就死了。
\par }{\PP \VS{39}{\PN{大卫}}听见{\PN{拿八}}死了,就说:「应当称颂耶和华,因他伸了{\PN{拿八}}羞辱我的冤,又阻止仆人行恶;也使{\PN{拿八}}的恶归到{\PN{拿八}}的头上。」于是{\PN{大卫}}打发人去,与{\PN{亚比该}}说,要娶她为妻。
\VS{40}{\PN{大卫}}的仆人到了{\PN{迦密}}见{\PN{亚比该}},对她说:「{\PN{大卫}}打发我们来见你,想要娶你为妻。」
\VS{41}{\PN{亚比该}}就起来,俯伏在地,说:「我情愿作婢女,洗我主仆人的脚。」
\VS{42}{\PN{亚比该}}立刻起身,骑上驴,带着五个使女,跟从{\PN{大卫}}的使者去了,就作了{\PN{大卫}}的妻。
\par }{\PP \VS{43}{\PN{大卫}}先娶了{\PN{耶斯列}}人{\PN{亚希暖}},她们二人都作了他的妻。
\VS{44}{\PN{扫罗}}已将他的女儿{\PN{米甲}},就是{\PN{大卫}}的妻,给了{\PN{迦琳}}人{\PN{拉亿}}的儿子{\PN{帕提}}为妻。

\par }\Chap{26}{\SH 大卫又饶扫罗的命
\par }{\PP \VerseOne{1}{\PN{西弗}}人到{\PN{基比亚}}见{\PN{扫罗}},说:「{\PN{大卫}}不是在旷野前的{\PN{哈基拉山}}藏着吗?」
\VS{2}{\PN{扫罗}}就起身,带领{\PN{以色列}}人中挑选的三千精兵下到{\PN{西弗}}的旷野,要在那里寻索{\PN{大卫}}。
\VS{3}{\PN{扫罗}}在旷野前的{\PN{哈基拉山}},在道路上安营。{\PN{大卫}}住在旷野,听说{\PN{扫罗}}到旷野来追寻他,
\VS{4}就打发人去探听,便知道{\PN{扫罗}}果然来到。
\VS{5}{\PN{大卫}}起来,到{\PN{扫罗}}安营的地方,看见{\PN{扫罗}}和他的元帅{\PN{尼珥}}的儿子{\PN{押尼珥}}睡卧之处;{\PN{扫罗}}睡在辎重营里,百姓安营在他周围。
\par }{\PP \VS{6}{\PN{大卫}}对{\PN{赫}}人{\PN{亚希米勒}}和{\PN{洗鲁雅}}的儿子{\PN{约押}}的兄弟{\PN{亚比筛}}说:「谁同我下到{\PN{扫罗}}营里去?」{\PN{亚比筛}}说:「我同你下去。」
\VS{7}于是{\PN{大卫}}和{\PN{亚比筛}}夜间到了百姓那里,见{\PN{扫罗}}睡在辎重营里;他的枪在头旁,插在地上。{\PN{押尼珥}}和百姓睡在他周围。
\VS{8}{\PN{亚比筛}}对{\PN{大卫}}说:「现在 神将你的仇敌交在你手里,求你容我拿枪将他刺透在地,一刺就成,不用再刺。」
\VS{9}{\PN{大卫}}对{\PN{亚比筛}}说:「不可害死他。有谁伸手害耶和华的受膏者而无罪呢?」
\VS{10}{\PN{大卫}}又说:「我指着永生的耶和华起誓,他或被耶和华击打,或是死期到了,或是出战阵亡;
\VS{11}我在耶和华面前,万不敢伸手害耶和华的受膏者。现在你可以将他头旁的枪和水瓶拿来,我们就走。」
\VS{12}{\PN{大卫}}从{\PN{扫罗}}的头旁拿了枪和水瓶,二人就走了,没有人看见,没有人知道,也没有人醒起,都睡着了,因为耶和华使他们沉沉地睡了。
\par }{\PP \VS{13}{\PN{大卫}}过到那边去,远远地站在山顶上,与他们相离甚远。
\VS{14}{\PN{大卫}}呼叫百姓和{\PN{尼珥}}的儿子{\PN{押尼珥}}说:「{\PN{押尼珥}}啊,你为何不答应呢?」{\PN{押尼珥}}说:「你是谁?竟敢呼叫王呢?」
\VS{15}{\PN{大卫}}对{\PN{押尼珥}}说:「你不是个勇士吗?{\PN{以色列}}中谁能比你呢?民中有人进来要害死王—你的主,你为何没有保护王—你的主呢?
\VS{16}你这样是不好的!我指着永生的耶和华起誓,你们都是该死的;因为没有保护你们的主,就是耶和华的受膏者。现在你看看王头旁的枪和水瓶在哪里。」
\par }{\PP \VS{17}{\PN{扫罗}}听出是{\PN{大卫}}的声音,就说:「我儿{\PN{大卫}},这是你的声音吗?」{\PN{大卫}}说:「主—我的王啊,是我的声音」;
\VS{18}又说:「我做了什么?我手里有什么恶事?我主竟追赶仆人呢?
\VS{19}求我主我王听仆人的话:若是耶和华激发你攻击我,愿耶和华收纳祭物;若是人激发你,愿他在耶和华面前受咒诅;因为他现今赶逐我,不容我在耶和华的产业上有分,说:『你去事奉别神吧!』
\VS{20}现在求王不要使我的血流在离耶和华远的地方。{\PN{以色列}}王出来是寻找一个虼蚤,如同人在山上猎取一个鹧鸪一般。」
\par }{\PP \VS{21}{\PN{扫罗}}说:「我有罪了!我儿{\PN{大卫}},你可以回来,因你今日看我的性命为宝贵;我必不再加害于你。我是糊涂人,大大错了。」
\VS{22}{\PN{大卫}}说:「王的枪在这里,可以吩咐一个仆人过来拿去。
\VS{23}今日耶和华将王交在我手里,我却不肯伸手害耶和华的受膏者。耶和华必照各人的公义诚实报应他。
\VS{24}我今日重看你的性命,愿耶和华也重看我的性命,并且拯救我脱离一切患难。」
\VS{25}{\PN{扫罗}}对{\PN{大卫}}说:「我儿{\PN{大卫}},愿你得福!你必做大事,也必得胜。」于是{\PN{大卫}}起行,{\PN{扫罗}}回他的本处去了。

\par }\Chap{27}{\SH 大卫在非利士人中间
\par }{\PP \VerseOne{1}{\PN{大卫}}心里说:「必有一日我死在{\PN{扫罗}}手里,不如逃奔{\PN{非利士}}地去。{\PN{扫罗}}见我不在{\PN{以色列}}的境内,就必绝望,不再寻索我;这样我可以脱离他的手。」
\VS{2}于是{\PN{大卫}}起身,和跟随他的六百人投奔{\PN{迦特}}王—{\PN{玛俄}}的儿子{\PN{亚吉}}去了。
\VS{3}{\PN{大卫}}和他的两个妻,就是{\PN{耶斯列}}人{\PN{亚希暖}}和作过{\PN{拿八}}妻的{\PN{迦密}}人{\PN{亚比该}},并跟随他的人,连各人的眷属,都住在{\PN{迦特}}的{\PN{亚吉}}那里。
\VS{4}有人告诉{\PN{扫罗}}说:「{\PN{大卫}}逃到{\PN{迦特}}。」{\PN{扫罗}}就不再寻索他了。
\par }{\PP \VS{5}{\PN{大卫}}对{\PN{亚吉}}说:「我若在你眼前蒙恩,求你在京外的城邑中赐我一个地方居住。仆人何必与王同住京都呢?」
\VS{6}当日{\PN{亚吉}}将{\PN{洗革拉}}赐给他,因此{\PN{洗革拉}}属{\PN{犹大}}王,直到今日。
\VS{7}{\PN{大卫}}在{\PN{非利士}}地住了一年零四个月。
\par }{\PP \VS{8}{\PN{大卫}}和跟随他的人上去,侵夺{\PN{基述}}人、{\PN{基色}}人、{\PN{亚玛力}}人之地。这几族历来住在那地,从{\PN{书珥}}直到{\PN{埃及}}。
\VS{9}{\PN{大卫}}击杀那地的人,无论男女都没有留下一个,又夺获牛、羊、骆驼、驴,并衣服,回来见{\PN{亚吉}}。
\VS{10}{\PN{亚吉}}说:「你们今日侵夺了什么地方呢?」{\PN{大卫}}说:「侵夺了{\PN{犹大}}的南方、{\PN{耶拉篾}}的南方、{\PN{基尼}}的南方。」
\VS{11}无论男女,{\PN{大卫}}没有留下一个带到{\PN{迦特}}来。他说:「恐怕他们将我们的事告诉人,说{\PN{大卫}}住在{\PN{非利士}}地的时候常常这样行。」
\VS{12}{\PN{亚吉}}信了{\PN{大卫}},心里说:「{\PN{大卫}}使本族{\PN{以色列}}人憎恶他,所以他必永远作我的仆人了。」

\par }\Chap{28}{\PP \VerseOne{1}那时,{\PN{非利士}}人聚集军旅,要与{\PN{以色列}}人打仗。{\PN{亚吉}}对{\PN{大卫}}说:「你当知道,你和跟随你的人都要随我出战。」
\VS{2}{\PN{大卫}}对{\PN{亚吉}}说:「仆人所能做的事,王必知道。」{\PN{亚吉}}对{\PN{大卫}}说:「这样,我立你永远作我的护卫长。」
\par }{\SH 扫罗求问女巫
\par }{\PP \VS{3}那时{\PN{撒母耳}}已经死了,{\PN{以色列}}众人为他哀哭,葬他在{\PN{拉玛}},就是在他本城里。{\PN{扫罗}}曾在国内不容有交鬼的和行巫术的人。
\VS{4}{\PN{非利士}}人聚集,来到{\PN{书念}}安营;{\PN{扫罗}}聚集{\PN{以色列}}众人在{\PN{基利波}}安营。
\VS{5}{\PN{扫罗}}看见{\PN{非利士}}的军旅就惧怕,心中发颤。
\VS{6}{\PN{扫罗}}求问耶和华,耶和华却不借梦,或乌陵,或先知回答他。
\VS{7}{\PN{扫罗}}吩咐臣仆说:「当为我找一个交鬼的妇人,我好去问她。」臣仆说:「在{\PN{隐·多珥}}有一个交鬼的妇人。」
\par }{\PP \VS{8}于是{\PN{扫罗}}改了装,穿上别的衣服,带着两个人,夜里去见那妇人。{\PN{扫罗}}说:「求你用交鬼的法术,将我所告诉你的死人,为我招上来。」
\VS{9}妇人对他说:「你知道{\PN{扫罗}}从国中剪除交鬼的和行巫术的。你为何陷害我的性命,使我死呢?」
\VS{10}{\PN{扫罗}}向妇人指着耶和华起誓说:「我指着永生的耶和华起誓,你必不因这事受刑。」
\VS{11}妇人说:「我为你招谁上来呢?」回答说:「为我招{\PN{撒母耳}}上来。」
\VS{12}妇人看见{\PN{撒母耳}},就大声呼叫,对{\PN{扫罗}}说:「你是{\PN{扫罗}},为什么欺哄我呢?」
\VS{13}王对妇人说:「不要惧怕,你看见了什么呢?」妇人对{\PN{扫罗}}说:「我看见有神从地里上来。」
\VS{14}{\PN{扫罗}}说:「他是怎样的形状?」妇人说:「有一个老人上来,身穿长衣。」{\PN{扫罗}}知道是{\PN{撒母耳}},就屈身,脸伏于地下拜。
\par }{\PP \VS{15}{\PN{撒母耳}}对{\PN{扫罗}}说:「你为什么搅扰我,招我上来呢?」{\PN{扫罗}}回答说:「我甚窘急;因为{\PN{非利士}}人攻击我, 神也离开我,不再借先知或梦回答我。因此请你上来,好指示我应当怎样行。」
\VS{16}{\PN{撒母耳}}说:「耶和华已经离开你,且与你为敌,你何必问我呢?
\VS{17}耶和华照他借我说的话,已经从你手里夺去国权,赐与别人,就是{\PN{大卫}}。
\VS{18}因你没有听从耶和华的命令;他恼怒{\PN{亚玛力}}人,你没有灭绝他们,所以今日耶和华向你这样行,
\VS{19}并且耶和华必将你和{\PN{以色列}}人交在{\PN{非利士}}人的手里。明日你和你众子必与我在一处了;耶和华必将{\PN{以色列}}的军兵交在{\PN{非利士}}人手里。」
\par }{\PP \VS{20}{\PN{扫罗}}猛然仆倒,挺身在地,因{\PN{撒母耳}}的话甚是惧怕;那一昼一夜,没有吃什么,就毫无气力。
\VS{21}妇人到{\PN{扫罗}}面前,见他极其惊恐,对他说:「婢女听从你的话,不顾惜自己的性命,遵从你所吩咐的。
\VS{22}现在求你听婢女的话,容我在你面前摆上一点食物,你吃了,可以有气力行路。」
\VS{23}{\PN{扫罗}}不肯,说:「我不吃。」但他的仆人和妇人再三劝他,他才听了他们的话,从地上起来,坐在床上。
\VS{24}妇人急忙将家里的一只肥牛犊宰了,又拿面抟成无酵饼烤了,
\VS{25}摆在{\PN{扫罗}}和他仆人面前。他们吃完,当夜就起身走了。

\par }\Chap{29}{\SH 非利士人不要大卫
\par }{\PP \VerseOne{1}{\PN{非利士}}人将他们的军旅聚到{\PN{亚弗}};{\PN{以色列}}人在{\PN{耶斯列}}的泉旁安营。
\VS{2}{\PN{非利士}}人的首领各率军队,或百或千,挨次前进;{\PN{大卫}}和跟随他的人同着{\PN{亚吉}}跟在后边。
\VS{3}{\PN{非利士}}人的首领说:「这些{\PN{希伯来}}人{\ADD{在这里}}做什么呢?」{\PN{亚吉}}对他们说:「这不是{\PN{以色列}}王{\PN{扫罗}}的臣子{\PN{大卫}}吗?他在我这里有些年日了。自从他投降我直到今日,我未曾见他有过错。」
\par }{\PP \VS{4}{\PN{非利士}}人的首领向{\PN{亚吉}}发怒,对他说:「你要叫这人回你所安置他的地方,不可叫他同我们出战,恐怕他在阵上反为我们的敌人。他用什么与他主人复和呢?岂不是用我们这些人的首级吗?
\VS{5}从前{\PN{以色列}}的妇女跳舞唱和说:『{\PN{扫罗}}杀死千千,{\PN{大卫}}杀死万万』,所说的不是这个{\PN{大卫}}吗?」
\par }{\PP \VS{6}{\PN{亚吉}}叫{\PN{大卫}}来,对他说:「我指着永生的耶和华起誓,你是正直人。你随我在军中出入,我看你甚好。自从你投奔我到如今,我未曾见你有什么过失;只是众首领不喜悦你。
\VS{7}现在你可以平平安安地回去,免得{\PN{非利士}}人的首领不欢喜你。」
\VS{8}{\PN{大卫}}对{\PN{亚吉}}说:「我做了什么呢?自从仆人到你面前,直到今日,你查出我有什么过错,使我不去攻击主—我王的仇敌呢?」
\VS{9}{\PN{亚吉}}说:「我知道你在我眼前是好人,如同 神的使者一般;只是{\PN{非利士}}人的首领说:『这人不可同我们出战。』
\VS{10}故此你和跟随你的人,就是你本主的仆人,要明日早晨起来,等到天亮回去吧!」
\VS{11}于是{\PN{大卫}}和跟随他的人早晨起来,回往{\PN{非利士}}地去。{\PN{非利士}}人也上{\PN{耶斯列}}去了。

\par }\Chap{30}{\SH 与亚玛力人争战
\par }{\PP \VerseOne{1}第三日,{\PN{大卫}}和跟随他的人到了{\PN{洗革拉}}。{\PN{亚玛力}}人已经侵夺南地,攻破{\PN{洗革拉}},用火焚烧,
\VS{2}掳了城内的妇女和其中的大小人口,却没有杀一个,都带着走了。
\VS{3}{\PN{大卫}}和跟随他的人到了那城,不料,城已烧毁,他们的妻子儿女都被掳去了。
\VS{4}{\PN{大卫}}和跟随他的人就放声大哭,直哭得没有气力。
\VS{5}{\PN{大卫}}的两个妻—{\PN{耶斯列}}人{\PN{亚希暖}}和作过{\PN{拿八}}妻的{\PN{迦密}}人{\PN{亚比该}},也被掳去了。
\VS{6}{\PN{大卫}}甚是焦急,因众人为自己的儿女苦恼,说:「要用石头打死他。」{\PN{大卫}}却倚靠耶和华—他的 神,心里坚固。
\par }{\PP \VS{7}{\PN{大卫}}对{\PN{亚希米勒}}的儿子祭司{\PN{亚比亚他}}说:「请你将以弗得拿过来。」{\PN{亚比亚他}}就将以弗得拿到{\PN{大卫}}面前。
\VS{8}{\PN{大卫}}求问耶和华说:「我追赶敌军,追得上追不上呢?」耶和华说:「你可以追,必追得上,都救得回来。」
\VS{9}于是,{\PN{大卫}}和跟随他的六百人来到{\PN{比梭溪}};有不能前去的就留在那里。
\VS{10}{\PN{大卫}}却带着四百人往前追赶,有二百人疲乏,不能过{\PN{比梭溪}},所以留在那里。
\par }{\PP \VS{11}这四百人在田野遇见一个{\PN{埃及}}人,就带他到{\PN{大卫}}面前,给他饼吃,给他水喝,
\VS{12}又给他一块无花果饼,两个葡萄饼。他吃了,就精神复原;因为他三日三夜没有吃饼,没有喝水。
\VS{13}{\PN{大卫}}问他说:「你是属谁的?你是哪里的人?」他回答说:「我是{\PN{埃及}}的少年人,是{\PN{亚玛力}}人的奴仆;因我三日前患病,我主人就把我撇弃了。
\VS{14}我们侵夺了{\PN{基利提}}的南方和属{\PN{犹大}}的地,并{\PN{迦勒}}地的南方,又用火烧了{\PN{洗革拉}}。」
\VS{15}{\PN{大卫}}问他说:「你肯领我们到敌军那里不肯?」他回答说:「你要向我指着 神起誓,不杀我,也不将我交在我主人手里,我就领你下到敌军那里。」
\par }{\PP \VS{16}那人领{\PN{大卫}}下去,见他们散在地上,吃喝跳舞,因为从{\PN{非利士}}地和{\PN{犹大}}地所掳来的财物甚多。
\VS{17}{\PN{大卫}}从黎明直到次日晚上,击杀他们,除了四百骑骆驼的少年人之外,没有一个逃脱的。
\VS{18}{\PN{亚玛力}}人所掳去的财物,{\PN{大卫}}全都夺回,并救回他的两个妻来。
\VS{19}凡{\PN{亚玛力}}人所掳去的,无论大小、儿女、财物,{\PN{大卫}}都夺回来,没有失落一个。
\VS{20}{\PN{大卫}}所夺来的牛群羊群,跟随他的人赶在{\ADD{原有的}}群畜前边,说:「这是{\PN{大卫}}的掠物。」
\par }{\PP \VS{21}{\PN{大卫}}到了那疲乏不能跟随、留在{\PN{比梭溪}}的二百人那里。他们出来迎接{\PN{大卫}}并跟随的人。{\PN{大卫}}前来问他们安。
\VS{22}跟随{\PN{大卫}}人中的恶人和匪类说:「这些人既然没有和我们同去,我们所夺的财物就不分给他们,只将他们各人的妻子儿女给他们,使他们带去就是了。」
\VS{23}{\PN{大卫}}说:「弟兄们,耶和华所赐给我们的,不可不分给他们;因为他保佑我们,将那攻击我们的敌军交在我们手里。
\VS{24}这事谁肯依从你们呢?上阵的得多少,看守器具的也得多少;应当大家平分。」
\VS{25}{\PN{大卫}}定此为{\PN{以色列}}的律例典章,从那日直到今日。
\par }{\PP \VS{26}{\PN{大卫}}到了{\PN{洗革拉}},从掠物中取些送给他朋友{\PN{犹大}}的长老,说:「这是从耶和华仇敌那里夺来的,送你们为礼物。」
\VS{27}他送礼物给住{\PN{伯特利}}的,南地{\PN{拉末}}的,{\PN{雅提珥}}的;
\VS{28}住{\PN{亚罗珥}}的,{\PN{息末}}的,{\PN{以实提莫}}的;
\VS{29}住{\PN{拉哈勒}}的,{\PN{耶拉篾}}各城的,{\PN{基尼}}各城的;
\VS{30}住{\PN{何珥玛}}的,{\PN{歌拉珊}}的,{\PN{亚挞}}的;
\VS{31}住{\PN{希伯
}}的,并{\PN{大卫}}和跟随他的人素来所到之处的人。

\par }\Chap{31}{\SH 扫罗和他儿子的死
\par }{\R (代上10·1—12)
\par }{\PP \VerseOne{1}{\PN{非利士}}人与{\PN{以色列}}人争战。{\PN{以色列}}人在{\PN{非利士}}人面前逃跑,在{\PN{基利波}}有被杀仆倒的。
\VS{2}{\PN{非利士}}人紧追{\PN{扫罗}}和他儿子们,就杀了{\PN{扫罗}}的儿子{\PN{约拿单}}、{\PN{亚比拿达}}、{\PN{麦基舒亚}}。
\VS{3}势派甚大,{\PN{扫罗}}被弓箭手追上,射伤甚重,
\VS{4}就吩咐拿他兵器的人说:「你拔出刀来,将我刺死,免得那些未受割礼的人来刺我,凌辱我。」但拿兵器的人甚惧怕,不肯刺他;{\PN{扫罗}}就自己伏在刀上死了。
\VS{5}拿兵器的人见{\PN{扫罗}}已死,也伏在刀上死了。
\VS{6}这样,{\PN{扫罗}}和他三个儿子,与拿他兵器的人,以及跟随他的人,都一同死亡。
\VS{7}住平原那边并{\PN{约旦河}}西的{\PN{以色列}}人,见{\PN{以色列}}军兵逃跑,{\PN{扫罗}}和他儿子都死了,也就弃城逃跑。{\PN{非利士}}人便来住在其中。
\par }{\PP \VS{8}次日,{\PN{非利士}}人来剥那被杀之人的衣服,看见{\PN{扫罗}}和他三个儿子仆倒在{\PN{基利波山}},
\VS{9}就割下他的首级,剥了他的军装,打发人到\FTNT{}{{\FR 31:9: }或译:送到}{\PN{非利士}}地的四境,报信与他们庙里的偶像和众民;
\VS{10}又将{\PN{扫罗}}的军装放在{\PN{亚斯她录}}庙里,将他的尸身钉在{\PN{伯·珊}}的城墙上。
\VS{11}{\PN{基列·雅比}}的居民听见{\PN{非利士}}人向{\PN{扫罗}}所行的事,
\VS{12}他们中间所有的勇士就起身,走了一夜,将{\PN{扫罗}}和他儿子的尸身从{\PN{伯·珊}}城墙上取下来,送到{\PN{雅比}}那里,用火烧了;
\VS{13}将他们骸骨葬在{\PN{雅比}}的垂丝柳树下,就禁食七日。
\par }