\NormalFont\ShortTitle{箴言}
{\MT 箴言

\par }\ChapOne{1}{\SH 箴言的价值
\par }{\Q \VerseOne{1}{\PN{以色列}}王{\PN{大卫}}儿子{\PN{所罗门}}的箴言:
\par }{\Q \VS{2}要使人晓得智慧和训诲,
\par }{\Q 分辨通达的言语,
\par }{\Q \VS{3}使人处事领受智慧、
\par }{\Q 仁义、公平、正直的训诲,
\par }{\Q \VS{4}使愚人灵明,
\par }{\Q 使少年人有知识和谋略,
\par }{\Q \VS{5}使智慧人听见,增长学问,
\par }{\Q 使聪明人得着智谋,
\par }{\Q \VS{6}使人明白箴言和譬喻,
\par }{\Q 懂得智慧人的言词和谜语。
\par }{\SH 给年青人的忠告
\par }{\Q \VS{7}敬畏耶和华是知识的开端;
\par }{\Q 愚妄人藐视智慧和训诲。
\par }{\BB \par }{\Q \VS{8}我儿,要听你父亲的训诲,
\par }{\Q 不可离弃你母亲的法则\FTNT{}{{\FR 1:8: }或译:指教};
\par }{\Q \VS{9}因为这要作你头上的华冠,
\par }{\Q 你项上的{\ADD{金}}链。
\par }{\BB \par }{\Q \VS{10}我儿,恶人若引诱你,
\par }{\Q 你不可随从。
\par }{\Q \VS{11}他们若说:你与我们同去,
\par }{\Q 我们要埋伏流人之血,
\par }{\Q 要蹲伏害无罪之人;
\par }{\Q \VS{12}我们好像阴间,把他们活活吞下;
\par }{\Q 他们如同下坑的人,
\par }{\Q 被我们囫囵吞了;
\par }{\Q \VS{13}我们必得各样宝物,
\par }{\Q 将所掳来的,装满房屋;
\par }{\Q \VS{14}你与我们大家同分,
\par }{\Q 我们共用一个囊袋;
\par }{\Q \VS{15}我儿,不要与他们同行一道,
\par }{\Q 禁止你脚走他们的路。
\par }{\Q \VS{16}因为,他们的脚奔跑行恶;
\par }{\Q 他们急速流人的血,
\par }{\Q \VS{17}好像飞鸟,
\par }{\Q 网罗设在眼前仍不躲避。
\par }{\Q \VS{18}这些人埋伏,是为自流己血;
\par }{\Q 蹲伏,是为自害己命。
\par }{\Q \VS{19}凡贪恋财利的,所行之路都是如此;
\par }{\Q 这贪恋之心乃夺去得财者之命。
\par }{\SH 智慧呼唤人
\par }{\Q \VS{20}智慧在街市上呼喊,
\par }{\Q 在宽阔处发声,
\par }{\Q \VS{21}在热闹街头喊叫,
\par }{\Q 在城门口,在城中发出言语,
\par }{\Q \VS{22}说:你们愚昧人喜爱愚昧,
\par }{\Q 亵慢人喜欢亵慢,
\par }{\Q 愚顽人恨恶知识,要到几时呢?
\par }{\Q \VS{23}你们当因我的责备回转;
\par }{\Q 我要将我的灵浇灌你们,
\par }{\Q 将我的话指示你们。
\par }{\Q \VS{24}我呼唤,你们不肯听从;
\par }{\Q 我伸手,无人理会;
\par }{\Q \VS{25}反轻弃我一切的劝戒,
\par }{\Q 不肯受我的责备。
\par }{\Q \VS{26}你们遭灾难,我就发笑;
\par }{\Q 惊恐临到你们,我必嗤笑。
\par }{\Q \VS{27}惊恐临到你们,好像狂风;
\par }{\Q 灾难来到,如同暴风;
\par }{\Q 急难痛苦临到你们身上。
\par }{\Q \VS{28}那时,你们必呼求我,我却不答应,
\par }{\Q 恳切地寻找我,却寻不见。
\par }{\Q \VS{29}因为,你们恨恶知识,
\par }{\Q 不喜爱敬畏耶和华,
\par }{\Q \VS{30}不听我的劝戒,
\par }{\Q 藐视我一切的责备,
\par }{\Q \VS{31}所以必吃自结的果子,
\par }{\Q 充满自设的计谋。
\par }{\Q \VS{32}愚昧人背道,必杀己身;
\par }{\Q 愚顽人安逸,必害己命。
\par }{\Q \VS{33}惟有听从我的,必安然居住,
\par }{\Q 得享安静,不怕灾祸。

\par }\Chap{2}{\SH 智慧的赏赐
\par }{\Q \VerseOne{1}我儿,你若领受我的言语,
\par }{\Q 存记我的命令,
\par }{\Q \VS{2}侧耳听智慧,
\par }{\Q 专心求聪明,
\par }{\Q \VS{3}呼求明哲,
\par }{\Q 扬声求聪明,
\par }{\Q \VS{4}寻找它,如寻找银子,
\par }{\Q 搜求它,如搜求隐藏的珍宝,
\par }{\Q \VS{5}你就明白敬畏耶和华,
\par }{\Q 得以认识 神。
\par }{\Q \VS{6}因为,耶和华赐人智慧;
\par }{\Q 知识和聪明都由他口而出。
\par }{\Q \VS{7}他给正直人存留真智慧,
\par }{\Q 给行为纯正的人作盾牌,
\par }{\Q \VS{8}为要保守公平人的路,
\par }{\Q 护庇虔敬人的道。
\par }{\BB \par }{\Q \VS{9}你也必明白仁义、公平、
\par }{\Q 正直、一切的善道。
\par }{\Q \VS{10}智慧必入你心;
\par }{\Q 你的灵要以知识为美。
\par }{\Q \VS{11}谋略必护卫你;
\par }{\Q 聪明必保守你,
\par }{\Q \VS{12}要救你脱离恶道\FTNT{}{{\FR 2:12: }或译:恶人的道},
\par }{\Q 脱离说乖谬话的人。
\par }{\Q \VS{13}那等人舍弃正直的路,
\par }{\Q 行走黑暗的道,
\par }{\Q \VS{14}欢喜作恶,
\par }{\Q 喜爱恶{\ADD{人}}的乖僻,
\par }{\Q \VS{15}在他们的道中弯曲,
\par }{\Q 在他们的路上偏僻。
\par }{\BB \par }{\Q \VS{16}智{\ADD{慧}}要救你脱离淫妇,
\par }{\Q 就是那油嘴滑舌的外女。
\par }{\Q \VS{17}她离弃幼年的配偶,
\par }{\Q 忘了 神的盟约。
\par }{\Q \VS{18}她的家陷入死地;
\par }{\Q 她的路偏向阴间。
\par }{\Q \VS{19}凡到她那里去的,不得转回,
\par }{\Q 也得不着生命的路。
\par }{\BB \par }{\Q \VS{20}{\ADD{智慧}}必使你行善人的道,
\par }{\Q 守义人的路。
\par }{\Q \VS{21}正直人必在世上居住;
\par }{\Q 完全人必在地上存留。
\par }{\Q \VS{22}惟有恶人必然剪除;
\par }{\Q 奸诈的,必然拔出。

\par }\Chap{3}{\SH 给青年人的忠告
\par }{\Q \VerseOne{1}我儿,不要忘记我的法则\FTNT{}{{\FR 3:1: }或译:指教};
\par }{\Q 你心要谨守我的诫命;
\par }{\Q \VS{2}因为它必将长久的日子,
\par }{\Q 生命的年数与平安,加给你。
\par }{\Q \VS{3}不可使慈爱、诚实离开你,
\par }{\Q 要系在你颈项上,刻在你心版上。
\par }{\Q \VS{4}这样,你必在 神和世人眼前
\par }{\Q 蒙恩宠,有聪明。
\par }{\BB \par }{\Q \VS{5}你要专心仰赖耶和华,
\par }{\Q 不可倚靠自己的聪明,
\par }{\Q \VS{6}在你一切所行的事上都要认定他,
\par }{\Q 他必指引你的路。
\par }{\Q \VS{7}不要自以为有智慧;
\par }{\Q 要敬畏耶和华,远离恶事。
\par }{\Q \VS{8}这便医治你的肚脐,
\par }{\Q 滋润你的百骨。
\par }{\BB \par }{\Q \VS{9}你要以财物
\par }{\Q 和一切初熟的土产尊荣耶和华。
\par }{\Q \VS{10}这样,你的仓房必充满有余;
\par }{\Q 你的酒榨有新酒盈溢。
\par }{\BB \par }{\Q \VS{11}我儿,你不可轻看耶和华的管教\FTNT{}{{\FR 3:11: }或译:惩治},
\par }{\Q 也不可厌烦他的责备;
\par }{\Q \VS{12}因为耶和华所爱的,他必责备,
\par }{\Q 正如父亲责备所喜爱的儿子。
\par }{\BB \par }{\Q \VS{13}得智慧,得聪明的,
\par }{\Q 这人便为有福。
\par }{\Q \VS{14}因为得智慧胜过得银子,
\par }{\Q 其利益强如精金,
\par }{\Q \VS{15}比珍珠\FTNT{}{{\FR 3:15: }或译:红宝石}宝贵;
\par }{\Q 你一切所喜爱的,都不足与比较。
\par }{\Q \VS{16}她右手有长寿,
\par }{\Q 左手有富贵。
\par }{\Q \VS{17}她的道是安乐;
\par }{\Q 她的路全是平安。
\par }{\Q \VS{18}她与持守她的作生命树;
\par }{\Q 持定她的,俱各有福。
\par }{\Q \VS{19}耶和华以智慧立地,
\par }{\Q 以聪明定天,
\par }{\Q \VS{20}以知识使深渊裂开,
\par }{\Q 使天空滴下甘露。
\par }{\BB \par }{\Q \VS{21}我儿,要谨守真智慧和谋略,
\par }{\Q 不可使她离开你的眼目。
\par }{\Q \VS{22}这样,她必作你的生命,
\par }{\Q 颈项的美饰。
\par }{\Q \VS{23}你就坦然行路,
\par }{\Q 不致碰脚。
\par }{\Q \VS{24}你躺下,必不惧怕;
\par }{\Q 你躺卧,睡得香甜。
\par }{\Q \VS{25}忽然来的惊恐,不要害怕;
\par }{\Q 恶人遭毁灭,也不要恐惧。
\par }{\Q \VS{26}因为耶和华是你所倚靠的;
\par }{\Q 他必保守你的脚不陷入网罗。
\par }{\BB \par }{\Q \VS{27}你手若有行善的力量,不可推辞,
\par }{\Q 就当向那应得的人施行。
\par }{\Q \VS{28}你那里若有现成的,不可对邻舍说:
\par }{\Q 去吧,明天再来,我必给你。
\par }{\Q \VS{29}你的邻舍既在你附近安居,
\par }{\Q 你不可设计害他。
\par }{\Q \VS{30}人未曾加害与你,
\par }{\Q 不可无故与他相争。
\par }{\Q \VS{31}不可嫉妒强暴的人,
\par }{\Q 也不可选择他所行的路。
\par }{\Q \VS{32}因为,乖僻人为耶和华所憎恶;
\par }{\Q 正直人为他所亲密。
\par }{\Q \VS{33}耶和华咒诅恶人的家庭,
\par }{\Q 赐福与义人的居所。
\par }{\Q \VS{34}他讥诮那好讥诮的人,
\par }{\Q 赐恩给谦卑的人。
\par }{\Q \VS{35}智慧人必承受尊荣;
\par }{\Q 愚昧人高升也成为羞辱。

\par }\Chap{4}{\SH 智慧的益处
\par }{\Q \VerseOne{1}众子啊,要听父亲的教训,
\par }{\Q 留心得知聪明。
\par }{\Q \VS{2}因我所给你们的是好教训;
\par }{\Q 不可离弃我的法则\FTNT{}{{\FR 4:2: }或译:指教}。
\par }{\Q \VS{3}我在父亲面前为孝子,
\par }{\Q 在母亲眼中为独一的娇儿。
\par }{\Q \VS{4}父亲教训我说:你心要存记我的言语,
\par }{\Q 遵守我的命令,便得存活。
\par }{\Q \VS{5}要得智慧,要得聪明,不可忘记,
\par }{\Q 也不可偏离我口中的言语。
\par }{\Q \VS{6}不可离弃智慧,智慧就护卫你;
\par }{\Q 要爱她,她就保守你。
\par }{\Q \VS{7}智慧为首;
\par }{\Q {\ADD{所以}},要得智慧。
\par }{\Q 在你一切所得之内必得聪明
\FTNT{}{{\FR 4:7: }或译:用你一切所得的去换聪明}。
\par }{\Q \VS{8}高举智慧,她就使你高升;
\par }{\Q 怀抱智慧,她就使你尊荣。
\par }{\Q \VS{9}她必将华冠加在你头上,
\par }{\Q 把荣冕交给你。
\par }{\BB \par }{\Q \VS{10}我儿,你要听受我的言语,
\par }{\Q 就必延年益寿。
\par }{\Q \VS{11}我已指教你走智慧的道,
\par }{\Q 引导你行正直的路。
\par }{\Q \VS{12}你行走,脚步必不致狭窄;
\par }{\Q 你奔跑,也不致跌倒。
\par }{\Q \VS{13}要持定训诲,不可放松;
\par }{\Q 必当谨守,因为它是你的生命。
\par }{\Q \VS{14}不可行恶人的路;
\par }{\Q 不要走坏人的道。
\par }{\Q \VS{15}要躲避,不可经过;
\par }{\Q 要转身而去。
\par }{\Q \VS{16}这等人若不行恶,不得睡觉;
\par }{\Q 不使人跌倒,睡卧不安;
\par }{\Q \VS{17}因为他们以奸恶吃饼,
\par }{\Q 以强暴喝酒。
\par }{\BB \par }{\Q \VS{18}但义人的路好像黎明的光,
\par }{\Q 越照越明,直到日午。
\par }{\Q \VS{19}恶人的道好像幽暗,
\par }{\Q 自己不知因什么跌倒。
\par }{\BB \par }{\Q \VS{20}我儿,要留心听我的言词,
\par }{\Q 侧耳听我的话语,
\par }{\BB \par }{\Q \VS{21}都不可离你的眼目,
\par }{\Q 要存记在你心中。
\par }{\Q \VS{22}因为得着它的,就得了生命,
\par }{\Q 又得了医全体的良药。
\par }{\Q \VS{23}你要保守你心,胜过保守一切\FTNT{}{{\FR 4:23: }或译:你要切切保守你心},
\par }{\Q 因为一生的果效是由心发出。
\par }{\Q \VS{24}你要除掉邪僻的口,
\par }{\Q 弃绝乖谬的嘴。
\par }{\Q \VS{25}你的眼目要向前正看;
\par }{\Q 你的眼睛\FTNT{}{{\FR 4:25: }原文是皮}当向前直观。
\par }{\Q \VS{26}要修平你脚下的路,
\par }{\Q 坚定你一切的道。
\par }{\Q \VS{27}不可偏向左右;
\par }{\Q 要使你的脚离开邪恶。

\par }\Chap{5}{\SH 警告勿犯淫乱
\par }{\Q \VerseOne{1}我儿,要留心我智慧的话语,
\par }{\Q 侧耳听我聪明的言词,
\par }{\Q \VS{2}为要使你谨守谋略,
\par }{\Q 嘴唇保存知识。
\par }{\Q \VS{3}因为淫妇的嘴滴下蜂蜜;
\par }{\Q 她的口比油更滑,
\par }{\Q \VS{4}至终却苦似茵 ,
\par }{\Q 快如两刃的刀。
\par }{\Q \VS{5}她的脚下入死地;
\par }{\Q 她脚步踏住阴间,
\par }{\Q \VS{6}以致她找不着生命平坦的道。
\par }{\Q 她的路变迁不定,
\par }{\Q 自己还不知道。
\par }{\BB \par }{\Q \VS{7}众子啊,现在要听从我;
\par }{\Q 不可离弃我口中的话。
\par }{\Q \VS{8}你所行的道要离她远,
\par }{\Q 不可就近她的房门,
\par }{\Q \VS{9}恐怕将你的尊荣给别人,
\par }{\Q 将你的岁月给残忍的人;
\par }{\Q \VS{10}恐怕外人满得你的力量,
\par }{\Q 你劳碌得来的,归入外人的家;
\par }{\Q \VS{11}终久,你皮肉和身体消毁,
\par }{\Q 你就悲叹,
\par }{\Q \VS{12}说:我怎么恨恶训诲,
\par }{\Q 心中藐视责备,
\par }{\Q \VS{13}也不听从我师傅的话,
\par }{\Q 又不侧耳听那教训我的人?
\par }{\Q \VS{14}我在{\ADD{圣}}会里,
\par }{\Q 几乎落在诸般恶中。
\par }{\BB \par }{\Q \VS{15}你要喝自己池中的水,
\par }{\Q 饮自己井里的活水。
\par }{\Q \VS{16}你的泉源岂可涨溢在外?
\par }{\Q 你的河水岂可流在街上?
\par }{\Q \VS{17}惟独归你一人,
\par }{\Q 不可与外人同用。
\par }{\Q \VS{18}要使你的泉源蒙福;
\par }{\Q 要喜悦你幼年所娶的妻。
\par }{\Q \VS{19}她如可爱的麀鹿,可喜的母鹿;
\par }{\Q 愿她的胸怀使你时时知足,
\par }{\Q 她的爱情使你常常恋慕。
\par }{\Q \VS{20}我儿,你为何恋慕淫妇?
\par }{\Q 为何抱外女的胸怀?
\par }{\BB \par }{\Q \VS{21}因为,人所行的道都在耶和华眼前;
\par }{\Q 他也修平人一切的路。
\par }{\Q \VS{22}恶人必被自己的罪孽捉住;
\par }{\Q 他必被自己的罪恶如绳索缠绕。
\par }{\Q \VS{23}他因不受训诲就必死亡;
\par }{\Q 又因愚昧过甚,必走差了路。

\par }\Chap{6}{\SH 更多的警告
\par }{\Q \VerseOne{1}我儿,你若为朋友作保,
\par }{\Q 替外人击掌,
\par }{\Q \VS{2}你就被口中的话语缠住,
\par }{\Q 被嘴里的言语捉住。
\par }{\Q \VS{3}我儿,你既落在朋友手中,
\par }{\Q 就当这样行才可救自己:
\par }{\Q 你要自卑,去恳求你的朋友。
\par }{\Q \VS{4}不要容你的眼睛睡觉;
\par }{\Q 不要容你的眼皮打盹。
\par }{\Q \VS{5}要救自己,如鹿脱离{\ADD{猎户的}}手,
\par }{\Q 如鸟脱离捕鸟人的手。
\par }{\BB \par }{\Q \VS{6}懒惰人哪,
\par }{\Q 你去察看蚂蚁的动作就可得智慧。
\par }{\Q \VS{7}蚂蚁没有元帅,
\par }{\Q 没有官长,没有君王,
\par }{\Q \VS{8}尚且在夏天预备食物,
\par }{\Q 在收割时聚敛粮食。
\par }{\Q \VS{9}懒惰人哪,你要睡到几时呢?
\par }{\Q 你何时睡醒呢?
\par }{\Q \VS{10}再睡片时,打盹片时,
\par }{\Q 抱着手躺卧片时,
\par }{\Q \VS{11}你的贫穷就必如强盗速来,
\par }{\Q 你的缺乏仿佛拿兵器的人来到。
\par }{\BB \par }{\Q \VS{12}无赖的恶徒,
\par }{\Q 行动就用乖僻的口,
\par }{\Q \VS{13}用眼传神,
\par }{\Q 用脚示意,
\par }{\Q 用指点划,
\par }{\Q \VS{14}心中乖僻,
\par }{\Q 常设恶谋,
\par }{\Q 布散纷争。
\par }{\Q \VS{15}所以,灾难必忽然临到他身;
\par }{\Q 他必顷刻败坏,无法可治。
\par }{\BB \par }{\Q \VS{16}耶和华所恨恶的有六样,
\par }{\Q 连他心所憎恶的共有七样:
\par }{\Q \VS{17}就是高傲的眼,
\par }{\Q 撒谎的舌,
\par }{\Q 流无辜人血的手,
\par }{\Q \VS{18}图谋恶计的心,
\par }{\Q 飞跑行恶的脚,
\par }{\Q \VS{19}吐谎言的假见证,
\par }{\Q 并弟兄中布散纷争的人。
\par }{\SH 警告勿犯淫乱
\par }{\Q \VS{20}我儿,要谨守你父亲的诫命;
\par }{\Q 不可离弃你母亲的法则\FTNT{}{{\FR 6:20: }或译:指教},
\par }{\Q \VS{21}要常系在你心上,
\par }{\Q 挂在你项上。
\par }{\Q \VS{22}你行走,它必引导你;
\par }{\Q 你躺卧,它必保守你;
\par }{\Q 你睡醒,它必与你谈论。
\par }{\Q \VS{23}因为诫命是灯,法则\FTNT{}{{\FR 6:23: }或译:指教}是光,
\par }{\Q 训诲的责备是生命的道,
\par }{\Q \VS{24}能保你远离恶妇,
\par }{\Q 远离外女谄媚的舌头。
\par }{\Q \VS{25}你心中不要恋慕她的美色,
\par }{\Q 也不要被她眼皮勾引。
\par }{\Q \VS{26}因为,妓女{\ADD{能使人}}只剩一块饼;
\par }{\Q 淫妇猎取人宝贵的生命。
\par }{\Q \VS{27}人若怀里搋火,
\par }{\Q 衣服岂能不烧呢?
\par }{\Q \VS{28}人若在火炭上走,
\par }{\Q 脚岂能不烫呢?
\par }{\Q \VS{29}亲近邻舍之妻的,也是如此;
\par }{\Q 凡挨近她的,不免受罚。
\par }{\Q \VS{30}贼因饥饿偷窃充饥,
\par }{\Q 人不藐视他,
\par }{\Q \VS{31}若被找着,他必赔还七倍,
\par }{\Q 必将家中所有的尽都偿还。
\par }{\Q \VS{32}与妇人行淫的,便是无知;
\par }{\Q 行这事的,必丧掉生命。
\par }{\Q \VS{33}他必受伤损,必被凌辱;
\par }{\Q 他的羞耻不得涂抹。
\par }{\Q \VS{34}因为人的嫉恨成了烈怒,
\par }{\Q 报仇的时候决不留情。
\par }{\Q \VS{35}什么赎价,他都不顾;
\par }{\Q 你虽送许多礼物,他也不肯干休。

\par }\PoetryChap{7}{\Q \VerseOne{1}我儿,你要遵守我的言语,
\par }{\Q 将我的命令存记在心。
\par }{\Q \VS{2}遵守我的命令就得存活;
\par }{\Q 保守我的法则\FTNT{}{{\FR 7:2: }或译:指教},
\par }{\Q 好像保守眼中的瞳人,
\par }{\Q \VS{3}系在你指头上,
\par }{\Q 刻在你心版上。
\par }{\Q \VS{4}对智慧说:你是我的姊妹,
\par }{\Q 称呼聪明为你的亲人,
\par }{\Q \VS{5}她就保你远离淫妇,
\par }{\Q 远离说谄媚话的外女。
\par }{\SH 淫乱的妇人
\par }{\Q \VS{6}我曾在我房屋的窗户内,
\par }{\Q 从我窗棂之间往外观看:
\par }{\Q \VS{7}见愚蒙人内,少年人中,
\par }{\Q 分明有一个无知的少年人,
\par }{\Q \VS{8}从街上经过,走近淫妇的巷口,
\par }{\Q 直往通她家的路去,
\par }{\Q \VS{9}在黄昏,或晚上,
\par }{\Q 或半夜,或黑暗之中。
\par }{\BB \par }{\Q \VS{10}看哪,有一个妇人来迎接他,
\par }{\Q 是妓女的打扮,有诡诈的心思。
\par }{\Q \VS{11}这妇人喧嚷,不守约束,
\par }{\Q 在家里停不住脚,
\par }{\Q \VS{12}有时在街市上,有时在宽阔处,
\par }{\Q 或在各巷口蹲伏,
\par }{\Q \VS{13}拉住那少年人,与他亲嘴,
\par }{\Q 脸无羞耻对他说:
\par }{\Q \VS{14}平安祭在我这里,
\par }{\Q 今日才还了我所许的愿。
\par }{\Q \VS{15}因此,我出来迎接你,
\par }{\Q 恳切求见你的面,恰巧遇见了你。
\par }{\Q \VS{16}我已经用绣花毯子
\par }{\Q 和{\PN{埃及}}线织的花纹布铺了我的床。
\par }{\Q \VS{17}我又用没药、沉香、桂皮
\par }{\Q 薰了我的榻。
\par }{\Q \VS{18}你来,我们可以饱享爱情,直到早晨;
\par }{\Q 我们可以彼此亲爱欢乐。
\par }{\Q \VS{19}因为我丈夫不在家,出门行远路;
\par }{\Q \VS{20}他手拿银囊,必到月望才回家。
\par }{\BB \par }{\Q \VS{21}淫妇用许多巧言诱他随从,
\par }{\Q 用谄媚的嘴逼他同行。
\par }{\Q \VS{22}少年人立刻跟随她,好像牛往宰杀之地;
\par }{\Q 又像愚昧人带锁链去受刑罚,
\par }{\Q \VS{23}直等箭穿他的肝;
\par }{\Q 如同雀鸟急入网罗,却不知是自丧己命。
\par }{\BB \par }{\Q \VS{24}众子啊,现在要听从我,
\par }{\Q 留心听我口中的话。
\par }{\Q \VS{25}你的心不可偏向淫妇的道,
\par }{\Q 不要入她的迷途。
\par }{\Q \VS{26}因为,被她伤害仆倒的不少;
\par }{\Q 被她杀戮的而且甚多。
\par }{\Q \VS{27}她的家是在阴间之路,
\par }{\Q 下到死亡之宫。

\par }\Chap{8}{\SH 智慧颂
\par }{\Q \VerseOne{1}智慧岂不呼叫?
\par }{\Q 聪明岂不发声?
\par }{\Q \VS{2}她在道旁高处的顶上,
\par }{\Q 在十字路口站立,
\par }{\Q \VS{3}在城门旁,在城门口,
\par }{\Q 在城门洞,大声说:
\par }{\Q \VS{4}众人哪,我呼叫你们,
\par }{\Q 我向世人发声。
\par }{\Q \VS{5}说:愚蒙人哪,你们要会悟灵明;
\par }{\Q 愚昧人哪,你们当心里明白。
\par }{\Q \VS{6}你们当听,因我要说极美的话;
\par }{\Q 我张嘴要论正直的事。
\par }{\Q \VS{7}我的口要发出真理;
\par }{\Q 我的嘴憎恶邪恶。
\par }{\Q \VS{8}我口中的言语都是公义,
\par }{\Q 并无弯曲乖僻。
\par }{\Q \VS{9}有聪明的,以为明显,
\par }{\Q 得知识的,以为正直。
\par }{\Q \VS{10}你们当受我的教训,不受白银;
\par }{\Q 宁得知识,胜过黄金。
\par }{\BB \par }{\Q \VS{11}因为智慧比珍珠\FTNT{}{{\FR 8:11: }或译:红宝石}更美;
\par }{\Q 一切可喜爱的都不足与比较。
\par }{\Q \VS{12}我—智慧以灵明为居所,
\par }{\Q 又寻得知识和谋略。
\par }{\Q \VS{13}敬畏耶和华在乎恨恶邪恶;
\par }{\Q 那骄傲、狂妄,并恶道,
\par }{\Q 以及乖谬的口,都为我所恨恶。
\par }{\Q \VS{14}我有谋略和真知识;
\par }{\Q 我乃聪明,我有能力。
\par }{\Q \VS{15}帝王借我坐国位;
\par }{\Q 君王借我定公平。
\par }{\Q \VS{16}王子和首领,
\par }{\Q 世上一切的审判官,都是借我掌权。
\par }{\Q \VS{17}爱我的,我也爱他;
\par }{\Q 恳切寻求我的,必寻得见。
\par }{\Q \VS{18}丰富尊荣在我;
\par }{\Q 恒久的财并公义也在我。
\par }{\Q \VS{19}我的果实胜过黄金,强如精金;
\par }{\Q 我的出产超乎高银。
\par }{\Q \VS{20}我在公义的道上走,
\par }{\Q 在公平的路中行,
\par }{\Q \VS{21}使爱我的,承受货财,
\par }{\Q 并充满他们的府库。
\par }{\BB \par }{\Q \VS{22}在耶和华造化的起头,
\par }{\Q 在太初创造万物之先,就有了我。
\par }{\Q \VS{23}从亘古,从太初,
\par }{\Q 未有世界以前,我已被立。
\par }{\Q \VS{24}没有深渊,
\par }{\Q 没有大水的泉源,我已生出。
\par }{\Q \VS{25}大山未曾奠定,
\par }{\Q 小山未有之先,我已生出。
\par }{\Q \VS{26}耶和华还没有创造大地和田野,
\par }{\Q 并世上的土质,我已生出。
\par }{\Q \VS{27}他立高天,我在那里;
\par }{\Q 他在渊面的周围,划出圆圈。
\par }{\Q \VS{28}上使穹苍坚硬,
\par }{\Q 下使渊源稳固,
\par }{\Q \VS{29}为沧海定出界限,使水不越过他的命令,
\par }{\Q 立定大地的根基。
\par }{\Q \VS{30}那时,我在他那里为工师,
\par }{\Q 日日为他所喜爱,
\par }{\Q 常常在他面前踊跃,
\par }{\Q \VS{31}踊跃在他为人{\ADD{预备}}可住之地,
\par }{\Q 也喜悦住在世人之间。
\par }{\BB \par }{\Q \VS{32}众子啊,现在要听从我,
\par }{\Q 因为谨守我道的,便为有福。
\par }{\Q \VS{33}要听教训就得智慧,
\par }{\Q 不可弃绝。
\par }{\Q \VS{34}听从我、日日在我门口仰望、
\par }{\Q 在我门框旁边等候的,那人便为有福。
\par }{\Q \VS{35}因为寻得我的,就寻得生命,
\par }{\Q 也必蒙耶和华的恩惠。
\par }{\Q \VS{36}得罪我的,却害了自己的性命;
\par }{\Q 恨恶我的,都喜爱死亡。

\par }\Chap{9}{\SH 智慧和愚蠢
\par }{\Q \VerseOne{1}智慧建造房屋,
\par }{\Q 凿成七根柱子,
\par }{\Q \VS{2}宰杀牲畜,
\par }{\Q 调和旨酒,
\par }{\Q 设摆筵席;
\par }{\Q \VS{3}打发使女出去,
\par }{\Q 自己在城中至高处呼叫,
\par }{\Q \VS{4}说:谁是愚蒙人,可以转到这里来!
\par }{\Q 又对那无知的人说:
\par }{\Q \VS{5}你们来,吃我的饼,
\par }{\Q 喝我调和的酒。
\par }{\Q \VS{6}你们愚蒙人,要舍弃{\ADD{愚蒙}},
\par }{\Q 就得存活,并要走光明的道。
\par }{\BB \par }{\Q \VS{7}指斥亵慢人的,必受辱骂;
\par }{\Q 责备恶人的,必被玷污。
\par }{\Q \VS{8}不要责备亵慢人,恐怕他恨你;
\par }{\Q 要责备智慧人,他必爱你。
\par }{\Q \VS{9}{\ADD{教导}}智慧人,他就越发有智慧;
\par }{\Q 指示义人,他就增长学问。
\par }{\BB \par }{\Q \VS{10}敬畏耶和华是智慧的开端;
\par }{\Q 认识至圣者便是聪明。
\par }{\Q \VS{11}你借着我,日子必增多,
\par }{\Q 年岁也必加添。
\par }{\Q \VS{12}你若有智慧,是与自己有益;
\par }{\Q 你若亵慢,就必独自担当。
\par }{\BB \par }{\Q \VS{13}愚昧的妇人喧嚷;
\par }{\Q 她是愚蒙,一无所知。
\par }{\Q \VS{14}她坐在自己的家门口,
\par }{\Q 坐在城中高处的座位上,
\par }{\Q \VS{15}呼叫过路的,
\par }{\Q 就是直行其道的人,
\par }{\Q \VS{16}说:谁是愚蒙人,可以转到这里来!
\par }{\Q 又对那无知的人说:
\par }{\Q \VS{17}偷来的水是甜的,
\par }{\Q 暗{\ADD{吃的}}饼是好的。
\par }{\Q \VS{18}人却不知有阴魂在她那里;
\par }{\Q 她的客在阴间的深处。

\par }\Chap{10}{\SH 所罗门的箴言
\par }{\Q \VerseOne{1}{\PN{所罗门}}的箴言:
\par }{\Q 智慧之子使父亲欢乐;
\par }{\Q 愚昧之子叫母亲担忧。
\par }{\Q \VS{2}不义之财毫无益处;
\par }{\Q 惟有公义能救人脱离死亡。
\par }{\Q \VS{3}耶和华不使义人受饥饿;
\par }{\Q 恶人所欲的,他必推开。
\par }{\Q \VS{4}手懒的,要受贫穷;
\par }{\Q 手勤的,却要富足。
\par }{\Q \VS{5}夏天聚敛的,是智慧之子;
\par }{\Q 收割时沉睡的,是贻羞之子。
\par }{\Q \VS{6}福祉临到义人的头;
\par }{\Q 强暴蒙蔽恶人的口。
\par }{\Q \VS{7}义人的纪念被称赞;
\par }{\Q 恶人的名字必朽烂。
\par }{\Q \VS{8}心中智慧的,必受命令;
\par }{\Q 口里愚妄的,必致倾倒。
\par }{\Q \VS{9}行正直路的,步步安稳;
\par }{\Q 走弯曲道的,必致败露。
\par }{\Q \VS{10}以眼传神的,使人忧患;
\par }{\Q 口里愚妄的,必致倾倒。
\par }{\Q \VS{11}义人的口是生命的泉源;
\par }{\Q 强暴蒙蔽恶人的口。
\par }{\Q \VS{12}恨能挑启争端;
\par }{\Q 爱能遮掩一切过错。
\par }{\Q \VS{13}明哲人嘴里有智慧;
\par }{\Q 无知人背上受刑杖。
\par }{\Q \VS{14}智慧人积存知识;
\par }{\Q 愚妄人的口速致败坏。
\par }{\Q \VS{15}富户的财物是他的坚城;
\par }{\Q 穷人的贫乏是他的败坏。
\par }{\Q \VS{16}义人的勤劳致生;
\par }{\Q 恶人的进项致死\FTNT{}{{\FR 10:16: }死:原文是罪}。
\par }{\Q \VS{17}谨守训诲的,乃在生命的道上;
\par }{\Q 违弃责备的,便失迷了路。
\par }{\Q \VS{18}隐藏怨恨的,有说谎的嘴;
\par }{\Q 口出谗谤的,是愚妄的人。
\par }{\Q \VS{19}多言多语难免有过;
\par }{\Q 禁止嘴唇是有智慧。
\par }{\Q \VS{20}义人的舌乃{\ADD{似}}高银;
\par }{\Q 恶人的心所值无几。
\par }{\Q \VS{21}义人的口教养多人;
\par }{\Q 愚昧人因无知而死亡。
\par }{\Q \VS{22}耶和华所赐的福使人富足,
\par }{\Q 并不加上忧虑。
\par }{\Q \VS{23}愚妄人以行恶为戏耍;
\par }{\Q 明哲人却以智慧为乐。
\par }{\Q \VS{24}恶人所怕的,必临到他;
\par }{\Q 义人所愿的,必蒙应允。
\par }{\Q \VS{25}暴风一过,恶人归于无有;
\par }{\Q 义人的根基却是永久。
\par }{\Q \VS{26}懒惰人叫差他的人
\par }{\Q 如醋倒牙,如烟薰目。
\par }{\Q \VS{27}敬畏耶和华使人日子加多;
\par }{\Q 但恶人的年岁必被减少。
\par }{\Q \VS{28}义人的盼望{\ADD{必得}}喜乐;
\par }{\Q 恶人的指望必致灭没。
\par }{\Q \VS{29}耶和华的道是正直人的保障,
\par }{\Q 却成了作孽人的败坏。
\par }{\Q \VS{30}义人永不挪移;
\par }{\Q 恶人不得住在地上。
\par }{\Q \VS{31}义人的口滋生智慧;
\par }{\Q 乖谬的舌必被割断。
\par }{\Q \VS{32}义人的嘴能令人喜悦;
\par }{\Q 恶人的口{\ADD{说}}乖谬的话。

\par }\PoetryChap{11}{\Q \VerseOne{1}诡诈的天平为耶和华所憎恶;
\par }{\Q 公平的法码为他所喜悦。
\par }{\Q \VS{2}骄傲来,羞耻也来;
\par }{\Q 谦逊人却有智慧。
\par }{\Q \VS{3}正直人的纯正必引导自己;
\par }{\Q 奸诈人的乖僻必毁灭自己。
\par }{\Q \VS{4}发怒的日子资财无益;
\par }{\Q 惟有公义能救人脱离死亡。
\par }{\Q \VS{5}完全人的义必指引他的路;
\par }{\Q 但恶人必因自己的恶跌倒。
\par }{\Q \VS{6}正直人的义必拯救自己;
\par }{\Q 奸诈人必陷在自己的罪孽中。
\par }{\Q \VS{7}恶人一死,他的指望必灭绝;
\par }{\Q 罪人的盼望也必灭没。
\par }{\Q \VS{8}义人得脱离患难,
\par }{\Q 有恶人来代替他。
\par }{\Q \VS{9}不虔敬的人用口败坏邻舍;
\par }{\Q 义人却因知识得救。
\par }{\Q \VS{10}义人享福,合城喜乐;
\par }{\Q 恶人灭亡,人都欢呼。
\par }{\Q \VS{11}城因正直人祝福便高举,
\par }{\Q 却因邪恶人的口就倾覆。
\par }{\Q \VS{12}藐视邻舍的,毫无智慧;
\par }{\Q 明哲人却静默不言。
\par }{\Q \VS{13}往来传舌的,泄漏密事;
\par }{\Q 心中诚实的,遮隐事情。
\par }{\Q \VS{14}无智谋,民就败落;
\par }{\Q 谋士多,人便安居。
\par }{\Q \VS{15}为外人作保的,必受亏损;
\par }{\Q 恨恶击掌的,却得安稳。
\par }{\Q \VS{16}恩德的妇女得尊荣;
\par }{\Q 强暴的男子得资财。
\par }{\Q \VS{17}仁慈的人善待自己;
\par }{\Q 残忍的人扰害己身。
\par }{\Q \VS{18}恶人经营,得虚浮的工价;
\par }{\Q 撒义种的,得实在的果效。
\par }{\Q \VS{19}恒心为义的,{\ADD{必得}}生命;
\par }{\Q 追求邪恶的,{\ADD{必致}}死亡。
\par }{\Q \VS{20}心中乖僻的,为耶和华所憎恶;
\par }{\Q 行事完全的,为他所喜悦。
\par }{\Q \VS{21}恶人{\ADD{虽然}}连手,必不免受罚;
\par }{\Q 义人的后裔必得拯救。
\par }{\Q \VS{22}妇女美貌而无见识,
\par }{\Q {\ADD{如同}}金环带在猪鼻上。
\par }{\Q \VS{23}义人的心愿尽得好处;
\par }{\Q 恶人的指望致干忿怒。
\par }{\Q \VS{24}有施散的,却更增添;
\par }{\Q 有吝惜过度的,反致穷乏。
\par }{\Q \VS{25}好施舍的,必得丰裕;
\par }{\Q 滋润人的,必得滋润。
\par }{\Q \VS{26}屯粮不卖的,民必咒诅他;
\par }{\Q 情愿出卖的,人必为他祝福。
\par }{\Q \VS{27}恳切求善的,就求得恩惠;
\par }{\Q 惟独求恶的,恶必临到他身。
\par }{\Q \VS{28}倚仗自己财物的,必跌倒;
\par }{\Q 义人必发旺,如青叶。
\par }{\Q \VS{29}扰害己家的,必承受清风;
\par }{\Q 愚昧人必作慧心人的仆人。
\par }{\Q \VS{30}义人所结的果子就是生命树;
\par }{\Q 有智慧的,必能得人。
\par }{\Q \VS{31}看哪,义人在世尚且受报,
\par }{\Q 何况恶人和罪人呢?

\par }\PoetryChap{12}{\Q \VerseOne{1}喜爱管教的,就是喜爱知识;
\par }{\Q 恨恶责备的,却是畜类。
\par }{\Q \VS{2}善人必蒙耶和华的恩惠;
\par }{\Q 设诡计的人,耶和华必定他的罪。
\par }{\Q \VS{3}人靠恶行不能坚立;
\par }{\Q 义人的根必不动摇。
\par }{\Q \VS{4}才德的妇人是丈夫的冠冕;
\par }{\Q 贻羞的妇人如同朽烂在她丈夫的骨中。
\par }{\Q \VS{5}义人的思念是公平;
\par }{\Q 恶人的计谋是诡诈。
\par }{\Q \VS{6}恶人的言论是埋伏流人的血;
\par }{\Q 正直人的口必拯救人。
\par }{\Q \VS{7}恶人倾覆,归于无有;
\par }{\Q 义人的家必站得住。
\par }{\Q \VS{8}人必按自己的智慧被称赞;
\par }{\Q 心中乖谬的,必被藐视。
\par }{\Q \VS{9}被人轻贱,却有仆人,
\par }{\Q 强如自尊,缺少食物。
\par }{\Q \VS{10}义人顾惜他牲畜的命;
\par }{\Q 恶人的怜悯也是残忍。
\par }{\Q \VS{11}耕种自己田地的,必得饱食;
\par }{\Q 追随虚浮的,却是无知。
\par }{\Q \VS{12}恶人想得坏人的网罗;
\par }{\Q 义人的根得以结实。
\par }{\Q \VS{13}恶人嘴中的过错是自己的网罗;
\par }{\Q 但义人必脱离患难。
\par }{\Q \VS{14}人因口所结的果子,必饱得美福;
\par }{\Q 人手所做的,必为自己的报应。
\par }{\Q \VS{15}愚妄人所行的,在自己眼中看为正直;
\par }{\Q 惟智慧人肯听人的劝教。
\par }{\Q \VS{16}愚妄人的恼怒立时显露;
\par }{\Q 通达人能忍辱藏羞。
\par }{\Q \VS{17}说出真话的,显明公义;
\par }{\Q 作假见证的,显出诡诈。
\par }{\Q \VS{18}说话浮躁的,如刀刺人;
\par }{\Q 智慧人的舌头却为医人的良药。
\par }{\Q \VS{19}口吐真言,永远坚立;
\par }{\Q 舌说谎话,只存片时。
\par }{\Q \VS{20}图谋恶事的,心存诡诈;
\par }{\Q 劝人和睦的,便得喜乐。
\par }{\Q \VS{21}义人不遭灾害;恶人满受祸患。
\par }{\Q \VS{22}说谎言的嘴为耶和华所憎恶;
\par }{\Q 行事诚实的,为他所喜悦。
\par }{\Q \VS{23}通达人隐藏知识;
\par }{\Q 愚昧人的心彰显愚昧。
\par }{\Q \VS{24}殷勤人的手必掌权;
\par }{\Q 懒惰的人必服苦。
\par }{\Q \VS{25}人心忧虑,屈而不伸;
\par }{\Q 一句良言,使心欢乐。
\par }{\Q \VS{26}义人引导他的邻舍;
\par }{\Q 恶人的道叫人失迷。
\par }{\Q \VS{27}懒惰的人不烤打猎所得的;
\par }{\Q 殷勤的人却得宝贵的财物。
\par }{\Q \VS{28}在公义的道上有生命;
\par }{\Q 其路之中并无死亡。

\par }\PoetryChap{13}{\Q \VerseOne{1}智慧子{\ADD{听}}父亲的教训;
\par }{\Q 亵慢人不听责备。
\par }{\Q \VS{2}人因口所结的果子,必享美福;
\par }{\Q 奸诈人{\ADD{必遭}}强暴。
\par }{\Q \VS{3}谨守口的,得保生命;
\par }{\Q 大张嘴的,必致败亡。
\par }{\Q \VS{4}懒惰人羡慕,却无所得;
\par }{\Q 殷勤人必得丰裕。
\par }{\Q \VS{5}义人恨恶谎言;
\par }{\Q 恶人有臭名,且致惭愧。
\par }{\Q \VS{6}行为正直的,有公义保守;
\par }{\Q 犯罪的,被邪恶倾覆。
\par }{\Q \VS{7}假作富足的,却一无所有;
\par }{\Q 装作穷乏的,却广有财物。
\par }{\Q \VS{8}人的资财是他生命的赎价;
\par }{\Q 穷乏人却听不见威吓的话。
\par }{\Q \VS{9}义人的光明亮\FTNT{}{{\FR 13:9: }原文是欢喜};
\par }{\Q 恶人的灯要熄灭。
\par }{\Q \VS{10}骄傲只启争竞;
\par }{\Q 听劝言的,却有智慧。
\par }{\Q \VS{11}不劳而得之财必然消耗;
\par }{\Q 勤劳积蓄的,必见加增。
\par }{\Q \VS{12}所盼望的迟延未得,令人心忧;
\par }{\Q 所愿意的临到,却是生命树。
\par }{\Q \VS{13}藐视训言的,自取灭亡;
\par }{\Q 敬畏诫命的,必得善报。
\par }{\Q \VS{14}智慧人的法则\FTNT{}{{\FR 13:14: }或译:指教}是生命的泉源,
\par }{\Q 可以使人离开死亡的网罗。
\par }{\Q \VS{15}美好的聪明使人蒙恩;
\par }{\Q 奸诈人的道路崎岖难行。
\par }{\Q \VS{16}凡通达人都凭知识行事;
\par }{\Q 愚昧人张扬自己的愚昧。
\par }{\Q \VS{17}奸恶的使者必陷在祸患里;
\par }{\Q 忠信的使臣乃医人的良药。
\par }{\Q \VS{18}弃绝管教的,{\ADD{必致}}贫受辱;
\par }{\Q 领受责备的,必得尊荣。
\par }{\Q \VS{19}所欲的成就,心觉甘甜;
\par }{\Q 远离恶事,为愚昧人所憎恶。
\par }{\Q \VS{20}与智慧人同行的,必得智慧;
\par }{\Q 和愚昧人作伴的,必受亏损。
\par }{\Q \VS{21}祸患追赶罪人;
\par }{\Q 义人必得善报。
\par }{\Q \VS{22}善人给子孙遗留产业;
\par }{\Q 罪人为义人积存资财。
\par }{\Q \VS{23}穷人耕种多得粮食,
\par }{\Q 但因不义,有消灭的。
\par }{\Q \VS{24}不忍用杖打儿子的,是恨恶他;
\par }{\Q 疼爱儿子的,随时管教。
\par }{\Q \VS{25}义人吃得饱足;
\par }{\Q 恶人肚腹缺粮。

\par }\PoetryChap{14}{\Q \VerseOne{1}智慧妇人建立家室;
\par }{\Q 愚妄妇人亲手拆毁。
\par }{\Q \VS{2}行动正直的,敬畏耶和华;
\par }{\Q 行事乖僻的,却藐视他。
\par }{\Q \VS{3}愚妄人口中骄傲,如杖责打己身;
\par }{\Q 智慧人的嘴必保守自己。
\par }{\Q \VS{4}家里无牛,槽头干净;
\par }{\Q 土产加多乃凭牛力。
\par }{\Q \VS{5}诚实见证人不说谎话;
\par }{\Q 假见证人吐出谎言。
\par }{\Q \VS{6}亵慢人寻智慧,却寻不着;
\par }{\Q 聪明人易得知识。
\par }{\Q \VS{7}到愚昧人面前,
\par }{\Q 不见他嘴中有知识。
\par }{\Q \VS{8}通达人的智慧在乎明白己道;
\par }{\Q 愚昧人的愚妄乃是诡诈\FTNT{}{{\FR 14:8: }或译:自叹}。
\par }{\Q \VS{9}愚妄人犯罪,以为戏耍\FTNT{}{{\FR 14:9: }或译:赎愆祭愚弄愚妄人};
\par }{\Q 正直人互相喜悦。
\par }{\Q \VS{10}心中的苦楚,自己知道;
\par }{\Q 心里的喜乐,外人无干。
\par }{\Q \VS{11}奸恶人的房屋必倾倒;
\par }{\Q 正直人的帐棚必兴盛。
\par }{\Q \VS{12}有一条路,人以为正,
\par }{\Q 至终成为死亡之路。
\par }{\Q \VS{13}人在喜笑中,心也忧愁;
\par }{\Q 快乐至极就生愁苦。
\par }{\Q \VS{14}心中背道的,必满得自己的结果;
\par }{\Q 善人必从自己{\ADD{的行为得以知足}}。
\par }{\Q \VS{15}愚蒙人是话都信;
\par }{\Q 通达人步步谨慎。
\par }{\Q \VS{16}智慧人惧怕,就远离恶事;
\par }{\Q 愚妄人却狂傲自恃。
\par }{\Q \VS{17}轻易发怒的,行事愚妄;
\par }{\Q 设立诡计的,被人恨恶。
\par }{\Q \VS{18}愚蒙人得愚昧为产业;
\par }{\Q 通达人得知识为冠冕。
\par }{\Q \VS{19}坏人俯伏在善人面前;
\par }{\Q 恶人俯伏在义人门口。
\par }{\Q \VS{20}贫穷人连邻舍也恨他;
\par }{\Q 富足人朋友最多。
\par }{\Q \VS{21}藐视邻舍的,这人有罪;
\par }{\Q 怜悯贫穷的,这人有福。
\par }{\Q \VS{22}谋恶的,岂非走入迷途吗?
\par }{\Q 谋善的,{\ADD{必得}}慈爱和诚实。
\par }{\Q \VS{23}诸般勤劳都有益处;
\par }{\Q 嘴上多言乃致穷乏。
\par }{\Q \VS{24}智慧人的财为自己的冠冕;
\par }{\Q 愚妄人的愚昧终是愚昧。
\par }{\Q \VS{25}作真见证的,救人性命;
\par }{\Q 吐出谎言的,{\ADD{施行}}诡诈。
\par }{\Q \VS{26}敬畏耶和华的,大有倚靠;
\par }{\Q 他的儿女也有避难所。
\par }{\Q \VS{27}敬畏耶和华就是生命的泉源,
\par }{\Q 可以使人离开死亡的网罗。
\par }{\Q \VS{28}帝王荣耀在乎民多;
\par }{\Q 君王衰败在乎民少。
\par }{\Q \VS{29}不轻易发怒的,大有聪明;
\par }{\Q 性情暴躁的,大显愚妄。
\par }{\Q \VS{30}心中安静是肉体的生命;
\par }{\Q 嫉妒是骨中的朽烂。
\par }{\Q \VS{31}欺压贫寒的,是辱没造他的主;
\par }{\Q 怜悯穷乏的,乃是尊敬主。
\par }{\Q \VS{32}恶人在所行的恶上必被推倒;
\par }{\Q 义人临死,有所投靠。
\par }{\Q \VS{33}智慧存在聪明人心中;
\par }{\Q 愚昧人心里所存的,显而易见。
\par }{\Q \VS{34}公义使邦国高举;
\par }{\Q 罪恶是人民的羞辱。
\par }{\Q \VS{35}智慧的臣子蒙王恩惠;
\par }{\Q 贻羞的仆人遭其震怒。

\par }\PoetryChap{15}{\Q \VerseOne{1}回答柔和,使怒消退;
\par }{\Q 言语暴戾,触动怒气。
\par }{\Q \VS{2}智慧人的舌善发知识;
\par }{\Q 愚昧人的口吐出愚昧。
\par }{\Q \VS{3}耶和华的眼目无处不在;
\par }{\Q 恶人善人,他都鉴察。
\par }{\Q \VS{4}温良的舌是生命树;
\par }{\Q 乖谬的嘴使人心碎。
\par }{\Q \VS{5}愚妄人藐视父亲的管教;
\par }{\Q 领受责备的,得着见识。
\par }{\Q \VS{6}义人家中多有财宝;
\par }{\Q 恶人得利反受扰害。
\par }{\Q \VS{7}智慧人的嘴播扬知识;
\par }{\Q 愚昧人的心并不如此。
\par }{\Q \VS{8}恶人献祭,为耶和华所憎恶;
\par }{\Q 正直人祈祷,为他所喜悦。
\par }{\Q \VS{9}恶人的道路,为耶和华所憎恶;
\par }{\Q 追求公义的,为他所喜爱。
\par }{\Q \VS{10}舍弃正路的,必受严刑;
\par }{\Q 恨恶责备的,必致死亡。
\par }{\Q \VS{11}阴间和灭亡尚在耶和华眼前,
\par }{\Q 何况世人的心呢?
\par }{\Q \VS{12}亵慢人不爱受责备;
\par }{\Q 他也不就近智慧人。
\par }{\Q \VS{13}心中喜乐,面带笑容;
\par }{\Q 心里忧愁,灵被损伤。
\par }{\Q \VS{14}聪明人心求知识;
\par }{\Q 愚昧人口吃愚昧。
\par }{\Q \VS{15}困苦人的日子都是愁苦;
\par }{\Q 心中欢畅的,常享丰筵。
\par }{\Q \VS{16}少有财宝,敬畏耶和华,
\par }{\Q 强如多有财宝,烦乱不安。
\par }{\Q \VS{17}吃素菜,彼此相爱,
\par }{\Q 强如吃肥牛,彼此相恨。
\par }{\Q \VS{18}暴怒的人挑启争端;
\par }{\Q 忍怒的人止息纷争。
\par }{\Q \VS{19}懒惰人的道像荆棘的篱笆;
\par }{\Q 正直人的路是平坦的大道。
\par }{\Q \VS{20}智慧子使父亲喜乐;
\par }{\Q 愚昧人藐视母亲。
\par }{\Q \VS{21}无知的人以愚妄为乐;
\par }{\Q 聪明的人按正直而行。
\par }{\Q \VS{22}不先商议,所谋无效;
\par }{\Q 谋士众多,所谋乃成。
\par }{\Q \VS{23}口善应对,自觉喜乐;
\par }{\Q 话合其时,何等美好。
\par }{\Q \VS{24}智慧人从生命的道上升,
\par }{\Q 使他远离在下的阴间。
\par }{\Q \VS{25}耶和华必拆毁骄傲人的家,
\par }{\Q 却要立定寡妇的地界。
\par }{\Q \VS{26}恶谋为耶和华所憎恶;
\par }{\Q 良言乃为纯净。
\par }{\Q \VS{27}贪恋财利的,扰害己家;
\par }{\Q 恨恶贿赂的,必得存活。
\par }{\Q \VS{28}义人的心,思量如何回答;
\par }{\Q 恶人的口吐出恶言。
\par }{\Q \VS{29}耶和华远离恶人,
\par }{\Q 却听义人的祷告。
\par }{\Q \VS{30}眼有光,使心喜乐;
\par }{\Q 好信息,使骨滋润。
\par }{\Q \VS{31}听从生命责备的,
\par }{\Q 必常在智慧人中。
\par }{\Q \VS{32}弃绝管教的,轻看自己的生命;
\par }{\Q 听从责备的,却得智慧。
\par }{\Q \VS{33}敬畏耶和华是智慧的训诲;
\par }{\Q 尊荣以前,必有谦卑。

\par }\PoetryChap{16}{\Q \VerseOne{1}心中的谋算在乎人;
\par }{\Q 舌头的应对由于耶和华。
\par }{\Q \VS{2}人一切所行的,在自己眼中看为清洁;
\par }{\Q 惟有耶和华衡量人心。
\par }{\Q \VS{3}你所做的,要交托耶和华,
\par }{\Q 你所谋的,就必成立。
\par }{\Q \VS{4}耶和华所造的,各适其用;
\par }{\Q 就是恶人也为祸患的日子所造。
\par }{\Q \VS{5}凡心里骄傲的,为耶和华所憎恶;
\par }{\Q {\ADD{虽然}}连手,他必不免受罚。
\par }{\Q \VS{6}因怜悯诚实,罪孽得赎;
\par }{\Q 敬畏耶和华的,远离恶事。
\par }{\Q \VS{7}人所行的,若蒙耶和华喜悦,
\par }{\Q 耶和华也使他的仇敌与他和好。
\par }{\Q \VS{8}多有财利,行事不义,
\par }{\Q 不如少有财利,行事公义。
\par }{\Q \VS{9}人心筹算自己的道路;
\par }{\Q 惟耶和华指引他的脚步。
\par }{\Q \VS{10}王的嘴中有神语,
\par }{\Q 审判之时,他的口必不差错。
\par }{\Q \VS{11}公道的天平和秤都属耶和华;
\par }{\Q 囊中一切法码都为他所定。
\par }{\Q \VS{12}作恶,为王所憎恶,
\par }{\Q 因国位是靠公义坚立。
\par }{\Q \VS{13}公义的嘴为王所喜悦;
\par }{\Q 说正直话的,为王所喜爱。
\par }{\Q \VS{14}王的震怒如杀人的使者;
\par }{\Q 但智慧人能止息王怒。
\par }{\Q \VS{15}王的脸光使人有生命;
\par }{\Q 王的恩典好像春云时雨。
\par }{\Q \VS{16}得智慧胜似得金子;
\par }{\Q 选聪明强如选银子。
\par }{\Q \VS{17}正直人的道是远离恶事;
\par }{\Q 谨守己路的,是保全性命。
\par }{\Q \VS{18}骄傲在败坏以先;
\par }{\Q 狂心在跌倒之前。
\par }{\Q \VS{19}心里谦卑与穷乏人来往,
\par }{\Q 强如将掳物与骄傲人同分。
\par }{\Q \VS{20}谨守训言的,必得好处;
\par }{\Q 倚靠耶和华的,便为有福。
\par }{\Q \VS{21}心中有智慧,必称为通达人;
\par }{\Q 嘴中的甜言,加增人的学问。
\par }{\Q \VS{22}人有智慧就有生命的泉源;
\par }{\Q 愚昧人必被愚昧惩治。
\par }{\Q \VS{23}智慧人的心教训他的口,
\par }{\Q 又使他的嘴增长学问。
\par }{\Q \VS{24}良言{\ADD{如同}}蜂房,
\par }{\Q 使心觉甘甜,使骨得医治。
\par }{\Q \VS{25}有一条路,人以为正,
\par }{\Q 至终成为死亡之路。
\par }{\Q \VS{26}劳力人的胃口使他劳力,
\par }{\Q 因为他的口腹催逼他。
\par }{\Q \VS{27}匪徒图谋奸恶,
\par }{\Q 嘴上仿佛有烧焦的火。
\par }{\Q \VS{28}乖僻人播散纷争;
\par }{\Q 传舌的,离间密友。
\par }{\Q \VS{29}强暴人诱惑邻舍,
\par }{\Q 领他走不善之道。
\par }{\Q \VS{30}眼目紧合的,图谋乖僻;
\par }{\Q 嘴唇紧闭的,成就邪恶。
\par }{\Q \VS{31}白发是荣耀的冠冕,
\par }{\Q 在公义的道上必能得着。
\par }{\Q \VS{32}不轻易发怒的,胜过勇士;
\par }{\Q 治服己心的,强如取城。
\par }{\Q \VS{33}签放在怀里,
\par }{\Q 定事由耶和华。

\par }\PoetryChap{17}{\Q \VerseOne{1}设筵满屋,大家相争,
\par }{\Q 不如有块干饼,大家相安。
\par }{\Q \VS{2}仆人办事聪明,必管辖贻羞之子,
\par }{\Q 又在众子中同分产业。
\par }{\Q \VS{3}鼎为炼银,炉为炼金;
\par }{\Q 惟有耶和华熬炼人心。
\par }{\Q \VS{4}行恶的,留心听奸诈之言;
\par }{\Q 说谎的,侧耳听邪恶之语。
\par }{\Q \VS{5}戏笑穷人的,是辱没造他的主;
\par }{\Q 幸灾乐祸的,必不免受罚。
\par }{\Q \VS{6}子孙为老人的冠冕;
\par }{\Q 父亲是儿女的荣耀。
\par }{\Q \VS{7}愚顽人说美言本不相宜,
\par }{\Q 何况君王说谎话呢?
\par }{\Q \VS{8}贿赂在馈送的人眼中看为宝玉,
\par }{\Q 随处运动都得顺利。
\par }{\Q \VS{9}遮掩人过的,寻求人爱;
\par }{\Q 屡次挑错的,离间密友。
\par }{\Q \VS{10}一句责备话深入聪明人的心,
\par }{\Q 强如责打愚昧人一百下。
\par }{\Q \VS{11}恶人只寻背叛,
\par }{\Q 所以必有严厉的使者奉差攻击他。
\par }{\Q \VS{12}宁可遇见丢崽子的母熊,
\par }{\Q 不可遇见正行愚妄的愚昧人。
\par }{\Q \VS{13}以恶报善的,
\par }{\Q 祸患必不离他的家。
\par }{\Q \VS{14}纷争的起头{\ADD{如}}水放开,
\par }{\Q 所以,在争闹之先必当止息争竞。
\par }{\Q \VS{15}定恶人为义的,定义人为恶的,
\par }{\Q 这都为耶和华所憎恶。
\par }{\Q \VS{16}愚昧人既无聪明,
\par }{\Q 为何手拿价银买智慧呢?
\par }{\Q \VS{17}朋友乃时常亲爱,
\par }{\Q 弟兄为患难而生。
\par }{\Q \VS{18}在邻舍面前击掌作保
\par }{\Q 乃是无知的人。
\par }{\Q \VS{19}喜爱争竞的,是喜爱过犯;
\par }{\Q 高立家门的,乃自取败坏。
\par }{\Q \VS{20}心存邪僻的,寻不着好处;
\par }{\Q 舌弄是非的,陷在祸患中。
\par }{\Q \VS{21}生愚昧子的,必自愁苦;
\par }{\Q 愚顽人的父毫无喜乐。
\par }{\Q \VS{22}喜乐的心乃是良药;
\par }{\Q 忧伤的灵使骨枯干。
\par }{\Q \VS{23}恶人暗中受贿赂,
\par }{\Q 为要颠倒判断。
\par }{\Q \VS{24}明哲人眼前有智慧;
\par }{\Q 愚昧人眼望地极。
\par }{\Q \VS{25}愚昧子使父亲愁烦,
\par }{\Q 使母亲忧苦。
\par }{\Q \VS{26}刑罚义人为不善;
\par }{\Q 责打君子为不义。
\par }{\Q \VS{27}寡少言语的,有知识;
\par }{\Q 性情温良的,有聪明。
\par }{\Q \VS{28}愚昧人若静默不言也可算为智慧;
\par }{\Q 闭口不说也可算为聪明。

\par }\PoetryChap{18}{\Q \VerseOne{1}与众寡合的,独自寻求心愿,
\par }{\Q 并恼恨一切真智慧。
\par }{\Q \VS{2}愚昧人不喜爱明哲,
\par }{\Q 只喜爱显露心意。
\par }{\Q \VS{3}恶人来,藐视随来;
\par }{\Q 羞耻到,辱骂同到。
\par }{\Q \VS{4}人口中的言语如同深水;
\par }{\Q 智慧的泉源好像涌流的河水。
\par }{\Q \VS{5}瞻徇恶人的情面,
\par }{\Q 偏断义人的案件,都为不善。
\par }{\Q \VS{6}愚昧人张嘴启争端,
\par }{\Q 开口招鞭打。
\par }{\Q \VS{7}愚昧人的口自取败坏;
\par }{\Q 他的嘴是他生命的网罗。
\par }{\Q \VS{8}传舌人的言语如同美食,
\par }{\Q 深入人的心腹。
\par }{\Q \VS{9}做工懈怠的,
\par }{\Q 与浪费人为弟兄。
\par }{\Q \VS{10}耶和华的名是坚固台;
\par }{\Q 义人奔入便得安稳。
\par }{\Q \VS{11}富足人的财物是他的坚城,
\par }{\Q 在他心想,犹如高墙。
\par }{\Q \VS{12}败坏之先,人心骄傲;
\par }{\Q 尊荣以前,必有谦卑。
\par }{\Q \VS{13}未曾听完先回答的,
\par }{\Q 便是他的愚昧和羞辱。
\par }{\Q \VS{14}人有疾病,心能忍耐;
\par }{\Q 心灵忧伤,谁能承当呢?
\par }{\Q \VS{15}聪明人的心得知识;
\par }{\Q 智慧人的耳求知识。
\par }{\Q \VS{16}人的礼物为他开路,
\par }{\Q 引他到高位的人面前。
\par }{\Q \VS{17}先诉情由的,似乎有理;
\par }{\Q 但邻舍来到,就察出实情。
\par }{\Q \VS{18}掣签能止息争竞,
\par }{\Q 也能解散强胜的人。
\par }{\Q \VS{19}弟兄结怨,{\ADD{劝他和好}},比取坚固城还难;
\par }{\Q {\ADD{这样的}}争竞如同坚寨的门闩。
\par }{\Q \VS{20}人口中所结的果子,必充满肚腹;
\par }{\Q 他嘴所出的,必使他饱足。
\par }{\Q \VS{21}生死在舌头的权下,
\par }{\Q 喜爱它的,必吃它所结的果子。
\par }{\Q \VS{22}得着{\ADD{贤}}妻的,是得着好处,
\par }{\Q 也是蒙了耶和华的恩惠。
\par }{\Q \VS{23}贫穷人说哀求的话;
\par }{\Q 富足人用威吓的话回答。
\par }{\Q \VS{24}滥交朋友的,自取败坏;
\par }{\Q 但有一朋友比弟兄更亲密。

\par }\PoetryChap{19}{\Q \VerseOne{1}行为纯正的贫穷人
\par }{\Q 胜过乖谬愚妄的{\ADD{富足}}人。
\par }{\Q \VS{2}心无知识的,乃为不善;
\par }{\Q 脚步急快的,难免犯罪。
\par }{\Q \VS{3}人的愚昧倾败他的道;
\par }{\Q 他的心也抱怨耶和华。
\par }{\Q \VS{4}财物使朋友增多;
\par }{\Q 但穷人朋友远离。
\par }{\Q \VS{5}作假见证的,必不免受罚;
\par }{\Q 吐出谎言的,终不能逃脱。
\par }{\Q \VS{6}好施散的,有多人求他的恩情;
\par }{\Q 爱送礼的,人都为他的朋友。
\par }{\Q \VS{7}贫穷人,弟兄都恨他;
\par }{\Q 何况他的朋友,更远离他!
\par }{\Q 他用言语追随,他们却走了。
\par }{\Q \VS{8}得着智慧的,爱惜生命;
\par }{\Q 保守聪明的,必得好处。
\par }{\Q \VS{9}作假见证的,不免受罚;
\par }{\Q 吐出谎言的,也必灭亡。
\par }{\Q \VS{10}愚昧人宴乐度日是不合宜的;
\par }{\Q 何况仆人管辖王子呢?
\par }{\Q \VS{11}人有见识就不轻易发怒;
\par }{\Q 宽恕人的过失便是自己的荣耀。
\par }{\Q \VS{12}王的忿怒好像狮子吼叫;
\par }{\Q 他的恩典却如草上的甘露。
\par }{\Q \VS{13}愚昧的儿子是父亲的祸患;
\par }{\Q 妻子的争吵{\ADD{如雨}}连连滴漏。
\par }{\Q \VS{14}房屋钱财是祖宗所遗留的;
\par }{\Q 惟有贤慧的妻是耶和华所赐的。
\par }{\Q \VS{15}懒惰使人沉睡;
\par }{\Q 懈怠的人必受饥饿。
\par }{\Q \VS{16}谨守诫命的,保全生命;
\par }{\Q 轻忽己路的,必致死亡。
\par }{\Q \VS{17}怜悯贫穷的,就是借给耶和华;
\par }{\Q 他的善行,耶和华必偿还。
\par }{\Q \VS{18}趁有指望,管教你的儿子;
\par }{\Q 你的心不可任他死亡。
\par }{\Q \VS{19}暴怒的人必受刑罚;
\par }{\Q 你若救他,必须再救。
\par }{\Q \VS{20}你要听劝教,受训诲,
\par }{\Q 使你终久有智慧。
\par }{\Q \VS{21}人心多有计谋;
\par }{\Q 惟有耶和华的筹算才能立定。
\par }{\Q \VS{22}施行仁慈的,令人爱慕;
\par }{\Q 穷人强如说谎言的。
\par }{\Q \VS{23}敬畏耶和华的,{\ADD{得着}}生命;
\par }{\Q 他必恒久知足,不遭祸患。
\par }{\Q \VS{24}懒惰人放手在盘子里,
\par }{\Q 就是向口撤回,他也不肯。
\par }{\Q \VS{25}鞭打亵慢人,愚蒙人必长见识;
\par }{\Q 责备明哲人,他就明白知识。
\par }{\Q \VS{26}虐待父亲、撵出母亲的,
\par }{\Q 是贻羞致辱之子。
\par }{\Q \VS{27}我儿,不可听了教训
\par }{\Q 而又偏离知识的言语。
\par }{\Q \VS{28}匪徒作见证戏笑公平;
\par }{\Q 恶人的口吞下罪孽。
\par }{\Q \VS{29}刑罚是为亵慢人预备的;
\par }{\Q 鞭打是为愚昧人的背预备的。

\par }\PoetryChap{20}{\Q \VerseOne{1}酒能使人亵慢,浓酒使人喧嚷;
\par }{\Q 凡因酒错误的,就无智慧。
\par }{\Q \VS{2}王的威吓如同狮子吼叫;
\par }{\Q 惹动他怒的,是自害己命。
\par }{\Q \VS{3}远离纷争是人的尊荣;
\par }{\Q 愚妄人都爱争闹。
\par }{\Q \VS{4}懒惰人因冬寒不肯耕种,
\par }{\Q 到收割的时候,他必讨饭而无所得。
\par }{\Q \VS{5}人心怀藏谋略,好像深水,
\par }{\Q 惟明哲人才能汲引出来。
\par }{\Q \VS{6}人多述说自己的仁慈,
\par }{\Q 但忠信人谁能遇着呢?
\par }{\Q \VS{7}行为纯正的义人,
\par }{\Q 他的子孙是有福的!
\par }{\Q \VS{8}王坐在审判的位上,
\par }{\Q 以眼目驱散诸恶。
\par }{\Q \VS{9}谁能说,我洁净了我的心,
\par }{\Q 我脱净了我的罪?
\par }{\Q \VS{10}两样的法码,两样的升斗,
\par }{\Q 都为耶和华所憎恶。
\par }{\Q \VS{11}孩童的动作是清洁,是正直,
\par }{\Q 都显明他的本性。
\par }{\Q \VS{12}能听的耳,能看的眼,
\par }{\Q 都是耶和华所造的。
\par }{\Q \VS{13}不要贪睡,免致贫穷;
\par }{\Q 眼要睁开,你就吃饱。
\par }{\Q \VS{14}买物的说:不好,不好;
\par }{\Q 及至{\ADD{买}}去,他便自夸。
\par }{\Q \VS{15}有金子和许多珍珠\FTNT{}{{\FR 20:15: }或译:红宝石},
\par }{\Q 惟有知识的嘴乃为贵重的珍宝。
\par }{\Q \VS{16}谁为生人作保,就拿谁的衣服;
\par }{\Q 谁为外人{\ADD{作保}},谁就要承当。
\par }{\Q \VS{17}以虚谎而得的食物,人觉甘甜;
\par }{\Q 但后来,他的口必充满尘沙。
\par }{\Q \VS{18}计谋都凭筹算立定;
\par }{\Q 打仗要凭智谋。
\par }{\Q \VS{19}往来传舌的,泄漏密事;
\par }{\Q 大张嘴的,不可与他结交。
\par }{\Q \VS{20}咒骂父母的,他的灯必灭,
\par }{\Q 变为漆黑的黑暗。
\par }{\Q \VS{21}起初速得的产业,
\par }{\Q 终久却不为福。
\par }{\Q \VS{22}你不要说,我要以恶报恶;
\par }{\Q 要等候耶和华,他必拯救你。
\par }{\Q \VS{23}两样的法码为耶和华所憎恶;
\par }{\Q 诡诈的天平也为不善。
\par }{\Q \VS{24}人的脚步为耶和华所定;
\par }{\Q 人岂能明白自己的路呢?
\par }{\Q \VS{25}人冒失说,{\ADD{这是}}圣物,
\par }{\Q 许愿之后才查问,就是自陷网罗。
\par }{\Q \VS{26}智慧的王簸散恶人,
\par }{\Q 用碌碡滚轧他们。
\par }{\Q \VS{27}人的灵是耶和华的灯,
\par }{\Q 鉴察人的心腹。
\par }{\Q \VS{28}王因仁慈和诚实得以保全他的国位,
\par }{\Q 也因仁慈立稳。
\par }{\Q \VS{29}强壮乃少年人的荣耀;
\par }{\Q 白发为老年人的尊荣。
\par }{\Q \VS{30}鞭伤除净人的罪恶;
\par }{\Q 责打{\ADD{能入}}人的心腹。

\par }\PoetryChap{21}{\Q \VerseOne{1}王的心在耶和华手中,
\par }{\Q 好像陇沟的水随意流转。
\par }{\Q \VS{2}人所行的,在自己眼中都看为正;
\par }{\Q 惟有耶和华衡量人心。
\par }{\Q \VS{3}行仁义公平
\par }{\Q 比献祭更蒙耶和华悦纳。
\par }{\Q \VS{4}恶人发达\FTNT{}{{\FR 21:4: }发达:原文是灯},眼高心傲,
\par }{\Q 这乃是罪。
\par }{\Q \VS{5}殷勤筹划的,足致丰裕;
\par }{\Q 行事急躁的,都必缺乏。
\par }{\Q \VS{6}用诡诈之舌求财的,就是自己取死;
\par }{\Q 所得之财乃是吹来吹去的浮云。
\par }{\Q \VS{7}恶人的强暴必将自己扫除,
\par }{\Q 因他们不肯按公平行事。
\par }{\Q \VS{8}负罪之人的路甚是弯曲;
\par }{\Q 至于清洁的人,他所行的乃是正直。
\par }{\Q \VS{9}宁可住在房顶的角上,
\par }{\Q 不在宽阔的房屋与争吵的妇人同住。
\par }{\Q \VS{10}恶人的心乐人受祸;
\par }{\Q 他眼并不怜恤邻舍。
\par }{\Q \VS{11}亵慢的人受刑罚,愚蒙的人就得智慧;
\par }{\Q 智慧人受训诲,便得知识。
\par }{\Q \VS{12}义人思想恶人的家,
\par }{\Q 知道恶人倾倒,必致灭亡。
\par }{\Q \VS{13}塞耳不听穷人哀求的,
\par }{\Q 他将来呼吁也不蒙应允。
\par }{\Q \VS{14}暗中送的礼物挽回怒气;
\par }{\Q 怀中搋的贿赂止息暴怒。
\par }{\Q \VS{15}秉公行义使义人喜乐,
\par }{\Q 使作孽的人败坏。
\par }{\Q \VS{16}迷离通达道路的,
\par }{\Q 必住在阴魂的会中。
\par }{\Q \VS{17}爱宴乐的,必致穷乏;
\par }{\Q 好酒,爱膏油的,必不富足。
\par }{\Q \VS{18}恶人作了义人的赎价;
\par }{\Q 奸诈人代替正直人。
\par }{\Q \VS{19}宁可住在旷野,
\par }{\Q 不与争吵使气的妇人同住。
\par }{\Q \VS{20}智慧人家中{\ADD{积蓄}}宝物膏油;
\par }{\Q 愚昧人随得来随吞下。
\par }{\Q \VS{21}追求公义仁慈的,
\par }{\Q 就寻得生命、公义,和尊荣。
\par }{\Q \VS{22}智慧人爬上勇士的城墙,
\par }{\Q 倾覆他所倚靠的坚垒。
\par }{\Q \VS{23}谨守口与舌的,
\par }{\Q 就保守自己免受灾难。
\par }{\Q \VS{24}心骄气傲的人名叫亵慢;
\par }{\Q 他行事狂妄,都出于骄傲。
\par }{\Q \VS{25}懒惰人的心愿将他杀害,
\par }{\Q 因为他手不肯做工。
\par }{\Q \VS{26}有终日贪得无厌的;
\par }{\Q 义人施舍而不吝惜。
\par }{\Q \VS{27}恶人的祭物是可憎的;
\par }{\Q 何况他存恶意来献呢?
\par }{\Q \VS{28}作假见证的必灭亡;
\par }{\Q 惟有听真情而言的,其言长存。
\par }{\Q \VS{29}恶人脸无羞耻;
\par }{\Q 正直人行事坚定。
\par }{\Q \VS{30}没有人能以智慧、聪明、
\par }{\Q 谋略敌挡耶和华。
\par }{\Q \VS{31}马是为打仗之日预备的;
\par }{\Q 得胜乃在乎耶和华。

\par }\PoetryChap{22}{\Q \VerseOne{1}美名胜过大财;
\par }{\Q 恩宠强如金银。
\par }{\Q \VS{2}富户穷人在世相遇,
\par }{\Q 都为耶和华所造。
\par }{\Q \VS{3}通达人见祸藏躲;
\par }{\Q 愚蒙人前往受害。
\par }{\Q \VS{4}敬畏耶和华心存谦卑,
\par }{\Q 就得富有、尊荣、生命为赏赐。
\par }{\Q \VS{5}乖僻人的路上有荆棘和网罗;
\par }{\Q 保守自己生命的,必要远离。
\par }{\Q \VS{6}教养孩童,使他走当行的道,
\par }{\Q 就是到老他也不偏离。
\par }{\Q \VS{7}富户管辖穷人;
\par }{\Q 欠债的是债主的仆人。
\par }{\Q \VS{8}撒罪孽的,必收灾祸;
\par }{\Q 他逞怒的杖也必废掉。
\par }{\Q \VS{9}眼目慈善的,就必蒙福,
\par }{\Q 因他将食物分给穷人。
\par }{\Q \VS{10}赶出亵慢人,争端就消除;
\par }{\Q 纷争和羞辱也必止息。
\par }{\Q \VS{11}喜爱清心的人因他嘴上的恩言,
\par }{\Q 王必与他为友。
\par }{\Q \VS{12}耶和华的眼目眷顾聪明人,
\par }{\Q 却倾败奸诈人的言语。
\par }{\Q \VS{13}懒惰人说:外头有狮子;
\par }{\Q 我在街上就必被杀。
\par }{\Q \VS{14}淫妇的口为深坑;
\par }{\Q 耶和华所憎恶的,必陷在其中。
\par }{\Q \VS{15}愚蒙迷住孩童的心,
\par }{\Q 用管教的杖可以远远赶除。
\par }{\Q \VS{16}欺压贫穷为要利己的,
\par }{\Q 并送{\ADD{礼}}与富户的,都必缺乏。
\par }{\SH 智言三十则
\par }{\Q \VS{17}你须侧耳听受智慧人的言语,
\par }{\Q 留心领会我的知识。
\par }{\Q \VS{18}你若心中存记,
\par }{\Q 嘴上咬定,这便为美。
\par }{\Q \VS{19}我今日以此特特指教你,
\par }{\Q 为要使你倚靠耶和华。
\par }{\Q \VS{20}谋略和知识的美事,
\par }{\Q 我岂没有写给你吗?
\par }{\Q \VS{21}要使你知道真言的实理,
\par }{\Q 你好将真言回复那打发你来的人。
\par }{\BB \par }{\Q \VS{22}贫穷人,你不可因他贫穷就抢夺他的物,
\par }{\Q 也不可在城门口欺压困苦人;
\par }{\Q \VS{23}因耶和华必为他辨屈;
\par }{\Q 抢夺他的,耶和华必夺取那人的命。
\par }{\BB \par }{\Q \VS{24}好生气的人,不可与他结交;
\par }{\Q 暴怒的人,不可与他来往;
\par }{\Q \VS{25}恐怕你效法他的行为,
\par }{\Q 自己就陷在网罗里。
\par }{\BB \par }{\Q \VS{26}不要与人击掌,
\par }{\Q 不要为欠债的作保。
\par }{\Q \VS{27}你若没有什么偿还,
\par }{\Q 何必使人夺去你睡卧的床呢?
\par }{\Q \VS{28}你先祖所立的地界,
\par }{\Q 你不可挪移。
\par }{\BB \par }{\Q \VS{29}你看见办事殷勤的人吗?
\par }{\Q 他必站在君王面前,
\par }{\Q 必不站在下贱人面前。

\par }\PoetryChap{23}{\Q \VerseOne{1}你若与官长坐席,
\par }{\Q 要留意在你面前的是谁。
\par }{\Q \VS{2}你若是贪食的,
\par }{\Q 就当拿刀放在喉咙上。
\par }{\Q \VS{3}不可贪恋他的美食,
\par }{\Q 因为是哄人的食物。
\par }{\BB \par }{\Q \VS{4}不要劳碌求富,
\par }{\Q 休仗自己的聪明。
\par }{\Q \VS{5}你岂要定睛在虚无的{\ADD{钱财}}上吗?
\par }{\Q 因{\ADD{钱财}}必长翅膀,如鹰向天飞去。
\par }{\Q \VS{6}不要吃恶眼人的饭,
\par }{\Q 也不要贪他的美味;
\par }{\Q \VS{7}因为他心怎样思量,
\par }{\Q 他为人就是怎样。
\par }{\Q 他虽对你说,请吃,请喝,
\par }{\Q 他的心却与你相背。
\par }{\Q \VS{8}你所吃的那点食物必吐出来;
\par }{\Q 你所说的甘美言语也必落空。
\par }{\BB \par }{\Q \VS{9}你不要说话给愚昧人听,
\par }{\Q 因他必藐视你智慧的言语。
\par }{\BB \par }{\Q \VS{10}不可挪移古时的地界,
\par }{\Q 也不可侵入孤儿的田地;
\par }{\Q \VS{11}因他们的救赎主大有能力,
\par }{\Q 他必向你为他们辨屈。
\par }{\BB \par }{\Q \VS{12}你要留心领受训诲,
\par }{\Q 侧耳听从知识的言语。
\par }{\BB \par }{\Q \VS{13}不可不管教孩童;
\par }{\Q 你用杖打他,他必不至于死。
\par }{\Q \VS{14}你要用杖打他,
\par }{\Q 就可以救他的灵魂免下阴间。
\par }{\BB \par }{\Q \VS{15}我儿,你心若存智慧,
\par }{\Q 我的心也甚欢喜。
\par }{\Q \VS{16}你的嘴若说正直话,
\par }{\Q 我的心肠也必快乐。
\par }{\BB \par }{\Q \VS{17}你心中不要嫉妒罪人,
\par }{\Q 只要终日敬畏耶和华;
\par }{\Q \VS{18}因为至终必有善报,
\par }{\Q 你的指望也不致断绝。
\par }{\BB \par }{\Q \VS{19}我儿,你当听,当存智慧,
\par }{\Q 好在正道上引导你的心。
\par }{\Q \VS{20}好饮酒的,好吃肉的,
\par }{\Q 不要与他们来往;
\par }{\Q \VS{21}因为好酒贪食的,必致贫穷;
\par }{\Q 好睡觉的,必穿破烂衣服。
\par }{\BB \par }{\Q \VS{22}你要听从生你的父亲;
\par }{\Q 你母亲老了,也不可藐视她。
\par }{\Q \VS{23}你当买真理;
\par }{\Q 就是智慧、训诲,和聪明也都不可卖。
\par }{\Q \VS{24}义人的父亲必大得快乐;
\par }{\Q 人生智慧的儿子,必因他欢喜。
\par }{\Q \VS{25}你要使父母欢喜,
\par }{\Q 使生你的快乐。
\par }{\BB \par }{\Q \VS{26}我儿,要将你的心归我;
\par }{\Q 你的眼目也要喜悦我的道路。
\par }{\Q \VS{27}妓女是深坑;
\par }{\Q 外女是窄阱。
\par }{\Q \VS{28}她埋伏好像强盗;
\par }{\Q 她使人中多有奸诈的。
\par }{\BB \par }{\Q \VS{29}谁有祸患?谁有忧愁?
\par }{\Q 谁有争斗?谁有哀叹\FTNT{}{{\FR 23:29: }或译:怨言}?
\par }{\Q 谁无故受伤?谁眼目红赤?
\par }{\Q \VS{30}就是那流连饮酒、
\par }{\Q 常去寻找调和酒的人。
\par }{\Q \VS{31-32}酒发红,在杯中闪烁,
\par }{\Q 你不可观看,
\par }{\Q {\ADD{虽然}}下咽舒畅,
\par }{\Q 终久是咬你如蛇,刺你如毒蛇。
\par }{\Q \VS{33}你眼必看见异怪的事\FTNT{}{{\FR 23:33: }或译:淫妇};
\par }{\Q 你心必发出乖谬的话。
\par }{\Q \VS{34}你必像躺在海中,
\par }{\Q 或像卧在桅杆上。
\par }{\Q \VS{35}{\ADD{你必}}说:人打我,我却未受伤;
\par }{\Q 人鞭打我,我竟不觉得。
\par }{\Q 我几时清醒,我仍去寻酒。

\par }\PoetryChap{24}{\Q \VerseOne{1}你不要嫉妒恶人,
\par }{\Q 也不要起意与他们相处;
\par }{\Q \VS{2}因为,他们的心图谋强暴,
\par }{\Q 他们的口谈论奸恶。
\par }{\BB \par }{\Q \VS{3}房屋因智慧建造,
\par }{\Q 又因聪明立稳;
\par }{\Q \VS{4}其中因知识充满各样美好宝贵的财物。
\par }{\BB \par }{\Q \VS{5}智慧人大有能力;
\par }{\Q 有知识的人力上加力。
\par }{\Q \VS{6}你去打仗,要凭智谋;
\par }{\Q 谋士众多,人便得胜。
\par }{\BB \par }{\Q \VS{7}智慧极高,非愚昧人所能及,
\par }{\Q 所以在城门内不敢开口。
\par }{\BB \par }{\Q \VS{8}设计作恶的,
\par }{\Q 必称为奸人。
\par }{\Q \VS{9}愚妄人的思念乃是罪恶;
\par }{\Q 亵慢者为人所憎恶。
\par }{\BB \par }{\Q \VS{10}你在患难之日若胆怯,
\par }{\Q 你的力量就微小。
\par }{\BB \par }{\Q \VS{11}人被拉到死地,你要解救;
\par }{\Q 人将被杀,你须拦阻。
\par }{\Q \VS{12}你若说:这事我未曾知道,
\par }{\Q 那衡量人心的岂不明白吗?
\par }{\Q 保守你命的岂不知道吗?
\par }{\Q 他岂不按各人所行的报应各人吗?
\par }{\BB \par }{\Q \VS{13}我儿,你要吃蜜,因为是好的;
\par }{\Q 吃蜂房下滴的蜜便觉甘甜。
\par }{\Q \VS{14}你心得了智慧,也必觉得如此。
\par }{\Q 你若找着,至终必有善报;
\par }{\Q 你的指望也不致断绝。
\par }{\BB \par }{\Q \VS{15}你这恶人,不要埋伏攻击义人的家;
\par }{\Q 不要毁坏他安居之所。
\par }{\Q \VS{16}因为,义人虽七次跌倒,仍必兴起;
\par }{\Q 恶人却被祸患倾倒。
\par }{\BB \par }{\Q \VS{17}你仇敌跌倒,你不要欢喜;
\par }{\Q 他倾倒,你心不要快乐;
\par }{\Q \VS{18}恐怕耶和华看见就不喜悦,
\par }{\Q 将怒气从仇敌身上转过来。
\par }{\BB \par }{\Q \VS{19}不要为作恶的心怀不平,
\par }{\Q 也不要嫉妒恶人;
\par }{\Q \VS{20}因为,恶人终不得善报;
\par }{\Q 恶人的灯也必熄灭。
\par }{\BB \par }{\Q \VS{21}我儿,你要敬畏耶和华与君王,
\par }{\Q 不要与反复无常的人结交,
\par }{\Q \VS{22}因为他们的灾难必忽然而起。
\par }{\Q 耶和华与君王所施行的毁灭,
\par }{\Q 谁能知道呢?
\par }{\SH 其他智言
\par }{\Q \VS{23}以下也是智慧人的{\ADD{箴言}}:
\par }{\BB \par }{\Q 审判时看人情面是不好的。
\par }{\Q \VS{24}对恶人说「你是义人」的,
\par }{\Q 这人万民必咒诅,列邦必憎恶。
\par }{\Q \VS{25}责备{\ADD{恶人的}},必得喜悦;
\par }{\Q 美好的福也必临到他。
\par }{\Q \VS{26}应对正直的,犹如与人亲嘴。
\par }{\BB \par }{\Q \VS{27}你要在外头预备工料,
\par }{\Q 在田间办理整齐,
\par }{\Q 然后建造房屋。
\par }{\BB \par }{\Q \VS{28}不可无故作见证陷害邻舍,
\par }{\Q 也不可用嘴欺骗人。
\par }{\Q \VS{29}不可说:人怎样待我,我也怎样待他;
\par }{\Q 我必照他所行的报复他。
\par }{\BB \par }{\Q \VS{30}我经过懒惰人的田地、
\par }{\Q 无知人的葡萄园,
\par }{\Q \VS{31}荆棘长满了地皮,
\par }{\Q 刺草遮盖了田面,
\par }{\Q 石墙也坍塌了。
\par }{\Q \VS{32}我看见就留心思想;
\par }{\Q 我看着就领了训诲。
\par }{\Q \VS{33}再睡片时,打盹片时,
\par }{\Q 抱着手躺卧片时,
\par }{\Q \VS{34}你的贫穷就必如强盗速来,
\par }{\Q 你的缺乏仿佛拿兵器的人来到。

\par }\Chap{25}{\SH 所罗门的另一些箴言
\par }{\Q \VerseOne{1}以下也是{\PN{所罗门}}的箴言,是{\PN{犹大}}王{\PN{希西家}}的人所誊录的。
\par }{\Q \VS{2}将事隐秘乃 神的荣耀;
\par }{\Q 将事察清乃君王的荣耀。
\par }{\Q \VS{3}天之高,地之厚,
\par }{\Q 君王之心也测不透。
\par }{\Q \VS{4}除去银子的渣滓就有{\ADD{银子}}出来,
\par }{\Q 银匠能以做器皿。
\par }{\Q \VS{5}除去王面前的恶人,
\par }{\Q 国位就靠公义坚立。
\par }{\Q \VS{6}不要在王面前妄自尊大;
\par }{\Q 不要在大人的位上站立。
\par }{\Q \VS{7}宁可有人说:请你上来,
\par }{\Q 强如在你觐见的王子面前叫你退下。
\par }{\BB \par }{\Q \VS{8}不要冒失出去与人争竞,
\par }{\Q 免得至终被他羞辱,
\par }{\Q 你就{\ADD{不知道}}怎样行了。
\par }{\Q \VS{9}你与邻舍争讼,
\par }{\Q 要与他一人辩论,
\par }{\Q 不可泄漏人的密事,
\par }{\Q \VS{10}恐怕听见的人骂你,
\par }{\Q 你的臭名就难以脱离。
\par }{\BB \par }{\Q \VS{11}一句话说得合宜,
\par }{\Q 就如金苹果在银网子里。
\par }{\Q \VS{12}智慧人的劝戒,在顺从的人耳中,
\par }{\Q 好像金耳环和精金的妆饰。
\par }{\Q \VS{13}忠信的使者叫差他的人心里舒畅,
\par }{\Q 就如在收割时有冰雪的凉气。
\par }{\Q \VS{14}空夸赠送礼物的,
\par }{\Q 好像无雨的风云。
\par }{\BB \par }{\Q \VS{15}恒常忍耐可以劝动君王;
\par }{\Q 柔和的舌头能折断骨头。
\par }{\Q \VS{16}你得了蜜吗?只可吃够而已,
\par }{\Q 恐怕你过饱就呕吐出来。
\par }{\Q \VS{17}你的脚要少进邻舍的家,
\par }{\Q 恐怕他厌烦你,恨恶你。
\par }{\Q \VS{18}作假见证陷害邻舍的,
\par }{\Q 就是大槌,是利刀,是快箭。
\par }{\Q \VS{19}患难时倚靠不忠诚的人,
\par }{\Q 好像破坏的牙,错骨缝的脚。
\par }{\Q \VS{20}对伤心的人唱歌,
\par }{\Q 就如冷天脱衣服,
\par }{\Q 又如硷上倒醋。
\par }{\Q \VS{21}你的仇敌若饿了,就给他饭吃;
\par }{\Q 若渴了,就给他水喝;
\par }{\Q \VS{22}因为,你{\ADD{这样行}}就是把炭火堆在他的头上;
\par }{\Q 耶和华也必赏赐你。
\par }{\Q \VS{23}北风生雨,
\par }{\Q 谗谤人的舌头也生怒容。
\par }{\Q \VS{24}宁可住在房顶的角上,
\par }{\Q 不在宽阔的房屋与争吵的妇人同住。
\par }{\Q \VS{25}有好消息从远方来,
\par }{\Q 就如拿凉水给口渴的人喝。
\par }{\Q \VS{26}义人在恶人面前退缩,
\par }{\Q 好像趟浑之泉,弄浊之井。
\par }{\Q \VS{27}吃蜜过多是不好的;
\par }{\Q 考究自己的荣耀也是可厌的。
\par }{\Q \VS{28}人不制伏自己的心,
\par }{\Q 好像毁坏的城邑没有墙垣。

\par }\PoetryChap{26}{\Q \VerseOne{1}夏天落雪,收割时下雨,都不相宜;
\par }{\Q 愚昧人得尊荣也是如此。
\par }{\Q \VS{2}麻雀往来,燕子翻飞;
\par }{\Q 这样,无故的咒诅也必不临到。
\par }{\Q \VS{3}鞭子是为打马,辔头是为勒驴;
\par }{\Q 刑杖是为打愚昧人的背。
\par }{\Q \VS{4}不要照愚昧人的愚妄话回答他,
\par }{\Q 恐怕你与他一样。
\par }{\Q \VS{5}要照愚昧人的愚妄话回答他,
\par }{\Q 免得他自以为有智慧。
\par }{\Q \VS{6}借愚昧人手寄信的,
\par }{\Q 是砍断{\ADD{自己的}}脚,自受\FTNT{}{{\FR 26:6: }原文是:喝}损害。
\par }{\Q \VS{7}瘸子的脚空存无用;
\par }{\Q 箴言在愚昧人的口中也是如此。
\par }{\Q \VS{8}将尊荣给愚昧人的,
\par }{\Q 好像人把石子包在机弦里。
\par }{\Q \VS{9}箴言在愚昧人的口中,
\par }{\Q 好像荆棘刺入醉汉的手。
\par }{\Q \VS{10}雇愚昧人的,与雇过路人的,
\par }{\Q 就像射伤众人的弓箭手。
\par }{\Q \VS{11}愚昧人行愚妄事,行了又行,
\par }{\Q 就如狗转过来吃它所吐的。
\par }{\Q \VS{12}你见自以为有智慧的人吗?
\par }{\Q 愚昧人比他更有指望。
\par }{\Q \VS{13}懒惰人说:道上有猛狮,
\par }{\Q 街上有壮狮。
\par }{\Q \VS{14}门在枢纽转动,
\par }{\Q 懒惰人在床上也是如此。
\par }{\Q \VS{15}懒惰人放手在盘子里,
\par }{\Q 就是向口撤回也以为劳乏。
\par }{\Q \VS{16}懒惰人看自己比七个善于应对的人更有智慧。
\par }{\Q \VS{17}过路被事激动,管理不干己的争竞,
\par }{\Q 好像人揪住狗耳。
\par }{\Q \VS{18-19}人欺凌邻舍,却说:
\par }{\Q 我岂不是戏耍吗?
\par }{\Q 他就像疯狂的人抛掷火把、利箭,
\par }{\Q 与杀人的兵器\FTNT{}{{\FR 26:18-19: }原文是死亡}。
\par }{\Q \VS{20}火缺了柴就必熄灭;
\par }{\Q 无人传舌,争竞便止息。
\par }{\Q \VS{21}好争竞的人煽惑争端,
\par }{\Q 就如余火加炭,火上加柴一样。
\par }{\Q \VS{22}传舌人的言语,如同美食,
\par }{\Q 深入人的心腹。
\par }{\Q \VS{23}火热的嘴,奸恶的心,
\par }{\Q 好像银渣包的瓦器。
\par }{\Q \VS{24}怨恨人的,用嘴粉饰,
\par }{\Q 心里却藏着诡诈;
\par }{\Q \VS{25}他用甜言蜜语,你不可信他,
\par }{\Q 因为他心中有七样可憎恶的。
\par }{\Q \VS{26}他虽用诡诈遮掩自己的怨恨,
\par }{\Q 他的邪恶必在会中显露。
\par }{\Q \VS{27}挖陷坑的,自己必掉在其中;
\par }{\Q 滚石头的,石头必反滚在他身上。
\par }{\Q \VS{28}虚谎的舌恨他所压伤的人;
\par }{\Q 谄媚的口败坏人的事。

\par }\PoetryChap{27}{\Q \VerseOne{1}不要为明日自夸,
\par }{\Q 因为一日要生何事,你尚且不能知道。
\par }{\Q \VS{2}要别人夸奖你,不可用口自夸;
\par }{\Q 等外人称赞你,不可用嘴自称。
\par }{\Q \VS{3}石头重,沙土沉,
\par }{\Q 愚妄人的恼怒比这两样更重。
\par }{\Q \VS{4}忿怒为残忍,怒气为狂澜,
\par }{\Q 惟有嫉妒,谁能敌得住呢?
\par }{\Q \VS{5}当面的责备强如背地的爱情。
\par }{\Q \VS{6}朋友加的伤痕出于忠诚;
\par }{\Q 仇敌连连亲嘴却是多余。
\par }{\Q \VS{7}人吃饱了,厌恶蜂房的蜜;
\par }{\Q 人饥饿了,一切苦物都觉甘甜。
\par }{\Q \VS{8}人离本处飘流,
\par }{\Q 好像雀鸟离窝游飞。
\par }{\Q \VS{9}膏油与香料使人心喜悦;
\par }{\Q 朋友诚实的劝教也是如此甘美。
\par }{\Q \VS{10}你的朋友和父亲的朋友,
\par }{\Q 你都不可离弃。
\par }{\Q 你遭难的日子,不要上弟兄的家去;
\par }{\Q 相近的邻舍强如远方的弟兄。
\par }{\Q \VS{11}我儿,你要作智慧人,好叫我的心欢喜,
\par }{\Q 使我可以回答那讥诮我的人。
\par }{\Q \VS{12}通达人见祸藏躲;
\par }{\Q 愚蒙人前往受害。
\par }{\Q \VS{13}谁为生人作保,就拿谁的衣服;
\par }{\Q 谁为外女{\ADD{作保}},谁就承当。
\par }{\Q \VS{14}清晨起来,大声给朋友祝福的,
\par }{\Q 就算是咒诅他。
\par }{\Q \VS{15}大雨之日连连滴漏,
\par }{\Q 和争吵的妇人一样;
\par }{\Q \VS{16}想拦阻她的,便是拦阻风,
\par }{\Q 也是右手抓油。
\par }{\Q \VS{17}铁磨铁,磨出刃来;
\par }{\Q 朋友相感\FTNT{}{{\FR 27:17: }原文是磨朋友的脸}也是如此。
\par }{\Q \VS{18}看守无花果树的,必吃树上的果子;
\par }{\Q 敬奉主人的,必得尊荣。
\par }{\Q \VS{19}水中照脸,彼此{\ADD{相符}};
\par }{\Q 人与人,心也{\ADD{相对}}。
\par }{\Q \VS{20}阴间和灭亡永不满足;
\par }{\Q 人的眼目也是如此。
\par }{\Q \VS{21}鼎为炼银,炉为炼金,
\par }{\Q 人的称赞也{\ADD{试炼}}人。
\par }{\Q \VS{22}你虽用杵将愚妄人与打碎的麦子一同捣在臼中,
\par }{\Q 他的愚妄还是离不了他。
\par }{\BB \par }{\Q \VS{23}你要详细知道你羊群的景况,
\par }{\Q 留心料理你的牛群;
\par }{\Q \VS{24}因为资财不能永有,
\par }{\Q 冠冕岂能存到万代?
\par }{\Q \VS{25}干草割去,嫩草发现,
\par }{\Q 山上的菜蔬也被收敛。
\par }{\Q \VS{26}羊羔{\ADD{之毛}}是为你作衣服;
\par }{\Q 山羊是为作田地的价值,
\par }{\Q \VS{27}并有母山羊奶够你吃,
\par }{\Q 也够你的家眷吃,
\par }{\Q 且够养你的婢女。

\par }\PoetryChap{28}{\Q \VerseOne{1}恶人虽无人追赶也逃跑;
\par }{\Q 义人却胆壮像狮子。
\par }{\Q \VS{2}邦国因有罪过,君王就多{\ADD{更换}};
\par }{\Q 因有聪明知识的人,国必长存。
\par }{\Q \VS{3}穷人欺压贫民,
\par }{\Q 好像暴雨{\ADD{冲}}没粮食。
\par }{\Q \VS{4}违弃律法的,夸奖恶人;
\par }{\Q 遵守律法的,却与恶人相争。
\par }{\Q \VS{5}坏人不明白公义;
\par }{\Q 惟有寻求耶和华的,无不明白。
\par }{\Q \VS{6}行为纯正的穷乏人
\par }{\Q 胜过行事乖僻的富足人。
\par }{\Q \VS{7}谨守律法的,是智慧之子;
\par }{\Q 与贪食人作伴的,却羞辱其父。
\par }{\Q \VS{8}人以{\ADD{厚}}利加增财物,
\par }{\Q 是给那怜悯穷人者积蓄的。
\par }{\Q \VS{9}转耳不听律法的,
\par }{\Q 他的祈祷也为可憎。
\par }{\Q \VS{10}诱惑正直人行恶道的,必掉在自己的坑里;
\par }{\Q 惟有完全人必承受福分。
\par }{\Q \VS{11}富足人自以为有智慧,
\par }{\Q 但聪明的贫穷人能将他查透。
\par }{\Q \VS{12}义人得志,有大荣耀;
\par }{\Q 恶人兴起,人就躲藏。
\par }{\Q \VS{13}遮掩自己罪过的,必不亨通;
\par }{\Q 承认离弃罪过的,必蒙怜恤。
\par }{\Q \VS{14}常存敬畏的,便为有福;
\par }{\Q 心存刚硬的,必陷在祸患里。
\par }{\Q \VS{15}暴虐的君王辖制贫民,
\par }{\Q 好像吼叫的狮子、觅食的熊。
\par }{\Q \VS{16}无知的君多行暴虐;
\par }{\Q 以贪财为可恨的,必年长日久。
\par }{\Q \VS{17}背负流人血之罪的,必往坑里奔跑,
\par }{\Q 谁也不可拦阻他。
\par }{\Q \VS{18}行动正直的,必蒙拯救;
\par }{\Q 行事弯曲的,立时跌倒。
\par }{\Q \VS{19}耕种自己田地的,必得饱食;
\par }{\Q 追随虚浮的,足受穷乏。
\par }{\Q \VS{20}诚实人必多得福;
\par }{\Q 想要急速发财的,不免受罚。
\par }{\Q \VS{21}看人的情面乃为不好;
\par }{\Q 人因一块饼枉法也为不好。
\par }{\Q \VS{22}人有恶眼想要急速发财,
\par }{\Q 却不知穷乏必临到他身。
\par }{\Q \VS{23}责备人的,后来蒙人喜悦,
\par }{\Q 多于那用舌头谄媚人的。
\par }{\Q \VS{24}偷窃父母的,说:这不是罪,
\par }{\Q 此人就是与强盗同类。
\par }{\Q \VS{25}心中贪婪的,挑起争端;
\par }{\Q 倚靠耶和华的,必得丰裕。
\par }{\Q \VS{26}心中自是的,便是愚昧人;
\par }{\Q 凭智慧行事的,必蒙拯救。
\par }{\Q \VS{27}周济贫穷的,不致缺乏;
\par }{\Q 佯为不见的,必多受咒诅。
\par }{\Q \VS{28}恶人兴起,人就躲藏;
\par }{\Q 恶人败亡,义人增多。

\par }\PoetryChap{29}{\Q \VerseOne{1}人屡次受责罚,仍然硬着颈项;
\par }{\Q 他必顷刻败坏,无法可治。
\par }{\Q \VS{2}义人增多,民就喜乐;
\par }{\Q 恶人掌权,民就叹息。
\par }{\Q \VS{3}爱慕智慧的,使父亲喜乐;
\par }{\Q 与妓女结交的,却浪费钱财。
\par }{\Q \VS{4}王借公平,使国坚定;
\par }{\Q 索要贿赂,使国倾败。
\par }{\Q \VS{5}谄媚邻舍的,
\par }{\Q 就是设网罗绊他的脚。
\par }{\Q \VS{6}恶人犯罪,自陷网罗;
\par }{\Q 惟独义人欢呼喜乐。
\par }{\Q \VS{7}义人知道查明穷人的案;
\par }{\Q 恶人没有聪明,就不得而知。
\par }{\Q \VS{8}亵慢人煽惑通城;
\par }{\Q 智慧人止息众怒。
\par }{\Q \VS{9}智慧人与愚妄人相争,
\par }{\Q 或怒或笑,总不能使他止息。
\par }{\Q \VS{10}好流人血的,恨恶完全人,
\par }{\Q 索取正直人的性命。
\par }{\Q \VS{11}愚妄人怒气全发;
\par }{\Q 智慧人忍气含怒。
\par }{\Q \VS{12}君王若听谎言,
\par }{\Q 他一切臣仆都是奸恶。
\par }{\Q \VS{13}贫穷人、强暴人在世相遇;
\par }{\Q 他们的眼目都蒙耶和华光照。
\par }{\Q \VS{14}君王凭诚实判断穷人;
\par }{\Q 他的国位必永远坚立。
\par }{\Q \VS{15}杖打和责备能加增智慧;
\par }{\Q 放纵的儿子使母亲羞愧。
\par }{\Q \VS{16}恶人加多,过犯也加多,
\par }{\Q 义人必看见他们跌倒。
\par }{\Q \VS{17}管教你的儿子,他就使你得安息,
\par }{\Q 也必使你心里喜乐。
\par }{\Q \VS{18}没有异象\FTNT{}{{\FR 29:18: }或译:默示},民就放肆;
\par }{\Q 惟遵守律法的,便为有福。
\par }{\Q \VS{19}只用言语,仆人不肯受管教;
\par }{\Q 他虽然明白,也不留意。
\par }{\Q \VS{20}你见言语急躁的人吗?
\par }{\Q 愚昧人比他更有指望。
\par }{\Q \VS{21}人将仆人从小娇养,
\par }{\Q 这仆人终久必成了他的儿子。
\par }{\Q \VS{22}好气的人挑启争端;
\par }{\Q 暴怒的人多多犯罪。
\par }{\Q \VS{23}人的高傲必使他卑下;
\par }{\Q 心里谦逊的,必得尊荣。
\par }{\Q \VS{24}人与盗贼分赃,是恨恶自己的性命;
\par }{\Q 他听见叫人发誓的声音,却不言语。
\par }{\Q \VS{25}惧怕人的,陷入网罗;
\par }{\Q 惟有倚靠耶和华的,必得安稳。
\par }{\Q \VS{26}求王恩的人多;
\par }{\Q 定人事乃在耶和华。
\par }{\Q \VS{27}为非作歹的,被义人憎嫌;
\par }{\Q 行事正直的,被恶人憎恶。

\par }\Chap{30}{\SH 亚古珥语录
\par }{\Q \VerseOne{1}{\PN{雅基}}的儿子{\PN{亚古珥}}的言语就是真言。
\par }{\Q 这人对{\PN{以铁}}和{\PN{乌甲}}说:
\par }{\Q \VS{2}我比众人更蠢笨,
\par }{\Q 也没有人的聪明。
\par }{\Q \VS{3}我没有学好智慧,
\par }{\Q 也不认识至圣者。
\par }{\Q \VS{4}谁升天又降下来?
\par }{\Q 谁聚风在掌握中?
\par }{\Q 谁包水在衣服里?
\par }{\Q 谁立定地的四极?
\par }{\Q 他名叫什么?
\par }{\Q 他儿子名叫什么?
\par }{\Q 你知道吗?
\par }{\BB \par }{\Q \VS{5}神的言语句句都是炼净的;
\par }{\Q 投靠他的,他便作他们的盾牌。
\par }{\Q \VS{6}他的言语,你不可加添,
\par }{\Q 恐怕他责备你,你就显为说谎言的。
\par }{\SH 其他箴言
\par }{\Q \VS{7}我求你两件事,
\par }{\Q 在我未死之先,不要不赐给我:
\par }{\Q \VS{8}求你使虚假和谎言远离我;
\par }{\Q 使我也不贫穷也不富足;
\par }{\Q 赐给我需用的饮食,
\par }{\Q \VS{9}恐怕我饱足不认{\ADD{你}},说:
\par }{\Q 耶和华是谁呢?
\par }{\Q 又恐怕我贫穷就偷窃,
\par }{\Q 以致亵渎我 神的名。
\par }{\BB \par }{\Q \VS{10}你不要向主人谗谤仆人,
\par }{\Q 恐怕他咒诅你,你便算为有罪。
\par }{\BB \par }{\Q \VS{11}有一宗人\FTNT{}{{\FR 30:11: }宗:原文是代;下同},咒诅父亲,
\par }{\Q 不给母亲祝福。
\par }{\Q \VS{12}有一宗人,自以为清洁,
\par }{\Q 却没有洗去自己的污秽。
\par }{\Q \VS{13}有一宗人,眼目何其高傲,
\par }{\Q 眼皮也是高举。
\par }{\Q \VS{14}有一宗人,牙如剑,齿如刀,
\par }{\Q 要吞灭地上的困苦人和世间的穷乏人。
\par }{\BB \par }{\Q \VS{15}蚂蟥有两个女儿,
\par }{\Q {\ADD{常说}}:给呀,给呀!
\par }{\Q 有三样不知足的,
\par }{\Q 连不说「够的」共有四样:
\par }{\Q \VS{16}就是阴间和石胎,
\par }{\Q 浸水不足的地,并火。
\par }{\BB \par }{\Q \VS{17}戏笑父亲、藐视而不听从母亲的,
\par }{\Q 他的眼睛必为谷中的乌鸦啄出来,为鹰雏所吃。
\par }{\BB \par }{\Q \VS{18}我所测不透的奇妙有三样,
\par }{\Q 连我所不知道的共有四样:
\par }{\Q \VS{19}就是鹰在空中飞的道;
\par }{\Q 蛇在磐石上爬的道;
\par }{\Q 船在海中行的道;
\par }{\Q 男与女交合的道。
\par }{\Q \VS{20}淫妇的道也是这样:
\par }{\Q 她吃了,把嘴一擦就说:
\par }{\Q 我没有行恶。
\par }{\BB \par }{\Q \VS{21}使地震动的有三样,
\par }{\Q 连地担不起的共有四样:
\par }{\Q \VS{22}就是仆人作王;
\par }{\Q 愚顽人吃饱;
\par }{\Q \VS{23}丑恶的女子出嫁;
\par }{\Q 婢女接续主母。
\par }{\BB \par }{\Q \VS{24}地上有四样小物,却甚聪明:
\par }{\Q \VS{25}蚂蚁是无力之类,
\par }{\Q 却在夏天预备粮食。
\par }{\Q \VS{26}沙番是软弱之类,
\par }{\Q 却在磐石中造房。
\par }{\Q \VS{27}蝗虫没有君王,
\par }{\Q 却分队而出。
\par }{\Q \VS{28}守宫用爪抓{\ADD{墙}},
\par }{\Q 却住在王宫。
\par }{\BB \par }{\Q \VS{29}步行威武的有三样,
\par }{\Q 连行走威武的共有四样:
\par }{\Q \VS{30}就是狮子—乃百兽中最为猛烈、无所躲避的,
\par }{\Q \VS{31}猎狗,公山羊,和无人能敌的君王。
\par }{\BB \par }{\Q \VS{32}你若行事愚顽,自高自傲,
\par }{\Q 或是怀了恶念,就当用手{\ADD{捂口}}。
\par }{\Q \VS{33}摇牛奶必成奶油;
\par }{\Q 扭鼻子必出血。
\par }{\Q 照样,激动怒气必起争端。

\par }\Chap{31}{\SH 给一君王的忠告
\par }{\Q \VerseOne{1}{\PN{利慕伊勒}}王的言语,是他母亲教训他的真言。
\par }{\Q \VS{2}我的儿啊,我腹中生的儿啊,
\par }{\Q 我许愿得的儿啊!我当怎样{\ADD{教训}}你呢?
\par }{\Q \VS{3}不要将你的精力给妇女;
\par }{\Q 也不要有败坏君王的行为。
\par }{\Q \VS{4}{\PN{利慕伊勒}}啊,君王喝酒,君王喝酒不相宜;
\par }{\Q 王子说浓酒在那里也不相宜;
\par }{\Q \VS{5}恐怕喝了就忘记律例,
\par }{\Q 颠倒一切困苦人的是非。
\par }{\Q \VS{6}可以把浓酒给将亡的人喝,
\par }{\Q 把{\ADD{清}}酒给苦心的人喝,
\par }{\Q \VS{7}让他喝了,就忘记他的贫穷,
\par }{\Q 不再记念他的苦楚。
\par }{\Q \VS{8}你当为哑巴\FTNT{}{{\FR 31:8: }或译:不能自辩的}开口,
\par }{\Q 为一切孤独的伸冤。
\par }{\Q \VS{9}你当开口按公义判断,
\par }{\Q 为困苦和穷乏的辨屈。
\par }{\SH 论贤妻
\par }{\Q \VS{10}才德的妇人谁能得着呢?
\par }{\Q 她的价值远胜过珍珠。
\par }{\Q \VS{11}她丈夫心里倚靠她,
\par }{\Q 必不缺少利益;
\par }{\Q \VS{12}她一生使丈夫有益无损。
\par }{\Q \VS{13}她寻找羊绒和麻,
\par }{\Q 甘心用手做工。
\par }{\Q \VS{14}她好像商船从远方运粮来,
\par }{\Q \VS{15}未到黎明她就起来,
\par }{\Q 把食物分给家中的人,
\par }{\Q 将当做的工分派婢女。
\par }{\Q \VS{16}她想得田地就买来;
\par }{\Q 用手所得之利栽种葡萄园。
\par }{\Q \VS{17}她以能力束腰,
\par }{\Q 使膀臂有力。
\par }{\Q \VS{18}她觉得所经营的有利;
\par }{\Q 她的灯终夜不灭。
\par }{\Q \VS{19}她手拿捻线竿,
\par }{\Q 手把纺线车。
\par }{\Q \VS{20}她张手周济困苦人,
\par }{\Q 伸手帮补穷乏人。
\par }{\Q \VS{21}她不因下雪为家里的人担心,
\par }{\Q 因为全家都穿着朱红衣服。
\par }{\Q \VS{22}她为自己制作绣花毯子;
\par }{\Q 她的衣服是细麻和紫色布做的。
\par }{\Q \VS{23}她丈夫在城门口与本地的长老同坐,
\par }{\Q 为众人所认识。
\par }{\Q \VS{24}她做细麻布衣裳出卖,
\par }{\Q 又将腰带卖与商家。
\par }{\Q \VS{25}能力和威仪是她的衣服;
\par }{\Q 她{\ADD{想到}}日后的景况就喜笑。
\par }{\Q \VS{26}她开口就发智慧;
\par }{\Q 她舌上有仁慈的法则。
\par }{\Q \VS{27}她观察家务,
\par }{\Q 并不吃闲饭。
\par }{\Q \VS{28}她的儿女起来称她有福;
\par }{\Q 她的丈夫也称赞她,
\par }{\Q \VS{29}说:才德的女子很多,
\par }{\Q 惟独你超过一切。
\par }{\Q \VS{30}艳丽是虚假的,美容是虚浮的;
\par }{\Q 惟敬畏耶和华的妇女必得称赞。
\par }{\Q \VS{31}愿她享受操作所得的;
\par }{\Q 愿她的工作在城门口荣耀她。
\par }