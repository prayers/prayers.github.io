\NormalFont\ShortTitle{传道书}
{\MT 传道书

\par }\ChapOne{1}{\SH 人生的虚空
\par }{\PP \VerseOne{1}在{\PN{耶路撒冷}}作王、{\PN{大卫}}的儿子、传道者的言语。
\par }{\Q \VS{2}传道者说:虚空的虚空,
\par }{\Q 虚空的虚空,凡事都是虚空。
\par }{\Q \VS{3}人一切的劳碌,
\par }{\Q 就是他在日光之下的劳碌,有什么益处呢?
\par }{\Q \VS{4}一代过去,一代又来,
\par }{\Q 地却永远长存。
\par }{\Q \VS{5}日头出来,日头落下,
\par }{\Q 急归所出之地。
\par }{\Q \VS{6}风往南刮,又向北转,
\par }{\Q 不住地旋转,而且返回转行原道。
\par }{\Q \VS{7}江河都往海里流,海却不满;
\par }{\Q 江河从何处流,仍归还何处。
\par }{\Q \VS{8}万事令人厌烦\FTNT{}{{\FR 1:8: }或译:万物满有困乏},
\par }{\Q 人不能说尽。
\par }{\Q 眼看,看不饱;
\par }{\Q 耳听,听不足。
\par }{\Q \VS{9}已有的事后必再有;
\par }{\Q 已行的事后必再行。
\par }{\Q 日光之下并无新事。
\par }{\Q \VS{10}岂有一件事人能指着说这是新的?
\par }{\Q {\ADD{哪知}},在我们以前的世代早已有了。
\par }{\Q \VS{11}已过的{\ADD{世代}},无人记念;
\par }{\Q 将来的{\ADD{世代}},后来的人也不记念。
\par }{\SH 传道者的经验
\par }{\PP \VS{12}我传道者在{\PN{耶路撒冷}}作过{\PN{以色列}}的王。
\VS{13}我专心用智慧寻求、查究天下所做的一切事,{\ADD{乃知}} 神叫世人所经练的是极重的劳苦。
\VS{14}我见日光之下所做的一切事,都是虚空,都是捕风。
\par }{\Q \VS{15}弯曲的,不能变直;
\par }{\Q 缺少的,不能足数。
\par }{\PP \VS{16}我心里议论说:我得了大智慧,胜过我以前在{\PN{耶路撒冷}}的众人,而且我心中多经历智慧和知识的事。
\VS{17}我又专心察明智慧、狂妄,和愚昧,乃知这也是捕风。
\par }{\Q \VS{18}因为多有智慧,就多有愁烦;
\par }{\Q 加增知识的,就加增忧伤。

\par }\Chap{2}{\PP \VerseOne{1}我心里说:「来吧,我以喜乐试试你,你好享福!」谁知,这也是虚空。
\VS{2}我指嬉笑说:「这是狂妄。」论喜乐说:「有何功效呢?」
\VS{3}我心里察究,如何用酒使我肉体舒畅,我心却仍以智慧引导{\ADD{我}};又如何持住愚昧,等我看明世人,在天下一生当行何事为美。
\VS{4}我为自己动大工程,建造房屋,栽种葡萄园,
\VS{5}修造园囿,在其中栽种各样果木树;
\VS{6}挖造水池,用以浇灌嫩小的树木。
\VS{7}我买了仆婢,也有生在家中的仆婢;又有许多牛群羊群,胜过以前在{\PN{耶路撒冷}}众人所有的。
\VS{8}我又为自己积蓄金银和君王的财宝,并各省的财宝;又得唱歌的男女和世人所喜爱的物,并许多的妃嫔。
\par }{\PP \VS{9}这样,我就日见昌盛,胜过以前在{\PN{耶路撒冷}}的众人。我的智慧仍然存留。
\VS{10}凡我眼所求的,我没有留下不给它的;我心所乐的,我没有禁止不享受的;因我的心为我一切所劳碌的快乐,这就是我从劳碌中所得的分。
\VS{11}后来,我察看我手所经营的一切事和我劳碌所成的功。谁知都是虚空,都是捕风;在日光之下毫无益处。
\VS{12}我转念观看智慧、狂妄,和愚昧。在王以后而来的人还能做什么呢?也不过行早先所行的就是了。
\VS{13}我便看出智慧胜过愚昧,如同光明胜过黑暗。
\VS{14}智慧人的眼目光明\FTNT{}{{\FR 2:14: }光明:原文是在他头上},愚昧人在黑暗里行。我却看明有一件事,这两等人都必遇见。
\VS{15}我就心里说:「愚昧人所遇见的,我也必遇见,我为何更有智慧呢?」我心里说,这也是虚空。
\VS{16}智慧人和愚昧人一样,永远无人记念,因为日后都被忘记;可叹智慧人死亡,与愚昧人无异。
\VS{17}我所以恨恶生命;因为在日光之下所行的事我都以为烦恼,都是虚空,都是捕风。
\par }{\PP \VS{18}我恨恶一切的劳碌,就是我在日光之下的劳碌,因为我得来的必留给我以后的人。
\VS{19}那人是智慧是愚昧,谁能知道?他竟要管理我劳碌所得的,就是我在日光之下用智慧所得的。这也是虚空。
\VS{20}故此,我转想我在日光之下所劳碌的一切工作,心便绝望。
\VS{21}因为有人用智慧、知识、灵巧所劳碌得来的,却要留给未曾劳碌的人为分。这也是虚空,也是大患。
\VS{22}人在日光之下劳碌累心,在他一切的劳碌上得着什么呢?
\VS{23}因为他日日忧虑,他的劳苦成为愁烦,连夜间心也不安。这也是虚空。
\par }{\PP \VS{24}人莫强如吃喝,且在劳碌中享福,我看这也是出于 神的手。
\VS{25}论到吃用、享福,谁能胜过我呢?
\VS{26}神喜悦谁,就给谁智慧、知识,和喜乐;惟有罪人, 神使他劳苦,叫他将所收聚的、所堆积的归给 神所喜悦的人。这也是虚空,也是捕风。

\par }\Chap{3}{\SH 万事均有定时
\par }{\Q \VerseOne{1}凡事都有定期,
\par }{\Q 天下万务都有定时。
\par }{\Q \VS{2}生有时,死有时;
\par }{\Q 栽种有时,拔出所栽种的也有时;
\par }{\Q \VS{3}杀戮有时,医治有时;
\par }{\Q 拆毁有时,建造有时;
\par }{\Q \VS{4}哭有时,笑有时;
\par }{\Q 哀恸有时,跳舞有时;
\par }{\Q \VS{5}抛掷石头有时,堆聚石头有时;
\par }{\Q 怀抱有时,不怀抱有时;
\par }{\Q \VS{6}寻找有时,失落有时;
\par }{\Q 保守有时,舍弃有时;
\par }{\Q \VS{7}撕裂有时,缝补有时;
\par }{\Q 静默有时,言语有时;
\par }{\Q \VS{8}喜爱有时,恨恶有时;
\par }{\Q 争战有时,和好有时。
\par }{\PP \VS{9}{\ADD{这样看来}},做事的人在他的劳碌上有什么益处呢?
\VS{10}我见 神叫世人劳苦,使他们在其中受经练。
\VS{11}神造万物,各按其时成为美好,又将永生\FTNT{}{{\FR 3:11: }原文是永远}安置在世人心里。然而 神从始至终的作为,人不能参透。
\VS{12}我知道世人,莫强如终身喜乐行善;
\VS{13}并且人人吃喝,在他一切劳碌中享福,这也是 神的恩赐。
\par }{\PP \VS{14}我知道 神一切所做的都必永存;无所增添,无所减少。 神这样行,是要人在他面前存敬畏的心。
\VS{15}现今的事早先就有了,将来的事早已也有了,并且 神使已过的事重新再来\FTNT{}{{\FR 3:15: }或译:并且 神再寻回已过的事}。
\par }{\SH 世上的不公平
\par }{\PP \VS{16}我又见日光之下,在审判之处有奸恶,在公义之处也有奸恶。
\VS{17}我心里说, 神必审判义人和恶人;因为在那里,各样事务,一切工作,都有定时。
\VS{18}我心里说,{\ADD{这乃}}为世人的缘故,是 神要试验他们,使他们觉得自己{\ADD{不过像}}兽一样。
\VS{19}因为世人遭遇的,兽也遭遇,所遭遇的都是一样:这个怎样死,那个也怎样死,气息都是一样。人不能强于兽,都是虚空。
\VS{20}都归一处,都是出于尘土,也都归于尘土。
\VS{21}谁知道人的灵是往上升,兽的魂是下入地呢?
\VS{22}故此,我见人莫强如在他经营的事上喜乐,因为这是他的分。他身后的事谁能使他回来得见呢?

\par }\Chap{4}{\PP \VerseOne{1}我又转念,见日光之下所行的一切欺压。看哪,受欺压的流泪,且无人安慰;欺压他们的有势力,也无人安慰他们。
\VS{2}因此,我赞叹那早已死的死人,胜过那还活着的活人。
\VS{3}并且我以为那未曾生的,就是未见过日光之下恶事的,比这两等人更强。
\par }{\PP \VS{4}我又见人为一切的劳碌和各样灵巧的工作就被邻舍嫉妒。这也是虚空,也是捕风。
\par }{\PP \VS{5}愚昧人抱着手,吃自己的肉。
\par }{\PP \VS{6}满了一把,得享安静,强如满了两把,劳碌捕风。
\par }{\PP \VS{7}我又转念,见日光之下有一件虚空的事:
\VS{8}有人孤单无二,无子无兄,竟劳碌不息,眼目也不以钱财为足。{\ADD{他说}}:「我劳劳碌碌,刻苦自己,不享福乐,到底是为谁呢?」这也是虚空,是极重的劳苦。
\par }{\PP \VS{9}两个人总比一个人好,因为二人劳碌同得美好的果效。
\VS{10}若是跌倒,这人可以扶起他的同伴;若是孤身跌倒,没有别人扶起他来,这人就有祸了。
\VS{11}再者,二人同睡就都暖和,一人独睡怎能暖和呢?
\VS{12}有人攻胜孤身一人,若有二人便能敌挡他;三股合成的绳子不容易折断。
\par }{\PP \VS{13}贫穷而有智慧的少年人胜过年老不肯纳谏的愚昧王。
\VS{14}这人是从监牢中出来作王,在他国中,生来原是贫穷的。
\VS{15}我见日光之下一切行动的活人都随从那第二位,就是起来代替老王的少年人。
\VS{16}他所治理的众人就是他的百姓,多得无数;在他后来的人尚且不喜悦他。这真是虚空,也是捕风。

\par }\Chap{5}{\SH 不可随便许愿
\par }{\PP \VerseOne{1}你到 神的殿要谨慎脚步;因为近前听,胜过愚昧人献祭\FTNT{}{{\FR 5:1: }或译:胜过献愚昧人的祭},他们本不知道所做的是恶。
\VS{2}你在 神面前不可冒失开口,也不可心急发言;因为 神在天上,你在地下,所以你的言语要寡少。
\par }{\PP \VS{3}事务多,就令人做梦;言语多,就显出愚昧。
\par }{\PP \VS{4}你向 神许愿,偿还不可迟延,因他不喜悦愚昧人,所以你许的愿应当偿还。
\VS{5}你许愿不还,不如不许。
\VS{6}不可任你的口使肉体犯罪,也不可在祭司\FTNT{}{{\FR 5:6: }原文是使者}面前说是错许了。为何使 神因你的声音发怒,败坏你手所做的呢?
\par }{\PP \VS{7}多梦和多言,其中多有虚幻,你只要敬畏 神。
\par }{\SH 人生是虚空的
\par }{\PP \VS{8}你若在一省之中见穷人受欺压,并夺去公义公平的事,不要因此诧异;因有一位高过居高位的鉴察,在他们以上还有更高的。
\VS{9}况且地的益处归众人,就是君王也受田地的供应。
\par }{\PP \VS{10}贪爱银子的,不因得银子知足;贪爱丰富的,也不因得利益知足。这也是虚空。
\par }{\PP \VS{11}货物增添,吃的人也增添,物主得什么益处呢?不过眼看而已!
\par }{\PP \VS{12}劳碌的人不拘吃多吃少,睡得香甜;富足人的丰满却不容他睡觉。
\par }{\PP \VS{13}我见日光之下有一宗大祸患,就是财主积存资财,反害自己。
\VS{14}因遭遇祸患,这些资财就消灭;那人若生了儿子,手里也一无所有。
\VS{15}他怎样从母胎赤身而来,也必照样赤身而去;他所劳碌得来的,手中分毫不能带去。
\VS{16}他来的情形怎样,他去的情形也怎样。这也是一宗大祸患。他为风劳碌有什么益处呢?
\VS{17}并且他终身在黑暗中吃喝,多有烦恼,又有病患呕气。
\par }{\PP \VS{18}我所见为善为美的,就是人在 神赐他一生的日子吃喝,享受日光之下劳碌得来的好处,因为这是他的分。
\VS{19}神赐人资财丰富,使他能以吃用,能取自己的分,在他劳碌中喜乐,这乃是 神的恩赐。
\VS{20}他不多思念自己一生的年日,因为 神应他的心使他喜乐。

\par }\Chap{6}{\PP \VerseOne{1}我见日光之下有一宗祸患重压在人身上,
\VS{2}就是人蒙 神赐他资财、丰富、尊荣,以致他心里所愿的一样都不缺,只是 神使他不能吃用,反有外人来吃用。这是虚空,也是祸患。
\VS{3}人若生一百个儿子,活许多岁数,以致他的年日甚多,心里却不得满享福乐,又不得埋葬;据我说,那不到期而落的胎比他倒好。
\VS{4}因为虚虚而来,暗暗而去,名字被黑暗遮蔽,
\VS{5}并且没有见过天日,也毫无知觉;这胎,比那人倒享安息。
\VS{6}那人虽然活千年,再活千年,却不享福,众人岂不都归一个地方去吗?
\par }{\PP \VS{7}人的劳碌都为口腹,心里却不知足。
\VS{8}这样看来,智慧人比愚昧人有什么长处呢?穷人在众人面前知道如何行,有什么长处呢?
\VS{9}眼睛所看的比心里妄想的倒好。这也是虚空,也是捕风。
\par }{\PP \VS{10}先前所有的,早已起了名,并知道何为人,他也不能与那比自己力大的相争。
\VS{11}加增虚浮的事既多,这与人有什么益处呢?
\VS{12}人一生虚度的日子,就如影儿经过,谁知道什么与他有益呢?谁能告诉他身后在日光之下有什么事呢?

\par }\Chap{7}{\SH 智愚之别
\par }{\Q \VerseOne{1}名誉强如美好的膏油;人死的日子胜过人生的日子。
\par }{\Q \VS{2}往遭丧的家去,
\par }{\Q 强如往宴乐的家去;
\par }{\Q 因为{\ADD{死}}是众人的结局,
\par }{\Q 活人也必将这事放在心上。
\par }{\Q \VS{3}忧愁强如喜笑;
\par }{\Q 因为面带愁容,终必使心喜乐。
\par }{\Q \VS{4}智慧人的心在遭丧之家;
\par }{\Q 愚昧人的心在快乐之家。
\par }{\Q \VS{5}听智慧人的责备,
\par }{\Q 强如听愚昧人的歌唱。
\par }{\Q \VS{6}愚昧人的笑声,
\par }{\Q 好像锅下烧荆棘的爆声;
\par }{\Q 这也是虚空。
\par }{\Q \VS{7}勒索使智慧人变为愚妄;
\par }{\Q 贿赂能败坏人的慧心。
\par }{\Q \VS{8}事情的终局强如事情的起头;
\par }{\Q 存心忍耐的,胜过居心骄傲的。
\par }{\Q \VS{9}你不要心里急躁恼怒,
\par }{\Q 因为恼怒存在愚昧人的怀中。
\par }{\Q \VS{10}不要说:
\par }{\Q 先前的日子强过如今的日子,
\par }{\Q 是什么缘故呢?
\par }{\Q 你这样问,不是出于智慧。
\par }{\Q \VS{11}智慧和产业并好,
\par }{\Q 而且见天日的人得智慧更为有益。
\par }{\Q \VS{12}因为智慧护庇人,
\par }{\Q 好像银钱护庇人一样。
\par }{\Q 惟独智慧能保全智慧人的生命。
\par }{\Q 这就是知识的益处。
\par }{\Q \VS{13}你要察看 神的作为;
\par }{\Q 因 神使为曲的,谁能变为直呢?
\par }{\PP \VS{14}遇亨通的日子你当喜乐;遭患难的日子你当思想;因为 神使这两样并列,为的是叫人查不出身后有什么事。
\par }{\PP \VS{15}有义人行义,反致灭亡;有恶人行恶,倒享长寿。这都是我在虚度之日中所见过的。
\VS{16}不要行义过分,也不要过于自逞智慧,何必自取败亡呢?
\VS{17}不要行恶过分,也不要为人愚昧,何必不到期而死呢?
\VS{18}你持守这个为美,那个也不要松手;因为敬畏 神的人,必从这两样出来。
\par }{\PP \VS{19}智慧使有智慧的人比城中十个官长更有能力。
\par }{\PP \VS{20}时常行善而不犯罪的义人,世上实在没有。
\par }{\PP \VS{21}人所说的一切话,你不要放在心上,恐怕听见你的仆人咒诅你。
\VS{22}因为你心里知道,自己也曾屡次咒诅别人。
\par }{\PP \VS{23}我曾用智慧试验这一切事;我说,要得智慧,智慧却离我远。
\VS{24}万事{\ADD{之理}},离{\ADD{我}}甚远,而且最深,谁能测透呢?
\VS{25}我转念,一心要知道,要考察,要寻求智慧和{\ADD{万事的}}理由;又要知道邪恶为愚昧,愚昧为狂妄。
\par }{\PP \VS{26}我得知有等妇人比死还苦:她的心是网罗,手是锁链。凡蒙 神喜悦的人必能躲避她;有罪的人却被她缠住了。
\VS{27-28}传道者说:「看哪,一千男子中,我找到一个{\ADD{正直人}},但众女子中,没有找到一个。」我{\ADD{将这事}}一一比较,要寻求其理,我心仍要寻找,却未曾找到。
\VS{29}我所找到的只有一件,就是 神造人原是正直,但他们寻出许多巧计。

\par }\PoetryChap{8}{\Q \VerseOne{1}谁如智慧人呢?
\par }{\Q 谁知道事情的解释呢?
\par }{\Q 人的智慧使他的脸发光,
\par }{\Q 并使他脸上的暴气改变。
\par }{\SH 服从君王
\par }{\PP \VS{2}我{\ADD{劝你}}遵守王的命令;既指 神起誓,理当如此。
\VS{3}不要急躁离开王的面前,不要固执行恶,因为他凡事都随自己的心意而行。
\VS{4}王的话本有权力,谁敢问他说「你做什么」呢?
\VS{5}凡遵守命令的,必不经历祸患;智慧人的心能辨明时候和定理\FTNT{}{{\FR 8:5: }原文是审判;下节同}。
\VS{6}各样事务{\ADD{成就}}都有时候和定理,因为人的苦难重压在他身上。
\VS{7}他不知道将来的事,因为将来如何,谁能告诉他呢?
\VS{8}无人有权力掌管生命,将生命留住;也无人有权力掌管死期;这场争战,无人能免;邪恶也不能救那好行邪恶的人。
\VS{9}这一切我都见过,也专心查考日光之下所做的一切事。有时这人管辖那人,令人受害。
\par }{\SH 恶人和义人
\par }{\PP \VS{10}我见恶人埋葬,归入{\ADD{坟墓}};又见行正直事的离开圣地,在城中被人忘记。这也是虚空。
\VS{11}因为断定罪名不立刻施刑,所以世人满心作恶。
\VS{12}罪人虽然作恶百次,倒享长久的年日;然而我准知道,敬畏 神的,就是在他面前敬畏的人,终久必得福乐。
\VS{13}恶人却不得福乐,也不得长久的年日;这年日好像影儿,因他不敬畏 神。
\par }{\PP \VS{14}世上有一件虚空的事,就是义人所遭遇的,反照恶人所行的;又有恶人所遭遇的,反照义人所行的。我说,这也是虚空。
\VS{15}我就称赞快乐,原来人在日光之下,莫强如吃喝快乐;因为他在日光之下, 神赐他一生的年日,要从劳碌中,时常享受所得的。
\par }{\PP \VS{16}我专心求智慧,要看世上所做的事。(有昼夜不睡觉不合眼的。)
\VS{17}我就看明 神一切的作为,知道人查不出日光之下所做的事;任凭他费多少力寻查,都查不出来,就是智慧人虽想知道,也是查不出来。

\par }\Chap{9}{\PP \VerseOne{1}我将这一切事放在心上,详细考究,就知道义人和智慧人,并他们的作为都在 神手中;或是爱,或是恨,都在他们的前面,人不能知道。
\VS{2}凡临到众人的事都是一样:义人和恶人都遭遇一样的事;好人,洁净人和不洁净人,献祭的与不献祭的,也是一样。好人如何,罪人也如何;起誓的如何,怕起誓的也如何。
\VS{3}在日光之下所行的一切事上有一件祸患,就是众人所遭遇的都是一样,并且世人的心充满了恶;活着的时候心里狂妄,后来就归死人那里去了。
\VS{4}与一切活人相连的,那人还有指望,因为活着的狗比死了的狮子更强。
\VS{5}活着的人知道必死;死了的人毫无所知,也不再得赏赐;他们的名无人记念。
\VS{6}他们的爱,他们的恨,他们的嫉妒,早都消灭了。在日光之下所行的一切事上,他们永不再有分了。
\par }{\PP \VS{7}你只管去欢欢喜喜吃你的饭,心中快乐喝你的酒,因为 神已经悦纳你的作为。
\par }{\PP \VS{8}你的衣服当时常洁白,你头上也不要缺少膏油。
\par }{\PP \VS{9}在你一生虚空的年日,就是 神赐你在日光之下虚空的年日,当同你所爱的妻,快活度日,因为那是你生前在日光之下劳碌的事上所得的分。
\VS{10}凡你手所当做的事要尽力去做;因为在你所必去的阴间没有工作,没有谋算,没有知识,也没有智慧。
\par }{\PP \VS{11}我又转念:见日光之下,快跑的未必能赢;力战的未必得胜;智慧的未必得粮食;明哲的未必得资财;灵巧的未必得喜悦。所临到众人的是在乎当时的机会。
\VS{12}原来人也不知道自己的定期。鱼被恶网圈住,鸟被网罗捉住,祸患忽然临到的时候,世人陷在其中也是如此。
\par }{\SH 智慧胜过武力
\par }{\PP \VS{13}我见日光之下有一样智慧,据我看乃是广大,
\VS{14}就是有一小城,其中的人数稀少,有大君王来攻击,修筑营垒,将城围困。
\VS{15}城中有一个贫穷的智慧人,他用智慧救了那城,却没有人记念那穷人。
\VS{16}我就说,智慧胜过勇力;然而那贫穷人的智慧被人藐视,他的话也无人听从。
\par }{\PP \VS{17}宁可在安静之中听智慧人的言语,不听掌管愚昧人的喊声。
\VS{18}智慧胜过打仗的兵器;但一个罪人能败坏许多善事。

\par }\PoetryChap{10}{\Q \VerseOne{1}死苍蝇使做香的膏油发出臭气;
\par }{\Q {\ADD{这样}},一点愚昧也能败坏智慧和尊荣。
\par }{\Q \VS{2}智慧人的心居右;
\par }{\Q 愚昧人的心居左。
\par }{\Q \VS{3}并且愚昧人行路显出无知,
\par }{\Q 对众人说,他是愚昧人。
\par }{\Q \VS{4}掌权者的心若向你发怒,
\par }{\Q 不要离开你的本位,
\par }{\Q 因为柔和能免大过。
\par }{\Q \VS{5}我见日光之下有一件祸患,
\par }{\Q 似乎出于掌权的错误,
\par }{\Q \VS{6}就是愚昧人立在高位;
\par }{\Q 富足人坐在低位。
\par }{\Q \VS{7}我见过仆人骑马,
\par }{\Q 王子像仆人在地上步行。
\par }{\Q \VS{8}挖陷坑的,自己必掉在其中;
\par }{\Q 拆墙垣的,必为蛇所咬。
\par }{\Q \VS{9}凿开\FTNT{}{{\FR 10:9: }或译:挪移}石头的,必受损伤;
\par }{\Q 劈开木头的,必遭危险。
\par }{\Q \VS{10}铁器钝了,若不将刃磨快,就必多费气力;
\par }{\Q 但得智慧指教,便有益处。
\par }{\Q \VS{11}未行法术以先,蛇若咬人,
\par }{\Q 后行法术也是无益。
\par }{\Q \VS{12}智慧人的口说出恩言;
\par }{\Q 愚昧人的嘴吞灭自己。
\par }{\Q \VS{13}他口中的言语起头是愚昧;
\par }{\Q 他话的末尾是奸恶的狂妄。
\par }{\Q \VS{14}愚昧人多有言语,
\par }{\Q 人却不知将来有什么事;
\par }{\Q 他身后的事谁能告诉他呢?
\par }{\Q \VS{15}凡愚昧人,他的劳碌使自己困乏,
\par }{\Q 因为连进城的路,他也不知道。
\par }{\Q \VS{16}邦国啊,你的王若是孩童,
\par }{\Q 你的群臣早晨宴乐,
\par }{\Q 你就有祸了!
\par }{\Q \VS{17}邦国啊,你的王若是贵胄之子,
\par }{\Q 你的群臣按时吃喝,
\par }{\Q 为要补力,不为酒醉,
\par }{\Q 你就有福了!
\par }{\Q \VS{18}因人懒惰,房顶塌下;
\par }{\Q 因人手懒,房屋滴漏。
\par }{\Q \VS{19}设摆筵席是为喜笑。
\par }{\Q 酒能使人快活;
\par }{\Q 钱能叫万事应心。
\par }{\Q \VS{20}你不可咒诅君王,
\par }{\Q 也不可心怀此念;
\par }{\Q 在你卧房也不可咒诅富户。
\par }{\Q 因为空中的鸟必传扬这声音,
\par }{\Q 有翅膀的也必述说这事。

\par }\Chap{11}{\SH 智慧人的作为
\par }{\Q \VerseOne{1}当将你的粮食撒在水面,
\par }{\Q 因为日久必能得着。
\par }{\Q \VS{2}你要分给七人,或分给八人,
\par }{\Q 因为你不知道将来有什么灾祸临到地上。
\par }{\Q \VS{3}云若满了雨,就必倾倒在地上。
\par }{\Q 树若向南倒,或向北倒,
\par }{\Q 树倒在何处,就存在何处。
\par }{\Q \VS{4}看风的,必不撒种;
\par }{\Q 望云的,必不收割。
\par }{\PP \VS{5}风从何道来,骨头在怀孕妇人的胎中如何长成,你尚且不得知道;这样,行万事之 神的作为,你更不得知道。
\par }{\PP \VS{6}早晨要撒你的种,晚上也不要歇你的手,因为你不知道哪一样发旺;或是早撒的,或是晚撒的,或是两样都好。
\par }{\PP \VS{7}光本是佳美的,眼见日光也是可悦的。
\par }{\PP \VS{8}人活多年,就当快乐多年;然而也当想到黑暗的日子。因为这日子必多,所要来的都是虚空。
\par }{\SH 对年轻人的忠告
\par }{\PP \VS{9}少年人哪,你在幼年时当快乐。在幼年的日子,使你的心欢畅,行你心所愿行的,看你眼所爱看的;却要知道,为这一切的事, 神必审问你。
\par }{\PP \VS{10}所以,你当从心中除掉愁烦,从肉体克去邪恶;因为一生的开端和幼年之时,都是虚空的。

\par }\Chap{12}{\PP \VerseOne{1}你趁着年幼、衰败的日子尚未来到,就是你所说,我毫无喜乐的那些年日未曾临近之先,当记念造你的主。
\VS{2}不要等到日头、光明、月亮、星宿变为黑暗,雨后云彩反回,
\VS{3}看守房屋的发颤,有力的屈身,推磨的稀少就止息,从窗户往外看的都昏暗;
\VS{4}街门关闭,推磨的响声微小,雀鸟一叫,人就起来,唱歌的女子也都衰微。
\VS{5}人怕高处,路上有惊慌,杏树开花,蚱蜢成为重担,人所愿的也都废掉;因为人归他永远的家,吊丧的在街上往来。
\VS{6}银链折断,金罐破裂,瓶子在泉旁损坏,水轮在井口破烂,
\VS{7}尘土仍归于地,灵仍归于赐灵的 神。
\VS{8}传道者说:「虚空的虚空,凡事都是虚空。」
\par }{\SH 结论
\par }{\PP \VS{9}再者,传道者因有智慧,仍将知识教训众人;又默想,又考查,又陈说许多箴言。
\VS{10}传道者专心寻求可喜悦的言语,是凭正直写的诚实话。
\par }{\PP \VS{11}智慧人的言语好像刺棍;会中之师的{\ADD{言语}}又像钉稳的钉子,都是一个牧者所赐的。
\VS{12}我儿,还有一层,你当受劝戒:著书多,没有穷尽;读书多,身体疲倦。
\par }{\PP \VS{13}这些事都已听见了,总意就是:敬畏 神,谨守他的诫命,这是人所当尽的{\ADD{本分}}\FTNT{}{{\FR 12:13: }或译:这是众人的本分}。
\VS{14}因为人所做的事,连一切隐藏的事,无论是善是恶, 神都必审问。
\par }