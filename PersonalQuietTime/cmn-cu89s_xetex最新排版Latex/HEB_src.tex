\NormalFont\ShortTitle{希伯来书}
{\MT 希伯来书

\par }\ChapOne{1}{\SH  神借着他儿子说话
\par }{\PP \VerseOne{1}神既在古时借着众先知多次多方地晓谕列祖,
\VS{2}就在这末世借着他儿子晓谕我们;又早已立他为承受万有的,也曾借着他创造诸世界。
\VS{3}他是 神荣耀所发的光辉,是 神本体的真像,常用他权能的命令托住万有。他洗净了人的罪,就坐在高天至大者的右边。
\VS{4}他所承受的名,既比天使的名更尊贵,就远超过天使。
\par }{\SH 儿子远超过天使
\par }{\PP \VS{5}所有的天使, 神从来对哪一个说:
\par }{\Q 你是我的儿子,
\par }{\Q 我今日生你?
\par }{\MM 又指着哪一个说:
\par }{\Q 我要作他的父,
\par }{\Q 他要作我的子?
\par }{\MM \VS{6}再者, 神使长子到世上来的时候\FTNT{}{{\FR 1:6: }或译: 神再使长子到世上来的时候},就说:
\par }{\Q  神的使者都要拜他。
\par }{\MM \VS{7}论到使者,又说:
\par }{\Q  神以风为使者,
\par }{\Q 以火焰为仆役;
\par }{\MM \VS{8}论到子却说:
\par }{\Q  神啊,你的宝座是永永远远的;
\par }{\Q 你的国权是正直的。
\par }{\Q \VS{9}你喜爱公义,恨恶罪恶;
\par }{\Q 所以 神,就是你的 神,用喜乐油膏你,
\par }{\Q 胜过膏你的同伴;
\par }{\Q \VS{10}又说:主啊,你起初立了地的根基;
\par }{\Q 天也是你手所造的。
\par }{\Q \VS{11}天地都要灭没,你却要长存。
\par }{\Q 天地都要像衣服渐渐旧了;
\par }{\Q \VS{12}你要将天地卷起来,像一件外衣,
\par }{\Q 天地就都改变了。
\par }{\Q 惟有你永不改变;
\par }{\Q 你的年数没有穷尽。
\par }{\PP \VS{13}所有的天使, 神从来对哪一个说:
\par }{\Q 你坐在我的右边,
\par }{\Q 等我使你仇敌作你的脚凳?
\par }{\MM \VS{14}天使岂不都是服役的灵、奉差遣为那将要承受救恩的人效力吗?

\par }\Chap{2}{\SH 这么大的救恩
\par }{\PP \VerseOne{1}所以,我们当越发郑重所听见的道理,恐怕我们随流失去。
\VS{2}那借着天使所传的话既是确定的;凡干犯悖逆的都受了该受的报应。
\VS{3}我们若忽略这么大的救恩,怎能逃罪呢?这救恩起先是主亲自讲的,后来是听见的人给我们证实了。
\VS{4}神又按自己的旨意,用神迹、奇事和百般的异能,并圣灵的恩赐,同他们作见证。
\par }{\SH 救恩的元帅
\par }{\PP \VS{5}我们所说将来的世界, 神原没有交给天使管辖。
\VS{6}但有人在{\ADD{经上}}某处证明说:
\par }{\Q 人算什么,你竟顾念他?
\par }{\Q 世人算什么,你竟眷顾他?
\par }{\Q \VS{7}你叫他比天使微小一点\FTNT{}{{\FR 2:7: }或译:你叫他暂时比天使小},
\par }{\Q 赐他荣耀尊贵为冠冕,
\par }{\Q 并将你手所造的都派他管理,
\par }{\Q \VS{8}叫万物都服在他的脚下。
\par }{\MM 既叫万物都服他,就没有剩下一样不服他的。只是如今我们还不见万物都服他。
\VS{9}惟独见那成为比天使小一点的耶稣\FTNT{}{{\FR 2:9: }或译:惟独见耶稣暂时比天使小};因为受死的苦,就得了尊贵荣耀为冠冕,叫他因着 神的恩,为人人尝了死味。
\VS{10}原来那为万物所属、为万物所本的,要领许多的儿子进荣耀里去,使救他们的元帅,因受苦难得以完全,本是合宜的。
\VS{11}因那使人成圣的和那些得以成圣的,都是出于一。所以,他称他们为弟兄也不以为耻,
\VS{12}说:
\par }{\Q 我要将你的名传与我的弟兄,
\par }{\Q 在会中我要颂扬你;
\par }{\PP \VS{13}又说:
\par }{\Q 我要倚赖他;
\par }{\MM 又说:
\par }{\Q 看哪,我与 神所给我的儿女。
\par }{\MM \VS{14}儿女既同有血肉之体,他也照样亲自成了血肉之体,特要借着死败坏那掌死权的,就是魔鬼,
\VS{15}并要释放那些一生因怕死而为奴仆的人。
\VS{16}他并不救拔天使,乃是救拔{\PN{亚伯拉罕}}的后裔。
\VS{17}所以,他凡事该与他的弟兄相同,为要在 神的事上成为慈悲忠信的大祭司,为百姓的罪献上挽回祭。
\VS{18}他自己既然被试探而受苦,就能搭救被试探的人。

\par }\Chap{3}{\SH 耶稣超越摩西
\par }{\PP \VerseOne{1}同蒙天召的圣洁弟兄啊,你们应当思想我们所认为使者、为大祭司的耶稣。
\VS{2}他为那设立他的尽忠,如同{\PN{摩西}}在 神的全家尽忠一样。
\VS{3}他比{\PN{摩西}}算是更配多得荣耀,好像建造房屋的比房屋更尊荣;
\VS{4}因为房屋都必有人建造,但建造万物的就是 神。
\VS{5}{\PN{摩西}}为仆人,在 神的全家诚然尽忠,为要证明将来必传说的事。
\VS{6}但基督为儿子,治理 神的家;我们若将可夸的盼望和胆量坚持到底,便是他的家了。
\par }{\SH  神子民的安息
\par }{\PP \VS{7}圣灵有话说:
\par }{\Q 你们今日若听他的话,
\par }{\Q \VS{8}就不可硬着心,
\par }{\Q 像在旷野惹他发怒、
\par }{\Q 试探他的时候一样。
\par }{\Q \VS{9}在那里,你们的祖宗试我探我,
\par }{\Q 并且观看我的作为有四十年之久。
\par }{\Q \VS{10}所以,我厌烦那世代的人,说:
\par }{\Q 他们心里常常迷糊,
\par }{\Q 竟不晓得我的作为!
\par }{\Q \VS{11}我就在怒中起誓说:
\par }{\Q 他们断不可进入我的安息。
\par }{\PP \VS{12}弟兄们,你们要谨慎,免得你们中间或有人存着不信的恶心,把永生 神离弃了。
\VS{13}总要趁着还有今日,天天彼此相劝,免得你们中间有人被罪迷惑,心里就刚硬了。
\VS{14}我们若将起初确实的信心坚持到底,就在基督里有分了。
\VS{15}{\ADD{经上}}说:「你们今日若听他的话,就不可硬着心,像惹他发怒的日子一样。」
\VS{16}那时,听见他话惹他发怒的是谁呢?岂不是跟着{\PN{摩西}}从{\PN{埃及}}出来的众人吗?
\VS{17}神四十年之久,又厌烦谁呢?岂不是那些犯罪、尸首倒在旷野的人吗?
\VS{18}又向谁起誓,不容他们进入他的安息呢?岂不是向那些不信从的人吗?
\VS{19}这样看来,他们不能进入{\ADD{安息}}是因为不信的缘故了。

\par }\Chap{4}{\PP \VerseOne{1}我们既蒙留下,有进入他安息的应许,就当畏惧,免得我们\FTNT{}{{\FR 4:1: }原文是你们}中间或有人似乎是赶不上了。
\VS{2}因为有福音传给我们,像传给他们一样;只是所听见的道与他们无益,因为他们没有信心与所听见的道调和。
\VS{3}但我们已经相信的人得以进入那安息,正如 神所说:「我在怒中起誓说:『他们断不可进入我的安息!』」其实造物之工,从创世以来已经成全了。
\VS{4}论到第七日,有一处说:「到第七日, 神就歇了他一切的工。」
\VS{5}又有一处说:「他们断不可进入我的安息!」
\VS{6}既有必进安息的人,那先前听见福音的,因为不信从,不得进去。
\VS{7}所以过了多年,就在{\PN{大卫}}的书上,又限定一日,如以上所引的说:「你们今日若听他的话,就不可硬着心。」
\VS{8}若是{\PN{约书亚}}已叫他们享了安息,后来 神就不再提别的日子了。
\VS{9}这样看来,必另有一安息日的安息为 神的子民存留。
\VS{10}因为那进入安息的,乃是歇了自己的工,正如 神歇了他的工一样。
\VS{11}所以,我们务必竭力进入那安息,免得有人学那不信从的样子跌倒了。
\par }{\PP \VS{12}神的道是活泼的,是有功效的,比一切两刃的剑更快,甚至魂与灵,骨节与骨髓,都能刺入、剖开,连心中的思念和主意都能辨明。
\VS{13}并且被造的没有一样在他面前不显然的;原来万物在那与我们有关系的主眼前,都是赤露敞开的。
\par }{\SH 大祭司耶稣
\par }{\PP \VS{14}我们既然有一位已经升入高天尊荣的大祭司,就是 神的儿子耶稣,便当持定所承认的道。
\VS{15}因我们的大祭司并非不能体恤我们的软弱。他也曾凡事受过试探,与我们一样,只是他没有犯罪。
\VS{16}所以,我们只管坦然无惧地来到施恩的宝座前,为要得怜恤,蒙恩惠,作随时的帮助。

\par }\Chap{5}{\PP \VerseOne{1}凡从人间挑选的大祭司,是奉派替人办理属 神的事,为要献上礼物和赎罪祭\FTNT{}{{\FR 5:1: }或译:要为罪献上礼物和祭物}。
\VS{2}他能体谅那愚蒙的和失迷的人,因为他自己也是被软弱所困。
\VS{3}故此,他理当为百姓和自己献祭赎罪。
\VS{4}这{\ADD{大祭司的}}尊荣,没有人自取。惟要蒙 神所召,像{\PN{亚伦}}一样。
\par }{\PP \VS{5}如此,基督也不是自取荣耀作大祭司,乃是在乎向他说「你是我的儿子,我今日生你」的那一位;
\VS{6}就如{\ADD{经上}}又有一处说:「你是照着{\PN{麦基洗德}}的等次永远为祭司。」
\VS{7}基督在肉体的时候,既大声哀哭,流泪祷告,恳求那能救他免死的主,就因他的虔诚蒙了应允。
\VS{8}他虽然为儿子,还是因所受的苦难学了顺从。
\VS{9}他既得以完全,就为凡顺从他的人成了永远得救的根源,
\VS{10}并蒙 神照着{\PN{麦基洗德}}的等次称他为大祭司。
\par }{\SH 警戒叛道的人
\par }{\PP \VS{11}论到{\PN{麦基洗德}},我们有好些话,并且难以解明,因为你们听不进去。
\VS{12}看你们学习的工夫,本该作师傅,谁知还得有人将 神圣言小学的开端另教导你们,并且成了那必须吃奶、不能吃干粮的人。
\VS{13}凡只能吃奶的都不熟练仁义的道理,因为他是婴孩;
\VS{14}惟独长大成人的才能吃干粮;他们的心窍习练得通达,就能分辨好歹了。

\par }\Chap{6}{\PP \VerseOne{1}所以,我们应当离开基督道理的开端,竭力进到完全的地步,不必再立根基,就如那懊悔死行、信靠 神、
\VS{2}各样洗礼、按手之礼、死人复活,以及永远审判各等教训。
\VS{3}神若许我们,我们必如此行。
\VS{4}论到那些已经蒙了光照、尝过天恩的滋味、又于圣灵有分,
\VS{5}并尝过 神善道的滋味、觉悟来世权能的人,
\VS{6}若是离弃{\ADD{道}}理,就不能叫他们从新懊悔了。因为他们把 神的儿子重钉十字架,明明地羞辱他。
\VS{7}就如一块田地,吃过屡次下的雨水,生长菜蔬,合乎耕种的人用,就从 神得福;
\VS{8}若长荆棘和蒺藜,必被废弃,近于咒诅,结局就是焚烧。
\VS{9}亲爱的弟兄们,我们虽是这样说,却深信你们的行为强过这些,而且近乎得救。
\VS{10}因为 神并非不公义,竟忘记你们所做的工和你们为他名所显的爱心,就是先前伺候圣徒,如今还是伺候。
\VS{11}我们愿你们各人都显出这样的殷勤,使你们有满足的指望,一直到底。
\VS{12}并且不懈怠,总要效法那些凭信心和忍耐承受应许的人。
\par }{\SH  神确切的应许
\par }{\PP \VS{13}当初 神应许{\PN{亚伯拉罕}}的时候,因为没有比自己更大可以指着起誓的,就指着自己起誓,说:
\VS{14}「论福,我必赐大福给你;论子孙,我必叫你的子孙多起来。」
\VS{15}这样,{\PN{亚伯拉罕}}既恒久忍耐,就得了所应许的。
\VS{16}人都是指着比自己大的起誓,并且以起誓为实据,了结各样的争论。
\VS{17}照样, 神愿意为那承受应许的人格外显明他的旨意是不更改的,就起誓为证。
\VS{18}借这两件不更改的事, 神决不能说谎,好叫我们这逃往避难所、持定摆在我们前头指望的人可以大得勉励。
\VS{19}我们有这指望,如同灵魂的锚,又坚固又牢靠,且通入幔内。
\VS{20}作先锋的耶稣,既照着{\PN{麦基洗德}}的等次成了永远的大祭司,就为我们进入幔内。

\par }\Chap{7}{\SH 麦基洗德的祭司等次
\par }{\PP \VerseOne{1}这{\PN{麦基洗德}}就是{\PN{撒冷}}王,又是至高 神的祭司,本是长远为祭司的。他当{\PN{亚伯拉罕}}杀败诸王回来的时候,就迎接他,给他祝福。
\VS{2}{\PN{亚伯拉罕}}也将自己所得来的,取十分之一给他。他头一个名翻出来就是仁义王,他又名{\PN{撒冷}}王,就是平安王的意思。
\VS{3}他无父,无母,无族谱,无生之始,无命之终,乃是与 神的儿子相似。
\par }{\PP \VS{4}你们想一想,先祖{\PN{亚伯拉罕}}将自己所掳来上等之物取十分之一给他,这人是何等尊贵呢!
\VS{5}那得祭司职任的{\PN{利未}}子孙,领命照例向百姓取十分之一,这百姓是自己的弟兄,虽是从{\PN{亚伯拉罕}}身\FTNT{}{{\FR 7:5: }原文是腰}中生的,还是照例{\ADD{取十分之一}}。
\VS{6}独有{\PN{麦基洗德}},不与他们同谱,倒收纳{\PN{亚伯拉罕}}的十分之一,为那蒙应许的{\PN{亚伯拉罕}}祝福。
\VS{7}从来位分大的给位分小的祝福,这是驳不倒的理。
\VS{8}在这里收十分之一的都是必死的人;但在那里收十分之一的,有为他作见证的说,他是活的;
\VS{9}并且可说那受十分之一的{\PN{利未}},也是借着{\PN{亚伯拉罕}}纳了十分之一。
\VS{10}因为{\PN{麦基洗德}}迎接{\PN{亚伯拉罕}}的时候,{\PN{利未}}已经在他先祖的身\FTNT{}{{\FR 7:10: }原文是腰}中。
\par }{\PP \VS{11}从前百姓在{\PN{利未}}人祭司职任以下受律法,倘若借这职任能得完全,又何用另外兴起一位祭司,照{\PN{麦基洗德}}的等次,不照{\PN{亚伦}}的等次呢?
\VS{12}祭司的职任既已更改,律法也必须更改。
\VS{13}因为这话所指的人本属别的支派,那支派里从来没有一人伺候祭坛。
\VS{14}我们的主分明是从{\PN{犹大}}出来的;但这支派,{\PN{摩西}}并没有提到祭司。
\VS{15}倘若照{\PN{麦基洗德}}的样式,另外兴起一位祭司来,{\ADD{我的话}}更是显而易见的了。
\VS{16}他成为祭司,并不是照属肉体的条例,乃是照无穷\FTNT{}{{\FR 7:16: }原文是不能毁坏}之生命的大能。
\VS{17}因为有给他作见证的说:「你是照着{\PN{麦基洗德}}的等次永远为祭司。」
\VS{18}先前的条例,因软弱无益,所以废掉了,
\VS{19}(律法原来一无所成)就引进了更美的指望;靠这指望,我们便可以进到 神面前。
\par }{\PP \VS{20}再者,{\ADD{耶稣为祭司}},并不是不起誓立的。
\VS{21}至于那些祭司,原不是起誓立的,只有耶稣是起誓立的;因为那立他的对他说:「主起了誓,决不后悔,你是永远为祭司。」
\VS{22}既是起誓立的,耶稣就作了更美之约的中保。
\VS{23}那些成为祭司的,数目本来多,是因为有死阻隔,不能长久。
\VS{24}这位既是永远常存的,他祭司的职任就长久不更换。
\VS{25}凡靠着他进到 神面前的人,他都能拯救到底;因为他是长远活着,替他们祈求。
\par }{\PP \VS{26}像这样圣洁、无邪恶、无玷污、远离罪人、高过诸天的大祭司,原是与我们合宜的。
\VS{27}他不像那些大祭司,每日必须先为自己的罪,后为百姓的罪献祭;因为他只一次将自己献上,就把这事成全了。
\VS{28}律法本是立软弱的人为大祭司;但在律法以后起誓的话,是立儿子为大祭司,乃是成全到永远的。

\par }\Chap{8}{\SH 更美之约的大祭司
\par }{\PP \VerseOne{1}我们所讲的事,其中第一要紧的,就是我们有这样的大祭司,已经坐在天上至大者宝座的右边,
\VS{2}在圣所,就是真帐幕里,作执事;这帐幕是主所支的,不是人所支的。
\VS{3}凡大祭司都是为献礼物和祭物设立的,所以这位大祭司也必须有所献的。
\VS{4}他若在地上,必不得为祭司,因为已经有照律法献礼物的{\ADD{祭司}}。
\VS{5}他们供奉的事本是天上事的形状和影像,正如{\PN{摩西}}将要造帐幕的时候,蒙 {\ADD{神}}警戒他,说:「你要谨慎,作各样的物件都要照着在山上指示你的样式。」
\par }{\PP \VS{6}如今耶稣所得的职任是更美的,正如他作更美之约的中保;这约原是凭更美之应许立的。
\VS{7}那前约若没有瑕疵,就无处寻求后约了。
\VS{8}所以主指责他的百姓说\FTNT{}{{\FR 8:8: }或译:所以主指前约的缺欠说}:
\par }{\Q 日子将到,
\par }{\Q 我要与{\PN{以色列}}家和{\PN{犹大}}家另立新约,
\par }{\Q \VS{9}不像我拉着他们祖宗的手,
\par }{\Q 领他们出{\PN{埃及}}的时候,
\par }{\Q 与他们所立的约。
\par }{\Q 因为他们不恒心守我的约,
\par }{\Q 我也不理他们。
\par }{\Q 这是主说的。
\par }{\Q \VS{10}主又说:
\par }{\Q 那些日子以后,
\par }{\Q 我与{\PN{以色列}}家所立的约乃是这样:
\par }{\Q 我要将我的律法放在他们里面,
\par }{\Q 写在他们心上;
\par }{\Q 我要作他们的 神;
\par }{\Q 他们要作我的子民。
\par }{\Q \VS{11}他们不用各人教导自己的乡邻和自己的弟兄,
\par }{\Q 说:你该认识主;
\par }{\Q 因为他们从最小的到至大的,都必认识我。
\par }{\Q \VS{12}我要宽恕他们的不义,
\par }{\Q 不再记念他们的罪愆。
\par }{\MM \VS{13}既说新约,就以前约为旧了;但那渐旧渐衰的,就必快归无有了。

\par }\Chap{9}{\SH 地上和天上的圣所
\par }{\PP \VerseOne{1}原来前约有礼拜的条例和属世界的圣幕。
\VS{2}因为有预备的帐幕,头一层叫作圣所,里面有灯台、桌子,和陈设饼。
\VS{3}第二幔子后又有一层帐幕,叫作至圣所,
\VS{4}有金香炉\FTNT{}{{\FR 9:4: }炉:或译坛},有包金的约柜,柜里有盛吗哪的金罐和{\PN{亚伦}}发过芽的杖,并两块约版;
\VS{5}柜上面有荣耀基路伯的影罩着施恩\FTNT{}{{\FR 9:5: }施恩:原文是蔽罪}座。这几件我现在不能一一细说。
\par }{\PP \VS{6}这些物件既如此预备齐了,众祭司就常进头一层帐幕,行拜 神的礼。
\VS{7}至于第二层帐幕,惟有大祭司一年一次独自进去,没有不带着血为自己和百姓的过错献上。
\VS{8}圣灵用此指明,头一层帐幕仍存的时候,进入至圣所的路还未显明。
\VS{9}那头一层帐幕作现今的一个表样,所献的礼物和祭物,就着良心说,都不能叫礼拜的人得以完全。
\VS{10}这些事,连那饮食和诸般洗濯的规矩,都不过是属肉体的条例,命定到振兴的时候为止。
\par }{\PP \VS{11}但现在基督已经来到,作了将来美事的大祭司,经过那更大更全备的帐幕,不是人手所造,也不是属乎这世界的;
\VS{12}并且不用山羊和牛犊的血,乃用自己的血,只一次进入圣所,成了永远赎罪的事。
\VS{13}若山羊和公牛的血,并母牛犊的灰,洒在不洁的人身上,尚且叫人成圣,身体洁净,
\VS{14}何况基督借着永远的灵,将自己无瑕无疵献给 神,他的血岂不更能洗净你们的心\FTNT{}{{\FR 9:14: }原文是良心},除去你们的死行,使你们事奉那永生 神吗?
\par }{\PP \VS{15}为此,他作了新约的中保,既然受死赎了人在前约之时所犯的罪过,便叫蒙召之人得着所应许永远的产业。
\VS{16}凡有遗命必须等到留遗命\FTNT{}{{\FR 9:16: }遗命:原文与约字同}的人死了;
\VS{17}因为人死了,遗命才有效力,若留遗命的尚在,那遗命还有用处吗?
\VS{18}所以,前约也不是不用血立的;
\VS{19}因为{\PN{摩西}}当日照着律法将各样诫命传给众百姓,就拿朱红色绒和牛膝草,把牛犊山羊的血和水洒在书上,又洒在众百姓身上,说:
\VS{20}「这血就是 神与你们立约{\ADD{的凭据}}。」
\VS{21}他又照样把血洒在帐幕和各样器皿上。
\VS{22}按着律法,凡物差不多都是用血洁净的;若不流血,罪就不得赦免了。
\par }{\SH 基督献己为祭除掉了罪
\par }{\PP \VS{23}照着天上样式做的物件必须用这些祭物去洁净;但那天上的本物自然当用更美的祭物去洁净。
\VS{24}因为基督并不是进了人手所造的圣所(这不过是真圣所的影相),乃是进了天堂,如今为我们显在 神面前;
\VS{25}也不是多次将自己献上,像那大祭司每年带着牛羊的血\FTNT{}{{\FR 9:25: }牛羊的血:原文作不是自己的血}进入圣所,
\VS{26}如果这样,他从创世以来,就必多次受苦了。但如今在这末世显现一次,把自己献为祭,好除掉罪。
\VS{27}按着定命,人人都有一死,死后且有审判。
\VS{28}像这样,基督既然一次被献,担当了多人的罪,将来要向那等候他的人第二次显现,并与罪无关,乃是为拯救他们。

\par }\Chap{10}{\PP \VerseOne{1}律法既是将来美事的影儿,不是本物的真像,总不能借着每年常献一样的祭物叫那近前来的人得以完全。
\VS{2}若不然,献祭的事岂不早已止住了吗?因为礼拜的人,良心既被洁净,就不再觉得有罪了。
\VS{3}但这些{\ADD{祭物}}是叫人每年想起罪来;
\VS{4}因为公牛和山羊的血,断不能除罪。
\par }{\PP \VS{5}所以基督到世上来的时候,就说:
\par }{\Q  {\ADD{神啊}},祭物和礼物是你不愿意的;
\par }{\Q 你曾给我预备了身体。
\par }{\Q \VS{6}燔祭和赎罪{\ADD{祭}}是你不喜欢的。
\par }{\Q \VS{7}那时我说: 神啊,我来了,
\par }{\Q 为要照你的旨意行;
\par }{\Q 我的事在经卷上已经记载了。
\par }{\MM \VS{8}以上说:「祭物和礼物,燔祭和赎罪祭,是你不愿意的,也是你不喜欢的(这都是按着律法献的)」;
\VS{9}后又说:「我来了为要照你的旨意行」;{\ADD{可见}}他是除去在先的,为要立定在后的。
\VS{10}我们凭这旨意,靠耶稣基督,只一次献上他的身体,就得以成圣。
\par }{\PP \VS{11}凡祭司天天站着事奉 {\ADD{神}},屡次献上一样的祭物,这祭物永不能除罪。
\VS{12}但基督献了一次永远的赎罪祭,就在 神的右边坐下了。
\VS{13}从此,等候他仇敌成了他的脚凳。
\VS{14}因为他一次献祭,便叫那得以成圣的人永远完全。
\VS{15}圣灵也对我们作见证;因为他既已说过:
\par }{\Q \VS{16}主说:那些日子以后,
\par }{\Q 我与他们所立的约乃是这样:
\par }{\Q 我要将我的律法写在他们心上,
\par }{\Q 又要放在他们的里面。
\par }{\Q \VS{17}{\ADD{以后就说}}:
\par }{\Q 我不再记念他们的罪愆和他们的过犯。
\par }{\Q \VS{18}这些罪过既已赦免,就不用再为罪献祭了。
\par }{\SH 劝勉和警告
\par }{\PP \VS{19}弟兄们,我们既因耶稣的血得以坦然进入至圣所,
\VS{20}是借着他给我们开了一条又新又活的路,从幔子经过,这幔子就是他的身体。
\VS{21}又有一位大祭司治理 神的家,
\VS{22}并我们心中天良的亏欠已经洒去,身体用清水洗净了,就当存着诚心和充足的信心来到 神面前;
\VS{23}也要坚守我们所承认的指望,不至摇动,因为那应许我们的是信实的。
\VS{24}又要彼此相顾,激发爱心,勉励行善。
\VS{25}你们不可停止聚会,好像那些停止惯了的人,倒要彼此劝勉,既知道\FTNT{}{{\FR 10:25: }原文是看见}那日子临近,就更当如此。
\par }{\PP \VS{26}因为我们得知真道以后,若故意犯罪,赎罪的祭就再没有了;
\VS{27}惟有战惧等候审判和那烧灭众敌人的烈火。
\VS{28}人干犯{\PN{摩西}}的律法,凭两三个见证人,尚且不得怜恤而死,
\VS{29}何况人践踏 神的儿子,将那使他成圣之约的血当作平常,又亵慢施恩的{\ADD{圣}}灵,你们想,他要受的刑罚该怎样加重呢!
\VS{30}因为我们知道谁说:「伸冤在我,我必报应」;又{\ADD{说}}:「主要审判他的百姓。」
\VS{31}落在永生 神的手里,真是可怕的!
\par }{\PP \VS{32}你们要追念往日,蒙了光照以后所忍受大争战的各样苦难:
\VS{33}一面被毁谤,遭患难,成了戏景,叫众人观看;一面陪伴那些受这样苦难的人。
\VS{34}因为你们体恤了那些被捆锁的人,并且你们的家业被人抢去,也甘心忍受,知道自己有更美长存的家业。
\VS{35}所以,你们不可丢弃勇敢的心;存这样的心必得大赏赐。
\VS{36}你们必须忍耐,使你们行完了 神的旨意,就可以得着所应许的。
\par }{\Q \VS{37}因为还有一点点时候,
\par }{\Q 那要来的就来,并不迟延;
\par }{\Q \VS{38}只是义人\FTNT{}{{\FR 10:38: }有古卷:我的义人}必因信得生。
\par }{\Q 他若退后,我心里就不喜欢他。
\par }{\PP \VS{39}我们却不是退后入沉沦的那等人,乃是有信心以致灵魂得救的人。

\par }\Chap{11}{\SH 论信
\par }{\PP \VerseOne{1}信就是所望之事的实底,是未见之事的确据。
\VS{2}古人在这信上得了美好的证据。
\par }{\PP \VS{3}我们因着信,就知道诸世界是借 神的话造成的;这样,所看见的,并不是从显然之物造出来的。
\par }{\PP \VS{4}{\PN{亚伯}}因着信,献祭与 神,比{\PN{该隐}}所献的更美,因此便得了称义的见证,就是 神指他礼物作的见证。他虽然死了,却因这信,仍旧说话。
\VS{5}{\PN{以诺}}因着信,被接去,不至于见死,人也找不着他,因为 神已经把他接去了;只是他被接去以先,已经得了 神喜悦他的明证。
\VS{6}人非有信,就不能得 神的喜悦;因为到 神面前来的人必须信有 神,且信他赏赐那寻求他的人。
\VS{7}{\PN{挪亚}}因着信,既蒙 {\ADD{神}}指示他未见的事,动了敬畏的心,预备了一只方舟,使他全家得救。因此就定了那世代的罪,自己也承受了那从信而来的义。
\par }{\PP \VS{8}{\PN{亚伯拉罕}}因着信,蒙召的时候就遵命出去,往将来要得为业的地方去;出去的时候,还不知往哪里去。
\VS{9}他因着信,就在所应许之地作客,好像在异地居住帐棚,与那同蒙一个应许的{\PN{以撒}}、{\PN{雅各}}一样。
\VS{10}因为他等候那座有根基的城,就是 神所经营所建造的。
\VS{11}因着信,连{\PN{撒拉}}自己,虽然过了生育的岁数,还能怀孕,因她以为那应许她的是可信的。
\VS{12}所以从一个仿佛已死的人就生出{\ADD{子孙}},如同天上的星那样众多,海边的沙那样无数。
\par }{\PP \VS{13}这些人都是存着信心死的,并没有得着所应许的;却从远处望见,且欢喜迎接,又承认自己在世上是客旅,是寄居的。
\VS{14}说这样话的人是表明自己要找一个家乡。
\VS{15}他们若想念所离开的家乡,还有可以回去的机会。
\VS{16}他们却羡慕一个更美的家乡,就是在天上的。所以 神被称为他们的 神,并不以为耻,因为他已经给他们预备了一座城。
\par }{\PP \VS{17}{\PN{亚伯拉罕}}因着信,被试验的时候,就把{\PN{以撒}}献上;这便是那欢喜领受应许的,将自己独生的儿子献上。
\VS{18}{\ADD{论到这儿子}},曾有话说:「从{\PN{以撒}}生的才要称为你的后裔。」
\VS{19}他以为 神还能叫人从死里复活;他也仿佛从死中得回他的儿子来。
\VS{20}{\PN{以撒}}因着信,就指着将来的事给{\PN{雅各}}、{\PN{以扫}}祝福。
\VS{21}{\PN{雅各}}因着信,临死的时候,给{\PN{约瑟}}的两个儿子各自祝福,扶着杖头敬拜 神。
\VS{22}{\PN{约瑟}}因着信,临终的时候,提到{\PN{以色列}}族将来要出{\ADD{
{\PN{埃及}}}},并为自己的骸骨留下遗命。
\par }{\PP \VS{23}{\PN{摩西}}生下来,他的父母见他是个俊美的孩子,就因着信,把他藏了三个月,并不怕王命。
\VS{24}{\PN{摩西}}因着信,长大了就不肯称为法老女儿之子。
\VS{25}他宁可和 神的百姓同受苦害,也不愿暂时享受罪中之乐。
\VS{26}他看为基督受的凌辱比{\PN{埃及}}的财物更宝贵,因他想望所要得的赏赐。
\VS{27}他因着信,就离开{\PN{埃及}},不怕王怒;因为他恒心忍耐,如同看见那不能看见的{\ADD{主}}。
\VS{28}他因着信,就守\FTNT{}{{\FR 11:28: }或译:立}逾越节,行洒血的礼,免得那灭长子的临近{\PN{以色列}}人。
\VS{29}他们因着信,过{\PN{红海}}如行干地;{\PN{埃及}}人试着要过去,就被吞灭了。
\VS{30}{\PN{以色列}}人因着信,围绕{\PN{耶利哥}}城七日,城墙就倒塌了。
\VS{31}妓女{\PN{喇合}}因着信,曾和和平平地接待探子,就不与那些不顺从的人一同灭亡。
\par }{\PP \VS{32}我又何必再说呢?若要一一细说,{\PN{基甸}}、{\PN{巴拉}}、{\PN{参孙}}、{\PN{耶弗他}}、{\PN{大卫}}、{\PN{撒母耳}},和众先知的事,时候就不够了。
\VS{33}他们因着信,制伏了敌国,行了公义,得了应许,堵了狮子的口,
\VS{34}灭了烈火的猛势,脱了刀剑的锋刃;软弱变为刚强,争战显出勇敢,打退外邦的全军。
\VS{35}有妇人得自己的死人复活。又有人忍受严刑,不肯苟且得释放\FTNT{}{{\FR 11:35: }原文是赎},为要得着更美的复活。
\VS{36}又有人忍受戏弄、鞭打、捆锁、监禁、各等的磨炼,
\VS{37}被石头打死,被锯锯死,受试探,被刀杀,披着绵羊山羊的皮各处奔跑,受穷乏、患难、苦害,
\VS{38}在旷野、山岭、山洞、地穴,飘流无定,本是世界不配有的人。
\par }{\PP \VS{39}这些人都是因信得了美好的证据,却仍未得着所应许的;
\VS{40}因为 神给我们预备了更美的事,叫他们若不与我们同得,就不能完全。

\par }\Chap{12}{\SH 主的管教
\par }{\PP \VerseOne{1}我们既有这许多的见证人,如同云彩围着我们,就当放下各样的重担,脱去容易缠累我们的罪,存心忍耐,奔那摆在我们前头的路程,
\VS{2}仰望为我们信心创始成终的耶稣\FTNT{}{{\FR 12:2: }或译:仰望那将真道创始成终的耶稣}。他因那摆在前面的喜乐,就轻看羞辱,忍受了十字架{\ADD{的苦难}},便坐在 神宝座的右边。
\VS{3}那忍受罪人这样顶撞的,你们要思想,免得疲倦灰心。
\par }{\PP \VS{4}你们与罪恶相争,还没有抵挡到流血的地步。
\VS{5}你们又忘了那劝你们如同劝儿子的话,说:
\par }{\Q 我儿,你不可轻看主的管教,
\par }{\Q 被他责备的时候也不可灰心;
\par }{\Q \VS{6}因为主所爱的,他必管教,
\par }{\Q 又鞭打凡所收纳的儿子。
\par }{\MM \VS{7}你们所忍受的,是 神管教你们,待你们如同待儿子。焉有儿子不被父亲管教的呢?
\VS{8}管教原是众子所共受的。你们若不受管教,就是私子,不是儿子了。
\VS{9}再者,我们曾有生身的父管教我们,我们尚且敬重他,何况万灵的父,我们岂不更当顺服他得生吗?
\VS{10}生身的父都是暂随己意管教我们;惟有万灵的父管教我们,是要我们得益处,使我们在他的圣洁上有分。
\VS{11}凡管教的事,当时不觉得快乐,反觉得愁苦;后来却为那经练过的人结出平安的果子,就是义。
\par }{\PP \VS{12}所以,你们要把下垂的手、发酸的腿挺起来;
\VS{13}也要为自己的脚,把道路修直了,使瘸子不致歪脚\FTNT{}{{\FR 12:13: }或译:差路},反得痊愈。
\par }{\SH 警告弃绝 神恩典的人
\par }{\PP \VS{14}你们要追求与众人和睦,并要追求圣洁;非圣洁没有人能见主。
\VS{15}又要谨慎,恐怕有人失了 神的恩;恐怕有毒根生出来扰乱你们,因此叫众人沾染污秽;
\VS{16}恐怕有淫乱的,有贪恋世俗如{\PN{以扫}}的,他因一点食物把自己长子的名分卖了。
\VS{17}后来想要承受父所祝的福,竟被弃绝,虽然号哭切求,却得不着门路使他{\ADD{父亲}}的心意回转。这是你们知道的。
\par }{\PP \VS{18}你们原不是来到那能摸的山;此山有火焰、密云、黑暗、暴风、
\VS{19}角声与说话的声音。那些听见这声音的,都求不要再向他们说话;
\VS{20}因为他们当不起所命他们的话,说:「靠近这山的,即便是走兽,也要用石头打死。」
\VS{21}所见的极其可怕,甚至{\PN{摩西}}说:「我甚是恐惧战兢。」
\VS{22}你们乃是来到{\PN{锡安山}},永生 神的城邑,就是天上的{\PN{耶路撒冷}}。那里有千万的天使,
\VS{23}有名录在天上诸长子之会所共聚的总会,有审判众人的 神和被成全之义人的灵魂,
\VS{24}并新约的中保耶稣,以及所洒的血;这血所说的比{\PN{亚伯}}的血所说的更美。
\par }{\PP \VS{25}你们总要谨慎,不可弃绝那向你们说话的。因为,那些弃绝在地上警戒他们的尚且不能逃罪,何况我们违背那从天上警戒我们的呢?
\VS{26}当时他的声音震动了地,但如今他应许说:「再一次我不单要震动地,还要震动天。」
\VS{27}这再一次的话,是指明被震动的,就是受造之物都要挪去,使那不被震动的常存。
\VS{28}所以我们既得了不能震动的国,就当感恩,照 神所喜悦的,用虔诚、敬畏的心事奉 神。
\VS{29}因为我们的 神乃是烈火。

\par }\Chap{13}{\SH 蒙 神悦纳的服务
\par }{\PP \VerseOne{1}你们务要常存弟兄相爱的心。
\VS{2}不可忘记用爱心接待客旅;因为曾有接待客旅的,不知不觉就接待了天使。
\VS{3}你们要记念被捆绑的人,好像与他们同受捆绑;也要记念遭苦害的人,想到自己也在肉身之内。
\VS{4}婚姻,人人都当尊重,床也不可污秽;因为苟合行淫的人, 神必要审判。
\VS{5}你们存心不可贪爱钱财,要以自己所有的为足;因为主曾说:「我总不撇下你,也不丢弃你。」
\VS{6}所以我们可以放胆说:
\par }{\Q 主是帮助我的,我必不惧怕;
\par }{\Q 人能把我怎么样呢?
\par }{\PP \VS{7}从前引导你们、传 神之道给你们的人,你们要想念他们,效法他们的信心,留心看他们为人的结局。
\VS{8}耶稣基督昨日、今日、一直到永远,是一样的。
\VS{9}你们不要被那诸般怪异的教训勾引了去;因为人心靠恩得坚固才是好的,并不是靠饮食。那在饮食上专心的从来没有得着益处。
\VS{10}我们有一祭坛,上面的祭物是那些在帐幕中供职的人不可同吃的。
\VS{11}原来牲畜的血被大祭司带入圣所作赎罪祭;牲畜的身子被烧在营外。
\VS{12}所以,耶稣要用自己的血叫百姓成圣,也就在城门外受苦。
\VS{13}这样,我们也当出到营外,就了他去,忍受他所受的凌辱。
\VS{14}我们在这里本没有常存的城,乃是寻求那将来的城。
\VS{15}我们应当靠着耶稣,常常以颂赞为祭献给 神,这就是那承认主名之人嘴唇的果子。
\VS{16}只是不可忘记行善和捐输的事,因为这样的祭是 神所喜悦的。
\par }{\PP \VS{17}你们要依从那些引导你们的,且要顺服;因他们为你们的灵魂时刻警醒,好像那将来交帐的人。你们要使他们交的时候有快乐,不致忧愁;若忧愁就与你们无益了。
\par }{\PP \VS{18}请你们为我们祷告,因我们自觉良心无亏,愿意凡事按正道而行。
\VS{19}我更求你们为我祷告,使我快些回到你们那里去。
\par }{\SH 祝福和问安
\par }{\PP \VS{20}但愿赐平安的 神,就是那凭永约之血、使群羊的大牧人—我主耶稣从死里复活的 神,
\VS{21}在各样善事上成全你们,叫你们遵行他的旨意;又借着耶稣基督在你们心里行他所喜悦的事。愿荣耀归给他,直到永永远远。阿们!
\par }{\PP \VS{22}弟兄们,我略略写信给你们,望你们听我劝勉的话。
\VS{23}你们该知道,我们的兄弟{\PN{提摩太}}已经释放了;他若快来,我必同他去见你们。
\par }{\PP \VS{24}请你们问引导你们的诸位和众圣徒安。从{\PN{意大利}}来的人也问你们安。
\VS{25}愿恩惠{\ADD{常}}与你们众人同在。阿们!
\par }