\NormalFont\ShortTitle{马太福音}
{\MT 马太福音

\par }\ChapOne{1}{\SH 耶稣基督的家谱
\par }{\R (路3·23—38)
\par }{\PP \VerseOne{1}{\PN{亚伯拉罕}}的后裔,{\PN{大卫}}的子孙\FTNT{}{{\FR 1:1: }后裔,子孙:原文是儿子;下同},耶稣基督的家谱:
\par }{\PP \VS{2}{\PN{亚伯拉罕}}生{\PN{以撒}};{\PN{以撒}}生{\PN{雅各}};{\PN{雅各}}生{\PN{犹大}}和他的弟兄;
\VS{3}{\PN{犹大}}从{\PN{她玛}}氏生{\PN{法勒斯}}和{\PN{谢拉}};{\PN{法勒斯}}生{\PN{希斯
}};{\PN{希斯
}}生{\PN{亚兰}};
\VS{4}{\PN{亚兰}}生{\PN{亚米拿达}};{\PN{亚米拿达}}生{\PN{拿顺}};{\PN{拿顺}}生{\PN{撒门}};
\VS{5}{\PN{撒门}}从{\PN{喇合}}氏生{\PN{波阿斯}};{\PN{波阿斯}}从{\PN{路得}}氏生{\PN{俄备得}};{\PN{俄备得}}生{\PN{耶西}};
\VS{6}{\PN{耶西}}生{\PN{大卫}}王。
\par }{\PP {\PN{大卫}}从{\PN{乌利亚}}的妻子生{\PN{所罗门}};
\VS{7}{\PN{所罗门}}生{\PN{罗波安}};{\PN{罗波安}}生{\PN{亚比雅}};{\PN{亚比雅}}生{\PN{亚撒}};
\VS{8}{\PN{亚撒}}生{\PN{约沙法}};{\PN{约沙法}}生{\PN{约兰}};
{\PN{约兰}}生{\PN{乌西雅}};
\VS{9}{\PN{乌西雅}}生{\PN{约坦}};{\PN{约坦}}生{\PN{亚哈斯}};{\PN{亚哈斯}}生{\PN{希西家}};
\VS{10}{\PN{希西家}}生{\PN{玛拿西}};{\PN{玛拿西}}生{\PN{亚们}};{\PN{亚们}}生{\PN{约西亚}};
\VS{11}百姓被迁到{\PN{巴比伦}}的时候,{\PN{约西亚}}生{\PN{耶哥尼雅}}和他的弟兄。
\par }{\PP \VS{12}迁到{\PN{巴比伦}}之后,{\PN{耶哥尼雅}}生{\PN{撒拉铁}};{\PN{撒拉铁}}生{\PN{所罗巴伯}};
\VS{13}{\PN{所罗巴伯}}生{\PN{亚比玉}};{\PN{亚比玉}}生{\PN{以利亚敬}};{\PN{以利亚敬}}生{\PN{亚所}};
\VS{14}{\PN{亚所}}生{\PN{撒督}};{\PN{撒督}}生{\PN{亚金}};{\PN{亚金}}生{\PN{以律}};
\VS{15}{\PN{以律}}生{\PN{以利亚撒}};{\PN{以利亚撒}}生{\PN{马但}};{\PN{马但}}生{\PN{雅各}};
\VS{16}{\PN{雅各}}生{\PN{约瑟}},就是{\PN{马利亚}}的丈夫。那称为基督的耶稣是从{\PN{马利亚}}生的。
\par }{\PP \VS{17}这样,从{\PN{亚伯拉罕}}到{\PN{大卫}}共有十四代;从{\PN{大卫}}到迁至{\PN{巴比伦}}的时候也有十四代;从迁至{\PN{巴比伦}}的时候到基督又有十四代。
\par }{\SH 耶稣基督降生
\par }{\R (路2·1—7)
\par }{\PP \VS{18}耶稣基督降生的事记在下面:他母亲{\PN{马利亚}}已经许配了{\PN{约瑟}},还没有迎娶,{\PN{马利亚}}就从圣灵怀了孕。
\VS{19}她丈夫{\PN{约瑟}}是个义人,不愿意明明地羞辱她,想要暗暗地把她休了。
\VS{20}正思念这事的时候,有主的使者向他梦中显现,{\ADD{说}}:「{\PN{大卫}}的子孙{\PN{约瑟}},不要怕!只管娶过你的妻子{\PN{马利亚}}来,因她所怀的孕是从圣灵来的。
\VS{21}她将要生一个儿子,你要给他起名叫耶稣,因他要将自己的百姓从罪恶里救出来。」
\VS{22}这一切的事成就是要应验主借先知所说的话,
\VS{23}说:
\par }{\Q 必有童女怀孕生子;
\par }{\Q 人要称他的名为{\PN{以马内利}}。
\par }{\MM ({\PN{以马内利}}翻出来就是「 神与我们同在」。)
\VS{24}{\PN{约瑟}}醒了,起来,就遵着主使者的吩咐把妻子娶过来;
\VS{25}只是没有和她同房,等她生了儿子\FTNT{}{{\FR 1:25: }有古卷:等她生了头胎的儿子},就给他起名叫耶稣。

\par }\Chap{2}{\SH 博士朝拜
\par }{\PP \VerseOne{1}当{\PN{希律}}王的时候,耶稣生在{\PN{犹太}}的{\PN{伯利恒}}。有几个博士从东方来到{\PN{耶路撒冷}},{\ADD{说}}:
\VS{2}「那生下来作{\PN{犹太}}人之王的在哪里?我们在东方看见他的星,特来拜他。」
\VS{3}{\PN{希律}}王听见了,就心里不安;{\PN{耶路撒冷}}合城的人也都不安。
\VS{4}他就召齐了祭司长和民间的文士,问他们说:「基督当生在何处?」
\VS{5}他们回答说:「在{\PN{犹太}}的{\PN{伯利恒}}。因为有先知记着,说:
\par }{\PP \VS{6}{\PN{犹大}}地的{\PN{伯利恒}}啊,
\par }{\Q 你在{\PN{犹大}}诸城中并不是最小的;
\par }{\Q 因为将来有一位君王要从你那里出来,
\par }{\Q 牧养我{\PN{以色列}}民。」
\par }{\PP \VS{7}当下,{\PN{希律}}暗暗地召了博士来,细问那星是什么时候出现的,
\VS{8}就差他们往{\PN{伯利恒}}去,说:「你们去仔细寻访那小孩子,寻到了就来报信,我也好去拜他。」
\VS{9}他们听见王的话就去了。在东方所看见的那星忽然在他们前头行,直行到小孩子的地方,就在上头停住了。
\VS{10}他们看见那星,就大大地欢喜;
\VS{11}进了房子,看见小孩子和他母亲{\PN{马利亚}},就俯伏拜那小孩子,揭开宝盒,拿黄金、乳香、没药为礼物献给他。
\VS{12}博士因为在梦中被{\ADD{主}}指示不要回去见{\PN{希律}},就从别的路回本地去了。
\par }{\SH 逃往埃及
\par }{\PP \VS{13}他们去后,有主的使者向{\PN{约瑟}}梦中显现,说:「起来!带着小孩子同他母亲逃往{\PN{埃及}},住在那里,等我吩咐你;因为{\PN{希律}}必寻找小孩子,要除灭他。」
\VS{14}{\PN{约瑟}}就起来,夜间带着小孩子和他母亲往{\PN{埃及}}去,
\VS{15}住在那里,直到{\PN{希律}}死了。这是要应验主借先知所说的话,说:「我从{\PN{埃及}}召出我的儿子来。」
\par }{\SH 屠杀男孩
\par }{\PP \VS{16}{\PN{希律}}见自己被博士愚弄,就大大发怒,差人将{\PN{伯利恒}}城里并四境所有的男孩,照着他向博士仔细查问的时候,凡两岁以里的,都杀尽了。
\VS{17}这就应了先知{\PN{耶利米}}的话,说:
\par }{\PP \VS{18}在{\PN{拉玛}}听见号咷大哭的声音,
\par }{\Q 是{\PN{拉结}}哭她儿女,
\par }{\Q 不肯受安慰,
\par }{\Q 因为他们都不在了。
\par }{\SH 从埃及回来
\par }{\PP \VS{19}{\PN{希律}}死了以后,有主的使者在{\PN{埃及}}向{\PN{约瑟}}梦中显现,说:
\VS{20}「起来!带着小孩子和他母亲往{\PN{以色列}}地去,因为要害小孩子性命的人已经死了。」
\VS{21}{\PN{约瑟}}就起来,把小孩子和他母亲带到{\PN{以色列}}地去;
\VS{22}只因听见{\PN{亚基老}}接着他父亲{\PN{希律}}作了{\PN{犹太}}王,就怕往那里去,又在梦中被{\ADD{主}}指示,便往{\PN{加利利}}境内去了,
\VS{23}到了一座城,名叫{\PN{拿撒勒}},就住在那里。这是要应验先知所说,他将称为{\PN{拿撒勒}}人的话了。

\par }\Chap{3}{\SH 施洗约翰传道
\par }{\R (可1·1—8;路3·1—18;约1·19—28)
\par }{\PP \VerseOne{1}那时,有施洗的{\PN{约翰}}出来,在{\PN{犹太}}的旷野传道,说:
\VS{2}「天国近了,你们应当悔改!」
\VS{3}这人就是先知{\PN{以赛亚}}所说的。他说:「在旷野有人声喊着说:
\par }{\Q 预备主的道,
\par }{\Q 修直他的路!」
\par }{\PP \VS{4}这{\PN{约翰}}身穿骆驼毛的衣服,腰束皮带,吃的是蝗虫、野蜜。
\VS{5}那时,{\PN{耶路撒冷}}和{\PN{犹太}}全地,并{\PN{约旦河}}一带地方的人,都出去到{\PN{约翰}}那里,
\VS{6}承认他们的罪,在{\PN{约旦河}}里受他的洗。
\VS{7}{\PN{约翰}}看见许多法利赛人和撒都该人也来受洗,就对他们说:「毒蛇的种类!谁指示你们逃避将来的愤怒呢?
\VS{8}你们要结出果子来,与悔改的心相称。
\VS{9}不要自己心里说:『有{\PN{亚伯拉罕}}为我们的祖宗。』我告诉你们, 神能从这些石头中给{\PN{亚伯拉罕}}兴起子孙来。
\VS{10}现在斧子已经放在树根上,凡不结好果子的树就砍下来,丢在火里。
\VS{11}我是用水给你们施洗,叫你们悔改。但那在我以后来的,能力比我更大,我就是给他提鞋也不配。他要用圣灵与火给你们施洗。
\VS{12}他手里拿着簸箕,要扬净他的场,把麦子收在仓里,把糠用不灭的火烧尽了。」
\par }{\SH 耶稣受洗
\par }{\R (可1·9—11;路3·21—22)
\par }{\PP \VS{13}当下,耶稣从{\PN{加利利}}来到{\PN{约旦河}},见了{\PN{约翰}},要受他的洗。
\VS{14}{\PN{约翰}}想要拦住他,说:「我当受你的洗,你反倒上我这里来吗?」
\VS{15}耶稣回答说:「你暂且许我,因为我们理当这样尽诸般的义\FTNT{}{{\FR 3:15: }或译:礼}。」于是{\PN{约翰}}许了他。
\VS{16}耶稣受了洗,随即从水里上来。天忽然为他开了,他就看见 神的灵仿佛鸽子降下,落在他身上。
\VS{17}从天上有声音说:「这是我的爱子,我所喜悦的。」

\par }\Chap{4}{\SH 耶稣受试探
\par }{\R (可1·12—13;路4·1—13)
\par }{\PP \VerseOne{1}当时,耶稣被{\ADD{圣}}灵引到旷野,受魔鬼的试探。
\VS{2}他禁食四十昼夜,后来就饿了。
\VS{3}那试探人的进前来,对他说:「你若是 神的儿子,可以吩咐这些石头变成食物。」
\VS{4}耶稣却回答说:「{\ADD{经上}}记着说:
\par }{\Q 人活着,不是单靠食物,
\par }{\Q 乃是靠 神口里所出的一切话。」
\par }{\PP \VS{5}魔鬼就带他进了圣城,叫他站在殿顶\FTNT{}{{\FR 4:5: }顶:原文是翅}上,
\VS{6}对他说:「你若是 神的儿子,可以跳下去,因为{\ADD{经上}}记着说:
\par }{\Q 主要为你吩咐他的使者
\par }{\Q 用手托着你,
\par }{\Q 免得你的脚碰在石头上。」
\par }{\PP \VS{7}耶稣对他说:「{\ADD{经上}}又记着说:『不可试探主—你的 神。』」
\VS{8}魔鬼又带他上了一座最高的山,将世上的万国与万国的荣华都指给他看,
\VS{9}对他说:「你若俯伏拜我,我就把这一切都赐给你。」
\VS{10}耶稣说:「撒但\FTNT{}{{\FR 4:10: }就是抵挡的意思,乃魔鬼的别名},退去吧!因为{\ADD{经上}}记着说:
\par }{\Q 当拜主—你的 神,
\par }{\Q 单要事奉他。」
\par }{\PP \VS{11}于是,魔鬼离了耶稣,有天使来伺候他。
\par }{\SH 开始在加利利传道
\par }{\R (可1·14—15;路4·14—15)
\par }{\PP \VS{12}耶稣听见{\PN{约翰}}下了监,就退到{\PN{加利利}}去;
\VS{13}后又离开{\PN{拿撒勒}},往{\PN{迦百农}}去,就住在那里。那地方靠海,在{\PN{西布伦}}和{\PN{拿弗他利}}的边界上。
\VS{14}这是要应验先知{\PN{以赛亚}}的话,
\VS{15}说:
\par }{\Q {\PN{西布伦}}地,{\PN{拿弗他利}}地,
\par }{\Q 就是沿海的路,{\PN{约旦河}}外,
\par }{\Q 外邦人的{\PN{加利利}}地—
\par }{\Q \VS{16}那坐在黑暗里的百姓看见了大光;
\par }{\Q 坐在死荫之地的人有光发现照着他们。
\par }{\PP \VS{17}从那时候,耶稣就传起道来,说:「天国近了,你们应当悔改!」
\par }{\SH 呼召四个渔夫
\par }{\R (可1·16—20;路5·1—11)
\par }{\PP \VS{18}耶稣在{\PN{加利利海}}边行走,看见弟兄二人,就是那称呼{\PN{彼得}}的{\PN{西门}}和他兄弟{\PN{安得烈}},在海里撒网;他们本是打鱼的。
\VS{19}耶稣对他们说:「来跟从我,我要叫你们得人如得鱼一样。」
\VS{20}他们就立刻舍了网,跟从了他。
\VS{21}从那里往前走,又看见弟兄二人,就是{\PN{西庇太}}的儿子{\PN{雅各}}和他兄弟{\PN{约翰}},同他们的父亲{\PN{西庇太}}在船上补网,耶稣就招呼他们,
\VS{22}他们立刻舍了船,别了父亲,跟从了耶稣。
\par }{\SH 耶稣向广大的群众传道
\par }{\R (路6·17—19)
\par }{\PP \VS{23}耶稣走遍{\PN{加利利}},在各会堂里教训人,传天国的福音,医治百姓各样的病症。
\VS{24}他的名声就传遍了{\PN{叙利亚}}。那里的人把一切害病的,就是害各样疾病、各样疼痛的和被鬼附的、癫痫的、瘫痪的,都带了来,耶稣就治好了他们。
\VS{25}当下,有许多人从{\PN{加利利}}、{\PN{低加坡里}}、{\PN{耶路撒冷}}、{\PN{犹太}}、{\PN{约旦河}}外来跟着他。

\par }\Chap{5}{\SH 山上宝训
\par }{\PP \VerseOne{1}耶稣看见这许多的人,就上了山,既已坐下,门徒到他跟前来,
\VS{2}他就开口教训他们,说:
\par }{\SH 论福
\par }{\R (路6·20—23)
\par }{\Q \VS{3}虚心的人有福了!
\par }{\Q 因为天国是他们的。
\par }{\Q \VS{4}哀恸的人有福了!
\par }{\Q 因为他们必得安慰。
\par }{\Q \VS{5}温柔的人有福了!
\par }{\Q 因为他们必承受地土。
\par }{\Q \VS{6}饥渴慕义的人有福了!
\par }{\Q 因为他们必得饱足。
\par }{\Q \VS{7}怜恤人的人有福了!
\par }{\Q 因为他们必蒙怜恤。
\par }{\Q \VS{8}清心的人有福了!
\par }{\Q 因为他们必得见 神。
\par }{\Q \VS{9}使人和睦的人有福了!
\par }{\Q 因为他们必称为 神的儿子。
\par }{\Q \VS{10}为义受逼迫的人有福了!
\par }{\Q 因为天国是他们的。
\par }{\PP \VS{11}「人若因我辱骂你们,逼迫你们,捏造各样坏话毁谤你们,你们就有福了!
\VS{12}应当欢喜快乐,因为你们在天上的赏赐是大的。在你们以前的先知,人也是这样逼迫他们。」
\par }{\SH 盐和光
\par }{\R (可9·50;路14·34—35)
\par }{\PP \VS{13}「你们是世上的盐。盐若失了味,怎能叫它再咸呢?以后无用,不过丢在外面,被人践踏了。
\VS{14}你们是世上的光。城造在山上是不能隐藏的。
\VS{15}人点灯,不放在斗底下,是放在灯台上,就照亮一家的人。
\VS{16}你们的光也当这样照在人前,叫他们看见你们的好行为,便将荣耀归给你们在天上的父。」
\par }{\SH 论律法
\par }{\PP \VS{17}「莫想我来要废掉律法和先知。我来不是要废掉,乃是要成全。
\VS{18}我实在告诉你们,就是到天地都废去了,律法的一点一画也不能废去,都要成全。
\VS{19}所以,无论何人废掉这诫命中最小的一条,又教训人这样做,他在天国要称为最小的。但无论何人遵行这诫命,又教训人遵行,他在天国要称为大的。
\VS{20}我告诉你们,你们的义若不胜于文士和法利赛人的义,断不能进天国。」
\par }{\SH 论发怒
\par }{\PP \VS{21}「你们听见有吩咐古人的话,说:『不可杀人』;{\ADD{又说}}:『凡杀人的难免受审判。』
\VS{22}只是我告诉你们,凡\FTNT{}{{\FR 5:22: }有古卷在凡字下加:无缘无故地}向弟兄动怒的,难免受审断;凡骂弟兄是拉加的,难免公会{\ADD{的审断}};凡骂弟兄是魔利的,难免地狱的火。
\VS{23}所以,你在祭坛上献礼物的时候,若想起弟兄向你怀怨,
\VS{24}就把礼物留在坛前,先去同弟兄和好,然后来献礼物。
\VS{25}你同告你的对头还在路上,就赶紧与他和息,恐怕他把你送给审判官,审判官交付衙役,你就下在监里了。
\VS{26}我实在告诉你,若有一文钱没有还清,你断不能从那里出来。」
\par }{\SH 论奸淫
\par }{\PP \VS{27}「你们听见有话说:『不可奸淫。』
\VS{28}只是我告诉你们,凡看见妇女就动淫念的,这人心里已经与她犯奸淫了。
\VS{29}若是你的右眼叫你跌倒,就剜出来丢掉,宁可失去百体中的一体,不叫全身丢在地狱里。
\VS{30}若是右手叫你跌倒,就砍下来丢掉,宁可失去百体中的一体,不叫全身下入地狱。」
\par }{\SH 论休妻
\par }{\R (太19·9;可10·11—12;路16·18)
\par }{\PP \VS{31}「又有话说:『人若休妻,就当给她休书。』
\VS{32}只是我告诉你们,凡休妻的,若不是为淫乱的缘故,就是叫她作淫妇了;人若娶这被休的妇人,也是犯奸淫了。」
\par }{\SH 论起誓
\par }{\PP \VS{33}「你们又听见有吩咐古人的话,说:『不可背誓,所起的誓总要向主谨守。』
\VS{34}只是我告诉你们,什么誓都不可起。不可指着天起誓,因为天是 神的座位;
\VS{35}不可指着地起誓,因为地是他的脚凳;也不可指着{\PN{耶路撒冷}}起誓,因为{\PN{耶路撒冷}}是大君的京城;
\VS{36}又不可指着你的头起誓,因为你不能使一根头发变黑变白了。
\VS{37}你们的话,是,就说是;不是,就说不是;若再多说就是出于那恶者\FTNT{}{{\FR 5:37: }或译:就是从恶里出来的}。」
\par }{\SH 论报复
\par }{\R (路6·29—30)
\par }{\PP \VS{38}「你们听见有话说:『以眼还眼,以牙还牙。』
\VS{39}只是我告诉你们,不要与恶人作对。有人打你的右脸,连左脸也转过来由他打;
\VS{40}有人想要告你,要拿你的里衣,连外衣也由他拿去;
\VS{41}有人强逼你走一里路,你就同他走二里;
\VS{42}有求你的,就给他;有向你借贷的,不可推辞。」
\par }{\SH 论爱仇敌
\par }{\R (路6·27—28;32—36)
\par }{\PP \VS{43}「你们听见有话说:『当爱你的邻舍,恨你的仇敌。』
\VS{44}只是我告诉你们,要爱你们的仇敌,为那逼迫你们的祷告。
\VS{45}{\ADD{这样}}就可以作你们天父的儿子;因为他叫日头照好人,也照歹人;降雨给义人,也给不义的人。
\VS{46}你们若{\ADD{单}}爱那爱你们的人,有什么赏赐呢?就是税吏不也是这样行吗?
\VS{47}你们若单请你弟兄的安,比人有什么长处呢?就是外邦人不也是这样行吗?
\VS{48}所以,你们要完全,像你们的天父完全一样。」

\par }\Chap{6}{\SH 论施舍
\par }{\PP \VerseOne{1}「你们要小心,不可将善事行在人的面前,故意叫他们看见,若是这样,就不能得你们天父的赏赐了。
\VS{2}所以,你施舍的时候,不可在你前面吹号,像那假冒为善的人在会堂里和街道上所行的,故意要得人的荣耀。我实在告诉你们,他们已经得了他们的赏赐。
\VS{3}你施舍的时候,不要叫左手知道右手所做的,
\VS{4}要叫你施舍的事行在暗中。你父在暗中察看,必然报答你\FTNT{}{{\FR 6:4: }有古卷:必在明处报答你}。」
\par }{\SH 论祷告
\par }{\R (路11·2—4)
\par }{\PP \VS{5}「你们祷告的时候,不可像那假冒为善的人,爱站在会堂里和十字路口上祷告,故意叫人看见。我实在告诉你们,他们已经得了他们的赏赐。
\VS{6}你祷告的时候,要进你的内屋,关上门,祷告你在暗中的父;你父在暗中察看,必然报答你。
\VS{7}你们祷告,不可像外邦人,用许多重复话,他们以为话多了必蒙垂听。
\VS{8}你们不可效法他们;因为你们没有祈求以先,你们所需用的,你们的父早已知道了。
\VS{9}所以,你们祷告要这样说:
\par }{\Q 我们在天上的父:
\par }{\Q 愿人都尊你的名为圣。
\par }{\Q \VS{10}愿你的国降临;
\par }{\Q 愿你的旨意行在地上,
\par }{\Q 如同行在天上。
\par }{\Q \VS{11}我们日用的饮食,今日赐给我们。
\par }{\Q \VS{12}免我们的债,
\par }{\Q 如同我们免了人的债。
\par }{\Q \VS{13}不叫我们遇见试探;
\par }{\Q 救我们脱离凶恶\FTNT{}{{\FR 6:13: }或译:脱离恶者}。
\par }{\Q 因为国度、权柄、荣耀,全是你的,
\par }{\Q 直到永远。阿们\FTNT{}{{\FR 6:13: }有古卷没有因为…阿们等字}!
\par }{\PP \VS{14}「你们饶恕人的过犯,你们的天父也必饶恕你们{\ADD{的过犯}};
\VS{15}你们不饶恕人的过犯,你们的{\ADD{天}}父也必不饶恕你们的过犯。」
\par }{\SH 论禁食
\par }{\PP \VS{16}「你们禁食的时候,不可像那假冒为善的人,脸上带着愁容;因为他们把脸弄得难看,故意叫人看出他们是禁食。我实在告诉你们,他们已经得了他们的赏赐。
\VS{17}你禁食的时候,要梳头洗脸,
\VS{18}不叫人看出你禁食来,只叫你暗中的父看见;你父在暗中察看,必然报答你。」
\par }{\SH 论天上的财宝
\par }{\R (路12·33—34)
\par }{\PP \VS{19}「不要为自己积攒财宝在地上;地上有虫子咬,能锈坏,也有贼挖窟窿来偷。
\VS{20}只要积攒财宝在天上;天上没有虫子咬,不能锈坏,也没有贼挖窟窿来偷。
\VS{21}因为你的财宝在哪里,你的心也在那里。」
\par }{\SH 论心里的光
\par }{\R (路11·34—36)
\par }{\PP \VS{22}「眼睛就是身上的灯。你的眼睛若了亮,全身就光明;
\VS{23}你的眼睛若昏花,全身就黑暗。你里头的光若黑暗了,那黑暗是何等大呢!」
\par }{\SH 论 神和财利
\par }{\R (路16·13)
\par }{\PP \VS{24}「一个人不能事奉两个主;不是恶这个、爱那个,就是重这个、轻那个。你们不能又事奉 神,又事奉玛门\FTNT{}{{\FR 6:24: }玛门:财利的意思}。」
\par }{\SH 不要忧虑
\par }{\R (路12·22—31)
\par }{\PP \VS{25}「所以我告诉你们,不要为生命忧虑吃什么,喝什么;为身体忧虑穿什么。生命不胜于饮食吗?身体不胜于衣裳吗?
\VS{26}你们看那天上的飞鸟,也不种,也不收,也不积蓄在仓里,你们的天父尚且养活它。你们不比飞鸟贵重得多吗?
\VS{27}你们哪一个能用思虑使寿数多加一刻呢\FTNT{}{{\FR 6:27: }或译:使身量多加一肘呢}?
\VS{28}何必为衣裳忧虑呢?你想野地里的百合花怎么长起来;它也不劳苦,也不纺线。
\VS{29}然而我告诉你们,就是{\PN{所罗门}}极荣华的时候,他所穿戴的,还不如这花一朵呢!
\VS{30}你们这小信的人哪!野地里的草今天还在,明天就丢在炉里, 神还给它这样的妆饰,何况你们呢!
\VS{31}所以,不要忧虑说,吃什么?喝什么?穿什么?
\VS{32}这都是外邦人所求的。你们需用的这一切东西,你们的天父是知道的。
\VS{33}你们要先求他的国和他的义,这些东西都要加给你们了。
\VS{34}所以,不要为明天忧虑,因为明天自有明天的忧虑;一天的难处一天当就够了。」

\par }\Chap{7}{\SH 不要论断人
\par }{\R (路6·37—38,41—42)
\par }{\PP \VerseOne{1}「你们不要论断人,免得你们被论断。
\VS{2}因为你们怎样论断人,也必怎样被论断;你们用什么量器量给人,也必用什么量器量给你们。
\VS{3}为什么看见你弟兄眼中有刺,却不想自己眼中有梁木呢?
\VS{4}你自己眼中有梁木,怎能对你弟兄说:『容我去掉你眼中的刺』呢?
\VS{5}你这假冒为善的人!先去掉自己眼中的梁木,然后才能看得清楚,去掉你弟兄眼中的刺。
\VS{6}不要把圣物给狗,也不要把你们的珍珠丢在猪前,恐怕它践踏了珍珠,转过来咬你们。」
\par }{\SH 祈求、寻找、叩门
\par }{\R (路11·9—13)
\par }{\PP \VS{7}「你们祈求,就给你们;寻找,就寻见;叩门,就给你们开门。
\VS{8}因为凡祈求的,就得着;寻找的,就寻见;叩门的,就给他开门。
\VS{9}你们中间谁有儿子求饼,反给他石头呢?
\VS{10}求鱼,反给他蛇呢?
\VS{11}你们虽然不好,尚且知道拿好东西给儿女,何况你们在天上的父,岂不更把好东西给求他的人吗?
\VS{12}所以,无论何事,你们愿意人怎样待你们,你们也要怎样待人,因为这就是律法和先知{\ADD{的道理}}。」
\par }{\SH 窄门
\par }{\R (路13·24)
\par }{\PP \VS{13}「你们要进窄门。因为引到灭亡,那门是宽的,路是大的,进去的人也多;
\VS{14}引到{\ADD{永}}生,那门是窄的,路是小的,找着的人也少。」
\par }{\SH 两种果树
\par }{\R (路6·43—44)
\par }{\PP \VS{15}「你们要防备假先知。他们到你们这里来,外面披着羊皮,里面却是残暴的狼。
\VS{16}凭着他们的果子,就可以认出他们来。荆棘上岂能摘葡萄呢?蒺藜里岂能摘无花果呢?
\VS{17}这样,凡好树都结好果子,惟独坏树结坏果子。
\VS{18}好树不能结坏果子;坏树不能结好果子。
\VS{19}凡不结好果子的树就砍下来,丢在火里。
\VS{20}所以,凭着他们的果子就可以认出他们来。」
\par }{\SH 我不认识你们
\par }{\R (路13·25—27)
\par }{\PP \VS{21}「凡称呼我『主啊,主啊』的人不能都进天国;惟独遵行我天父旨意的人才能进去。
\VS{22}当那日必有许多人对我说:『主啊,主啊,我们不是奉你的名传道,奉你的名赶鬼,奉你的名行许多异能吗?』
\VS{23}我就明明地告诉他们说:『我从来不认识你们,你们这些作恶的人,离开我去吧!』」
\par }{\SH 两种盖房子的人
\par }{\R (路6·47—49)
\par }{\PP \VS{24}「所以,凡听见我这话就去行的,好比一个聪明人,把房子盖在磐石上;
\VS{25}雨淋,水冲,风吹,撞着那房子,房子总不倒塌,因为根基立在磐石上。
\VS{26}凡听见我这话不去行的,好比一个无知的人,把房子盖在沙土上;
\VS{27}雨淋,水冲,风吹,撞着那房子,房子就倒塌了,并且倒塌得很大。」
\par }{\PP \VS{28}耶稣讲完了这些话,众人都希奇他的教训;
\VS{29}因为他教训他们,正像有权柄的人,不像他们的文士。

\par }\Chap{8}{\SH 洁净长大麻风的人
\par }{\R (可1·40—45;路5·12—16)
\par }{\PP \VerseOne{1}耶稣下了山,有许多人跟着他。
\VS{2}有一个长大麻风的来拜他,说:「主若肯,必能叫我洁净了。」
\VS{3}耶稣伸手摸他,说:「我肯,你洁净了吧!」他的大麻风立刻就洁净了。
\VS{4}耶稣对他说:「你切不可告诉人,只要去把身体给祭司察看,献上{\PN{摩西}}所吩咐的礼物,对众人作证据。」
\par }{\SH 治好百夫长的仆人
\par }{\R (路7·1—10)
\par }{\PP \VS{5}耶稣进了{\PN{迦百农}},有一个百夫长进前来,求他说:
\VS{6}「主啊,我的仆人害瘫痪病,躺在家里,甚是疼苦。」
\VS{7}耶稣说:「我去医治他。」
\VS{8}百夫长回答说:「主啊,你到我舍下,我不敢当;只要你说一句话,我的仆人就必好了。
\VS{9}因为我在人的权下,也有兵在我以下;对这个说『去!』他就去;对那个说『来!』他就来;对我的仆人说:『你做这事!』他就去做。」
\VS{10}耶稣听见就希奇,对跟从的人说:「我实在告诉你们,这么大的信心,就是在{\PN{以色列}}中,我也没有遇见过。
\VS{11}我又告诉你们,从东从西,将有许多人来,在天国里与{\PN{亚伯拉罕}}、{\PN{以撒}}、{\PN{雅各}}一同坐席;
\VS{12}惟有本国的子民竟被赶到外边黑暗里去,在那里必要哀哭切齿了。」
\VS{13}耶稣对百夫长说:「你回去吧!照你的信心,给你成全了。」那时,他的仆人就好了。
\par }{\SH 治好许多病人
\par }{\R (可1·29—34;路4·38—41)
\par }{\PP \VS{14}耶稣到了{\PN{彼得}}家里,见{\PN{彼得}}的岳母害热病躺着。
\VS{15}耶稣把她的手一摸,热就退了;她就起来服事耶稣。
\VS{16}到了晚上,有人带着许多被鬼附的来到耶稣跟前,他只用一句话就把鬼都赶出去,并且治好了一切有病的人。
\VS{17}这是要应验先知{\PN{以赛亚}}的话,说:
\par }{\Q 他代替我们的软弱,
\par }{\Q 担当我们的疾病。
\par }{\SH 要跟从耶稣的人
\par }{\R (路9·57—62)
\par }{\PP \VS{18}耶稣见许多人围着他,就吩咐渡到那边去。
\VS{19}有一个文士来,对他说:「夫子,你无论往哪里去,我要跟从你。」
\VS{20}耶稣说:「狐狸有洞,天空的飞鸟有窝,人子却没有枕头的地方。」
\VS{21}又有一个门徒对耶稣说:「主啊,容我先回去埋葬我的父亲。」
\VS{22}耶稣说:「任凭死人埋葬他们的死人;你跟从我吧!」
\par }{\SH 平静风和海
\par }{\R (可4·35—41;路8·22—25)
\par }{\PP \VS{23}耶稣上了船,门徒跟着他。
\VS{24}海里忽然起了暴风,甚至船被波浪掩盖;耶稣却睡着了。
\VS{25}门徒来叫醒了他,说:「主啊,救我们,我们丧命啦!」
\VS{26}耶稣说:「你们这小信的人哪,为什么胆怯呢?」于是起来,斥责风和海,风和海就大大地平静了。
\VS{27}众人希奇,说:「这是怎样的人?连风和海也听从他了!」
\par }{\SH 治好加大拉被鬼附的人
\par }{\R (可5·1—20;路8·26—39)
\par }{\PP \VS{28}耶稣既渡到那边去,来到{\PN{加大拉}}人的地方,就有两个被鬼附的人从坟茔里出来迎着他,极其凶猛,甚至没有人能从那条路上经过。
\VS{29}他们喊着说:「 神的儿子,我们与你有什么相干?时候还没有到,你就上这里来叫我们受苦吗?」
\VS{30}离他们很远,有一大群猪吃食。
\VS{31}鬼就央求耶稣,说:「若把我们赶出去,就打发我们进入猪群吧!」
\VS{32}耶稣说:「去吧!」鬼就出来,进入猪群。全群忽然闯下山崖,投在海里淹死了。
\VS{33}放猪的就逃跑进城,将这一切事和被鬼附的人所遭遇的都告诉人。
\VS{34}合城的人都出来迎见耶稣,既见了就央求他离开他们的境界。

\par }\Chap{9}{\SH 治好瘫痪病人
\par }{\R (可2·1—12;路5·17—26)
\par }{\PP \VerseOne{1}耶稣上了船,渡过{\ADD{海}},来到自己的城里。
\VS{2}有人用褥子抬着一个瘫子到耶稣跟前来。耶稣见他们的信心,就对瘫子说:「小子,放心吧!你的罪赦了。」
\VS{3}有几个文士心里说:「这个人说僭妄的话了。」
\VS{4}耶稣知道他们的心意,就说:「你们为什么心里怀着恶念呢?
\VS{5}或说『你的罪赦了』,或说『你起来行走』,哪一样容易呢?
\VS{6}但要叫你们知道,人子在地上有赦罪的权柄」;就对瘫子说:「起来!拿你的褥子回家去吧。」
\VS{7}那人就起来,回家去了。
\VS{8}众人看见都惊奇,就归荣耀与 神,因为他将这样的权柄赐给人。
\par }{\SH 呼召马太
\par }{\R (可2·13—17;路5·27—32)
\par }{\PP \VS{9}耶稣从那里往前走,看见一个人名叫{\PN{马太}},坐在税关上,就对他说:「你跟从我来。」他就起来跟从了耶稣。
\par }{\PP \VS{10}耶稣在屋里坐席的时候,有好些税吏和罪人来,与耶稣和他的门徒一同坐席。
\VS{11}法利赛人看见,就对耶稣的门徒说:「你们的先生为什么和税吏并罪人一同吃饭呢?」
\VS{12}耶稣听见,就说:「康健的人用不着医生,有病的人才用得着。
\VS{13}{\ADD{经上说}}:『我喜爱怜恤,不喜爱祭祀。』这句话的意思,你们且去揣摩。我来本不是召义人,乃是召罪人。」
\par }{\SH 禁食的问题
\par }{\R (可2·18—22;路5·33—39)
\par }{\PP \VS{14}那时,{\PN{约翰}}的门徒来见耶稣,说:「我们和法利赛人常常禁食,你的门徒倒不禁食,这是为什么呢?」
\VS{15}耶稣对他们说:「新郎和陪伴之人同在的时候,陪伴之人岂能哀恸呢?但日子将到,新郎要离开他们,那时候他们就要禁食。
\VS{16}没有人把新布补在旧衣服上;因为所补上的反带坏了那衣服,破的就更大了。
\VS{17}也没有人把新酒装在旧皮袋里;若是这样,皮袋就裂开,酒漏出来,连皮袋也坏了。惟独把新酒装在新皮袋里,两样就都保全了。」
\par }{\SH 管会堂者的女儿和血漏的女人
\par }{\R (可5·21—43;路8·40—56)
\par }{\PP \VS{18}耶稣说这话的时候,有一个管{\ADD{会堂的}}来拜他,说:「我女儿刚才死了,求你去按手在她身上,她就必活了。」
\VS{19}耶稣便起来跟着他去;门徒也跟了去。
\VS{20}有一个女人,患了十二年的血漏,来到耶稣背后,摸他的衣裳 子;
\VS{21}因为她心里说:「我只摸他的衣裳,就必痊愈。」
\VS{22}耶稣转过来,看见她,就说:「女儿,放心!你的信救了你。」从那时候,女人就痊愈了。
\VS{23}耶稣到了管{\ADD{会堂}}的家里,看见有吹手,又有许多人乱嚷,
\VS{24}就说:「退去吧!这闺女不是死了,是睡着了。」他们就嗤笑他。
\VS{25}众人既被撵出,耶稣就进去,拉着闺女的手,闺女便起来了。
\VS{26}于是这风声传遍了那地方。
\par }{\SH 两个瞎子得医治
\par }{\PP \VS{27}耶稣从那里往前走,有两个瞎子跟着他,喊叫说:「{\PN{大卫}}的子孙,可怜我们吧!」
\VS{28}耶稣进了房子,瞎子就来到他跟前。耶稣说:「你们信我能做这事吗?」他们说:「主啊,我们信。」
\VS{29}耶稣就摸他们的眼睛,说:「照着你们的信给你们成全了吧。」
\par }{\PP \VS{30}他们的眼睛就开了。耶稣切切地嘱咐他们说:「你们要小心,不可叫人知道。」
\VS{31}他们出去,竟把他的名声传遍了那地方。
\par }{\SH 治好哑巴
\par }{\PP \VS{32}他们出去的时候,有人将鬼所附的一个哑巴带到耶稣跟前来。
\VS{33}鬼被赶出去,哑巴就说出话来。众人都希奇,说:「在{\PN{以色列}}中,从来没有见过这样的事。」
\VS{34}法利赛人却说:「他是靠着鬼王赶鬼。」
\par }{\SH 耶稣的怜悯
\par }{\PP \VS{35}耶稣走遍各城各乡,在会堂里教训人,宣讲天国的福音,又医治各样的病症。
\VS{36}他看见许多的人,就怜悯他们;因为他们困苦流离,如同羊没有牧人一般。
\VS{37}于是对门徒说:「要收的庄稼多,做工的人少。
\VS{38}所以,你们当求庄稼的主打发工人出去收他的庄稼。」

\par }\Chap{10}{\SH 十二使徒
\par }{\R (可3·13—19;路6·12—16)
\par }{\PP \VerseOne{1}耶稣叫了十二个门徒来,给他们权柄,能赶逐污鬼,并医治各样的病症。
\VS{2}这十二使徒的名:头一个叫{\PN{西门}}(又称{\PN{彼得}}),还有他兄弟{\PN{安得烈}},{\PN{西庇太}}的儿子{\PN{雅各}}和{\PN{雅各}}的兄弟{\PN{约翰}},
\VS{3}{\PN{腓力}}和{\PN{巴多罗买}},{\PN{多马}}和税吏{\PN{马太}},{\PN{亚勒腓}}的儿子{\PN{雅各}},和{\PN{达太}},
\VS{4}奋锐党的{\PN{西门}},还有卖耶稣的{\PN{加略}}人{\PN{犹大}}。
\par }{\SH 十二使徒的使命
\par }{\R (可6·7—13;路9·1—6)
\par }{\PP \VS{5}耶稣差这十二个人去,吩咐他们说:「外邦人的路,你们不要走;{\PN{撒马利亚}}人的城,你们不要进;
\VS{6}宁可往{\PN{以色列}}家迷失的羊那里去。
\VS{7}随走随传,说『天国近了!』
\VS{8}医治病人,叫死人复活,叫长大麻风的洁净,把鬼赶出去。你们白白地得来,也要白白地舍去。
\VS{9}腰袋里不要带金银铜钱;
\VS{10}行路不要带口袋;不要带两件褂子,也不要带鞋和拐杖。因为工人得饮食是应当的。
\VS{11}你们无论进哪一城,哪一村,要打听那里谁是好人,就住在他家,直住到走的时候。
\VS{12}进他家里去,要请他的安。
\VS{13}那家若配得平安,你们所求的平安就必临到那家;若不配得,你们所求的平安仍归你们。
\VS{14}凡不接待你们、不听你们话的人,你们离开那家,或是那城的时候,就把脚上的尘土跺下去。
\VS{15}我实在告诉你们,当审判的日子,{\PN{所多玛}}和{\PN{蛾摩拉}}所受的,比那城还容易受呢!」
\par }{\SH 将来的逼迫
\par }{\R (可13·9—13;路21·12—19)
\par }{\PP \VS{16}「我差你们去,如同羊进入狼群;所以你们要灵巧像蛇,驯良像鸽子。
\VS{17}你们要防备人;因为他们要把你们交给公会,也要在会堂里鞭打你们,
\VS{18}并且你们要为我的缘故被送到诸侯君王面前,对他们和外邦人作见证。
\VS{19}你们被交的时候,不要思虑怎样说话,或说什么话。到那时候,必赐给你们当说的话;
\VS{20}因为不是你们自己说的,乃是你们父的灵在你们里头说的。
\VS{21}弟兄要把弟兄,父亲要把儿子,送到死地;儿女要与父母为敌,害死他们;
\VS{22}并且你们要为我的名被众人恨恶。惟有忍耐到底的必然得救。
\VS{23}有人在这城里逼迫你们,就逃到那城里去。我实在告诉你们,{\PN{以色列}}的城邑,你们还没有走遍,人子就到了。
\VS{24}学生不能高过先生;仆人不能高过主人。
\VS{25}学生和先生一样,仆人和主人一样,也就罢了。人既骂家主是别西卜\FTNT{}{{\FR 10:25: }别西卜:是鬼王的名},何况他的家人呢?」
\par }{\SH 该怕的是谁
\par }{\R (路12·2—7)
\par }{\PP \VS{26}「所以,不要怕他们;因为掩盖的事没有不露出来的,隐藏的事没有不被人知道的。
\VS{27}我在暗中告诉你们的,你们要在明处说出来;你们耳中所听的,要在房上宣扬出来。
\VS{28}那杀身体、不能杀灵魂的,不要怕他们;惟有能把身体和灵魂都灭在地狱里的,正要怕他。
\VS{29}两个麻雀不是卖一分银子吗?若是你们的父不许,一个也不能掉在地上;
\VS{30}就是你们的头发也都被数过了。
\VS{31}所以,不要惧怕,你们比许多麻雀还贵重!」
\par }{\SH 在人的面前承认基督
\par }{\R (路12·8—9)
\par }{\PP \VS{32}「凡在人面前认我的,我在我天上的父面前也必认他;
\VS{33}凡在人面前不认我的,我在我天上的父面前也必不认他。」
\par }{\SH 不是和平,而是刀剑
\par }{\R (路12·51—53;14·26—27)
\par }{\PP \VS{34}「你们不要想我来是叫地上太平;我来并不是叫地上太平,乃是叫地上动刀兵。
\VS{35}因为我来是叫
\par }{\Q 人与父亲生疏,
\par }{\Q 女儿与母亲生疏,
\par }{\Q 媳妇与婆婆生疏。
\par }{\Q \VS{36}人的仇敌就是自己家里的人。
\par }{\PP \VS{37}「爱父母过于爱我的,不配作我的门徒;爱儿女过于爱我的,不配作我的门徒;
\VS{38}不背着他的十字架跟从我的,也不配作我的门徒。
\VS{39}得着生命的,将要失丧生命;为我失丧生命的,将要得着生命。」
\par }{\SH 论报赏
\par }{\R (可9·41)
\par }{\PP \VS{40}「人接待你们就是接待我;接待我就是接待那差我来的。
\VS{41}人因为先知的名接待先知,必得先知所得的赏赐;人因为义人的名接待义人,必得义人所得的赏赐。
\VS{42}无论何人,因为门徒的名,只把一杯凉水给这小子里的一个喝,我实在告诉你们,这人不能不得赏赐。」

\par }\Chap{11}{\PP \VerseOne{1}耶稣吩咐完了十二个门徒,就离开那里,往各城去传道、教训人。
\par }{\SH 施洗约翰的门徒来见耶稣
\par }{\R (路7·18—35)
\par }{\PP \VS{2}{\PN{约翰}}在监里听见基督所做的事,就打发{\ADD{两个}}门徒去,
\VS{3}问他说:「那将要来的是你吗?还是我们等候别人呢?」
\VS{4}耶稣回答说:「你们去,把所听见、所看见的事告诉{\PN{约翰}}。
\VS{5}就是瞎子看见,瘸子行走,长大麻风的洁净,聋子听见,死人复活,穷人有福音传给他们。
\VS{6}凡不因我跌倒的就有福了!」
\VS{7}他们走的时候,耶稣就对众人讲论{\PN{约翰}}说:「你们从前出到旷野是要看什么呢?要看风吹动的芦苇吗?
\VS{8}你们出去到底是要看什么?要看穿细软{\ADD{衣服}}的人吗?那穿细软{\ADD{衣服}}的人是在王宫里。
\VS{9}你们出去究竟是为什么?是要看先知吗?我告诉你们,是的,他比先知大多了。
\VS{10}{\ADD{经上}}记着说:『我要差遣我的使者在你前面预备道路』,所说的就是这个人。
\VS{11}我实在告诉你们,凡妇人所生的,没有一个兴起来大过施洗{\PN{约翰}}的;然而天国里最小的比他还大。
\VS{12}从施洗{\PN{约翰}}的时候到如今,天国是努力进入的,努力的人就得着了。
\VS{13}因为众先知和律法说预言,到{\PN{约翰}}为止。
\VS{14}你们若肯领受,这人就是那应当来的{\PN{以利亚}}。
\VS{15}有耳可听的,就应当听!
\VS{16}我可用什么比这世代呢?好像孩童坐在街市上招呼同伴,说:
\par }{\PP \VS{17}我们向你们吹笛,
\par }{\Q 你们不跳舞;
\par }{\Q 我们向你们举哀,
\par }{\Q 你们不捶胸。
\par }{\MM \VS{18}{\PN{约翰}}来了,也不吃也不喝,人就说他是被鬼附着的;
\VS{19}人子来了,也吃也喝,人又说他是贪食好酒的人,是税吏和罪人的朋友。但智慧之子总以智慧为是\FTNT{}{{\FR 11:19: }有古卷:但智慧在行为上就显为是}。」
\par }{\SH 耶稣责备不悔改的城
\par }{\R (路10·13—15)
\par }{\PP \VS{20}耶稣在诸城中行了许多异能,那些城的人终不悔改,就在那时候责备他们,说:
\VS{21}「{\PN{哥拉汛}}哪,你有祸了!{\PN{伯赛大}}啊,你有祸了!因为在你们中间所行的异能,若行在{\PN{泰尔}}、{\PN{西顿}},他们早已披麻蒙灰悔改了。
\VS{22}但我告诉你们,当审判的日子,{\PN{泰尔}}、{\PN{西顿}}所受的,比你们还容易受呢!
\VS{23}{\PN{迦百农}}啊,你已经升到天上\FTNT{}{{\FR 11:23: }或译:你将要升到天上吗},将来必坠落阴间;因为在你那里所行的异能,若行在{\PN{所多玛}},它还可以存到今日。
\VS{24}但我告诉你们,当审判的日子,{\PN{所多玛}}所受的,比你还容易受呢!」
\par }{\SH 到我这里来
\par }{\R (路10·21—22)
\par }{\PP \VS{25}那时,耶稣说:「父啊,天地的主,我感谢你!因为你将这些事向聪明通达人就藏起来,向婴孩就显出来。
\VS{26}父啊,是的,因为你的美意本是如此。
\VS{27}一切所有的,都是我父交付我的;除了父,没有人知道子;除了子和子所愿意指示的,没有人知道父。
\VS{28}凡劳苦担重担的人可以到我这里来,我就使你们得安息。
\VS{29}我心里柔和谦卑,你们当负我的轭,学我的样式;这样,你们心里就必得享安息。
\VS{30}因为我的轭是容易的,我的担子是轻省的。」

\par }\Chap{12}{\SH 安息日的问题
\par }{\R (可2·23—28;路6·1—5)
\par }{\PP \VerseOne{1}那时,耶稣在安息日从麦地经过。他的门徒饿了,就掐起麦穗来吃。
\VS{2}法利赛人看见,就对耶稣说:「看哪,你的门徒做安息日不可做的事了!」
\VS{3}耶稣对他们说:「{\ADD{经上记着}}{\PN{大卫}}和跟从他的人饥饿之时所做的事,你们没有念过吗?
\VS{4}他怎么进了 神的殿,吃了陈设饼,这饼不是他和跟从他的人可以吃得,惟独祭司才可以吃。
\VS{5}再者,律法上所记的,当安息日,祭司在殿里犯了安息日还是没有罪,你们没有念过吗?
\VS{6}但我告诉你们,在这里有一人比殿更大。
\VS{7}『我喜爱怜恤,不喜爱祭祀。』你们若明白这话的意思,就不将无罪的当作有罪的了。
\VS{8}因为人子是安息日的主。」
\par }{\SH 治好枯干了一只手的人
\par }{\R (可3·1—6;路6·6—11)
\par }{\PP \VS{9}耶稣离开那地方,进了一个会堂。
\VS{10}那里有一个人枯干了一只手。有人问耶稣说:「安息日治病可以不可以?」意思是要控告他。
\VS{11}耶稣说:「你们中间谁有一只羊,当安息日掉在坑里,不把它抓住、拉上来呢?
\VS{12}人比羊何等贵重呢!所以,在安息日做善事是可以的。」
\VS{13}于是对那人说:「伸出手来!」他把手一伸,手就复了原,和那只手一样。
\VS{14}法利赛人出去,商议怎样可以除灭耶稣。
\par }{\SH  神所拣选的仆人
\par }{\PP \VS{15}耶稣知道了,就离开那里,有许多人跟着他。他把其中有病的人都治好了;
\VS{16}又嘱咐他们,不要给他传名。
\VS{17}这是要应验先知{\PN{以赛亚}}的话,说:
\par }{\Q \VS{18}看哪!我的仆人,
\par }{\Q 我所拣选、所亲爱、心里所喜悦的,
\par }{\Q 我要将我的灵赐给他;
\par }{\Q 他必将公理传给外邦。
\par }{\Q \VS{19}他不争竞,不喧嚷;
\par }{\Q 街上也没有人听见他的声音。
\par }{\Q \VS{20}压伤的芦苇,他不折断;
\par }{\Q 将残的灯火,他不吹灭;
\par }{\Q 等他施行公理,叫公理得胜。
\par }{\Q \VS{21}外邦人都要仰望他的名。
\par }{\SH 耶稣和别西卜
\par }{\R (可3·20—30;路11·14—23)
\par }{\PP \VS{22}当下,有人将一个被鬼附着、又瞎又哑的人带到耶稣那里,耶稣就医治他,甚至那哑巴又能说话,又能看见。
\VS{23}众人都惊奇,说:「这不是{\PN{大卫}}的子孙吗?」
\VS{24}但法利赛人听见,就说:「这个人赶鬼,无非是靠着鬼王别西卜啊。」
\VS{25}耶稣知道他们的意念,就对他们说:「凡一国自相纷争,就成为荒场;一城一家自相纷争,必站立不住;
\VS{26}若撒但赶逐撒但,就是自相纷争,他的国怎能站得住呢?
\VS{27}我若靠着别西卜赶鬼,你们的子弟赶鬼又靠着谁呢?这样,他们就要断定你们的是非。
\VS{28}我若靠着 神的灵赶鬼,这就是 神的国临到你们了。
\VS{29}人怎能进壮士家里,抢夺他的家具呢?除非先捆住那壮士,才可以抢夺他的{\ADD{家财}}。
\VS{30}不与我相合的,就是敌我的;不同我收聚的,就是分散的。」
\VS{31}所以我告诉你们:「人一切的罪和亵渎的话都可得赦免,惟独亵渎圣灵,总不得赦免。
\VS{32}凡说话干犯人子的,还可得赦免;惟独说话干犯圣灵的,今世来世总不得赦免。」
\par }{\SH 树和果子
\par }{\R (路6·43—45)
\par }{\PP \VS{33}「你们或以为树好,果子也好;树坏,果子也坏;因为看果子就可以知道树。
\VS{34}毒蛇的种类!你们既是恶人,怎能说出好话来呢?因为心里所充满的,口里就说出来。
\VS{35}善人从他{\ADD{心里}}所存的善就发出善来;恶人从他{\ADD{心里}}所存的恶就发出恶来。
\VS{36}我又告诉你们,凡人所说的闲话,当审判的日子,必要句句供出来;
\VS{37}因为要凭你的话定你为义,也要凭你的话定你有罪。」
\par }{\SH 要求神迹
\par }{\R (可8·11—12;路11·29—32)
\par }{\PP \VS{38}当时,有几个文士和法利赛人对耶稣说:「夫子,我们愿意你显个神迹给我们看。」
\VS{39}耶稣回答说:「一个邪恶淫乱的世代求看神迹,除了先知{\PN{约拿}}的神迹以外,再没有神迹给他们看。
\VS{40}{\PN{约拿}}三日三夜在大鱼肚腹中,人子也要这样三日三夜在地里头。
\VS{41}当审判的时候,{\PN{尼尼微}}人要起来定这世代的罪,因为{\PN{尼尼微}}人听了{\PN{约拿}}所传的就悔改了。看哪,在这里有一人比{\PN{约拿}}更大!
\VS{42}当审判的时候,南方的女王要起来定这世代的罪;因为她从地极而来,要听{\PN{所罗门}}的智慧话。看哪,在这里有一人比{\PN{所罗门}}更大!」
\par }{\SH 污鬼回来
\par }{\R (路11·24—26)
\par }{\PP \VS{43}「污鬼离了人身,就在无水之地过来过去,寻求安歇{\ADD{之处}},却寻不着。
\VS{44}于是说:『我要回到我所出来的屋里去。』到了,就看见里面空闲,打扫干净,修饰好了,
\VS{45}便去另带了七个比自己更恶的鬼来,都进去住在那里。那人末后的景况比先前更不好了。这邪恶的世代也要如此。」
\par }{\SH 耶稣的母亲和兄弟们
\par }{\R (可3·31—35;路8·19—21)
\par }{\PP \VS{46}耶稣还对众人说话的时候,不料他母亲和他弟兄站在外边,要与他说话。
\VS{47}有人告诉他说:「看哪,你母亲和你弟兄站在外边,要与你说话。」
\VS{48}他却回答那人说:「谁是我的母亲?谁是我的弟兄?」
\VS{49}就伸手指着门徒,说:「看哪,我的母亲,我的弟兄。
\VS{50}凡遵行我天父旨意的人,就是我的弟兄姊妹和母亲了。」

\par }\Chap{13}{\SH 撒种的比喻
\par }{\R (可4·1—9;路8·4—8)
\par }{\PP \VerseOne{1}当那一天,耶稣从房子里出来,坐在海边。
\VS{2}有许多人到他那里聚集,他只得上船坐下,众人都站在岸上。
\VS{3}他用比喻对他们讲许多道理,说:「有一个撒种的出去撒种;
\VS{4}撒的时候,有落在路旁的,飞鸟来吃尽了;
\VS{5}有落在土浅石头地上的,土既不深,发苗最快,
\VS{6}日头出来一晒,因为没有根,就枯干了;
\VS{7}有落在荆棘里的,荆棘长起来,把它挤住了;
\VS{8}又有落在好土里的,就结实,有一百倍的,有六十倍的,有三十倍的。
\VS{9}有耳{\ADD{可听}}的,就应当听!」
\par }{\SH 用比喻的目的
\par }{\R (可4·10—12;路8·9—10)
\par }{\PP \VS{10}门徒进前来,问耶稣说:「对众人讲话,为什么用比喻呢?」
\VS{11}耶稣回答说:「因为天国的奥秘只叫你们知道,不叫他们知道。
\VS{12}凡有的,还要加给他,叫他有余;凡没有的,连他所有的,也要夺去。
\VS{13}所以我用比喻对他们讲,是因他们看也看不见,听也听不见,也不明白。
\VS{14}在他们身上,正应了{\PN{以赛亚}}的预言,说:
\par }{\Q 你们听是要听见,却不明白;
\par }{\Q 看是要看见,却不晓得;
\par }{\Q \VS{15}因为这百姓油蒙了心,
\par }{\Q 耳朵发沉,
\par }{\Q 眼睛闭着,
\par }{\Q 恐怕眼睛看见,
\par }{\Q 耳朵听见,
\par }{\Q 心里明白,回转过来,
\par }{\Q 我就医治他们。
\par }{\PP \VS{16}「但你们的眼睛是有福的,因为看见了;你们的耳朵也是有福的,因为听见了。
\VS{17}我实在告诉你们,从前有许多先知和义人要看你们所看的,却没有看见,要听你们所听的,却没有听见。」
\par }{\SH 解明撒种的比喻
\par }{\R (可4·13—20;路8·11—15)
\par }{\PP \VS{18}「所以,你们当听这撒种的比喻。
\VS{19}凡听见天国道理不明白的,那恶者就来,把所撒在他心里的夺了去;这就是撒在路旁的了。
\VS{20}撒在石头地上的,就是人听了道,当下欢喜领受,
\VS{21}只因心里没有根,不过是暂时的,及至为道遭了患难,或是受了逼迫,立刻就跌倒了。
\VS{22}撒在荆棘里的,就是人听了道,后来有世上的思虑、钱财的迷惑把道挤住了,不能结实。
\VS{23}撒在好地上的,就是人听道明白了,后来结实,有一百倍的,有六十倍的,有三十倍的。」
\par }{\SH 稗子的比喻
\par }{\PP \VS{24}耶稣又设个比喻对他们说:「天国好像人撒好种在田里,
\VS{25}及至人睡觉的时候,有仇敌来,将稗子撒在麦子里就走了。
\VS{26}到长苗吐穗的时候,稗子也显出来。
\VS{27}田主的仆人来告诉他说:『主啊,你不是撒好种在田里吗?从哪里来的稗子呢?』
\VS{28}主人说:『这是仇敌做的。』仆人说:『你要我们去薅出来吗?』
\VS{29}主人说:『不必,恐怕薅稗子,连麦子也拔出来。
\VS{30}容这两样一齐长,等着收割。当收割的时候,我要对收割的人说,先将稗子薅出来,捆成捆,留着烧;惟有麦子要收在仓里。』」
\par }{\SH 芥菜种和面酵的比喻
\par }{\R (可4·30—32;路13·18—21)
\par }{\PP \VS{31}他又设个比喻对他们说:「天国好像一粒芥菜种,有人拿去种在田里。
\VS{32}这原是百种里最小的,等到长起来,却比各样的菜都大,且成了树,天上的飞鸟来宿在它的枝上。」
\VS{33}他又对他们讲个比喻说:「天国好像面酵,有妇人拿来,藏在三斗面里,直等全团都发起来。」
\par }{\SH 用比喻的原因
\par }{\R (可4·33—34)
\par }{\PP \VS{34}这都是耶稣用比喻对众人说的话;若不用比喻,就不对他们说什么。
\VS{35}这是要应验先知的话,说:
\par }{\Q 我要开口用比喻,
\par }{\Q 把创世以来所隐藏的事发明出来。
\par }{\SH 解明稗子的比喻
\par }{\PP \VS{36}当下,耶稣离开众人,进了房子。他的门徒进前来,说:「请把田间稗子的比喻讲给我们听。」
\VS{37}他回答说:「那撒好种的就是人子;
\VS{38}田地就是世界;好种就是天国之子;稗子就是那恶者之子;
\VS{39}撒稗子的仇敌就是魔鬼;收割的时候就是世界的末了;收割的人就是天使。
\VS{40}将稗子薅出来用火焚烧,世界的末了也要如此。
\VS{41}人子要差遣使者,把一切叫人跌倒的和作恶的,从他国里挑出来,
\VS{42}丢在火炉里;在那里必要哀哭切齿了。
\VS{43}那时,义人在他们父的国里,要发出光来,像太阳一样。有耳{\ADD{可听}}的,就应当听!」
\par }{\SH 藏宝、寻珠、撒网的比喻
\par }{\PP \VS{44}「天国好像宝贝藏在地里,人遇见了就把它藏起来,欢欢喜喜地去变卖一切所有的,买这块地。
\VS{45}天国又好像买卖人寻找好珠子,
\VS{46}遇见一颗重价的珠子,就去变卖他一切所有的,买了这颗珠子。
\VS{47}天国又好像网撒在海里,聚拢各样水族,
\VS{48}网既满了,人就拉上岸来,坐下,拣好的收在器具里,将不好的丢弃了。
\VS{49}世界的末了也要这样。天使要出来,从义人中把恶人分别出来,
\VS{50}丢在火炉里;在那里必要哀哭切齿了。」
\par }{\SH 新旧的东西
\par }{\PP \VS{51}{\ADD{耶稣说}}:「这一切的话你们都明白了吗?」他们说:「我们明白了。」
\VS{52}他说:「凡文士受教作天国的门徒,就像一个家主从他库里拿出新旧的东西来。」
\par }{\SH 拿撒勒人厌弃耶稣
\par }{\R (可6·1—6;路4·16—30)
\par }{\PP \VS{53}耶稣说完了这些比喻,就离开那里,
\VS{54}来到自己的家乡,在会堂里教训人,甚至他们都希奇,说:「这人从哪里有这等智慧和异能呢?
\VS{55}这不是木匠的儿子吗?他母亲不是叫{\PN{马利亚}}吗?他弟兄们不是叫{\PN{雅各}}、{\PN{约西}}\FTNT{}{{\FR 13:55: }有古卷:约瑟}、{\PN{西门}}、{\PN{犹大}}吗?
\VS{56}他妹妹们不是都在我们这里吗?这人从哪里有这一切的事呢?」
\VS{57}他们就厌弃他\FTNT{}{{\FR 13:57: }厌弃他:原文是因他跌倒}。耶稣对他们说:「大凡先知,除了本地本家之外,没有不被人尊敬的。」
\VS{58}耶稣因为他们不信,就在那里不多行异能了。

\par }\Chap{14}{\SH 施洗约翰的死
\par }{\R (可6·14—29;路9·7—9)
\par }{\PP \VerseOne{1}那时,分封的王{\PN{希律}}听见耶稣的名声,
\VS{2}就对臣仆说:「这是施洗的{\PN{约翰}}从死里复活,所以这些异能从他里面发出来。」
\VS{3}起先,{\PN{希律}}为他兄弟{\PN{腓力}}的妻子{\PN{希罗底}}的缘故,把{\PN{约翰}}拿住,锁在监里。
\VS{4}因为{\PN{约翰}}曾对他说:「你娶这妇人是不合理的。」
\VS{5}{\PN{希律}}就想要杀他,只是怕百姓,因为他们以{\PN{约翰}}为先知。
\VS{6}到了{\PN{希律}}的生日,{\PN{希罗底}}的女儿在众人面前跳舞,使{\PN{希律}}欢喜。
\VS{7}{\PN{希律}}就起誓,应许随她所求的给她。
\VS{8}女儿被母亲所使,就说:「请把施洗{\PN{约翰}}的头放在盘子里,拿来给我。」
\VS{9}王便忧愁,但因他所起的誓,又因同席的人,就吩咐给她;
\VS{10}于是打发人去,在监里斩了{\PN{约翰}},
\VS{11}把头放在盘子里,拿来给了女子;女子拿去给她母亲。
\VS{12}{\PN{约翰}}的门徒来,把尸首领去埋葬了,就去告诉耶稣。
\par }{\SH 耶稣给五千人吃饱
\par }{\R (可6·30—44;路9·10—17;约6·1—14)
\par }{\PP \VS{13}耶稣听见了,就上船从那里独自退到野地里去。众人听见,就从各城里步行跟随他。
\VS{14}耶稣出来,见有许多的人,就怜悯他们,治好了他们的病人。
\VS{15}天将晚的时候,门徒进前来,说:「这是野地,时候已经过了,请叫众人散开,他们好往村子里去,自己买吃的。」
\VS{16}耶稣说:「不用他们去,你们给他们吃吧!」
\VS{17}门徒说:「我们这里只有五个饼,两条鱼。」
\VS{18}耶稣说:「拿过来给我。」
\VS{19}于是吩咐众人坐在草地上,就拿着这五个饼,两条鱼,望着天祝福,擘开饼,递给门徒,门徒又递给众人。
\VS{20}他们都吃,并且吃饱了,把剩下的零碎收拾起来,装满了十二个篮子。
\VS{21}吃的人,除了妇女孩子,约有五千。
\par }{\SH 耶稣在海面上行走
\par }{\R (可6·45—52;约6·15—21)
\par }{\PP \VS{22}耶稣随即催门徒上船,先渡到那边去,等他叫众人散开。
\VS{23}散了众人以后,他就独自上山去祷告。到了晚上,只有他一人在那里。
\VS{24}那时船在海中,因风不顺,被浪摇撼。
\VS{25}夜里四更天,耶稣在海面上走,往门徒那里去。
\VS{26}门徒看见他在海面上走,就惊慌了,说:「是个鬼怪!」便害怕,喊叫起来。
\VS{27}耶稣连忙对他们说:「你们放心,是我,不要怕!」
\VS{28}{\PN{彼得}}说:「主,如果是你,请叫我从水面上走到你那里去。」
\VS{29}耶稣说:「你来吧。」{\PN{彼得}}就从船上下去,在水面上走,要到耶稣那里去;
\VS{30}只因见风甚大,就害怕,将要沉下去,便喊着说:「主啊,救我!」
\VS{31}耶稣赶紧伸手拉住他,说:「你这小信的人哪,为什么疑惑呢?」
\VS{32}他们上了船,风就住了。
\VS{33}在船上的人都拜他,说:「你真是 神的儿子了。」
\par }{\SH 治好革尼撒勒的病人
\par }{\R (可6·53—56)
\par }{\PP \VS{34}他们过了{\ADD{海}},来到{\PN{革尼撒勒}}地方。
\VS{35}那里的人一认出是耶稣,就打发人到周围地方去,把所有的病人带到他那里,
\VS{36}只求耶稣准他们摸他的衣裳 子;摸着的人就都好了。

\par }\Chap{15}{\SH 古人的传统
\par }{\R (可7·1—23)
\par }{\PP \VerseOne{1}那时,有法利赛人和文士从{\PN{耶路撒冷}}来见耶稣,说:
\VS{2}「你的门徒为什么犯古人的遗传呢?因为吃饭的时候,他们不洗手。」
\VS{3}耶稣回答说:「你们为什么因着你们的遗传犯 神的诫命呢?
\VS{4}神说:『当孝敬父母』;又说:『咒骂父母的,必治死他。』
\VS{5}你们倒说:『无论何人对父母说:我所当奉给你的已经作了供献,
\VS{6}他就可以不孝敬父母。』这就是你们借着遗传,废了 神的诫命。
\VS{7}假冒为善的人哪,{\PN{以赛亚}}指着你们说的预言是不错的。他说:
\par }{\Q \VS{8}这百姓用嘴唇尊敬我,
\par }{\Q 心却远离我;
\par }{\Q \VS{9}他们将人的吩咐当作道理教导人,
\par }{\Q 所以拜我也是枉然。」
\par }{\PP \VS{10}耶稣就叫了众人来,对他们说:「你们要听,也要明白。
\VS{11}入口的不能污秽人,出口的乃能污秽人。」
\VS{12}当时,门徒进前来对他说:「法利赛人听见这话,不服\FTNT{}{{\FR 15:12: }原文是跌倒},你知道吗?」
\VS{13}耶稣回答说:「凡栽种的物,若不是我天父栽种的,必要拔出来。
\VS{14}任凭他们吧!他们是瞎眼领路的;若是瞎子领瞎子,两个人都要掉在坑里。」
\VS{15}{\PN{彼得}}对耶稣说:「请将这比喻讲给我们听。」
\VS{16}耶稣说:「你们到如今还不明白吗?
\VS{17}岂不知凡入口的,是运到肚子里,又落在茅厕里吗?
\VS{18}惟独出口的,是从心里发出来的,这才污秽人。
\VS{19}因为从心里发出来的,有恶念、凶杀、奸淫、苟合、偷盗、妄证、谤 。
\VS{20}这都是污秽人的;至于不洗手吃饭,那却不污秽人。」
\par }{\SH 迦南妇人的信心
\par }{\R (可7·24—30)
\par }{\PP \VS{21}耶稣离开那里,退到{\PN{泰尔}}、{\PN{西顿}}的境内去。
\VS{22}有一个{\PN{迦南}}妇人,从那地方出来,喊着说:「主啊,{\PN{大卫}}的子孙,可怜我!我女儿被鬼附得甚苦。」
\VS{23}耶稣却一言不答。门徒进前来,求他说:「这妇人在我们后头喊叫,请打发她走吧。」
\VS{24}耶稣说:「我奉差遣不过是到{\PN{以色列}}家迷失的羊那里去。」
\VS{25}那妇人来拜他,说:「主啊,帮助我!」
\VS{26}他回答说:「不好拿儿女的饼丢给狗吃。」
\VS{27}妇人说:「主啊,不错;但是狗也吃它主人桌子上掉下来的碎渣儿。」
\VS{28}耶稣说:「妇人,你的信心是大的!照你所要的,给你成全了吧。」从那时候,她女儿就好了。
\par }{\SH 治好许多病人
\par }{\PP \VS{29}耶稣离开那地方,来到靠近{\PN{加利利}}的海边,就上山坐下。
\VS{30}有许多人到他那里,带着瘸子、瞎子、哑巴、有残疾的,和好些别的病人,都放在他脚前;他就治好了他们。
\VS{31}甚至众人都希奇;因为看见哑巴说话,残疾的痊愈,瘸子行走,瞎子看见,他们就归荣耀给{\PN{以色列}}的 神。
\par }{\SH 耶稣给四千人吃饱
\par }{\R (可8·1—10)
\par }{\PP \VS{32}耶稣叫门徒来,说:「我怜悯这众人;因为他们同我在这里已经三天,也没有吃的了。我不愿意叫他们饿着回去,恐怕在路上困乏。」
\VS{33}门徒说:「我们在这野地,哪里有这么多的饼叫这许多人吃饱呢?」
\VS{34}耶稣说:「你们有多少饼?」他们说:「有七个,还有几条小鱼。」
\VS{35}他就吩咐众人坐在地上,
\VS{36}拿着这七个饼和几条鱼,祝谢了,擘开,递给门徒;门徒又递给众人。
\VS{37}众人都吃,并且吃饱了,收拾剩下的零碎,装满了七个筐子。
\VS{38}吃的人,除了妇女孩子,共有四千。
\VS{39}耶稣叫众人散去,就上船,来到{\PN{马加丹}}的境界。

\par }\Chap{16}{\SH 求主显个神迹
\par }{\R (可8·11—13;路12·54—56)
\par }{\PP \VerseOne{1}法利赛人和撒都该人来试探耶稣,请他从天上显个神迹给他们看。
\VS{2}耶稣回答说:「晚上天发红,你们就说:『天必要晴。』
\VS{3}早晨天发红,又发黑,你们就说:『今日必有风雨。』你们知道分辨天上的气色,倒不能分辨这时候的神迹。
\VS{4}一个邪恶淫乱的世代求神迹,除了{\PN{约拿}}的神迹以外,再没有神迹给他看。」耶稣就离开他们去了。
\par }{\SH 防备法利赛人和撒都该人的酵
\par }{\R (可8·14—21)
\par }{\PP \VS{5}门徒渡到那边去,忘了带饼。
\VS{6}耶稣对他们说:「你们要谨慎,防备法利赛人和撒都该人的酵。」
\VS{7}门徒彼此议论说:「这是因为我们没有带饼吧。」
\VS{8}耶稣看出来,就说:「你们这小信的人,为什么因为没有饼彼此议论呢?
\VS{9}你们还不明白吗?不记得那五个饼分给五千人、又收拾了多少篮子{\ADD{的零碎}}吗?
\VS{10}也不记得那七个饼分给四千人、又收拾了多少筐子{\ADD{的零碎}}吗?
\VS{11}我对你们说:『要防备法利赛人和撒都该人的酵』,这话不是指着饼说的,你们怎么不明白呢?」
\VS{12}门徒这才晓得他说的不是叫他们防备饼的酵,乃是防备法利赛人和撒都该人的教训。
\par }{\SH 彼得认耶稣为基督
\par }{\R (可8·27—30;路9·18—21)
\par }{\PP \VS{13}耶稣到了{\PN{凯撒利亚·腓立比}}的境内,就问门徒说:「人说我\FTNT{}{{\FR 16:13: }有古卷没有我字}—人子是谁?」
\VS{14}他们说:「有人说是施洗的{\PN{约翰}};有人说是{\PN{以利亚}};又有人说是{\PN{耶利米}}或是先知里的一位。」
\VS{15}耶稣说:「你们说我是谁?」
\VS{16}{\PN{西门·彼得}}回答说:「你是基督,是永生 神的儿子。」
\VS{17}耶稣对他说:「{\PN{西门·巴·约拿}},你是有福的!因为这不是属血肉{\ADD{的}}指示你的,乃是我在天上的父指示的。
\VS{18}我还告诉你,你是{\PN{彼得}},我要把我的教会建造在这磐石上;阴间的权柄\FTNT{}{{\FR 16:18: }权柄:原文是门}不能胜过他。
\VS{19}我要把天国的钥匙给你,凡你在地上所捆绑的,在天上也要捆绑;凡你在地上所释放的,在天上也要释放。」
\VS{20}当下,耶稣嘱咐门徒,不可对人说他是基督。
\par }{\SH 耶稣预言受难和复活
\par }{\R (可8·31—9·1;路9·22—27)
\par }{\PP \VS{21}从此,耶稣才指示门徒,他必须上{\PN{耶路撒冷}}去,受长老、祭司长、文士许多的苦,并且被杀,第三日复活。
\VS{22}{\PN{彼得}}就拉着他,劝他说:「主啊,万不可如此!这事必不临到你身上。」
\VS{23}耶稣转过来,对{\PN{彼得}}说:「撒但,退我后边去吧!你是绊我脚的;因为你不体贴 神的意思,只体贴人的意思。」
\VS{24}于是耶稣对门徒说:「若有人要跟从我,就当舍己,背起他的十字架来跟从我。
\VS{25}因为,凡要救自己生命\FTNT{}{{\FR 16:25: }生命:或译灵魂;下同}的,必丧掉生命;凡为我丧掉生命的,必得着生命。
\VS{26}人若赚得全世界,赔上自己的生命,有什么益处呢?人还能拿什么换生命呢?
\VS{27}人子要在他父的荣耀里,同着众使者降临;那时候,他要照各人的行为报应各人。
\VS{28}我实在告诉你们,站在这里的,有人在没尝死味以前必看见人子降临在他的国里。」

\par }\Chap{17}{\SH 耶稣改变形象
\par }{\R (可9·2—13;路9·28—36)
\par }{\PP \VerseOne{1}过了六天,耶稣带着{\PN{彼得}}、{\PN{雅各}},和{\PN{雅各}}的兄弟{\PN{约翰}},暗暗地上了高山,
\VS{2}就在他们面前变了形象,脸面明亮如日头,衣裳洁白如光。
\VS{3}忽然,有{\PN{摩西}}、{\PN{以利亚}}向他们显现,同耶稣说话。
\VS{4}{\PN{彼得}}对耶稣说:「主啊,我们在这里真好!你若愿意,我就在这里搭三座棚,一座为你,一座为{\PN{摩西}},一座为{\PN{以利亚}}。」
\VS{5}说话之间,忽然有一朵光明的云彩遮盖他们,且有声音从云彩里出来,说:「这是我的爱子,我所喜悦的。你们要听他!」
\VS{6}门徒听见,就俯伏在地,极其害怕。
\VS{7}耶稣进前来,摸他们,说:「起来,不要害怕!」
\VS{8}他们举目不见一人,只见耶稣在那里。
\VS{9}下山的时候,耶稣吩咐他们说:「人子还没有从死里复活,你们不要将所看见的告诉人。」
\VS{10}门徒问耶稣说:「文士为什么说{\PN{以利亚}}必须先来?」
\VS{11}耶稣回答说:「{\PN{以利亚}}固然先来,并要复兴万事;
\VS{12}只是我告诉你们,{\PN{以利亚}}已经来了,人却不认识他,竟任意待他。人子也将要这样受他们的害。」
\VS{13}门徒这才明白耶稣所说的是指着施洗的{\PN{约翰}}。
\par }{\SH 治好被鬼附身的孩子
\par }{\R (可9·14—29;路9·37—43)
\par }{\PP \VS{14}耶稣和门徒到了众人那里,有一个人来见耶稣,跪下,
\VS{15}说:「主啊,怜悯我的儿子!他害癫痫的病很苦,屡次跌在火里,屡次跌在水里。
\VS{16}我带他到你门徒那里,他们却不能医治他。」
\VS{17}耶稣说:「嗳!这又不信又悖谬的世代啊,我在你们这里要到几时呢?我忍耐你们要到几时呢?把他带到我这里来吧!」
\VS{18}耶稣斥责那鬼,鬼就出来;从此孩子就痊愈了。
\VS{19}门徒暗暗地到耶稣跟前,说:「我们为什么不能赶出那鬼呢?」
\VS{20}耶稣说:「是因你们的信心小。我实在告诉你们,你们若有信心,像一粒芥菜种,就是对这座山说:『你从这边挪到那边』,它也必挪去;并且你们没有一件不能做的事了。
\VS{21}至于这一类的鬼,若不祷告、禁食,他就不出来\FTNT{}{{\FR 17:21: }或译:不能赶他出来}。」
\par }{\SH 耶稣第二次预言受难和复活
\par }{\R (可9·30—32;路9·43—45)
\par }{\PP \VS{22}他们还住在{\PN{加利利}}的时候,耶稣对门徒说:「人子将要被交在人手里。
\VS{23}他们要杀害他,第三日他要复活。」门徒就大大地忧愁。
\par }{\SH 缴纳殿税
\par }{\PP \VS{24}到了{\PN{迦百农}},有收丁税的人来见{\PN{彼得}},说:「你们的先生不纳丁税\FTNT{}{{\FR 17:24: }丁税约有半块钱}吗?」
\VS{25}{\PN{彼得}}说:「纳。」他进了屋子,耶稣先向他说:「{\PN{西门}},你的意思如何?世上的君王向谁征收关税、丁税?是向自己的儿子呢?是向外人呢?」
\VS{26}{\PN{彼得}}说:「是向外人。」耶稣说:「既然如此,儿子就可以免税了。
\VS{27}但恐怕触犯\FTNT{}{{\FR 17:27: }触犯:原文是绊倒}他们,你且往海边去钓鱼,把先钓上来的鱼拿起来,开了它的口,必得一块钱,可以拿去给他们,作你我的税银。」

\par }\Chap{18}{\SH 天国里最伟大的人
\par }{\R (可9·33—37;路9·46—48)
\par }{\PP \VerseOne{1}当时,门徒进前来,问耶稣说:「天国里谁是最大的?」
\VS{2}耶稣便叫一个小孩子来,使他站在他们当中,
\VS{3}说:「我实在告诉你们,你们若不回转,变成小孩子的样式,断不得进天国。
\VS{4}所以,凡自己谦卑像这小孩子的,他在天国里就是最大的。
\VS{5}凡为我的名接待一个像这小孩子的,就是接待我。」
\par }{\SH 罪的诱惑
\par }{\R (可9·42—48;路17·1—2)
\par }{\PP \VS{6}「凡使这信我的一个小子跌倒的,倒不如把大磨石拴在这人的颈项上,沉在深海里。
\VS{7}这世界有祸了,因为将人绊倒;绊倒人的事是免不了的,但那绊倒人的有祸了!
\VS{8}倘若你一只手,或是一只脚,叫你跌倒,就砍下来丢掉。你缺一只手,或是一只脚,进入永生,强如有两手两脚被丢在永火里。
\VS{9}倘若你一只眼叫你跌倒,就把它剜出来丢掉。你只有一只眼进入{\ADD{永}}生,强如有两只眼被丢在地狱的火里。」
\par }{\SH 迷羊的比喻
\par }{\R (路15·3—7)
\par }{\PP \VS{10}「你们要小心,不可轻看这小子里的一个;我告诉你们,他们的使者在天上,常见我天父的面。\FTNT{}{{\FR 18:10: }有古卷加:11
人子来,为要拯救失丧的人。}
\VS{12}一个人若有一百只羊,一只走迷了路,你们的意思如何?他岂不撇下这九十九只,往山里去找那只迷路的羊吗?
\VS{13}若是找着了,我实在告诉你们,他为这一只羊欢喜,比为那没有迷路的九十九只欢喜还大呢!
\VS{14}你们在天上的父也是这样,不愿意这小子里失丧一个。」
\par }{\SH 如何对待犯过错的弟兄
\par }{\PP \VS{15}「倘若你的弟兄得罪你,你就去,趁着只有他和你在一处的时候,指出他的错来。他若听你,你便得了你的弟兄;
\VS{16}他若不听,你就另外带一两个人同去,要凭两三个人的口作见证,句句都可定准。
\VS{17}若是不听他们,就告诉教会;若是不听教会,就看他像外邦人和税吏一样。
\par }{\PP \VS{18}「我实在告诉你们,凡你们在地上所捆绑的,在天上也要捆绑;凡你们在地上所释放的,在天上也要释放。
\VS{19}我又告诉你们,若是你们中间有两个人在地上同心合意地求什么事,我在天上的父必为他们成全。
\VS{20}因为无论在哪里,有两三个人奉我的名聚会,那里就有我在他们中间。」
\par }{\SH 不饶恕人的恶仆
\par }{\PP \VS{21}那时,{\PN{彼得}}进前来,对耶稣说:「主啊,我弟兄得罪我,我当饶恕他几次呢?到七次可以吗?」
\VS{22}耶稣说:「我对你说,不是到七次,乃是到七十个七次。
\VS{23}天国好像一个王要和他仆人算帐。
\VS{24}才算的时候,有人带了一个欠一千万银子的来。
\VS{25}因为他没有什么偿还之物,主人吩咐把他和他妻子儿女,并一切所有的都卖了偿还。
\VS{26}那仆人就俯伏拜他,说:『主啊,宽容我,将来我都要还清。』
\VS{27}那仆人的主人就动了慈心,把他释放了,并且免了他的债。
\par }{\PP \VS{28}「那仆人出来,遇见他的一个同伴欠他十两银子,便揪着他,掐住他的喉咙,说:『你把所欠的还我!』
\VS{29}他的同伴就俯伏央求他,说:『宽容我吧,将来我必还清。』
\VS{30}他不肯,竟去把他下在监里,等他还了所欠的债。
\VS{31}众同伴看见他所做的事就甚忧愁,去把这事都告诉了主人。
\VS{32}于是主人叫了他来,对他说:『你这恶奴才!你央求我,我就把你所欠的都免了,
\VS{33}你不应当怜恤你的同伴,像我怜恤你吗?』
\VS{34}主人就大怒,把他交给掌刑的,等他还清了所欠的债。
\VS{35}你们各人若不从心里饶恕你的弟兄,我天父也要这样待你们了。」

\par }\Chap{19}{\SH 休妻的问题
\par }{\R (可10·1—12)
\par }{\PP \VerseOne{1}耶稣说完了这些话,就离开{\PN{加利利}},来到{\PN{犹太}}的境界{\PN{约旦河}}外。
\VS{2}有许多人跟着他,他就在那里把他们{\ADD{的病人}}治好了。
\par }{\PP \VS{3}有法利赛人来试探耶稣,说:「人无论什么缘故都可以休妻吗?」
\VS{4}耶稣回答说:「那起初造人的,是造男造女,
\VS{5}并且说:『因此,人要离开父母,与妻子连合,二人成为一体。』{\ADD{这经}}你们没有念过吗?
\VS{6}既然如此,夫妻不再是两个人,乃是一体的了。所以, 神配合的,人不可分开。」
\VS{7}法利赛人说:「这样,{\PN{摩西}}为什么吩咐给妻子休书,就可以休她呢?」
\VS{8}耶稣说:「{\PN{摩西}}因为你们的心硬,所以许你们休妻,但起初并不是这样。
\VS{9}我告诉你们,凡休妻另娶的,若不是为淫乱的缘故,就是犯奸淫了;有人娶那被休的妇人,也是犯奸淫了。」
\VS{10}门徒对耶稣说:「人和妻子既是这样,倒不如不娶。」
\VS{11}耶稣说:「这话不是人都能领受的,惟独赐给谁,谁才能领受。
\VS{12}因为有生来是阉人,也有被人阉的,并有为天国的缘故自阉的。这话谁能领受就可以领受。」
\par }{\SH 耶稣为小孩祝福
\par }{\R (可10·13—16;路18·15—17)
\par }{\PP \VS{13}那时,有人带着小孩子来见耶稣,要耶稣给他们按手祷告,门徒就责备那些人。
\VS{14}耶稣说:「让小孩子到我这里来,不要禁止他们;因为在天国的,正是这样的人。」
\VS{15}耶稣给他们按手,就离开那地方去了。
\par }{\SH 少年财主
\par }{\R (可10·17—31;路18·18—30)
\par }{\PP \VS{16}有一个人来见耶稣,说:「夫子\FTNT{}{{\FR 19:16: }有古卷:良善的夫子},我该做什么善事才能得永生?」
\VS{17}耶稣对他说:「你为什么以善事问我呢?只有一位是善的\FTNT{}{{\FR 19:17: }有古卷:你为什么称我是良善的?除了 神以外,没有一个良善的}。你若要进入{\ADD{永}}生,就当遵守诫命。」
\VS{18}他说:「什么诫命?」耶稣说:「就是不可杀人;不可奸淫;不可偷盗;不可作假见证;
\VS{19}当孝敬父母;又当爱人如己。」
\VS{20}那少年人说:「这一切我都遵守了,还缺少什么呢?」
\VS{21}耶稣说:「你若愿意作完全人,可去变卖你所有的,分给穷人,就必有财宝在天上;你还要来跟从我。」
\VS{22}那少年人听见这话,就忧忧愁愁地走了,因为他的产业很多。
\par }{\PP \VS{23}耶稣对门徒说:「我实在告诉你们,财主进天国是难的。
\VS{24}我又告诉你们,骆驼穿过针的眼,比财主进 神的国还容易呢!」
\VS{25}门徒听见这话,就希奇得很,说:「这样谁能得救呢?」
\VS{26}耶稣看着他们,说:「在人这是不能的,在 神凡事都能。」
\VS{27}{\PN{彼得}}就对他说:「看哪,我们已经撇下所有的跟从你,将来我们要得什么呢?」
\VS{28}耶稣说:「我实在告诉你们,你们这跟从我的人,到复兴的时候,人子坐在他荣耀的宝座上,你们也要坐在十二个宝座上,审判{\PN{以色列}}十二个支派。
\VS{29}凡为我的名撇下房屋,或是弟兄、姊妹、父亲、母亲、\FTNT{}{{\FR 19:29: }有古卷加:妻子、} 儿女、田地的,必要得着百倍,并且承受永生。
\VS{30}然而,有许多在前的,将要在后;在后的,将要在前。」

\par }\Chap{20}{\SH 葡萄园的比喻
\par }{\PP \VerseOne{1}「因为天国好像家主清早去雇人进他的葡萄园做工,
\VS{2}和工人讲定一天一钱银子,就打发他们进葡萄园去。
\VS{3}约在巳初出去,看见市上还有闲站的人,
\VS{4}就对他们说:『你们也进葡萄园去,所当给的,我必给你们。』他们也进去了。
\VS{5}约在午正和申初又出去,也是这样行。
\VS{6}约在酉初出去,看见还有人站在那里,就问他们说:『你们为什么整天在这里闲站呢?』
\VS{7}他们说:『因为没有人雇我们。』他说:『你们也进葡萄园去。』
\VS{8}到了晚上,园主对管事的说:『叫工人都来,给他们工钱,从后来的起,到先来的为止。
\VS{9}约在酉初雇的人来了,各人得了一钱银子。
\VS{10}及至那先雇的来了,他们以为必要多得;谁知也是各得一钱。
\VS{11}他们得了,就埋怨家主说:
\VS{12}『我们整天劳苦受热,那后来的只做了一小时,你竟叫他们和我们一样吗?』
\VS{13}家主回答其中的一人说:『朋友,我不亏负你,你与我讲定的不是一钱银子吗?
\VS{14}拿你的走吧!我给那后来的和给你一样,这是我愿意的。
\VS{15}我的东西难道不可随我的意思用吗?因为我作好人,你就红了眼吗?』
\VS{16}这样,那在后的,将要在前;在前的,将要在后了。\FTNT{}{{\FR 20:16: }有古卷加:因为被召的人多,选上的人少。}」
\par }{\SH 耶稣第三次预言受难和复活
\par }{\R (可10·32—34;路18·31—34)
\par }{\PP \VS{17}耶稣上{\PN{耶路撒冷}}去的时候,在路上把十二个门徒带到一边,对他们说:
\VS{18}「看哪,我们上{\PN{耶路撒冷}}去,人子要被交给祭司长和文士。他们要定他死罪,
\VS{19}又交给外邦人,将他戏弄,鞭打,钉在十字架上;第三日他要复活。」
\par }{\SH 母亲的请求
\par }{\R (可10·35—45)
\par }{\PP \VS{20}那时,{\PN{西庇太}}儿子的母亲同她两个儿子上前来拜耶稣,求他一件事。
\VS{21}耶稣说:「你要什么呢?」她说:「愿你叫我这两个儿子在你国里,一个坐在你右边,一个坐在你左边。」
\VS{22}耶稣回答说:「你们不知道所求的是什么;我将要喝的杯,你们能喝吗?」他们说:「我们能。」
\VS{23}耶稣说:「我所喝的杯,你们必要喝;只是坐在我的左右,不是我可以赐的,乃是我父为谁预备的,就赐给谁。」
\VS{24}那十个门徒听见,就恼怒他们弟兄二人。
\VS{25}耶稣叫了他们来,说:「你们知道外邦人有君王为主治理他们,有大臣操权管束他们。
\VS{26}只是在你们中间,不可这样;你们中间谁愿为大,就必作你们的用人;
\VS{27}谁愿为首,就必作你们的仆人。
\VS{28}正如人子来,不是要受人的服事,乃是要服事人,并且要舍命,作多人的赎价。」
\par }{\SH 治好两个瞎子
\par }{\R (可10·46—52;路18·35—43)
\par }{\PP \VS{29}他们出{\PN{耶利哥}}的时候,有极多的人跟随他。
\VS{30}有两个瞎子坐在路旁,听说是耶稣经过,就喊着说:「主啊,{\PN{大卫}}的子孙,可怜我们吧!」
\VS{31}众人责备他们,不许他们作声;他们却越发喊着说:「主啊,{\PN{大卫}}的子孙,可怜我们吧!」
\VS{32}耶稣就站住,叫他们来,说:「要我为你们做什么?」
\VS{33}他们说:「主啊,要我们的眼睛能看见!」
\VS{34}耶稣就动了慈心,把他们的眼睛一摸,他们立刻看见,就跟从了耶稣。

\par }\Chap{21}{\SH 光荣地进耶路撒冷
\par }{\R (可11·1—11;路19·28—40;约12·12—19)
\par }{\PP \VerseOne{1}耶稣和门徒将近{\PN{耶路撒冷}},到了{\PN{伯法其}},在{\PN{橄榄山}}那里。
\VS{2}耶稣就打发两个门徒,对他们说:「你们往对面村子里去,必看见一匹驴拴在那里,还有驴驹同在一处;你们解开,牵到我这里来。
\VS{3}若有人对你们说什么,你们就说:『主要用它。』那人必立时让你们牵来。
\VS{4}这事成就是要应验先知的话,说:
\VS{5}要对{\PN{锡安}}的居民\FTNT{}{{\FR 21:5: }原文是女子}说:
\par }{\Q 看哪,你的王来到你这里,
\par }{\Q 是温柔的,又骑着驴,
\par }{\Q 就是骑着驴驹子。」
\par }{\PP \VS{6}门徒就照耶稣所吩咐的去行,
\VS{7}牵了驴和驴驹来,把自己的衣服搭在上面,耶稣就骑上。
\VS{8}众人多半把衣服铺在路上;还有人砍下树枝来铺在路上。
\VS{9}前行后随的众人喊着说:
\par }{\Q 和散那\FTNT{}{{\FR 21:9: }原有求救的意思,在此是称颂的话}归于{\PN{大卫}}的子孙!
\par }{\Q 奉主名来的是应当称颂的!
\par }{\Q 高高在上和散那!
\par }{\PP \VS{10}耶稣既进了{\PN{耶路撒冷}},合城都惊动了,说:「这是谁?」
\VS{11}众人说:「这是{\PN{加利利}}{\PN{拿撒勒}}的先知耶稣。」
\par }{\SH 洁净圣殿
\par }{\R (可11·15—19;路19·45—48;约2·13—22)
\par }{\PP \VS{12}耶稣进了 神的殿,赶出殿里一切做买卖的人,推倒兑换银钱之人的桌子,和卖鸽子之人的凳子,
\VS{13}对他们说:「{\ADD{经上}}记着说:
\par }{\Q 我的殿必称为祷告的殿,
\par }{\Q 你们倒使它成为贼窝了。」
\par }{\PP \VS{14}在殿里有瞎子、瘸子到耶稣跟前,他就治好了他们。
\VS{15}祭司长和文士看见耶稣所行的奇事,又见小孩子在殿里喊着说:「和散那归于{\PN{大卫}}的子孙!」就甚恼怒,
\VS{16}对他说:「这些人所说的,你听见了吗?」耶稣说:「是的。{\ADD{经上}}说『你从婴孩和吃奶的口中完全了赞美』的话,你们没有念过吗?」
\VS{17}于是离开他们,出城到{\PN{伯大尼}}去,在那里住宿。
\par }{\SH 咒诅无花果树
\par }{\R (可11·12—14,20—24)
\par }{\PP \VS{18}早晨回城的时候,他饿了,
\VS{19}看见路旁有一棵无花果树,就走到跟前,在树上找不着什么,不过有叶子,就对树说:「从今以后,你永不结果子。」那无花果树就立刻枯干了。
\VS{20}门徒看见了,便希奇说:「无花果树怎么立刻枯干了呢?」
\VS{21}耶稣回答说:「我实在告诉你们,你们若有信心,不疑惑,不但能行无花果树上所行的事,就是对这座山说:『你挪开此地,投在海里!』也必成就。
\VS{22}你们祷告,无论求什么,只要信,就必得着。」
\par }{\SH 质问耶稣的权柄
\par }{\R (可11·27—33;路20·1—8)
\par }{\PP \VS{23}耶稣进了殿,正教训人的时候,祭司长和民间的长老来问他说:「你仗着什么权柄做这些事?给你这权柄的是谁呢?」
\VS{24}耶稣回答说:「我也要问你们一句话,你们若告诉我,我就告诉你们我仗着什么权柄做这些事。
\VS{25}{\PN{约翰}}的洗礼是从哪里来的?是从天上来的?是从人间来的呢?」他们彼此商议说:「我们若说『从天上来』,他必对我们说:『这样,你们为什么不信他呢?』
\VS{26}若说『从人间来』,我们又怕百姓,因为他们都以{\PN{约翰}}为先知。」
\VS{27}于是回答耶稣说:「我们不知道。」耶稣说:「我也不告诉你们我仗着什么权柄做这些事。」
\par }{\SH 两个儿子的比喻
\par }{\PP \VS{28}又说:「一个人有两个儿子。他来对大儿子说:『我儿,你今天到葡萄园里去做工。』
\VS{29}他回答说:『我不去』,以后自己懊悔,就去了。
\VS{30}又来对小儿子也是这样说。他回答说:『父啊,我去』,他却不去。
\VS{31}你们想,这两个儿子是哪一个遵行父命呢?」他们说:「大儿子。」耶稣说:「我实在告诉你们,税吏和娼妓倒比你们先进 神的国。
\VS{32}因为{\PN{约翰}}遵着义路到你们这里来,你们却不信他;税吏和娼妓倒信他。你们看见了,后来还是不懊悔去信他。」
\par }{\SH 凶恶园户的比喻
\par }{\R (可12·1—12;路20·9—19)
\par }{\PP \VS{33}「你们再听一个比喻:有个家主栽了一个葡萄园,周围圈上篱笆,里面挖了一个压酒池,盖了一座楼,租给园户,就往外国去了。
\VS{34}收果子的时候近了,就打发仆人到园户那里去收果子。
\VS{35}园户拿住仆人,打了一个,杀了一个,用石头打死一个。
\VS{36}主人又打发别的仆人去,比先前更多;园户还是照样待他们。
\VS{37}后来打发他的儿子到他们那里去,{\ADD{意思}}说:『他们必尊敬我的儿子。』
\VS{38}不料,园户看见他儿子,就彼此说:『这是承受产业的。来吧,我们杀他,占他的产业!』
\VS{39}他们就拿住他,推出葡萄园外,杀了。
\VS{40}园主来的时候要怎样处治这些园户呢?」
\VS{41}他们说:「要下毒手除灭那些恶人,将葡萄园另租给那按着时候交果子的园户。」
\VS{42}耶稣说:「{\ADD{经上写着}}:
\par }{\Q 匠人所弃的石头
\par }{\Q 已作了房角的头块石头。
\par }{\Q 这是主所做的,
\par }{\Q 在我们眼中看为希奇。
\par }{\MM 这经你们没有念过吗?
\VS{43}所以我告诉你们, 神的国必从你们夺去,赐给那能结果子的百姓。
\VS{44}谁掉在这石头上,必要跌碎;这石头掉在谁的身上,就要把谁砸得稀烂。」
\VS{45}祭司长和法利赛人听见他的比喻,就看出他是指着他们说的。
\VS{46}他们想要捉拿他,只是怕众人,因为众人以他为先知。

\par }\Chap{22}{\SH 喜筵的比喻
\par }{\R (路14·15—24)
\par }{\PP \VerseOne{1}耶稣又用比喻对他们说:
\VS{2}「天国好比一个王为他儿子摆设娶亲的筵席,
\VS{3}就打发仆人去,请那些被召的人来赴席,他们却不肯来。
\VS{4}王又打发别的仆人,说:『你们告诉那被召的人,我的筵席已经预备好了,牛和肥畜已经宰了,各样都齐备,请你们来赴席。』
\VS{5}那些人不理就走了;一个到自己田里去;一个做买卖去;
\VS{6}其余的拿住仆人,凌辱他们,把他们杀了。
\VS{7}王就大怒,发兵除灭那些凶手,烧毁他们的城。
\VS{8}于是对仆人说:『喜筵已经齐备,只是所召的人不配。
\VS{9}所以你们要往岔路口上去,凡遇见的,都召来赴席。』
\VS{10}那些仆人就出去,到大路上,凡遇见的,不论善恶都召聚了来,筵席上就坐满了客。
\VS{11}王进来观看宾客,见那里有一个没有穿礼服的,
\VS{12}就对他说:『朋友,你到这里来怎么不穿礼服呢?』那人无言可答。
\VS{13}于是王对使唤的人说:『捆起他的手脚来,把他丢在外边的黑暗里;在那里必要哀哭切齿了。』
\VS{14}因为被召的人多,选上的人少。」
\par }{\SH 纳税给凯撒的问题
\par }{\R (可12·13—17;路20·19—26)
\par }{\PP \VS{15}当时,法利赛人出去商议,怎样就着耶稣的话陷害他,
\VS{16}就打发他们的门徒同{\PN{希律}}党的人去见耶稣,说:「夫子,我们知道你是诚实人,并且诚诚实实传 神的道,什么人你都不徇情面,因为你不看人的外貌。
\VS{17}请告诉我们,你的意见如何?纳税给凯撒可以不可以?」
\VS{18}耶稣看出他们的恶意,就说:「假冒为善的人哪,为什么试探我?
\VS{19}拿一个上税的钱给我看!」他们就拿一个银钱来给他。
\VS{20}耶稣说:「这像和这号是谁的?」
\VS{21}他们说:「是凯撒的。」耶稣说:「这样,凯撒的物当归给凯撒; 神的物当归给 神。」
\VS{22}他们听见就希奇,离开他走了。
\par }{\SH 复活的问题
\par }{\R (可12·18—27;路20·27—40)
\par }{\PP \VS{23}撒都该人常说没有复活的事。那天,他们来问耶稣说:
\VS{24}「夫子,{\PN{摩西}}说:『人若死了,没有孩子,他兄弟当娶他的妻,为哥哥生子立后。』
\VS{25}从前,在我们这里有弟兄七人,第一个娶了妻,死了,没有孩子,撇下妻子给兄弟。
\VS{26}第二、第三,直到第七个,都是如此。
\VS{27}末后,妇人也死了。
\VS{28}这样,当复活的时候,她是七个人中哪一个的妻子呢?因为他们都娶过她。」
\VS{29}耶稣回答说:「你们错了;因为不明白圣经,也不晓得 神的大能。
\VS{30}当复活的时候,人也不娶也不嫁,乃像天上的使者一样。
\VS{31}论到死人复活, 神{\ADD{在经上}}向你们所说的,你们没有念过吗?
\VS{32}他说:『我是{\PN{亚伯拉罕}}的 神,{\PN{以撒}}的 神,{\PN{雅各}}的 神。』 神不是死人的 神,乃是活人的 神。」
\VS{33}众人听见这话,就希奇他的教训。
\par }{\SH 最大的诫命
\par }{\R (可12·28—34;路10·25—28)
\par }{\PP \VS{34}法利赛人听见耶稣堵住了撒都该人的口,他们就聚集。
\VS{35}内中有一个人是律法师,要试探耶稣,就问他说:
\VS{36}「夫子,律法上的诫命,哪一条是最大的呢?」
\VS{37}耶稣对他说:「你要尽心、尽性、尽意爱主—你的 神。
\VS{38}这是诫命中的第一,且是最大的。
\VS{39}其次也相仿,就是要爱人如己。
\VS{40}这两条诫命是律法和先知一切{\ADD{道理}}的总纲。」
\par }{\SH 大卫子孙的问题
\par }{\R (可12·35—37;路20·41—44)
\par }{\PP \VS{41}法利赛人聚集的时候,耶稣问他们说:
\VS{42}「论到基督,你们的意见如何?他是谁的子孙呢?」他们回答说:「是{\PN{大卫}}的子孙。」
\VS{43}耶稣说:「这样,{\PN{大卫}}被圣灵感动,怎么还称他为主,说:
\par }{\PP \VS{44}主对我主说:
\par }{\Q 你坐在我的右边,
\par }{\Q 等我把你仇敌放在你的脚下。
\par }{\MM \VS{45}{\PN{大卫}}既称他为主,他怎么又是{\PN{大卫}}的子孙呢?」
\VS{46}他们没有一个人能回答一言。从那日以后,也没有人敢再问他什么。

\par }\Chap{23}{\SH 谴责文士和法利赛人
\par }{\R (可12·38—40;路11·37—52;20·45—47)
\par }{\PP \VerseOne{1}那时,耶稣对众人和门徒讲论,
\VS{2}说:「文士和法利赛人坐在{\PN{摩西}}的位上,
\VS{3}凡他们所吩咐你们的,你们都要谨守遵行。但不要效法他们的行为;因为他们能说,不能行。
\VS{4}他们把难担的重担捆起来,搁在人的肩上,但自己一个指头也不肯动。
\VS{5}他们一切所做的事都是要叫人看见,所以将佩戴的经文做宽了,{\ADD{衣裳的}} 子做长了,
\VS{6}喜爱筵席上的首座,会堂里的高位,
\VS{7}又喜爱人在街市上问他安,称呼他拉比\FTNT{}{{\FR 23:7: }拉比就是夫子}。
\VS{8}但你们不要受拉比的称呼,因为只有一位是你们的夫子;你们都是弟兄。
\VS{9}也不要称呼地上的人为父,因为只有一位是你们的父,就是在天上的父。
\VS{10}也不要受师尊的称呼,因为只有一位是你们的师尊,就是基督。
\VS{11}你们中间谁为大,谁就要作你们的用人。
\VS{12}凡自高的,必降为卑;自卑的,必升为高。
\par }{\PP \VS{13}「你们这假冒为善的文士和法利赛人有祸了!因为你们正当人前,把天国{\ADD{的门}}关了,自己不进去,正要进去的人,你们也不容他们进去。\FTNT{}{{\FR 23:13: }有古卷加:
14你们这假冒为善的文士和法利赛人有祸了!因为你们侵吞寡妇的家产,假意做很长的祷告,所以要受更重的刑罚。}
\par }{\PP \VS{15}「你们这假冒为善的文士和法利赛人有祸了!因为你们走遍洋海陆地,勾引一个人入教,既入了教,却使他作地狱之子,比你们还加倍。
\par }{\PP \VS{16}「你们这瞎眼领路的有祸了!你们说:『凡指着殿起誓的,这算不得什么;只是凡指着殿中金子起誓的,他就该谨守。』
\VS{17}你们这无知瞎眼的人哪,什么是大的?是金子呢?还是叫金子成圣的殿呢?
\VS{18}你们又说:『凡指着坛起誓的,这算不得什么;只是凡指着坛上礼物起誓的,他就该谨守。』
\VS{19}你们这瞎眼的人哪,什么是大的?是礼物呢?还是叫礼物成圣的坛呢?
\VS{20}所以,人指着坛起誓,就是指着坛和坛上一切所有的起誓;
\VS{21}人指着殿起誓,就是指着殿和那住在殿里的起誓;
\VS{22}人指着天起誓,就是指着 神的宝座和那坐在上面的起誓。
\par }{\PP \VS{23}「你们这假冒为善的文士和法利赛人有祸了!因为你们将薄荷、茴香、芹菜献上十分之一,那律法上更重的事,就是公义、怜悯、信实,反倒不行了。这{\ADD{更重的}}是你们当行的;那也是不可不行的。
\VS{24}你们这瞎眼领路的,蠓虫你们就滤出来,骆驼你们倒吞下去。
\par }{\PP \VS{25}「你们这假冒为善的文士和法利赛人有祸了!因为你们洗净杯盘的外面,里面却盛满了勒索和放荡。
\VS{26}你这瞎眼的法利赛人,先洗净杯盘的里面,好叫外面也干净了。
\par }{\PP \VS{27}「你们这假冒为善的文士和法利赛人有祸了!因为你们好像粉饰的坟墓,外面好看,里面却装满了死人的骨头和一切的污秽。
\VS{28}你们也是如此,在人前,外面显出公义来,里面却装满了假善和不法的事。
\par }{\PP \VS{29}「你们这假冒为善的文士和法利赛人有祸了!因为你们建造先知的坟,修饰义人的墓,说:
\VS{30}『若是我们在我们祖宗的时候,必不和他们同流先知的血。』
\VS{31}这就是你们自己证明是杀害先知者的子孙了。
\VS{32}你们去充满你们祖宗的恶贯吧!
\VS{33}你们这些蛇类、毒蛇之种啊,怎能逃脱地狱的刑罚呢?
\VS{34}所以我差遣先知和智慧人并文士到你们这里来,有的你们要杀害,要钉十字架;有的你们要在会堂里鞭打,从这城追逼到那城,
\VS{35}叫世上所流义人的血都归到你们身上,从义人{\PN{亚伯}}的血起,直到你们在殿和坛中间所杀的{\PN{巴拉加}}的儿子{\PN{撒迦利亚}}的血为止。
\VS{36}我实在告诉你们,这一切的罪都要归到这世代了。」
\par }{\SH 为耶路撒冷哀哭
\par }{\R (路13·34—35)
\par }{\PP \VS{37}「{\PN{耶路撒冷}}啊,{\PN{耶路撒冷}}啊,你常杀害先知,又用石头打死那奉差遣到你这里来的人。我多次愿意聚集你的儿女,好像母鸡把小鸡聚集在翅膀底下,只是你们不愿意。
\VS{38}看哪,你们的家成为荒场留给你们。
\VS{39}我告诉你们,从今以后,你们不得再见我,直等到你们说:『奉主名来的是应当称颂的。』」

\par }\Chap{24}{\SH 预言圣殿被毁
\par }{\R (可13·1—2;路21·5—6)
\par }{\PP \VerseOne{1}耶稣出了{\ADD{圣}}殿,正走的时候,门徒进前来,把殿宇指给他看。
\VS{2}耶稣对他们说:「你们不是看见这殿宇吗?我实在告诉你们,将来在这里没有一块石头留在石头上,不被拆毁了。」
\par }{\SH 灾难的起头
\par }{\R (可13·3—13;路21·7—19)
\par }{\PP \VS{3}耶稣在{\PN{橄榄山}}上坐着,门徒暗暗地来说:「请告诉我们,什么时候有这些事?你降临和世界的末了有什么预兆呢?」
\VS{4}耶稣回答说:「你们要谨慎,免得有人迷惑你们。
\VS{5}因为将来有好些人冒我的名来,说:『我是基督』,并且要迷惑许多人。
\VS{6}你们也要听见打仗和打仗的风声,总不要惊慌;因为{\ADD{这些事}}是必须有的,只是末期还没有到。
\VS{7}民要攻打民,国要攻打国;多处必有饥荒、地震。
\VS{8}这都是灾难\FTNT{}{{\FR 24:8: }灾难:原文是生产之难}的起头。
\VS{9}那时,人要把你们陷在患难里,也要杀害你们;你们又要为我的名被万民恨恶。
\VS{10}那时,必有许多人跌倒,也要彼此陷害,彼此恨恶;
\VS{11}且有好些假先知起来,迷惑多人。
\VS{12}只因不法的事增多,许多人的爱心才渐渐冷淡了。
\VS{13}惟有忍耐到底的,必然得救。
\VS{14}这天国的福音要传遍天下,对万民作见证,然后末期才来到。」
\par }{\SH 大灾难
\par }{\R (可13·14—23;路21·20—24)
\par }{\PP \VS{15}「你们看见先知{\PN{但以理}}所说的『那行毁坏可憎的』站在圣地(读{\ADD{这经}}的人须要会意)。
\VS{16}那时,在{\PN{犹太}}的,应当逃到山上;
\VS{17}在房上的,不要下来拿家里的东西;
\VS{18}在田里的,也不要回去取衣裳。
\VS{19}当那些日子,怀孕的和奶孩子的有祸了。
\VS{20}你们应当祈求,叫你们逃走的时候,不遇见冬天或是安息日。
\VS{21}因为那时必有大灾难,从世界的起头直到如今,没有这样的灾难,后来也必没有。
\VS{22}若不减少那日子,凡有血气的总没有一个得救的;只是为选民,那日子必减少了。
\VS{23}那时,若有人对你们说:『基督在这里』,或说:『基督在那里』,你们不要信!
\VS{24}因为假基督、假先知将要起来,显大神迹、大奇事,倘若能行,连选民也就迷惑了。
\VS{25}看哪,我预先告诉你们了。
\VS{26}若有人对你们说:『看哪,基督在旷野里』,你们不要出去!{\ADD{或说}}:『看哪,基督在内屋中』,你们不要信!
\VS{27}闪电从东边发出,直照到西边。人子降临也要这样。
\VS{28}尸首在哪里,鹰也必聚在那里。」
\par }{\SH 人子的降临
\par }{\R (可13·24—27;路21·25—28)
\par }{\PP \VS{29}「那些日子的灾难一过去,
\par }{\Q 日头就变黑了,
\par }{\Q 月亮也不放光,
\par }{\Q 众星要从天上坠落,
\par }{\Q 天势都要震动。
\par }{\MM \VS{30}那时,人子的兆头要显在天上,地上的万族都要哀哭。他们要看见人子,有能力,有大荣耀,驾着天上的云降临。
\VS{31}他要差遣使者,用号筒的大声,将他的选民,从四方\FTNT{}{{\FR 24:31: }方:原文是风},从天这边到天那边,都招聚了来。」
\par }{\SH 从无花果树学个比方
\par }{\R (可13·28—31;路21·29—33)
\par }{\PP \VS{32}「你们可以从无花果树学个比方:当树枝发嫩长叶的时候,你们就知道夏天近了。
\VS{33}这样,你们看见这一切的事,也该知道人子近了,正在门口了。
\VS{34}我实在告诉你们,这世代还没有过去,这些事都要成就。
\VS{35}天地要废去,我的话却不能废去。」
\par }{\SH 那日那时没有人知道
\par }{\R (可13·32—37;路17·26—30,34—36)
\par }{\PP \VS{36}「但那日子,那时辰,没有人知道,连天上的使者也不知道,子也不知道,惟独父知道。
\VS{37}{\PN{挪亚}}的日子怎样,人子降临也要怎样。
\VS{38}当洪水以前的日子,人照常吃喝嫁娶,直到{\PN{挪亚}}进方舟的那日;
\VS{39}不知不觉洪水来了,把他们全都冲去。人子降临也要这样。
\VS{40}那时,两个人在田里,取去一个,撇下一个。
\VS{41}两个女人推磨,取去一个,撇下一个。
\VS{42}所以,你们要警醒,因为不知道你们的主是哪一天来到。
\VS{43}家主若知道几更天有贼来,就必警醒,不容人挖透房屋;这是你们所知道的。
\VS{44}所以,你们也要预备,因为你们想不到的时候,人子就来了。」
\par }{\SH 忠心和不忠心的仆人
\par }{\R (路12·41—48)
\par }{\PP \VS{45}「谁是忠心有见识的仆人,为主人所派,管理家里的人,按时分粮给他们呢?
\VS{46}主人来到,看见他这样行,那仆人就有福了。
\VS{47}我实在告诉你们,主人要派他管理一切所有的。
\VS{48}倘若那恶仆心里说:『我的主人必来得迟』,
\VS{49}就动手打他的同伴,又和酒醉的人一同吃喝。
\VS{50}在想不到的日子,不知道的时辰,那仆人的主人要来,
\VS{51}重重地处治他\FTNT{}{{\FR 24:51: }或译:把他腰斩了},定他和假冒为善的人同罪;在那里必要哀哭切齿了。」

\par }\Chap{25}{\SH 十童女的比喻
\par }{\PP \VerseOne{1}「那时,天国好比十个童女拿着灯出去迎接新郎。
\VS{2}其中有五个是愚拙的,五个是聪明的。
\VS{3}愚拙的拿着灯,却不预备油;
\VS{4}聪明的拿着灯,又预备油在器皿里。
\VS{5}新郎迟延的时候,她们都打盹,睡着了。
\VS{6}半夜有人喊着说:『新郎来了,你们出来迎接他!』
\VS{7}那些童女就都起来收拾灯。
\VS{8}愚拙的对聪明的说:『请分点油给我们,因为我们的灯要灭了。』
\VS{9}聪明的回答说:『恐怕不够你我用的;不如你们自己到卖油的那里去买吧。』
\VS{10}她们去买的时候,新郎到了。那预备好了的,同他进去坐席,门就关了。
\VS{11}其余的童女随后也来了,说:『主啊,主啊,给我们开门!』
\VS{12}他却回答说:『我实在告诉你们,我不认识你们。』
\VS{13}所以,你们要警醒;因为那日子,那时辰,你们不知道。」
\par }{\SH 按才干接受托付的比喻
\par }{\R (路19·11—27)
\par }{\PP \VS{14}「{\ADD{天国}}又好比一个人要往外国去,就叫了仆人来,把他的家业交给他们,
\VS{15}按着各人的才干给他们银子:一个给了五千,一个给了二千,一个给了一千,就往外国去了。
\VS{16}那领五千的随即拿去做买卖,另外赚了五千。
\VS{17}那领二千的也照样另赚了二千。
\VS{18}但那领一千的去掘开地,把主人的银子埋藏了。
\VS{19}过了许久,那些仆人的主人来了,和他们算帐。
\VS{20}那领五千银子的又带着那另外的五千来,说:『主啊,你交给我五千银子。请看,我又赚了五千。』
\VS{21}主人说:『好,你这又良善又忠心的仆人,你在不多的事上有忠心,我要把许多事派你管理;可以进来享受你主人的快乐。』
\VS{22}那领二千的也来,说:『主啊,你交给我二千银子。请看,我又赚了二千。』
\VS{23}主人说:『好,你这又良善又忠心的仆人,你在不多的事上有忠心,我要把许多事派你管理;可以进来享受你主人的快乐。』
\VS{24}那领一千的也来,说:『主啊,我知道你是忍心的人,没有种的地方要收割,没有散的地方要聚敛,
\VS{25}我就害怕,去把你的一千银子埋藏在地里。请看,你的原银子在这里。』
\VS{26}主人回答说:『你这又恶又懒的仆人,你既知道我没有种的地方要收割,没有散的地方要聚敛,
\VS{27}就当把我的银子放给兑换银钱的人,到我来的时候,可以连本带利收回。
\VS{28}夺过他这一千来,给那有一万的。
\VS{29}因为凡有的,还要加给他,叫他有余;没有的,连他所有的也要夺过来。
\VS{30}把这无用的仆人丢在外面黑暗里;在那里必要哀哭切齿了。』」
\par }{\SH 万民受审判
\par }{\PP \VS{31}「当人子在他荣耀里、同着众天使降临的时候,要坐在他荣耀的宝座上。
\VS{32}万民都要聚集在他面前。他要把他们分别出来,好像牧羊的分别绵羊山羊一般,
\VS{33}把绵羊安置在右边,山羊在左边。
\VS{34}于是王要向那右边的说:『你们这蒙我父赐福的,可来承受那创世以来为你们所预备的国;
\VS{35}因为我饿了,你们给我吃,渴了,你们给我喝;我作客旅,你们留我住;
\VS{36}我赤身露体,你们给我穿;我病了,你们看顾我;我在监里,你们来看我。』
\VS{37}义人就回答说:『主啊,我们什么时候见你饿了,给你吃,渴了,给你喝?
\VS{38}什么时候见你作客旅,留你住,或是赤身露体,给你穿?
\VS{39}又什么时候见你病了,或是在监里,来看你呢?』
\VS{40}王要回答说:『我实在告诉你们,这些事你们既做在我这弟兄中一个最小的身上,就是做在我身上了。』
\VS{41}王又要向那左边的说:『你们这被咒诅的人,离开我!进入那为魔鬼和他的使者所预备的永火里去!
\VS{42}因为我饿了,你们不给我吃,渴了,你们不给我喝;
\VS{43}我作客旅,你们不留我住;我赤身露体,你们不给我穿;我病了,我在监里,你们不来看顾我。』
\VS{44}他们也要回答说:『主啊,我们什么时候见你饿了,或渴了,或作客旅,或赤身露体,或病了,或在监里,不伺候你呢?』
\VS{45}王要回答说:『我实在告诉你们,这些事你们既不做在我这{\ADD{弟兄}}中一个最小的身上,就是不做在我身上了。』
\VS{46}这些人要往永刑里去;那些义人要往永生里去。」

\par }\Chap{26}{\SH 杀害耶稣的阴谋
\par }{\R (可14·1—2;路22·1—2;约11·45—53)
\par }{\PP \VerseOne{1}耶稣说完了这一切的话,就对门徒说:
\VS{2}「你们知道,过两天是逾越节,人子将要被交给人,钉在十字架上。」
\VS{3}那时,祭司长和民间的长老聚集在大祭司称为{\PN{该亚法}}的院里。
\VS{4}大家商议要用诡计拿住耶稣,杀他,
\VS{5}只是说:「当节的日子不可,恐怕民间生乱。」
\par }{\SH 在伯大尼受膏
\par }{\R (可14·3—9;约12·1—8)
\par }{\PP \VS{6}耶稣在{\PN{伯大尼}}长大麻风的{\PN{西门}}家里,
\VS{7}有一个女人拿着一玉瓶极贵的香膏来,趁耶稣坐席的时候,浇在他的头上。
\VS{8}门徒看见就很不喜悦,说:「何用这样的枉费呢!
\VS{9}这{\ADD{香膏}}可以卖许多钱,周济穷人。」
\VS{10}耶稣看出他们的意思,就说:「为什么难为这女人呢?她在我身上做的是一件美事。
\VS{11}因为常有穷人和你们同在;只是你们不常有我。
\VS{12}她将这香膏浇在我身上是为我安葬做的。
\VS{13}我实在告诉你们,普天之下,无论在什么地方传这福音,也要述说这女人所行的,作个纪念。」
\par }{\SH 犹大出卖耶稣
\par }{\R (可14·10—11;路22·3—6)
\par }{\PP \VS{14}当下,十二门徒里有一个称为{\PN{加略}}人{\PN{犹大}}的,去见祭司长,
\VS{15}说:「我把他交给你们,你们愿意给我多少钱?」他们就给了他三十块钱。
\VS{16}从那时候,他就找机会要把耶稣交给他们。
\par }{\SH 和门徒同度逾越节
\par }{\R (可14·12—21;路22·7—14,21—23;约13·21—30)
\par }{\PP \VS{17}除酵节的第一天,门徒来问耶稣说:「你吃逾越{\ADD{节的筵席}},要我们在哪里给你预备?」
\VS{18}耶稣说:「你们进城去,到某人那里,对他说:『夫子说:我的时候快到了,我与门徒要在你家里守逾越{\ADD{节}}。』」
\VS{19}门徒遵着耶稣所吩咐的就去预备了逾越{\ADD{节的筵席}}。
\VS{20}到了晚上,耶稣和十二个门徒坐席。
\VS{21}正吃的时候,耶稣说:「我实在告诉你们,你们中间有一个人要卖我了。」
\VS{22}他们就甚忧愁,一个一个地问他说:「主,是我吗?」
\VS{23}耶稣回答说:「同我蘸手在盘子里的,就是他要卖我。
\VS{24}人子必要去世,正如{\ADD{经上}}指着他所写的;但卖人子的人有祸了!那人不生在世上倒好。」
\VS{25}卖耶稣的{\PN{犹大}}问他说:「拉比,是我吗?」耶稣说:「你说的是。」
\par }{\SH 设立主的晚餐
\par }{\R (可14·22—26;路22·15—20;林前11·23—25)
\par }{\PP \VS{26}他们吃的时候,耶稣拿起饼来,祝福,就擘开,递给门徒,说:「你们拿着吃,这是我的身体」;
\VS{27}又拿起杯来,祝谢了,递给他们,说:「你们都喝这个;
\VS{28}因为这是我立约的血,为多人流出来,使罪得赦。
\VS{29}但我告诉你们,从今以后,我不再喝这葡萄汁,直到我在我父的国里同你们喝新的那日子。」
\VS{30}他们唱了诗,就出来往{\PN{橄榄山}}去。
\par }{\SH 预言彼得不认主
\par }{\R (可14·27—31;路22·31—34;约13·36—38)
\par }{\PP \VS{31}那时,耶稣对他们说:「今夜,你们为我的缘故都要跌倒。因为{\ADD{经上}}记着说:
\par }{\Q 我要击打牧人,
\par }{\Q 羊就分散了。
\par }{\MM \VS{32}但我复活以后,要在你们以先往{\PN{加利利}}去。」
\VS{33}{\PN{彼得}}说:「众人虽然为你的缘故跌倒,我却永不跌倒。」
\VS{34}耶稣说:「我实在告诉你,今夜鸡叫以先,你要三次不认我。」
\VS{35}{\PN{彼得}}说:「我就是必须和你同死,也总不能不认你。」众门徒都是这样说。
\par }{\SH 在客西马尼祷告
\par }{\R (可14·32—42;路22·39—46)
\par }{\PP \VS{36}耶稣同门徒来到一个地方,名叫{\PN{客西马尼}},就对他们说:「你们坐在这里,等我到那边去祷告。」
\VS{37}于是带着{\PN{彼得}}和{\PN{西庇太}}的两个儿子同去,就忧愁起来,极其难过,
\VS{38}便对他们说:「我心里甚是忧伤,几乎要死;你们在这里等候,和我一同警醒。」
\VS{39}他就稍往前走,俯伏在地,祷告说:「我父啊,倘若可行,求你叫这杯离开我。然而,不要照我的意思,只要照你的意思。」
\VS{40}来到门徒那里,见他们睡着了,就对{\PN{彼得}}说:「怎么样?你们不能同我警醒片时吗?
\VS{41}总要警醒祷告,免得入了迷惑。你们心灵固然愿意,肉体却软弱了。」
\VS{42}第二次又去祷告说:「我父啊,这杯若不能离开我,必要我喝,就愿你的意旨成全。」
\VS{43}又来,见他们睡着了,因为他们的眼睛困倦。
\VS{44}耶稣又离开他们去了。第三次祷告,说的话还是与先前一样。
\VS{45}于是来到门徒那里,对他们说:「现在你们仍然睡觉安歇吧\FTNT{}{{\FR 26:45: }吧:或译吗?}!时候到了,人子被卖在罪人手里了。
\VS{46}起来!我们走吧。看哪,卖我的人近了!」
\par }{\SH 耶稣被捕
\par }{\R (可14·43—50;路22·47—53;约18·3—12)
\par }{\PP \VS{47}说话之间,那十二个门徒里的{\PN{犹大}}来了,并有许多人带着刀棒,从祭司长和民间的长老那里与他同来。
\VS{48}那卖耶稣的给了他们一个暗号,说:「我与谁亲嘴,谁就是他。你们可以拿住他。」
\VS{49}{\PN{犹大}}随即到耶稣跟前,说:「请拉比安」,就与他亲嘴。
\VS{50}耶稣对他说:「朋友,你来要做的事,就做吧。」于是那些人上前,下手拿住耶稣。
\VS{51}有跟随耶稣的一个人伸手拔出刀来,将大祭司的仆人砍了一刀,削掉了他一个耳朵。
\VS{52}耶稣对他说:「收刀入鞘吧!凡动刀的,必死在刀下。
\VS{53}你想,我不能求我父现在为我差遣十二营多天使来吗?
\VS{54}若是这样,经上所说,事情必须如此的话怎么应验呢?」
\VS{55}当时,耶稣对众人说:「你们带着刀棒出来拿我,如同拿强盗吗?我天天坐在殿里教训人,你们并没有拿我。
\VS{56}但这一切的事成就了,为要应验先知书上的话。」当下,门徒都离开他,逃走了。
\par }{\SH 耶稣在公会里受审
\par }{\R (可14·53—65;路22·54—55,63—71;约18·13—14,19—24)
\par }{\PP \VS{57}拿耶稣的人把他带到大祭司{\PN{该亚法}}那里去;文士和长老已经在那里聚会。
\VS{58}{\PN{彼得}}远远地跟着耶稣,直到大祭司的院子,进到里面,就和差役同坐,要看这事到底怎样。
\VS{59}祭司长和全公会寻找假见证控告耶稣,要治死他。
\VS{60}虽有好些人来作假见证,总得不着实据。末后有两个人前来,说:
\VS{61}「这个人曾说:『我能拆毁 神的殿,三日内又建造起来。』」
\VS{62}大祭司就站起来,对耶稣说:「你什么都不回答吗?这些人作见证告你的是什么呢?」
\VS{63}耶稣却不言语。大祭司对他说:「我指着永生 神叫你起誓告诉我们,你是 神的儿子基督不是?」
\VS{64}耶稣对他说:「你说的是。然而,我告诉你们,后来你们要看见人子坐在那权能者的右边,驾着天上的云降临。」
\VS{65}大祭司就撕开衣服,说:「他说了僭妄的话,我们何必再用见证人呢?这僭妄的话,现在你们都听见了。
\VS{66}你们的意见如何?」他们回答说:「他是该死的。」
\VS{67}他们就吐唾沫在他脸上,用拳头打他,也有用手掌打他的,说:
\VS{68}「基督啊!你是先知,告诉我们打你的是谁?」
\par }{\SH 彼得三次不认主
\par }{\R (可14·66—72;路22·56—62;约18·15—18,25—27)
\par }{\PP \VS{69}{\PN{彼得}}在外面院子里坐着,有一个使女前来,说:「你素来也是同那{\PN{加利利}}人耶稣一伙的。」
\VS{70}{\PN{彼得}}在众人面前却不承认,说:「我不知道你说的是什么!」
\VS{71}既出去,到了门口,又有一个{\ADD{使女}}看见他,就对那里的人说:「这个人也是同{\PN{拿撒勒}}人耶稣一伙的。」
\VS{72}{\PN{彼得}}又不承认,并且起誓说:「我不认得那个人。」
\VS{73}过了不多的时候,旁边站着的人前来,对{\PN{彼得}}说:「你真是他们一党的,你的口音把你露出来了。」
\VS{74}{\PN{彼得}}就发咒起誓地说:「我不认得那个人。」立时,鸡就叫了。
\VS{75}{\PN{彼得}}想起耶稣所说的话:「鸡叫以先,你要三次不认我。」他就出去痛哭。

\par }\Chap{27}{\SH 耶稣被交给彼拉多
\par }{\R (可15·1;路23·1—2;约18·28—32)
\par }{\PP \VerseOne{1}到了早晨,众祭司长和民间的长老大家商议要治死耶稣,
\VS{2}就把他捆绑,解去,交给巡抚{\PN{彼拉多}}。
\par }{\SH 犹大的死
\par }{\R (徒1·18—19)
\par }{\PP \VS{3}这时候,卖耶稣的{\PN{犹大}}看见耶稣已经定了罪,就后悔,把那三十块钱拿回来给祭司长和长老,说:
\VS{4}「我卖了无辜之人的血是有罪了。」他们说:「那与我们有什么相干?你自己承当吧!」
\VS{5}{\PN{犹大}}就把那银钱丢在殿里,出去吊死了。
\VS{6}祭司长拾起银钱来,说:「这是血价,不可放在库里。」
\VS{7}他们商议,就用那银钱买了窑户的一块田,为要埋葬外乡人。
\VS{8}所以那块田直到今日还叫做「血田」。
\VS{9}这就应了先知{\PN{耶利米}}的话,说:「他们用那三十块钱,就是被估定之人的价钱,是{\PN{以色列}}人中所估定的,
\VS{10}买了窑户的一块田;这是照着主所吩咐我的。」
\par }{\SH 耶稣在彼拉多面前受审
\par }{\R (可15·2—5;路23·3—5;约18·33—38)
\par }{\PP \VS{11}耶稣站在巡抚面前;巡抚问他说:「你是{\PN{犹太}}人的王吗?」耶稣说:「你说的是。」
\VS{12}他被祭司长和长老控告的时候,什么都不回答。
\VS{13}{\PN{彼拉多}}就对他说:「他们作见证告你这么多的事,你没有听见吗?」
\VS{14}耶稣仍不回答,连一句话也不说,以致巡抚甚觉希奇。
\par }{\SH 耶稣被判死刑
\par }{\R (可15·6—15;路23·13—25;约18·39—19·16)
\par }{\PP \VS{15}巡抚有一个常例,每逢这节期,随众人所要的释放一个囚犯给他们。
\VS{16}当时有一个出名的囚犯叫{\PN{巴拉巴}}。
\VS{17}众人聚集的时候,{\PN{彼拉多}}就对他们说:「你们要我释放哪一个给你们?是{\PN{巴拉巴}}呢?是称为基督的耶稣呢?」
\VS{18}巡抚原知道他们是因为嫉妒才把他解了来。
\VS{19}正坐堂的时候,他的夫人打发人来说:「这义人的事,你一点不可管,因为我今天在梦中为他受了许多的苦。」
\VS{20}祭司长和长老挑唆众人,求释放{\PN{巴拉巴}},除灭耶稣。
\VS{21}巡抚对众人说:「这两个人,你们要我释放哪一个给你们呢?」他们说:「{\PN{巴拉巴}}。」
\VS{22}{\PN{彼拉多}}说:「这样,那称为基督的耶稣我怎么办他呢?」他们都说:「把他钉十字架!」
\VS{23}巡抚说:「为什么呢?他做了什么恶事呢?」他们便极力地喊着说:「把他钉十字架!」
\VS{24}{\PN{彼拉多}}见说也无济于事,反要生乱,就拿水在众人面前洗手,说:「流这义人的血,罪不在我,你们承当吧。」
\VS{25}众人都回答说:「他的血归到我们和我们的子孙身上。」
\VS{26}于是{\PN{彼拉多}}释放{\PN{巴拉巴}}给他们,把耶稣鞭打了,交给人钉十字架。
\par }{\SH 兵丁戏弄耶稣
\par }{\R (可15·16—20;约19·2—3)
\par }{\PP \VS{27}巡抚的兵就把耶稣带进衙门,叫全营的兵都聚集在他那里。
\VS{28}他们给他脱了衣服,穿上一件朱红色袍子,
\VS{29}用荆棘编做冠冕,戴在他头上,拿一根苇子放在他右手里,跪在他面前,戏弄他,说:「恭喜,{\PN{犹太}}人的王啊!」
\VS{30}又吐唾沫在他脸上,拿苇子打他的头。
\VS{31}戏弄完了,就给他脱了袍子,仍穿上他自己的衣服,带他出去,要钉十字架。
\par }{\SH 耶稣被钉十字架
\par }{\R (可15·21—32;路23·26—43;约19·17—27)
\par }{\PP \VS{32}他们出来的时候,遇见一个{\PN{古利奈}}人,名叫{\PN{西门}},就勉强他同去,好背着耶稣的十字架。
\VS{33}到了一个地方名叫{\PN{各各他}},意思就是「髑髅地」。
\VS{34}兵丁拿苦胆调和的酒给耶稣喝;他尝了,就不肯喝。
\VS{35}他们既将他钉在十字架上,就拈阄分他的衣服,
\VS{36}又坐在那里看守他。
\VS{37}在他头以上安一个牌子,写着他的罪状,说:「这是{\PN{犹太}}人的王耶稣。」
\VS{38}当时,有两个强盗和他同钉十字架,一个在右边,一个在左边。
\VS{39}从那里经过的人讥诮他,摇着头,说:
\VS{40}「你这拆毁{\ADD{圣}}殿、三日又建造起来的,可以救自己吧!你如果是 神的儿子,就从十字架上下来吧!」
\VS{41}祭司长和文士并长老也是这样戏弄他,说:
\VS{42}「他救了别人,不能救自己。他是{\PN{以色列}}的王,现在可以从十字架上下来,我们就信他。
\VS{43}他倚靠 神, 神若喜悦他,现在可以救他;因为他曾说:『我是 神的儿子。』」
\VS{44}那和他同钉的强盗也是这样地讥诮他。
\par }{\SH 耶稣的死
\par }{\R (可15·33—41;路23·44—49;约19·28—30)
\par }{\PP \VS{45}从午正到申初,遍地都黑暗了。
\VS{46}约在申初,耶稣大声喊着说:「以利!以利!拉马撒巴各大尼?」就是说:「我的 神!我的 神!为什么离弃我?」
\VS{47}站在那里的人,有的听见就说:「这个人呼叫{\PN{以利亚}}呢!」
\VS{48}内中有一个人赶紧跑去,拿海绒蘸满了醋,绑在苇子上,送给他喝。
\VS{49}其余的人说:「且等着,看{\PN{以利亚}}来救他不来。」
\VS{50}耶稣又大声喊叫,气就断了。
\VS{51}忽然,殿里的幔子从上到下裂为两半,地也震动,磐石也崩裂,
\VS{52}坟墓也开了,已睡圣徒的身体多有起来的。
\VS{53}到耶稣复活以后,他们从坟墓里出来,进了圣城,向许多人显现。
\VS{54}百夫长和一同看守耶稣的人看见地震并所经历的事,就极其害怕,说:「这真是 神的儿子了!」
\VS{55}有好些妇女在那里,远远地观看;她们是从{\PN{加利利}}跟随耶稣来服事他的。
\VS{56}内中有{\PN{抹大拉}}的{\PN{马利亚}},又有{\PN{雅各}}和{\PN{约西}}的母亲{\PN{马利亚}},并有{\PN{西庇太}}两个儿子的母亲。
\par }{\SH 耶稣的安葬
\par }{\R (可15·42—47;路23·50—56;约19·38—42)
\par }{\PP \VS{57}到了晚上,有一个财主,名叫{\PN{约瑟}},是{\PN{亚利马太}}来的,他也是耶稣的门徒。
\VS{58}这人去见{\PN{彼拉多}},求耶稣的身体;{\PN{彼拉多}}就吩咐给他。
\VS{59}{\PN{约瑟}}取了身体,用干净细麻布裹好,
\VS{60}安放在自己的新坟墓里,就是他凿在磐石里的。他又把大石头滚到墓门口,就去了。
\VS{61}有{\PN{抹大拉}}的{\PN{马利亚}}和那个{\PN{马利亚}}在那里,对着坟墓坐着。
\par }{\SH 封石妥守
\par }{\PP \VS{62}次日,就是预备日的第二天,祭司长和法利赛人聚集来见{\PN{彼拉多}},说:
\VS{63}「大人,我们记得那诱惑人的还活着的时候曾说:『三日后我要复活。』
\VS{64}因此,请吩咐人将坟墓把守妥当,直到第三日,恐怕他的门徒来,把他偷了去,就告诉百姓说:『他从死里复活了。』这样,那后来的迷惑比先前的更利害了!」
\VS{65}{\PN{彼拉多}}说:「你们有看守的兵,去吧!尽你们所能的把守妥当。」
\VS{66}他们就带着看守的兵同去,封了石头,将坟墓把守妥当。

\par }\Chap{28}{\SH 耶稣复活
\par }{\R (可16·1—10;路24·1—12;约20·1—10)
\par }{\PP \VerseOne{1}安息日将尽,七日的头一日,天快亮的时候,{\PN{抹大拉}}的{\PN{马利亚}}和那个{\PN{马利亚}}来看坟墓。
\VS{2}忽然,地大震动;因为有主的使者从天上下来,把石头滚开,坐在上面。
\VS{3}他的相貌如同闪电,衣服洁白如雪。
\VS{4}看守的人就因他吓得浑身乱战,甚至和死人一样。
\VS{5}天使对妇女说:「不要害怕!我知道你们是寻找那钉十字架的耶稣。
\VS{6}他不在这里,照他所说的,已经复活了。你们来看安放主的地方。
\VS{7}快去告诉他的门徒,说他从死里复活了,并且在你们以先往{\PN{加利利}}去,在那里你们要见他。看哪,我已经告诉你们了。」
\VS{8}妇女们就急忙离开坟墓,又害怕,又大大地欢喜,跑去要报给他的门徒。
\VS{9}忽然,耶稣遇见她们,说:「愿你们平安!」她们就上前抱住他的脚拜他。
\VS{10}耶稣对她们说:「不要害怕!你们去告诉我的弟兄,叫他们往{\PN{加利利}}去,在那里必见我。」
\par }{\SH 看守的兵报告祭司长
\par }{\PP \VS{11}她们去的时候,看守的兵有几个进城去,将所经历的事都报给祭司长。
\VS{12}祭司长和长老聚集商议,就拿许多银钱给兵丁,说:
\VS{13}「你们要这样说:『夜间我们睡觉的时候,他的门徒来,把他偷去了。』
\VS{14}倘若这话被巡抚听见,有我们劝他,保你们无事。」
\VS{15}兵丁受了银钱,就照所嘱咐他们的去行。这话就传说在{\PN{犹太}}人中间,直到今日。
\par }{\SH 门徒奉差遣
\par }{\R (可16·14—18;路24·36—49;约20·19—23;徒1·6—8)
\par }{\PP \VS{16}十一个门徒往{\PN{加利利}}去,到了耶稣约定的山上。
\VS{17}他们见了耶稣就拜他,然而还有人疑惑。
\VS{18}耶稣进前来,对他们说:「天上地下所有的权柄都赐给我了。
\VS{19}所以,你们要去,使万民作我的门徒,奉父、子、圣灵的名给他们施洗\FTNT{}{{\FR 28:19: }或译:给他们施洗,归于父、子、圣灵的名}。
\VS{20}凡我所吩咐你们的,都教训他们遵守,我就常与你们同在,直到世界的末了。」
\par }