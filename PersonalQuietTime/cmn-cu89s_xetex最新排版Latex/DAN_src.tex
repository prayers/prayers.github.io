\NormalFont\ShortTitle{但以理书}
{\MT 但以理书

\par }\ChapOne{1}{\SH 但以理和他的朋友们
\par }{\R (1·1—6·28)
\par }{\SH 但以理在尼布甲尼撒的宫廷里
\par }{\PP \VerseOne{1}{\PN{犹大}}王{\PN{约雅敬}}在位第三年,{\PN{巴比伦}}王{\PN{尼布甲尼撒}}来到{\PN{耶路撒冷}},将城围困。
\VS{2}主将{\PN{犹大}}王{\PN{约雅敬}},并 神殿中器皿的几分交付他手。他就把这器皿带到{\PN{示拿}}地,收入他神的庙里,放在他神的库中。
\par }{\PP \VS{3}王吩咐太监长{\PN{亚施毗拿}},从{\PN{以色列}}人的宗室和贵胄中带进几个人来,
\VS{4}就是年少没有残疾、相貌俊美、通达各样学问、知识聪明俱备、足能侍立在王宫里的,要教他们{\PN{迦勒底}}的文字言语。
\VS{5}王派定将自己所用的膳和所饮的酒,每日赐他们一分,养他们三年。满了三年,好叫他们在王面前侍立。
\VS{6}他们中间有{\PN{犹大}}族的人:{\PN{但以理}}、{\PN{哈拿尼雅}}、{\PN{米沙利}}、{\PN{亚撒利雅}}。
\VS{7}太监长给他们起名:称{\PN{但以理}}为{\PN{伯提沙撒}},称{\PN{哈拿尼雅}}为{\PN{沙得拉}},称{\PN{米沙利}}为{\PN{米煞}},称{\PN{亚撒利雅}}为{\PN{亚伯尼歌}}。
\par }{\PP \VS{8}{\PN{但以理}}却立志不以王的膳和王所饮的酒玷污自己,所以求太监长容他不玷污自己。
\VS{9}神使{\PN{但以理}}在太监长眼前蒙恩惠,受怜悯。
\VS{10}太监长对{\PN{但以理}}说:「我惧怕我主我王,他已经派定你们的饮食,倘若他见你们的面貌比你们同岁的少年人肌瘦,怎么好呢?这样,你们就使我的头在王那里难保。」
\VS{11}{\PN{但以理}}对太监长所派管理{\PN{但以理}}、{\PN{哈拿尼雅}}、{\PN{米沙利}}、{\PN{亚撒利雅}}的委办说:
\VS{12}「求你试试仆人们十天,给我们素菜吃,白水喝,
\VS{13}然后看看我们的面貌和用王膳那少年人的面貌,就照你所看的待仆人吧!」
\par }{\PP \VS{14}委办便允准他们这件事,试看他们十天。
\VS{15}过了十天,见他们的面貌比用王膳的一切少年人更加俊美肥胖。
\VS{16}于是委办撤去派他们用的膳,饮的酒,给他们素菜吃。
\par }{\PP \VS{17}这四个少年人, 神在各样文字学问\FTNT{}{{\FR 1:17: }学问:原文是智慧}上赐给他们聪明知识;{\PN{但以理}}又明白各样的异象和梦兆。
\par }{\PP \VS{18}{\PN{尼布甲尼撒}}王预定带进少年人来的日期满了,太监长就把他们带到王面前。
\VS{19}王与他们谈论,见少年人中无一人能比{\PN{但以理}}、{\PN{哈拿尼雅}}、{\PN{米沙利}}、{\PN{亚撒利雅}},所以留他们在王面前侍立。
\VS{20}王考问他们一切事,就见他们的智慧聪明比通国的术士和用法术的胜过十倍。
\VS{21}到{\PN{塞鲁士}}王元年,{\PN{但以理}}还在。

\par }\Chap{2}{\PP \VerseOne{1}{\PN{尼布甲尼撒}}在位第二年,他做了梦,心里烦乱,不能睡觉。
\VS{2}王吩咐人将术士、用法术的、行邪术的,和{\PN{迦勒底}}人召来,要他们将王的梦告诉王,他们就来站在王前。
\VS{3}王对他们说:「我做了一梦,心里烦乱,要知道这是什么梦。」
\VS{4}{\PN{迦勒底}}人用{\PN{亚兰}}的言语对王说:「愿王万岁!请将那梦告诉仆人,仆人就可以讲解。」
\VS{5}王回答{\PN{迦勒底}}人说:「梦我已经忘了\FTNT{}{{\FR 2:5: }或译:我已定命;八节同},你们若不将梦和梦的讲解告诉我,就必被凌迟,你们的房屋必成为粪堆;
\VS{6}你们若将梦和梦的讲解告诉我,就必从我这里得赠品和赏赐,并大尊荣。现在你们要将梦和梦的讲解告诉我。」
\VS{7}他们第二次对王说:「请王将梦告诉仆人,仆人就可以讲解。」
\VS{8}王回答说:「我准知道你们是故意迟延,因为你们知道那梦我已经忘了。
\VS{9}你们若不将梦告诉我,只有一法待你们;因为你们预备了谎言乱语向我说,要等候时势改变。现在你们要将梦告诉我,因我知道你们能将梦的讲解告诉我。」
\par }{\PP \VS{10}{\PN{迦勒底}}人在王面前回答说:「世上没有人能将王所问的事说出来;因为没有君王、大臣、掌权的向术士,或用法术的,或{\PN{迦勒底}}人问过这样的事。
\VS{11}王所问的事甚难。除了不与世人同居的神明,没有人在王面前能说出来。」
\par }{\PP \VS{12}因此,王气忿忿地大发烈怒,吩咐灭绝{\PN{巴比伦}}所有的哲士。
\VS{13}于是命令发出,哲士将要见杀,人就寻找{\PN{但以理}}和他的同伴,要杀他们。
\par }{\SH  神把梦的意思指示但以理
\par }{\PP \VS{14}王的护卫长{\PN{亚略}}出来,要杀{\PN{巴比伦}}的哲士,{\PN{但以理}}就用婉言回答他,
\VS{15}向王的护卫长{\PN{亚略}}说:「王的命令为何这样紧急呢?」{\PN{亚略}}就将情节告诉{\PN{但以理}}。
\VS{16}{\PN{但以理}}遂进去求王宽限,就可以将{\ADD{梦的}}讲解告诉王。
\par }{\PP \VS{17}{\PN{但以理}}回到他的居所,将这事告诉他的同伴{\PN{哈拿尼雅}}、{\PN{米沙利}}、{\PN{亚撒利雅}},
\VS{18}要他们祈求天上的 神施怜悯,将这奥秘的事指明,免得{\PN{但以理}}和他的同伴与{\PN{巴比伦}}其余的哲士一同灭亡。
\VS{19}这奥秘的事就在夜间异象中给{\PN{但以理}}显明,{\PN{但以理}}便称颂天上的 神。
\par }{\PP \VS{20}{\PN{但以理}}说:「 神的名是应当称颂的!从亘古直到永远,因为智慧能力都属乎他。
\VS{21}他改变时候、日期,废王,立王,将智慧赐与智慧人,将知识赐与聪明人。
\VS{22}他显明深奥隐秘的事,知道暗中所有的,光明也与他同居。
\VS{23}我列祖的 神啊,我感谢你,赞美你;因你将智慧才能赐给我,允准我们所求的,把王的事给我们指明。」
\par }{\SH 但以理为王解梦
\par }{\PP \VS{24}于是,{\PN{但以理}}进去见{\PN{亚略}},就是王所派灭绝{\PN{巴比伦}}哲士的,对他说:「不要灭绝{\PN{巴比伦}}的哲士,求你领我到王面前,我要将{\ADD{梦的}}讲解告诉王。」
\par }{\PP \VS{25}{\PN{亚略}}就急忙将{\PN{但以理}}领到王面前,对王说:「我在被掳的{\PN{犹大}}人中遇见一人,他能将{\ADD{梦的}}讲解告诉王。」
\VS{26}王问称为{\PN{伯提沙撒}}的{\PN{但以理}}说:「你能将我所做的梦和梦的讲解告诉我吗?」
\VS{27}{\PN{但以理}}在王面前回答说:「王所问的那奥秘事,哲士、用法术的、术士、观兆的都不能告诉王,
\VS{28}只有一位在天上的 神能显明奥秘的事。他已将日后必有的事指示{\PN{尼布甲尼撒}}王。你的梦和你在床上脑中的异象是这样:
\VS{29}王啊,你在床上想到后来的事,那显明奥秘事的主把将来必有的事指示你。
\VS{30}至于那奥秘的事显明给我,并非因我的智慧胜过一切活人,乃为使王知道{\ADD{梦的}}讲解和心里的思念。
\par }{\PP \VS{31}「王啊,你梦见一个大像,这像甚高,极其光耀,站在你面前,形状甚是可怕。
\VS{32}这像的头是精金的,胸膛和膀臂是银的,肚腹和腰是铜的,
\VS{33}腿是铁的,脚是半铁半泥的。
\VS{34}你观看,见有一块非人手凿出来的石头打在这像半铁半泥的脚上,把脚砸碎;
\VS{35}于是金、银、铜、铁、泥都一同砸得粉碎,成如夏天禾场上的糠秕,被风吹散,无处可寻。打碎这像的石头变成一座大山,充满天下。
\par }{\PP \VS{36}「这就是那梦;我们在王面前要讲解那梦。
\VS{37}王啊,你是诸王之王。天上的 神已将国度、权柄、能力、尊荣都赐给你。
\VS{38}凡世人所住之地的走兽,并天空的飞鸟,他都交付你手,使你掌管这一切。你就是那金头。
\VS{39}在你以后必另兴一国,不及于你;又有第三国,就是铜的,必掌管天下。
\VS{40}第四国,必坚壮如铁,铁能打碎克制百物,又能压碎一切,那国也必打碎压制{\ADD{列国}}。
\VS{41}你既见像的脚和脚指头,一半是窑匠的泥,一半是铁,那国将来也必分开。你既见铁与泥搀杂,那国也必有铁的力量。
\VS{42}那脚指头,既是半铁半泥,那国也必半强半弱。
\VS{43}你既见铁与泥搀杂,那国民也必与各种人搀杂,却不能彼此相合,正如铁与泥不能相合一样。
\VS{44}当那列王在位的时候,天上的 神必另立一国,永不败坏,也不归别国的人,却要打碎灭绝那一切国,这国必存到永远。
\VS{45}你既看见非人手凿出来的一块石头从山而出,打碎金、银、铜、铁、泥,那就是至大的 神把后来必有的事给王指明。这梦准是这样,这讲解也是确实的。」
\par }{\SH 王奖赏但以理
\par }{\PP \VS{46}当时,{\PN{尼布甲尼撒}}王俯伏在地,向{\PN{但以理}}下拜,并且吩咐人给他奉上供物和香品。
\VS{47}王对{\PN{但以理}}说:「你既能显明这奥秘的事,你们的 神诚然是万神之神、万王之主,又是显明奥秘事的。」
\VS{48}于是王高抬{\PN{但以理}},赏赐他许多上等礼物,派他管理{\PN{巴比伦}}全省,又立他为总理,掌管{\PN{巴比伦}}的一切哲士。
\VS{49}{\PN{但以理}}求王,王就派{\PN{沙得拉}}、{\PN{米煞}}、{\PN{亚伯尼歌}}管理{\PN{巴比伦}}省的事务,只是{\PN{但以理}}常在朝中侍立。

\par }\Chap{3}{\SH 尼布甲尼撒下令拜金像
\par }{\PP \VerseOne{1}{\PN{尼布甲尼撒}}王造了一个金像,高六十肘,宽六肘,立在{\PN{巴比伦}}省{\PN{杜拉}}平原。
\VS{2}{\PN{尼布甲尼撒}}王差人将总督、钦差、巡抚、臬司、藩司、谋士、法官,和各省的官员都召了来,为{\PN{尼布甲尼撒}}王所立的像行开光之礼。
\VS{3}于是总督、钦差、巡抚、臬司、藩司、谋士、法官,和各省的官员都聚集了来,要为{\PN{尼布甲尼撒}}王所立的像行开光之礼,就站在{\PN{尼布甲尼撒}}所立的像前。
\VS{4}那时传令的大声呼叫说:「各方、各国、各族\FTNT{}{{\FR 3:4: }原文是舌;下同}的人哪,有令传与你们:
\VS{5}你们一听见角、笛、琵琶、琴、瑟、笙,和各样乐器的声音,就当俯伏敬拜{\PN{尼布甲尼撒}}王所立的金像。
\VS{6}凡不俯伏敬拜的,必立时扔在烈火的窑中。」
\VS{7}因此各方、各国、各族的人民一听见角、笛、琵琶、琴、瑟,和各样乐器的声音,就都俯伏敬拜{\PN{尼布甲尼撒}}王所立的金像。
\par }{\SH 但以理的三个朋友被控违令
\par }{\PP \VS{8}那时,有几个{\PN{迦勒底}}人进前来控告{\PN{犹大}}人。
\VS{9}他们对{\PN{尼布甲尼撒}}王说:「愿王万岁!
\VS{10}王啊,你曾降旨说,凡听见角、笛、琵琶、琴、瑟、笙,和各样乐器声音的都当俯伏敬拜金像。
\VS{11}凡不俯伏敬拜的,必扔在烈火的窑中。
\VS{12}现在有几个{\PN{犹大}}人,就是王所派管理{\PN{巴比伦}}省事务的{\PN{沙得拉}}、{\PN{米煞}}、{\PN{亚伯尼歌}};王啊,这些人不理你,不事奉你的神,也不敬拜你所立的金像。」
\par }{\PP \VS{13}当时,{\PN{尼布甲尼撒}}冲冲大怒,吩咐人把{\PN{沙得拉}}、{\PN{米煞}}、{\PN{亚伯尼歌}}带过来,他们就把那些人带到王面前。
\VS{14}{\PN{尼布甲尼撒}}问他们说:「{\PN{沙得拉}}、{\PN{米煞}}、{\PN{亚伯尼歌}},你们不事奉我的神,也不敬拜我所立的金像,是故意的吗?
\VS{15}你们再听见角、笛、琵琶、琴、瑟、笙,和各样乐器的声音,若俯伏敬拜我所造的像,却还可以;若不敬拜,必立时扔在烈火的窑中,有何神能救你们脱离我手呢?」
\par }{\PP \VS{16}{\PN{沙得拉}}、{\PN{米煞}}、{\PN{亚伯尼歌}}对王说:「{\PN{尼布甲尼撒}}啊,这件事我们不必回答你,
\VS{17}即便如此,我们所事奉的 神能将我们从烈火的窑中救出来。王啊,他也必救我们脱离你的手;
\VS{18}即或不然,王啊,你当知道我们决不事奉你的神,也不敬拜你所立的金像。」
\par }{\SH 但以理的三个朋友被扔进火窑
\par }{\PP \VS{19}当时,{\PN{尼布甲尼撒}}怒气填胸,向{\PN{沙得拉}}、{\PN{米煞}}、{\PN{亚伯尼歌}}变了脸色,吩咐人把窑烧热,比寻常更加七倍;
\VS{20}又吩咐他军中的几个壮士,将{\PN{沙得拉}}、{\PN{米煞}}、{\PN{亚伯尼歌}}捆起来,扔在烈火的窑中。
\VS{21}这三人穿着裤子、内袍、外衣,和{\ADD{别的}}衣服,被捆起来扔在烈火的窑中。
\VS{22}因为王命紧急,窑又甚热,那抬{\PN{沙得拉}}、{\PN{米煞}}、{\PN{亚伯尼歌}}的人都被火焰烧死。
\VS{23}{\PN{沙得拉}}、{\PN{米煞}}、{\PN{亚伯尼歌}}这三个人都被捆着落在烈火的窑中。
\par }{\PP \VS{24}那时,{\PN{尼布甲尼撒}}王惊奇,急忙起来,对谋士说:「我们捆起来扔在火里的不是三个人吗?」他们回答王说:「王啊,是。」
\VS{25}王说:「看哪,我见有四个人,并没有捆绑,在火中游行,也没有受伤;那第四个的相貌好像神子。」
\par }{\SH 三人获释且得高升
\par }{\PP \VS{26}于是,{\PN{尼布甲尼撒}}就近烈火窑门,说:「至高 神的仆人{\PN{沙得拉}}、{\PN{米煞}}、{\PN{亚伯尼歌}}出来,上这里来吧!」{\PN{沙得拉}}、{\PN{米煞}}、{\PN{亚伯尼歌}}就从火中出来了。
\VS{27}那些总督、钦差、巡抚,和王的谋士一同聚集看这三个人,见火无力伤他们的身体,头发也没有烧焦,衣裳也没有变色,并没有火燎的气味。
\VS{28}{\PN{尼布甲尼撒}}说:「{\PN{沙得拉}}、{\PN{米煞}}、{\PN{亚伯尼歌}}的 神是应当称颂的!他差遣使者救护倚靠他的仆人,他们不遵王命,舍去己身,在他们 神以外不肯事奉敬拜别神。
\VS{29}现在我降旨,无论何方、何国、何族的人,谤
{\PN{沙得拉}}、{\PN{米煞}}、{\PN{亚伯尼歌}}之 神的,必被凌迟,他的房屋必成粪堆,因为没有别神能这样施行拯救。」
\VS{30}那时王在{\PN{巴比伦}}省,高升了{\PN{沙得拉}}、{\PN{米煞}}、{\PN{亚伯尼歌}}。

\par }\Chap{4}{\SH 尼布甲尼撒的第二个梦
\par }{\PP \VerseOne{1}{\PN{尼布甲尼撒}}王晓谕住在全地各方、各国、各族的人说:「愿你们大享平安!
\VS{2}我乐意将至高的 神向我所行的神迹奇事宣扬出来。
\par }{\Q \VS{3}他的神迹何其大!
\par }{\Q 他的奇事何其盛!
\par }{\Q 他的国是永远的;
\par }{\Q 他的权柄存到万代!
\par }{\PP \VS{4}「我—{\PN{尼布甲尼撒}}安居在宫中,平顺在殿内。
\VS{5}我做了一梦,使我惧怕。我在床上的思念,并脑中的异象,使我惊惶。
\VS{6}所以我降旨召{\PN{巴比伦}}的一切哲士到我面前,叫他们把梦的讲解告诉我。
\VS{7}于是那些术士、用法术的、{\PN{迦勒底}}人、观兆的都进来,我将那梦告诉了他们,他们却不能把梦的讲解告诉我。
\VS{8}末后那照我神的名,称为{\PN{伯提沙撒}}的{\PN{但以理}}来到我面前,他里头有圣神的灵,我将梦告诉他{\ADD{说}}:
\VS{9}『术士的领袖{\PN{伯提沙撒}}啊,因我知道你里头有圣神的灵,什么奥秘的事都不能使你为难。现在要把我梦中所见的异象和梦的讲解告诉我。』
\par }{\PP \VS{10}「我在床上脑中的异象是这样:我看见地当中有一棵树,极其高大。
\VS{11}那树渐长,而且坚固,高得顶天,从地极都能看见,
\VS{12}叶子华美,果子甚多,可作众生的食物;田野的走兽卧在荫下,天空的飞鸟宿在枝上;凡有血气的都从这树得食。
\par }{\PP \VS{13}「我在床上脑中的异象,见有一位守望的圣者从天而降。
\VS{14}大声呼叫说:『伐倒这树!砍下枝子!摇掉叶子!抛散果子!使走兽离开树下,飞鸟躲开树枝。
\VS{15}树丕却要留在地内,用铁圈和铜圈箍住,在田野的青草中让天露滴湿,使他与地上的兽一同吃草,
\VS{16}使他的心改变,不如人心,给他一个兽心,使他经过七期\FTNT{}{{\FR 4:16: }期:或译年;本章同}。
\VS{17}这是守望者所发的命,圣者所出的令,好叫世人知道至高者在人的国中掌权,要将国赐与谁就赐与谁,{\ADD{或}}立极卑微的人执掌国权。』
\par }{\PP \VS{18}「这是我—{\PN{尼布甲尼撒}}王所做的梦。{\PN{伯提沙撒}}啊,你要说明这{\ADD{梦的}}讲解;因为我国中的一切哲士都不能将{\ADD{梦}}的讲解告诉我,惟独你能,因你里头有圣神的灵。」
\par }{\SH 但以理解梦
\par }{\PP \VS{19}于是称为{\PN{伯提沙撒}}的{\PN{但以理}}惊讶片时,心意惊惶。王说:「{\PN{伯提沙撒}}啊,不要因梦和梦的讲解惊惶。」{\PN{伯提沙撒}}回答说:「我主啊,愿这梦归与恨恶你的人,讲解归与你的敌人。
\VS{20}你所见的树渐长,而且坚固,高得顶天,从地极都能看见;
\VS{21}叶子华美,果子甚多,可作众生的食物;田野的走兽住在其下;天空的飞鸟宿在枝上。
\par }{\PP \VS{22}「王啊,这渐长又坚固的树就是你。你的威势渐长及天,你的权柄管到地极。
\VS{23}王既看见一位守望的圣者从天而降,说:『将这树砍伐毁坏,树丕却要留在地内,用铁圈和铜圈箍住;在田野的青草中,让天露滴湿,使他与地上的兽一同吃草,直到经过七期。』
\par }{\PP \VS{24}「王啊,讲解就是这样:临到我主我王的事是出于至高者的命。
\VS{25}你必被赶出离开世人,与野地的兽同居,吃草如牛,被天露滴湿,且要经过七期。等你知道至高者在人的国中掌权,要将国赐与谁就赐与谁。
\VS{26}守望者既吩咐存留树丕,等你知道诸天掌权,以后你的国必定归你。
\VS{27}王啊,求你悦纳我的谏言,以施行公义断绝罪过,以怜悯穷人除掉罪孽,或者你的平安可以延长。」
\par }{\PP \VS{28}这事都临到{\PN{尼布甲尼撒}}王。
\VS{29}过了十二个月,他游行在{\PN{巴比伦}}王宫里\FTNT{}{{\FR 4:29: }原文是上}。
\VS{30}他说:「这大{\PN{巴比伦}}不是我用大能大力建为京都,要显我威严的荣耀吗?」
\VS{31}这话在王口中尚未说完,有声音从天降下,{\ADD{说}}:「{\PN{尼布甲尼撒}}王啊,有话对你说,你的国位离开你了。
\VS{32}你必被赶出离开世人,与野地的兽同居,吃草如牛,且要经过七期。等你知道至高者在人的国中掌权,要将国赐与谁就赐与谁。」
\VS{33}当时这话就应验在{\PN{尼布甲尼撒}}的身上,他被赶出离开世人,吃草如牛,身被天露滴湿,头发长长,好像鹰{\ADD{毛}};指甲长长,如同鸟{\ADD{爪}}。
\par }{\SH 尼布甲尼撒颂赞 神
\par }{\PP \VS{34}日子满足,我—{\PN{尼布甲尼撒}}举目望天,我的聪明复归于我,我便称颂至高者,赞美尊敬活到永远的 {\ADD{神}}。
\par }{\Q 他的权柄是永有的;
\par }{\Q 他的国存到万代。
\par }{\Q \VS{35}世上所有的居民都算为虚无;
\par }{\Q 在天上的万军和世上的居民中,
\par }{\Q 他都凭自己的意旨行事。
\par }{\Q 无人能拦住他手,
\par }{\Q 或问他说,你做什么呢?
\par }{\PP \VS{36}那时,我的聪明复归于我,为我国的荣耀、威严,和光耀也都复归于我;并且我的谋士和大臣也来朝见我。我又得坚立在国位上,至大的权柄加增于我。
\VS{37}现在我—{\PN{尼布甲尼撒}}赞美、尊崇、恭敬天上的王;因为他所做的全都诚实,他所行的也都公平。那行动骄傲的,他能降为卑。

\par }\Chap{5}{\SH 伯沙撒王的宴会
\par }{\PP \VerseOne{1}{\PN{伯沙撒}}王为他的一千大臣设摆盛筵,与这一千人对面饮酒。
\VS{2}{\PN{伯沙撒}}欢饮之间,吩咐人将他父\FTNT{}{{\FR 5:2: }或译:祖;下同}{\PN{尼布甲尼撒}}从{\PN{耶路撒冷}}殿中所掠的金银器皿拿来,王与大臣、皇后、妃嫔好用这器皿饮酒。
\VS{3}于是他们把{\PN{耶路撒冷}} 神殿{\ADD{库}}房中所掠的金器皿拿来,王和大臣、皇后、妃嫔就用这器皿饮酒。
\VS{4}他们饮酒,赞美金、银、铜、铁、木、石所造的神。
\par }{\PP \VS{5}当时,{\ADD{忽}}有人的指头显出,在王宫与灯台相对的粉墙上写字。王看见写字的指头
\VS{6}就变了脸色,心意惊惶,腰骨{\ADD{好像}}脱节,双膝彼此相碰,
\VS{7}大声吩咐将用法术的和{\PN{迦勒底}}人并观兆的领进来,对{\PN{巴比伦}}的哲士说,谁能读这文字,把讲解告诉我,他必身穿紫袍,项带金链,在我国中位列第三。
\VS{8}于是王的一切哲士都进来,却不能读那文字,也不能把讲解告诉王。
\VS{9}{\PN{伯沙撒}}王就甚惊惶,脸色改变,他的大臣也都惊奇。
\par }{\PP \VS{10}太后\FTNT{}{{\FR 5:10: }或译:皇后;下同}因王和他大臣所说的话,就进入宴宫,说:「愿王万岁!你心意不要惊惶,脸面不要变色。
\VS{11}在你国中有一人,他里头有圣神的灵,你父在世的日子,这人心中光明,又有聪明智慧,好像神的智慧。你父{\PN{尼布甲尼撒}}王,就是王的父,立他为术士、用法术的,和{\PN{迦勒底}}人,并观兆的领袖。
\VS{12}在他里头有美好的灵性,又有知识聪明,能圆梦,释谜语,解疑惑。这人名叫{\PN{但以理}},{\PN{尼布甲尼撒}}王又称他为{\PN{伯提沙撒}},现在可以召他来,他必解明这意思。」
\par }{\SH 但以理解释墙上的字
\par }{\PP \VS{13}{\PN{但以理}}就被领到王前。王问{\PN{但以理}}说:「你是被掳之{\PN{犹大}}人中的{\PN{但以理}}吗?就是我父王从{\PN{犹大}}掳来的吗?
\VS{14}我听说你里头有神的灵,心中光明,又有聪明和美好的智慧。
\VS{15}现在哲士和用法术的都领到我面前,为叫他们读这文字,把讲解告诉我,无奈他们都不能把讲解说出来。
\VS{16}我听说你善于讲解,能解疑惑;现在你若能读这文字,把讲解告诉我,就必身穿紫袍,项戴金链,在我国中位列第三。」
\par }{\PP \VS{17}{\PN{但以理}}在王面前回答说:「你的赠品可以归你自己,你的赏赐可以归给别人;我却要为王读这文字,把讲解告诉王。
\VS{18}王啊,至高的 神曾将国位、大权、荣耀、威严赐与你父{\PN{尼布甲尼撒}};
\VS{19}因 神所赐他的大权,各方、各国、各族的人都在他面前战兢恐惧。他可以随意生杀,随意升降。
\VS{20}但他心高气傲,灵也刚愎,甚至行事狂傲,就被革去王位,夺去荣耀。
\VS{21}他被赶出离开世人,他的心变如兽心,与野驴同居,吃草如牛,身被天露滴湿,等他知道至高的 神在人的国中掌权,凭自己的意旨立人治国。
\VS{22}{\PN{伯沙撒}}啊,你是他的儿子\FTNT{}{{\FR 5:22: }或译:孙子},你虽知道这一切,你心仍不自卑,
\VS{23}竟向天上的主自高,使人将他殿中的器皿拿到你面前,你和大臣、皇后、妃嫔用这器皿饮酒。你又赞美那不能看、不能听、无知无识、金、银、铜、铁、木、石所造的神,却没有将荣耀归与那手中有你气息,管理你一切行动的 神。
\VS{24}因此从 神那里显出指头来写这文字。
\par }{\PP \VS{25}「所写的文字是:『弥尼,弥尼,提客勒,乌法珥新。』
\VS{26}讲解是这样:弥尼,就是 神已经数算你国的年日到此完毕。
\VS{27}提客勒,就是你被称在天平里,显出你的亏欠。
\VS{28}毗勒斯\FTNT{}{{\FR 5:28: }与乌法珥新同义},就是你的国分裂,归与{\PN{米底亚}}人和{\PN{波斯}}人。」
\VS{29}{\PN{伯沙撒}}下令,人就把紫袍给{\PN{但以理}}穿上,把金链给他戴在颈项上,又传令使他在国中位列第三。
\VS{30}当夜,{\PN{迦勒底}}王{\PN{伯沙撒}}被杀。
\VS{31}{\PN{米底亚}}人{\PN{大流士}}年六十二岁,取了{\PN{迦勒底}}国。

\par }\Chap{6}{\SH 但以理在狮子坑中
\par }{\PP \VerseOne{1}{\PN{大流士}}随心所愿,立一百二十个总督,治理通国。
\VS{2}又在他们以上立总长三人({\PN{但以理}}在其中),使总督在他们三人面前回复事务,免得王受亏损。
\VS{3}因这{\PN{但以理}}有美好的灵性,所以显然超乎其余的总长和总督,王又想立他治理通国。
\par }{\PP \VS{4}那时,总长和总督寻找{\PN{但以理}}误国的把柄,为要参他;只是找不着他的错误过失,因他忠心办事,毫无错误过失。
\VS{5}那些人便说:「我们要找参这{\PN{但以理}}的把柄,除非在他 神的律法中就寻不着。」
\par }{\PP \VS{6}于是,总长和总督纷纷聚集来见王,说:「愿{\PN{大流士}}王万岁!
\VS{7}国中的总长、钦差、总督、谋士,和巡抚彼此商议,要立一条坚定的禁令\FTNT{}{{\FR 6:7: }或译:求王下旨要立一条......},三十日内,不拘何人,若在王以外,或向神或向人求什么,就必扔在狮子坑中。
\VS{8}王啊,现在求你立这禁令,加盖玉玺,使禁令决不更改;照{\PN{米底亚}}和{\PN{波斯}}人的例是不可更改的。」
\VS{9}于是{\PN{大流士}}王立这禁令,加盖玉玺。
\VS{10}{\PN{但以理}}知道这{\ADD{禁令}}盖了玉玺,就到自己家里(他楼上的窗户开向{\PN{耶路撒冷}}),一日三次,双膝跪在他 神面前,祷告感谢,与素常一样。
\par }{\PP \VS{11}那些人就纷纷聚集,见{\PN{但以理}}在他 神面前祈祷恳求。
\VS{12}他们便进到王前,提王的禁令,说:「王啊,三十日内不拘何人,若在王以外,或向神或向人求什么,必被扔在狮子坑中。王不是在这禁令上盖了玉玺吗?」王回答说:「实有这事,照{\PN{米底亚}}和{\PN{波斯}}人的例是不可更改的。」
\VS{13}他们对王说:「王啊,那被掳之{\PN{犹大}}人中的{\PN{但以理}}不理你,也不遵你盖了玉玺的禁令,他竟一日三次祈祷。」
\VS{14}王听见这话,就甚愁烦,一心要救{\PN{但以理}},筹划解救他,直到日落的时候。
\VS{15}那些人就纷纷聚集来见王,说:「王啊,当知道{\PN{米底亚}}人和{\PN{波斯}}人有例,凡王所立的禁令和律例都不可更改。」
\par }{\PP \VS{16}王下令,人就把{\PN{但以理}}带来,扔在狮子坑中。王对{\PN{但以理}}说:「你所常事奉的 神,他必救你。」
\VS{17}有人搬石头放在坑口,王用自己的玺和大臣的印,封闭那坑,使惩办{\PN{但以理}}的事毫无更改。
\VS{18}王回宫,终夜禁食,无人拿乐器到他面前,并且睡不着觉。
\par }{\PP \VS{19}次日黎明,王就起来,急忙往狮子坑那里去。
\VS{20}临近坑边,哀声呼叫{\PN{但以理}},对{\PN{但以理}}说:「永生 神的仆人{\PN{但以理}}啊,你所常事奉的 神能救你脱离狮子吗?」
\VS{21}{\PN{但以理}}对王说:「愿王万岁!
\VS{22}我的 神差遣使者,封住狮子的口,叫狮子不伤我;因我在 神面前无辜,我在王面前也没有行过亏损的事。」
\VS{23}王就甚喜乐,吩咐人将{\PN{但以理}}从坑里系上来。于是{\PN{但以理}}从坑里被系上来,身上毫无伤损,因为信靠他的 神。
\VS{24}王下令,人就把那些控告{\PN{但以理}}的人,连他们的妻子儿女都带来,扔在狮子坑中。他们还没有到坑底,狮子就抓住\FTNT{}{{\FR 6:24: }原文是胜了}他们,咬碎他们的骨头。
\par }{\PP \VS{25}那时,{\PN{大流士}}王传旨,晓谕住在全地各方、各国、各族的人说:「愿你们大享平安!
\VS{26}现在我降旨晓谕我所统辖的全国人民,要在{\PN{但以理}}的 神面前,战兢恐惧。
\par }{\Q 因为他是永远长存的活 神,
\par }{\Q 他的国永不败坏;
\par }{\Q 他的权柄永存无极!
\par }{\Q \VS{27}他护庇人,搭救人,
\par }{\Q 在天上地下施行神迹奇事,
\par }{\Q 救了{\PN{但以理}}脱离狮子的口。」
\par }{\PP \VS{28}如此,这{\PN{但以理}},当{\PN{大流士}}{\ADD{王}}在位的时候和{\PN{波斯}}{\ADD{王}}{\PN{塞鲁士}}在位的时候,大享亨通。

\par }\Chap{7}{\SH 但以理叙述异象
\par }{\R (7·1—12·13)
\par }{\SH 四兽的异象
\par }{\PP \VerseOne{1}{\PN{巴比伦}}王{\PN{伯沙撒}}元年,{\PN{但以理}}在床上做梦,见了脑中的异象,就记录这梦,述说其中的大意。
\VS{2}{\PN{但以理}}说:
\par }{\PP 我夜里见异象,看见天的四风陡起,刮在大海之上。
\VS{3}有四个大兽从海中上来,{\ADD{形状}}各有不同:
\VS{4}头一个像狮子,有鹰的翅膀;我正观看的时候,兽的翅膀被拔去,兽从地上得立起来,用两脚站立,像人一样,又得了人心。
\VS{5}又有一兽如熊,就是第二兽,旁跨而坐,口齿内衔着三根肋骨。有吩咐这兽的说:「起来吞吃多肉。」
\VS{6}此后我观看,又有一兽如豹,背上有鸟的四个翅膀;这兽有四个头,又得了权柄。
\VS{7}其后我在夜间的异象中观看,见第四兽甚是可怕,极其强壮,大有力量,有大铁牙,吞吃嚼碎,所剩下的用脚践踏。这兽与前三兽大不相同,头有十角。
\VS{8}我正观看这些角,见其中又长起一个小角;先前的角中有三角在这角前,连根被它拔出来。这角有眼,像人的眼,有口说夸大的话。
\par }{\SH 永存者的异象
\par }{\Q \VS{9}我观看,
\par }{\Q 见有宝座设立,
\par }{\Q 上头坐着亘古常在者。
\par }{\Q 他的衣服洁白如雪,
\par }{\Q 头发如纯净的羊毛。
\par }{\Q 宝座乃火焰,
\par }{\Q 其轮乃烈火。
\par }{\Q \VS{10}从他面前有火,像河发出;
\par }{\Q 事奉他的有千千,
\par }{\Q 在他面前侍立的有万万;
\par }{\Q 他坐着要行审判,
\par }{\Q 案卷都展开了。
\par }{\PP \VS{11}那时我观看,见那兽因小角说夸大话的声音被杀,身体损坏,扔在火中焚烧。
\VS{12}其余的兽,权柄都被夺去,生命却仍存留,直到所定的时候和日期。
\par }{\Q \VS{13}我在夜间的异象中观看,
\par }{\Q 见有一位像人子的,
\par }{\Q 驾着天云而来,
\par }{\Q 被领到亘古常在者面前,
\par }{\Q \VS{14}得了权柄、荣耀、国度,
\par }{\Q 使各方、各国、各族的人都事奉他。
\par }{\Q 他的权柄是永远的,不能废去;
\par }{\Q 他的国必不败坏。
\par }{\SH 解释异象
\par }{\PP \VS{15}至于我—{\PN{但以理}},我的灵在我里面愁烦,我脑中的异象使我惊惶。
\VS{16}我就近一位侍立者,问他这一切的真情。他就告诉我,将那事的讲解给我说明:
\VS{17}这四个大兽就是四王将要在世上兴起。
\VS{18}然而,至高者的圣民,必要得国享受,直到永永远远。
\VS{19}那时我愿知道第四兽的真情,它为何与那三兽的真情大不相同,甚是可怕,有铁牙铜爪,吞吃嚼碎,所剩下的用脚践踏;
\VS{20}头有十角和那另长的一角,在这角前有三角被它打落。这角有眼,有说夸大话的口,形状强横,过于它的同类。
\VS{21}我观看,见这角与圣民争战,胜了他们。
\VS{22}直到亘古常在者来给至高者的圣民伸冤,圣民得国的时候就到了。
\par }{\Q \VS{23}那侍立者这样说:
\par }{\Q 第四兽就是世上必有的第四国,
\par }{\Q 与一切国大不相同,
\par }{\Q 必吞吃全地,
\par }{\Q 并且践踏嚼碎。
\par }{\Q \VS{24}至于那十角,就是从这国中必兴起的十王,
\par }{\Q 后来又兴起一王,
\par }{\Q 与先前的不同;
\par }{\Q 他必制伏三王。
\par }{\Q \VS{25}他必向至高者说{\ADD{夸大的}}话,
\par }{\Q 必折磨至高者的圣民,
\par }{\Q 必想改变节期和律法。
\par }{\Q 圣民必交付他手一载、二载、半载。
\par }{\Q \VS{26}然而,审判者必坐着行审判;
\par }{\Q 他的权柄必被夺去,
\par }{\Q 毁坏,灭绝,一直到底。
\par }{\Q \VS{27}国度、权柄,和天下诸国的大权
\par }{\Q 必赐给至高者的圣民。
\par }{\Q 他的国是永远的;
\par }{\Q 一切掌权的都必事奉他,顺从他。
\par }{\PP \VS{28}那事至此完毕。至于我—{\PN{但以理}},心中甚是惊惶,脸色也改变了,却将那事存记在心。

\par }\Chap{8}{\SH 公绵羊、公山羊的异象
\par }{\PP \VerseOne{1}{\PN{伯沙撒}}王在位第三年,有异象现与我—{\PN{但以理}},是在先前所见的异象之后。
\VS{2}我见了异象的时候,我{\ADD{以为}}在{\PN{以拦}}省{\PN{书珊}}城\FTNT{}{{\FR 8:2: }或译:宫}中;我见异象又如在{\PN{乌莱河}}边。
\VS{3}我举目观看,见有双角的公绵羊站在河边,两角都高。这角高过那角,更高的是后长的。
\VS{4}我见那公绵羊往西、往北、往南抵触。兽在它面前都站立不住,也没有能救护脱离它手的;但它任意而行,自高自大。
\par }{\PP \VS{5}我正思想的时候,见有一只公山羊从西而来,遍行全地,{\ADD{脚}}不沾尘。这山羊两眼当中有一非常的角。
\VS{6}它往我所看见、站在河边有双角的公绵羊那里去,大发忿怒,向它直闯。
\VS{7}我见公山羊就近公绵羊,向它发烈怒,抵触它,折断它的两角。绵羊在它面前站立不住;它将绵羊触倒在地,用脚践踏,没有能救绵羊脱离它手的。
\VS{8}这山羊极其自高自大,正强盛的时候,那大角折断了,又在角根上向天的四方\FTNT{}{{\FR 8:8: }原文是风}长出四个非常的角来。
\VS{9}四角之中有一角长出一个小角,向南、向东、向荣美之{\ADD{地}},渐渐成为强大。
\VS{10}它渐渐强大,高及天象,将些天象和星宿抛落在地,用脚践踏。
\VS{11}并且它自高自大,以为高及天象之君;除掉常献给君的{\ADD{燔祭}},毁坏君的圣所。
\VS{12}因罪过的缘故,有军旅和常献的{\ADD{燔祭}}交付它。它将真理抛在地上,{\ADD{任意}}而行,无不顺利。
\par }{\PP \VS{13}我听见有一位圣者说话,又有一位圣者问那说话的圣者说:「这{\ADD{除掉}}常献的{\ADD{燔祭}}和施行毁坏的罪过,将圣所与军旅\FTNT{}{{\FR 8:13: }或译:以色列的军}践踏的异象,要到几时才{\ADD{应验}}呢?」
\VS{14}他对我说:「到二千三百日,圣所就必洁净。」
\par }{\SH 天使加百列解释异象
\par }{\PP \VS{15}我—{\PN{但以理}}见了这异象,愿意明白其中的意思。忽有一位形状像人的站在我面前。
\VS{16}我又听见{\PN{乌莱河}}{\ADD{两岸}}中有人声呼叫说:「{\PN{加百列}}啊,要使此人明白这异象。」
\VS{17}他便来到我所站的地方。他一来,我就惊慌俯伏在地;他对我说:「人子啊,你要明白,因为这是关乎末后的异象。」
\par }{\PP \VS{18}他与我说话的时候,我面伏在地沉睡;他就摸我,扶我站起来,
\VS{19}说:「我要指示你恼怒临完必有的事,因为这是关乎末后的定期。
\VS{20}你所看见双角的公绵羊,就是{\PN{米底亚}}和{\PN{波斯}}王。
\VS{21}那公山羊就是{\PN{希腊}}王\FTNT{}{{\FR 8:21: }希腊:原文是雅完;下同};两眼当中的大角就是头一王。
\VS{22}至于那折断了的角,在其根上又长出四角,这四角就是四国,必从这国里兴起来,只是权势都不及他。
\VS{23}这四国末时,犯法的人罪恶满盈,必有一王兴起,面貌凶恶,能用双关的诈语。
\VS{24}他的权柄必大,却不是因自己的能力;他必行非常的毁灭,事情顺利,{\ADD{任意}}而行;又必毁灭有能力的和圣民。
\VS{25}他用权术成就手中的诡计,心里自高自大,在人坦然无备的时候,毁灭多人;又要站起来攻击万君之君,至终却非因人手而灭亡。
\VS{26}所说二千三百日的异象是真的,但你要将这异象封住,因为关乎{\ADD{后来}}许多的日子。」
\par }{\PP \VS{27}于是我—{\PN{但以理}}昏迷不醒,病了数日,然后起来办理王的事务。我因这异象惊奇,却无人能明白其中的意思。

\par }\Chap{9}{\SH 但以理为同胞祷告
\par }{\PP \VerseOne{1}{\PN{米底亚}}族{\PN{亚哈随鲁}}的儿子{\PN{大流士}}立为{\PN{迦勒底}}国的王元年,
\VS{2}就是他在位第一年,我—{\PN{但以理}}从书上得知耶和华的话临到先知{\PN{耶利米}},论{\PN{耶路撒冷}}荒凉的年数,七十年为满。
\par }{\PP \VS{3}我便禁食,披麻蒙灰,定意向主 神祈祷恳求。
\VS{4}我向耶和华—我的 神祈祷、认罪,说:「主啊,大而可畏的 神,向爱主、守主诫命的人守约施慈爱。
\VS{5}我们犯罪作孽,行恶叛逆,偏离你的诫命典章,
\VS{6}没有听从你仆人众先知奉你名向我们君王、首领、列祖,和国中一切百姓所说的话。
\VS{7}主啊,你是公义的,我们是脸上蒙羞的;因我们{\PN{犹大}}人和{\PN{耶路撒冷}}的居民,并{\PN{以色列}}众人,或在近处,或在远处,被你赶到各国的人,都得罪了你,正如今日一样。
\VS{8}主啊,我们和我们的君王、首领、列祖因得罪了你,就都脸上蒙羞。
\VS{9}主—我们的 神是怜悯饶恕人的,我们却违背了他,
\VS{10}也没有听从耶和华—我们 神的话,没有遵行他借仆人众先知向我们所陈明的律法。
\VS{11}{\PN{以色列}}众人都犯了你的律法,偏行,不听从你的话;因此,在你仆人{\PN{摩西}}律法上所写的咒诅和誓言都倾在我们身上,因我们得罪了 神。
\VS{12}他使大灾祸临到我们,成就了警戒我们和审判我们官长的话;原来在普天之下未曾行过像在{\PN{耶路撒冷}}所行的。
\VS{13}这一切灾祸临到我们身上是照{\PN{摩西}}律法上所写的,我们却没有求耶和华—我们 神的恩典,使我们回头离开罪孽,明白你的真理。
\VS{14}所以耶和华留意使这灾祸临到我们身上,因为耶和华—我们的 神在他所行的事上都是公义;我们并没有听从他的话。
\VS{15}主—我们的 神啊,你曾用大能的手领你的子民出{\PN{埃及}}地,使自己得了名,正如今日一样。我们犯了罪,作了恶。
\VS{16}主啊,求你按你的大仁大义,使你的怒气和忿怒转离你的城{\PN{耶路撒冷}},就是你的圣山。{\PN{耶路撒冷}}和你的子民,因我们的罪恶和我们列祖的罪孽被四围的人羞辱。
\VS{17}我们的 神啊,现在求你垂听仆人的祈祷恳求,为自己使脸光照你荒凉的圣所。
\VS{18}我的 神啊,求你侧耳而听,睁眼而看,{\ADD{眷顾}}我们荒凉{\ADD{之地}}和称为你名下的城。我们在你面前恳求,原不是因自己的义,乃因你的大怜悯。
\VS{19}求主垂听,求主赦免,求主应允而行,为你自己不要迟延。我的 神啊,因这城和这民都是称为你名下的。」
\par }{\SH 加百列解释预言
\par }{\PP \VS{20}我说话,祷告,承认我的罪和本国之民{\PN{以色列}}的罪,为我 神的圣山,在耶和华—我 神面前恳求。
\VS{21}我正祷告的时候,先前在异象中所见的那位{\PN{加百列}},奉命迅速飞来,约在献晚祭的时候,按手在我身上。
\VS{22}他指教我说:「{\PN{但以理}}啊,现在我出来要使你有智慧,有聪明。
\VS{23}你初恳求的时候,就发出命令,我来告诉你,因你大蒙眷爱;所以你要思想明白这{\ADD{以下的}}事和异象。
\par }{\PP \VS{24}「为你本国之民和你圣城,已经定了七十个七。要止住罪过,除净罪恶,赎尽罪孽,引进\FTNT{}{{\FR 9:24: }或译:彰显}永义,封住异象和预言,并膏至圣者\FTNT{}{{\FR 9:24: }者:或译所}。
\VS{25}你当知道,当明白,从出令重新建造{\PN{耶路撒冷}},直到有受膏君的时候,必有七个七和六十二个七。正在艰难的时候,{\PN{耶路撒冷}}城连街带濠都必重新建造。
\VS{26}过了六十二个七,那\FTNT{}{{\FR 9:26: }或译:有}受膏者必被剪除,一无所有;必有一王的民来毁灭这城和圣所,至终必如洪水冲没。必有争战,一直到底,荒凉的事已经定了。
\VS{27}一七之内,他必与许多人坚定盟约;一七之半,他必使祭祀与供献止息。那行毁坏可憎的\FTNT{}{{\FR 9:27: }或译:使地荒凉的}如飞而来,并且有{\ADD{忿怒}}倾在那行毁坏的身上\FTNT{}{{\FR 9:27: }或译:倾在那荒凉之地},直到所定的结局。」

\par }\Chap{10}{\SH 底格里斯大河边的异象
\par }{\PP \VerseOne{1}{\PN{波斯}}王{\PN{塞鲁士}}第三年,有事显给称为{\PN{伯提沙撒}}的{\PN{但以理}}。这事是真的,是指着大争战;{\PN{但以理}}通达这事,明白这异象。
\par }{\PP \VS{2}当那时,我—{\PN{但以理}}悲伤了三个七日。
\VS{3}美味我没有吃,酒肉没有入我的口,也没有用油抹我的身,直到满了三个七日。
\par }{\PP \VS{4}正月二十四日,我在{\PN{底格里斯}}大河边,
\VS{5}举目观看,见有一人身穿细麻衣,腰束{\PN{乌法}}精金带。
\VS{6}他身体如水苍玉,面貌如闪电,眼目如火把,手和脚如光明的铜,说话的声音如大众的声音。
\VS{7}这异象惟有我—{\PN{但以理}}一人看见,同着我的人没有看见。他们却大大战兢,逃跑隐藏,
\VS{8}只剩下我一人。我见了这大异象便浑身无力,面貌失色,毫无气力。
\VS{9}我却听见他说话的声音,一听见就面伏在地沉睡了。
\par }{\PP \VS{10}忽然,有一手按在我身上,使我用膝和手掌支持微起。
\VS{11}他对我说:「大蒙眷爱的{\PN{但以理}}啊,要明白我与你所说的话,只管站起来,因为我现在奉差遣来到你这里。」他对我说这话,我便战战兢兢地立起来。
\VS{12}他就说:「{\PN{但以理}}啊,不要惧怕!因为从你第一日专心求明白{\ADD{将来的事}},又在你 神面前刻苦己心,你的言语已蒙应允;我是因你的言语而来。
\VS{13}但{\PN{波斯}}国的{\ADD{魔}}君拦阻我二十一日。忽然有大君\FTNT{}{{\FR 10:13: }就是天使长;二十一节同}中的一位{\PN{米迦勒}}来帮助我,我就停留在{\PN{波斯}}诸王那里。
\VS{14}现在我来,要使你明白本国之民日后必遭遇的事,因为这异象关乎后来{\ADD{许多的}}日子。」
\VS{15}他向我这样说,我就脸面朝地,哑口无声。
\VS{16}不料,有一位像人的,摸我的嘴唇,我便开口向那站在我面前的说:「我主啊,因见这异象,我大大愁苦,毫无气力。
\VS{17}我主的仆人怎能与我主说话呢?我{\ADD{一见异象}}就浑身无力,毫无气息。」
\par }{\PP \VS{18}有一位形状像人的又摸我,使我有力量。
\VS{19}他说:「大蒙眷爱的人哪,不要惧怕,愿你平安!你总要坚强。」他一向我说话,我便觉得有力量,说:「我主请说,因你使我有力量。」
\VS{20}他就说:「你知道我为何来见你吗?现在我要回去与{\PN{波斯}}的{\ADD{魔}}君争战,我去后,{\PN{希腊}}\FTNT{}{{\FR 10:20: }原文是雅完}的{\ADD{魔}}君必来。
\VS{21}但我要将那录在真确书上的事告诉你。除了你们的{\ADD{大}}君{\PN{米迦勒}}之外,没有帮助我抵挡这两{\ADD{魔}}君的。」

\par }\Chap{11}{\PP \VerseOne{1}{\ADD{又说}}:「当{\PN{米底亚}}王{\PN{大流士}}元年,我曾起来扶助{\PN{米迦勒}},使他坚强。」
\VS{2}现在我将真事指示你:
\par }{\SH 南方王和北方王
\par }{\PP 「{\PN{波斯}}还有三王兴起,第四王必富足远胜诸王。他因富足成为强盛,就必激动大众攻击{\PN{希腊}}国。
\VS{3}必有一个勇敢的王兴起,执掌大权,随意而行。
\VS{4}他兴起的时候,他的国必破裂,向天的四方\FTNT{}{{\FR 11:4: }方:原文是风}分开,却不归他的后裔,治国的权势也都不及他;因为他的国必被拔出,归与他后裔之外的人。
\par }{\PP \VS{5}「南方的王必强盛,他将帅中必有一个比他更强盛,执掌权柄,他的权柄甚大。
\VS{6}过些年后,他们必互相连合,南方王的女儿必就了北方王来立约;但这女子帮助之力存立不住,王和他所倚靠之力也不能存立。这女子和引导她来的,并生她的,以及当时扶助她的,都必交与{\ADD{死地}}。
\VS{7}但这女子的本家\FTNT{}{{\FR 11:7: }原文是根}必另生一子\FTNT{}{{\FR 11:7: }子:原文是枝}继续王位,他必率领军队进入北方王的保障,攻击他们,而且得胜;
\VS{8}并将他们的神像和铸成的偶像,与金银的宝器掠到{\PN{埃及}}去。数年之内,他不去攻击北方的王。
\VS{9}北方的王\FTNT{}{{\FR 11:9: }原文是他}必入南方王的国,却要仍回本地。
\par }{\PP \VS{10}「北方王\FTNT{}{{\FR 11:10: }原文是他}的二子必动干戈,招聚许多军兵。这军兵前去,如洪水泛滥,又必再去争战,直到南方王的保障。
\VS{11}南方王必发烈怒,出来与北方王争战,摆列大军;北方王的军兵必交付他手。
\VS{12}他的众军高傲,他的心也必自高;他虽使数万人仆倒,却不得{\ADD{常}}胜。
\par }{\PP \VS{13}「北方王必回来摆列大军,比先前的更多。满了所定的年数,他必率领大军,带极多的军装来。
\VS{14}那时,必有许多人起来攻击南方王,并且你本国的强暴人必兴起,要应验那异象,他们却要败亡。
\VS{15}北方王必来筑垒攻取坚固城;南方的军兵必站立不住,就是选择的精兵\FTNT{}{{\FR 11:15: }原文是民}也无力站住。
\VS{16}来攻击他的,必任意而行,无人在北方王\FTNT{}{{\FR 11:16: }原文是他}面前站立得住。他必站在那荣美之地,用手施行毁灭。
\par }{\PP \VS{17}「他必定意用全国之力而来,立公正的约,照约而行,将自己的女儿给南方王为妻,想要败坏他\FTNT{}{{\FR 11:17: }或译:埃及},这计却不得成就,与自己毫无益处。
\VS{18}其后他必转回夺取了许多海岛。但有一大帅,除掉他令人受的羞辱,并且使这羞辱归他本身。
\VS{19}他就必转向本地的保障,却要绊跌仆倒,归于无有。
\par }{\PP \VS{20}「那时,必有一人兴起接续他为王,使横征暴敛的人通行国中的荣美地。这王不多日就必灭亡,却不因忿怒,也不因争战。」
\par }{\SH 北方的恶王
\par }{\PP \VS{21}「必有一个卑鄙的人兴起接续为王,人未曾将国的尊荣给他,他却趁人坦然无备的时候,用谄媚的话得国。
\VS{22}必有无数的军兵势如洪水,在他面前冲没败坏;同盟的君也必如此。
\VS{23}与那君结盟之后,他必行诡诈,因为他必上来以微小的军\FTNT{}{{\FR 11:23: }原文是民}成为强盛。
\VS{24}趁人坦然无备的时候,他必来到国中极肥美之地,行他列祖和他列祖之祖所未曾行的,将掳物、掠物,和财宝散给众人,又要设计攻打保障,然而这都是暂时的。
\par }{\PP \VS{25}「他必奋勇向前,率领大军攻击南方王;南方王也必以极大极强的军兵与他争战,却站立不住,因为有人设计谋害南方王。
\VS{26}吃王膳的,必败坏他;他的军队必被冲没,而且被杀的甚多。
\VS{27}至于这二王,他们心怀恶计,同席说谎,{\ADD{计谋}}却不成就;因为到了定期,事就了结。
\VS{28}北方王\FTNT{}{{\FR 11:28: }原文是他}必带许多财宝回往本国,他的心反对圣约,{\ADD{任意}}而行,回到本地。
\par }{\PP \VS{29}「到了定期,他必返回,来到南方。后一次却不如前一次,
\VS{30}因为{\PN{基提}}战船必来攻击他,他就丧胆而回,又要恼恨圣约,{\ADD{任意}}而行;他必回来联络背弃圣约的人。
\VS{31}他必兴兵,这兵必亵渎圣地,就是保障,除掉常献的{\ADD{燔祭}},设立那行毁坏可憎的。
\VS{32}作恶违背{\ADD{圣}}约的人,他必用巧言勾引;惟独认识 神的子民必刚强行事。
\VS{33}民间的智慧人必训诲多人;然而他们多日必倒在刀下,或被火烧,或被掳掠抢夺。
\VS{34}他们仆倒的时候,稍得扶助,却有许多人用谄媚的话亲近他们。
\VS{35}智慧人中有些仆倒的,为要熬炼其余的人,使他们清净洁白,直到末了;因为到了定期,事就了结。
\par }{\PP \VS{36}「王必任意而行,自高自大,超过所有的神,又用奇异的话攻击万神之神。他必行事亨通,直到{\ADD{主的}}忿怒完毕,因为所定的事必然成就。
\VS{37}他必不顾他列祖的神,也不顾妇女所羡慕的神,无论何神他都不顾;因为他必自大,高过一切。
\VS{38}他倒要敬拜保障的神,用金、银、宝石和可爱之物敬奉他列祖所不认识的神。
\VS{39}他必靠外邦神的帮助,攻破最坚固的保障。凡承认他的,他必将荣耀加给他们,使他们管辖许多人,又为贿赂分地{\ADD{与他们}}。
\par }{\PP \VS{40}「到末了,南方王要与他交战。北方王必用战车、马兵,和许多战船,势如暴风来攻击他,也必进入列国,如洪水泛滥。
\VS{41}又必进入那荣美之地,有许多{\ADD{国}}就被倾覆,但{\PN{以东}}人、{\PN{摩押}}人,和一大半{\PN{亚扪}}人必脱离他的手。
\VS{42}他必伸手攻击列国;{\PN{埃及}}地也不得脱离。
\VS{43}他必把持{\PN{埃及}}的金银财宝和各样的宝物。{\PN{利比亚}}人和{\PN{古实}}人都必跟从他。
\VS{44}但从东方和北方必有消息扰乱他,他就大发烈怒出去,要将多人杀灭净尽。
\VS{45}他必在海和荣美的圣山中间设立他如宫殿的帐幕;然而到了他的结局,必无人能帮助他。」

\par }\Chap{12}{\SH 末日
\par }{\PP \VerseOne{1}「那时,保佑你本国之民的天使长\FTNT{}{{\FR 12:1: }原文是大君}{\PN{米迦勒}}必站起来,并且有大艰难,从有国以来直到此时,没有这样的。你本国的民中,凡名录在册上的,必得拯救。
\VS{2}睡在尘埃中的,必有多人复醒。其中有得永生的,有受羞辱永远被憎恶的。
\VS{3}智慧人必发光如同天上的光;那使多人归义的,必发光如星,直到永永远远。
\VS{4}{\PN{但以理}}啊,你要隐藏这话,封闭这书,直到末时。必有多人来往奔跑\FTNT{}{{\FR 12:4: }或译:切心研究},知识就必增长。」
\par }{\PP \VS{5}我—{\PN{但以理}}观看,见另有两个人站立:一个在河这边,一个在河那边。
\VS{6}有一个问那站在河水以上、穿细麻衣的说:「这奇异的事到几时才应验呢?」
\par }{\PP \VS{7}我听见那站在河水以上、穿细麻衣的,向天举起左右手,指着活到永远的{\ADD{主}}起誓说:「要到一载、二载、半载,打破圣民权力的时候,这一切事就都应验了。」
\VS{8}我听见这话,却不明白,就说:「我主啊,这些事的结局是怎样呢?」
\par }{\PP \VS{9}他说:「{\PN{但以理}}啊,你只管去;因为这话已经隐藏封闭,直到末时。
\VS{10}必有许多人使自己清净洁白,且被熬炼;但恶人仍必行恶,一切恶人都不明白,惟独智慧人能明白。
\VS{11}从除掉常献的{\ADD{燔祭}},并设立那行毁坏可憎之物的时候,必有一千二百九十日。
\VS{12}等到一千三百三十五日的,那人便为有福。
\par }{\PP \VS{13}「你且去等候结局,因为你必安歇。到了末期,你必起来,享受你的{\ADD{福}}分。」
\par }