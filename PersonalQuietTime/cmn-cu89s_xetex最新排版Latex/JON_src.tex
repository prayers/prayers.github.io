\NormalFont\ShortTitle{约拿书}
{\MT 约拿书

\par }\ChapOne{1}{\SH 约拿违背耶和华
\par }{\PP \VerseOne{1}耶和华的话临到{\PN{亚米太}}的儿子{\PN{约拿}},说:
\VS{2}「你起来往{\PN{尼尼微}}大城去,向其中的居民呼喊,因为他们的恶达到我面前。」
\VS{3}{\PN{约拿}}却起来,逃往{\PN{他施}}去躲避耶和华;下到{\PN{约帕}},遇见一只船,要往{\PN{他施}}去。他就给了船价,上了船,要与船上的人同往{\PN{他施}}去躲避耶和华。
\par }{\PP \VS{4}然而耶和华使海中起大风,海就狂风大作,甚至船几乎破坏。
\VS{5}水手便惧怕,各人哀求自己的神。他们将船上的货物抛在海中,为要使船轻些。{\PN{约拿}}已下到底舱,躺卧沉睡。
\VS{6}船主到他那里对他说:「你这沉睡的人哪,为何这样呢?起来,求告你的神,或者神顾念我们,使我们不致灭亡。」
\par }{\PP \VS{7}船上的人彼此说:「来吧,我们掣签,看看这灾临到我们是因谁的缘故。」于是他们掣签,掣出{\PN{约拿}}来。
\VS{8}众人对他说:「请你告诉我们,这灾临到我们是因谁的缘故?你以何事为业?你从哪里来?你是哪一国?属哪一族的人?」
\VS{9}他说:「我是{\PN{希伯来}}人。我敬畏耶和华—那创造沧海旱地之天上的 神。」
\VS{10}他们就大大惧怕,对他说:「你做的是什么事呢?」他们已经知道他躲避耶和华,因为他告诉了他们。
\par }{\PP \VS{11}他们问他说:「我们当向你怎样行,使海浪平静呢?」这话是因海浪越发翻腾。
\VS{12}他对他们说:「你们将我抬起来,抛在海中,海就平静了;我知道你们遭这大风是因我的缘故。」
\VS{13}然而那些人竭力荡桨,要把船拢岸,却是不能,因为海浪越发向他们翻腾。
\VS{14}他们便求告耶和华说:「耶和华啊,我们恳求你,不要因这人的性命使我们死亡,不要使流无辜血的罪归与我们;因为你—耶和华是随自己的意旨行事。」
\VS{15}他们遂将{\PN{约拿}}抬起,抛在海中,海的狂浪就平息了。
\VS{16}那些人便大大敬畏耶和华,向耶和华献祭,并且许愿。
\par }{\PP \VS{17}耶和华安排一条大鱼吞了{\PN{约拿}},他在鱼腹中三日三夜。

\par }\Chap{2}{\SH 约拿的祷告
\par }{\PP \VerseOne{1}{\PN{约拿}}在鱼腹中祷告耶和华—他的 神,
\VS{2}说:
\par }{\Q 我遭遇患难求告耶和华,
\par }{\Q 你就应允我;
\par }{\Q 从阴间的深处呼求,
\par }{\Q 你就俯听我的声音。
\par }{\Q \VS{3}你将我投下深渊,
\par }{\Q 就是海的深处;
\par }{\Q 大水环绕我,
\par }{\Q 你的波浪洪涛都漫过我身。
\par }{\Q \VS{4}我说:我从你眼前虽被驱逐,
\par }{\Q 我仍要仰望你的圣殿。
\par }{\Q \VS{5}诸水环绕我,几乎淹没我;
\par }{\Q 深渊围住我;
\par }{\Q 海草缠绕我的头。
\par }{\Q \VS{6}我下到山根,
\par }{\Q 地的门将我永远{\ADD{关住}}。
\par }{\Q 耶和华—我的 神啊,
\par }{\Q 你却将我的性命从坑中救出来。
\par }{\Q \VS{7}我心在我里面发昏的时候,
\par }{\Q 我就想念耶和华。
\par }{\Q 我的祷告进入你的圣殿,
\par }{\Q 达到你的面前。
\par }{\Q \VS{8}那信奉虚无{\ADD{之神}}的人,
\par }{\Q 离弃怜爱他们的主;
\par }{\Q \VS{9}但我必用感谢的声音献祭与你。
\par }{\Q 我所许的愿,我必偿还。
\par }{\Q 救恩出于耶和华。
\par }{\Q \VS{10}耶和华吩咐鱼,鱼就把{\PN{约拿}}吐在旱地上。

\par }\Chap{3}{\SH 约拿顺服耶和华
\par }{\PP \VerseOne{1}耶和华的话二次临到{\PN{约拿}}说:
\VS{2}「你起来!往{\PN{尼尼微}}大城去,向其中的居民宣告我所吩咐你的话。」
\VS{3}{\PN{约拿}}便照耶和华的话起来,往{\PN{尼尼微}}去。这{\PN{尼尼微}}是极大的城,有三日的路程。
\VS{4}{\PN{约拿}}进城走了一日,宣告说:「再等四十日,{\PN{尼尼微}}必倾覆了!」
\VS{5}{\PN{尼尼微}}人信服 神,便宣告禁食,从最大的到至小的都穿麻衣\FTNT{}{{\FR 3:5: }或译:披上麻布}。
\par }{\PP \VS{6}这信息传到{\PN{尼尼微}}王的耳中,他就下了宝座,脱下朝服,披上麻布,坐在灰中。
\VS{7}他又使人遍告{\PN{尼尼微}}通城,说:「王和大臣有令,人不可尝什么,牲畜、牛羊不可吃草,也不可喝水。
\VS{8}人与牲畜都当披上麻布;人要切切求告 神。各人回头离开所行的恶道,丢弃手中的强暴。
\VS{9}或者 神转意后悔,不发烈怒,使我们不致灭亡,也未可知。」
\par }{\PP \VS{10}于是 神察看他们的行为,见他们离开恶道,他就后悔,不把所说的灾祸降与他们了。

\par }\Chap{4}{\SH 约拿的忿怒和 神的怜悯
\par }{\PP \VerseOne{1}这事{\PN{约拿}}大大不悦,且甚发怒,
\VS{2}就祷告耶和华说:「耶和华啊,我在本国的时候岂不是这样说吗?我知道你是有恩典、有怜悯的 神,不轻易发怒,有丰盛的慈爱,并且后悔不降所说的灾,所以我急速逃往{\PN{他施}}去。
\VS{3}耶和华啊,现在求你取我的命吧!因为我死了比活着还好。」
\VS{4}耶和华说:「你这样发怒合乎理吗?」
\par }{\PP \VS{5}于是{\PN{约拿}}出城,坐在城的东边,在那里为自己搭了一座棚,坐在棚的荫下,要看看那城究竟如何。
\VS{6}耶和华 神安排一棵蓖麻,使其发生高过{\PN{约拿}},影儿遮盖他的头,救他脱离苦楚;{\PN{约拿}}因这棵蓖麻大大喜乐。
\VS{7}次日黎明, 神却安排一条虫子咬这蓖麻,以致枯槁。
\VS{8}日头出来的时候, 神安排炎热的东风,日头曝晒{\PN{约拿}}的头,使他发昏,他就为自己求死,说:「我死了比活着还好!」
\VS{9}神对{\PN{约拿}}说:「你因这棵蓖麻发怒合乎理吗?」他说:「我发怒以至于死,都合乎理!」
\VS{10}耶和华说:「这蓖麻不是你栽种的,也不是你培养的;一夜发生,一夜干死,你尚且爱惜;
\VS{11}何况这{\PN{尼尼微}}大城,其中不能分辨左手右手的有十二万多人,并有许多牲畜,我岂能不爱惜呢?」
\par }