\NormalFont\ShortTitle{哥林多前书}
{\MT 哥林多前书

\par }\ChapOne{1}{\SH 祝福和感谢
\par }{\PP \VerseOne{1}奉 神旨意,蒙召作耶稣基督使徒的{\PN{保罗}},同兄弟{\PN{所提尼}},
\VS{2}写信给在{\PN{哥林多}} 神的教会,就是在基督耶稣里成圣、蒙召作圣徒的,以及所有在各处求告我主耶稣基督之名的人。基督是他们的{\ADD{主}},也是我们的{\ADD{主}}。
\VS{3}愿恩惠、平安从 神我们的父并主耶稣基督归与你们。
\par }{\PP \VS{4}我常为你们感谢我的 神,因 神在基督耶稣里所赐给你们的恩惠;
\VS{5}又因你们在他里面凡事富足,口才、知识都全备,
\VS{6}正如{\ADD{我为}}基督{\ADD{作}}的见证,在你们心里得以坚固,
\VS{7}以致你们在恩赐上没有一样不及人的,等候我们的主耶稣基督显现。
\VS{8}他也必坚固你们到底,叫你们在我们主耶稣基督的日子无可责备。
\VS{9}神是信实的,你们原是被他所召,好与他儿子—我们的主耶稣基督一同得分。
\par }{\SH 在教会中有纷争
\par }{\PP \VS{10}弟兄们,我借我们主耶稣基督的名劝你们都说一样的话。你们中间也不可分党,只要一心一意,彼此相合。
\VS{11}因为{\PN{革来}}氏{\ADD{家里}}的人曾对我提起弟兄们来,说你们中间有纷争。
\VS{12}我的意思就是你们各人说:「我是属{\PN{保罗}}的」;「我是属{\PN{亚波罗}}的」;「我是属{\PN{矶法}}的」;「我是属基督的」。
\VS{13}基督是分开的吗?{\PN{保罗}}为你们钉了十字架吗?你们是奉{\PN{保罗}}的名受了洗吗?
\VS{14}我感谢 神,除了{\PN{基利司布}}并{\PN{该犹}}以外,我没有给你们一个人施洗,
\VS{15}免得有人说,你们是奉我的名受洗。
\VS{16}我也给{\PN{司提反}}家施过洗,此外给别人施洗没有,我却记不清。
\VS{17}基督差遣我,原不是为施洗,乃是为传福音,并不用智慧的言语,免得基督的十字架落了空。
\par }{\SH 基督是 神的能力和智慧
\par }{\PP \VS{18}因为十字架的道理,在那灭亡的人为愚拙;在我们得救的人,却为 神的大能。
\VS{19}就如{\ADD{经上}}所记:
\par }{\Q 我要灭绝智慧人的智慧,
\par }{\Q 废弃聪明人的聪明。
\par }{\MM \VS{20}智慧人在哪里?文士在哪里?这世上的辩士在哪里? 神岂不是叫这世上的智慧变成愚拙吗?
\VS{21}世人凭自己的智慧,既不认识 神, 神就乐意用{\ADD{人所当作}}愚拙的道理拯救那些信的人;这就是 神的智慧了。
\VS{22}{\PN{犹太}}人是要神迹,{\PN{希腊}}人是求智慧,
\VS{23}我们却是传钉在十字架的基督,在{\PN{犹太}}人为绊脚石,在外邦人为愚拙;
\VS{24}但在那蒙召的,无论是{\PN{犹太}}人、{\PN{希腊}}人,基督总为 神的能力, 神的智慧。
\VS{25}因 神的愚拙总比人智慧, 神的软弱总比人强壮。
\par }{\PP \VS{26}弟兄们哪,可见你们蒙召的,按着肉体有智慧的不多,有能力的不多,有尊贵的也不多。
\VS{27}神却拣选了世上愚拙的,叫有智慧的羞愧;又拣选了世上软弱的,叫那强壮的羞愧。
\VS{28}神也拣选了世上卑贱的,被人厌恶的,以及那无有的,为要废掉那有的,
\VS{29}使一切有血气的,在 神面前一个也不能自夸。
\VS{30}但你们得在基督耶稣里是本乎 神, 神又使他成为我们的智慧、公义、圣洁、救赎。
\VS{31}如{\ADD{经上}}所记:「夸口的,当指着主夸口。」

\par }\Chap{2}{\SH 传扬钉十字架的基督
\par }{\PP \VerseOne{1}弟兄们,从前我到你们那里去,并没有用高言大智对你们宣传 神的奥秘。
\VS{2}因为我曾定了主意,在你们中间不知道别的,只知道耶稣基督并他钉十字架。
\VS{3}我在你们那里,又软弱,又惧怕又甚战兢。
\VS{4}我说的话、讲的道,不是用智慧委婉的言语,乃是用圣灵和大能的明证,
\VS{5}叫你们的信不在乎人的智慧,只在乎 神的大能。
\par }{\SH  神借着圣灵启示我们
\par }{\PP \VS{6}然而,在完全的人中,我们也讲智慧。但不是这世上的智慧,也不是这世上有权有位、将要败亡之人的智慧。
\VS{7}我们讲的,乃是从前所隐藏、 神奥秘的智慧,就是 神在万世以前预定使我们得荣耀的。
\VS{8}这智慧世上有权有位的人没有一个知道的,他们若知道,就不把荣耀的主钉在十字架上了。
\VS{9}如{\ADD{经上}}所记:
\par }{\Q  神为爱他的人所预备的
\par }{\Q 是眼睛未曾看见,
\par }{\Q 耳朵未曾听见,
\par }{\Q 人心也未曾想到的。
\par }{\MM \VS{10}只有 神借着圣灵向我们显明了,因为圣灵参透万事,就是 神深奥的事也参透了。
\VS{11}除了在人里头的灵,谁知道人的事?像这样,除了 神的灵,也没有人知道 神的事。
\VS{12}我们所领受的,并不是世上的灵,乃是从 神来的灵,叫我们能知道 神开恩赐给我们的事。
\VS{13}并且我们讲说这些事,不是用人智慧所指教的言语,乃是用{\ADD{圣}}灵所指教的言语,将属灵的话解释属灵的事\FTNT{}{{\FR 2:13: }或译:将属灵的事讲与属灵的人}。
\VS{14}然而,属血气的人不领会 神{\ADD{圣}}灵的事,反倒以为愚拙,并且不能知道,因为这些事惟有属灵的人才能看透。
\VS{15}属灵的人能看透万事,却没有一人能看透了他。
\VS{16}谁曾知道主的心去教导他呢?但我们是有基督的心了。

\par }\Chap{3}{\SH  神的同工
\par }{\PP \VerseOne{1}弟兄们,我从前对你们说话,不能把你们当作属灵的,只得把你们当作属肉体,在基督里为婴孩的。
\VS{2}我是用奶喂你们,没有用饭喂你们。那时你们不能吃,就是如今还是不能。
\VS{3}你们仍是属肉体的,因为在你们中间有嫉妒、纷争,这岂不是属乎肉体、照着世人的样子行吗?
\VS{4}有说:「我是属{\PN{保罗}}的」;有说:「我是属{\PN{亚波罗}}的。」这岂不是你们和世人一样吗?
\VS{5}{\PN{亚波罗}}算什么?{\PN{保罗}}算什么?无非是执事,照主所赐给他们各人的,引导你们相信。
\VS{6}我栽种了,{\PN{亚波罗}}浇灌了,惟有 神叫他生长。
\VS{7}可见栽种的,算不得什么,浇灌的,也算不得什么;只在那叫他生长的 神。
\VS{8}栽种的和浇灌的,都是一样,但将来各人要照自己的工夫得自己的赏赐。
\VS{9}因为我们是与 神同工的;你们是 神所耕种的田地,所建造的房屋。
\VS{10}我照 神所给我的恩,好像一个聪明的工头,立好了根基,有别人在上面建造;只是各人要谨慎怎样在上面建造。
\VS{11}因为那已经立好的根基就是耶稣基督,此外没有人能立别的根基。
\VS{12}若有人用金、银、宝石、草木、禾秸在这根基上建造,
\VS{13}各人的工程必然显露,因为那日子要将它表明出来,有火发现;这火要试验各人的工程怎样。
\VS{14}人在那根基上所建造的工程若存得住,他就要得赏赐。
\VS{15}人的工程若被烧了,他就要受亏损,自己却要得救;虽然得救,乃像从火里经过的一样。
\VS{16}岂不知你们是 神的殿, 神的灵住在你们里头吗?
\VS{17}若有人毁坏 神的殿, 神必要毁坏那人;因为 神的殿是圣的,这殿就是你们。
\par }{\PP \VS{18}人不可自欺。你们中间若有人在这世界自以为有智慧,倒不如变作愚拙,好成为有智慧的。
\VS{19}因这世界的智慧,在 神看是愚拙。如{\ADD{经上}}记着说:「主叫有智慧的,中了自己的诡计」;
\VS{20}又说:「主知道智慧人的意念是虚妄的。」
\VS{21}所以无论谁,都不可拿人夸口,因为万有全是你们的。
\VS{22}或{\PN{保罗}},或{\PN{亚波罗}},或{\PN{矶法}},或世界,或生,或死,或现今的事,或将来的事,全是你们的;
\VS{23}并且你们是属基督的,基督又是属 神的。

\par }\Chap{4}{\SH 使徒的工作
\par }{\PP \VerseOne{1}人应当以我们为基督的执事,为 神奥秘事的管家。
\VS{2}所求于管家的,是要他有忠心。
\VS{3}我被你们论断,或被别人论断,我都以为极小的事;连我自己也不论断自己。
\VS{4}我虽不觉得自己有错,却也不能因此得以称义;但判断我的乃是主。
\VS{5}所以,时候未到,什么都不要论断,只等主来,他要照出暗中的隐情,显明人心的意念。那时,各人要从 神那里得着称赞。
\par }{\PP \VS{6}弟兄们,我为你们的缘故,拿这些事转比自己和{\PN{亚波罗}},叫你们效法我们不可过于{\ADD{圣经}}所记,免得你们自高自大,贵重这个,轻看那个。
\VS{7}使你与人不同的是谁呢?你有什么不是领受的呢?若是领受的,为何自夸,仿佛不是领受的呢?
\VS{8}你们已经饱足了!已经丰富了!不用我们,自己就作王了!我愿意你们果真作王,叫我们也得与你们一同作王。
\VS{9}我想 神把我们使徒明明列在末后,好像定死罪的囚犯;因为我们成了一台戏,给世人和天使观看。
\VS{10}我们为基督的缘故算是愚拙的,你们在基督里倒是聪明的;我们软弱,你们倒强壮;你们有荣耀,我们倒被藐视。
\VS{11}直到如今,我们还是又饥又渴,又赤身露体,又挨打,又没有一定的住处,
\VS{12}并且劳苦,亲手做工。被人咒骂,我们就祝福;被人逼迫,我们就忍受;
\VS{13}被人毁谤,我们就善劝。直到如今,人还把我们看作世界上的污秽,万物中的渣滓。
\par }{\PP \VS{14}我写这话,不是叫你们羞愧,乃是警戒你们,好像我所亲爱的儿女一样。
\VS{15}你们学基督的,师傅虽有一万,为父的却是不多,因我在基督耶稣里用福音生了你们。
\VS{16}所以,我求你们效法我。
\VS{17}因此我已打发{\PN{提摩太}}到你们那里去。他在主里面,是我所亲爱、有忠心的儿子。他必提醒你们,记念我在基督里怎样行事,在各处各教会中怎样教导人。
\VS{18}有些人自高自大,以为我不到你们那里去;
\VS{19}然而,主若许我,我必快到你们那里去,并且我所要知道的,不是那些自高自大之人的言语,乃是他们的权能。
\VS{20}因为 神的国不在乎言语,乃在乎权能。
\VS{21}你们愿意怎么样呢?是愿意我带着刑杖到你们那里去呢?还是要我存慈爱温柔的心呢?

\par }\Chap{5}{\SH 判断淫乱的事件
\par }{\PP \VerseOne{1}风闻在你们中间有淫乱的事。这样的淫乱连外邦人中也没有,就是有人收了他的继母。
\VS{2}你们还是自高自大,并不哀痛,把行这事的人从你们中间赶出去。
\VS{3}我身子虽不在你们那里,心却在你们那里,好像我亲自与你们同在,已经判断了行这事的人。
\VS{4}就是你们聚会的时候,我的心也同在。奉我们主耶稣的名,并用我们主耶稣的权能,
\VS{5}要把这样的人交给撒但,败坏他的肉体,使他的灵魂在主耶稣的日子可以得救。
\VS{6}你们这自夸是不好的。岂不知一点面酵能使全团发起来吗?
\VS{7}你们既是无酵的面,应当把旧酵除净,好使你们成为新团;因为我们逾越节的{\ADD{羔羊}}基督已经被杀献祭了。
\VS{8}所以,我们守这节不可用旧酵,也不可用恶毒\FTNT{}{{\FR 5:8: }或译:阴毒}、邪恶的酵,只用诚实真正的无酵饼。
\par }{\PP \VS{9}我先前写信给你们说,不可与淫乱的人相交。
\VS{10}此话不是指这世上一概行淫乱的,或贪婪的,勒索的,或拜偶像的;若是这样,你们除非离开世界方可。
\VS{11}但如今我写信给你们说,若有称为弟兄是行淫乱的,或贪婪的,或拜偶像的,或辱骂的,或醉酒的,或勒索的,这样的人不可与他相交,就是与他吃饭都不可。
\VS{12}因为审判{\ADD{教}}外的人与我何干?{\ADD{教}}内的人岂不是你们审判的吗?
\VS{13}至于外人有 神审判他们。你们应当把那恶人从你们中间赶出去。

\par }\Chap{6}{\SH 在不信主的人面前求审
\par }{\PP \VerseOne{1}你们中间有彼此相争的事,怎敢在不义的人面前求审,不在圣徒面前求审呢?
\VS{2}岂不知圣徒要审判世界吗?若世界为你们所审,难道你们不配审判这最小的事吗?
\VS{3}岂不知我们要审判天使吗?何况今生的事呢?
\VS{4}既是这样,你们若有今生的事当审判,是派教会所轻看的人审判吗?
\VS{5}我说这话是要叫你们羞耻。难道你们中间没有一个智慧人能审断弟兄们的事吗?
\VS{6}你们竟是弟兄与弟兄告状,而且告在不信{\ADD{主}}的人面前。
\VS{7}你们彼此告状,这已经是你们的大错了。为什么不情愿受欺呢?为什么不情愿吃亏呢?
\VS{8}你们倒是欺压人、亏负人,况且所欺压所亏负的就是弟兄。
\par }{\PP \VS{9}你们岂不知不义的人不能承受 神的国吗?不要自欺!无论是淫乱的、拜偶像的、奸淫的、作娈童的、亲男色的、
\VS{10}偷窃的、贪婪的、醉酒的、辱骂的、勒索的,都不能承受 神的国。
\VS{11}你们中间也有人从前是这样;但如今你们奉主耶稣基督的名,并借着我们 神的灵,已经洗净,成圣,称义了。
\par }{\SH 在身子上荣耀 神
\par }{\PP \VS{12}凡事我都可行,但不都有益处。凡事我都可行,但无论哪一件,我总不受它的辖制。
\VS{13}食物是为肚腹,肚腹是为食物;但 神要叫这两样都废坏。身子不是为淫乱,乃是为主;主也是为身子。
\VS{14}并且 神已经叫主复活,也要用自己的能力叫我们复活。
\VS{15}岂不知你们的身子是基督的肢体吗?我可以将基督的肢体作为娼妓的肢体吗?断乎不可!
\VS{16}岂不知与娼妓联合的,便是与她成为一体吗?因为主说:「二人要成为一体。」
\VS{17}但与主联合的,便是与主成为一灵。
\VS{18}你们要逃避淫行。人所犯的,无论什么罪,都在身子以外,惟有行淫的,是得罪自己的身子。
\VS{19}岂不知你们的身子就是圣灵的殿吗?这圣灵是从 神而来,住在你们里头的;并且你们不是自己的人,
\VS{20}因为你们是重价买来的。所以,要在你们的身子上荣耀 神。

\par }\Chap{7}{\SH 婚姻的问题
\par }{\PP \VerseOne{1}论到你们信上所提的事,{\ADD{我说}}男不近女倒好。
\VS{2}但要免淫乱的事,男子当各有自己的妻子;女子也当各有自己的丈夫。
\VS{3}丈夫当用合宜之分待妻子;妻子待丈夫也要如此。
\VS{4}妻子没有权柄主张自己的身子,乃在丈夫;丈夫也没有权柄主张自己的身子,乃在妻子。
\VS{5}夫妻不可彼此亏负,除非两相情愿,暂时分房,为要专心祷告方可;以后仍要同房,免得撒但趁着你们情不自禁,引诱你们。
\VS{6}我说这话,原是准{\ADD{你们}}的,不是命{\ADD{你们}}的。
\VS{7}我愿意众人像我一样;只是各人领受 神的恩赐,一个是这样,一个是那样。
\par }{\PP \VS{8}我对着没有嫁娶的和寡妇说,若他们常像我就好。
\VS{9}倘若自己禁止不住,就可以嫁娶。与其欲火攻心,倒不如嫁娶为妙。
\VS{10}至于那已经嫁娶的,我吩咐他们;其实不是我吩咐,乃是主吩咐说:妻子不可离开丈夫,
\VS{11}若是离开了,不可再嫁,或是仍同丈夫和好。丈夫也不可离弃妻子。
\VS{12}我对其余的人说(不是主说):倘若某弟兄有不信的妻子,妻子也情愿和他同住,他就不要离弃妻子。
\VS{13}妻子有不信的丈夫,丈夫也情愿和她同住,她就不要离弃丈夫。
\VS{14}因为不信的丈夫就因着妻子成了圣洁,并且不信的妻子就因着丈夫\FTNT{}{{\FR 7:14: }原文是弟兄}成了圣洁。不然,你们的儿女就不洁净,但如今他们是圣洁的了。
\VS{15}倘若那不信的人要离去,就由他离去吧!无论是弟兄,是姊妹,遇着这样的事都不必拘束。 神召我们原是要我们和睦。
\VS{16}你这作妻子的,怎么知道不能救你的丈夫呢?你这作丈夫的,怎么知道不能救你的妻子呢?
\par }{\SH 照主分给各人的去行
\par }{\PP \VS{17}只要照主所分给各人的,和 神所召各人的而行。我吩咐各教会都是这样。
\VS{18}有人已受割礼蒙召呢,就不要废割礼;有人未受割礼蒙召呢,就不要受割礼。
\VS{19}受割礼算不得什么,不受割礼也算不得什么,只要守 神的诫命就是了。
\VS{20}各人蒙召的时候是什么身分,仍要守住这身分。
\VS{21}你是作奴隶蒙召的吗?不要因此忧虑;若能以自由,就求自由更好。
\VS{22}因为作奴仆蒙召于主的,就是主所释放的人;作自由之人蒙召的,就是基督的奴仆。
\VS{23}你们是重价买来的,不要作人的奴仆。
\VS{24}弟兄们,你们各人蒙召的时候是什么身分,仍要在 神面前守住这身分。
\par }{\SH 未婚和寡居
\par }{\PP \VS{25}论到童身的人,我没有主的命令,但我既蒙主怜恤能作忠心的人,就把自己的意见告诉{\ADD{你们}}。
\VS{26}因现今的艰难,据我看来,人不如守素安常才好。
\VS{27}你有妻子缠着呢,就不要求脱离;你没有妻子缠着呢,就不要求妻子。
\VS{28}你若娶妻,并不是犯罪;处女若出嫁,也不是犯罪。然而这等人肉身必受苦难,我却愿意你们免这苦难。
\VS{29}弟兄们,我对你们说:时候减少了。从此以后,那有妻子的,要像没有妻子;
\VS{30}哀哭的,要像不哀哭;快乐的,要像不快乐;置买的,要像无有所得;
\VS{31}用世物的,要像不用世物,因为这世界的样子将要过去了。
\VS{32}我愿你们无所挂虑。没有娶妻的,是为主的事挂虑,想怎样叫主喜悦。
\VS{33}娶了妻的,是为世上的事挂虑,想怎样叫妻子喜悦。
\VS{34}妇人和处女也有分别。没有出嫁的,是为主的事挂虑,要身体、灵魂都圣洁;已经出嫁的,是为世上的事挂虑,想怎样叫丈夫喜悦。
\VS{35}我说这话是为你们的益处,不是要牢笼你们,乃是要叫你们行合宜的事,得以殷勤服事主,没有分心的事。
\par }{\PP \VS{36}若有人以为自己待他的女儿不合宜,女儿也过了年岁,事又当行,他就可随意办理,不算有罪,叫二人成亲就是了。
\VS{37}倘若人心里坚定,没有不得已的事,并且由得自己作主,心里又决定了留下女儿{\ADD{不出嫁}},如此行也好。
\VS{38}这样看来,叫自己的女儿出嫁是好,不叫她出嫁更是好。
\VS{39}丈夫活着的时候,妻子是被约束的;丈夫若死了,妻子就可以自由,随意再嫁,只是要嫁这在主里面的人。
\VS{40}然而按我的意见,若常守节更有福气。我也想自己是被 神的灵感动了。

\par }\Chap{8}{\SH 祭过偶像的食物
\par }{\PP \VerseOne{1}论到祭偶像之物,我们晓得我们都有知识。但知识是叫人自高自大,惟有爱心能造就人。
\VS{2}若有人以为自己知道什么,按他所当知道的,他仍是不知道。
\VS{3}若有人爱 神,这人乃是 神所知道的。
\VS{4}论到吃祭偶像之物,我们知道偶像在世上算不得什么,也知道 神只有一位,再没有别的 神。
\VS{5}虽有称为神的,或在天,或在地,就如那许多的神,许多的主;
\VS{6}然而我们只有一位 神,就是父—万物都本于他;我们也归于他—并有一位主,就是耶稣基督—万物都是借着他有的;我们也是借着他有的。
\par }{\PP \VS{7}但人不都有这等知识。有人到如今因拜惯了偶像,就以为所吃的是祭偶像之物。他们的良心既然软弱,也就污秽了。
\VS{8}其实食物不能叫 神看中我们,因为我们不吃也无损,吃也无益。
\VS{9}只是你们要谨慎,恐怕你们这自由竟成了那软弱人的绊脚石。
\VS{10}若有人见你这有知识的,在偶像的庙里坐席,这人的良心若是软弱,岂不放胆去吃那祭偶像之物吗?
\VS{11}因此,基督为他死的那软弱弟兄,也就因你的知识沉沦了。
\VS{12}你们这样得罪弟兄们,伤了他们软弱的良心,就是得罪基督。
\VS{13}所以,食物若叫我弟兄跌倒,我就永远不吃肉,免得叫我弟兄跌倒了。

\par }\Chap{9}{\SH 使徒的权利
\par }{\PP \VerseOne{1}我不是自由的吗?我不是使徒吗?我不是见过我们的主耶稣吗?你们不是我在主里面所做之工吗?
\VS{2}假若在别人,我不是使徒,在你们,我总是使徒,因为你们在主里正是我作使徒的印证。
\par }{\PP \VS{3}我对那盘问我的人就是这样分诉:
\VS{4}难道我们没有权柄{\ADD{靠福音}}吃喝吗?
\VS{5}难道我们没有权柄娶{\ADD{信主的}}姊妹为妻,带着一同往来,仿佛其余的使徒和主的弟兄并{\PN{矶法}}一样吗?
\VS{6}独有我与{\PN{巴拿巴}}没有权柄不做工吗?
\VS{7}有谁当兵自备粮饷呢?有谁栽葡萄园不吃园里的果子呢?有谁牧养牛羊不吃牛羊的奶呢?
\VS{8}我说这话,岂是照人的意见;律法不也是这样说吗?
\VS{9}就如{\PN{摩西}}的律法记着说:「牛{\ADD{在场上}}踹谷的时候,不可笼住它的嘴。」难道 神所挂念的是牛吗?
\VS{10}不全是为我们说的吗?分明是为我们说的。因为耕种的当存着指望去耕种;打场的也当存得{\ADD{粮}}的指望去打场。
\VS{11}我们若把属灵的种子撒在你们中间,就是从你们收割奉养肉身之物,这还算大事吗?
\VS{12}若别人在你们身上有这权柄,何况我们呢?
\par }{\PP 然而,我们没有用过这权柄,倒凡事忍受,免得基督的福音被阻隔。
\VS{13}你们岂不知为圣事劳碌的就吃殿中的物吗?伺候祭坛的就分领坛上的物吗?
\VS{14}主也是这样命定,叫传福音的靠着福音养生。
\VS{15}但这权柄我全没有用过。我写这话,并非要你们这样待我,因为我宁可死也不叫人使我所夸的落了空。
\VS{16}我传福音原没有可夸的,因为我是不得已的。若不传福音,我便有祸了。
\VS{17}我若甘心做这事,就有赏赐;若不甘心,责任却已经托付我了。
\VS{18}既是这样,我的赏赐是什么呢?就是我传福音的时候叫人不花钱得福音,免得用尽我传福音的权柄。
\par }{\PP \VS{19}我虽是自由的,无人辖管;然而我甘心作了众人的仆人,为要多得人。
\VS{20}向{\PN{犹太}}人,我就作{\PN{犹太}}人,为要得{\PN{犹太}}人;向律法以下的人,我虽不在律法以下,还是作律法以下的人,为要得律法以下的人。
\VS{21}向没有律法的人,我就作没有律法的人,为要得没有律法的人;其实我在 神面前,不是没有律法;在基督面前,正在律法之下。
\VS{22}向软弱的人,我就作软弱的人,为要得软弱的人。向什么样的人,我就作什么样的人。无论如何,总要救些人。
\VS{23}凡我所行的,都是为福音的缘故,为要与人同得这福音{\ADD{的好处}}。
\par }{\PP \VS{24}岂不知在场上赛跑的都跑,但得奖赏的只有一人?你们也当这样跑,好叫你们得着奖赏。
\VS{25}凡较力争胜的,诸事都有节制,他们不过是要得能坏的冠冕;我们却是要得不能坏的冠冕。
\VS{26}所以,我奔跑不像无定向的;我斗拳不像打空气的。
\VS{27}我是攻克己身,叫身服我,恐怕我传{\ADD{福音}}给别人,自己反被弃绝了。

\par }\Chap{10}{\SH 警戒拜偶像的事
\par }{\PP \VerseOne{1}弟兄们,我不愿意你们不晓得,我们的祖宗从前都在云下,都从海中经过,
\VS{2}都在云里、海里受洗归了{\PN{摩西}};
\VS{3}并且都吃了一样的灵食,
\VS{4}也都喝了一样的灵水。所喝的,是出于随着他们的灵磐石;那磐石就是基督。
\VS{5}但他们中间多半是 神不喜欢的人,所以在旷野倒毙。
\VS{6}这些事都是我们的鉴戒,叫我们不要贪恋恶事,像他们那样贪恋的;
\VS{7}也不要拜偶像,像他们有人拜的。如{\ADD{经上}}所记:「百姓坐下吃喝,起来玩耍。」
\VS{8}我们也不要行奸淫,像他们有人行的,一天就倒毙了二万三千人;
\VS{9}也不要试探主\FTNT{}{{\FR 10:9: }有古卷:基督},像他们有人试探的,就被蛇所灭。
\VS{10}你们也不要发怨言,像他们有发怨言的,就被灭命的所灭。
\VS{11}他们遭遇这些事都要作为鉴戒,并且写{\ADD{在经上}},正是警戒我们这末世的人。
\VS{12}所以,自己以为站得稳的,须要谨慎,免得跌倒。
\VS{13}你们所遇见的试探,无非是人所能受的。 神是信实的,必不叫你们受试探过于所能受的;在受试探的时候,总要给你们开一条出路,叫你们能忍受得住。
\par }{\PP \VS{14}我所亲爱的{\ADD{弟兄}}啊,你们要逃避拜偶像的事。
\VS{15}我好像对明白人说的,你们要审察我的话。
\VS{16}我们所祝福的杯,岂不是同领基督的血吗?我们所擘开的饼,岂不是同领基督的身体吗?
\VS{17}我们虽多,仍是一个饼,一个身体,因为我们都是分受这一个饼。
\VS{18}你们看属肉体的{\PN{以色列}}人,那吃祭物的岂不是在祭坛上有分吗?
\VS{19}我是怎么说呢?岂是说祭偶像之物算得什么呢?或说偶像算得什么呢?
\VS{20}我乃是说,外邦人所献的祭是祭鬼,不是祭 神。我不愿意你们与鬼相交。
\VS{21}你们不能喝主的杯又喝鬼的杯,不能吃主的筵席又吃鬼的筵席。
\VS{22}我们可惹主的愤恨吗?我们比他还有能力吗?
\par }{\SH 凡事为荣耀 神而行
\par }{\PP \VS{23}凡事都可行,但不都有益处。凡事都可行,但不都造就人。
\VS{24}无论何人,不要求自己的益处,乃要求别人的益处。
\VS{25}凡市上所卖的,你们只管吃,不要为良心的缘故问什么话,
\VS{26}因为地和其中所充满的都属乎主。
\VS{27}倘有一个不信的人请你们{\ADD{赴席}},你们若愿意去,凡摆在你们面前的,只管吃,不要为良心的缘故问什么话。
\VS{28}若有人对你们说:「这是献过祭的物」,就要为那告诉你们的人,并为良心的缘故不吃。
\VS{29}我说的良心不是你的,乃是他的。我这自由为什么被别人的良心论断呢?
\VS{30}我若谢恩而吃,为什么因我谢恩的物被人毁谤呢?
\VS{31}所以,你们或吃或喝,无论做什么,都要为荣耀 神而行。
\VS{32}不拘是{\PN{犹太}}人,是{\PN{希腊}}人,是 神的教会,你们都不要使他跌倒;
\VS{33}就好像我凡事都叫众人喜欢,不求自己的益处,只求众人的益处,叫他们得救。

\par }\Chap{11}{\PP \VerseOne{1}你们该效法我,像我效法基督一样。
\par }{\SH 礼拜时蒙头的问题
\par }{\PP \VS{2}我称赞你们,因你们凡事记念我,又坚守我所传给你们的。
\VS{3}我愿意你们知道,基督是各人的头;男人是女人的头; 神是基督的头。
\VS{4}凡男人祷告或是讲道\FTNT{}{{\FR 11:4: }或译:说预言;下同},若蒙着头,就羞辱自己的头。
\VS{5}凡女人祷告或是讲道,若不蒙着头,就羞辱自己的头,因为这就如同剃了头发一样。
\VS{6}女人若不蒙着头,就该剪了头发;女人若以剪发、剃发为羞愧,就该蒙着头。
\VS{7}男人本不该蒙着头,因为他是 神的形象和荣耀;但女人是男人的荣耀。
\VS{8}起初,男人不是由女人而出,女人乃是由男人而出;
\VS{9}并且男人不是为女人造的,女人乃是为男人造的。
\VS{10}因此,女人为天使的缘故,应当在头上有服权柄的{\ADD{记号}}。
\VS{11}然而照主的安排,女也不是无男,男也不是无女。
\VS{12}因为女人原是由男人而出,男人也是由女人而出;但万有都是出乎 神。
\VS{13}你们自己审察,女人祷告 神,不蒙着头是合宜的吗?
\VS{14}你们的本性不也指示你们,男人若有长头发,便是他的羞辱吗?
\VS{15}但女人有长头发,乃是她的荣耀,因为这头发是给她作盖头的。
\VS{16}若有人想要辩驳,我们却没有这样的规矩, 神的众教会也是没有的。
\par }{\SH 圣餐时的混乱
\par }{\PP \VS{17}我现今吩咐你们的话,不是称赞你们;因为你们聚会不是受益,乃是招损。
\VS{18}第一,我听说,你们聚会的时候彼此分门别类,我也稍微地信这话。
\VS{19}在你们中间不免有分门结党的事,好叫那些有经验的人显明出来。
\VS{20}你们聚会的时候,算不得吃主的晚餐;
\VS{21}因为吃的时候,各人先吃自己的饭,甚至这个饥饿,那个酒醉。
\VS{22}你们要吃喝,难道没有家吗?还是藐视 神的教会,叫那没有的羞愧呢?我向你们可怎么说呢?可因此称赞你们吗?我不称赞!
\par }{\SH 圣餐的设立
\par }{\PP \VS{23}我当日传给你们的,原是从主领受的,就是主耶稣被卖的那一夜,拿起饼来,
\VS{24}祝谢了,就擘开,说:「这是我的身体,为你们舍\FTNT{}{{\FR 11:24: }有古卷:擘开}的,你们应当如此行,为的是记念我。」
\VS{25}饭后,也照样拿起杯来,说:「这杯是用我的血所立的新约,你们每逢喝的时候,要如此行,为的是记念我。」
\VS{26}你们每逢吃这饼,喝这杯,是表明主的死,直等到他来。
\par }{\SH 不按理吃主的晚餐
\par }{\PP \VS{27}所以,无论何人,不按理吃主的饼,喝主的杯,就是干犯主的身、主的血了。
\VS{28}人应当自己省察,然后吃这饼、喝这杯。
\VS{29}因为人吃喝,若不分辨是{\ADD{主的}}身体,就是吃喝自己的罪了。
\VS{30}因此,在你们中间有好些软弱的与患病的,死\FTNT{}{{\FR 11:30: }原文是睡}的也不少。
\VS{31}我们若是先分辨自己,就不至于受审。
\VS{32}我们受审的时候,乃是被主惩治,免得我们和世人一同定罪。
\VS{33}所以我弟兄们,你们聚会吃的时候,要彼此等待。
\VS{34}若有人饥饿,可以在家里先吃,免得你们聚会,自己取罪。其余的事,我来的时候再安排。

\par }\Chap{12}{\SH 属灵的恩赐
\par }{\PP \VerseOne{1}弟兄们,论到属灵的{\ADD{恩赐}},我不愿意你们不明白。
\VS{2}你们作外邦人的时候,随事被牵引,受迷惑,去服事那哑巴偶像,这是你们知道的。
\VS{3}所以我告诉你们,被 神的灵感动的,没有说「耶稣是可咒诅」的;若不是被{\ADD{圣}}灵感动的,也没有能说「耶稣是主」的。
\par }{\PP \VS{4}恩赐原有分别,{\ADD{圣}}灵却是一位。
\VS{5}职事也有分别,主却是一位。
\VS{6}功用也有分别, 神却是一位,在众人里面运行一切的事。
\VS{7}{\ADD{圣}}灵显在各人身上,是叫人得益处。
\VS{8}这人蒙{\ADD{圣}}灵赐他智慧的言语,那人也蒙这位{\ADD{圣}}灵赐他知识的言语,
\VS{9}又有一人蒙这位{\ADD{圣}}灵赐他信心,还有一人蒙这位{\ADD{圣}}灵赐他医病的恩赐,
\VS{10}又叫一人能行异能,又叫一人能作先知,又叫一人能辨别诸灵,又叫一人能说方言,又叫一人能翻方言。
\VS{11}这一切都是这位{\ADD{圣}}灵所运行、随己意分给各人的。
\par }{\SH 一个身体有许多肢体
\par }{\PP \VS{12}就如身子是一个,却有许多肢体;而且肢体虽多,仍是一个身子;基督也是这样。
\VS{13}我们不拘是{\PN{犹太}}人,是{\PN{希腊}}人,是为奴的,是自主的,都从一位{\ADD{圣}}灵受洗,成了一个身体,饮于一位{\ADD{圣}}灵。
\VS{14}身子原不是一个肢体,乃是许多肢体。
\VS{15}设若脚说:「我不是手,所以不属乎身子,」它不能因此就不属乎身子。
\VS{16}设若耳说:「我不是眼,所以不属乎身子,」它也不能因此就不属乎身子。
\VS{17}若全身是眼,从哪里听声呢?若全身是耳,从哪里闻味呢?
\VS{18}但如今, 神随自己的意思把肢体俱各安排在身上了。
\VS{19}若都是一个肢体,身子在哪里呢?
\VS{20}但如今肢体是多的,身子却是一个。
\VS{21}眼不能对手说:「我用不着你」;头也不能对脚说:「我用不着你。」
\VS{22}不但如此,身上肢体人以为软弱的,更是不可少的。
\VS{23}身上肢体,我们看为不体面的,越发给它加上体面;不俊美的,越发得着俊美。
\VS{24}我们俊美的肢体,自然用不着装饰;但 神配搭这身子,把加倍的体面给那有缺欠的肢体,
\VS{25}免得身上分门别类,总要肢体彼此相顾。
\VS{26}若一个肢体受苦,所有的肢体就一同受苦;若一个肢体得荣耀,所有的肢体就一同快乐。
\VS{27}你们就是基督的身子,并且各自作肢体。
\VS{28}神在教会所设立的:第一是使徒,第二是先知,第三是教师,其次是行异能的,再次是得恩赐医病的,帮助人的,治理事的,说方言的。
\VS{29}岂都是使徒吗?岂都是先知吗?岂都是教师吗?岂都是行异能的吗?
\VS{30}岂都是得恩赐医病的吗?岂都是说方言的吗?岂都是翻方言的吗?
\VS{31}你们要切切地求那更大的恩赐。
\par }{\SH 爱
\par }{\PP 我现今把最妙的道指示你们。

\par }\Chap{13}{\PP \VerseOne{1}我若能说万人的方言,并天使的话语,却没有爱,我就成了鸣的锣,响的钹一般。
\VS{2}我若有先知讲道{\ADD{之能}},也明白各样的奥秘,各样的知识,而且有全备的信,叫我能够移山,却没有爱,我就算不得什么。
\VS{3}我若将所有的周济{\ADD{穷人}},又舍己身叫人焚烧,却没有爱,仍然与我无益。
\par }{\PP \VS{4}爱是恒久忍耐,又有恩慈;爱是不嫉妒;爱是不自夸,不张狂,
\VS{5}不做害羞的事,不求自己的益处,不轻易发怒,不计算{\ADD{人的}}恶,
\VS{6}不喜欢不义,只喜欢真理;
\VS{7}凡事包容,凡事相信,凡事盼望,凡事忍耐。
\par }{\PP \VS{8}爱是永不止息。先知讲道之能终必归于无有;说方言之能终必停止;知识也终必归于无有。
\VS{9}我们现在所知道的有限,先知所讲的也有限,
\VS{10}等那完全的来到,这有限的必归于无有了。
\VS{11}我作孩子的时候,话语像孩子,心思像孩子,意念像孩子,既成了人,就把孩子的事丢弃了。
\VS{12}我们如今仿佛对着镜子观看,模糊不清\FTNT{}{{\FR 13:12: }原文是如同猜谜},到那时就要面对面了。我如今所知道的有限,到那时就全知道,如同主知道我一样。
\par }{\PP \VS{13}如今常存的有信,有望,有爱这三样,其中最大的是爱。

\par }\Chap{14}{\SH 说方言和作先知讲道
\par }{\PP \VerseOne{1}你们要追求爱,也要切慕属灵的{\ADD{恩赐}},其中更要羡慕的,是作先知讲道\FTNT{}{{\FR 14:1: }原文作:是说预言;下同}。
\VS{2}那说方言的,原不是对人说,乃是对 神说,因为没有人听出来。然而,他在心灵里却是讲说各样的奥秘。
\VS{3}但作先知讲道的,是对人说,要造就、安慰、劝勉人。
\VS{4}说方言的,是造就自己;作先知讲道的,乃是造就教会。
\VS{5}我愿意你们都说方言,更愿意你们作先知讲道;因为说方言的,若不翻出来,使教会被造就,那作先知讲道的,就比他强了。
\par }{\PP \VS{6}弟兄们,我到你们那里去,若只说方言,不用启示,或知识,或预言,或教训,给你们讲解,我与你们有什么益处呢?
\VS{7}就是那有声无气的物,或箫,或琴,若发出来的声音没有分别,怎能知道所吹所弹的是什么呢?
\VS{8}若吹无定的号声,谁能预备打仗呢?
\VS{9}你们也是如此。舌头若不说容易明白的话,怎能知道所说的是什么呢?这就是向空说话了。
\VS{10}世上的声音,或者甚多,却没有一样是无意思的。
\VS{11}我若不明白那声音的意思,这说话的人必以我为化外之人,我也以他为化外之人。
\VS{12}你们也是如此,既是切慕属灵的{\ADD{恩赐}},就当求多得造就教会的恩赐。
\VS{13}所以那说方言的,就当求着能翻出来。
\VS{14}我若用方言祷告,是我的灵祷告,但我的悟性没有果效。
\VS{15}这却怎么样呢?我要用灵祷告,也要用悟性祷告;我要用灵歌唱,也要用悟性歌唱。
\VS{16}不然,你用灵祝谢,那在座不通方言的人,既然不明白你的话,怎能在你感谢的时候说「阿们」呢?
\VS{17}你感谢的固然是好,无奈不能造就别人。
\VS{18}我感谢 神,我说方言比你们众人还多。
\VS{19}但在教会中,宁可用悟性说五句教导人的话,强如说万句方言。
\par }{\PP \VS{20}弟兄们,在心志上不要作小孩子。然而,在恶事上要作婴孩,在心志上总要作大人。
\VS{21}律法上记着:
\par }{\Q 主说:我要用外邦人的舌头
\par }{\Q 和外邦人的嘴唇向这百姓说话;
\par }{\Q 虽然如此,
\par }{\Q 他们还是不听从我。
\par }{\MM \VS{22}这样看来,说方言不是为信的人作证据,乃是为不信的人;作先知讲道不是为不信的人{\ADD{作证据}},乃是为信的人。
\VS{23}所以,全教会聚在一处的时候,若都说方言,偶然有不通方言的,或是不信的人进来,岂不说你们癫狂了吗?
\VS{24}若都作先知讲道,偶然有不信的,或是不通方言的人进来,就被众人劝醒,被众人审明,
\VS{25}他心里的隐情显露出来,就必将脸伏地,敬拜 神,说:「 神真是在你们中间了。」
\par }{\SH 凡事都要按次序行
\par }{\PP \VS{26}弟兄们,这却怎么样呢?你们聚会的时候,各人或有诗歌,或有教训,或有启示,或有方言,或有翻出来的话,凡事都当造就人。
\VS{27}若有说方言的,只好两个人,至多三个人,且要轮流着说,也要一个人翻出来。
\VS{28}若没有人翻,就当在会中闭口,只对自己和 神说就是了。
\VS{29}至于作先知讲道的,只好两个人或是三个人,其余的就当{\ADD{慎思}}明辨。
\VS{30}若旁边坐着的得了启示,那先说话的就当闭口不言。
\VS{31}因为你们都可以一个一个地作先知讲道,叫众人学道理,叫众人得劝勉。
\VS{32}先知的灵原是顺服先知的;
\VS{33}因为 神不是叫人混乱,乃是叫人安静。
\par }{\PP \VS{34}妇女在会中要闭口不言,像在圣徒的众教会一样,因为不准她们说话。她们总要顺服,正如律法所说的。
\VS{35}她们若要学什么,可以在家里问自己的丈夫,因为妇女在会中说话原是可耻的。
\VS{36}神的道理岂是从你们出来吗?岂是单临到你们吗?
\par }{\PP \VS{37}若有人以为自己是先知,或是属灵的,就该知道,我所写给你们的是主的命令。
\VS{38}若有不知道的,就由他不知道吧!
\VS{39}所以我弟兄们,你们要切慕作先知讲道,也不要禁止说方言。
\VS{40}凡事都要规规矩矩地按着次序行。

\par }\Chap{15}{\SH 基督的复活
\par }{\PP \VerseOne{1}弟兄们,我如今把先前所传给你们的福音告诉你们知道;这福音你们也领受了,又靠着站立得住,
\VS{2}并且你们若不是徒然相信,能以持守我所传给你们的,就必因这福音得救。
\VS{3}我当日所领受又传给你们的:第一,就是基督照圣经所说,为我们的罪死了,
\VS{4}而且埋葬了;又照圣经所说,第三天复活了,
\VS{5}并且显给{\PN{矶法}}看,然后显给十二使徒看;
\VS{6}后来一时显给五百多弟兄看,其中一大半到如今还在,却也有已经睡了的。
\VS{7}以后显给{\PN{雅各}}看,再显给众使徒看,
\VS{8}末了也显给我看;我如同未到产期而生的人一般。
\VS{9}我原是使徒中最小的,不配称为使徒,因为我从前逼迫 神的教会。
\VS{10}然而,我今日成了何等人,是蒙 神的恩才成的,并且他所赐我的恩不是徒然的。我比众使徒格外劳苦;这原不是我,乃是 神的恩与我同在。
\VS{11}不拘是我,是众使徒,我们如此传,你们也如此信了。
\par }{\SH 死人的复活
\par }{\PP \VS{12}既传基督是从死里复活了,怎么在你们中间有人说没有死人复活的事呢?
\VS{13}若没有死人复活的事,基督也就没有复活了。
\VS{14}若基督没有复活,我们所传的便是枉然,你们所信的也是枉然;
\VS{15}并且明显我们是为 神妄作见证的,因我们见证 神是叫基督复活了。若死人真不复活, 神也就没有叫基督复活了。
\VS{16}因为死人若不复活,基督也就没有复活了。
\VS{17}基督若没有复活,你们的信便是徒然,你们仍在罪里。
\VS{18}就是在基督里睡了的人也灭亡了。
\VS{19}我们若靠基督只在今生有指望,就算比众人更可怜。
\par }{\PP \VS{20}但基督已经从死里复活,成为睡了之人初熟的果子。
\VS{21}死既是因一人而来,死人复活也是因一人而来。
\VS{22}在{\PN{亚当}}里众人都死了;照样,在基督里众人也都要复活。
\VS{23}但各人是按着自己的次序复活:初熟的果子是基督;以后,在他来的时候,是那些属基督的。
\VS{24}再后,末期到了,那时基督既将一切执政的、掌权的、有能的都毁灭了,就把国交与父 神。
\VS{25}因为基督必要作王,等 神把一切仇敌都放在他的脚下。
\VS{26}尽末了所毁灭的仇敌就是死。
\VS{27}因为{\ADD{经上}}说:「 神叫万物都服在他的脚下。」既说万物都服了他,明显那叫万物服他的,不在其内了。
\VS{28}万物既服了他,那时子也要自己服那叫万物服他的,叫 神在万物之上,为万物之主。
\par }{\PP \VS{29}不然,那些为死人受洗的,将来怎样呢?若死人总不复活,因何为他们受洗呢?
\VS{30}我们又因何时刻冒险呢?
\VS{31}弟兄们,我在我主基督耶稣里,指着你们所夸的口极力地说,我是天天{\ADD{冒}}死。
\VS{32}我若当日像寻常人,在{\PN{以弗所}}同野兽战斗,那于我有什么益处呢?若死人不复活,
\par }{\Q 我们就吃吃喝喝吧!
\par }{\Q 因为明天要死了。
\par }{\MM \VS{33}你们不要自欺;滥交是败坏善行。
\VS{34}你们要醒悟为善,不要犯罪,因为有人不认识 神。我说这话是要叫你们羞愧。
\par }{\SH 复活的身体
\par }{\PP \VS{35}或有人问:「死人怎样复活,带着什么身体来呢?」
\VS{36}无知的人哪,你所种的,若不死就不能生。
\VS{37}并且你所种的不是那将来的形体,不过是子粒,即如麦子,或是别样的谷。
\VS{38}但 神随自己的意思给他一个形体,并叫各等子粒各有自己的形体。
\VS{39}凡肉体各有不同:人是一样,兽又是一样,鸟又是一样,鱼又是一样。
\VS{40}有天上的形体,也有地上的形体;但天上形体的荣光是一样,地上形体的荣光又是一样。
\VS{41}日有日的荣光,月有月的荣光,星有星的荣光;这星和那星的荣光也有分别。
\par }{\PP \VS{42}死人复活也是这样:所种的是必朽坏的,复活的是不朽坏的;
\VS{43}所种的是羞辱的,复活的是荣耀的;所种的是软弱的,复活的是强壮的;
\VS{44}所种的是血气的身体,复活的是灵性的身体。若有血气的身体,也必有灵性的身体。
\VS{45}{\ADD{经上}}也是这样记着说:「首先的人{\PN{亚当}}成了有灵\FTNT{}{{\FR 15:45: }灵:或译血气}的活人」;末后的{\PN{亚当}}成了叫人活的灵。
\VS{46}但属灵的不在先,属血气的在先,以后才有属灵的。
\VS{47}头一个人是出于地,乃属土;第二个人是出于天。
\VS{48}那属土的怎样,凡属土的也就怎样;属天的怎样,凡属天的也就怎样。
\VS{49}我们既有属土的形状,将来也必有属天的形状。
\par }{\PP \VS{50}弟兄们,我告诉你们说,血肉{\ADD{之体}}不能承受 神的国,必朽坏的不能承受不朽坏的。
\VS{51}我如今把一件奥秘的事告诉你们:我们不是都要睡觉,乃是都要改变,
\VS{52}就在一霎时,眨眼之间,号筒末次吹响的时候。因号筒要响,死人要复活成为不朽坏的,我们也要改变。
\VS{53}这必朽坏的总要变成\FTNT{}{{\FR 15:53: }变成:原文是穿;下同}不朽坏的,这必死的总要变成不死的。
\VS{54}这必朽坏的既变成不朽坏的,这必死的既变成不死的,那时{\ADD{经上}}所记「死被得胜吞灭」的话就应验了。
\par }{\Q \VS{55}死啊!你得胜的权势在哪里?
\par }{\Q 死啊!你的毒钩在哪里?
\par }{\MM \VS{56}死的毒钩就是罪,罪的权势就是律法。
\VS{57}感谢 神,使我们借着我们的主耶稣基督得胜。
\VS{58}所以,我亲爱的弟兄们,你们务要坚固,不可摇动,常常竭力多做主工;因为知道,你们的劳苦在主里面不是徒然的。

\par }\Chap{16}{\SH 捐助圣徒
\par }{\PP \VerseOne{1}论到为圣徒捐钱,我从前怎样吩咐{\PN{加拉太}}的众教会,你们也当怎样行。
\VS{2}每逢七日的第一日,各人要照自己的进项抽出来留着,免得我来的时候现凑。
\VS{3}及至我来到了,你们写信举荐谁,我就打发他们,把你们的捐资送到{\PN{耶路撒冷}}去。
\VS{4}若我也该去,他们可以和我同去。
\par }{\SH 行程的计划
\par }{\PP \VS{5}我要从{\PN{马其顿}}经过;既经过了,就要到你们那里去,
\VS{6}或者和你们同住几时,或者也过冬。无论我往哪里去,你们就可以给我送行。
\VS{7}我如今不愿意路过见你们;主若许我,我就指望和你们同住几时。
\VS{8}但我要仍旧住在{\PN{以弗所}},直等到五旬节;
\VS{9}因为有宽大又有功效的门为我开了,并且反对的人也多。
\par }{\PP \VS{10}若是{\PN{提摩太}}来到,你们要留心,叫他在你们那里无所惧怕;因为他劳力做主的工,像我一样。
\VS{11}所以,无论谁都不可藐视他,只要送他平安前行,叫他到我这里来,因我指望他和弟兄们同来。
\VS{12}至于兄弟{\PN{亚波罗}},我再三地劝他同弟兄们到你们那里去;但这时他决不愿意去,几时有了机会他必去。
\par }{\SH 请求和问安
\par }{\PP \VS{13}你们务要警醒,在真道上站立得稳,要作大丈夫,要刚强。
\VS{14}凡你们所做的都要凭爱心而做。
\par }{\PP \VS{15}弟兄们,你们晓得{\PN{司提反}}一家,是{\PN{亚该亚}}初结的果子,并且他们专以服事圣徒为念。
\VS{16}我劝你们顺服这样的人,并一切同工同劳的人。
\VS{17}{\PN{司提反}}和{\PN{福徒拿都}},并{\PN{亚该古}}到这里来,我很喜欢;因为你们待我有不及之处,他们补上了。
\VS{18}他们叫我和你们心里都快活。这样的人,你们务要敬重。
\par }{\PP \VS{19}{\PN{亚细亚}}的众教会问你们安。{\PN{亚居拉}}和{\PN{百基拉}}并在他们家里的教会,因主多多地问你们安。
\VS{20}众弟兄都问你们安。你们要亲嘴问安,彼此务要圣洁。
\par }{\PP \VS{21}我—{\PN{保罗}}亲笔问安。
\VS{22}若有人不爱主,这人可诅可咒。主必要来!
\VS{23}愿主耶稣基督的恩{\ADD{常}}与你们众人同在!
\VS{24}我在基督耶稣里的爱与你们众人同在。阿们!
\par }