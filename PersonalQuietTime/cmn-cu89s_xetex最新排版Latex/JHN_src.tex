\NormalFont\ShortTitle{约翰福音}
{\MT 约翰福音

\par }\ChapOne{1}{\SH 道成肉身
\par }{\PP \VerseOne{1}太初有道,道与 神同在,道就是 神。
\VS{2}这道太初与 神同在。
\VS{3}万物是借着他造的;凡被造的,没有一样不是借着他造的。
\VS{4}生命在他里头,这生命就是人的光。
\VS{5}光照在黑暗里,黑暗却不接受光。
\par }{\PP \VS{6}有一个人,是从 神那里差来的,名叫{\PN{约翰}}。
\VS{7}这人来,为要作见证,就是为光作见证,叫众人因他可以信。
\VS{8}他不是那光,乃是要为光作见证。
\VS{9}那光是真光,照亮一切生在世上的人。
\VS{10}他在世界,世界也是借着他造的,世界却不认识他。
\VS{11}他到自己的地方来,自己的人倒不接待他。
\VS{12}凡接待他的,就是信他名的人,他就赐他们权柄作 神的儿女。
\VS{13}这等人不是从血气生的,不是从情欲生的,也不是从人意生的,乃是从 神生的。
\par }{\PP \VS{14}道成了肉身,住在我们中间,充充满满地有恩典有真理。我们也见过他的荣光,正是父独生子的荣光。
\VS{15}{\PN{约翰}}为他作见证,喊着说:「这就是我曾说:『那在我以后来的,反成了在我以前的,因他本来在我以前。』」
\VS{16}从他丰满{\ADD{的恩典}}里,我们都领受了,而且恩上加恩。
\VS{17}律法本是借着{\PN{摩西}}传的;恩典和真理都是由耶稣基督来的。
\VS{18}从来没有人看见 神,只有在父怀里的独生子将他表明出来。
\par }{\SH 施洗约翰的见证
\par }{\R (太3·1—12;可1·1—8;路3·1—18)
\par }{\PP \VS{19}{\PN{约翰}}所作的见证记在下面:{\PN{犹太}}人从{\PN{耶路撒冷}}差祭司和{\PN{利未}}人到{\PN{约翰}}那里,问他说:「你是谁?」
\VS{20}他就明说,并不隐瞒,明说:「我不是基督。」
\VS{21}他们又问他说:「这样,你是谁呢?是{\PN{以利亚}}吗?」他说:「我不是。」「是那先知吗?」他回答说:「不是。」
\VS{22}于是他们说:「你到底是谁,叫我们好回复差我们来的人。你自己说,你是谁?」
\VS{23}他说:「我就是那在旷野有人声喊着说:『修直主的道路』,正如先知{\PN{以赛亚}}所说的。」
\VS{24}那些人是法利赛人差来的\FTNT{}{{\FR 1:24: }或译:那差来的是法利赛人};
\VS{25}他们就问他说:「你既不是基督,不是{\PN{以利亚}},也不是那先知,为什么施洗呢?」
\VS{26}{\PN{约翰}}回答说:「我是用水施洗,但有一位站在你们中间,是你们不认识的,
\VS{27}就是那在我以后来的,我给他解鞋带也不配。」
\VS{28}这是在{\PN{约旦河}}外{\PN{伯大尼}}\FTNT{}{{\FR 1:28: }有古卷:伯大巴喇},{\PN{约翰}}施洗的地方作的见证。
\par }{\SH  神的羔羊
\par }{\PP \VS{29}次日,{\PN{约翰}}看见耶稣来到他那里,就说:「看哪, 神的羔羊,除去\FTNT{}{{\FR 1:29: }或译:背负}世人罪孽的!
\VS{30}这就是我曾说:『有一位在我以后来、反成了在我以前的,因他本来在我以前。』
\VS{31}我先前不认识他,如今我来用水施洗,为要叫他显明给{\PN{以色列}}人。」
\VS{32}{\PN{约翰}}又作见证说:「我曾看见圣灵,仿佛鸽子从天降下,住在他的身上。
\VS{33}我先前不认识他,只是那差我来用水施洗的、对我说:『你看见圣灵降下来,住在谁的身上,谁就是用圣灵施洗的。』
\VS{34}我看见了,就证明这是 神的儿子。」
\par }{\SH 初次选召门徒
\par }{\PP \VS{35}再次日,{\PN{约翰}}同两个门徒站在那里。
\VS{36}他见耶稣行走,就说:「看哪,这是 神的羔羊!」
\VS{37}两个门徒听见他的话,就跟从了耶稣。
\VS{38}耶稣转过身来,看见他们跟着,就问他们说:「你们要什么?」他们说:「拉比,在哪里住?」(拉比翻出来就是夫子。)
\VS{39}耶稣说:「你们来看。」他们就去看他在哪里住,这一天便与他同住;那时约有申正了。
\VS{40}听见{\PN{约翰}}的话跟从耶稣的那两个人,一个是{\PN{西门·彼得}}的兄弟{\PN{安得烈}}。
\VS{41}他先找着自己的哥哥{\PN{西门}},对他说:「我们遇见弥赛亚了。」(弥赛亚翻出来就是基督。)
\VS{42}于是领他去见耶稣。耶稣看着他,说:「你是{\PN{约翰}}的儿子{\PN{西门}}\FTNT{}{{\FR 1:42: }约翰在马太十六章十七节称约拿},你要称为{\PN{矶法}}。」({\PN{矶法}}翻出来就是{\PN{彼得}}。)
\par }{\SH 呼召腓力和拿但业
\par }{\PP \VS{43}又次日,耶稣想要往{\PN{加利利}}去,遇见{\PN{腓力}},就对他说:「来跟从我吧。」
\VS{44}这{\PN{腓力}}是{\PN{伯赛大}}人,和{\PN{安得烈}}、{\PN{彼得}}同城。
\VS{45}{\PN{腓力}}找着{\PN{拿但业}},对他说:「{\PN{摩西}}在律法上所写的和众先知所记的那一位,我们遇见了,就是{\PN{约瑟}}的儿子{\PN{拿撒勒}}人耶稣。」
\VS{46}{\PN{拿但业}}对他说:「{\PN{拿撒勒}}还能出什么好的吗?」{\PN{腓力}}说:「你来看!」
\VS{47}耶稣看见{\PN{拿但业}}来,就指着他说:「看哪,这是个真{\PN{以色列}}人,他心里是没有诡诈的。」
\VS{48}{\PN{拿但业}}对耶稣说:「你从哪里知道我呢?」耶稣回答说:「{\PN{腓力}}还没有招呼你,你在无花果树底下,我就看见你了。」
\VS{49}{\PN{拿但业}}说:「拉比,你是 神的儿子,你是{\PN{以色列}}的王!」
\VS{50}耶稣对他说:「因为我说『在无花果树底下看见你』,你就信吗?你将要看见比这更大的事」;
\VS{51}又说:「我实实在在地告诉你们,你们将要看见天开了, 神的使者上去下来在人子身上。」

\par }\Chap{2}{\SH 迦拿的婚礼
\par }{\PP \VerseOne{1}第三日,在{\PN{加利利}}的{\PN{迦拿}}有娶亲的筵席,耶稣的母亲在那里。
\VS{2}耶稣和他的门徒也被请去赴席。
\VS{3}酒用尽了,耶稣的母亲对他说:「他们没有酒了。」
\VS{4}耶稣说:「母亲\FTNT{}{{\FR 2:4: }原文是妇人},我与你有什么相干?我的时候还没有到。」
\VS{5}他母亲对用人说:「他告诉你们什么,你们就做什么。」
\VS{6}照{\PN{犹太}}人洁净的规矩,有六口石缸摆在那里,每口可以盛两三桶水。
\VS{7}耶稣对用人说:「把缸倒满了水。」他们就倒满了,直到缸口。
\VS{8}耶稣又说:「现在可以舀出来,送给管筵席的。」他们就送了去。
\VS{9}管筵席的尝了那水变的酒,并不知道是哪里来的,只有舀水的用人知道。管筵席的便叫新郎来,
\VS{10}对他说:「人都是先摆上好酒,等客喝足了,才摆上次的,你倒把好酒留到如今!」
\VS{11}这是耶稣所行的头一件神迹,是在{\PN{加利利}}的{\PN{迦拿}}行的,显出他的荣耀来;他的门徒就信他了。
\par }{\PP \VS{12}这事以后,耶稣与他的母亲、弟兄,和门徒都下{\PN{迦百农}}去,在那里住了不多几日。
\par }{\SH 洁净圣殿
\par }{\R (太21·12—13;可11·15—17;路19·45—46)
\par }{\PP \VS{13}{\PN{犹太}}人的逾越节近了,耶稣就上{\PN{耶路撒冷}}去。
\VS{14}看见殿里有卖牛、羊、鸽子的,并有兑换银钱的人坐在那里,
\VS{15}耶稣就拿绳子做成鞭子,把牛羊都赶出殿去,倒出兑换银钱之人的银钱,推翻他们的桌子,
\VS{16}又对卖鸽子的说:「把这些东西拿去!不要将我父的殿当作买卖的地方。」
\VS{17}他的门徒就想起{\ADD{经上}}记着说:「我为你的殿心里焦急,如同火烧。」
\VS{18}因此{\PN{犹太}}人问他说:「你既做这些事,还显什么神迹给我们看呢?」
\VS{19}耶稣回答说:「你们拆毁这殿,我三日内要再建立起来。」
\VS{20}{\PN{犹太}}人便说:「这殿是四十六年才造成的,你三日内就再建立起来吗?」
\VS{21}但耶稣这话是以他的身体为殿。
\VS{22}所以到他从死里复活以后,门徒就想起他说过这话,便信了圣经和耶稣所说的。
\par }{\SH 耶稣洞悉人心
\par }{\PP \VS{23}当耶稣在{\PN{耶路撒冷}}过逾越节的时候,有许多人看见他所行的神迹,就信了他的名。
\VS{24}耶稣却不将自己交托他们;因为他知道万人,
\VS{25}也用不着谁见证人怎样,因他知道人心里所存的。

\par }\Chap{3}{\SH 耶稣与尼哥德慕
\par }{\PP \VerseOne{1}有一个法利赛人,名叫{\PN{尼哥德慕}},是{\PN{犹太}}人的官。
\VS{2}这人夜里来见耶稣,说:「拉比,我们知道你是由 神那里来作师傅的;因为你所行的神迹,若没有 神同在,无人能行。」
\VS{3}耶稣回答说:「我实实在在地告诉你,人若不重生,就不能见 神的国。」
\VS{4}{\PN{尼哥德慕}}说:「人已经老了,如何能重生呢?岂能再进母腹生出来吗?」
\VS{5}耶稣说:「我实实在在地告诉你,人若不是从水和{\ADD{圣}}灵生的,就不能进 神的国。
\VS{6}从肉身生的就是肉身;从灵生的就是灵。
\VS{7}我说:『你们必须重生』,你不要以为希奇。
\VS{8}风随着意思吹,你听见风的响声,却不晓得从哪里来,往哪里去;凡从{\ADD{圣}}灵生的,也是如此。」
\VS{9}{\PN{尼哥德慕}}问他说:「怎能有这事呢?」
\VS{10}耶稣回答说:「你是{\PN{以色列}}人的先生,还不明白这事吗?
\VS{11}我实实在在地告诉你,我们所说的是我们知道的;我们所见证的是我们见过的;你们却不领受我们的见证。
\VS{12}我对你们说地上的事,你们尚且不信,若说天上的事,如何能信呢?
\VS{13}除了从天降下、仍旧在天的人子,没有人升过天。
\VS{14}{\PN{摩西}}在旷野怎样举蛇,人子也必照样被举起来,
\VS{15}叫一切信他的都得永生\FTNT{}{{\FR 3:15: }或译:叫一切信的人在他里面得永生}。
\par }{\PP \VS{16}「 神爱世人,甚至将他的独生子赐给{\ADD{他们}},叫一切信他的,不致灭亡,反得永生。
\VS{17}因为 神差他的儿子降世,不是要定世人的罪\FTNT{}{{\FR 3:17: }或译:审判世人;下同},乃是要叫世人因他得救。
\VS{18}信他的人,不被定罪;不信的人,罪已经定了,因为他不信 神独生子的名。
\VS{19}光来到世间,世人因自己的行为是恶的,不爱光,倒爱黑暗,定他们的罪就是在此。
\VS{20}凡作恶的便恨光,并不来就光,恐怕他的行为受责备。
\VS{21}但行真理的必来就光,要显明他所行的是靠 神而行。」
\par }{\SH 耶稣和施洗约翰
\par }{\PP \VS{22}这事以后,耶稣和门徒到了{\PN{犹太}}地,在那里居住,施洗。
\VS{23}{\PN{约翰}}在靠近{\PN{撒冷}}的{\PN{哀嫩}}也施洗;因为那里水多,众人都去受洗。(
\VS{24}那时{\PN{约翰}}还没有下在监里。)
\VS{25}{\PN{约翰}}的门徒和一个{\PN{犹太}}人辩论洁净的礼,
\VS{26}就来见{\PN{约翰}},说:「拉比,从前同你在{\PN{约旦河}}外、你所见证的那位,现在施洗,众人都往他那里去了。」
\VS{27}{\PN{约翰}}说:「若不是从天上赐的,人就不能得什么。
\VS{28}我曾说:『我不是基督,是奉差遣在他前面的』,你们自己可以给我作见证。
\VS{29}娶新妇的就是新郎;新郎的朋友站着,听见新郎的声音就甚喜乐。故此,我这喜乐满足了。
\VS{30}他必兴旺,我必衰微。」
\par }{\SH 从天上来的那一位
\par }{\PP \VS{31}「从天上来的是在万有之上;从地上来的是属乎地,他所说的也是属乎地。从天上来的是在万有之上。
\VS{32}他将所见所闻的见证出来,只是没有人领受他的见证。
\VS{33}那领受他见证的,就印上印,证明 神是真的。
\VS{34}神所差来的就说 神的话,因为 神赐{\ADD{圣}}灵{\ADD{给他}}是没有限量的。
\VS{35}父爱子,已将万有交在他手里。
\VS{36}信子的人有永生;不信子的人得不着{\ADD{永}}生\FTNT{}{{\FR 3:36: }原文是不得见永生}, 神的震怒常在他身上。」

\par }\Chap{4}{\SH 耶稣和撒马利亚妇人
\par }{\PP \VerseOne{1}主知道法利赛人听见他收门徒,施洗,比{\PN{约翰}}还多,(
\VS{2}其实不是耶稣亲自施洗,乃是他的门徒施洗,)
\VS{3}他就离了{\PN{犹太}},又往{\PN{加利利}}去。
\VS{4}必须经过{\PN{撒马利亚}},
\VS{5}于是到了{\PN{撒马利亚}}的一座城,名叫{\PN{叙加}},靠近{\PN{雅各}}给他儿子{\PN{约瑟}}的那块地。
\VS{6}在那里有{\PN{雅各井}};耶稣因走路困乏,就坐在井旁。那时约有午正。
\par }{\PP \VS{7}有一个{\PN{撒马利亚}}的妇人来打水。耶稣对她说:「请你给我水喝。」(
\VS{8}那时门徒进城买食物去了。)
\VS{9}{\PN{撒马利亚}}的妇人对他说:「你既是{\PN{犹太}}人,怎么向我一个{\PN{撒马利亚}}妇人要水喝呢?」原来{\PN{犹太}}人和{\PN{撒马利亚}}人没有来往。
\VS{10}耶稣回答说:「你若知道 神的恩赐,和对你说『给我水喝』的是谁,你必早求他,他也必早给了你活水。」
\VS{11}妇人说:「先生,没有打水的器具,井又深,你从哪里得活水呢?
\VS{12}我们的祖宗{\PN{雅各}}将这井留给我们,他自己和儿子并牲畜也都喝这井里的水,难道你比他还大吗?」
\VS{13}耶稣回答说:「凡喝这水的还要再渴;
\VS{14}人若喝我所赐的水就永远不渴。我所赐的水要在他里头成为泉源,直涌到永生。」
\VS{15}妇人说:「先生,请把这水赐给我,叫我不渴,也不用来这么远打水。」
\VS{16}耶稣说:「你去叫你丈夫也到这里来。」
\VS{17}妇人说:「我没有丈夫。」耶稣说:「你说没有丈夫是不错的。
\VS{18}你已经有五个丈夫,你现在有的并不是你的丈夫。你这话是真的。」
\VS{19}妇人说:「先生,我看出你是先知。
\VS{20}我们的祖宗在这山上礼拜,你们倒说,应当礼拜的地方是在{\PN{耶路撒冷}}。」
\VS{21}耶稣说:「妇人,你当信我。时候将到,你们拜父,也不在这山上,也不在{\PN{耶路撒冷}}。
\VS{22}你们所拜的,你们不知道;我们所拜的,我们知道,因为救恩是从{\PN{犹太}}人出来的。
\VS{23}时候将到,如今就是了,那真正拜父的,要用心灵和诚实拜他,因为父要这样的人拜他。」
\VS{24}神是个灵\FTNT{}{{\FR 4:24: }或无个字},所以拜他的必须用心灵和诚实拜他。
\VS{25}妇人说:「我知道弥赛亚(就是那称为基督的)要来;他来了,必将一切的事都告诉我们。」
\VS{26}耶稣说:「这和你说话的就是他!」
\par }{\PP \VS{27}当下门徒回来,就希奇耶稣和一个妇人说话;只是没有人说:「你是要什么?」或说:「你为什么和她说话?」
\VS{28}那妇人就留下水罐子,往城里去,对众人说:
\VS{29}「你们来看!有一个人将我素来所行的一切事都给我说出来了,莫非这就是基督吗?」
\VS{30}众人就出城,往耶稣那里去。
\par }{\PP \VS{31}这其间,门徒对耶稣说:「拉比,请吃。」
\VS{32}耶稣说:「我有食物吃,是你们不知道的。」
\VS{33}门徒就彼此对问说:「莫非有人拿什么给他吃吗?」
\VS{34}耶稣说:「我的食物就是遵行差我来者的旨意,做成他的工。
\VS{35}你们岂不说『到收割的时候还有四个月』吗?我告诉你们,举目向田观看,庄稼已经熟了\FTNT{}{{\FR 4:35: }原文是发白},可以收割了。
\VS{36}收割的人得工价,积蓄五谷到永生,叫撒种的和收割的一同快乐。
\VS{37}俗语说:『那人撒种,这人收割』,这话可见是真的。
\VS{38}我差你们去收你们所没有劳苦的;别人劳苦,你们享受他们所劳苦的。」
\par }{\PP \VS{39}那城里有好些{\PN{撒马利亚}}人信了耶稣,因为那妇人作见证说:「他将我素来所行的一切事都给我说出来了。」
\VS{40}于是{\PN{撒马利亚}}人来见耶稣,求他在他们那里住下,他便在那里住了两天。
\VS{41}因耶稣的话,信的人就更多了,
\VS{42}便对妇人说:「现在我们信,不是因为你的话,是我们亲自听见了,知道这真是救世主。」
\par }{\SH 耶稣治好大臣的儿子
\par }{\PP \VS{43}过了那两天,耶稣离了那地方,往{\PN{加利利}}去。
\VS{44}因为耶稣自己作过见证说:「先知在本地是没有人尊敬的。」
\VS{45}到了{\PN{加利利}},{\PN{加利利}}人既然看见他在{\PN{耶路撒冷}}过节所行的一切事,就接待他,因为他们也是上去过节。
\par }{\PP \VS{46}耶稣又到了{\PN{加利利}}的{\PN{迦拿}},就是他从前变水为酒的地方。有一个大臣,他的儿子在{\PN{迦百农}}患病。
\VS{47}他听见耶稣从{\PN{犹太}}到了{\PN{加利利}},就来见他,求他下去医治他的儿子,因为他儿子快要死了。
\VS{48}耶稣就对他说:「若不看见神迹奇事,你们总是不信。」
\VS{49}那大臣说:「先生,求你趁着我的孩子还没有死就下去。」
\VS{50}耶稣对他说:「回去吧,你的儿子活了!」那人信耶稣所说的话就回去了。
\VS{51}正下去的时候,他的仆人迎见他,说他的儿子活了。
\VS{52}他就问什么时候见好的。他们说:「昨日未时热就退了。」
\VS{53}他便知道这正是耶稣对他说「你儿子活了」的时候;他自己和全家就都信了。
\VS{54}这是耶稣在{\PN{加利利}}行的第二件神迹,是他从{\PN{犹太}}回去以后行的。

\par }\Chap{5}{\SH 在毕士大池边治病
\par }{\PP \VerseOne{1}这事以后,到了{\PN{犹太}}人的一个节期,耶稣就上{\PN{耶路撒冷}}去。
\VS{2}在{\PN{耶路撒冷}},靠近羊{\ADD{门}}有一个池子,希伯来话叫作{\PN{毕士大}},{\ADD{旁边}}有五个廊子;
\VS{3}里面躺着瞎眼的、瘸腿的、血气枯干的许多病人。\FTNT{}{{\FR 5:3: }有古卷加:等候水动;4因为有天使按时下池子搅动那水,水动之后,谁先下去,无论害什么病就痊愈了。}
\VS{5}在那里有一个人,病了三十八年。
\VS{6}耶稣看见他躺着,知道他病了许久,就问他说:「你要痊愈吗?」
\VS{7}病人回答说:「先生,水动的时候,没有人把我放在池子里;我正去的时候,就有别人比我先下去。」
\VS{8}耶稣对他说:「起来,拿你的褥子走吧!」
\VS{9}那人立刻痊愈,就拿起褥子来走了。
\par }{\PP \VS{10}那天是安息日,所以{\PN{犹太}}人对那医好的人说:「今天是安息日,你拿褥子是不可的。」
\VS{11}他却回答说:「那使我痊愈的,对我说:『拿你的褥子走吧。』」
\VS{12}他们问他说:「对你说『拿褥子走』的是什么人?」
\VS{13}那医好的人不知道是谁;因为那里的人多,耶稣已经躲开了。
\VS{14}后来耶稣在殿里遇见他,对他说:「你已经痊愈了,不要再犯罪,恐怕你遭遇的更加利害。」
\VS{15}那人就去告诉{\PN{犹太}}人,使他痊愈的是耶稣。
\VS{16}所以{\PN{犹太}}人逼迫耶稣,因为他在安息日做了这事。
\VS{17}耶稣就对他们说:「我父做事直到如今,我也做事。」
\VS{18}所以{\PN{犹太}}人越发想要杀他;因他不但犯了安息日,并且称 神为他的父,将自己和 神当作平等。
\par }{\SH 子的权柄
\par }{\PP \VS{19}耶稣对他们说:「我实实在在地告诉你们,子凭着自己不能做什么,惟有看见父所做的,子才能做;父所做的事,子也照样做。
\VS{20}父爱子,将自己所做的一切事指给他看,还要将比这更大的事指给他看,叫你们希奇。
\VS{21}父怎样叫死人起来,使他们活着,子也照样随自己的意思使人活着。
\VS{22}父不审判什么人,乃将审判的事全交与子,
\VS{23}叫人都尊敬子如同尊敬父一样。不尊敬子的,就是不尊敬差子来的父。
\VS{24}我实实在在地告诉你们,那听我话、又信差我来者的,就有永生;不至于定罪,是已经出死入生了。
\VS{25}我实实在在地告诉你们,时候将到,现在就是了,死人要听见 神儿子的声音,听见的人就要活了。
\VS{26}因为父怎样在自己有生命,就赐给他儿子也照样在自己有生命,
\VS{27}并且因为他是人子,就赐给他行审判的权柄。
\VS{28}你们不要把这事看作希奇。时候要到,凡在坟墓里的,都要听见他的声音,就出来:
\VS{29}行善的,复活得生;作恶的,复活定罪。
\par }{\PP \VS{30}「我凭着自己不能做什么,我怎么听见就怎么审判。我的审判也是公平的;因为我不求自己的意思,只求那差我来者的意思。」
\par }{\SH 为耶稣作证
\par }{\PP \VS{31}「我若为自己作见证,我的见证就不真。
\VS{32}另有一位给我作见证,我也知道他给我作的见证是真的。
\VS{33}你们曾差人到{\PN{约翰}}那里,他为真理作过见证。
\VS{34}其实,我所受的见证不是从人来的;然而,我说这些话,为要叫你们得救。
\VS{35}{\PN{约翰}}是点着的明灯,你们情愿暂时喜欢他的光。
\VS{36}但我有比{\PN{约翰}}更大的见证;因为父交给我要我成就的事,就是我所做的事,这便见证我是父所差来的。
\VS{37}差我来的父也为我作过见证。你们从来没有听见他的声音,也没有看见他的形象。
\VS{38}你们并没有他的道存在心里;因为他所差来的,你们不信。
\VS{39}你们查考圣经\FTNT{}{{\FR 5:39: }或译:应当查考圣经},因你们以为内中有永生;给我作见证的就是这经。
\VS{40}然而,你们不肯到我这里来得生命。
\par }{\PP \VS{41}「我不受从人来的荣耀。
\VS{42}但我知道,你们心里没有 神的爱。
\VS{43}我奉我父的名来,你们并不接待我;若有别人奉自己的名来,你们倒要接待他。
\VS{44}你们互相受荣耀,却不求从独一之 神来的荣耀,怎能信{\ADD{我}}呢?
\VS{45}不要想我在父面前要告你们;有一位告你们的,就是你们所仰赖的{\PN{摩西}}。
\VS{46}你们如果信{\PN{摩西}},也必信我,因为他{\ADD{书上}}有指着我写的话。
\VS{47}你们若不信他的书,怎能信我的话呢?」

\par }\Chap{6}{\SH 耶稣给五千人吃饱
\par }{\R (太14·13—21;可6·30—44;路9·10—17)
\par }{\PP \VerseOne{1}这事以后,耶稣渡过{\PN{加利利海}},就是{\PN{提比哩亚海}}。
\VS{2}有许多人因为看见他在病人身上所行的神迹,就跟随他。
\VS{3}耶稣上了山,和门徒一同坐在那里。
\VS{4}那时{\PN{犹太}}人的逾越节近了。
\VS{5}耶稣举目看见许多人来,就对{\PN{腓力}}说:「我们从哪里买饼叫这些人吃呢?」(
\VS{6}他说这话是要试验{\PN{腓力}};他自己原知道要怎样行。)
\VS{7}{\PN{腓力}}回答说:「就是二十两银子的饼,叫他们各人吃一点也是不够的。」
\VS{8}有一个门徒,就是{\PN{西门·彼得}}的兄弟{\PN{安得烈}},对耶稣说:
\VS{9}「在这里有一个孩童,带着五个大麦饼、两条鱼,只是分给这许多人还算什么呢?」
\VS{10}耶稣说:「你们叫众人坐下。」原来那地方的草多,众人就坐下,数目约有五千。
\VS{11}耶稣拿起饼来,祝谢了,就分给那坐着的人;分鱼也是这样,都随着他们所要的。
\VS{12}他们吃饱了,耶稣对门徒说:「把剩下的零碎收拾起来,免得有糟蹋的。」
\VS{13}他们便将那五个大麦饼的零碎,就是众人吃了剩下的,收拾起来,装满了十二个篮子。
\VS{14}众人看见耶稣所行的神迹,就说:「这真是那要到世间来的先知!」
\VS{15}耶稣既知道众人要来强逼他作王,就独自又退到山上去了。
\par }{\SH 耶稣在海面上行走
\par }{\R (太14·22—33;可6·45—52)
\par }{\PP \VS{16}到了晚上,他的门徒下海边去,
\VS{17}上了船,要过海往{\PN{迦百农}}去。天已经黑了,耶稣还没有来到他们那里。
\VS{18}忽然狂风大作,海就翻腾起来。
\VS{19}门徒摇橹,约行了十里多路,看见耶稣在海面上走,渐渐近了船,他们就害怕。
\VS{20}耶稣对他们说:「是我,不要怕!」
\VS{21}门徒就喜欢接他上船,船立时到了他们所要去的地方。
\par }{\SH 耶稣是生命的粮
\par }{\PP \VS{22}第二日,站在海那边的众人知道那里没有别的船,只有一只小船,又知道耶稣没有同他的门徒上船,乃是门徒自己去的。
\VS{23}然而,有几只小船从{\PN{提比哩亚}}来,靠近主祝谢后{\ADD{分饼给人}}吃的地方。
\VS{24}众人见耶稣和门徒都不在那里,就上了船,往{\PN{迦百农}}去找耶稣。
\VS{25}既在海那边找着了,就对他说:「拉比,是几时到这里来的?」
\VS{26}耶稣回答说:「我实实在在地告诉你们,你们找我,并不是因见了神迹,乃是因吃饼得饱。
\VS{27}不要为那必坏的食物劳力,要为那存到永生的食物劳力,就是人子要赐给你们的,因为人子是父 神所印证的。」
\VS{28}众人问他说:「我们当行什么才算做 神的工呢?」
\VS{29}耶稣回答说:「信 神所差来的,这就是做 神的工。」
\VS{30}他们又说:「你行什么神迹,叫我们看见就信你;你到底做什么事呢?
\VS{31}我们的祖宗在旷野吃过吗哪,如{\ADD{经上}}写着说:『他从天上赐下粮来给他们吃。』」
\VS{32}耶稣说:「我实实在在地告诉你们,那从天上来的粮不是{\PN{摩西}}赐给你们的,乃是我父将天上来的真粮赐给你们。
\VS{33}因为 神的粮就是那从天上降下来、赐生命给世界的。」
\par }{\PP \VS{34}他们说:「主啊,常将这粮赐给我们!」
\VS{35}耶稣说:「我就是生命的粮。到我这里来的,必定不饿;信我的,永远不渴。
\VS{36}只是我对你们说过,你们已经看见我,还是不信。
\VS{37}凡父所赐给我的人必到我这里来;到我这里来的,我总不丢弃他。
\VS{38}因为我从天上降下来,不是要按自己的意思行,乃是要按那差我来者的意思行。
\VS{39}差我来者的意思就是:他所赐给我的,叫我一个也不失落,在末日却叫他复活。
\VS{40}因为我父的意思是叫一切见子而信的人得永生,并且在末日我要叫他复活。」
\par }{\PP \VS{41}{\PN{犹太}}人因为耶稣说「我是从天上降下来的粮」,就私下议论他,
\VS{42}说:「这不是{\PN{约瑟}}的儿子耶稣吗?他的父母我们岂不认得吗?他如今怎么说『我是从天上降下来的』呢?」
\VS{43}耶稣回答说:「你们不要大家议论。
\VS{44}若不是差我来的父吸引人,就没有能到我这里来的;{\ADD{到我这里来的}},在末日我要叫他复活。
\VS{45}在先知书上写着说:『他们都要蒙 神的教训。』凡听见父{\ADD{之教训}}又学习的,就到我这里来。
\VS{46}这不是说有人看见过父,惟独从 神来的,他看见过父。
\VS{47}我实实在在地告诉你们,信的人有永生。
\VS{48}我就是生命的粮。
\VS{49}你们的祖宗在旷野吃过吗哪,还是死了。
\VS{50}这是从天上降下来的粮,叫人吃了就不死。
\VS{51}我是从天上降下来生命的粮;人若吃这粮,就必永远活着。我所要赐的粮就是我的肉,为世人之生命所赐的。」
\par }{\PP \VS{52}因此,{\PN{犹太}}人彼此争论说:「这个人怎能把他的肉给我们吃呢?」
\VS{53}耶稣说:「我实实在在地告诉你们,你们若不吃人子的肉,不喝人子的血,就没有生命在你们里面。
\VS{54}吃我肉、喝我血的人就有永生,在末日我要叫他复活。
\VS{55}我的肉真是可吃的,我的血真是可喝的。
\VS{56}吃我肉、喝我血的人常在我里面,我也常在他里面。
\VS{57}永活的父怎样差我来,我又因父活着;照样,吃我{\ADD{肉}}的人也要因我活着。
\VS{58}这就是从天上降下来的粮。吃这粮的人就永远活着,不像{\ADD{你们的}}祖宗吃过{\ADD{吗哪}}还是死了。」
\VS{59}这些话是耶稣在{\PN{迦百农}}会堂里教训人说的。
\par }{\SH 永生的话语
\par }{\PP \VS{60}他的门徒中有好些人听见了,就说:「这话甚难,谁能听呢?」
\VS{61}耶稣心里知道门徒为这话议论,就对他们说:「这话是叫你们厌弃\FTNT{}{{\FR 6:61: }原文是跌倒}吗?
\VS{62}倘或你们看见人子升到他原来所在之处,怎么样呢?
\VS{63}叫人活着的乃是灵,肉体是无益的。我对你们所说的话就是灵,就是生命。
\VS{64}只是你们中间有不信的人。」耶稣从起头就知道谁不信他,谁要卖他。
\VS{65}耶稣又说:「所以我对你们说过,若不是蒙我父的恩赐,没有人能到我这里来。」
\par }{\PP \VS{66}从此,他门徒中多有退去的,不再和他同行。
\VS{67}耶稣就对那十二个门徒说:「你们也要去吗?」
\VS{68}{\PN{西门·彼得}}回答说:「主啊,你有永生之道,我们还归从谁呢?
\VS{69}我们已经信了,又知道你是 神的圣者。」
\VS{70}耶稣说:「我不是拣选了你们十二个门徒吗?但你们中间有一个是魔鬼。」
\VS{71}耶稣这话是指着{\PN{加略}}人{\PN{西门}}的儿子{\PN{犹大}}说的;他本是十二个门徒里的一个,后来要卖耶稣的。

\par }\Chap{7}{\SH 耶稣的兄弟不信
\par }{\PP \VerseOne{1}这事以后,耶稣在{\PN{加利利}}游行,不愿在{\PN{犹太}}游行,因为{\PN{犹太}}人想要杀他。
\VS{2}当时{\PN{犹太}}人的住棚节近了。
\VS{3}耶稣的弟兄就对他说:「你离开这里上{\PN{犹太}}去吧,叫你的门徒也看见你所行的事。
\VS{4}人要显扬名声,没有在暗处行事的;你如果行这些事,就当将自己显明给世人看。」
\VS{5}因为连他的弟兄{\ADD{说这话}},是因为不信他。
\VS{6}耶稣就对他们说:「我的时候还没有到;你们的时候常是方便的。
\VS{7}世人不能恨你们,却是恨我,因为我指证他们所做的事是恶的。
\VS{8}你们上去过节吧,我现在不上去过这节,因为我的时候还没有满。」
\VS{9}耶稣说了这话,仍旧住在{\PN{加利利}}。
\par }{\SH 耶稣过住棚节
\par }{\PP \VS{10}但他弟兄上去以后,他也上去过节,不是明去,似乎是暗去的。
\VS{11}正在节期,{\PN{犹太}}人寻找耶稣,说:「他在哪里?」
\VS{12}众人为他纷纷议论,有的说:「他是好人。」有的说:「不然,他是迷惑众人的。」
\VS{13}只是没有人明明地讲论他,因为怕{\PN{犹太}}人。
\par }{\PP \VS{14}到了节期,耶稣上殿里去教训人。
\VS{15}{\PN{犹太}}人就希奇,说:「这个人没有学过,怎么明白书呢?」
\VS{16}耶稣说:「我的教训不是我自己的,乃是那差我来者的。
\VS{17}人若立志遵着他的旨意行,就必晓得这教训或是出于 神,或是我凭着自己说的。
\VS{18}人凭着自己说,是求自己的荣耀;惟有求那差他来者的荣耀,这人是真的,在他心里没有不义。
\VS{19}{\PN{摩西}}岂不是传律法给你们吗?你们却没有一个人守律法。为什么想要杀我呢?」
\VS{20}众人回答说:「你是被鬼附着了!谁想要杀你?」
\VS{21}耶稣说:「我做了一件事,你们都以为希奇。
\VS{22}{\PN{摩西}}传割礼给你们(其实不是从{\PN{摩西}}起的,乃是从祖先起的),因此你们也在安息日给人行割礼。
\VS{23}人若在安息日受割礼,免得违背{\PN{摩西}}的律法,我在安息日叫一个人全然好了,你们就向我生气吗?
\VS{24}不可按外貌断定是非,总要按公平断定是非。」
\par }{\SH 这是基督吗?
\par }{\PP \VS{25}{\PN{耶路撒冷}}人中有的说:「这不是他们想要杀的人吗?
\VS{26}你看他还明明地讲道,他们也不向他说什么,难道官长真知道这是基督吗?
\VS{27}然而,我们知道这个人从哪里来;只是基督来的时候,没有人知道他从哪里来。」
\VS{28}那时,耶稣在殿里教训人,大声说:「你们也知道我,也知道我从哪里来;我来并不是由于自己。但那差我来的是真的。你们不认识他,
\VS{29}我却认识他;因为我是从他来的,他也是差了我来。」
\VS{30}他们就想要捉拿耶稣;只是没有人下手,因为他的时候还没有到。
\VS{31}但众人中间有好些信他的,说:「基督来的时候,他所行的神迹岂能比这人所行的更多吗?」
\par }{\SH 差役捉拿耶稣
\par }{\PP \VS{32}法利赛人听见众人为耶稣这样纷纷议论,祭司长和法利赛人就打发差役去捉拿他。
\VS{33}于是耶稣说:「我还有不多的时候和你们同在,以后就回到差我来的那里去。
\VS{34}你们要找我,却找不着;我所在的地方你们不能到。」
\VS{35}{\PN{犹太}}人就彼此对问说:「这人要往哪里去,叫我们找不着呢?难道他要往散住{\PN{希腊}}中的{\PN{犹太}}人那里去教训{\PN{希腊}}人吗?
\VS{36}他说:『你们要找我,却找不着;我所在的地方,你们不能到』,这话是什么意思呢?」
\par }{\SH 活水的江河
\par }{\PP \VS{37}节期的末日,就是最大之日,耶稣站着高声说:「人若渴了,可以到我这里来喝。
\VS{38}信我的人就如经上所说:『从他腹中要流出活水的江河来。』」
\VS{39}耶稣这话是指着信他之人要受{\ADD{圣}}灵说的。那时还没有{\ADD{赐下圣}}灵{\ADD{来}},因为耶稣尚未得着荣耀。
\par }{\SH 群众因耶稣而纷争
\par }{\PP \VS{40}众人听见这话,有的说:「这真是那先知。」
\VS{41}有的说:「这是基督。」但也有的说:「基督岂是从{\PN{加利利}}出来的吗?
\VS{42}经上岂不是说『基督是{\PN{大卫}}的后裔,从{\PN{大卫}}本乡{\PN{伯利恒}}出来的』吗?」
\VS{43}于是众人因着耶稣起了纷争。
\VS{44}其中有人要捉拿他,只是无人下手。
\par }{\SH 犹太领袖的不信
\par }{\PP \VS{45}差役回到祭司长和法利赛人那里。他们对差役说:「你们为什么没有带他来呢?」
\VS{46}差役回答说:「从来没有像他这样说话的!」
\VS{47}法利赛人说:「你们也受了迷惑吗?
\VS{48}官长或是法利赛人岂有信他的呢?
\VS{49}但这些不明白律法的百姓是被咒诅的!」
\VS{50}内中有{\PN{尼哥德慕}},就是从前去见耶稣的,对他们说:
\VS{51}「不先听本人的口供,不知道他所做的事,难道我们的律法还定他的罪吗?」
\VS{52}他们回答说:「你也是出于{\PN{加利利}}吗?你且去查考,就可知道{\PN{加利利}}没有出过先知。」

\par }\Chap{8}{\SH 行淫时被捉的女人
\par }{\PP \VerseOne{1}于是各人都回家去了;耶稣却往{\PN{橄榄山}}去,
\VS{2}清早又回到殿里。众百姓都到他那里去,他就坐下,教训他们。
\VS{3}文士和法利赛人带着一个行淫时被拿的妇人来,叫她站在当中,
\VS{4}就对耶稣说:「夫子,这妇人是正行淫之时被拿的。
\VS{5}{\PN{摩西}}在律法上吩咐我们把这样的妇人用石头打死。你说该把她怎么样呢?」
\VS{6}他们说这话,乃试探耶稣,要得着告他的把柄。耶稣却弯着腰,用指头在地上画字。
\VS{7}他们还是不住地问他,耶稣就直起腰来,对他们说:「你们中间谁是没有罪的,谁就可以先拿石头打她。」
\VS{8}于是又弯着腰,用指头在地上画字。
\VS{9}他们听见这话,就从老到少,一个一个地都出去了,只剩下耶稣一人,还有那妇人仍然站在当中。
\VS{10}耶稣就直起腰来,对她说:「妇人,那些人在哪里呢?没有人定你的罪吗?」
\VS{11}她说:「主啊,没有。」耶稣说:「我也不定你的罪。去吧,从此不要再犯罪了!」
\par }{\SH 耶稣是世界的光
\par }{\PP \VS{12}耶稣又对众人说:「我是世界的光。跟从我的,就不在黑暗里走,必要得着生命的光。」
\VS{13}法利赛人对他说:「你是为自己作见证,你的见证不真。」
\VS{14}耶稣说:「我虽然为自己作见证,我的见证还是真的;因我知道我从哪里来,往哪里去;你们却不知道我从哪里来,往哪里去。
\VS{15}你们是以外貌\FTNT{}{{\FR 8:15: }原文是凭肉身}判断人,我却不判断人。
\VS{16}就是判断人,我的判断也是真的;因为不是我独自在这里,还有差我来的父与我同在。
\VS{17}你们的律法上也记着说:『两个人的见证是真的。』
\VS{18}我是为自己作见证,还有差我来的父也是为我作见证。」
\VS{19}他们就问他说:「你的父在哪里?」耶稣回答说:「你们不认识我,也不认识我的父;若是认识我,也就认识我的父。」
\VS{20}这些话是耶稣在殿里的库房、教训人时所说的,也没有人拿他,因为他的时候还没有到。
\par }{\SH 我所去的地方你们不能到
\par }{\PP \VS{21}耶稣又对他们说:「我要去了,你们要找我,并且你们要死在罪中。我所去的地方,你们不能到。」
\VS{22}{\PN{犹太}}人说:「他说『我所去的地方,你们不能到』,难道他要自尽吗?」
\VS{23}耶稣对他们说:「你们是从下头来的,我是从上头来的;你们是属这世界的,我不是属这世界的。
\VS{24}所以我对你们说,你们要死在罪中。你们若不信我是{\ADD{基督}},必要死在罪中。」
\VS{25}他们就问他说:「你是谁?」耶稣对他们说:「就是我从起初所告诉你们的。
\VS{26}我有许多事讲论你们,判断你们;但那差我来的是真的,我在他那里所听见的,我就传给世人。」
\VS{27}他们不明白耶稣是指着父说的。
\VS{28}所以耶稣说:「你们举起人子以后,必知道我是{\ADD{基督}},并且知道我没有一件事是凭着自己做的。我说这些话乃是照着父所教训我的。
\VS{29}那差我来的是与我同在;他没有撇下我独自在这里,因为我常做他所喜悦的事。」
\VS{30}耶稣说这话的时候,就有许多人信他。
\par }{\SH 真理必使你们得自由
\par }{\PP \VS{31}耶稣对信他的{\PN{犹太}}人说:「你们若常常遵守我的道,就真是我的门徒;
\VS{32}你们必晓得真理,真理必叫你们得以自由。」
\VS{33}他们回答说:「我们是{\PN{亚伯拉罕}}的后裔,从来没有作过谁的奴仆。你怎么说『你们必得以自由』呢?」
\VS{34}耶稣回答说:「我实实在在地告诉你们,所有犯罪的就是罪的奴仆。
\VS{35}奴仆不能永远住在家里;儿子是永远住在家里。
\VS{36}所以{\ADD{天父的}}儿子若叫你们自由,你们就真自由了。
\VS{37}我知道你们是{\PN{亚伯拉罕}}的子孙,你们却想要杀我,因为你们心里容不下我的道。
\VS{38}我所说的是在我父那里看见的;你们所行的是在你们的父那里听见的。」
\par }{\SH 你们的父是魔鬼
\par }{\PP \VS{39}他们说:「我们的父就是{\PN{亚伯拉罕}}。」耶稣说:「你们若是{\PN{亚伯拉罕}}的儿子,就必行{\PN{亚伯拉罕}}所行的事。
\VS{40}我将在 神那里所听见的真理告诉了你们,现在你们却想要杀我,这不是{\PN{亚伯拉罕}}所行的事。
\VS{41}你们是行你们父所行的事。」他们说:「我们不是从淫乱生的;我们只有一位父,就是 神。」
\VS{42}耶稣说:「倘若 神是你们的父,你们就必爱我;因为我本是出于 神,也是从 神而来,并不是由着自己来,乃是他差我来。
\VS{43}你们为什么不明白我的话呢?无非是因你们不能听我的道。
\VS{44}你们是出于你们的父魔鬼,你们父的私欲你们偏要行。他从起初是杀人的,不守真理,因他心里没有真理。他说谎是出于自己;因他本来是说谎的,也是说谎之人的父。
\VS{45}我将真理告诉你们,你们就因此不信我。
\VS{46}你们中间谁能指证我有罪呢?我既然将真理告诉你们,为什么不信我呢?
\VS{47}出于 神的,必听 神的话;你们不听,因为你们不是出于 神。」
\par }{\SH 还没有亚伯拉罕就有了我
\par }{\PP \VS{48}{\PN{犹太}}人回答说:「我们说你是{\PN{撒马利亚}}人,并且是鬼附着的,这话岂不正对吗?」
\VS{49}耶稣说:「我不是鬼附着的;我尊敬我的父,你们倒轻慢我。
\VS{50}我不求自己的荣耀,有一位{\ADD{为我}}求{\ADD{荣耀}}、定是非的。
\VS{51}我实实在在地告诉你们,人若遵守我的道,就永远不见死。」
\VS{52}{\PN{犹太}}人对他说:「现在我们知道你是鬼附着的。{\PN{亚伯拉罕}}死了,众先知也死了,你还说:『人若遵守我的道,就永远不尝死味。』
\VS{53}难道你比我们的祖宗{\PN{亚伯拉罕}}还大吗?他死了,众先知也死了,你将自己当作什么人呢?」
\VS{54}耶稣回答说:「我若荣耀自己,我的荣耀就算不得什么;荣耀我的乃是我的父,就是你们所说是你们的 神。
\VS{55}你们未曾认识他;我却认识他。我若说不认识他,我就是说谎的,像你们一样;但我认识他,也遵守他的道。
\VS{56}你们的祖宗{\PN{亚伯拉罕}}欢欢喜喜地仰望我的日子,既看见了就快乐。」
\VS{57}{\PN{犹太}}人说:「你还没有五十岁,岂见过{\PN{亚伯拉罕}}呢?」
\VS{58}耶稣说:「我实实在在地告诉你们,还没有{\PN{亚伯拉罕}}就有了我。」
\VS{59}于是他们拿石头要打他;耶稣却躲藏,从殿里出去了。

\par }\Chap{9}{\SH 治好生来瞎眼的
\par }{\PP \VerseOne{1}耶稣过去的时候,看见一个人生来是瞎眼的。
\VS{2}门徒问耶稣说:「拉比,这人生来是瞎眼的,是谁犯了罪?是这人呢?是他父母呢?」
\VS{3}耶稣回答说:「也不是这人犯了罪,也不是他父母犯了罪,是要在他身上显出 神的作为来。
\VS{4}趁着白日,我们必须做那差我来者的工;黑夜将到,就没有人能做工了。
\VS{5}我在世上的时候,是世上的光。」
\VS{6}耶稣说了这话,就吐唾沫在地上,用唾沫和泥抹在瞎子的眼睛上,
\VS{7}对他说:「你往{\PN{西罗亚}}池子里去洗。」({\PN{西罗亚}}翻出来就是「奉差遣」。)他去一洗,回头就看见了。
\VS{8}他的邻舍和那素常见他是讨饭的,就说:「这不是那从前坐着讨饭的人吗?」
\VS{9}有人说:「是他」;又有人说:「不是,却是像他。」他自己说:「是我。」
\VS{10}他们对他说:「你的眼睛是怎么开的呢?」
\VS{11}他回答说:「有一个人,名叫耶稣,他和泥抹我的眼睛,对我说:『你往{\PN{西罗亚}}{\ADD{池子}}去洗。』我去一洗,就看见了。」
\VS{12}他们说:「那个人在哪里?」他说:「我不知道。」
\par }{\SH 法利赛人盘问医治的事
\par }{\PP \VS{13}他们把从前瞎眼的人带到法利赛人那里。
\VS{14}耶稣和泥开他眼睛的日子是安息日。
\VS{15}法利赛人也问他是怎么得看见的。瞎子对他们说:「他把泥抹在我的眼睛上,我去一洗,就看见了。」
\VS{16}法利赛人中有的说:「这个人不是从 神来的,因为他不守安息日。」又有人说:「一个罪人怎能行这样的神迹呢?」他们就起了纷争。
\VS{17}他们又对瞎子说:「他既然开了你的眼睛,你说他是怎样的人呢?」他说:「是个先知。」
\VS{18}{\PN{犹太}}人不信他从前是瞎眼,后来能看见的,等到叫了他的父母来,
\VS{19}问他们说:「这是你们的儿子吗?你们说他生来是瞎眼的,如今怎么能看见了呢?」
\VS{20}他父母回答说:「他是我们的儿子,生来就瞎眼,这是我们知道的。
\VS{21}至于他如今怎么能看见,我们却不知道;是谁开了他的眼睛,我们也不知道。他已经成了人,你们问他吧,他自己必能说。」
\VS{22}他父母说这话,是怕{\PN{犹太}}人;因为{\PN{犹太}}人已经商议定了,若有认耶稣是基督的,要把他赶出会堂。
\VS{23}因此他父母说:「他已经成了人,你们问他吧。」
\VS{24}所以法利赛人第二次叫了那从前瞎眼的人来,对他说:「你该将荣耀归给 神,我们知道这人是个罪人。」
\VS{25}他说:「他是个罪人不是,我不知道;有一件事我知道,从前我是眼瞎的,如今能看见了。」
\VS{26}他们就问他说:「他向你做什么?是怎么开了你的眼睛呢?」
\VS{27}他回答说:「我方才告诉你们,你们不听,为什么又要听呢?莫非你们也要作他的门徒吗?」
\VS{28}他们就骂他说:「你是他的门徒;我们是{\PN{摩西}}的门徒。
\VS{29}神对{\PN{摩西}}说话是我们知道的;只是这个人,我们不知道他从哪里来!」
\VS{30}那人回答说:「他开了我的眼睛,你们竟不知道他从哪里来,这真是奇怪!
\VS{31}我们知道 神不听罪人,惟有敬奉 神、遵行他旨意的, 神才听他。
\VS{32}从创世以来,未曾听见有人把生来是瞎子的眼睛开了。
\VS{33}这人若不是从 神来的,什么也不能做。」
\VS{34}他们回答说:「你全然生在罪孽中,还要教训我们吗?」于是把他赶出去了。
\par }{\SH 灵性的盲目
\par }{\PP \VS{35}耶稣听说他们把他赶出去,后来遇见他,就说:「你信 神的儿子吗?」
\VS{36}他回答说:「主啊,谁是 神的儿子,叫我信他呢?」
\VS{37}耶稣说:「你已经看见他,现在和你说话的就是他。」
\VS{38}他说:「主啊,我信!」就拜耶稣。
\VS{39}耶稣说:「我为审判到这世上来,叫不能看见的,可以看见;能看见的,反瞎了眼。」
\VS{40}同他在那里的法利赛人听见这话,就说:「难道我们也瞎了眼吗?」
\VS{41}耶稣对他们说:「你们若瞎了眼,就没有罪了;但如今你们说『我们能看见』,{\ADD{所以}}你们的罪还在。」

\par }\Chap{10}{\SH 羊圈的比喻
\par }{\PP \VerseOne{1}「我实实在在地告诉你们,人进羊圈,不从门进去,倒从别处爬进去,那人就是贼,就是强盗。
\VS{2}从门进去的,才是羊的牧人。
\VS{3}看门的就给他开门;羊也听他的声音。他按着名叫自己的羊,把羊领出来。
\VS{4}既放出自己的羊来,就在前头走,羊也跟着他,因为认得他的声音。
\VS{5}羊不跟着生人;因为不认得他的声音,必要逃跑。」
\VS{6}耶稣将这比喻告诉他们,但他们不明白所说的是什么意思。
\par }{\SH 好牧人耶稣
\par }{\PP \VS{7}所以,耶稣又对他们说:「我实实在在地告诉你们,我就是羊的门。
\VS{8}凡在我以先来的都是贼,是强盗;羊却不听他们。
\VS{9}我就是门;凡从我进来的,必然得救,并且出入得草吃。
\VS{10}盗贼来,无非要偷窃,杀害,毁坏;我来了,是要叫羊\FTNT{}{{\FR 10:10: }或译:人}得生命,并且得的更丰盛。
\VS{11}我是好牧人;好牧人为羊舍命。
\VS{12}若是雇工,不是牧人,羊也不是他自己的,他看见狼来,就撇下羊逃走;狼抓住羊,赶散了羊群。
\VS{13}雇工{\ADD{逃走}},因他是雇工,并不顾念羊。
\VS{14}我是好牧人;我认识我的羊,我的羊也认识我,
\VS{15}正如父认识我,我也认识父一样;并且我为羊舍命。
\VS{16}我另外有羊,不是这圈里的;我必须领他们来,他们也要听我的声音,并且要合成一群,归一个牧人了。
\VS{17}我父爱我;因我将命舍去,好再取回来。
\VS{18}没有人夺我的命去,是我自己舍的。我有权柄舍了,也有权柄取回来。这是我从我父所受的命令。」
\par }{\PP \VS{19}{\PN{犹太}}人为这些话又起了纷争。
\VS{20}内中有好些人说:「他是被鬼附着,而且疯了,为什么听他呢?」
\VS{21}又有人说:「这不是鬼附之人所说的话。鬼岂能叫瞎子的眼睛开了呢?」
\par }{\SH 被犹太人弃绝
\par }{\PP \VS{22}在{\PN{耶路撒冷}}有修殿节,是冬天的时候。
\VS{23}耶稣在殿里{\PN{所罗门}}的廊下行走。
\VS{24}{\PN{犹太}}人围着他,说:「你叫我们犹疑不定到几时呢?你若是基督,就明明地告诉我们。」
\VS{25}耶稣回答说:「我已经告诉你们,你们不信。我奉我父之名所行的事可以为我作见证;
\VS{26}只是你们不信,因为你们不是我的羊。
\VS{27}我的羊听我的声音,我也认识他们,他们也跟着我。
\VS{28}我又赐给他们永生;他们永不灭亡,谁也不能从我手里把他们夺去。
\VS{29}我父把羊赐给我,他比万有都大,谁也不能从我父手里把他们夺去。
\VS{30}我与父原为一。」
\par }{\PP \VS{31}{\PN{犹太}}人又拿起石头来要打他。
\VS{32}耶稣对他们说:「我从父显出许多善事给你们看,你们是为哪一件拿石头打我呢?」
\VS{33}{\PN{犹太}}人回答说:「我们不是为善事拿石头打你,是为你说僭妄的话;又为你是个人,反将自己当作 神。」
\VS{34}耶稣说:「你们的律法上岂不是写着『我曾说你们是神』吗?
\VS{35}经上的话是不能废的;若那些承受 神道的人尚且称为神,
\VS{36}父所分别为圣、又差到世间来的,他自称是 神的儿子,你们还向他说『你说僭妄的话』吗?
\VS{37}我若不行我父的事,你们就不必信我;
\VS{38}我若行了,你们纵然不信我,也当信这些事,叫你们又知道又明白父在我里面,我也在父里面。」
\VS{39}他们又要拿他,他却逃出他们的手走了。
\par }{\PP \VS{40}耶稣又往{\PN{约旦河}}外去,到了{\PN{约翰}}起初施洗的地方,就住在那里。
\VS{41}有许多人来到他那里。他们说:「{\PN{约翰}}一件神迹没有行过,但{\PN{约翰}}指着这人所说的一切话都是真的。」
\VS{42}在那里,信耶稣的人就多了。

\par }\Chap{11}{\SH 拉撒路的死
\par }{\PP \VerseOne{1}有一个患病的人,名叫{\PN{拉撒路}},住在{\PN{伯大尼}},就是{\PN{马利亚}}和她姊姊{\PN{马大}}的村庄。
\VS{2}这{\PN{马利亚}}就是那用香膏抹主,又用头发擦他脚的;患病的{\PN{拉撒路}}是她的兄弟。
\VS{3}她姊妹两个就打发人去见耶稣,说:「主啊,你所爱的人病了。」
\VS{4}耶稣听见,就说:「这病不至于死,乃是为 神的荣耀,叫 神的儿子因此得荣耀。」
\VS{5}耶稣素来爱{\PN{马大}}和她妹子并{\PN{拉撒路}}。
\VS{6}听见{\PN{拉撒路}}病了,就在所居之地仍住了两天。
\VS{7}然后对门徒说:「我们再往{\PN{犹太}}去吧。」
\VS{8}门徒说:「拉比,{\PN{犹太}}人近来要拿石头打你,你还往那里去吗?」
\VS{9}耶稣回答说:「白日不是有十二小时吗?人在白日走路,就不至跌倒,因为看见这世上的光。
\VS{10}若在黑夜走路,就必跌倒,因为他没有光。」
\VS{11}耶稣说了这话,随后对他们说:「我们的朋友{\PN{拉撒路}}睡了,我去叫醒他。」
\VS{12}门徒说:「主啊,他若睡了,就必好了。」
\VS{13}耶稣这话是指着他死说的,他们却以为是说照常睡了。
\VS{14}耶稣就明明地告诉他们说:「{\PN{拉撒路}}死了。
\VS{15}我没有在那里就欢喜,这是为你们的缘故,好叫你们相信。{\ADD{如今}}我们可以往他那里去吧。」
\VS{16}{\PN{多马}},又称为{\PN{低土马}},就对那同作门徒的说:「我们也去和他同死吧。」
\par }{\SH 复活在我,生命在我
\par }{\PP \VS{17}耶稣到了,就知道{\PN{拉撒路}}在坟墓里已经四天了。
\VS{18}{\PN{伯大尼}}离{\PN{耶路撒冷}}不远,约有六里路。
\VS{19}有好些{\PN{犹太}}人来看{\PN{马大}}和{\PN{马利亚}},要为她们的兄弟安慰她们。
\VS{20}{\PN{马大}}听见耶稣来了,就出去迎接他;{\PN{马利亚}}却仍然坐在家里。
\VS{21}{\PN{马大}}对耶稣说:「主啊,你若早在这里,我兄弟必不死。
\VS{22}就是现在,我也知道,你无论向 神求什么, 神也必赐给你。」
\VS{23}耶稣说:「你兄弟必然复活。」
\VS{24}{\PN{马大}}说:「我知道在末日复活的时候,他必复活。」
\VS{25}耶稣对她说:「复活在我,生命也在我。信我的人虽然死了,也必复活;
\VS{26}凡活着信我的人必永远不死。你信这话吗?」
\VS{27}{\PN{马大}}说:「主啊,是的,我信你是基督,是 神的儿子,就是那要临到世界的。」
\par }{\SH 耶稣哭了
\par }{\PP \VS{28}{\PN{马大}}说了这话,就回去暗暗地叫她妹子{\PN{马利亚}},说:「夫子来了,叫你。」
\VS{29}{\PN{马利亚}}听见了,就急忙起来,到耶稣那里去。
\VS{30}那时,耶稣还没有进村子,仍在{\PN{马大}}迎接他的地方。
\VS{31}那些同{\PN{马利亚}}在家里安慰她的{\PN{犹太}}人,见她急忙起来出去,就跟着她,以为她要往坟墓那里去哭。
\VS{32}{\PN{马利亚}}到了耶稣那里,看见他,就俯伏在他脚前,说:「主啊,你若早在这里,我兄弟必不死。」
\VS{33}耶稣看见她哭,并看见与她同来的{\PN{犹太}}人也哭,就心里悲叹,又甚忧愁,
\VS{34}便说:「你们把他安放在哪里?」他们回答说:「请主来看。」
\VS{35}耶稣哭了。
\VS{36}{\PN{犹太}}人就说:「你看他爱这人是何等恳切。」
\VS{37}其中有人说:「他既然开了瞎子的眼睛,岂不能叫这人不死吗?」
\par }{\SH 拉撒路复活
\par }{\PP \VS{38}耶稣又心里悲叹,来到坟墓前;那坟墓是个洞,有一块石头挡着。
\VS{39}耶稣说:「你们把石头挪开。」那死人的姊姊{\PN{马大}}对他说:「主啊,他现在必是臭了,因为他{\ADD{死了}}已经四天了。」
\VS{40}耶稣说:「我不是对你说过,你若信,就必看见 神的荣耀吗?」
\VS{41}他们就把石头挪开。耶稣举目{\ADD{望天}},说:「父啊,我感谢你,因为你已经听我。
\VS{42}我也知道你常听我,但我说这话是为周围站着的众人,叫他们信是你差了我来。」
\VS{43}说了这话,就大声呼叫说:「{\PN{拉撒路}}出来!」
\VS{44}那死人就出来了,手脚裹着布,脸上包着手巾。耶稣对他们说:「解开,叫他走!」
\par }{\SH 杀害耶稣的阴谋
\par }{\R (太26·1—5;可14·1—2;路22·1—2)
\par }{\PP \VS{45}那些来看{\PN{马利亚}}的{\PN{犹太}}人见了耶稣所做的事,就多有信他的;
\VS{46}但其中也有去见法利赛人的,将耶稣所做的事告诉他们。
\VS{47}祭司长和法利赛人聚集公会,说:「这人行好些神迹,我们怎么办呢?
\VS{48}若这样由着他,人人都要信他,{\PN{罗马}}人也要来夺我们的地土和我们的百姓。」
\VS{49}内中有一个人,名叫{\PN{该亚法}},本年作大祭司,对他们说:「你们不知道什么。
\VS{50}独不想一个人替百姓死,免得通国灭亡,就是你们的益处。」
\VS{51}他这话不是出于自己,是因他本年作大祭司,所以预言耶稣将要替这一国死;
\VS{52}也不但替这一国死,并要将 神四散的子民都聚集归一。
\VS{53}从那日起,他们就商议要杀耶稣
\par }{\PP \VS{54}所以,耶稣不再显然行在{\PN{犹太}}人中间,就离开那里往靠近旷野的地方去,到了一座城,名叫{\PN{以法莲}},就在那里和门徒同住。
\par }{\PP \VS{55}{\PN{犹太}}人的逾越{\ADD{节}}近了,有许多人从乡下上{\PN{耶路撒冷}}去,要在节前洁净自己。
\VS{56}他们就寻找耶稣,站在殿里彼此说:「你们的意思如何,他不来过节吗?」
\VS{57}那时,祭司长和法利赛人早已吩咐说,若有人知道耶稣在哪里,就要报明,好去拿他。

\par }\Chap{12}{\SH 在伯大尼受膏
\par }{\R (太26·6—13;可14·3—9)
\par }{\PP \VerseOne{1}逾越节前六日,耶稣来到{\PN{伯大尼}},就是他叫{\PN{拉撒路}}从死里复活之处。
\VS{2}有人在那里给耶稣预备筵席;{\PN{马大}}伺候,{\PN{拉撒路}}也在那同耶稣坐席的人中。
\VS{3}{\PN{马利亚}}就拿着一斤极贵的真哪哒香膏,抹耶稣的脚,又用自己头发去擦,屋里就满了膏的香气。
\VS{4}有一个门徒,就是那将要卖耶稣的{\PN{加略}}人{\PN{犹大}},
\VS{5}说:「这香膏为什么不卖三十两银子周济穷人呢?」
\VS{6}他说这话,并不是挂念穷人,乃因他是个贼,又带着钱囊,常取其中所存的。
\VS{7}耶稣说:「由她吧!她是为我安葬之日存留的。
\VS{8}因为常有穷人和你们同在,只是你们不常有我。」
\par }{\SH 杀害拉撒路的阴谋
\par }{\PP \VS{9}有许多{\PN{犹太}}人知道耶稣在那里,就来了,不但是为耶稣的缘故,也是要看他从死里所复活的{\PN{拉撒路}}。
\VS{10}但祭司长商议连{\PN{拉撒路}}也要杀了;
\VS{11}因有好些{\PN{犹太}}人为{\PN{拉撒路}}的缘故,回去信了耶稣。
\par }{\SH 光荣地进耶路撒冷
\par }{\R (太21·1—11;可11·1—11;路19·28—40)
\par }{\PP \VS{12}第二天,有许多上来过节的人听见耶稣将到{\PN{耶路撒冷}},
\VS{13}就拿着棕树枝出去迎接他,喊着说:
\par }{\Q 和散那!
\par }{\Q 奉主名来的{\PN{以色列}}王是应当称颂的!
\par }{\MM \VS{14}耶稣得了一个驴驹,就骑上,如{\ADD{经}}上所记的说:
\par }{\Q \VS{15}{\PN{锡安}}的民\FTNT{}{{\FR 12:15: }原文是女子}哪,不要惧怕!
\par }{\Q 你的王骑着驴驹来了。
\par }{\PP \VS{16}这些事门徒起先不明白,等到耶稣得了荣耀以后才想起这话是指着他写的,并且众人果然向他这样行了。
\VS{17}当耶稣呼唤{\PN{拉撒路}},叫他从死复活出坟墓的时候,同耶稣在那里的众人就作见证。
\VS{18}众人因听见耶稣行了这神迹,就去迎接他。
\VS{19}法利赛人彼此说:「看哪,你们是徒劳无益,世人都随从他去了。」
\par }{\SH 希腊人求见耶稣
\par }{\PP \VS{20}那时,上来过节礼拜的人中,有几个{\PN{希腊}}人。
\VS{21}他们来见{\PN{加利利、伯赛大}}的{\PN{腓力}},求他说:「先生,我们愿意见耶稣。」
\VS{22}{\PN{腓力}}去告诉{\PN{安得烈}},{\PN{安得烈}}同{\PN{腓力}}去告诉耶稣。
\VS{23}耶稣说:「人子得荣耀的时候到了。
\VS{24}我实实在在地告诉你们,一粒麦子不落在地里死了,仍旧是一粒,若是死了,就结出许多子粒来。
\VS{25}爱惜自己生命的,就失丧生命;在这世上恨恶自己生命的,就要保守生命到永生。
\VS{26}若有人服事我,就当跟从我;我在哪里,服事我的人也要在那里;若有人服事我,我父必尊重他。」
\par }{\SH 人子必须被举起来
\par }{\PP \VS{27}「我现在心里忧愁,我说什么才好呢?父啊,救我脱离这时候;但我原是为这时候来的。
\VS{28}父啊,愿你荣耀你的名!」当时就有声音从天上来,说:「我已经荣耀了我的名,还要再荣耀。」
\VS{29}站在旁边的众人听见,就说:「打雷了。」还有人说:「有天使对他说话。」
\VS{30}耶稣说:「这声音不是为我,是为你们来的。
\VS{31}现在这世界受审判,这世界的王要被赶出去。
\VS{32}我若从地上被举起来,就要吸引万人来归我。」
\VS{33}耶稣这话原是指着自己将要怎样死说的。
\VS{34}众人回答说:「我们听见律法上有话说,基督是永存的,你怎么说『人子必须被举起来』呢?这人子是谁呢?」
\VS{35}耶稣对他们说:「光在你们中间还有不多的时候,应当趁着有光行走,免得黑暗临到你们;那在黑暗里行走的,不知道往何处去。
\VS{36}你们应当趁着有光,信从这光,使你们成为光明之子。」
\par }{\SH 犹太人的不信
\par }{\PP 耶稣说了这话,就离开他们隐藏了。
\VS{37}他虽然在他们面前行了许多神迹,他们还是不信他。
\VS{38}这是要应验先知{\PN{以赛亚}}的话,说:
\par }{\Q 主啊,我们所传的有谁信呢?
\par }{\Q 主的膀臂向谁显露呢?
\par }{\MM \VS{39}他们所以不能信,因为{\PN{以赛亚}}又说:
\par }{\Q \VS{40}主叫他们瞎了眼,
\par }{\Q 硬了心,
\par }{\Q 免得他们眼睛看见,
\par }{\Q 心里明白,回转过来,
\par }{\Q 我就医治他们。
\par }{\PP \VS{41}{\PN{以赛亚}}因为看见他的荣耀,就指着他说这话。
\VS{42}虽然如此,官长中却有好些信他的,只因法利赛人的缘故,就不承认,恐怕被赶出会堂。
\VS{43}这是因他们爱人的荣耀过于爱 神的荣耀。
\par }{\SH 耶稣的话要审判人
\par }{\PP \VS{44}耶稣大声说:「信我的,不是信我,乃是信那差我来的。
\VS{45}人看见我,就是看见那差我来的。
\VS{46}我到世上来,乃是光,叫凡信我的,不住在黑暗里。
\VS{47}若有人听见我的话不遵守,我不审判他。我来本不是要审判世界,乃是要拯救世界。
\VS{48}弃绝我、不领受我话的人,有审判他的—就是我所讲的道在末日要审判他。
\VS{49}因为我没有凭着自己讲,惟有差我来的父已经给我命令,叫我说什么,讲什么。
\VS{50}我也知道他的命令就是永生。故此,我所讲的话正是照着父对我所说的。」

\par }\Chap{13}{\SH 耶稣为门徒洗脚
\par }{\PP \VerseOne{1}逾越节以前,耶稣知道自己离世归父的时候到了。他既然爱世间属自己的人,就爱他们到底。
\VS{2}吃晚饭的时候,魔鬼已将卖耶稣的意思放在{\PN{西门}}的儿子{\PN{加略}}人{\PN{犹大}}心里。
\VS{3}耶稣知道父已将万有交在他手里,且知道自己是从 神出来的,又要归到 神那里去,
\VS{4}就离席站起来,脱了衣服,拿一条手巾束腰,
\VS{5}随后把水倒在盆里,就洗门徒的脚,并用自己所束的手巾擦干。
\VS{6}挨到{\PN{西门·彼得}},{\PN{彼得}}对他说:「主啊,你洗我的脚吗?」
\VS{7}耶稣回答说:「我所做的,你如今不知道,后来必明白。」
\VS{8}{\PN{彼得}}说:「你永不可洗我的脚!」耶稣说:「我若不洗你,你就与我无分了。」
\VS{9}{\PN{西门·彼得}}说:「主啊,不但我的脚,连手和头也要洗。」
\VS{10}耶稣说:「凡洗过澡的人,只要把脚一洗,全身就干净了。你们是干净的,然而不都是干净的。」
\VS{11}耶稣原知道要卖他的是谁,所以说:「你们不都是干净的。」
\par }{\PP \VS{12}耶稣洗完了他们的脚,就穿上衣服,又坐下,对他们说:「我向你们所做的,你们明白吗?
\VS{13}你们称呼我夫子,称呼我主,你们说的不错,我本来是。
\VS{14}我是你们的主,你们的夫子,尚且洗你们的脚,你们也当彼此洗脚。
\VS{15}我给你们作了榜样,叫你们照着我向你们所做的去做。
\VS{16}我实实在在地告诉你们,仆人不能大于主人,差人也不能大于差他的人。
\VS{17}你们既知道这事,若是去行就有福了。
\VS{18}我这话不是指着你们众人说的,我知道我所拣选的是谁。现在要应验经上的话,说:『同我吃饭的人用脚踢我。』
\VS{19}如今事情还没有成就,我要先告诉你们,叫你们到事情成就的时候可以信我是{\ADD{基督}}。
\VS{20}我实实在在地告诉你们,有人接待我所差遣的,就是接待我;接待我,就是接待那差遣我的。」
\par }{\SH 预言自己将被出卖
\par }{\R (太26·20—25;可14·17—21;路22·21—23)
\par }{\PP \VS{21}耶稣说了这话,心里忧愁,就明说:「我实实在在地告诉你们,你们中间有一个人要卖我了。」
\VS{22}门徒彼此对看,猜不透所说的是谁。
\VS{23}有一个门徒,是耶稣所爱的,侧身挨近耶稣的怀里。
\VS{24}{\PN{西门·彼得}}点头对他说:「你告诉我们,主是指着谁说的。」
\VS{25}那门徒便就势靠着耶稣的胸膛,问他说:「主啊,是谁呢?」
\VS{26}耶稣回答说:「我蘸一点饼给谁,就是谁。」耶稣就蘸了一点饼,递给{\PN{加略}}人{\PN{西门}}的儿子{\PN{犹大}}。
\VS{27}他吃了以后,撒但就入了他的心。耶稣便对他说:「你所做的,快做吧!」
\VS{28}同席的人没有一个知道是为什么对他说这话。
\VS{29}有人因{\PN{犹大}}带着钱囊,以为耶稣是对他说:「你去买我们过节所应用的东西」,或是叫他拿什么周济穷人。
\VS{30}{\PN{犹大}}受了那点饼,立刻就出去。那时候是夜间了。
\par }{\SH 新的命令
\par }{\PP \VS{31}他既出去,耶稣就说:「如今人子得了荣耀, 神在人子身上也得了荣耀。
\VS{32}神要因自己荣耀人子,并且要快快地荣耀他。
\VS{33}小子们,我还有不多的时候与你们同在;后来你们要找我,但我所去的地方你们不能到。这话我曾对{\PN{犹太}}人说过,如今也照样对你们说。
\VS{34}我赐给你们一条新命令,乃是叫你们彼此相爱;我怎样爱你们,你们也要怎样相爱。
\VS{35}你们若有彼此相爱的心,众人因此就认出你们是我的门徒了。」
\par }{\SH 预言彼得不认主
\par }{\PP \VS{36}{\PN{西门·彼得}}问耶稣说:「主往哪里去?」耶稣回答说:「我所去的地方,你现在不能跟我去,后来却要跟我去。」
\VS{37}{\PN{彼得}}说:「主啊,我为什么现在不能跟你去?我愿意为你舍命!」
\VS{38}耶稣说:「你愿意为我舍命吗?我实实在在地告诉你,鸡叫以先,你要三次不认我。」

\par }\Chap{14}{\SH 耶稣是道路、真理、生命
\par }{\PP \VerseOne{1}「你们心里不要忧愁;你们信 神,也当信我。
\VS{2}在我父的家里有许多住处;若是没有,我就早已告诉你们了。我去原是为你们预备地方去。
\VS{3}我若去为你们预备了地方,就必再来接你们到我那里去,我在哪里,叫你们也在那里。
\VS{4}我往哪里去,你们知道;那条路,你们也知道\FTNT{}{{\FR 14:4: }有古卷:我往哪里去,你们知道那条路}。」
\VS{5}{\PN{多马}}对他说:「主啊,我们不知道你往哪里去,怎么知道那条路呢?」
\VS{6}耶稣说:「我就是道路、真理、生命;若不借着我,没有人能到父那里去。
\VS{7}你们若认识我,也就认识我的父。从今以后,你们认识他,并且已经看见他。」
\VS{8}{\PN{腓力}}对他说:「求主将父显给我们看,我们就知足了。」
\VS{9}耶稣对他说:「{\PN{腓力}},我与你们同在这样长久,你还不认识我吗?人看见了我,就是看见了父;你怎么说『将父显给我们看』呢?
\VS{10}我在父里面,父在我里面,你不信吗?我对你们所说的话,不是凭着自己说的,乃是住在我里面的父做他自己的事。
\VS{11}你们当信我,我在父里面,父在我里面;即或不信,也当因我所做的事信我。
\VS{12}我实实在在地告诉你们,我所做的事,信我的人也要做,并且要做比这更大的事,因为我往父那里去。
\VS{13}你们奉我的名无论求什么,我必成就,叫父因儿子得荣耀。
\VS{14}你们若奉我的名求什么,我必成就。」
\par }{\SH 应许赐圣灵
\par }{\PP \VS{15}「你们若爱我,就必遵守我的命令。
\VS{16}我要求父,父就另外赐给你们一位保惠师\FTNT{}{{\FR 14:16: }或译:训慰师;下同},叫他永远与你们同在,
\VS{17}就是真理的{\ADD{圣}}灵,乃世人不能接受的;因为不见他,也不认识他。你们却认识他,因他常与你们同在,也要在你们里面。
\VS{18}我不撇下你们为孤儿,我必到你们这里来。
\VS{19}还有不多的时候,世人不再看见我,你们却看见我;因为我活着,你们也要活着。
\VS{20}到那日,你们就知道我在父里面,你们在我里面,我也在你们里面。
\VS{21}有了我的命令又遵守的,这人就是爱我的;爱我的必蒙我父爱他,我也要爱他,并且要向他显现。」
\VS{22}{\PN{犹大}}(不是{\PN{加略}}人{\PN{犹大}})问耶稣说:「主啊,为什么要向我们显现,不向世人显现呢?」
\VS{23}耶稣回答说:「人若爱我,就必遵守我的道;我父也必爱他,并且我们要到他那里去,与他同住。
\VS{24}不爱我的人就不遵守我的道。你们所听见的道不是我的,乃是差我来之父的道。
\par }{\PP \VS{25}「我还与你们同住的时候,已将这些话对你们说了。
\VS{26}但保惠师,就是父因我的名所要差来的圣灵,他要将一切的事指教你们,并且要叫你们想起我对你们所说的一切话。
\VS{27}我留下平安给你们;我将我的平安赐给你们。我所赐的,不像世人所赐的。你们心里不要忧愁,也不要胆怯。
\VS{28}你们听见我对你们说了,我去还要到你们这里来。你们若爱我,因我到父那里去,就必喜乐,因为父是比我大的。
\VS{29}现在事情还没有成就,我预先告诉你们,叫你们到事情成就的时候就可以信。
\VS{30}以后我不再和你们多说话,因为这世界的王将到。他在我里面是毫无所有;
\VS{31}但要叫世人知道我爱父,并且父怎样吩咐我,我就怎样行。起来,我们走吧!」

\par }\Chap{15}{\SH 耶稣是真葡萄树
\par }{\PP \VerseOne{1}「我是真葡萄树,我父是栽培的人。
\VS{2}凡属我不结果子的枝子,他就剪去;凡结果子的,他就修理干净,使枝子结果子更多。
\VS{3}现在你们因我讲给你们的道,已经干净了。
\VS{4}你们要常在我里面,我也常在你们里面。枝子若不常在葡萄树上,自己就不能结果子;你们若不常在我里面,也是这样。
\VS{5}我是葡萄树,你们是枝子。常在我里面的,我也常在他里面,这人就多结果子;因为离了我,你们就不能做什么。
\VS{6}人若不常在我里面,就像枝子丢在外面枯干,人拾起来,扔在火里烧了。
\VS{7}你们若常在我里面,我的话也常在你们里面,凡你们所愿意的,祈求,就给你们成就。
\VS{8}你们多结果子,我父就因此得荣耀,你们也就是我的门徒了。
\VS{9}我爱你们,正如父爱我一样;你们要常在我的爱里。
\VS{10}你们若遵守我的命令,就常在我的爱里,正如我遵守了我父的命令,常在他的爱里。
\par }{\PP \VS{11}「这些事我已经对你们说了,是要叫我的喜乐存在你们心里,并叫你们的喜乐可以满足。
\VS{12}你们要彼此相爱,像我爱你们一样;这就是我的命令。
\VS{13}人为朋友舍命,人的爱心没有比这个大的。
\VS{14}你们若遵行我所吩咐的,就是我的朋友了。
\VS{15}以后我不再称你们为仆人,因仆人不知道主人所做的事。我乃称你们为朋友;因我从我父所听见的,已经都告诉你们了。
\VS{16}不是你们拣选了我,是我拣选了你们,并且分派你们去结果子,叫你们的果子常存,使你们奉我的名,无论向父求什么,他就赐给你们。
\VS{17}我这样吩咐你们,是要叫你们彼此相爱。」
\par }{\SH 世人的憎恨
\par }{\PP \VS{18}「世人若恨你们,你们知道\FTNT{}{{\FR 15:18: }或译:该知道},恨你们以先已经恨我了。
\VS{19}你们若属世界,世界必爱属自己的;只因你们不属世界,乃是我从世界中拣选了你们,所以世界就恨你们。
\VS{20}你们要记念我从前对你们所说的话:『仆人不能大于主人。』他们若逼迫了我,也要逼迫你们;若遵守了我的话,也要遵守你们的话。
\VS{21}但他们因我的名要向你们行这一切的事,因为他们不认识那差我来的。
\VS{22}我若没有来教训他们,他们就没有罪;但如今他们的罪无可推诿了。
\VS{23}恨我的,也恨我的父。
\VS{24}我若没有在他们中间行过别人未曾行的事,他们就没有罪;但如今连我与我的父,他们也看见也恨恶了。
\VS{25}这要应验他们律法上所写的话,说:『他们无故地恨我。』
\VS{26}但我要从父那里差保惠师来,就是从父出来真理的圣灵;他来了,就要为我作见证。
\VS{27}你们也要作见证,因为你们从起头就与我同在。」

\par }\Chap{16}{\PP \VerseOne{1}「我已将这些事告诉你们,使你们不至于跌倒。
\VS{2}人要把你们赶出会堂,并且时候将到,凡杀你们的就以为是事奉 神。
\VS{3}他们这样行,是因未曾认识父,也未曾认识我。
\VS{4}我将这事告诉你们,是叫你们到了时候可以想起我对你们说过了。」
\par }{\SH 圣灵的工作
\par }{\PP 「我起先没有将这事告诉你们,因为我与你们同在。
\VS{5}现今我往差我来的{\ADD{父}}那里去,你们中间并没有人问我:『你往哪里去?』
\VS{6}只因我将这事告诉你们,你们就满心忧愁。
\VS{7}然而,我将真情告诉你们,我去是与你们有益的;我若不去,保惠师就不到你们这里来;我若去,就差他来。
\VS{8}他既来了,就要叫世人为罪、为义、为审判,自己责备自己。
\VS{9}为罪,是因他们不信我;
\VS{10}为义,是因我往父那里去,你们就不再见我;
\VS{11}为审判,是因这世界的王受了审判。
\par }{\PP \VS{12}「我还有好些事要告诉你们,但你们现在担当不了\FTNT{}{{\FR 16:12: }或译:不能领会}。
\VS{13}只等真理的圣灵来了,他要引导你们明白\FTNT{}{{\FR 16:13: }原文是进入}一切的真理;因为他不是凭自己说的,乃是把他所听见的都说出来,并要把将来的事告诉你们。
\VS{14}他要荣耀我,因为他要将受于我的告诉你们。
\VS{15}凡父所有的,都是我的;所以我说,他要将受于我的告诉你们。」
\par }{\SH 忧愁变为喜乐
\par }{\PP \VS{16}「等不多时,你们就不得见我;再等不多时,你们还要见我。」
\VS{17}有几个门徒就彼此说:「他对我们说:『等不多时,你们就不得见我;再等不多时,你们还要见我』;又{\ADD{说}}:『因我往父那里去。』这是什么意思呢?」
\VS{18}门徒彼此说:「他说『等不多时』到底是什么意思呢?我们不明白他所说的话。」
\VS{19}耶稣看出他们要问他,就说:「我说『等不多时,你们就不得见我;再等不多时,你们还要见我』,你们为这话彼此相问吗?
\VS{20}我实实在在地告诉你们,你们将要痛哭、哀号,世人倒要喜乐;你们将要忧愁,然而你们的忧愁要变为喜乐。
\VS{21}妇人生产的时候就忧愁,因为她的时候到了;既生了孩子,就不再记念那苦楚,因为欢喜世上生了一个人。
\VS{22}你们现在也是忧愁,但我要再见你们,你们的心就喜乐了;这喜乐也没有人能夺去。
\VS{23}到那日,你们什么也就不问我了。我实实在在地告诉你们,你们若向父求什么,他必因我的名赐给你们。
\VS{24}向来你们没有奉我的名求什么,如今你们求,就必得着,叫你们的喜乐可以满足。」
\par }{\SH 我已经胜过世界
\par }{\PP \VS{25}「这些事,我是用比喻对你们说的;时候将到,我不再用比喻对你们说,乃要将父明明地告诉你们。
\VS{26}到那日,你们要奉我的名祈求;我并不对你们说,我要为你们求父。
\VS{27}父自己爱你们;因为你们已经爱我,又信我是从父出来的。
\VS{28}我从父出来,到了世界;我又离开世界,往父那里去。」
\VS{29}门徒说:「如今你是明说,并不用比喻了。
\VS{30}现在我们晓得你凡事都知道,也不用人问你,因此我们信你是从 神出来的。」
\VS{31}耶稣说:「现在你们信吗?
\VS{32}看哪,时候将到,且是已经到了,你们要分散,各归自己的地方去,留下我独自一人;其实我不是独自一人,因为有父与我同在。
\VS{33}我将这些事告诉你们,是要叫你们在我里面有平安。在世上,你们有苦难;但你们可以放心,我已经胜了世界。」

\par }\Chap{17}{\SH 耶稣的祷告
\par }{\PP \VerseOne{1}耶稣说了这话,就举目望天,说:「父啊,时候到了,愿你荣耀你的儿子,使儿子也荣耀你;
\VS{2}正如你曾赐给他权柄管理凡有血气的,叫他将永生赐给你所赐给他的人。
\VS{3}认识你—独一的真神,并且认识你所差来的耶稣基督,这就是永生。
\VS{4}我在地上已经荣耀你,你所托付我的事,我已成全了。
\VS{5}父啊,现在求你使我同你享荣耀,就是未有世界以先,我同你所有的荣耀。
\par }{\PP \VS{6}「你从世上赐给我的人,我已将你的名显明与他们。他们本是你的,你将他们赐给我,他们也遵守了你的道。
\VS{7}如今他们知道,凡你所赐给我的,都是从你那里来的;
\VS{8}因为你所赐给我的道,我已经赐给他们,他们也领受了,又确实知道,我是从你出来的,并且信你差了我来。
\VS{9}我为他们祈求,不为世人祈求,却为你所赐给我的人祈求,因他们本是你的。
\VS{10}凡是我的,都是你的;你的也是我的,并且我因他们得了荣耀。
\VS{11}从今以后,我不在世上,他们却在世上;我往你那里去。圣父啊,求你因你所赐给我的名保守他们,叫他们合而为一像我们一样。
\VS{12}我与他们同在的时候,因你所赐给我的名保守了他们,我也护卫了他们;其中除了那灭亡之子,没有一个灭亡的,好叫经上的话得应验。
\VS{13}现在我往你那里去,我还在世上说这话,是叫他们心里充满我的喜乐。
\VS{14}我已将你的道赐给他们。世界又恨他们;因为他们不属世界,正如我不属世界一样。
\VS{15}我不求你叫他们离开世界,只求你保守他们脱离那恶者\FTNT{}{{\FR 17:15: }或译:脱离罪恶}。
\VS{16}他们不属世界,正如我不属世界一样。
\VS{17}求你用真理使他们成圣;你的道就是真理。
\VS{18}你怎样差我到世上,我也照样差他们到世上。
\VS{19}我为他们的缘故,自己分别为圣,叫他们也因真理成圣。
\par }{\PP \VS{20}「我不但为这些人祈求,也为那些因他们的话信我的人祈求,
\VS{21}使他们都合而为一。正如你父在我里面,我在你里面,使他们也在我们里面,叫世人可以信你差了我来。
\VS{22}你所赐给我的荣耀,我已赐给他们,使他们合而为一,像我们合而为一。
\VS{23}我在他们里面,你在我里面,使他们完完全全地合而为一,叫世人知道你差了我来,也知道你爱他们如同爱我一样。
\VS{24}父啊,我在哪里,愿你所赐给我的人也同我在那里,叫他们看见你所赐给我的荣耀;因为创立世界以前,你已经爱我了。
\VS{25}公义的父啊,世人未曾认识你,我却认识你;这些人也知道你差了我来。
\VS{26}我已将你的名指示他们,还要指示他们,使你所爱我的爱在他们里面,我也在他们里面。」

\par }\Chap{18}{\SH 耶稣被捕
\par }{\R (太26·47—56;可14·43—50;路22·47—53)
\par }{\PP \VerseOne{1}耶稣说了这话,就同门徒出去,过了{\PN{汲沦溪}}。在那里有一个园子,他和门徒进去了。
\VS{2}卖耶稣的{\PN{犹大}}也知道那地方,因为耶稣和门徒屡次上那里去聚集。
\VS{3}{\PN{犹大}}领了一队兵,和祭司长并法利赛人的差役,拿着灯笼、火把、兵器,就来到园里。
\VS{4}耶稣知道将要临到自己的一切事,就出来对他们说:「你们找谁?」
\VS{5}他们回答说:「找{\PN{拿撒勒}}人耶稣。」耶稣说:「我就是!」卖他的{\PN{犹大}}也同他们站在那里。
\VS{6}耶稣一说「我就是」,他们就退后倒在地上。
\VS{7}他又问他们说:「你们找谁?」他们说:「找{\PN{拿撒勒}}人耶稣。」
\VS{8}耶稣说:「我已经告诉你们,我就是。你们若找我,就让这些人去吧。」
\VS{9}这要应验耶稣从前的话,说:「你所赐给我的人,我没有失落一个。」
\VS{10}{\PN{西门·彼得}}带着一把刀,就拔出来,将大祭司的仆人砍了一刀,削掉他的右耳;那仆人名叫{\PN{马勒古}}。
\VS{11}耶稣就对{\PN{彼得}}说:「收刀入鞘吧,我父所给我的那杯,我岂可不喝呢?」
\par }{\SH 耶稣被带到亚那面前
\par }{\PP \VS{12}那队兵和千夫长,并{\PN{犹太}}人的差役就拿住耶稣,把他捆绑了,
\VS{13}先带到{\PN{亚那}}面前,因为{\PN{亚那}}是本年作大祭司{\PN{该亚法}}的岳父。
\VS{14}这{\PN{该亚法}}就是从前向{\PN{犹太}}人发议论说「一个人替百姓死是有益的」那位。
\par }{\SH 彼得不认耶稣
\par }{\PP \VS{15}{\PN{西门·彼得}}跟着耶稣,还有一个门徒跟着。那门徒是大祭司所认识的,他就同耶稣进了大祭司的院子。
\VS{16}{\PN{彼得}}却站在门外。大祭司所认识的那个门徒出来,和看门的使女说了一声,就领{\PN{彼得}}进去。
\VS{17}那看门的使女对{\PN{彼得}}说:「你不也是这人的门徒吗?」他说:「我不是。」
\VS{18}仆人和差役因为天冷,就生了炭火,站在那里烤火;{\PN{彼得}}也同他们站着烤火。
\par }{\SH 大祭司盘问耶稣
\par }{\R (太26·59—66;可14·55—64;路22·66—71)
\par }{\PP \VS{19}大祭司就以耶稣的门徒和他的教训盘问他。
\VS{20}耶稣回答说:「我从来是明明地对世人说话。我常在会堂和殿里,就是{\PN{犹太}}人聚集的地方教训人;我在暗地里并没有说什么。
\VS{21}你为什么问我呢?可以问那听见的人,我对他们说的是什么;我所说的,他们都知道。」
\VS{22}耶稣说了这话,旁边站着的一个差役用手掌打他,说:「你这样回答大祭司吗?」
\VS{23}耶稣说:「我若说的不是,你可以指证那不是;我若说的是,你为什么打我呢?」
\VS{24}{\PN{亚那}}就把耶稣解到大祭司{\PN{该亚法}}那里,仍是捆着解去的。
\par }{\SH 彼得再次不认耶稣
\par }{\R (太26·71—75;可14·69—72;路22·58—62)
\par }{\PP \VS{25}{\PN{西门·彼得}}正站着烤火,有人对他说:「你不也是他的门徒吗?」{\PN{彼得}}不承认,说:「我不是。」
\VS{26}有大祭司的一个仆人,是{\PN{彼得}}削掉耳朵那人的亲属,说:「我不是看见你同他在园子里吗?」
\VS{27}{\PN{彼得}}又不承认。立时鸡就叫了。
\par }{\SH 耶稣在彼拉多面前受审
\par }{\R (太27·1—2,11—14;可15·1—5;路23·1—5)
\par }{\PP \VS{28}众人将耶稣从{\PN{该亚法}}那里往衙门内解去,那时天还早。他们自己却不进衙门,恐怕染了污秽,不能吃逾越{\ADD{节的筵席}}。
\VS{29}{\PN{彼拉多}}就出来,到他们那里,说:「你们告这人是为什么事呢?」
\VS{30}他们回答说:「这人若不是作恶的,我们就不把他交给你。」
\VS{31}{\PN{彼拉多}}说:「你们自己带他去,按着你们的律法审问他吧。」{\PN{犹太}}人说:「我们没有杀人的权柄。」
\VS{32}这要应验耶稣所说自己将要怎样死的话了。
\VS{33}{\PN{彼拉多}}又进了衙门,叫耶稣来,对他说:「你是{\PN{犹太}}人的王吗?」
\VS{34}耶稣回答说:「这话是你自己说的,还是别人论我对你说的呢?」
\VS{35}{\PN{彼拉多}}说:「我岂是{\PN{犹太}}人呢?你本国的人和祭司长把你交给我。你做了什么事呢?」
\VS{36}耶稣回答说:「我的国不属这世界;我的国若属这世界,我的臣仆必要争战,使我不至于被交给{\PN{犹太}}人。只是我的国不属这世界。」
\VS{37}{\PN{彼拉多}}就对他说:「这样,你是王吗?」耶稣回答说:「你说我是王。我为此而生,也为此来到世间,特为给真理作见证。凡属真理的人就听我的话。」
\VS{38}{\PN{彼拉多}}说:「真理是什么呢?」
\par }{\SH 耶稣被判死刑
\par }{\R (太27·15—31;可15·6—20;路23·13—25)
\par }{\PP 说了这话,又出来到{\PN{犹太}}人那里,对他们说:「我查不出他有什么罪来。
\VS{39}但你们有个规矩,在逾越节要我给你们释放一个人,你们要我给你们释放{\PN{犹太}}人的王吗?」
\VS{40}他们又喊着说:「不要这人,要{\PN{巴拉巴}}!」这{\PN{巴拉巴}}是个强盗。

\par }\Chap{19}{\PP \VerseOne{1}当下{\PN{彼拉多}}将耶稣鞭打了。
\VS{2}兵丁用荆棘编做冠冕戴在他头上,给他穿上紫袍,
\VS{3}又挨近他,说:「恭喜,{\PN{犹太}}人的王啊!」他们就用手掌打他。
\VS{4}{\PN{彼拉多}}又出来对众人说:「我带他出来见你们,叫你们知道我查不出他有什么罪来。」
\VS{5}耶稣出来,戴着荆棘冠冕,穿着紫袍。{\ADD{
{\PN{彼拉多}}}}对他们说:「你们看这个人!」
\VS{6}祭司长和差役看见他,就喊着说:「钉他十字架!钉他十字架!」{\PN{彼拉多}}说:「你们自己把他钉十字架吧!我查不出他有什么罪来。」
\VS{7}{\PN{犹太}}人回答说:「我们有律法,按那律法,他是该死的,因他以自己为 神的儿子。」
\par }{\PP \VS{8}{\PN{彼拉多}}听见这话,越发害怕,
\VS{9}又进衙门,对耶稣说:「你是哪里来的?」耶稣却不回答。
\VS{10}{\PN{彼拉多}}说:「你不对我说话吗?你岂不知我有权柄释放你,也有权柄把你钉十字架吗?」
\VS{11}耶稣回答说:「若不是从上头赐给你的,你就毫无权柄办我。所以,把我交给你的那人罪更重了。」
\VS{12}从此,{\PN{彼拉多}}想要释放耶稣,无奈{\PN{犹太}}人喊着说:「你若释放这个人,就不是凯撒的忠臣\FTNT{}{{\FR 19:12: }原文是朋友}。凡以自己为王的,就是背叛凯撒了。」
\par }{\PP \VS{13}{\PN{彼拉多}}听见这话,就带耶稣出来,到了一个地方,名叫「铺华石处」,希伯来话叫{\PN{厄巴大}},就在那里坐堂。
\VS{14}那日是预备逾越{\ADD{节}}的日子,约有午正。{\PN{彼拉多}}对{\PN{犹太}}人说:「看哪,这是你们的王!」
\VS{15}他们喊着说:「除掉他!除掉他!钉他在十字架上!」{\PN{彼拉多}}说:「我可以把你们的王钉十字架吗?」祭司长回答说:「除了凯撒,我们没有王。」
\VS{16}于是{\PN{彼拉多}}将耶稣交给他们去钉十字架。
\par }{\SH 耶稣被钉十字架
\par }{\R (太27·32—44;可15·21—32;路23·26—43)
\par }{\PP \VS{17}他们就把耶稣带了去。耶稣背着自己的十字架出来,到了一个地方,名叫「髑髅地」,希伯来话叫{\PN{各各他}}。
\VS{18}他们就在那里钉他在十字架上,还有两个人和他一同钉着,一边一个,耶稣在中间。
\VS{19}{\PN{彼拉多}}又{\ADD{用牌子}}写了一个名号,安在十字架上,写的是:「{\PN{犹太}}人的王,{\PN{拿撒勒}}人耶稣。」
\VS{20}有许多{\PN{犹太}}人念这名号;因为耶稣被钉十字架的地方与城相近,并且是用{\PN{希伯来}}、{\PN{罗马}}、{\PN{希腊}}三样文字写的。
\VS{21}{\PN{犹太}}人的祭司长就对{\PN{彼拉多}}说:「不要写『{\PN{犹太}}人的王』,要写『他自己说:我是{\PN{犹太}}人的王』。」
\VS{22}{\PN{彼拉多}}说:「我所写的,我已经写上了。」
\par }{\PP \VS{23}兵丁既然将耶稣钉在十字架上,就拿他的衣服分为四分,每兵一分;又拿他的里衣,这件里衣原来没有缝儿,是上下一片织成的。
\VS{24}他们就彼此说:「我们不要撕开,只要拈阄,看谁得着。」这要应验经上的话说:
\par }{\Q 他们分了我的外衣,
\par }{\Q 为我的里衣拈阄。
\par }{\MM 兵丁果然做了这事。
\VS{25}站在耶稣十字架旁边的,有他母亲与他母亲的姊妹,并{\PN{革罗罢}}的妻子{\PN{马利亚}},和{\PN{抹大拉}}的{\PN{马利亚}}。
\VS{26}耶稣见母亲和他所爱的那门徒站在旁边,就对他母亲说:「母亲\FTNT{}{{\FR 19:26: }原文是妇人},看,你的儿子!」
\VS{27}又对那门徒说:「看,你的母亲!」从此,那门徒就接她到自己家里去了。
\par }{\SH 耶稣的死
\par }{\R (太27·45—56;可15·33—41;路23·44—49)
\par }{\PP \VS{28}这事以后,耶稣知道各样的事已经成了,为要使经上的话应验,就说:「我渴了。」
\VS{29}有一个器皿盛满了醋,放在那里;他们就拿海绒蘸满了醋,绑在牛膝草上,送到他口。
\VS{30}耶稣尝\FTNT{}{{\FR 19:30: }原文是受}了那醋,就说:「成了!」便低下头,将灵魂交付 {\ADD{神}}了。
\par }{\SH 肋旁被扎
\par }{\PP \VS{31}{\PN{犹太}}人因这日是预备日,又因那安息日是个大日,就求{\PN{彼拉多}}叫人打断他们的腿,把他们拿去,免得尸首当安息日留在十字架上。
\VS{32}于是兵丁来,把头一个人的腿,并与耶稣同钉第二个人的腿,都打断了。
\VS{33}只是来到耶稣那里,见他已经死了,就不打断他的腿。
\VS{34}惟有一个兵拿枪扎他的肋旁,随即有血和水流出来。
\VS{35}看见{\ADD{这事}}的那人就作见证—他的见证也是真的,并且他知道自己所说的是真的—叫你们也可以信。
\VS{36}这些事成了,为要应验经上的话说:「他的骨头一根也不可折断。」
\VS{37}经上又有一句说:「他们要仰望自己所扎的人。」
\par }{\SH 耶稣的安葬
\par }{\R (太27·57—61;可15·42—47;路23·50—56)
\par }{\PP \VS{38}这些事以后,有{\PN{亚利马太}}人{\PN{约瑟}},是耶稣的门徒,只因怕{\PN{犹太}}人,就暗暗地作门徒。他来求{\PN{彼拉多}},要把耶稣的身体领去。{\PN{彼拉多}}允准,他就把耶稣的身体领去了。
\VS{39}又有{\PN{尼哥德慕}},就是先前夜里去见耶稣的,带着没药和沉香约有一百斤前来。
\VS{40}他们就照{\PN{犹太}}人殡葬的规矩,把耶稣的身体用细麻布加上香料裹好了。
\VS{41}在耶稣钉十字架的地方有一个园子,园子里有一座新坟墓,是从来没有葬过人的。
\VS{42}只因是{\PN{犹太}}人的预备日,又因那坟墓近,他们就把耶稣安放在那里。

\par }\Chap{20}{\SH 耶稣复活
\par }{\R (太28·1—8;可16·1—8;路24·1—12)
\par }{\PP \VerseOne{1}七日的第一日清早,天还黑的时候,{\PN{抹大拉}}的{\PN{马利亚}}来到坟墓那里,看见石头从坟墓挪开了,
\VS{2}就跑来见{\PN{西门·彼得}}和耶稣所爱的那个门徒,对他们说:「有人把主从坟墓里挪了去,我们不知道放在哪里。」
\VS{3}{\PN{彼得}}和那门徒就出来,往坟墓那里去。
\VS{4}两个人同跑,那门徒比{\PN{彼得}}跑得更快,先到了坟墓,
\VS{5}低头往里看,就见细麻布还放在那里,只是没有进去。
\VS{6}{\PN{西门·彼得}}随后也到了,进坟墓里去,就看见细麻布还放在那里,
\VS{7}又看见耶稣的裹头巾没有和细麻布放在一处,是另在一处卷着。
\VS{8}先到坟墓的那门徒也进去,看见就信了。(
\VS{9}因为他们还不明白圣经的{\ADD{意思}},就是耶稣必要从死里复活。)
\VS{10}于是两个门徒回自己的住处去了。
\par }{\SH 向抹大拉的马利亚显现
\par }{\R (太28·9—10;可16·9—11)
\par }{\PP \VS{11}{\PN{马利亚}}却站在坟墓外面哭。哭的时候,低头往坟墓里看,
\VS{12}就见两个天使,穿着白衣,在安放耶稣身体的地方坐着,一个在头,一个在脚。
\VS{13}天使对她说:「妇人,你为什么哭?」她说:「因为有人把我主挪了去,我不知道放在哪里。」
\VS{14}说了这话,就转过身来,看见耶稣站在那里,却不知道是耶稣。
\VS{15}耶稣问她说:「妇人,为什么哭?你找谁呢?」{\PN{马利亚}}以为是看园的,就对他说:「先生,若是你把他移了去,请告诉我,你把他放在哪里,我便去取他。」
\VS{16}耶稣说:「{\PN{马利亚}}。」{\PN{马利亚}}就转过来,用希伯来话对他说:「拉波尼!」(拉波尼就是夫子的意思。)
\VS{17}耶稣说:「不要摸我,因我还没有升上去见我的父。你往我弟兄那里去,告诉他们说,我要升上去见我的父,也是你们的父,见我的 神,也是你们的 神。」
\VS{18}{\PN{抹大拉}}的{\PN{马利亚}}就去告诉门徒说:「我已经看见了主。」她又将主对她说的这话告诉他们。
\par }{\SH 向门徒显现
\par }{\R (太28·16—20;可16·14—18;路24·36—49)
\par }{\PP \VS{19}那日(就是七日的第一日)晚上,门徒所在的地方,因怕{\PN{犹太}}人,门都关了。耶稣来,站在当中,对他们说:「愿你们平安!」
\VS{20}说了这话,就把手和肋旁指给他们看。门徒看见主,就喜乐了。
\VS{21}耶稣又对他们说:「愿你们平安!父怎样差遣了我,我也照样差遣你们。」
\VS{22}说了这话,就向他们吹一口气,说:「你们受圣灵!
\VS{23}你们赦免谁的罪,谁的罪就赦免了;你们留下谁的罪,谁的罪就留下了。」
\par }{\SH 耶稣和多马
\par }{\PP \VS{24}那十二个门徒中,有称为{\PN{低土马}}的{\PN{多马}};耶稣来的时候,他没有和他们同在。
\VS{25}那些门徒就对他说:「我们已经看见主了。」{\PN{多马}}却说:「我非看见他手上的钉痕,用指头探入那钉痕,又用手探入他的肋旁,我总不信。」
\VS{26}过了八日,门徒又在屋里,{\PN{多马}}也和他们同在,门都关了。耶稣来,站在当中说:「愿你们平安!」
\VS{27}就对{\PN{多马}}说:「伸过你的指头来,摸\FTNT{}{{\FR 20:27: }原文是看}我的手;伸出你的手来,探入我的肋旁。不要疑惑,总要信!」
\VS{28}{\PN{多马}}说:「我的主!我的 神!」
\VS{29}耶稣对他说:「你因看见了我才信;那没有看见就信的有福了。」
\par }{\SH 本书的目的
\par }{\PP \VS{30}耶稣在门徒面前另外行了许多神迹,没有记在这书上。
\VS{31}但记这些事要叫你们信耶稣是基督,是 神的儿子,并且叫你们信了他,就可以因他的名得生命。

\par }\Chap{21}{\SH 耶稣向七个门徒显现
\par }{\PP \VerseOne{1}这些事以后,耶稣在{\PN{提比哩亚海}}边又向门徒显现。他怎样显现记在下面:
\VS{2}有{\PN{西门·彼得}}和称为{\PN{低土马}}的{\PN{多马}},并{\PN{加利利}}的{\PN{迦拿}}人{\PN{拿但业}},还有{\PN{西庇太}}的两个儿子,又有两个门徒,都在一处。
\VS{3}{\PN{西门·彼得}}对他们说:「我打鱼去。」他们说:「我们也和你同去。」他们就出去,上了船;那一夜并没有打着什么。
\VS{4}天将亮的时候,耶稣站在岸上,门徒却不知道是耶稣。
\VS{5}耶稣就对他们说:「小子!你们有吃的没有?」他们回答说:「没有。」
\VS{6}耶稣说:「你们把网撒在船的右边,就必得着。」他们便撒下网去,竟拉不上来了,因为鱼甚多。
\VS{7}耶稣所爱的那门徒对{\PN{彼得}}说:「是主!」那时{\PN{西门·彼得}}赤着身子,一听见是主,就束上一件外衣,跳在海里。
\VS{8}其余的门徒离岸不远,约有二百肘\FTNT{}{{\FR 21:8: }古时以肘为尺,一肘约有今时尺半},就在小船上把那网鱼拉过来。
\VS{9}他们上了岸,就看见那里有炭火,上面有鱼,又有饼。
\VS{10}耶稣对他们说:「把刚才打的鱼拿几条来。」
\VS{11}{\PN{西门·彼得}}就去\FTNT{}{{\FR 21:11: }或译:上船},把网拉到岸上。那网满了大鱼,共一百五十三条;鱼虽这样多,网却没有破。
\VS{12}耶稣说:「你们来吃早饭。」门徒中没有一个敢问他:「你是谁?」因为知道是主。
\VS{13}耶稣就来拿饼和鱼给他们。
\VS{14}耶稣从死里复活以后,向门徒显现,这是第三次。
\par }{\SH 耶稣和彼得
\par }{\PP \VS{15}他们吃完了早饭,耶稣对{\PN{西门·彼得}}说:「{\PN{约翰}}\FTNT{}{{\FR 21:15: }在马太十六章十七节称约拿}的儿子{\PN{西门}},你爱我比这些更深吗?」{\PN{彼得}}说:「主啊,是的,你知道我爱你。」耶稣对他说:「你喂养我的小羊。」
\VS{16}耶稣第二次又对他说:「{\PN{约翰}}的儿子{\PN{西门}},你爱我吗?」{\PN{彼得}}说:「主啊,是的,你知道我爱你。」耶稣说:「你牧养我的羊。」
\VS{17}第三次对他说:「{\PN{约翰}}的儿子{\PN{西门}},你爱我吗?」{\PN{彼得}}因为耶稣第三次对他说「你爱我吗」,就忧愁,对耶稣说:「主啊,你是无所不知的;你知道我爱你。」耶稣说:「你喂养我的羊。
\VS{18}我实实在在地告诉你,你年少的时候,自己束上{\ADD{带子}},随意往来;但年老的时候,你要伸出手来,别人要把你束上,带你到不愿意去的地方。」
\VS{19}(耶稣说这话是指着{\PN{彼得}}要怎样死,荣耀 神。)说了这话,就对他说:「你跟从我吧!」
\par }{\SH 耶稣和他所爱的那门徒
\par }{\PP \VS{20}{\PN{彼得}}转过来,看见耶稣所爱的那门徒跟着,(就是在晚饭的时候,靠着耶稣胸膛说:「主啊,卖你的是谁?」的那门徒。)
\VS{21}{\PN{彼得}}看见他,就问耶稣说:「主啊,这人将来如何?」
\VS{22}耶稣对他说:「我若要他等到我来的时候,与你何干?你跟从我吧!」
\VS{23}于是这话传在弟兄中间,说那门徒不死。其实,耶稣不是说他不死,乃是说:「我若要他等到我来的时候,与你何干?」
\par }{\PP \VS{24}为这些事作见证,并且记载这些事的就是这门徒;我们也知道他的见证是真的。
\par }{\PP \VS{25}耶稣所行的事还有许多,若是一一地都写出来,我想,所写的书就是世界也容不下了。
\par }