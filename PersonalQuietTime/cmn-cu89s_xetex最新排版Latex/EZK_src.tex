\NormalFont\ShortTitle{以西结书}
{\MT 以西结书

\par }\ChapOne{1}{\SH 以西结所见的第一个异象
\par }{\R (1·1—7·27)
\par }{\SH  神的宝座
\par }{\PP \VerseOne{1}当三十年四{\ADD{月}}初五{\ADD{日}},{\PN{以西结}}\FTNT{}{{\FR 1:1: }原文是我}在{\PN{迦巴鲁河}}边被掳的人中,天就开了,得见 神的异象。
\VS{2}正是{\PN{约雅斤}}王被掳去第五年四月初五日,
\VS{3}在{\PN{迦勒底}}人之地、{\PN{迦巴鲁河}}边,耶和华的话特特临到{\PN{布西}}的儿子祭司{\PN{以西结}};耶和华的灵\FTNT{}{{\FR 1:3: }原文是手}降在他身上。
\par }{\PP \VS{4}我观看,见狂风从北方刮来,随着有一朵包括闪烁火的大云,周围有光辉;从其中的火内发出好像光耀的精金;
\VS{5}又从其中显出四个活物的形象来。他们的形状是这样:有人的形象,
\VS{6}各有四个脸面,四个翅膀。
\VS{7}他们的腿是直的,脚掌好像牛犊之蹄,都灿烂如光明的铜。
\VS{8}在四面的翅膀以下有人的手。这四个活物的脸和翅膀乃是这样:
\VS{9}翅膀彼此相接,行走并不转身,俱各直往前行。
\VS{10}至于脸的形象:{\ADD{前面}}各有人的脸,右面各有狮子的脸,左面各有牛的脸,{\ADD{后面}}各有鹰的脸。
\VS{11}各展开上边的两个翅膀相接,各以{\ADD{下边的}}两个{\ADD{翅膀}}遮体。
\VS{12}他们俱各直往前行。灵往哪里去,他们就往那里去,行走并不转身。
\VS{13}至于四活物的形象,就如烧着火炭的形状,又如火把的形状。{\ADD{火}}在四活物中间上去下来,这火有光辉,从火中发出闪电。
\VS{14}这活物往来奔走,好像电光一闪。
\par }{\PP \VS{15}我正观看活物的时候,见活物的脸旁各有一轮在地上。
\VS{16}轮的形状和颜色\FTNT{}{{\FR 1:16: }原文是作法}好像水苍玉。四轮都是一个样式,形状和作法好像轮中套轮。
\VS{17}轮行走的时候,向四方都能直行,并不掉转。
\VS{18}至于轮辋,高而可畏;四个轮辋周围满有眼睛。
\VS{19}活物行走,轮也在旁边行走;活物从地上升,轮也都上升。
\VS{20}灵往哪里去,活物就往那里去;活物上升,轮也在活物旁边上升,因为活物的灵在轮中。
\VS{21}那些行走,这些也行走;那些站住,这些也站住;那些从地上升,轮也在旁边上升,因为活物的灵在轮中。
\par }{\PP \VS{22}活物的头以上有穹苍的形象,看着像可畏的水晶,铺张在活物的头以上。
\VS{23}穹苍以下,活物的翅膀直张,彼此相对;每活物有两个翅膀遮体。
\VS{24}活物行走的时候,我听见翅膀的响声,像大水的声音,像全能者的声音,也像军队哄嚷的声音。活物站住的时候,便将翅膀垂下。
\VS{25}在他们头以上的穹苍之上有声音。他们站住的时候,便将翅膀垂下。
\par }{\PP \VS{26}在他们头以上的穹苍之上有宝座的形象,仿佛蓝宝石;在宝座形象以上有仿佛人的形状。
\VS{27}我见从他腰以上有仿佛光耀的精金,周围都有火的形状,又见从他腰以下有仿佛火的形状,周围也有光辉。
\VS{28}下雨的日子,云中虹的形状怎样,周围光辉的形状也是怎样。
\par }{\PP 这就是耶和华荣耀的形象。我一看见就俯伏在地,又听见一位说话的声音。

\par }\Chap{2}{\SH  神选召以西结作先知
\par }{\PP \VerseOne{1}他对我说:「人子啊,你站起来,我要和你说话。」
\VS{2}他对我说话的时候,灵就进入我里面,使我站起来,我便听见那位对我说话的{\ADD{声音}}。
\VS{3}他对我说:「人子啊,我差你往悖逆的国民{\PN{以色列}}人那里去。他们是悖逆我的,他们和他们的列祖违背我,直到今日。
\VS{4}这众子面无羞耻,心里刚硬。我差你往他们那里去,你要对他们说:主耶和华如此说。
\VS{5}他们或听,或不听,(他们是悖逆之家),必知道在他们中间有了先知。
\VS{6}人子啊,虽有荆棘和蒺藜在你那里,你又住在蝎子中间,总不要怕他们,也不要怕他们的话;他们虽是悖逆之家,还不要怕他们的话,也不要因他们的脸色惊惶。
\VS{7}他们或听,或不听,你只管将我的话告诉他们;他们是极其悖逆的。
\par }{\PP \VS{8}「人子啊,要听我对你所说的话,不要悖逆像那悖逆之家,你要开口吃我所赐给你的。」
\VS{9}我观看,见有一只手向我伸出来,手中有一书卷。
\VS{10}他将书卷在我面前展开,内外都写着字,其上所写的有哀号、叹息、悲痛的话。

\par }\Chap{3}{\PP \VerseOne{1}他对我说:「人子啊,要吃你所得的,要吃这书卷,好去对{\PN{以色列}}家讲说。」
\VS{2}于是我开口,他就使我吃这书卷,
\VS{3}又对我说:「人子啊,要吃我所赐给你的这书卷,充满你的肚腹。」我就吃了,口中觉得其甜如蜜。
\par }{\PP \VS{4}他对我说:「人子啊,你往{\PN{以色列}}家那里去,将我的话对他们讲说。
\VS{5}你奉差遣不是往那说话深奥、言语难懂的民那里去,乃是往{\PN{以色列}}家去;
\VS{6}不是往那说话深奥、言语难懂的多国去,他们的话语是你不懂得的。我若差你往他们那里去,他们必听从你。
\VS{7}{\PN{以色列}}家却不肯听从你,因为他们不肯听从我;原来{\PN{以色列}}全家是额坚心硬的人。
\VS{8}看哪,我使你的脸硬过他们的脸,使你的额硬过他们的额。
\VS{9}我使你的额像金钢钻,比火石更硬。他们虽是悖逆之家,你不要怕他们,也不要因他们的脸色惊惶。」
\VS{10}他又对我说:「人子啊,我对你所说的一切话,要心里领会,耳中听闻。
\VS{11}你往你本国被掳的子民那里去,他们或听,或不听,你要对他们讲说,告诉他们这是主耶和华说的。」
\par }{\PP \VS{12}那时,灵将我举起,我就听见在我身后有震动轰轰的声音,{\ADD{说}}:「从耶和华的所在显出来的荣耀是该称颂的!」
\VS{13}我又{\ADD{听见}}那活物翅膀相碰,与活物旁边轮子旋转震动轰轰的响声。
\VS{14}于是灵将我举起,带我而去。我心中甚苦,灵性忿激,并且耶和华的灵\FTNT{}{{\FR 3:14: }原文是手}在我身上大有能力。
\VS{15}我就来到{\PN{提勒·亚毕}},住在{\PN{迦巴鲁河}}边被掳的人那里,到他们所住的地方,在他们中间忧忧闷闷地坐了七日。
\par }{\SH 耶和华派以西结作守望者
\par }{\R (33·1—9)
\par }{\PP \VS{16}过了七日,耶和华的话临到我说:
\VS{17}「人子啊,我立你作{\PN{以色列}}家守望的人,所以你要听我口中的话,替我警戒他们。
\VS{18}我何时指着恶人说:他必要死;你若不警戒他,也不劝戒他,使他离开恶行,拯救他的性命,这恶人必死在罪孽之中;我却要向你讨他丧命的罪\FTNT{}{{\FR 3:18: }原文是血}。
\VS{19}倘若你警戒恶人,他仍不转离罪恶,也不离开恶行,他必死在罪孽之中,你却救自己脱离了罪。
\VS{20}再者,义人何时离义而犯罪,我将绊脚石放在他面前,他就必死;因你没有警戒他,他必死在罪中,他素来所行的义不被记念;我却要向你讨他丧命的罪\FTNT{}{{\FR 3:20: }原文是血}。
\VS{21}倘若你警戒义人,使他不犯罪,他就不犯罪;他因受警戒就必存活,你也救自己脱离了罪。」
\par }{\SH 以西结暂成哑巴
\par }{\PP \VS{22}耶和华的灵\FTNT{}{{\FR 3:22: }原文是手}在那里降在我身上。他对我说:「你起来往平原去,我要在那里和你说话。」
\VS{23}于是我起来往平原去,不料,耶和华的荣耀正如我在{\PN{迦巴鲁河}}边所见的一样,停在那里,我就俯伏于地。
\VS{24}灵就进入我里面,使我站起来。耶和华对我说:「你进房屋去,将门关上。
\VS{25}人子啊,人必用绳索捆绑你,你就不能出去在他们中间来往。
\VS{26}我必使你的舌头贴住上膛,以致你哑口,不能作责备他们的人;他们原是悖逆之家。
\VS{27}但我对你说话的时候,必使你开口,你就要对他们说:『主耶和华如此说。』听的可以听,不听的任他不听,因为他们是悖逆之家。」

\par }\Chap{4}{\SH 表演耶路撒冷被困
\par }{\PP \VerseOne{1}「人子啊,你要拿一块砖,摆在你面前,将一座{\PN{耶路撒冷}}城画在其上,
\VS{2}又围困这城,造台筑垒,安营攻击,在四围安设撞锤攻城,
\VS{3}又要拿个铁鏊,放在你和城的中间,作为铁墙。你要对面攻击这城,使城被困;这样,好作{\PN{以色列}}家的预兆。
\par }{\PP \VS{4}「你要向左侧卧,承当{\PN{以色列}}家的罪孽;要{\ADD{按}}你向左侧卧的日数,担当他们的罪孽。
\VS{5}因为我已将他们作孽的年数定为你{\ADD{向左侧卧}}的日数,就是三百九十日,你要这样担当{\PN{以色列}}家的罪孽。
\VS{6}再者,你满了这些日子,还要向右侧卧,担当{\PN{犹大}}家的罪孽。我给你定规{\ADD{侧卧}}四十日,一日顶一年。
\VS{7}你要露出膀臂,面向被困的{\PN{耶路撒冷}},说预言攻击这城。
\VS{8}我用绳索捆绑你,使你不能辗转,直等你满了困{\ADD{城}}的日子。
\par }{\PP \VS{9}「你要取小麦、大麦、豆子、红豆、小米、粗麦,装在一个器皿中,用以为自己做饼;要{\ADD{按}}你侧卧的三百九十日吃这饼。
\VS{10}你所吃的要按分两吃,每日二十舍客勒,按时而吃。
\VS{11}你喝水也要按制子,{\ADD{每日喝}}一欣六分之一,按时而喝。
\VS{12}你吃这饼像吃大麦饼一样,要用人粪在众人眼前烧烤。」
\VS{13}耶和华说:「{\PN{以色列}}人在我所赶他们到的各国中,也必这样吃不洁净的食物。」
\VS{14}我说:「哎!主耶和华啊,我素来未曾被玷污,从幼年到如今没有吃过自死的,或被野兽撕裂的,那可憎的肉也未曾入我的口。」
\VS{15}于是他对我说:「看哪,我给你牛粪代替人粪,你要将你的饼烤在其上。」
\VS{16}他又对我说:「人子啊,我必在{\PN{耶路撒冷}}折断他们的杖,就是断绝他们的粮。他们吃饼要按分两,忧虑而吃;喝水也要按制子,惊惶而喝;
\VS{17}使他们缺粮缺水,彼此惊惶,因自己的罪孽消灭。」

\par }\Chap{5}{\SH 以西结剃发
\par }{\PP \VerseOne{1}「人子啊,你要拿一把快刀,当作剃头刀,用这刀剃你的头发和你的胡须,用天平将须发平分。
\VS{2}围困{\ADD{城}}的日子满了,你要将三分之一在城中用火焚烧,将三分之一在城的四围用刀砍碎,将三分之一任风吹散;我也要拔刀追赶。
\VS{3}你要从其中取几根包在衣襟里,
\VS{4}再从这几根中取些扔在火中焚烧,从里面必有火出来烧入{\PN{以色列}}全家。
\VS{5}主耶和华如此说:这就是{\PN{耶路撒冷}}。我曾将她安置在列邦之中;列国都在她的四围。
\VS{6}她行恶,违背我的典章,过于列国;干犯我的律例,过于四围的列邦,因为她弃掉我的典章。至于我的律例,她并没有遵行。
\VS{7}所以主耶和华如此说:因为你们纷争过于四围的列国,也不遵行我的律例,不谨守我的典章,并以遵从四围列国的恶规尚不满意,
\VS{8}所以主耶和华如此说:看哪,我与你反对,必在列国的眼前,在你中间,施行审判;
\VS{9}并且因你一切可憎的事,我要在你中间行我所未曾行的,以后我也不再照着行。
\VS{10}在你中间父亲要吃儿子,儿子要吃父亲。我必向你施行审判,我必将你所剩下的分散四方\FTNT{}{{\FR 5:10: }方:原文是风}。」
\VS{11}主耶和华说:「我指着我的永生起誓,因你用一切可憎的物、可厌的事玷污了我的圣所,故此,我定要使{\ADD{你人数}}减少,我眼必不顾惜你,也不可怜你。
\VS{12}你的民三分之一必遭瘟疫而死,在你中间必因饥荒消灭;三分之一必在你四围倒在刀下;我必将三分之一分散四方\FTNT{}{{\FR 5:12: }方:原文是风},并要拔刀追赶他们。
\par }{\PP \VS{13}「我要这样成就怒中所定的;我向他们发的忿怒止息了,自己就得着安慰。我在他们身上成就怒中所定的那时,他们就知道我—耶和华所说的是出于热心;
\VS{14}并且我必使你在四围的列国中,在经过的众人眼前,成了荒凉和羞辱。
\VS{15}这样,我必以怒气和忿怒,并烈怒的责备,向你施行审判。那时,你就在四围的列国中成为羞辱、讥刺、警戒、惊骇。这是我—耶和华说的。
\VS{16}那时,我要将灭人、使人饥荒的恶箭,就是射去灭人的,射在你们身上,并要加增你们的饥荒,断绝你们所倚靠的粮食;
\VS{17}又要使饥荒和恶兽到你那里,叫你丧子,瘟疫和流血的事也必盛行在你那里;我也要使刀剑临到你。这是我—耶和华说的。」

\par }\Chap{6}{\SH 谴责拜偶像的罪
\par }{\PP \VerseOne{1}耶和华的话临到我说:
\VS{2}「人子啊,你要面向{\PN{以色列}}的众山说预言,
\VS{3}说:{\PN{以色列}}的众山哪,要听主耶和华的话。主耶和华对大山、小冈、水沟、山谷如此说:我必使刀剑临到你们,也必毁灭你们的邱坛。
\VS{4}你们的祭坛必然荒凉,你们的日像必被打碎。我要使你们被杀的人倒在你们的偶像面前;
\VS{5}我也要将{\PN{以色列}}人的尸首放在他们的偶像面前,将你们的骸骨抛散在你们祭坛的四围。
\VS{6}在你们一切的住处,城邑要变为荒场,邱坛必然凄凉,使你们的祭坛荒废,将你们的偶像打碎。你们的日像被砍倒,你们的工作被毁灭。
\VS{7}被杀的人必倒在你们中间,你们就知道我是耶和华。
\par }{\PP \VS{8}「你们分散在各国的时候,我必在列邦中使你们有剩下脱离刀剑的人。
\VS{9}那脱离{\ADD{刀剑}}的人必在所掳到的各国中记念我,为他们心中何等伤破,是因他们起淫心,远离我,眼对偶像行邪淫。他们因行一切可憎的恶事,必厌恶自己。
\VS{10}他们必知道我是耶和华;我说要使这灾祸临到他们身上,并非空话。
\par }{\PP \VS{11}「主耶和华如此说:你当拍手顿足,说:哀哉!{\PN{以色列}}家行这一切可憎的恶事,他们必倒在刀剑、饥荒、瘟疫之下。
\VS{12}在远处的,必遭瘟疫而死;在近处的,必倒在刀剑之下;那存留被围困的,必因饥荒而死;我必这样在他们身上成就我怒中所定的。
\VS{13}他们被杀的人{\ADD{倒在}}他们祭坛四围的偶像中,就是各高冈、各山顶、各青翠树下、各茂密的橡树下,乃是他们献馨香的祭牲给一切偶像的地方。那时,他们就知道我是耶和华。
\VS{14}我必伸手攻击他们,使他们的地从旷野到{\PN{第伯拉他}}一切住处极其荒凉,他们就知道我是耶和华。」

\par }\Chap{7}{\SH 以色列的终局近了
\par }{\PP \VerseOne{1}耶和华的话又临到我说:
\VS{2}「人子啊,主耶和华对{\PN{以色列}}地如此说:结局到了,结局到了地的四境!
\VS{3}现在你的结局已经临到,我必使我的怒气归与你,也必按你的行为审判你,照你一切可憎的事刑罚你。
\VS{4}我眼必不顾惜你,也不可怜你,却要按你所行的报应你,照你中间可憎的事刑罚你。你就知道我是耶和华。
\par }{\PP \VS{5}「主耶和华如此说:有一灾,独有一灾;看哪,临近了!
\VS{6}结局来了,结局来了,向你兴起。看哪,来到了!
\VS{7}境内的居民哪,所定的灾临到你,时候到了,日子近了,乃是哄嚷并非在山上欢呼的日子。
\VS{8}我快要将我的忿怒倾在你身上,向你成就我怒中所定的,按你的行为审判你,照你一切可憎的事刑罚你。
\VS{9}我眼必不顾惜你,也不可怜你,必按你所行的报应你,照你中间可憎的事刑罚你。你就知道击打你的是我耶和华。
\par }{\PP \VS{10}「看哪,看哪,日子快到了;所定的灾已经发出!杖已经开花,骄傲已经发芽。
\VS{11}强暴兴起,成了罚恶的杖。{\PN{以色列}}人,或是他们的群众,或是他们的财宝,无一{\ADD{存留}};他们中间也没有得尊荣的。
\VS{12}时候到了,日子近了,买主不可欢喜,卖主不可愁烦,因为烈怒已经临到他们众人身上。
\VS{13}卖主虽然存活,却不能归回再得所卖的,因为这异象关乎他们众人。谁都不得归回,也没有人在他的罪孽中坚立自己。」
\par }{\SH 惩罚以色列的罪恶
\par }{\PP \VS{14}「他们已经吹角,预备齐全,却无一人出战,因为我的烈怒临到他们众人身上。
\VS{15}在外有刀剑;在内有瘟疫、饥荒。在田野的,必遭刀剑而死;在城中的,必有饥荒、瘟疫吞灭他。
\VS{16}其中所逃脱的就必逃脱,各人因自己的罪孽在山上发出悲声,好像谷中的鸽子哀鸣。
\VS{17}手都发软,膝弱如水。
\VS{18}要用麻布束腰,被战兢所盖;各人脸上羞愧,头上光秃。
\VS{19}他们要将银子抛在街上,金子看如污秽之物。当耶和华发怒的日子,他们的金银不能救他们,不能使心里知足,也不能使肚腹饱满,因为这金银作了他们罪孽的绊脚石。
\VS{20}论到耶和华妆饰华美{\ADD{的殿}},他建立得威严,他们却在其中制造可憎可厌的偶像,所以{\ADD{这殿}}我使他们看如污秽之物。
\VS{21}我必将{\ADD{这殿}}交付外邦人为掠物,交付地上的恶人为掳物;他们也必亵渎{\ADD{这殿}}。
\VS{22}我必转脸不顾{\PN{以色列}}人,他们亵渎我隐密之{\ADD{所}},强盗也必进去亵渎。
\par }{\PP \VS{23}「要制造锁链;因为这地遍满流血的罪,城邑充满强暴的事。
\VS{24}所以,我必使列国中最恶的人来占据他们的房屋;我必使强暴人的骄傲止息,他们的圣所都要被亵渎。
\VS{25}毁灭临近了;他们要求平安,却无平安可得。
\VS{26}灾害加上灾害,风声接连风声;他们必向先知求异象,但祭司讲的律法、长老设的谋略都必断绝。
\VS{27}君要悲哀,王要披凄凉为衣,国民的手都发颤。我必照他们的行为待他们,按他们应得的审判他们,他们就知道我是耶和华。」

\par }\Chap{8}{\SH 以西结所见的第二个异象
\par }{\R (8·1—10·22)
\par }{\SH 耶路撒冷城里的偶像崇拜
\par }{\PP \VerseOne{1}第六年六{\ADD{月}}初五{\ADD{日}},我坐在家中;{\PN{犹大}}的众长老坐在我面前。在那里主耶和华的灵\FTNT{}{{\FR 8:1: }原文是手}降在我身上。
\VS{2}我观看,见有形象仿佛火的形状,从他腰以下的形状有火,从他腰以上有光辉的形状,仿佛光耀的精金。
\VS{3}他伸出仿佛一只手的样式,抓住我的一绺头发,灵就将我举到天地中间,在 神的异象中,带我到{\PN{耶路撒冷}}朝北的内{\ADD{院}}门口,在那里有触动{\ADD{主}}怒偶像的坐位,就是惹动忌邪的。
\VS{4}谁知,在那里有{\PN{以色列}} 神的荣耀,形状与我在平原所见的一样。
\par }{\PP \VS{5}神对我说:「人子啊,你举目向北观看。」我就举目向北观看,见祭坛门的北边,在门口有这{\ADD{惹}}忌邪的偶像;
\VS{6}又对我说:「人子啊,{\PN{以色列}}家所行的,就是在此行这大可憎的事,使我远离我的圣所,你看见了吗?你还要看见另有大可憎的事。」
\par }{\PP \VS{7}他领我到院门口。我观看,见墙上有个窟窿。
\VS{8}他对我说:「人子啊,你要挖墙。」我一挖墙,见有一门。
\VS{9}他说:「你进去,看他们在这里所行可憎的恶事。」
\VS{10}我进去一看,谁知,在四面墙上画着各样爬物和可憎的走兽,并{\PN{以色列}}家一切的偶像。
\VS{11}在这些像前有{\PN{以色列}}家的七十个长老站立,{\PN{沙番}}的儿子{\PN{雅撒尼亚}}也站在其中。各人手拿香炉,烟云的香气上腾。
\VS{12}他对我说:「人子啊,{\PN{以色列}}家的长老暗中在各人画像屋里所行的,你看见了吗?他们常说:『耶和华看不见我们;耶和华已经离弃这地。』」
\VS{13}他又说:「你还要看见他们另外行大可憎的事。」
\par }{\PP \VS{14}他领我到耶和华殿外院朝北的门口。谁知,在那里有妇女坐着,为{\PN{搭模斯}}哭泣。
\VS{15}他对我说:「人子啊,你看见了吗?你还要看见比这更可憎的事。」
\par }{\PP \VS{16}他又领我到耶和华殿的内院。谁知,在耶和华的殿门口、廊子和祭坛中间,约有二十五个人背向耶和华的殿,面向东方拜日头。
\VS{17}他对我说:「人子啊,你看见了吗?{\PN{犹大}}家在此行这可憎的事还算为小吗?他们在这地遍行强暴,再三惹我发怒,他们手拿枝条举向鼻前。
\VS{18}因此,我也要以忿怒行事,我眼必不顾惜,也不可怜他们;他们虽向我耳中大声呼求,我还是不听。」

\par }\Chap{9}{\SH 耶路撒冷受惩罚
\par }{\PP \VerseOne{1}他向我耳中大声喊叫说:「要使那监管这城的人手中各拿灭命的兵器前来。」
\VS{2}忽然有六个人从朝北的上门而来,各人手拿杀人的兵器;内中有一人身穿细麻衣,腰间带着墨盒子。他们进来,站在铜祭坛旁。
\par }{\PP \VS{3}{\PN{以色列}} 神的荣耀本在基路伯上,现今从那里升到殿的门槛。 神将那身穿细麻衣、腰间带着墨盒子的人召来。
\VS{4}耶和华对他说:「你去走遍{\PN{耶路撒冷}}全城,那些因城中所行可憎之事叹息哀哭的人,画记号在额上。」
\VS{5}我耳中听见他对其余的人说:「要跟随他走遍全城,以行击杀。你们的眼不要顾惜,也不要可怜他们。
\VS{6}要将年老的、年少的,并处女、婴孩,和妇女,从圣所起全都杀尽,只是凡有记号的人不要挨近他。」于是他们从殿前的长老杀起。
\VS{7}他对他们说:「要污秽这殿,使院中充满被杀的人。你们出去吧!」他们就出去,在城中击杀。
\VS{8}他们击杀的时候,我被留下,我就俯伏在地,说:「哎!主耶和华啊,你将忿怒倾在{\PN{耶路撒冷}},岂要将{\PN{以色列}}所剩下的人都灭绝吗?」
\par }{\PP \VS{9}他对我说:「{\PN{以色列}}家和{\PN{犹大}}家的罪孽极其重大。遍地有流血的事,满城有冤屈,因为他们说:『耶和华已经离弃这地,他看不见我们。』
\VS{10}故此,我眼必不顾惜,也不可怜他们,要照他们所行的报应在他们头上。」
\par }{\PP \VS{11}那穿细麻衣、腰间带着墨盒子的人将这事回复说:「我已经照你所吩咐的行了。」

\par }\Chap{10}{\SH 耶和华的荣耀离开圣城
\par }{\PP \VerseOne{1}我观看,见基路伯头上的穹苍之中,显出蓝宝石的形状,仿佛宝座的形象。
\VS{2}主对那穿细麻衣的人说:「你进去,在旋转的轮内基路伯以下,从基路伯中间将火炭取满两手,撒在城上。」
\par }{\PP 我就见他进去。
\VS{3}那人进去的时候,基路伯站在殿的右边,云彩充满了内院。
\VS{4}耶和华的荣耀从基路伯那里上升,{\ADD{停}}在门槛以上;殿内满了云彩,院宇也被耶和华荣耀的光辉充满。
\VS{5}基路伯翅膀的响声听到外院,好像全能 神说话的声音。
\par }{\PP \VS{6}他吩咐那穿细麻衣的人说:「要从旋转的轮内基路伯中间取火。」那人就进去站在一个轮子旁边。
\VS{7}有一个基路伯从基路伯中伸手到基路伯中间的火那里,取些放在那穿细麻衣的人两手中,那人就拿出去了。
\VS{8}在基路伯翅膀之下,显出有人手的样式。
\par }{\PP \VS{9}我又观看,见基路伯旁边有四个轮子。这基路伯旁有一个轮子,那基路伯旁有一个轮子,每基路伯都是如此;轮子的颜色\FTNT{}{{\FR 10:9: }原文是形状}仿佛水苍玉。
\VS{10}至于四轮的形状,都是一个样式,仿佛轮中套轮。
\VS{11}轮行走的时候,向四方都能直行,并不掉转。头向何方,他们也随向何方,行走的时候并不掉转。
\VS{12}他们全身,连背带手和翅膀,并轮周围都满了眼睛。这四个基路伯的轮子都是如此。
\VS{13}至于这些轮子,我耳中听见说是旋转的。
\VS{14}基路伯各有四脸:第一是基路伯的脸,第二是人的脸,第三是狮子的脸,第四是鹰的脸。
\par }{\PP \VS{15}基路伯升上去了;这是我在{\PN{迦巴鲁河}}边所见的活物。
\VS{16}基路伯行走,轮也在旁边行走。基路伯展开翅膀,离地上升,轮也不转离他们旁边。
\VS{17}那些站住,这些也站住;那些上升,这些也一同上升,因为活物的灵在轮中。
\par }{\PP \VS{18}耶和华的荣耀从殿的门槛那里出去,停在基路伯以上。
\VS{19}基路伯出去的时候,就展开翅膀,在我眼前离地上升。轮也在他们的旁边,都停在耶和华殿的东门口。在他们以上有{\PN{以色列}} 神的荣耀。
\par }{\PP \VS{20}这是我在{\PN{迦巴鲁河}}边所见、{\PN{以色列}} 神荣耀以下的活物,我就知道他们是基路伯。
\VS{21}各有四个脸面,四个翅膀,翅膀以下有人手的样式。
\VS{22}至于他们脸的模样,并身体的形象,是我从前在{\PN{迦巴鲁河}}边所看见的。他们俱各直往前行。

\par }\Chap{11}{\SH 耶路撒冷被判罪
\par }{\PP \VerseOne{1}灵将我举起,带到耶和华殿向东的东门。谁知,在门口有二十五个人,我见其中有民间的首领{\PN{押朔}}的儿子{\PN{雅撒尼亚}}和{\PN{比拿雅}}的儿子{\PN{毗拉提}}。
\VS{2}耶和华对我说:「人子啊,这就是图谋罪孽的人,在这城中给人设恶谋。
\VS{3}他们说:『盖房屋的时候尚未临近;这城是锅,我们是肉。』
\VS{4}人子啊,因此你当说预言,说预言攻击他们。」
\par }{\PP \VS{5}耶和华的灵降在我身上,对我说:「你当说,耶和华如此说:{\PN{以色列}}家啊,你们口中所说的,心里所想的,我都知道。
\VS{6}你们在这城中杀人增多,使被杀的人充满街道。
\VS{7}所以主耶和华如此说:你们杀在城中的人就是肉,这城就是锅;你们却要从其中被带出去。
\VS{8}你们怕刀剑,我必使刀剑临到你们。这是主耶和华说的。
\VS{9}我必从这城中带出你们去,交在外邦人的手中,且要在你们中间施行审判。
\VS{10}你们必倒在刀下;我必在{\PN{以色列}}的境界审判你们,你们就知道我是耶和华。
\VS{11}这城必不作你们的锅,你们也不作其中的肉。我必在{\PN{以色列}}的境界审判你们,
\VS{12}你们就知道我是耶和华;因为你们没有遵行我的律例,也没有顺从我的典章,却随从你们四围列国的恶规。」
\par }{\PP \VS{13}我正说预言的时候,{\PN{比拿雅}}的儿子{\PN{毗拉提}}死了。于是我俯伏在地,大声呼叫说:「哎!主耶和华啊,你要将{\PN{以色列}}剩下的人灭绝净尽吗?」
\par }{\SH  神对流亡者的应许
\par }{\PP \VS{14}耶和华的话临到我说:
\VS{15}「人子啊,{\PN{耶路撒冷}}的居民对你的弟兄、你的本族、你的亲属、{\PN{以色列}}全家,就是对大众说:『你们远离耶和华吧!这地是赐给我们为业的。』
\VS{16}所以你当说:『耶和华如此说:我虽将{\PN{以色列}}全家远远迁移到列国中,将他们分散在列邦内,我还要在他们所到的列邦,暂作他们的圣所。』
\VS{17}你当说:『主耶和华如此说:我必从万民中招聚你们,从分散的列国内聚集你们,又要将{\PN{以色列}}地赐给你们。』
\VS{18}他们必到那里,也必从其中除掉一切可憎可厌的物。
\VS{19}我要使他们有合一的心,也要将新灵放在他们里面,又从他们肉体中除掉石心,赐给他们肉心,
\VS{20}使他们顺从我的律例,谨守遵行我的典章。他们要作我的子民,我要作他们的 神。
\VS{21}至于那些心中随从可憎可厌之物的,我必照他们所行的报应在他们头上。这是主耶和华说的。」
\par }{\SH  神的荣耀离开耶路撒冷
\par }{\PP \VS{22}于是,基路伯展开翅膀,轮子都在他们旁边;在他们以上有{\PN{以色列}} 神的荣耀。
\VS{23}耶和华的荣耀从城中上升,停在城东的那座山上。
\VS{24}灵将我举起,在异象中借着 神的灵将我带进{\PN{迦勒底}}地,到被掳的人那里;我所见的异象就离我上升去了。
\VS{25}我便将耶和华所指示我的一切事都说给被掳的人听。

\par }\Chap{12}{\SH 先知扮演难民
\par }{\PP \VerseOne{1}耶和华的话又临到我说:
\VS{2}「人子啊,你住在悖逆的家中。他们有眼睛看不见,有耳朵听不见,因为他们是悖逆之家。
\VS{3}所以人子啊,你要预备掳去使用的物件,在白日当他们眼前从你所住的地方移到别处去;他们虽是悖逆之家,或者可以揣摩思想。
\VS{4}你要在白日当他们眼前带出你的物件去,好像预备掳去使用的物件。到了晚上,你要在他们眼前亲自出去,像被掳的人出去一样。
\VS{5}你要在他们眼前挖通了墙,从其中将{\ADD{物件}}带出去。
\VS{6}到天黑时,你要当他们眼前搭在肩头上带出去,并要蒙住脸看不见地,因为我立你作{\PN{以色列}}家的预兆。」
\par }{\PP \VS{7}我就照着所吩咐的去行,白日带出我的物件,好像预备掳去使用的物件。到了晚上,我用手挖通了墙。天黑的时候,就当他们眼前搭在肩头上带出去。
\par }{\PP \VS{8}次日早晨,耶和华的话临到我说:
\VS{9}「人子啊,{\PN{以色列}}家,就是那悖逆之家,岂不是问你说:『你做什么呢?』
\VS{10}你要对他们说:『主耶和华如此说:这是{\ADD{关乎}}{\PN{耶路撒冷}}的君王和他周围{\PN{以色列}}全家的预表\FTNT{}{{\FR 12:10: }原文是担子}。』
\VS{11}你要说:『我作你们的预兆:我怎样行,他们所遭遇的也必怎样,他们必被掳去。』
\VS{12}他们中间的君王也必在天黑的时候将物件搭在肩头上带出去。他们要挖通了墙,从其中带出去。他必蒙住脸,眼看不见地。
\VS{13}我必将我的网撒在他身上,他必在我的网罗中缠住。我必带他到{\PN{迦勒底}}人之地的{\PN{巴比伦}};他虽死在那里,却看不见那地。
\VS{14}周围一切帮助他的和他所有的军队,我必分散四方\FTNT{}{{\FR 12:14: }方:原文是风},也要拔刀追赶他们。
\VS{15}我将他们四散在列国、分散在列邦的时候,他们就知道我是耶和华。
\VS{16}我却要留下他们几个人得免刀剑、饥荒、瘟疫,使他们在所到的各国中述说他们一切可憎的事,人就知道我是耶和华。」
\par }{\SH 忧虑惊惶的预兆
\par }{\PP \VS{17}耶和华的话又临到我说:
\VS{18}「人子啊,你吃饭必胆战,喝水必惶惶忧虑。
\VS{19}你要对这地的百姓说:主耶和华论{\PN{耶路撒冷}}和{\PN{以色列}}地的居民如此说,他们吃饭必忧虑,喝水必惊惶。因其中居住的众人所行强暴的事,这地必然荒废,一无所存。
\VS{20}有居民的城邑必变为荒场,地也必变为荒废;你们就知道我是耶和华。」
\par }{\SH 平凡的俗语和非凡的信息
\par }{\PP \VS{21}耶和华的话临到我说:
\VS{22}「人子啊,在你们{\PN{以色列}}地怎么有这俗语,说『日子迟延,一切异象都落了空』呢?
\VS{23}你要告诉他们说:『主耶和华如此说:我必使这俗语止息,{\PN{以色列}}中不再用这俗语。』你却要对他们说:『日子临近,一切的异象必都应验。』
\VS{24}从此,在{\PN{以色列}}家中必不再有虚假的异象和奉承的占卜。
\VS{25}我—耶和华说话,所说的必定成就,不再耽延。你们这悖逆之家,我所说的话必趁你们在世的日子成就。这是主耶和华说的。」
\par }{\PP \VS{26}耶和华的话又临到我说:
\VS{27}「人子啊,{\PN{以色列}}家的人说:『他所见的异象是关乎后来许多的日子,所说的预言是指着极远的时候。』
\VS{28}所以你要对他们说:『主耶和华如此说:我的话没有一句再耽延的,我所说的必定成就。这是主耶和华说的。』」

\par }\Chap{13}{\SH 斥责假先知
\par }{\PP \VerseOne{1}耶和华的话临到我说:
\VS{2}「人子啊,你要说预言攻击{\PN{以色列}}中说预言的先知,对那些本己心发预言的说:『你们当听耶和华的话。』」
\VS{3}主耶和华如此说:「愚顽的先知有祸了,他们随从自己的心意,却一无所见。
\VS{4}{\PN{以色列}}啊,你的先知好像荒场中的狐狸,
\VS{5}没有上去堵挡破口,也没有为{\PN{以色列}}家重修墙垣,使他们当耶和华的日子在阵上站立得住。
\VS{6}这些人所见的是虚假,是谎诈的占卜。他们说是耶和华说的,其实耶和华并没有差遣他们,他们倒使人指望那话必然立定。
\VS{7}你们岂不是见了虚假的异象吗?岂不是说了谎诈的占卜吗?你们说,这是耶和华说的,其实我没有说。」
\par }{\PP \VS{8}所以主耶和华如此说:「因你们说的是虚假,见的是谎诈,我就与你们反对。这是主耶和华说的。
\VS{9}我的手必攻击那见虚假异象、用谎诈占卜的先知,他们必不列在我百姓的会中,不录在{\PN{以色列}}家的册上,也不进入{\PN{以色列}}地;你们就知道我是主耶和华。
\VS{10}因为他们诱惑我的百姓,说:『平安!』其实没有平安,就像有人立起墙壁,他们倒用未泡透的{\ADD{灰}}抹上。
\VS{11}所以你要对那些抹上未泡透{\ADD{灰}}的人说:『墙要倒塌,必有暴雨漫过。大冰雹啊,你们要降下,狂风也要吹裂这墙。』
\VS{12}这墙倒塌之后,人岂不问你们说:『你们抹上未泡透的{\ADD{灰}}在哪里呢?』」
\VS{13}所以主耶和华如此说:「我要发怒,使狂风吹裂这墙,在怒中使暴雨漫过,又发怒降下大冰雹,毁灭这墙。
\VS{14}我要这样拆毁你们那未泡透{\ADD{灰}}所抹的墙,拆平到地,以致根基露出,墙必倒塌,你们也必在其中灭亡;你们就知道我是耶和华。
\VS{15}我要这样向墙和用未泡透{\ADD{灰}}抹墙的人成就我怒中所定的,并要对你们说:『墙和抹墙的人都没有了。』
\VS{16}这{\ADD{抹墙的}}就是{\PN{以色列}}的先知,他们指着{\PN{耶路撒冷}}说预言,为这城见了平安的异象,其实没有平安。这是主耶和华说的。」
\par }{\SH 斥责假的女先知
\par }{\PP \VS{17}「人子啊,你要面向本民中、从己心发预言的女子说预言,攻击她们,
\VS{18}说主耶和华如此说:『这些妇女有祸了!她们为众人的膀臂缝靠枕,给高矮之人做下垂的头巾,为要猎取人的性命。难道你们要猎取我百姓的性命,为利己将人救活吗?
\VS{19}你们为两把大麦,为几块饼,在我民中亵渎我,对肯听谎言的民说谎,杀死不该死的人,救活不该活的人。』」
\par }{\PP \VS{20}所以主耶和华如此说:「看哪,我与你们的靠枕反对,就是你们用以猎取人、使人的性命{\ADD{如鸟}}飞的。我要将靠枕从你们的膀臂上扯去,释放你们猎取{\ADD{如鸟}}飞的人。
\VS{21}我也必撕裂你们下垂的头巾,救我百姓脱离你们的手,不再被猎取,落在你们手中。你们就知道我是耶和华。
\VS{22}我不使义人伤心,你们却以谎话使他伤心,又坚固恶人的手,使他不回头离开恶道得以救活。
\VS{23}你们就不再见虚假的异象,也不再行占卜的事;我必救我的百姓脱离你们的手;你们就知道我是耶和华。」

\par }\Chap{14}{\SH  神斥责拜偶像的罪
\par }{\PP \VerseOne{1}有几个{\PN{以色列}}长老到我这里来,坐在我面前。
\VS{2}耶和华的话就临到我说:
\VS{3}「人子啊,这些人已将他们的假神接到心里,把陷于罪的绊脚石放在面前,我岂能丝毫被他们求问吗?
\VS{4}所以你要告诉他们:『主耶和华如此说:{\PN{以色列}}家的人中,凡将他的假神接到心里,把陷于罪的绊脚石放在面前,又就了先知来的,我—耶和华在他所求的事上,必按他众多的假神回答他\FTNT{}{{\FR 14:4: }或译:必按他拜许多假神的罪报应他},
\VS{5}好在{\PN{以色列}}家的心事上捉住他们,因为他们都借着假神与我生疏。』
\par }{\PP \VS{6}「所以你要告诉{\PN{以色列}}家说:『主耶和华如此说:回头吧!离开你们的偶像,转脸莫从你们一切可憎的事。』
\VS{7}因为{\PN{以色列}}家的人,或在{\PN{以色列}}中寄居的外人,凡与我隔绝,将他的假神接到心里,把陷于罪的绊脚石放在面前,又就了先知来要为自己的事求问我的,我—耶和华必亲自回答他。
\VS{8}我必向那人变脸,使他作了警戒,笑谈,令人惊骇,并且我要将他从我民中剪除;你们就知道我是耶和华。
\VS{9}先知若被迷惑说一句预言,是我—耶和华任那先知受迷惑,我也必向他伸手,将他从我民{\PN{以色列}}中除灭。
\VS{10}他们必担当自己的罪孽。先知的罪孽和求问之人的罪孽都是一样,
\VS{11}好使{\PN{以色列}}家不再走迷离开我,不再因各样的罪过玷污自己,只要作我的子民,我作他们的 神。这是主耶和华说的。」
\par }{\SH 挪亚、但以理、约伯
\par }{\PP \VS{12}耶和华的话临到我说:
\VS{13}「人子啊,若有一国犯罪干犯我,我也向他伸手折断他们的杖,就是断绝他们的粮,使饥荒临到那地,将人与牲畜从其中剪除;
\VS{14}其中虽有{\PN{挪亚}}、{\PN{但以理}}、{\PN{约伯}}这三人,他们只能因他们的义救自己的性命。这是主耶和华说的。
\VS{15}我若使恶兽经过糟践那地,使地荒凉,以致因这些兽,人都不得经过;
\VS{16}虽有这三人在其中,主耶和华说:我指着我的永生起誓,他们连儿带女都不能得救,只能自己得救,那地仍然荒凉。
\VS{17}或者我使刀剑临到那地,说:刀剑哪,要经过那地,以致我将人与牲畜从其中剪除;
\VS{18}虽有这三人在其中,主耶和华说:我指着我的永生起誓,他们连儿带女都不能得救,只能自己得救。
\VS{19}或者我叫瘟疫流行那地,使我灭命\FTNT{}{{\FR 14:19: }原文是带血}的忿怒倾在其上,好将人与牲畜从其中剪除;
\VS{20}虽有{\PN{挪亚}}、{\PN{但以理}}、{\PN{约伯}}在其中,主耶和华说:我指着我的永生起誓,他们连儿带女都不能救,只能因他们的义救自己的性命。」
\par }{\PP \VS{21}主耶和华如此说:「我将这四样大灾—就是刀剑、饥荒、恶兽、瘟疫降在{\PN{耶路撒冷}},将人与牲畜从其中剪除,岂不更重吗?
\VS{22}然而其中必有剩下的人,他们连儿带女必带到你们这里来,你们看见他们所行所为的,要因我降给{\PN{耶路撒冷}}的一切灾祸,便得了安慰。
\VS{23}你们看见他们所行所为的,得了安慰,就知道我在{\PN{耶路撒冷}}中所行的并非无故。这是主耶和华说的。」

\par }\Chap{15}{\SH 葡萄树的比喻
\par }{\PP \VerseOne{1}耶和华的话临到我说:
\VS{2}「人子啊,葡萄树比别样树有什么强处?葡萄枝比众树枝有什么好处?
\VS{3}其上可以取木料做什么工用,可以取来做钉子挂什么器皿吗?
\VS{4}看哪,已经抛在火中当作柴烧,火既烧了两头,中间也被烧了,还有益于工用吗?
\VS{5}完全的时候尚且不合乎什么工用,何况被火烧坏,还能合乎什么工用吗?」
\VS{6}所以,主耶和华如此说:「众树以内的葡萄树,我怎样使它在火中当柴,也必照样待{\PN{耶路撒冷}}的居民。
\VS{7}我必向他们变脸;他们虽从火中出来,火却要烧灭他们。我向他们变脸的时候,你们就知道我是耶和华。
\VS{8}我必使地土荒凉,因为他们行事干犯我。这是主耶和华说的。」

\par }\Chap{16}{\SH 不忠实的耶路撒冷
\par }{\PP \VerseOne{1}耶和华的话又临到我说:
\VS{2}「人子啊,你要使{\PN{耶路撒冷}}知道她那些可憎的事,
\VS{3}说主耶和华对{\PN{耶路撒冷}}如此说:你根本,你出世,是在{\PN{迦南}}地;你父亲是{\PN{亚摩利}}人,你母亲是{\PN{赫}}人。
\VS{4}论到你出世的景况,在你初生的日子没有为你断脐带,也没有用水洗你,使你洁净,丝毫没有撒盐在你身上,也没有用布裹你。
\VS{5}谁的眼也不可怜你,为你做一件这样的事怜恤你;但你初生的日子扔在田野,是因你被厌恶。
\par }{\PP \VS{6}「我从你旁边经过,见你滚在血中,就对你说:{\ADD{你虽}}在血中,仍可存活;{\ADD{你虽}}在血中,仍可存活。
\VS{7}我使你生长好像田间所长的,你就渐渐长大,以致极其俊美,两乳成形,头发长成,你却仍然赤身露体。
\par }{\PP \VS{8}「我从你旁边经过,看见你的时候正动爱情,便用衣襟搭在你身上,遮盖你的赤体;又向你起誓,与你结盟,你就归于我。这是主耶和华说的。
\VS{9}那时我用水洗你,洗净你身上的血,又用油抹你。
\VS{10}我也使你身穿绣花衣服,脚穿海狗皮鞋,并用细麻布给你束腰,用丝绸为衣披在你身上,
\VS{11}又用妆饰打扮你,将镯子戴在你手上,将金链戴在你项上。
\VS{12}我也将环子戴在你鼻子上,将耳环戴在你耳朵上,将华冠戴在你头上。
\VS{13}这样,你就有金银的妆饰,穿的是细麻衣和丝绸,并绣花衣;吃的是细面、蜂蜜,并油。你也极其美貌,发达到王后的尊荣。
\VS{14}你美貌的名声传在列邦中,你十分美貌,是因我加在你身上的威荣。这是主耶和华说的。
\par }{\PP \VS{15}「只是你仗着自己的美貌,又因你的名声就行邪淫。你纵情淫乱,使过路的任意而行。
\VS{16}你用衣服为自己在高处结彩,在其上行邪淫。{\ADD{这样的事}}将来必没有,也必不再行了。
\VS{17}你又将我所给你那华美的金银、宝器为自己制造人像,与他行邪淫;
\VS{18}又用你的绣花衣服给他披上,并将我的膏油和香料摆在他跟前;
\VS{19}又将我赐给你的食物,就是我赐给你吃的细面、油,和蜂蜜,都摆在他跟前为馨香{\ADD{的供物}}。这是主耶和华说的。
\VS{20}并且你将给我所生的儿女焚献给他。
\VS{21}你行淫乱岂是小事,竟将我的儿女杀了,使他们经{\ADD{火}}归与他吗?
\VS{22}你行这一切可憎和淫乱的事,并未追念你幼年赤身露体滚在血中的日子。」
\par }{\SH 耶路撒冷像淫妇
\par }{\PP \VS{23}「你行这一切恶事之后(主耶和华说:你有祸了!有祸了!)
\VS{24}又为自己建造圆顶花楼,在各街上做了高台。
\VS{25}你在一切市口上建造高台,使你的美貌变为可憎的,又与一切过路的多行淫乱。
\VS{26}你也和你邻邦放纵情欲的{\PN{埃及}}人行淫,加增你的淫乱,惹我发怒。
\VS{27}因此我伸手攻击你,减少你应用的{\ADD{粮食}},又将你交给恨你的{\PN{非利士}}众女\FTNT{}{{\FR 16:27: }众女是城邑的意思;本章下同},使她们任意待你。她们见你的淫行,为你羞耻。
\VS{28}你因贪色无厌,又与{\PN{亚述}}人行淫,与他们行淫之后,仍不满意,
\VS{29}并且多行淫乱,直到那贸易之地,就是{\PN{迦勒底}},你仍不满意。
\par }{\PP \VS{30}「主耶和华说:你行这一切事,都是不知羞耻妓女所行的,可见你的心是何等懦弱!
\VS{31}因你在一切市口上建造圆顶花楼,在各街上做了高台,你却藐视赏赐,不像妓女。
\VS{32}哎!你这行淫的妻啊,宁肯接外人,不接丈夫。
\VS{33}凡妓女是得人赠送,你反倒赠送你所爱的人,贿赂他们从四围来与你行淫。
\VS{34}你行淫与{\ADD{别的}}妇女相反,因为不是人从你行淫;你既赠送人,人并不赠送你;所以你与别的妇女相反。」
\par }{\SH  神审判耶路撒冷
\par }{\PP \VS{35}「你这妓女啊,要听耶和华的话。
\VS{36}主耶和华如此说:因你的污秽倾泄了,你与你所爱的行淫露出下体,又因你拜一切可憎的偶像,流儿女的血献给他,
\VS{37}我就要将你一切相欢相爱的和你一切所恨的都聚集来,从四围攻击你;又将你的下体露出,使他们看尽了。
\VS{38}我也要审判你,好像{\ADD{官长}}审判淫妇和流人血的妇女一样。我因忿怒忌恨,使流血的罪归到你身上。
\VS{39}我又要将你交在他们手中;他们必拆毁你的圆顶花楼,毁坏你的高台,剥去你的衣服,夺取你的华美宝器,留下你赤身露体。
\VS{40}他们也必带多人来攻击你,用石头打死你,用刀剑刺透你,
\VS{41}用火焚烧你的房屋,在许多妇人眼前向你施行审判。我必使你不再行淫,也不再赠送与人。
\VS{42}这样,我就止息向你发的忿怒,我的忌恨也要离开你,我要安静不再恼怒。
\VS{43}因你不追念你幼年的日子,在这一切的事上向我发烈怒,所以我必照你所行的报应在你头上,你就不再贪淫,行那一切可憎的事。这是主耶和华说的。」
\par }{\SH 有其母必有其女
\par }{\PP \VS{44}「凡说俗语的必用俗语攻击你,说:『母亲怎样,女儿也怎样。』
\VS{45}你正是你母亲的女儿,厌弃丈夫和儿女;你正是你姊妹的姊妹,厌弃丈夫和儿女。你母亲是{\PN{赫}}人,你父亲是{\PN{亚摩利}}人。
\VS{46}你的姊姊是{\PN{撒马利亚}},她和她的众女住在你左边;你的妹妹是{\PN{所多玛}},她和她的众女住在你右边。
\VS{47}你没有效法她们的行为,也没有照她们可憎的事去做,你以那为小事,你一切所行的倒比她们更坏。
\VS{48}主耶和华说:我指着我的永生起誓,你妹妹{\PN{所多玛}}与她的众女尚未行你和你众女所行的事。
\VS{49}看哪,你妹妹{\PN{所多玛}}的罪孽是这样:她和她的众女都心骄气傲,粮食饱足,大享安逸,并没有扶助困苦和穷乏人的手。
\VS{50}她们狂傲,在我面前行可憎的事,我看见便将她们除掉。
\VS{51}{\PN{撒马利亚}}没有犯你一半的罪,你行可憎的事比她更多,使你的姊妹因你所行一切可憎的事,倒显为义。
\VS{52}你既断定你姊妹为义\FTNT{}{{\FR 16:52: }为义:或译当受羞辱},就要担当自己的羞辱;因你所犯的罪比她们更为可憎,她们就比你更显为义;你既使你的姊妹显为义,你就要抱愧担当自己的羞辱。」
\par }{\SH 所多玛、撒马利亚要复兴
\par }{\PP \VS{53}「我必叫她们被掳的归回,就是叫{\PN{所多玛}}和她的众女,{\PN{撒马利亚}}和她的众女,并你们中间被掳的,都要归回,
\VS{54}好使你担当自己的羞辱,并因你一切所行的使她们得安慰,你就抱愧。
\VS{55}你的妹妹{\PN{所多玛}}和她的众女必归回原位;{\PN{撒马利亚}}和她的众女,你和你的众女,也必归回原位。
\VS{56-57}在你骄傲的日子,你的恶行没有显露以先,你的口就不提你的妹妹{\PN{所多玛}}。那受了凌辱的{\PN{亚兰}}众女和{\PN{亚兰}}四围{\PN{非利士}}的众女都恨恶你,藐视你。
\VS{58}耶和华说:你贪淫和可憎的事,你已经担当了。」
\par }{\SH 永远的约
\par }{\PP \VS{59}「主耶和华如此说:你这轻看誓言、背弃盟约的,我必照你所行的待你。
\VS{60}然而我要追念在你幼年时与你所立的约,也要与你立定永约。
\VS{61}你接待你姊姊和你妹妹的时候,你要追念你所行的,自觉惭愧;并且我要将她们赐你为女儿,却不是按着前约。
\VS{62}我要坚定与你所立的约(你就知道我是耶和华),
\VS{63}好使你在我赦免你一切所行的时候,心里追念,自觉抱愧,又因你的羞辱就不再开口。这是主耶和华说的。」

\par }\Chap{17}{\SH 老鹰和葡萄树的比喻
\par }{\PP \VerseOne{1}耶和华的话临到我说:
\VS{2}「人子啊,你要向{\PN{以色列}}家出谜语,设比喻,
\VS{3}说主耶和华如此说:有一大鹰,翅膀大,翎毛长,羽毛丰满,彩色俱备,来到{\PN{黎巴嫩}},将香柏树梢拧去,
\VS{4}就是折去香柏树尽尖的嫩枝,叼到贸易之地,放在买卖城中;
\VS{5}又将{\PN{以色列}}地的枝子栽于肥田里,插在大水旁,如插柳树,
\VS{6}就渐渐生长,成为蔓延矮小的葡萄树。其枝转向那鹰,其根在鹰以下,于是成了葡萄树,生出枝子,发出小枝。
\par }{\PP \VS{7}「又有一大鹰,翅膀大,羽毛多。这葡萄树从栽种的畦中向这鹰弯过根来,发出枝子,好得它的浇灌。
\VS{8}这树栽于肥田多水的旁边,好生枝子,结果子,成为佳美的葡萄树。
\VS{9}你要说,主耶和华如此说:这葡萄树岂能发旺呢?鹰岂不拔出它的根来,芟除它的果子,使它枯干,使它发的嫩叶都枯干了吗?也不用大力和多民,就拔出它的根来。
\VS{10}葡萄树虽然栽种,岂能发旺呢?一经东风,岂不全然枯干吗?必在生长的畦中枯干了。」
\par }{\SH 解释比喻
\par }{\PP \VS{11}耶和华的话临到我说:
\VS{12}「你对那悖逆之家说:你们不知道这些事是什么意思吗?你要告诉他们说,{\PN{巴比伦}}王曾到{\PN{耶路撒冷}},将其中的君王和首领带到{\PN{巴比伦}}自己那里去。
\VS{13}从{\PN{以色列}}的宗室中取一人与他立约,使他发誓,并将国中有势力的人掳去,
\VS{14}使国低微不能自强,惟因守盟约得以存立。
\VS{15}他却背叛{\PN{巴比伦}}王,打发使者往{\PN{埃及}}去,要他们给他马匹和多民。他岂能亨通呢?行这样事的人岂能逃脱呢?他背约岂能逃脱呢?
\VS{16}他轻看向王所起的誓,背弃王与他所立的约。主耶和华说:我指着我的永生起誓,他定要死在立他作王、{\PN{巴比伦}}王的京都。
\VS{17}敌人筑垒造台,与他打仗的时候,为要剪除多人,法老虽领大军队和大群众,还是不能帮助他。
\VS{18}他轻看誓言,背弃盟约,已经投降,却又做这一切的事,他必不能逃脱。」
\VS{19}所以主耶和华如此说:「我指着我的永生起誓,他既轻看指我所起的誓,背弃指我所立的约,我必要使这罪归在他头上。
\VS{20}我必将我的网撒在他身上,他必在我的网罗中缠住。我必带他到{\PN{巴比伦}},并要在那里因他干犯我的罪刑罚他。
\VS{21}他的一切军队,凡逃跑的,都必倒在刀下;所剩下的,也必分散四方\FTNT{}{{\FR 17:21: }方:原文是风}。你们就知道说这话的是我—耶和华。」
\par }{\SH  神对将来的应许
\par }{\PP \VS{22}主耶和华如此说:「我要将香柏树梢拧去栽上,就是从尽尖的嫩枝中折一嫩枝,栽于极高的山上;
\VS{23}在{\PN{以色列}}高处的山栽上。它就生枝子,结果子,成为佳美的香柏树,各类飞鸟都必宿在其下,就是宿在枝子的荫下。
\VS{24}田野的树木都必知道我—耶和华使高树矮小,矮树高大;青树枯干,枯树发旺。我—耶和华如此说,也如此行了。」

\par }\Chap{18}{\PP \VerseOne{1}耶和华的话又临到我说:
\VS{2}「你们在{\PN{以色列}}地怎么用这俗语说『父亲吃了酸葡萄,儿子的牙酸倒了』呢?」
\VS{3}主耶和华说:「我指着我的永生起誓,你们在{\PN{以色列}}中,必不再有用这俗语的{\ADD{因由}}。
\VS{4}看哪,世人都是属我的;为父的怎样属我,为子的也照样属我;犯罪的,他必死亡。
\par }{\PP \VS{5}「人若是公义,且行正直与合理的事:
\VS{6}未曾在山上吃过{\ADD{祭偶像之物}},未曾仰望{\PN{以色列}}家的偶像,未曾玷污邻舍的妻,未曾在妇人的经期内亲近她,
\VS{7}未曾亏负人,乃将欠债之人的当头还给他;未曾抢夺人的物件,却将食物给饥饿的人吃,将衣服给赤身的人穿;
\VS{8}未曾向借钱的{\ADD{弟兄}}取利,也未曾向借粮的{\ADD{弟兄}}多要,缩手不作罪孽,在两人之间,按至理判断;
\VS{9}遵行我的律例,谨守我的典章,按诚实行事—这人是公义的,必定存活。这是主耶和华说的。
\par }{\PP \VS{10-11}「他若生一个儿子,作强盗,是流人血的,不行以上所说之善,反行其中之恶,乃在山上吃过{\ADD{祭偶像之物}},并玷污邻舍的妻,
\VS{12}亏负困苦和穷乏的人,抢夺人的物,未曾将当头还给人,仰望偶像,并行可憎的事,
\VS{13}向借钱的{\ADD{弟兄}}取利,向借粮的弟兄多要—这人岂能存活呢?他必不能存活。他行这一切可憎的事,必要死亡,他的罪\FTNT{}{{\FR 18:13: }原文是血}必归到他身上。
\par }{\PP \VS{14}「他若生一个儿子,见父亲所犯的一切罪便惧怕\FTNT{}{{\FR 18:14: }有古卷:思量},不照样去做;
\VS{15}未曾在山上吃过{\ADD{祭偶像之物}},未曾仰望{\PN{以色列}}家的偶像,未曾玷污邻舍的妻,
\VS{16}未曾亏负人,未曾取人的当头,未曾抢夺人的物件,却将食物给饥饿的人吃,将衣服给赤身的人穿,
\VS{17}缩手不害贫穷人,未曾向借钱的{\ADD{弟兄}}取利,也未曾向借粮的{\ADD{弟兄}}多要;他顺从我的典章,遵行我的律例,就不因父亲的罪孽死亡,定要存活。
\VS{18}至于他父亲;因为欺人太甚,抢夺弟兄,在本国的民中行不善,他必因自己的罪孽死亡。
\par }{\PP \VS{19}「你们还说:『儿子为何不担当父亲的罪孽呢?』儿子行正直与合理的事,谨守遵行我的一切律例,他必定存活。
\VS{20}惟有犯罪的,他必死亡。儿子必不担当父亲的罪孽,父亲也不担当儿子的罪孽。义人的善果必归自己,恶人的恶报也必归自己。
\par }{\PP \VS{21}「恶人若回头离开所做的一切罪恶,谨守我一切的律例,行正直与合理的事,他必定存活,不致死亡。
\VS{22}他所犯的一切罪过都不被记念,因所行的义,他必存活。
\VS{23}主耶和华说:恶人死亡,岂是我喜悦的吗?不是喜悦他回头离开所行的道存活吗?
\VS{24}义人若转离义行而作罪孽,照着恶人所行一切可憎的事而行,他岂能存活吗?他所行的一切义都不被记念;他必因所犯的罪、所行的恶死亡。
\par }{\PP \VS{25}「你们还说:『主的道不公平!』{\PN{以色列}}家啊,你们当听,我的道岂不公平吗?你们的道岂不是不公平吗?
\VS{26}义人若转离义行而作罪孽死亡,他是因所作的罪孽死亡。
\VS{27}再者,恶人若回头离开所行的恶,行正直与合理的事,他必将性命救活了。
\VS{28}因为他思量,回头离开所犯的一切罪过,必定存活,不致死亡。
\VS{29}{\PN{以色列}}家还说:『主的道不公平!』{\PN{以色列}}家啊,我的道岂不公平吗?你们的道岂不是不公平吗?」
\par }{\PP \VS{30}所以主耶和华说:「{\PN{以色列}}家啊,我必按你们各人所行的审判你们。你们当回头离开所犯的一切罪过。这样,罪孽必不使你们败亡。
\VS{31}你们要将所犯的一切罪过尽行抛弃,自做一个新心和新灵。{\PN{以色列}}家啊,你们何必死亡呢?
\VS{32}主耶和华说:我不喜悦那死人之死,所以你们当回头而存活。」

\par }\Chap{19}{\SH 挽歌
\par }{\PP \VerseOne{1}你当为{\PN{以色列}}的王作起哀歌,
\VS{2}说:
\par }{\Q 你的母亲是什么呢?
\par }{\Q 是个母狮子,蹲伏在狮子中间,
\par }{\Q 在少壮狮子中养育小狮子。
\par }{\Q \VS{3}在它小狮子中养大一个,
\par }{\Q 成了少壮狮子,
\par }{\Q 学会抓食而吃人。
\par }{\Q \VS{4}列国听见了就把它捉在他们的坑中,
\par }{\Q 用钩子拉到{\PN{埃及}}地去。
\par }{\Q \VS{5}母狮见自己等候失了指望,
\par }{\Q 就从它小狮子中又将一个养为少壮狮子。
\par }{\Q \VS{6}它在众狮子中走来走去,
\par }{\Q 成了少壮狮子,
\par }{\Q 学会抓食而吃人。
\par }{\Q \VS{7}它知道列国的宫殿,
\par }{\Q 又使他们的城邑变为荒场;
\par }{\Q 因它咆哮的声音,
\par }{\Q 遍地和其中所有的就都荒废。
\par }{\Q \VS{8}于是四围邦国各省的人来攻击它,
\par }{\Q 将网撒在它身上,
\par }{\Q 捉在他们的坑中。
\par }{\Q \VS{9}他们用钩子钩住它,将它放在笼中,
\par }{\Q 带到{\PN{巴比伦}}王那里,
\par }{\Q 将它放入坚固之所,
\par }{\Q 使它的声音在{\PN{以色列}}山上不再听见。
\par }{\BB \par }{\Q \VS{10}你的母亲先前如葡萄树,
\par }{\Q 极其茂盛\FTNT{}{{\FR 19:10: }原文是在你血中},栽于水旁。
\par }{\Q 因为水多,
\par }{\Q 就多结果子,满生枝子;
\par }{\Q \VS{11}生出坚固的枝干,可作掌权者的杖。
\par }{\Q 这枝干高举在茂密的枝中,
\par }{\Q 而且它生长高大,枝子繁多,
\par }{\Q {\ADD{远远}}可见。
\par }{\Q \VS{12}但这葡萄树因忿怒被拔出摔在地上;
\par }{\Q 东风吹干其上的果子,
\par }{\Q 坚固的枝干折断枯干,
\par }{\Q 被火烧毁了;
\par }{\Q \VS{13}如今栽于旷野干旱无水之地。
\par }{\Q \VS{14}火也从它枝干中发出,
\par }{\Q 烧灭果子,
\par }{\Q 以致没有坚固的枝干可做掌权者的杖。
\par }{\BB \par }{\Q 这是哀歌,也必用以作哀歌。

\par }\Chap{20}{\SH  神的旨意,人的抗命
\par }{\PP \VerseOne{1}第七年五{\ADD{月}}初十{\ADD{日}},有{\PN{以色列}}的几个长老来求问耶和华,坐在我面前。
\VS{2}耶和华的话临到我说:
\VS{3}「人子啊,你要告诉{\PN{以色列}}的长老说,主耶和华如此说:你们来是求问我吗?主耶和华说:我指着我的永生起誓,我必不被你们求问。
\VS{4}人子啊,你要审问审问他们吗?你当使他们知道他们列祖那些可憎的事,
\VS{5}对他们说,主耶和华如此说:当日我拣选{\PN{以色列}},向{\PN{雅各}}家的后裔起誓,在{\PN{埃及}}地将自己向他们显现,说:我是耶和华—你们的 神。
\VS{6}那日我向他们起誓,必领他们出{\PN{埃及}}地,到我为他们察看的流奶与蜜之地;那地在万国中是有荣耀的。
\VS{7}我对他们说,你们各人要抛弃眼所喜爱那可憎之物,不可因{\PN{埃及}}的偶像玷污自己。我是耶和华—你们的 神。
\VS{8}他们却悖逆我,不肯听从我,不抛弃他们眼所喜爱那可憎之物,不离弃{\PN{埃及}}的偶像。
\par }{\PP 「我就说,我要将我的忿怒倾在他们身上,在{\PN{埃及}}地向他们成就我怒中所定的。
\VS{9}我却为我名的缘故没有这样行,免得我名在他们所住的列国人眼前被亵渎;我领他们出{\PN{埃及}}地,在这列国人的眼前将自己向他们显现。
\VS{10}这样,我就使他们出{\PN{埃及}}地,领他们到旷野,
\VS{11}将我的律例赐给他们,将我的典章指示他们;人若遵行就必因此活着。
\VS{12}又将我的安息日赐给他们,好在我与他们中间为证据,使他们知道我—耶和华是叫他们成为圣的。
\VS{13}{\PN{以色列}}家却在旷野悖逆我,不顺从我的律例,厌弃我的典章(人若遵行就必因此活着),大大干犯我的安息日。
\par }{\PP 「我就说,要在旷野将我的忿怒倾在他们身上,灭绝他们。
\VS{14}我却为我名的缘故,没有这样行,免得我的名在我领他们出{\PN{埃及}}的列国人眼前被亵渎。
\VS{15}并且我在旷野向他们起誓,必不领他们进入我所赐给他们流奶与蜜之地(那地在万国中是有荣耀的);
\VS{16}因为他们厌弃我的典章,不顺从我的律例,干犯我的安息日,他们的心随从自己的偶像。
\VS{17}虽然如此,我眼仍顾惜他们,不毁灭他们,不在旷野将他们灭绝净尽。
\par }{\PP \VS{18}「我在旷野对他们的儿女说:不要遵行你们父亲的律例,不要谨守他们的恶规,也不要因他们的偶像玷污自己。
\VS{19}我是耶和华—你们的 神,你们要顺从我的律例,谨守遵行我的典章,
\VS{20}且以我的安息日为圣。这日在我与你们中间为证据,使你们知道我是耶和华—你们的 神。
\VS{21}只是{\ADD{他们的}}儿女悖逆我,不顺从我的律例,也不谨守遵行我的典章(人若遵行就必因此活着),干犯我的安息日。
\par }{\PP 「我就说,要将我的忿怒倾在他们身上,在旷野向他们成就我怒中所定的。
\VS{22}虽然如此,我却为我名的缘故缩手没有这样行,免得我的名在我领他们出{\PN{埃及}}的列国人眼前被亵渎。
\VS{23}并且我在旷野向他们起誓,必将他们分散在列国,四散在列邦;
\VS{24}因为他们不遵行我的典章,竟厌弃我的律例,干犯我的安息日,眼目仰望他们父亲的偶像。
\VS{25}我也任他们遵行不美的律例,谨守不能使人活着的恶规。
\VS{26}因他们将一切头生的经{\ADD{火}},我就任凭他们在这供献的事上玷污自己,好叫他们凄凉,使他们知道我是耶和华。
\par }{\PP \VS{27}「人子啊,你要告诉{\PN{以色列}}家说,主耶和华如此说:你们的列祖在得罪我的事上亵渎我;
\VS{28}因为我领他们到了我起誓应许赐给他们的地,他们看见各高山、各茂密树,就在那里献祭,奉上惹我发怒的供物,也在那里焚烧馨香的祭牲,并浇上奠祭。
\VS{29}我就对他们说:你们所上的那高处叫什么呢?(那高处的名字叫{\PN{巴麻}}直到今日。)
\VS{30}所以你要对{\PN{以色列}}家说,主耶和华如此说:你们仍照你们列祖所行的玷污自己吗?仍照他们可憎的事行邪淫吗?
\VS{31}你们奉上供物使你们儿子经火的时候,仍将一切偶像玷污自己,直到今日吗?{\PN{以色列}}家啊,我岂被你们求问吗?主耶和华说:我指着我的永生起誓,我必不被你们求问。
\par }{\PP \VS{32}「你们说:我们要像外邦人和列国的宗族一样,去事奉木头与石头。你们所起的这心意万不能成就。」
\par }{\SH  神的惩罚和赦免
\par }{\PP \VS{33}主耶和华说:「我指着我的永生起誓,我总要作王,用大能的手和伸出来的膀臂,并倾出来的忿怒,治理你们。
\VS{34}我必用大能的手和伸出来的膀臂,并倾出来的忿怒,将你们从万民中领出来,从分散的列国内聚集你们。
\VS{35}我必带你们到外邦人的旷野,在那里当面刑罚你们。
\VS{36}我怎样在{\PN{埃及}}地的旷野刑罚你们的列祖,也必照样刑罚你们。这是主耶和华说的。
\VS{37}我必使你们从杖下经过,使你们被约拘束。
\VS{38}我必从你们中间除净叛逆和得罪我的人,将他们从所寄居的地方领出来,他们却不得入{\PN{以色列}}地。你们就知道我是耶和华。
\par }{\PP \VS{39}「{\PN{以色列}}家啊,至于你们,主耶和华如此说:从此以后若不听从我,就任凭你们去事奉偶像,只是不可再因你们的供物和偶像亵渎我的圣名。
\par }{\PP \VS{40}「主耶和华说:在我的圣山,就是{\PN{以色列}}高处的山,所有{\PN{以色列}}的全家都要事奉我。我要在那里悦纳你们,向你们要供物和初熟的土产,并一切的圣物。
\VS{41}我从万民中领你们出来,从分散的列国内聚集你们,那时我必悦纳你们好像馨香之祭,要在外邦人眼前在你们身上显为圣。
\VS{42}我领你们进入{\PN{以色列}}地,就是我起誓应许赐给你们列祖之地,那时你们就知道我是耶和华。
\VS{43}你们在那里要追念玷污自己的行动作为,又要因所做的一切恶事厌恶自己。
\VS{44}主耶和华说:{\PN{以色列}}家啊,我为我名的缘故,不照着你们的恶行和你们的坏事待你们;你们就知道我是耶和华。」
\par }{\SH 南方大火
\par }{\PP \VS{45}耶和华的话临到我说:
\VS{46}「人子啊,你要面向南方,向南滴下预言攻击南方田野的树林。
\VS{47}对南方的树林说,要听耶和华的话。主耶和华如此说:我必使火在你中间着起,烧灭你中间的一切青树和枯树,猛烈的火焰必不熄灭。从南到北,人的脸面都被烧焦。
\VS{48}凡有血气的都必知道是我—耶和华使火着起,这火必不熄灭。」
\VS{49}于是我说:「哎!主耶和华啊,人都指着我说:他岂不是说比喻的吗?」

\par }\Chap{21}{\SH 耶和华的刀
\par }{\PP \VerseOne{1}耶和华的话临到我说:
\VS{2}「人子啊,你要面向{\PN{耶路撒冷}}和圣所滴下预言,攻击{\PN{以色列}}地。
\VS{3}对{\PN{以色列}}地说,耶和华如此说:我与你为敌,并要拔刀出鞘,从你中间将义人和恶人一并剪除。
\VS{4}我既要从你中间剪除义人和恶人,所以我的刀要出鞘,自南至北攻击一切有血气的;
\VS{5}一切有血气的就知道我—耶和华已经拔刀出鞘,必不再入鞘。
\VS{6}人子啊,你要叹息,在他们眼前弯着腰,苦苦地叹息。
\VS{7}他们问你说:『为何叹息呢?』你就说:『因为有风声、{\ADD{灾祸}}要来。人心都必消化,手都发软,精神衰败,膝弱如水。看哪,这{\ADD{灾祸}}临近,必然成就。这是主耶和华说的。』」
\par }{\PP \VS{8}耶和华的话临到我说:
\VS{9}「人子啊,你要预言。耶和华吩咐我如此说:
\par }{\Q 有刀、有刀,
\par }{\Q 是磨快擦亮的;
\par }{\Q \VS{10}磨快为要行杀戮,
\par }{\Q 擦亮为要像闪电。
\par }{\MM 我们岂可快乐吗?{\ADD{罚}}我子的杖藐视各树。
\VS{11}这刀已经交给人擦亮,为要应手使用。这刀已经磨快擦亮,好交在行杀戮的人手中。
\VS{12}人子啊,你要呼喊哀号,因为这刀临到我的百姓和{\PN{以色列}}一切的首领。他们和我的百姓都交在刀下,所以你要拍腿{\ADD{叹息}}。
\VS{13}有试验的事;若那藐视的杖归于无有,怎么样呢?这是主耶和华说的。
\par }{\PP \VS{14}「人子啊,你要拍掌预言。我—{\ADD{耶和华}}要使这刀,就是致死伤的刀,一连三次加倍{\ADD{刺人}},进入他们的内屋,使大人受死伤的就是这刀。
\VS{15}我设立这恐吓人的刀,攻击他们的一切城门,使他们的心消化,加增他们跌倒的事。哎!这刀造得像闪电,磨得尖利,要行杀戮。
\VS{16}刀啊,你归在右边,摆在左边;你面向哪方,就向那方{\ADD{杀戮}}。
\VS{17}我也要拍掌,并要使我的忿怒止息。这是我—耶和华说的。」
\par }{\SH 巴比伦王的刀
\par }{\PP \VS{18}耶和华的话又临到我说:
\VS{19}「人子啊,你要定出两条路,好使{\PN{巴比伦}}王的刀来。这两条路必从一地分出来,又要在通城的路口上画出一只手来。
\VS{20}你要定出一条路,使刀来到{\PN{亚扪}}人的{\PN{拉巴}};又要定出一条路,使刀来到{\PN{犹大}}的坚固城{\PN{耶路撒冷}}。
\VS{21}因为{\PN{巴比伦}}王站在岔路那里,在两条路口上要占卜。他摇签\FTNT{}{{\FR 21:21: }原文是箭}求问神像,察看{\ADD{牺牲的}}肝;
\VS{22}在右手中拿着为{\PN{耶路撒冷}}占卜的签,使他安设撞城锤,张口叫杀,扬声呐喊,筑垒造台,以撞城锤,攻打城门。
\VS{23}据那些曾起誓的{\PN{犹大}}人看来,这是虚假的占卜;但{\PN{巴比伦}}王要使他们想起罪孽,以致将他们捉住。」
\par }{\PP \VS{24}主耶和华如此说:「因你们的过犯显露,使你们的罪孽被记念,以致你们的罪恶在行为上都彰显出来;又因你们被记念,就被捉住。
\VS{25}你这受死伤行恶的{\PN{以色列}}王啊,罪孽的尽头到了,受报的日子已到。
\VS{26}主耶和华如此说:当除掉冠,摘下冕,景况必不再像先前;要使卑者升为高,使高者降为卑。
\VS{27}我要将{\ADD{这国}}倾覆,倾覆,而又倾覆;这{\ADD{国}}也必不再有,直等到那应得的人来到,我就赐给{\ADD{他}}。」
\par }{\SH 刀和亚扪人
\par }{\PP \VS{28}「人子啊,要发预言说:主耶和华论到{\PN{亚扪}}人和他们的凌辱,吩咐我如此说:有刀,有拔出来的刀,已经擦亮,为行杀戮,使他像闪电以行吞灭。
\VS{29}人为你见虚假的异象,行谎诈的占卜,使你倒在受死伤之恶人的颈项上。他们罪孽到了尽头,受报的日子已到。
\VS{30}你将刀收入鞘吧!在你受造之处、生长之地,我必刑罚你。
\VS{31}我必将我的恼恨倒在你身上,将我烈怒的火喷在你身上;又将你交在善于杀灭的畜类人手中。
\VS{32}你必当柴被火焚烧;你的血必{\ADD{流}}在国中。你必不再被记念,因为这是我—耶和华说的。」

\par }\Chap{22}{\SH 耶路撒冷的过犯
\par }{\PP \VerseOne{1}耶和华的话又临到我说:
\VS{2}「人子啊,你要审问审问这流人血的城吗?当使她知道她一切可憎的事。
\VS{3}你要说,主耶和华如此说:哎!这城有流人血的事在其中,叫她受报的日期来到,又做偶像玷污自己,陷害自己。
\VS{4}你因流了人的血,就为有罪;你做了偶像,就玷污自己,使你受报之日临近,报应之年来到。所以我叫你受列国的凌辱和列邦的讥诮。
\VS{5}你这名臭、多乱的城啊,那些离你近、离你远的都必讥诮你。
\par }{\PP \VS{6}「看哪,{\PN{以色列}}的首领各逞其能,在你中间流人之血。
\VS{7}在你中间有轻慢父母的,有欺压寄居的,有亏负孤儿寡妇的。
\VS{8}你藐视了我的圣物,干犯了我的安息日。
\VS{9}在你中间有谗谤人、流人血的;有在山上吃过{\ADD{祭偶像之物}}的,有行淫乱的。
\VS{10}在你中间有露{\ADD{继}}母下体羞辱父亲的,有玷辱月经不洁净之妇人的。
\VS{11}这人与邻舍的妻行可憎的事;那人贪淫玷污儿妇;还有玷辱同父之姊妹的。
\VS{12}在你中间有为流人血受贿赂的;有向借钱的{\ADD{弟兄}}取利,向借粮的{\ADD{弟兄}}多要的。且因贪得无厌,欺压邻舍夺取财物,竟忘了我。这是主耶和华说的。
\par }{\PP \VS{13}「看哪,我因你所得不义之财和你中间所流的血,就拍掌{\ADD{叹息}}。
\VS{14}到了我惩罚你的日子,你的心还能忍受吗?你的手还能有力吗?我—耶和华说了这话,就必照着行。
\VS{15}我必将你分散在列国,四散在列邦。我也必从你中间除掉你的污秽。
\VS{16}你必在列国人的眼前因自己所行的被亵渎,你就知道我是耶和华。」
\par }{\SH  神的大炼炉
\par }{\PP \VS{17}耶和华的话临到我说:
\VS{18}「人子啊,{\PN{以色列}}家在我看为渣滓。他们都是炉中的铜、锡、铁、铅,都是银渣滓。
\VS{19}所以主耶和华如此说:因你们都成为渣滓,我必聚集你们在{\PN{耶路撒冷}}中。
\VS{20}人怎样将银、铜、铁、铅、锡聚在炉中,吹火熔化;照样,我也要发怒气和忿怒,将你们聚集放在城中,熔化你们。
\VS{21}我必聚集你们,把我烈怒的火吹在你们身上,你们就在其中熔化。
\VS{22}银子怎样熔化在炉中,你们也必照样熔化在城中,你们就知道我—耶和华是将忿怒倒在你们身上了。」
\par }{\SH 以色列领袖们的罪恶
\par }{\PP \VS{23}耶和华的话临到我说:
\VS{24}「人子啊,你要对这地说:你是未得洁净之地,在恼恨的日子也没有雨下在你以上。
\VS{25}其中的先知同谋背叛,如咆哮的狮子抓撕掠物。他们吞灭人民,抢夺财宝,使这地多有寡妇。
\VS{26}其中的祭司强解我的律法,亵渎我的圣物,不分别圣的和俗的,也不使人分辨洁净的和不洁净的,又遮眼不顾我的安息日;我也在他们中间被亵慢。
\VS{27}其中的首领仿佛豺狼抓撕掠物,杀人流血,伤害人命,要得不义之财。
\VS{28}其中的先知为百姓用未泡透的{\ADD{灰}}抹墙,就是为他们见虚假的异象,用谎诈的占卜,说:『主耶和华如此说』,其实耶和华没有说。
\VS{29}国内众民一味地欺压,惯行抢夺,亏负困苦穷乏的,背理欺压寄居的。
\VS{30}我在他们中间寻找一人重修墙垣,在我面前为这国站在破口防堵,使我不灭绝这国,却找不着一个。
\VS{31}所以我将恼恨倒在他们身上,用烈怒的火灭了他们,照他们所行的报应在他们头上。这是主耶和华说的。」

\par }\Chap{23}{\SH 两个犯罪的姊妹
\par }{\PP \VerseOne{1}耶和华的话又临到我说:
\VS{2}「人子啊,有两个女子,是一母所生,
\VS{3}她们在{\PN{埃及}}行邪淫,在幼年时行邪淫。她们在那里作处女的时候,有人拥抱她们的怀,抚摸她们的乳。
\VS{4}她们的名字,姊姊名叫{\PN{阿荷拉}},妹妹名叫{\PN{阿荷利巴}}。她们都归于我,生了儿女。论到她们的名字,{\PN{阿荷拉}}就是{\PN{撒马利亚}},{\PN{阿荷利巴}}就是{\PN{耶路撒冷}}。
\par }{\PP \VS{5}「{\PN{阿荷拉}}归我之后行邪淫,贪恋所爱的人,就是她的邻邦{\PN{亚述}}人。
\VS{6}这些人都穿蓝衣,作省长、副省长,都骑着马,是可爱的少年人。
\VS{7}{\PN{阿荷拉}}就与{\PN{亚述}}人中最美的男子放纵淫行,她因所恋爱之人的一切偶像,玷污自己。
\VS{8}自从在{\PN{埃及}}的时候,她就没有离开淫乱,因为她年幼作处女的时候,{\PN{埃及}}人与她行淫,抚摸她的乳,纵欲与她行淫。
\VS{9}因此,我将她交在她所爱的人手中,就是她所恋爱的{\PN{亚述}}人手中。
\VS{10}他们就露了她的下体,掳掠她的儿女,用刀杀了她,使她在妇女中留下臭名,因他们向她施行审判。
\par }{\PP \VS{11}「她妹妹{\PN{阿荷利巴}}虽然看见了,却还贪恋,比她姊姊更丑;行淫乱比她姊姊更多。
\VS{12}她贪恋邻邦的{\PN{亚述}}人,就是穿极华美的衣服,骑着马的省长、副省长,都是可爱的少年人。
\VS{13}我看见她被玷污了,她姊妹二人同行一路。
\VS{14}{\PN{阿荷利巴}}又加增淫行,因她看见人像画在墙上,就是用丹色所画{\PN{迦勒底}}人的像,
\VS{15}腰间系着带子,头上有下垂的裹头巾,都是军长的形状,仿照{\PN{巴比伦}}人的形象;他们的故土就是{\PN{迦勒底}}。
\VS{16}{\PN{阿荷利巴}}一看见就贪恋他们,打发使者往{\PN{迦勒底}}去见他们。
\VS{17}{\PN{巴比伦}}人就来登她爱情的床,与她行淫玷污她。她被玷污,随后心里与他们生疏。
\VS{18}这样,她显露淫行,又显露下体;我心就与她生疏,像先前与她姊姊生疏一样。
\VS{19}她还加增她的淫行,追念她幼年在{\PN{埃及}}地行邪淫的日子,
\VS{20}贪恋情人身壮精足,如驴如马。
\VS{21}这样,你就想起你幼年的淫行。那时,{\PN{埃及}}人拥抱你的怀,抚摸你的乳。」
\par }{\SH  神审判妹妹
\par }{\PP \VS{22}{\PN{阿荷利巴}}啊,主耶和华如此说:「我必激动你先爱而后生疏的人来攻击你。我必使他们来,在你四围攻击你。
\VS{23}所来的就是{\PN{巴比伦}}人、{\PN{迦勒底}}的众人、{\PN{比割}}人、{\PN{书亚}}人、{\PN{哥亚}}人,同着他们的还有{\PN{亚述}}众人,乃是作省长、副省长、作军长有名声的,都骑着马,是可爱的少年人。
\VS{24}他们必带兵器、战车、辎重车,率领大众来攻击你。他们要拿大小盾牌,顶盔摆阵,在你四围攻击你。我要将审判的事交给他们,他们必按着自己的条例审判你。
\VS{25}我必以忌恨攻击你;他们必以忿怒办你。他们必割去你的鼻子和耳朵;你遗留\FTNT{}{{\FR 23:25: }或译:余剩;下同}的人必倒在刀下。他们必掳去你的儿女;你所遗留的必被火焚烧。
\VS{26}他们必剥去你的衣服,夺取你华美的宝器。
\VS{27}这样,我必使你的淫行和你从{\PN{埃及}}地{\ADD{染}}来的淫乱止息了,使你不再仰望{\PN{亚述}},也不再追念{\PN{埃及}}。
\VS{28}主耶和华如此说:我必将你交在你所恨恶的人手中,就是你心与他生疏的人手中。
\VS{29}他们必以恨恶办你,夺取你一切劳碌得来的,留下你赤身露体。你淫乱的下体,连你的淫行,带你的淫乱,都被显露。
\VS{30}人必向你行这些事;因为你随从外邦人行邪淫,被他们的偶像玷污了。
\VS{31}你走了你姊姊所走的路,所以我必将她的杯交在你手中。」
\par }{\Q \VS{32}主耶和华如此说:
\par }{\Q 你必喝你姊姊所喝的杯;
\par }{\Q 那杯又深又广,盛得甚多,
\par }{\Q 使你被人嗤笑讥刺。
\par }{\Q \VS{33}你必酩酊大醉,满有愁苦,
\par }{\Q 喝干你姊姊{\PN{撒马利亚}}的杯,
\par }{\Q 就是令人惊骇凄凉的杯。
\par }{\Q \VS{34}你必喝这杯,以致喝尽。
\par }{\Q {\ADD{杯破}}又啃杯片,
\par }{\Q 撕裂自己的乳;
\par }{\Q 因为这事我曾说过。
\par }{\Q 这是主耶和华说的。
\par }{\PP \VS{35}主耶和华如此说:「因你忘记我,将我丢在背后,所以你要担当你淫行和淫乱{\ADD{的报应}}。」
\par }{\SH  神审判两姊妹
\par }{\PP \VS{36}耶和华又对我说:「人子啊,你要审问{\PN{阿荷拉}}与{\PN{阿荷利巴}}吗?当指出她们所行可憎的事。
\VS{37}她们行淫,手中有{\ADD{杀人的}}血,又与偶像行淫,并使她们为我所生的儿女经{\ADD{火}}烧{\ADD{给偶像}}。
\VS{38}此外,她们还有向我所行的,就是同日玷污我的圣所,干犯我的安息日。
\VS{39}她们杀了儿女献与偶像,当天又入我的圣所,将圣所亵渎了。她们在我殿中所行的乃是如此。
\par }{\PP \VS{40}「况且你们二妇打发使者去请远方人。使者到他们那里,他们就来了。你们为他们沐浴己身,粉饰眼目,佩戴妆饰,
\VS{41}坐在华美的床上,前面摆设桌案,将我的香料膏油摆在其上。
\VS{42}在那里有群众安逸欢乐的声音,并有粗俗的人和酒徒从旷野同来,把镯子戴在{\ADD{二妇}}的手上,把华冠戴在她们的头上。
\par }{\PP \VS{43}「我论这行淫衰老的妇人说:现在人还要与她行淫,她也要{\ADD{与人}}行淫。
\VS{44}人与{\PN{阿荷拉}},并{\PN{阿荷利巴}}二淫妇苟合,好像与妓女苟合。
\VS{45}必有义人,照审判淫妇和流人血的妇人之例,审判她们;因为她们是淫妇,手中有{\ADD{杀人的}}血。」
\par }{\PP \VS{46}主耶和华如此说:「我必使多人来攻击她们,使她们抛来抛去,被人抢夺。
\VS{47}这些人必用石头打死她们,用刀剑杀害她们,又杀戮她们的儿女,用火焚烧她们的房屋。
\VS{48}这样,我必使淫行从境内止息,好叫一切妇人都受警戒,不效法你们的淫行。
\VS{49}人必照着你们的淫行报应你们;你们要担当{\ADD{拜}}偶像的罪,就知道我是主耶和华。」

\par }\Chap{24}{\SH 生锈的锅
\par }{\PP \VerseOne{1}第九年十月初十日,耶和华的话又临到我说:
\VS{2}「人子啊,今日正是{\PN{巴比伦}}王就近{\PN{耶路撒冷}}的日子,你要将这日记下,
\VS{3}要向这悖逆之家设比喻说,主耶和华如此说:
\par }{\Q 将锅放在火上,
\par }{\Q 放好了,就倒水在其中;
\par }{\Q \VS{4}将肉块,就是一切肥美的肉块、腿,
\par }{\Q 和肩都聚在其中,
\par }{\Q 拿美好的骨头把锅装满;
\par }{\Q \VS{5}取羊群中最好的,
\par }{\Q 将{\ADD{柴}}堆在{\ADD{锅}}下,
\par }{\Q 使锅开滚,
\par }{\Q 好把骨头煮在其中。
\par }{\PP \VS{6}「主耶和华如此说:祸哉!这流人血的城,就是长锈的锅。其中的锈未曾除掉,须要将肉块从其中一一取出来,不必为它拈阄。
\VS{7}城中所流的血倒在净光的磐石上,不倒在地上,用土掩盖。
\VS{8}这城中所流的血倒在净光的磐石上,不得掩盖,乃是出于我,为要发忿怒施行报应。
\VS{9}所以主耶和华如此说:祸哉!这流人血的城,我也必大堆{\ADD{火柴}},
\VS{10}添上木柴,使火着旺,将肉煮烂,把汤熬浓,使骨头烤焦;
\VS{11}把锅倒空坐在炭火上,使锅烧热,使铜烧红,熔化其中的污秽,除净其上的锈。
\VS{12}这锅劳碌疲乏,所长的大锈仍未除掉;这锈就是用火{\ADD{也不能除掉}}。
\VS{13}在你污秽中有淫行,我洁净你,你却不洁净。你的污秽再不能洁净,直等我向你发的忿怒止息。
\VS{14}我—耶和华说过的必定成就,必照话而行,必不返回,必不顾惜,也不后悔。人必照你的举动行为审判你。这是主耶和华说的。」
\par }{\SH 先知妻子的死
\par }{\PP \VS{15}耶和华的话又临到我说:
\VS{16}「人子啊,我要将你眼目所喜爱的忽然取去,你却不可悲哀哭泣,也不可流泪,
\VS{17}只可叹息,不可出声,不可办理丧事;头上仍勒裹头巾,脚上仍穿鞋,不可蒙着嘴唇,也不可吃吊丧的食物。」
\VS{18}于是我将这事早晨告诉百姓,晚上我的妻就死了。次日早晨我便遵命而行。
\par }{\PP \VS{19}百姓问我说:「你这样行与我们有什么关系,你不告诉我们吗?」
\VS{20}我回答他们:「耶和华的话临到我说:
\VS{21}『你告诉{\PN{以色列}}家,主耶和华如此说:我必使我的圣所,就是你们势力所夸耀、眼里所喜爱、心中所爱惜的被亵渎,并且你们所遗留的儿女必倒在刀下。
\VS{22}那时,你们必行我{\ADD{仆人}}所行的,不蒙着嘴唇,也不吃吊丧的食物。
\VS{23}你们仍要头上勒裹头巾,脚上穿鞋;不可悲哀哭泣。你们必因自己的罪孽相对叹息,渐渐消灭。
\VS{24}{\PN{以西结}}必这样为你们作预兆;凡他所行的,你们也必照样行。那事来到,你们就知道我是主耶和华。』
\par }{\PP \VS{25}「人子啊,我除掉他们所倚靠、所欢喜的荣耀,并眼中所喜爱、心里所重看的儿女。
\VS{26}那日逃脱的人岂不来到你这里,使你耳闻这事吗?
\VS{27}你必向逃脱的人开口说话,不再哑口。你必这样为他们作预兆,他们就知道我是耶和华。」

\par }\Chap{25}{\SH 预言亚扪受惩罚
\par }{\PP \VerseOne{1}耶和华的话临到我说:
\VS{2}「人子啊,你要面向{\PN{亚扪}}人说预言,攻击他们,
\VS{3}说:你们当听主耶和华的话。主耶和华如此说:我的圣所被亵渎,{\PN{以色列}}地变荒凉,{\PN{犹大}}家被掳掠;那时,你便因这些事说:『阿哈!』
\VS{4}所以我必将你的地交给东方人为业;他们必在你的地上安营居住,吃你的果子,喝你的奶。
\VS{5}我必使{\PN{拉巴}}为骆驼场,使{\PN{亚扪}}人的地为羊群躺卧之处,你们就知道我是耶和华。」
\VS{6}主耶和华如此说:「因你拍手顿足,以满心的恨恶向{\PN{以色列}}地欢喜,
\VS{7}所以我伸手攻击你,将你交给列国作为掳物。我必从万民中剪除你,使你从万国中败亡。我必除灭你,你就知道我是耶和华。」
\par }{\SH 斥责摩押
\par }{\PP \VS{8}主耶和华如此说:「因{\PN{摩押}}和{\PN{西珥}}人说:『看哪,{\PN{犹大}}家与列国无异』,
\VS{9}所以我要破开{\PN{摩押}}边界上的城邑,就是{\PN{摩押}}人看为本国之荣耀的{\PN{伯·耶西末}}、{\PN{巴力·免}}、{\PN{基列亭}},
\VS{10}好使东方人来攻击{\PN{亚扪}}人。我必将{\PN{亚扪}}人之地交给他们为业,使{\PN{亚扪}}人在列国中不再被记念。
\VS{11}我必向{\PN{摩押}}施行审判,他们就知道我是耶和华。」
\par }{\SH 斥责以东
\par }{\PP \VS{12}主耶和华如此说:「因为{\PN{以东}}报仇雪恨,攻击{\PN{犹大}}家,向他们报仇,大大有罪,
\VS{13}所以主耶和华如此说:我必伸手攻击{\PN{以东}},剪除人与牲畜,使{\PN{以东}}从{\PN{提幔}}起,人必倒在刀下,地要变为荒凉,直到{\PN{底但}}。
\VS{14}我必借我民{\PN{以色列}}的手报复{\PN{以东}};{\PN{以色列}}民必照我的怒气,按我的忿怒在{\PN{以东}}施报,{\PN{以东}}人就知道是我施报。这是主耶和华说的。」
\par }{\SH 斥责非利士
\par }{\PP \VS{15}主耶和华如此说:「因{\PN{非利士}}人向{\PN{犹大}}人报仇,就是以恨恶的心报仇雪恨,永怀仇恨,要毁灭他们,
\VS{16}所以主耶和华如此说:我必伸手攻击{\PN{非利士}}人,剪除{\PN{基利提}}人,灭绝沿海剩下的居民。
\VS{17}我向他们大施报应,发怒斥责他们。我报复他们的时候,他们就知道我是耶和华。」

\par }\Chap{26}{\SH 斥责泰尔
\par }{\PP \VerseOne{1}第十一年{\ADD{十一}}月初一日,耶和华的话临到我说:
\VS{2}「人子啊,因{\PN{泰尔}}向{\PN{耶路撒冷}}说:『阿哈,那作众民之门的已经破坏,向我开放;她既变为荒场,我必丰盛。』
\VS{3}所以,主耶和华如此说:{\PN{泰尔}}啊,我必与你为敌,使许多国民上来攻击你,如同海使波浪涌上来一样。
\VS{4}他们必破坏{\PN{泰尔}}的墙垣,拆毁她的城楼。我也要刮净尘土,使她成为净光的磐石。
\VS{5}她必在海中作晒网的地方,也必成为列国的掳物。这是主耶和华说的。
\VS{6}属{\PN{泰尔}}城邑的居民\FTNT{}{{\FR 26:6: }原文是田间的众女;八节同}必被刀剑杀灭,他们就知道我是耶和华。」
\par }{\PP \VS{7}主耶和华如此说:「我必使诸王之王的{\PN{巴比伦}}王{\PN{尼布甲尼撒}}率领马匹车辆、马兵、军队,和许多人民从北方来攻击{\ADD{你}}{\PN{泰尔}}。
\VS{8}他必用刀剑杀灭属你城邑的居民,也必造台筑垒举盾牌攻击你。
\VS{9}他必安设撞城锤攻破你的墙垣,用铁器拆毁你的城楼。
\VS{10}因他的马匹众多,尘土扬起遮蔽你。他进入你的城门,好像人进入已有破口之城。那时,你的墙垣必因骑马的和战车、辎重车的响声震动。
\VS{11}他的马蹄必践踏你一切的街道;他必用刀杀戮你的居民。你坚固的柱子\FTNT{}{{\FR 26:11: }或译:柱像}必倒在地上。
\VS{12}人必以你的财宝为掳物,以你的货财为掠物,破坏你的墙垣,拆毁你华美的房屋,将你的石头、木头、尘土都抛在水中。
\VS{13}我必使你唱歌的声音止息;人也不再听见你弹琴的声音。
\VS{14}我必使你成为净光的磐石,作晒网的地方。你不得再被建造,因为这是主耶和华说的。」
\par }{\PP \VS{15}主耶和华对{\PN{泰尔}}如此说:「在你中间行杀戮,受伤之人唉哼的时候,因你倾倒的响声,海岛岂不都震动吗?
\VS{16}那时靠海的君王必都下位,除去朝服,脱下花衣,披上战兢,坐在地上,时刻发抖,为你惊骇。
\VS{17}他们必为你作起哀歌说:
\par }{\Q 你这有名之城,
\par }{\Q 素为航海之人居住,
\par }{\Q 在海上为最坚固的;
\par }{\Q 平日你和居民使一切住在那里的人无不惊恐;
\par }{\Q 现在何竟毁灭了?
\par }{\Q \VS{18}如今在你这倾覆的日子,
\par }{\Q 海岛都必战兢;
\par }{\Q 海中的群岛见你归于无有就都惊惶。」
\par }{\PP \VS{19}主耶和华如此说:「{\PN{泰尔}}啊,我使你变为荒凉,如无人居住的城邑;又使深水漫过你,大水淹没你。
\VS{20}那时,我要叫你下入阴府,与古时的人一同在地的深处、久已荒凉之地居住,使你不再有居民。我也要在活人之地显荣耀\FTNT{}{{\FR 26:20: }我也......荣耀:或译在活人之地不再有荣耀}。
\VS{21}我必叫你令人惊恐,不再存留于世;人虽寻找你,却永寻不见。这是主耶和华说的。」

\par }\Chap{27}{\SH 哀悼泰尔的挽歌
\par }{\PP \VerseOne{1}耶和华的话又临到我说:
\VS{2}「人子啊,要为{\PN{泰尔}}作起哀歌,
\VS{3}说:你居住海口,是众民的商埠;你的交易通到许多的海岛。主耶和华如此说:
\par }{\Q {\PN{泰尔}}啊,你曾说:
\par }{\Q 我是全然美丽的。
\par }{\Q \VS{4}你的境界在海中,
\par }{\Q 造你的使你全然美丽。
\par }{\Q \VS{5}他们用{\PN{示尼珥}}的松树做你的一切板,
\par }{\Q 用{\PN{黎巴嫩}}的香柏树做桅杆,
\par }{\Q \VS{6}用{\PN{巴珊}}的橡树做你的桨,
\par }{\Q 用象牙镶嵌{\PN{基提}}海岛的黄杨木为坐板\FTNT{}{{\FR 27:6: }或译:舱板}。
\par }{\Q \VS{7}你的篷帆是用{\PN{埃及}}绣花细麻布做的,
\par }{\Q 可以做你的大旗;
\par }{\Q 你的凉棚是用{\PN{以利沙岛}}的蓝色、紫色{\ADD{布}}做的。
\par }{\Q \VS{8}{\PN{西顿}}和{\PN{亚发}}的居民作你荡桨的。
\par }{\Q {\PN{泰尔}}啊,你中间的智慧人作掌舵的。
\par }{\Q \VS{9}{\PN{迦巴勒}}的老者和聪明人
\par }{\Q 都在你中间作补缝的;
\par }{\Q 一切泛海的船只和水手
\par }{\Q 都在你中间经营交易的事。
\par }{\PP \VS{10}「{\PN{波斯}}人、{\PN{路德}}人、{\PN{弗}}人在你军营中作战士;他们在你中间悬挂盾牌和头盔,彰显你的尊荣。
\VS{11}{\PN{亚发}}人和你的军队都在你四围的墙上,你的望楼也有勇士;他们悬挂盾牌,成全你的美丽。
\par }{\PP \VS{12}「{\PN{他施}}人因你多有各类的财物,就作你的客商,拿银、铁、锡、铅兑换你的货物。
\VS{13}{\PN{雅完}}人、{\PN{土巴}}人、{\PN{米设}}人都与你交易;他们用人口和铜器兑换你的货物。
\VS{14}{\PN{陀迦玛}}族用马和战马并骡子兑换你的货物。
\VS{15}{\PN{底但}}人与你交易,许多海岛作你的码头;他们拿象牙、乌木与你兑换\FTNT{}{{\FR 27:15: }或译:进贡}。
\VS{16}{\PN{亚兰}}人因你的工作很多,就作你的客商;他们用绿宝石、紫色{\ADD{布}}绣货、细麻布、珊瑚、红宝石兑换你的货物。
\VS{17}{\PN{犹大}}和{\PN{以色列}}地的人都与你交易;他们用{\PN{米匿}}的麦子、饼、蜜、油、乳香兑换你的货物。
\VS{18}{\PN{大马士革}}人因你的工作很多,又因你多有各类的财物,就拿{\PN{黑本}}酒和白羊毛与你交易。
\VS{19}{\PN{威但}}人和{\PN{雅完}}人拿纺成的线、亮铁、桂皮、菖蒲兑换你的货物。
\VS{20}{\PN{底但}}人用高贵的毯子、鞍、屉与你交易。
\VS{21}{\PN{阿拉伯}}人和{\PN{基达}}的一切首领都作你的客商,用羊羔、公绵羊、公山羊与你交易。
\VS{22}{\PN{示巴}}和{\PN{拉玛}}的商人与你交易,他们用各类上好的香料、各类的宝石,和黄金兑换你的货物。
\VS{23}{\PN{哈兰}}人、{\PN{干尼}}人、{\PN{伊甸}}人、{\PN{示巴}}的商人,和{\PN{亚述}}人、{\PN{基抹}}人与你交易。
\VS{24}这些商人以美好的货物包在绣花蓝色包袱内,又有华丽的衣服装在香柏木的箱子里,用绳捆着与你交易。
\par }{\Q \VS{25}{\PN{他施}}的船只接连成帮为你运货,
\par }{\Q 你便在海中丰富极其荣华。
\par }{\Q \VS{26}荡桨的已经把你荡到大水之处,
\par }{\Q 东风在海中将你打破。
\par }{\Q \VS{27}你的资财、物件、货物、
\par }{\Q 水手、掌舵的、
\par }{\Q 补缝的、经营交易的,
\par }{\Q 并你中间的战士和人民,
\par }{\Q 在你破坏的日子必都沉在海中。
\par }{\Q \VS{28}你掌舵的呼号之声一发,
\par }{\Q 郊野都必震动。
\par }{\Q \VS{29}凡荡桨的和水手,
\par }{\Q 并一切泛海掌舵的,
\par }{\Q 都必下船登岸。
\par }{\Q \VS{30}他们必为你放声痛哭,
\par }{\Q 把尘土撒在头上,
\par }{\Q 在灰中打滚;
\par }{\Q \VS{31}又为你使头上光秃,
\par }{\Q 用麻布束腰,
\par }{\Q 号咷痛哭,
\par }{\Q 苦苦悲哀。
\par }{\Q \VS{32}他们哀号的时候,
\par }{\Q 为你作起哀歌哀哭,
\par }{\Q {\ADD{说}}:有何城如{\PN{泰尔}}?
\par }{\Q 有何城如她在海中成为寂寞的呢?
\par }{\Q \VS{33}你由海上运出货物,
\par }{\Q 就使许多国民充足;
\par }{\Q 你以许多资财、货物
\par }{\Q 使地上的君王丰富。
\par }{\Q \VS{34}你在深水中被海浪打破的时候,
\par }{\Q 你的货物和你中间的一切人民,
\par }{\Q 就都沉下去了。
\par }{\Q \VS{35}海岛的居民为你惊奇;
\par }{\Q 他们的君王都甚恐慌,
\par }{\Q 面带愁容。
\par }{\Q \VS{36}各国民中的客商都向你发嘶声;
\par }{\Q 你令人惊恐,
\par }{\Q 不再存留于世,直到永远。」

\par }\Chap{28}{\SH 斥责泰尔王
\par }{\PP \VerseOne{1}耶和华的话又临到我说:
\VS{2}「人子啊,你对{\PN{泰尔}}君王说,主耶和华如此说:
\par }{\Q 因你心里高傲,说:我是神;
\par }{\Q 我在海中坐神之位。
\par }{\Q 你虽然居心自比神,
\par }{\Q 也不过是人,并不是神!
\par }{\Q \VS{3}看哪,你比{\PN{但以理}}更有智慧,
\par }{\Q 什么秘事都不能向你隐藏。
\par }{\Q \VS{4}你靠自己的智慧聪明得了金银财宝,
\par }{\Q 收入库中。
\par }{\Q \VS{5}你靠自己的大智慧和贸易增添资财,
\par }{\Q 又因资财心里高傲。
\par }{\Q \VS{6}所以主耶和华如此说:
\par }{\Q 因你居心自比神,
\par }{\Q \VS{7}我必使外邦人,
\par }{\Q 就是列国中的强暴人临到你这里;
\par }{\Q 他们必拔刀砍坏你用智慧得来的美物,
\par }{\Q 亵渎你的荣光。
\par }{\Q \VS{8}他们必使你下坑;
\par }{\Q 你必死在海中,
\par }{\Q 与被杀的人一样。
\par }{\Q \VS{9}在杀你的人面前你还能说「我是神」吗?
\par }{\Q 其实你在杀害你的人手中,
\par }{\Q 不过是人,并不是神。
\par }{\Q \VS{10}你必死在外邦人手中,
\par }{\Q 与未受割礼\FTNT{}{{\FR 28:10: }或译:不洁;下同}的人一样,
\par }{\Q 因为这是主耶和华说的。」
\par }{\SH 泰尔王的终局
\par }{\PP \VS{11}耶和华的话临到我说:
\VS{12}「人子啊,你为{\PN{泰尔}}王作起哀歌,说主耶和华如此说:
\par }{\Q 你无所不备,
\par }{\Q 智慧充足,全然美丽。
\par }{\Q \VS{13}你曾在{\PN{伊甸}} 神的园中,
\par }{\Q 佩戴各样宝石,
\par }{\Q 就是红宝石、红璧玺、金钢石、
\par }{\Q 水苍玉、红玛瑙、碧玉、
\par }{\Q 蓝宝石、绿宝石、红玉,和黄金;
\par }{\Q 又有精美的鼓笛在你那里,
\par }{\Q 都是在你受造之日预备齐全的。
\par }{\Q \VS{14}你是那受膏遮掩{\ADD{约柜}}的基路伯;
\par }{\Q 我将你安置在 神的圣山上;
\par }{\Q 你在发{\ADD{光如}}火{\ADD{的宝}}石中间往来。
\par }{\Q \VS{15}你从受造之日所行的都完全,
\par }{\Q 后来在你中间又察出不义。
\par }{\Q \VS{16}因你贸易很多,
\par }{\Q 就被强暴的事充满,以致犯罪,
\par }{\Q 所以我因你亵渎{\ADD{圣地}},
\par }{\Q 就从 神的山驱逐你。
\par }{\Q 遮掩{\ADD{约柜}}的基路伯啊,
\par }{\Q 我已将你从发{\ADD{光如}}火{\ADD{的宝}}石中除灭。
\par }{\Q \VS{17}你因美丽心中高傲,
\par }{\Q 又因荣光败坏智慧,
\par }{\Q 我已将你摔倒在地,
\par }{\Q 使你倒在君王面前,
\par }{\Q 好叫他们目睹眼见。
\par }{\Q \VS{18}你因罪孽众多,
\par }{\Q 贸易不公,
\par }{\Q 就亵渎你那里的圣所。
\par }{\Q 故此,我使火从你中间发出,烧灭你,
\par }{\Q 使你在所有观看的人眼前变为地上的炉灰。
\par }{\Q \VS{19}各国民中,凡认识你的,
\par }{\Q 都必为你惊奇。
\par }{\Q 你令人惊恐,
\par }{\Q 不再存留于世,直到永远。」
\par }{\SH 斥责西顿
\par }{\Q \VS{20}耶和华的话临到我说:
\VS{21}「人子啊,你要向{\PN{西顿}}预言攻击她,
\VS{22}说主耶和华如此说:
\par }{\Q {\PN{西顿}}哪,我与你为敌,
\par }{\Q 我必在你中间得荣耀。
\par }{\Q 我在你中间施行审判、显为圣的时候,
\par }{\Q 人就知道我是耶和华。
\par }{\Q \VS{23}我必使瘟疫进入{\PN{西顿}},
\par }{\Q 使血流在她街上。
\par }{\Q 被杀的必在其中仆倒,
\par }{\Q 四围有刀剑临到她,
\par }{\Q 人就知道我是耶和华。」
\par }{\SH 以色列要蒙福
\par }{\PP \VS{24}「四围恨恶{\PN{以色列}}家的人,必不再向他们作刺人的荆棘,伤人的蒺藜,人就知道我是主耶和华。」
\par }{\PP \VS{25}主耶和华如此说:「我将分散在万民中的{\PN{以色列}}家招聚回来,向他们在列邦人眼前显为圣的时候,他们就在我赐给我仆人{\PN{雅各}}之地,仍然居住。
\VS{26}他们要在这地上安然居住。我向四围恨恶他们的众人施行审判以后,他们要盖造房屋,栽种葡萄园,安然居住,就知道我是耶和华—他们的 神。」

\par }\Chap{29}{\SH 斥责埃及
\par }{\PP \VerseOne{1}第十年十{\ADD{月}}十二{\ADD{日}},耶和华的话临到我说:
\VS{2}「人子啊,你要向{\PN{埃及}}王法老预言攻击他和{\PN{埃及}}全地,
\VS{3}说主耶和华如此说:
\par }{\Q {\PN{埃及}}王法老啊,
\par }{\Q 我与你这卧在自己河中的大鱼为敌。
\par }{\Q 你曾说:这河是我的,
\par }{\Q 是我为自己造的。
\par }{\Q \VS{4}我—{\ADD{耶和华}}必用钩子钩住你的腮颊,
\par }{\Q 又使江河中的鱼贴住你的鳞甲;
\par }{\Q 我必将你和所有贴住你鳞甲的鱼,
\par }{\Q 从江河中拉上来,
\par }{\Q \VS{5}把你并江河中的鱼都抛在旷野;
\par }{\Q 你必倒在田间,
\par }{\Q 不被收殓,不被掩埋。
\par }{\Q 我已将你给地上野兽、空中飞鸟作食物。
\par }{\PP \VS{6}「{\PN{埃及}}一切的居民,因向{\PN{以色列}}家成了芦苇的杖,就知道我是耶和华。
\VS{7}他们用手持住你,你就断折,伤了他们的肩;他们倚靠你,你就断折,闪了他们的腰。
\VS{8}所以主耶和华如此说:我必使刀剑临到你,从你中间将人与牲畜剪除。
\VS{9}{\PN{埃及}}地必荒废凄凉,他们就知道我是耶和华。
\par }{\PP 「因为法老说:『这河是我的,是我所造的』,
\VS{10}所以我必与你并你的江河为敌,使{\PN{埃及}}地,从{\PN{色弗尼}}塔直到{\PN{古实}}境界,全然荒废凄凉。
\VS{11}人的脚、兽的蹄都不经过,四十年之久并无人居住。
\VS{12}我必使{\PN{埃及}}地在荒凉的国中成为荒凉,使{\PN{埃及}}城在荒废的城中变为荒废,共有四十年。我必将{\PN{埃及}}人分散在列国,四散在列邦。」
\par }{\PP \VS{13}主耶和华如此说:「满了四十年,我必招聚分散在各国民中的{\PN{埃及}}人。
\VS{14}我必叫{\PN{埃及}}被掳的人回来,使他们归回本地{\PN{巴忒罗}}。在那里必成为低微的国,
\VS{15}必为列国中最低微的,也不再自高于列国之上。我必减少他们,以致不再辖制列国。
\VS{16}{\PN{埃及}}必不再作{\PN{以色列}}家所倚靠的;{\PN{以色列}}家仰望{\PN{埃及}}人的时候,便思念罪孽。他们就知道我是主耶和华。」
\par }{\SH 尼布甲尼撒要征服埃及
\par }{\PP \VS{17}二十七年正{\ADD{月}}初一{\ADD{日}},耶和华的话临到我说:
\VS{18}「人子啊,{\PN{巴比伦}}王{\PN{尼布甲尼撒}}使他的军兵大大效劳,攻打{\PN{泰尔}},以致头都光秃,肩都磨破;然而他和他的军兵攻打{\PN{泰尔}},并没有从那里得什么酬劳。
\VS{19}所以主耶和华如此说:我必将{\PN{埃及}}地赐给{\PN{巴比伦}}王{\PN{尼布甲尼撒}};他必掳掠{\PN{埃及}}群众,抢其中的财为掳物,夺其中的货为掠物,这就可以作他军兵的酬劳。
\VS{20}我将{\PN{埃及}}地赐给他,酬他所效的劳,因王与军兵是为我勤劳。这是主耶和华说的。
\par }{\PP \VS{21}「当那日,我必使{\PN{以色列}}家的角发生,又必使你—{\PN{以西结}}在他们中间得以开口;他们就知道我是耶和华。」

\par }\Chap{30}{\SH 耶和华要惩罚埃及
\par }{\PP \VerseOne{1}耶和华的话又临到我说:
\VS{2}「人子啊,你要发预言说,主耶和华如此说:
\par }{\Q 哀哉这日!你们应当哭号。
\par }{\Q \VS{3}因为耶和华的日子临近,
\par }{\Q 就是密云之日,
\par }{\Q 列国{\ADD{受罚}}之期。
\par }{\Q \VS{4}必有刀剑临到{\PN{埃及}};
\par }{\Q 在{\PN{埃及}}被杀之人仆倒的时候,
\par }{\Q {\PN{古实}}人就有痛苦,
\par }{\Q 人民必被掳掠,
\par }{\Q 基址必被拆毁。
\par }{\PP \VS{5}{\PN{古实}}人、{\PN{弗}}人\FTNT{}{{\FR 30:5: }或译:利比亚}、{\PN{路德}}人、杂族的人民,并{\PN{古巴}}人,以及同盟之地的人都要与{\PN{埃及}}人一同倒在刀下。」
\par }{\Q \VS{6}耶和华如此说:
\par }{\Q 扶助{\PN{埃及}}的也必倾倒。
\par }{\Q {\PN{埃及}}因势力而有的骄傲必降低微;
\par }{\Q 其中的人民,从{\PN{色弗尼}}塔起\FTNT{}{{\FR 30:6: }见二十九章十节}必倒在刀下。
\par }{\Q 这是主耶和华说的。
\par }{\Q \VS{7}{\PN{埃及}}地在荒凉的国中必成为荒凉;
\par }{\Q {\PN{埃及}}城在荒废的城中也变为荒废。
\par }{\Q \VS{8}我在{\PN{埃及}}中使火着起;
\par }{\Q 帮助{\PN{埃及}}的,都被灭绝。
\par }{\Q 那时,他们就知道我是耶和华。
\par }{\PP \VS{9}「到那日,必有使者坐船,从我面前出去,使安逸无虑的{\PN{古实}}人惊惧;必有痛苦临到他们,好像{\PN{埃及}}{\ADD{遭灾}}的日子一样。看哪,这事临近了!
\par }{\Q \VS{10}主耶和华如此说:
\par }{\Q 我必借{\PN{巴比伦}}王{\PN{尼布甲尼撒}}的手,
\par }{\Q 除灭{\PN{埃及}}众人。
\par }{\Q \VS{11}他和随从他的人,
\par }{\Q 就是列国中强暴的,
\par }{\Q 必进来毁灭这地。
\par }{\Q 他们必拔刀攻击{\PN{埃及}},
\par }{\Q 使遍地有被杀的人。
\par }{\Q \VS{12}我必使江河干涸,
\par }{\Q 将地卖在恶人的手中;
\par }{\Q 我必借外邦人的手,
\par }{\Q 使这地和其中所有的变为凄凉。
\par }{\Q 这是我—耶和华说的。
\par }{\BB \par }{\Q \VS{13}主耶和华如此说:
\par }{\Q 我必毁灭偶像,
\par }{\Q 从{\PN{挪弗}}除灭神像;
\par }{\Q 必不再有君王出自{\PN{埃及}}地。
\par }{\Q 我要使{\PN{埃及}}地的人惧怕。
\par }{\Q \VS{14}我必使{\PN{巴忒罗}}荒凉,
\par }{\Q 在{\PN{琐安}}中使火着起,
\par }{\Q 向{\PN{挪}}施行审判。
\par }{\Q \VS{15}我必将我的忿怒倒在{\PN{埃及}}的保障上,
\par }{\Q 就是{\PN{训}}上,
\par }{\Q 并要剪除{\PN{挪}}的众人。
\par }{\Q \VS{16}我必在{\PN{埃及}}中使火着起;
\par }{\Q {\PN{训}}必大大痛苦;
\par }{\Q {\PN{挪}}必被攻破;
\par }{\Q {\PN{挪弗}}白日\FTNT{}{{\FR 30:16: }或译:终日}见仇敌。
\par }{\Q \VS{17}{\PN{亚文}}和{\PN{比伯实}}的少年人必倒在刀下;
\par }{\Q 这些{\ADD{城的人}}必被掳掠。
\par }{\Q \VS{18}我在{\PN{答比匿}}折断{\PN{埃及}}的诸轭,
\par }{\Q 使她因势力而有的骄傲在其中止息。
\par }{\Q 那时,日光必退去;
\par }{\Q 至于这城,必有密云遮蔽,
\par }{\Q 其中的女子必被掳掠。
\par }{\Q \VS{19}我必这样向{\PN{埃及}}施行审判,
\par }{\Q 他们就知道我是耶和华。」
\par }{\SH 埃及王破碎的力量
\par }{\PP \VS{20}十一年正{\ADD{月}}初七{\ADD{日}},耶和华的话临到我说:
\VS{21}「人子啊,我已打折{\PN{埃及}}王法老的膀臂;没有敷药,也没有用布缠好,使他有力持刀。
\VS{22}所以主耶和华如此说:看哪,我与{\PN{埃及}}王法老为敌,必将他有力的膀臂和已打折的膀臂全行打断,使刀从他手中坠落。
\VS{23}我必将{\PN{埃及}}人分散在列国,四散在列邦。
\VS{24}我必使{\PN{巴比伦}}王的膀臂有力,将我的刀交在他手中;却要打断法老的膀臂,他就在{\PN{巴比伦}}王面前唉哼,如同受死伤的人一样。
\VS{25}我必扶持{\PN{巴比伦}}王的膀臂,法老的膀臂却要下垂;我将我的刀交在{\PN{巴比伦}}王手中,他必举刀攻击{\PN{埃及}}地,他们就知道我是耶和华。
\VS{26}我必将{\PN{埃及}}人分散在列国,四散在列邦;他们就知道我是耶和华。」

\par }\Chap{31}{\SH 以香柏树比拟埃及
\par }{\PP \VerseOne{1}十一年三{\ADD{月}}初一{\ADD{日}},耶和华的话临到我说:
\VS{2}「人子啊,你要向{\PN{埃及}}王法老和他的众人说:
\par }{\Q 在威势上谁能与你相比呢?
\par }{\Q \VS{3}{\PN{亚述}}王曾如{\PN{黎巴嫩}}中的香柏树,
\par }{\Q 枝条荣美,影密如林,
\par }{\Q 极其高大,树尖插入云中。
\par }{\Q \VS{4}众水使它生长;
\par }{\Q 深水使它长大。
\par }{\Q 所栽之地有江河围流,
\par }{\Q 汊出的水道延到田野诸树。
\par }{\Q \VS{5}所以它高大超过田野诸树;
\par }{\Q 发旺的时候,枝子繁多,
\par }{\Q 因得大水之力枝条长长。
\par }{\Q \VS{6}空中的飞鸟都在枝子上搭窝;
\par }{\Q 田野的走兽都在枝条下生子;
\par }{\Q 所有大国的人民都在它荫下居住。
\par }{\Q \VS{7}树大条长,成为荣美,
\par }{\Q 因为根在众水之旁。
\par }{\Q \VS{8}神园中的香柏树不能遮蔽它;
\par }{\Q 松树不及它的枝子;
\par }{\Q 枫树不及它的枝条;
\par }{\Q  神园中的树都没有它荣美。
\par }{\Q \VS{9}我使它的枝条蕃多,成为荣美,
\par }{\Q 以致 神{\PN{伊甸园}}中的树都嫉妒它。」
\par }{\PP \VS{10}所以主耶和华如此说:「因它高大,树尖插入云中,心骄气傲,
\VS{11}我就必将它交给列国中大有威势的人;他必定办它。我因它的罪恶,已经驱逐它。
\VS{12}外邦人,就是列邦中强暴的,将它砍断弃掉。它的枝条落在山间和一切谷中,它的枝子折断,落在地的一切河旁。地上的众民已经走去,离开它的荫下。
\VS{13}空中的飞鸟都要宿在这败落的树上,田野的走兽都要卧在它的枝条下,
\VS{14}好使水旁的诸树不因高大而自尊,也不将树尖插入云中,并且那些得水滋润、有势力的,也不得高大自立。因为它们在世人中,和下坑的人都被交与死亡,到阴府去了。」
\par }{\PP \VS{15}主耶和华如此说:「它下阴间的那日,我便使人悲哀。我为它遮盖深渊,使江河凝结,大水停流;我也使{\PN{黎巴嫩}}为它凄惨,田野的诸树都因它发昏。
\VS{16}我将它扔到阴间,与下坑的人一同下去。那时,列国听见它坠落的响声就都震动,并且{\PN{伊甸}}的一切树—就是{\PN{黎巴嫩}}得水滋润、最佳最美的树—都在阴府受了安慰。
\VS{17}它们也与它同下阴间,到被杀的人那里。它们曾作它的膀臂,在列国中它的荫下居住。
\VS{18}在这样荣耀威势上,在{\PN{伊甸}}{\ADD{园}}诸树中,谁能与你相比呢?然而你要与{\PN{伊甸}}的诸树一同下到阴府,在未受割礼的人中,与被杀的人一同躺卧。
\par }{\PP 「法老和他的群众乃是如此。这是主耶和华说的。」

\par }\Chap{32}{\SH 以鳄鱼比拟埃及王
\par }{\PP \VerseOne{1}十二年十二月初一日,耶和华的话临到我说:
\VS{2}「人子啊,你要为{\PN{埃及}}王法老作哀歌,说:
\par }{\Q 从前你在列国中,如同少壮狮子;
\par }{\Q 现在你却像海中的大鱼。
\par }{\Q 你冲出江河,
\par }{\Q 用爪搅动诸水,
\par }{\Q 使江河浑浊。
\par }{\Q \VS{3}主耶和华如此说:
\par }{\Q 我必用多国的人民,
\par }{\Q 将我的网撒在你身上,
\par }{\Q 把你拉上来。
\par }{\Q \VS{4}我必将你丢在地上,
\par }{\Q 抛在田野,
\par }{\Q 使空中的飞鸟都落在你身上,
\par }{\Q 使遍地的野兽吃你得饱。
\par }{\Q \VS{5}我必将你的肉丢在山间,
\par }{\Q 用你高大的{\ADD{尸首}}填满山谷。
\par }{\Q \VS{6}我又必用你的血浇灌你所游泳之地,
\par }{\Q 漫过山顶;
\par }{\Q 河道都必充满。
\par }{\Q \VS{7}我将你扑灭的时候,
\par }{\Q 要把天遮蔽,
\par }{\Q 使众星昏暗,
\par }{\Q 以密云遮掩太阳,
\par }{\Q 月亮也不放光。
\par }{\Q \VS{8}我必使天上的亮光都在你以上变为昏暗,
\par }{\Q 使你的地上黑暗。
\par }{\Q 这是主耶和华说的。
\par }{\PP \VS{9}「我使你败亡{\ADD{的风声}}传到你所不认识的各国。那时,我必使多民的心因你愁烦。
\VS{10}我在许多国民和君王面前{\ADD{向你}}抡我的刀,国民就必因你惊奇,君王也必因你极其恐慌。在你仆倒的日子,他们各人为自己的性命时刻战兢。」
\VS{11}主耶和华如此说:「{\PN{巴比伦}}王的刀必临到你。
\VS{12}我必借勇士的刀使你的众民仆倒;这勇士都是列国中强暴的。
\par }{\Q 他们必使{\PN{埃及}}的骄傲归于无有;
\par }{\Q {\PN{埃及}}的众民必被灭绝。
\par }{\Q \VS{13}我必从{\PN{埃及}}多水旁除灭所有的走兽;
\par }{\Q 人脚兽蹄必不再搅浑这水。
\par }{\Q \VS{14}那时,我必使{\PN{埃及河}}澄清,
\par }{\Q 江河像油缓流。
\par }{\Q 这是主耶和华说的。
\par }{\Q \VS{15}我使{\PN{埃及}}地变为荒废凄凉;
\par }{\Q 这地缺少从前所充满的,
\par }{\Q 又击杀其中一切的居民。
\par }{\Q 那时,他们就知道我是耶和华。
\par }{\PP \VS{16}「人必用这哀歌去哀哭,列国的女子为{\PN{埃及}}和她的群众也必以此悲哀。这是主耶和华说的。」
\par }{\SH 为埃及哀号
\par }{\PP \VS{17}十二年{\ADD{十二}}月十五日,耶和华的话临到我说:
\VS{18}「人子啊,你要为{\PN{埃及}}群众哀号,又要将{\PN{埃及}}和有名之国的女子,并下坑的人,一同扔到阴府去。
\par }{\Q \VS{19}你{\PN{埃及}}的美丽胜过谁呢?
\par }{\Q 你下去与未受割礼的人一同躺卧吧!
\par }{\PP \VS{20}他们必在被杀的人中仆到。她被交给刀剑,要把她和她的群众拉去。
\VS{21}强盛的勇士要在阴间对{\PN{埃及}}王和帮助他的说话;他们是未受割礼被杀的人,已经下去,躺卧不动。
\par }{\PP \VS{22}「{\PN{亚述}}和她的众民都在那里,她{\ADD{民}}的坟墓在她四围;他们都是被杀倒在刀下的。
\VS{23}他们的坟墓在坑中极深之处。她的众民在她坟墓的四围,都是被杀倒在刀下的;他们曾在活人之地使人惊恐。
\par }{\PP \VS{24}「{\PN{以拦}}也在那里,她的群众在她坟墓的四围,都是被杀倒在刀下、未受割礼而下阴府的;他们曾在活人之地使人惊恐,并且与下坑的人一同担当羞辱。
\VS{25}人给她和她的群众在被杀的人中设立床榻。她{\ADD{民}}的坟墓在她四围,他们都是未受割礼被刀杀的;他们曾在活人之地使人惊恐,并且与下坑的人一同担当羞辱。{\PN{以拦}}已经放在被杀的人中。
\par }{\PP \VS{26}「{\PN{米设}}、{\PN{土巴}},和她们的群众都在那里。她{\ADD{民}}的坟墓在她四围,他们都是未受割礼被刀杀的;他们曾在活人之地使人惊恐。
\VS{27}他们不得与那未受割礼仆倒的勇士一同躺卧;这些勇士带着兵器下阴间,头枕刀剑,骨头上有本身的罪孽;{\ADD{他们}}曾在活人之地使勇士惊恐。
\par }{\PP \VS{28}「法老啊,你必在未受割礼的人中败坏,与那些被杀的人一同躺卧。
\par }{\PP \VS{29}「{\PN{以东}}也在那里。她君王和一切首领虽然仗着势力,还是放在被杀的人中;他们必与未受割礼的和下坑的人一同躺卧。
\par }{\PP \VS{30}「在那里有北方的众王子和一切{\PN{西顿}}人,都与被杀的人下去。他们虽然仗着势力使人惊恐,还是蒙羞。他们未受割礼,和被刀杀的一同躺卧,与下坑的人一同担当羞辱。
\par }{\PP \VS{31}「法老看见他们,便为他被杀的军队受安慰。这是主耶和华说的。
\VS{32}我任凭法老在活人之地使人惊恐,法老和他的群众必放在未受割礼和被杀的人中。这是主耶和华说的。」

\par }\Chap{33}{\SH  神指派以西结作守望者
\par }{\R (3·16—21)
\par }{\PP \VerseOne{1}耶和华的话临到我说:
\VS{2}「人子啊,你要告诉本国的子民说:我使刀剑临到哪一国,那一国的民从他们中间选立一人为守望的。
\VS{3}他见刀剑临到那地,若吹角警戒众民,
\VS{4}凡听见角声不受警戒的,刀剑若来除灭了他,他的罪\FTNT{}{{\FR 33:4: }原文是血}就必归到自己的头上。
\VS{5}他听见角声,不受警戒,他的罪必归到自己的身上;他若受警戒,便是救了自己的性命。
\VS{6}倘若守望的人见刀剑临到,不吹角,以致民不受警戒,刀剑来杀了他们中间的一个人,他虽然死在罪孽之中,我却要向守望的人讨他丧命的罪\FTNT{}{{\FR 33:6: }原文是血}。
\par }{\PP \VS{7}「人子啊,我照样立你作{\PN{以色列}}家守望的人。所以你要听我口中的话,替我警戒他们。
\VS{8}我对恶人说:『恶人哪,你必要死!』你—{\PN{以西结}}若不开口警戒恶人,使他离开所行的道,这恶人必死在罪孽之中,我却要向你讨他丧命的罪\FTNT{}{{\FR 33:8: }原文是血}。
\VS{9}倘若你警戒恶人转离所行的道,他仍不转离,他必死在罪孽之中,你却救自己脱离了罪。」
\par }{\SH 个人的责任
\par }{\PP \VS{10}「人子啊,你要对{\PN{以色列}}家说:『你们常说:我们的过犯罪恶在我们身上,我们必因此消灭,怎能存活呢?』
\VS{11}你对他们说,主耶和华说:我指着我的永生起誓,我断不喜悦恶人死亡,惟喜悦恶人转离所行的道而活。{\PN{以色列}}家啊,你们转回,转回吧!离开恶道,何必死亡呢?
\VS{12}人子啊,你要对本国的人民说:义人的义,在犯罪之日不能救他;至于恶人的恶,在他转离恶行之日也不能使他倾倒;义人在犯罪之日也不能因他的义存活。
\VS{13}我对义人说:『你必定存活!』他若倚靠他的义而作罪孽,他所行的义都不被记念。他必因所作的罪孽死亡。
\VS{14}再者,我对恶人说:『你必定死亡!』他若转离他的罪,行正直与合理的事:
\VS{15}还人的当头和所抢夺的,遵行生命的律例,不作罪孽,他必定存活,不致死亡。
\VS{16}他所犯的一切罪必不被记念。他行了正直与合理的事,必定存活。
\par }{\PP \VS{17}「你本国的子民还说:『主的道不公平。』其实他们的道不公平。
\VS{18}义人转离他的义而作罪孽,就必因此死亡。
\VS{19}恶人转离他的恶,行正直与合理的事,就必因此存活。
\VS{20}你们还说:『主的道不公平。』{\PN{以色列}}家啊,我必按你们各人所行的审判你们。」
\par }{\SH 耶路撒冷陷落的消息
\par }{\PP \VS{21}我们被掳之后十二年十{\ADD{月}}初五{\ADD{日}},有人从{\PN{耶路撒冷}}逃到我这里,说:「城已攻破。」
\VS{22}逃来的人未到前一日的晚上,耶和华的灵\FTNT{}{{\FR 33:22: }原文是手}降在我身上,开我的口。到第二日早晨,那人来到我这里,我口就开了,不再缄默。
\par }{\SH 百姓的罪
\par }{\PP \VS{23}耶和华的话临到我说:
\VS{24}「人子啊,住在{\PN{以色列}}荒废之地的人说:『{\PN{亚伯拉罕}}独自一人能得这地为业,我们人数众多,这地{\ADD{更}}是给我们为业的。』
\VS{25}所以你要对他们说,主耶和华如此说:你们吃带血的物,仰望偶像,并且杀人流血,你们还能得这地为业吗?
\VS{26}你们倚仗自己的刀剑行可憎的事,人人玷污邻舍的妻,你们还能得这地为业吗?
\VS{27}你要对他们这样说,主耶和华如此说:我指着我的永生起誓,在荒场中的,必倒在刀下;在田野间的,必交给野兽吞吃;在保障和洞里的,必遭瘟疫而死。
\VS{28}我必使这地荒凉,令人惊骇;她因势力而有的骄傲也必止息。{\PN{以色列}}的山都必荒凉,无人经过。
\VS{29}我因他们所行一切可憎的事使地荒凉,令人惊骇。那时,他们就知道我是耶和华。」
\par }{\SH 先知传讲信息的结果
\par }{\PP \VS{30}「人子啊,你本国的子民在墙垣旁边、在房屋门口谈论你。弟兄对弟兄彼此说:『来吧!听听有什么话从耶和华而出。』
\VS{31}他们来到你这里如同民来{\ADD{聚会}},坐在你面前仿佛是我的民。他们听你的话却不去行;因为他们的口多显爱情,心却追随财利。
\VS{32}他们看你如善于奏乐、声音幽雅之人所唱的雅歌,他们听你的话却不去行。
\VS{33}看哪,所说的快要应验;应验了,他们就知道在他们中间有了先知。」

\par }\Chap{34}{\SH 以色列的牧人
\par }{\PP \VerseOne{1}耶和华的话临到我说:
\VS{2}「人子啊,你要向{\PN{以色列}}的牧人发预言,攻击他们,说,主耶和华如此说:祸哉!{\PN{以色列}}的牧人只知牧养自己。牧人岂不当牧养群羊吗?
\VS{3}你们吃脂油、穿羊毛、宰肥壮的,却不牧养群羊。
\VS{4}瘦弱的,你们没有养壮;有病的,你们没有医治;受伤的,你们没有缠裹;被逐的,你们没有领回;失丧的,你们没有寻找;但用强暴严严地辖制。
\VS{5}因无牧人,羊就分散;既分散,便作了一切野兽的食物。
\VS{6}我的羊在诸山间、在各高冈上流离,在全地上分散,无人去寻,无人去找。
\par }{\PP \VS{7}「所以,你们这些牧人要听耶和华的话。
\VS{8}主耶和华说:我指着我的永生起誓,我的羊因无牧人就成为掠物,也作了一切野兽的食物。我的牧人不寻找我的羊;这些牧人只知牧养自己,并不牧养我的羊。
\VS{9}所以你们这些牧人要听耶和华的话。
\VS{10}主耶和华如此说:我必与牧人为敌,必向他们的手追讨我的羊,使他们不再牧放群羊;牧人也不再牧养自己。我必救我的羊脱离他们的口,不再作他们的食物。」
\par }{\SH 好牧人
\par }{\PP \VS{11}「主耶和华如此说:看哪,我必亲自寻找我的羊,将它们寻见。
\VS{12}牧人在羊群四散的日子怎样寻找他的羊,我必照样寻找我的羊。这些羊在密云黑暗的日子散到各处,我必从那里救回它们来。
\VS{13}我必从万民中领出它们,从各国内聚集它们,引导它们归回故土,也必在{\PN{以色列}}山上—一切溪水旁边、境内一切可居之处—牧养它们。
\VS{14}我必在美好的草场牧养它们。它们的圈必在{\PN{以色列}}高处的山上,它们必在佳美之圈中躺卧,也在{\PN{以色列}}山肥美的草场吃草。
\VS{15}主耶和华说:我必亲自作我羊的牧人,使它们得以躺卧。
\VS{16}失丧的,我必寻找;被逐的,我必领回;受伤的,我必缠裹;有病的,我必医治;只是肥的壮的,我必除灭,也要秉公牧养它们。
\par }{\PP \VS{17}「我的羊群哪,论到你们,主耶和华如此说:我必在羊与羊中间、公绵羊与公山羊中间施行判断。
\VS{18}你们这些{\ADD{肥壮的羊}},在美好的草场吃草还以为小事吗?剩下的草,你们竟用蹄践踏了;你们喝清水,剩下的水,你们竟用蹄搅浑了。
\VS{19}至于我的羊,只得吃你们所践踏的,喝你们所搅浑的。
\par }{\PP \VS{20}「所以,主耶和华如此说:我必在肥羊和瘦羊中间施行判断。
\VS{21}因为你们用胁用肩拥挤一切瘦弱的,又用角抵触,以致使它们四散。
\VS{22}所以,我必拯救我的群羊不再作掠物;我也必在羊和羊中间施行判断。
\VS{23}我必立一牧人照管他们,牧养他们,就是我的仆人{\PN{大卫}}。他必牧养他们,作他们的牧人。
\VS{24}我—耶和华必作他们的 神,我的仆人{\PN{大卫}}必在他们中间作王。这是耶和华说的。
\par }{\PP \VS{25}「我必与他们立平安的约,使恶兽从境内断绝,他们就必安居在旷野,躺卧在林中。
\VS{26}我必使他们与我山的四围成为福源,我也必叫时雨落下,必有福{\ADD{如}}甘霖而降。
\VS{27}田野的树必结果,地也必有出产;他们必在故土安然居住。我折断他们所负的轭,救他们脱离那以他们为奴之人的手;那时,他们就知道我是耶和华。
\VS{28}他们必不再作外邦人的掠物,地上的野兽也不再吞吃他们;却要安然居住,无人惊吓。
\VS{29}我必给他们兴起有名的植物;他们在境内不再为饥荒所灭,也不再受外邦人的羞辱,
\VS{30}必知道我、耶和华—他们的 神是与他们同在的,并知道他们—{\PN{以色列}}家是我的民。这是主耶和华说的。
\VS{31}你们作我的羊,我草场上的羊,乃是{\PN{以色列}}人,我也是你们的 神。这是主耶和华说的。」

\par }\Chap{35}{\SH  神惩罚以东
\par }{\PP \VerseOne{1}耶和华的话又临到我说:
\VS{2}「人子啊,你要面向{\PN{西珥山}}发预言,攻击它,
\VS{3}对它说,主耶和华如此说:{\PN{西珥山}}哪,我与你为敌,必向你伸手攻击你,使你荒凉,令人惊骇。
\VS{4}我必使你的城邑变为荒场,成为凄凉。你就知道我是耶和华。
\VS{5}因为你永怀仇恨,在{\PN{以色列}}人遭灾、罪孽到了尽头的时候,将他们交与刀剑,
\VS{6}所以主耶和华说:我指着我的永生起誓,我必使你遭遇流血的报应,罪\FTNT{}{{\FR 35:6: }原文是血;本节同}必追赶你;你既不恨恶杀人流血,所以这罪必追赶你。
\VS{7}我必使{\PN{西珥山}}荒凉,令人惊骇,来往经过的人我必剪除。
\VS{8}我必使{\PN{西珥山}}满有被杀的人。被刀杀的,必倒在你小山和山谷,并一切的溪水中。
\VS{9}我必使你永远荒凉,使你的城邑无人居住,你的民就知道我是耶和华。
\par }{\PP \VS{10}「因为你曾说:『这二国这二邦必归于我,我必得为业』(其实耶和华仍在那里),
\VS{11}所以主耶和华说:我指着我的永生起誓,我必照你的怒气和你从仇恨中向他们所发的嫉妒待你。我审判你的时候,必将自己显明在他们中间。
\VS{12}你也必知道我—耶和华听见了你的一切毁谤,就是你攻击{\PN{以色列}}山的话,说:『这些山荒凉,是归我们吞灭的。』
\VS{13}你们也用口向我夸大,增添与我反对的话,我都听见了。
\VS{14}主耶和华如此说:全地欢乐的时候,我必使你荒凉。
\VS{15}你怎样因{\PN{以色列}}家的地业荒凉而喜乐,我必照你所行的待你。{\PN{西珥山}}哪,你和{\PN{以东}}全地必都荒凉。你们就知道我是耶和华。」

\par }\Chap{36}{\SH  神赐福给以色列
\par }{\PP \VerseOne{1}「人子啊,你要对{\PN{以色列}}山发预言说:{\PN{以色列}}山哪,要听耶和华的话。
\VS{2}主耶和华如此说:因仇敌说:『阿哈!这永久的山冈都归我们为业了!』
\VS{3}所以要发预言说,主耶和华如此说:因为敌人使你荒凉,四围吞吃,好叫你归与其余的外邦人为业,并且多嘴多舌的人提起你来,百姓也说你有臭名。
\VS{4}故此,{\PN{以色列}}山要听主耶和华的话。大山小冈、水沟山谷、荒废之地、被弃之城,为四围其余的外邦人所占据、所讥刺的,
\VS{5}主耶和华对你们如此说:我真发愤恨如火,责备那其余的外邦人和{\PN{以东}}的众人。他们快乐满怀,心存恨恶,将我的地归自己为业,又看为被弃的掠物。
\VS{6}所以,你要指着{\PN{以色列}}地说预言,对大山小冈、水沟山谷说,主耶和华如此说:我发愤恨和忿怒说,因你们曾受外邦人的羞辱,
\VS{7}所以我起誓说:你们四围的外邦人总要担当自己的羞辱。这是主耶和华说的。
\par }{\PP \VS{8}「{\PN{以色列}}山哪,你必发枝条,为我的民{\PN{以色列}}结果子,因为他们快要来到。
\VS{9}看哪,我是帮助你的,也必向你转意,使你得以耕种。
\VS{10}我必使{\PN{以色列}}全家的人数在你上面增多,城邑有人居住,荒场再被建造。
\VS{11}我必使人和牲畜在你上面加增;他们必生养众多。我要使你照旧有人居住,并要赐福与你比先前更多,你就知道我是耶和华。
\VS{12}我必使人,就是我的民{\PN{以色列}},行在你上面。他们必得你为业;你也不再使他们丧子。
\VS{13}主耶和华如此说:因为人对你说:『你是吞吃人的,又使国民丧子』,
\VS{14}所以主耶和华说:你必不再吞吃人,也不再使国民丧子。
\VS{15}我使你不再听见各国的羞辱,不再受万民的辱骂,也不再使国民绊跌。这是主耶和华说的。」
\par }{\SH 以色列的新生活
\par }{\PP \VS{16}耶和华的话又临到我说:
\VS{17}「人子啊,{\PN{以色列}}家住在本地的时候,在行动作为上玷污那地。他们的行为在我面前,好像正在经期的妇人那样污秽。
\VS{18}所以我因他们在那地上流人的血,又因他们以偶像玷污那地,就把我的忿怒倾在他们身上。
\VS{19}我将他们分散在列国,四散在列邦,按他们的行动作为惩罚他们。
\VS{20}他们到了所去的列国,就使我的圣名被亵渎;因为人谈论他们说,这是耶和华的民,是从耶和华的地出来的。
\VS{21}我却顾惜我的圣名,就是{\PN{以色列}}家在所到的列国中所亵渎的。
\par }{\PP \VS{22}「所以,你要对{\PN{以色列}}家说,主耶和华如此说:{\PN{以色列}}家啊,我行这事不是为你们,乃是为我的圣名,就是在你们到的列国中所亵渎的。
\VS{23}我要使我的大名显为圣;这名在列国中已被亵渎,就是你们在他们中间所亵渎的。我在他们眼前,在你们身上显为圣的时候,他们就知道我是耶和华。这是主耶和华说的。
\VS{24}我必从各国收取你们,从列邦聚集你们,引导你们归回本地。
\VS{25}我必用清水洒在你们身上,你们就洁净了。我要洁净你们,使你们脱离一切的污秽,弃掉一切的偶像。
\VS{26}我也要赐给你们一个新心,将新灵放在你们里面,又从你们的肉体中除掉石心,赐给你们肉心。
\VS{27}我必将我的灵放在你们里面,使你们顺从我的律例,谨守遵行我的典章。
\VS{28}你们必住在我所赐给你们列祖之地。你们要作我的子民,我要作你们的 神。
\VS{29}我必救你们脱离一切的污秽,也必命五谷丰登,不使你们遭遇饥荒。
\VS{30}我必使树木多结果子,田地多出土产,好叫你们不再因饥荒受外邦人的讥诮。
\VS{31}那时,你们必追想你们的恶行和你们不善的作为,就因你们的罪孽和可憎的事厌恶自己。
\VS{32}主耶和华说:你们要知道,我这样行不是为你们。{\PN{以色列}}家啊,当为自己的行为抱愧蒙羞。
\par }{\PP \VS{33}「主耶和华如此说:我洁净你们,使你们脱离一切罪孽的日子,必使城邑有人居住,荒场再被建造。
\VS{34}过路的人虽看为荒废之地,现今这荒废之地仍得耕种。
\VS{35}他们必说:『这先前为荒废之地,现在成如{\PN{伊甸园}};这荒废凄凉、毁坏的城邑现在坚固有人居住。』
\VS{36}那时,在你们四围其余的外邦人必知道我—耶和华修造那毁坏之处,培植那荒废之地。我—耶和华说过,也必成就。
\par }{\PP \VS{37}「主耶和华如此说:我要加增{\PN{以色列}}家的人数,多如羊群。他们必为这事向我求问,我要给他们成就。
\VS{38}{\PN{耶路撒冷}}在守节作祭物所献的羊群怎样多,照样,荒凉的城邑必被人群充满。他们就知道我是耶和华。」

\par }\Chap{37}{\SH 枯骨的复苏
\par }{\PP \VerseOne{1}耶和华的灵\FTNT{}{{\FR 37:1: }原文是手}降在我身上。耶和华借他的灵带我出去,将我放在平原中;这平原遍满骸骨。
\VS{2}他使我从骸骨的四围经过,谁知在平原的骸骨甚多,而且极其枯干。
\VS{3}他对我说:「人子啊,这些骸骨能复活吗?」我说:「主耶和华啊,你是知道的。」
\VS{4}他又对我说:「你向这些骸骨发预言说:枯干的骸骨啊,要听耶和华的话。
\VS{5}主耶和华对这些骸骨如此说:『我必使气息进入你们里面,你们就要活了。
\VS{6}我必给你们加上筋,使你们长肉,又将皮遮蔽你们,使气息进入你们里面,你们就要活了;你们便知道我是耶和华。』」
\par }{\PP \VS{7}于是,我遵命说预言。正说预言的时候,不料,有响声,有地震;骨与骨互相联络。
\VS{8}我观看,见骸骨上有筋,也长了肉,又有皮遮蔽其上,只是还没有气息。
\VS{9}主对我说:「人子啊,你要发预言,向风发预言,说主耶和华如此说:气息啊,要从四方\FTNT{}{{\FR 37:9: }原文是风}而来,吹在这些被杀的人身上,使他们活了。」
\VS{10}于是我遵命说预言,气息就进入骸骨,骸骨便活了,并且站起来,成为极大的军队。
\par }{\PP \VS{11}主对我说:「人子啊,这些骸骨就是{\PN{以色列}}全家。他们说:『我们的骨头枯干了,我们的指望失去了,我们灭绝净尽了。』
\VS{12}所以你要发预言对他们说,主耶和华如此说:『我的民哪,我必开你们的坟墓,使你们从坟墓中出来,领你们进入{\PN{以色列}}地。
\VS{13}我的民哪,我开你们的坟墓,使你们从坟墓中出来,你们就知道我是耶和华。
\VS{14}我必将我的灵放在你们里面,你们就要活了。我将你们安置在本地,你们就知道我—耶和华如此说,也如此成就了。这是耶和华说的。』」
\par }{\SH 犹大和以色列组成联合王国
\par }{\PP \VS{15}耶和华的话又临到我说:
\VS{16}「人子啊,你要取一根木杖,在其上写『为{\PN{犹大}}和他的同伴{\PN{以色列}}人』;又取一根木杖,在其上写『为{\PN{约瑟}},就是为{\PN{以法莲}},又为他的同伴{\PN{以色列}}全家』。
\VS{17}你要使这两根木杖接连为一,在你手中成为一根。
\VS{18}你本国的子民问你说:『这是什么意思?你不指示我们吗?』
\VS{19}你就对他们说:『主耶和华如此说:我要将{\PN{约瑟}}和他同伴{\PN{以色列}}支派的杖,就是那在{\PN{以法莲}}手中的,与{\PN{犹大}}的杖一同接连为一,在我手中成为一根。』
\VS{20}你所写的那两根杖要在他们眼前拿在手中,
\VS{21}要对他们说,主耶和华如此说:我要将{\PN{以色列}}人从他们所到的各国收取,又从四围聚集他们,引导他们归回本地。
\VS{22}我要使他们在那地,在{\PN{以色列}}山上成为一国,有一王作他们众民的王。他们不再为二国,决不再分为二国;
\VS{23}也不再因偶像和可憎的物,并一切的罪过玷污自己。我却要救他们出离一切的住处,就是他们犯罪的地方;我要洁净他们,如此,他们要作我的子民,我要作他们的 神。
\par }{\PP \VS{24}「我的仆人{\PN{大卫}}必作他们的王;众民必归一个牧人。他们必顺从我的典章,谨守遵行我的律例。
\VS{25}他们必住在我赐给我仆人{\PN{雅各}}的地上,就是你们列祖所住之地。他们和他们的子孙,并子孙的子孙,都永远住在那里。我的仆人{\PN{大卫}}必作他们的王,直到永远。
\VS{26}并且我要与他们立平安的约,作为永约。我也要将他们安置在{\ADD{本地}},使他们的人数增多,又在他们中间设立我的圣所,直到永远。
\VS{27}我的居所必在他们中间;我要作他们的 神,他们要作我的子民。
\VS{28}我的圣所在{\PN{以色列}}人中间直到永远,外邦人就必知道我是叫{\PN{以色列}}成为圣的耶和华。」

\par }\Chap{38}{\SH 歌革—耶和华的工具
\par }{\PP \VerseOne{1}耶和华的话临到我说:
\VS{2}「人子啊,你要面向{\PN{玛各}}地的{\PN{歌革}},就是{\PN{罗施}}、{\PN{米设}}、{\PN{土巴}}的王发预言攻击他,
\VS{3}说主耶和华如此说:{\PN{罗施}}、{\PN{米设}}、{\PN{土巴}}的王{\PN{歌革}}啊,我与你为敌。
\VS{4}我必用钩子钩住你的腮颊,调转你,将你和你的军兵、马匹、马兵带出来,都披挂整齐,成了大队,有大小盾牌,各拿刀剑。
\VS{5}{\PN{波斯}}人、{\PN{古实}}人,和{\PN{弗}}人\FTNT{}{{\FR 38:5: }又作利比亚人},各拿盾牌,头上戴盔;
\VS{6}{\PN{歌篾}}人和他的军队,北方极处的{\PN{陀迦玛}}族和他的军队,这许多国的民都同着你。
\par }{\PP \VS{7}「那聚集到你这里的各队都当准备;你自己也要准备,作他们的大帅。
\VS{8}过了多日,你必被差派。到末后之年,你必来到脱离刀剑从列国收回之地,到{\PN{以色列}}常久荒凉的山上;但那从列国中招聚出来的必在其上安然居住。
\VS{9}你和你的军队,并同着你许多国的民,必如暴风上来,如密云遮盖地面。」
\par }{\PP \VS{10}主耶和华如此说:「到那时,你心必起意念,图谋恶计,
\VS{11}说:『我要上那无城墙的乡村,我要到那安静的民那里,他们都没有城墙,无门、无闩,安然居住。
\VS{12}我去要抢财为掳物,夺货为掠物,反手攻击那从前荒凉、现在有人居住之地,又攻击那住世界中间、从列国招聚、得了牲畜财货的民。』
\VS{13}{\PN{示巴}}人、{\PN{底但}}人、{\PN{他施}}的客商,和其间的少壮狮子都必问你说:『你来要抢财为掳物吗?你聚集军队要夺货为掠物吗?要夺取金银,掳去牲畜、财货吗?要抢夺许多财宝为掳物吗?』
\par }{\PP \VS{14}「人子啊,你要因此发预言,对{\PN{歌革}}说,主耶和华如此说:到我民{\PN{以色列}}安然居住之日,你岂不知道吗?
\VS{15}你必从本地,从北方的极处率领许多国的民来,都骑着马,乃一大队极多的军兵。
\VS{16}{\PN{歌革}}啊,你必上来攻击我的民{\PN{以色列}},如密云遮盖地面。末后的日子,我必带你来攻击我的地,到我在外邦人眼前,在你身上显为圣的时候,好叫他们认识我。
\VS{17}主耶和华如此说:我在古时借我的仆人{\PN{以色列}}的先知所说的,就是你吗?当日他们多年预言我必带你来攻击{\PN{以色列}}人。」
\par }{\SH  神惩罚歌革
\par }{\PP \VS{18}主耶和华说:「{\PN{歌革}}上来攻击{\PN{以色列}}地的时候,我的怒气要从鼻孔里发出。
\VS{19}我发愤恨和烈怒如火说:那日在{\PN{以色列}}地必有大震动,
\VS{20}甚至海中的鱼、天空的鸟、田野的兽,并地上的一切昆虫,和其上的众人,因见我的面就都震动;山岭必崩裂,陡岩必塌陷,墙垣都必坍倒。」
\VS{21}主耶和华说:「我必命我的诸山发刀剑来攻击{\PN{歌革}};人都要用刀剑杀害弟兄。
\VS{22}我必用瘟疫和流血的事刑罚他。我也必将暴雨、大雹与火,并硫磺降与他和他的军队,并他所率领的众民。
\VS{23}我必显为大,显为圣,在多国人的眼前显现;他们就知道我是耶和华。」

\par }\Chap{39}{\SH 歌革被击败
\par }{\PP \VerseOne{1}「人子啊,你要向{\PN{歌革}}发预言攻击他,说主耶和华如此说:{\PN{罗施}}、{\PN{米设}}、{\PN{土巴}}的王{\PN{歌革}}啊,我与你为敌。
\VS{2}我必调转你,领你前往,使你从北方的极处上来,带你到{\PN{以色列}}的山上。
\VS{3}我必从你左手打落你的弓,从你右手打掉你的箭。
\VS{4}你和你的军队,并同着你的列国人,都必倒在{\PN{以色列}}的山上。我必将你给各类的鸷鸟和田野的走兽作食物。
\VS{5}你必倒在田野,因为我曾说过。这是主耶和华说的。
\VS{6}我要降火在{\PN{玛各}}和海岛安然居住的人身上,他们就知道我是耶和华。
\par }{\PP \VS{7}「我要在我民{\PN{以色列}}中显出我的圣名,也不容我的圣名再被亵渎,列国人就知道我是耶和华—{\PN{以色列}}中的圣者。
\VS{8}主耶和华说:这日事情临近,也必成就,乃是我所说的日子。
\par }{\PP \VS{9}「住{\PN{以色列}}城邑的人必出去和捡器械,就是大小盾牌、弓箭、梃杖、枪矛,都当柴烧火,直烧七年,
\VS{10}甚至他们不必从田野捡柴,也不必从树林伐木;因为他们要用器械烧火,并且抢夺那抢夺他们的人,掳掠那掳掠他们的人。这是主耶和华说的。」
\par }{\SH 歌革的埋葬
\par }{\PP \VS{11}「当那日,我必将{\PN{以色列}}{\ADD{地}}的谷,就是海东人所经过的谷,赐给{\PN{歌革}}为坟地,使经过的人到此停步。在那里人必葬埋{\PN{歌革}}和他的群众,就称那地为{\PN{哈们·歌革谷}}。
\VS{12}{\PN{以色列}}家的人必用七个月葬埋他们,为要洁净全地。
\VS{13}全地的居民都必葬埋他们。当我得荣耀的日子,这事必叫他们得名声。这是主耶和华说的。
\VS{14}他们必分派人时常巡查遍地,与过路的人一同葬埋那剩在地面上的尸首,好洁净全地。过了七个月,他们还要巡查。
\VS{15}巡查遍地的人要经过全地,见有人的骸骨,就在旁边立一标记,等葬埋的人来将骸骨葬在{\PN{哈们·歌革谷}}。
\VS{16}他们必这样洁净那地,并有一城名叫{\PN{哈摩那}}。
\par }{\PP \VS{17}「人子啊,主耶和华如此说:你要对各类的飞鸟和田野的走兽说:你们聚集来吧,要从四方聚到我为你们献祭之地,就是在{\PN{以色列}}山上献大祭之地,好叫你们吃肉、喝血。
\VS{18}你们必吃勇士的肉,喝地上首领的血,就如吃公绵羊、羊羔、公山羊、公牛,都是{\PN{巴珊}}的肥畜。
\VS{19}你们吃我为你们所献的祭,必吃饱了脂油,喝醉了血。
\VS{20}你们必在我席上饱吃马匹和{\ADD{坐}}车{\ADD{的人}},并勇士和一切的战士。这是主耶和华说的。」
\par }{\SH 以色列的复兴
\par }{\PP \VS{21}「我必显我的荣耀在列国中;万民就必看见我所行的审判与我在他们身上所加的手。
\VS{22}这样,从那日以后,{\PN{以色列}}家必知道我是耶和华—他们的 神。
\VS{23}列国人也必知道{\PN{以色列}}家被掳掠是因他们的罪孽。他们得罪我,我就掩面不顾,将他们交在敌人手中,他们便都倒在刀下。
\VS{24}我是照他们的污秽和罪过待他们,并且我掩面不顾他们。」
\par }{\PP \VS{25}主耶和华如此说:「我要使{\PN{雅各}}被掳的人归回,要怜悯{\PN{以色列}}全家,又为我的圣名发热心。
\VS{26-27}他们在本地安然居住,无人惊吓,是我将他们从万民中领回,从仇敌之地召来。我在许多国的民眼前,在他们身上显为圣的时候,他们要担当自己的羞辱和干犯我的一切罪。
\VS{28}因我使他们被掳到外邦人中,后又聚集他们归回本地,他们就知道我是耶和华—他们的 神;我必不再留他们一人在外邦。
\VS{29}我也不再掩面不顾他们,因我已将我的灵浇灌{\PN{以色列}}家。这是主耶和华说的。」

\par }\Chap{40}{\SH 新圣殿的异象
\par }{\R (40·1—48·35)
\par }{\SH 以西结被带到耶路撒冷
\par }{\PP \VerseOne{1}我们被掳掠第二十五年,{\PN{耶路撒冷}}城攻破后十四年,正在年初,月之初十日,耶和华的灵\FTNT{}{{\FR 40:1: }原文是手}降在我身上,他把我带到{\PN{以色列}}地。
\VS{2}在 神的异象中带我到{\PN{以色列}}地,安置在至高的山上;在山上的南边有仿佛一座城建立。
\VS{3}他带我到那里,见有一人,颜色\FTNT{}{{\FR 40:3: }原文是形状}如铜,手拿麻绳和量度的竿,站在门口。
\VS{4}那人对我说:「人子啊,凡我所指示你的,你都要用眼看,用耳听,并要放在心上。我带你到这里来,特为要指示你;凡你所见的,你都要告诉{\PN{以色列}}家。」
\par }{\SH 东门
\par }{\PP \VS{5}我见殿四围有墙。那人手拿量度的竿,长六肘,每肘是一肘零一掌。他用竿量墙,厚一竿,高一竿。
\VS{6}他到了朝东的门,就上门的台阶,量门的这槛,宽一竿;又量门的那槛,宽一竿。
\VS{7}又有卫房,每房长一竿,宽一竿,相隔五肘。门槛,就是挨着向殿的廊门槛,宽一竿。
\VS{8}他又量向殿门的廊子,宽一竿。
\VS{9}又量门廊,宽八肘,墙柱厚二肘;那门的廊子向着殿。
\VS{10}东门洞有卫房:这旁三间,那旁三间,都是一样的尺寸;这边的柱子和那边的柱子,也是一样的尺寸。
\VS{11}他量门口,宽十肘,长十三肘。
\VS{12}卫房前{\ADD{展出}}的境界:{\ADD{这边}}一肘,那边一肘;卫房这边六肘,那边六肘。
\VS{13}又量门洞,从这卫房顶{\ADD{的后檐}}到那卫房顶{\ADD{的后檐}},宽二十五肘;{\ADD{卫房}}门与门相对。
\VS{14}又量\FTNT{}{{\FR 40:14: }原文是造}廊子六十肘\FTNT{}{{\FR 40:14: }七十士译本是二十肘},墙柱外是院子,有廊为界,在门洞两边。
\VS{15}从大门口到内廊前,共五十肘。
\VS{16}卫房和门洞两旁柱间并廊子,都有严紧的窗棂;里边都有窗棂,柱上有雕刻的棕树。
\par }{\SH 外院
\par }{\PP \VS{17}他带我到外院,见院的四围有铺石地;铺石地上有屋子三十间。
\VS{18}铺石地,就是矮铺石地在各门洞两旁,以门洞的长短为度。
\VS{19}他从下门量到内院外,共宽一百肘,东面北面都是如此。
\par }{\SH 北门
\par }{\PP \VS{20}他量外院朝北的门,长宽若干。
\VS{21}门洞的卫房,这旁三间,那旁三间。门洞的柱子和廊子,与第一门的尺寸一样。门洞长五十肘,宽二十五肘。
\VS{22}其窗棂和廊子,并雕刻的棕树,与朝东的门尺寸一样。登七层台阶上到这门,前面有廊子。
\VS{23}内院有门与这门相对,北面东面都是如此。他从这门量到那门,共一百肘。
\par }{\SH 南门
\par }{\PP \VS{24}他带我往南去,见朝南有门,又照先前的尺寸量门洞的柱子和廊子。
\VS{25}门洞两旁与廊子的周围都有窗棂,和先量的窗棂一样。门洞长五十肘,宽二十五肘。
\VS{26}登七层台阶上到这门,前面有廊子;柱上有雕刻的棕树,这边一棵,那边一棵。
\VS{27}内院朝南有门。从这门量到朝南的那门,共一百肘。
\par }{\SH 内院的南门
\par }{\PP \VS{28}他带我从南门到内院,就照先前的尺寸量南门。
\VS{29}卫房和柱子,并廊子都照先前的尺寸。门洞两旁与廊子的周围都有窗棂。门洞长五十肘,宽二十五肘。
\VS{30}周围有廊子,长二十五肘,宽五肘。
\VS{31}廊子朝着外院,柱上有雕刻的棕树。登八层台阶上到这门。
\par }{\SH 内院的东门
\par }{\PP \VS{32}他带我到内院的东面,就照先前的尺寸量东门。
\VS{33}卫房和柱子,并廊子都照先前的尺寸。门洞两旁与廊子的周围都有窗棂。门洞长五十肘,宽二十五肘。
\VS{34}廊子朝着外院。门洞两旁的柱子都有雕刻的棕树。登八层台阶上到这门。
\par }{\SH 内院的北门
\par }{\PP \VS{35}他带我到北门,就照先前的尺寸量那门,
\VS{36}就是量卫房和柱子,并廊子。门洞周围都有窗棂;门洞长五十肘,宽二十五肘。
\VS{37}廊柱朝着外院。门洞两旁的柱子都有雕刻的棕树。登八层台阶上到这门。
\par }{\SH 厢房
\par }{\PP \VS{38}门洞的柱旁有屋子和门;祭司\FTNT{}{{\FR 40:38: }原文是他们}在那里洗燔祭牲。
\VS{39}在门廊内,这边有两张桌子,那边有两张桌子,在其上可以宰杀燔祭牲、赎罪祭牲,和赎愆祭牲。
\VS{40}上到朝北的门口,这边有两张桌子,门廊那边也有两张桌子。
\VS{41}门这边有四张桌子,那边有四张桌子,共八张;在其上祭司宰杀{\ADD{牺牲}}。
\VS{42}为燔祭牲有四张桌子,是凿过的石头做成的,长一肘半,宽一肘半,高一肘。祭司将宰杀燔祭牲和{\ADD{平安}}祭牲所用的器皿放在其上。
\VS{43}有钩子,宽一掌,钉在廊内的四围。桌子上有牺牲的肉。
\par }{\PP \VS{44}在北门旁,内院里有屋子,为歌唱的人而设。这屋子朝南;在南\FTNT{}{{\FR 40:44: }南:原文是东}门旁,又有一间朝北。
\VS{45}他对我说:「这朝南的屋子是为看守殿宇的祭司;
\VS{46}那朝北的屋子是为看守祭坛的祭司。这些祭司是{\PN{利未}}人中{\PN{撒督}}的子孙,近前来事奉耶和华的。」
\par }{\SH 内院和圣殿
\par }{\PP \VS{47}他又量内院,长一百肘,宽一百肘,是见方的。祭坛在殿前。
\VS{48}于是他带我到殿前的廊子,量廊子的墙柱。这面厚五肘,那面厚五肘。门两旁,这边三肘,那边三肘。
\VS{49}廊子长二十肘,宽十一肘。上廊子有台阶。靠近墙柱又有柱子,这边一根,那边一根。

\par }\Chap{41}{\PP \VerseOne{1}他带我到殿那里量墙柱:这面厚六肘,那面厚六肘,宽窄与会幕相同。
\VS{2}门口宽十肘。门两旁,这边五肘,那边五肘。他量殿长四十肘,宽二十肘。
\VS{3}他到内殿量墙柱,各厚二肘。门口宽六肘,门两旁各宽七肘。
\VS{4}他量内殿,长二十肘,宽二十肘。他对我说:「这是至圣所。」
\par }{\SH 靠圣殿墙边的厢房
\par }{\PP \VS{5}他又量殿墙,厚六肘;围着殿有旁屋,各宽四肘。
\VS{6}旁屋有三层,层叠而上,每层排列三十间。旁屋{\ADD{的梁木}}搁在殿墙坎上,免得插入殿墙。
\VS{7}这围{\ADD{殿}}的旁屋越高越宽;因旁屋围殿悬叠而上,所以越上越宽,从下一层,由中一层,到上一层。
\VS{8}我又见围着殿有高月台。旁屋的根基,高足一竿,就是六大肘。
\VS{9}旁屋的外墙厚五肘。旁屋之外还有余地。
\VS{10}在旁屋与对面的房屋中间有{\ADD{空地}},宽二十肘。
\VS{11}旁屋的门都向余地:一门向北,一门向南。周围的余地宽五肘。
\par }{\SH 西边的屋子
\par }{\PP \VS{12}在西面空地之后有房子,宽七十肘,长九十肘,墙四围厚五肘。
\par }{\SH 圣殿的尺寸
\par }{\PP \VS{13}这样,他量殿,长一百肘,又量空地和那房子并墙,共长一百肘。
\VS{14}殿的前面和两旁的空地,宽一百肘。
\par }{\PP \VS{15}他量空地后面的那房子,并两旁的楼廊,共长一百肘。
\par }{\SH 圣殿的内部
\par }{\PP \VS{16}内殿、院廊、门槛、严紧的窗棂,并对着门槛的三层楼廊,从地到窗棂(窗棂都有蔽子),
\VS{17}直到门以上,就是到内殿和外殿内外四围墙壁,都按尺寸用木板遮蔽。
\VS{18}墙上雕刻基路伯和棕树。每二基路伯中间有一棵棕树,每基路伯有二脸。
\VS{19}这边有人脸向着棕树,那边有狮子脸向着棕树,殿内周围都是如此。
\VS{20}从地至门以上,都有基路伯和棕树。殿墙就是这样。
\par }{\PP \VS{21}殿的门柱是方的。至圣所的前面,形状和{\ADD{殿}}的形状一样。
\VS{22}坛是木头做的,高三肘,长二肘。坛角和坛面,并四旁,都是木头做的。他对我说:「这是耶和华面前的桌子。」
\par }{\SH 门
\par }{\PP \VS{23}殿和至圣所的门各有两扇。
\VS{24}每扇分两扇,这两扇是折叠的。这边门分两扇,那边门也分两扇。
\VS{25}殿的门扇上雕刻基路伯和棕树,与刻在墙上的一般。在外头廊前有木槛。
\VS{26}廊这边那边都有严紧的窗棂和棕树;殿的旁屋和槛就是这样。

\par }\Chap{42}{\SH 圣殿附近的两座屋子
\par }{\PP \VerseOne{1}他带我出来向北,到外院,又带我进入圣屋;这圣屋一排顺着空地,一排与北边{\ADD{铺石地}}之屋相对。
\VS{2}这圣屋长一百肘,宽五十肘,有向北的门。
\VS{3}对着内院那二十{\ADD{肘宽之空地}},又对着外院的铺石地,在第三层楼上有楼廊对着楼廊。
\VS{4}在圣屋前有一条夹道,宽十肘,长一百肘。房门都向北。
\VS{5}圣屋因为楼廊占去些地方,所以上层比中下两层窄些。
\VS{6}圣屋有三层,却无柱子,不像外院{\ADD{的屋子}}有柱子;所以{\ADD{上层}}比中下两层更窄。
\VS{7}圣屋外,东边有墙,靠着外院,长五十肘。
\VS{8}靠着外院的圣屋长五十肘。殿北面{\ADD{的圣屋}}长一百肘。
\VS{9}在圣屋以下,东头有进入之处,就是从外院进入之处。
\par }{\PP \VS{10}向南\FTNT{}{{\FR 42:10: }原文是东}在内院墙里有圣屋,一排与铺石地之屋相对,一排顺着空地。
\VS{11}这圣屋前的夹道与北边圣屋的{\ADD{夹道}}长宽一样;出入之处与北屋门的样式相同。
\VS{12}正在墙前、夹道的东头,有门可以进入,与向南圣屋的门一样。
\par }{\PP \VS{13}他对我说:「顺着空地的南屋北屋,都是圣屋;亲近耶和华的祭司当在那里吃至圣的物,也当在那里放至圣的物,就是素祭、赎罪祭,和赎愆祭,因此处为圣。
\VS{14}祭司进去出了圣所的时候,不可直到外院,但要在圣屋放下他们供职的衣服,因为是圣衣;要穿上别的衣服才可以到属民的{\ADD{外院}}。」
\par }{\SH 圣殿周围的尺寸
\par }{\PP \VS{15}他量完了内殿,就带我出朝东的门,量院的四围。
\VS{16}他用量度的竿量四围,量东面五百肘\FTNT{}{{\FR 42:16: }原文是竿;本章下同},
\VS{17}用竿量北面五百肘,
\VS{18}用竿量南面五百肘,
\VS{19}又转到西面,用竿量五百肘。
\VS{20}他量四面,四围有墙,长五百{\ADD{肘}},宽五百{\ADD{肘}},为要分别圣地与俗地。

\par }\Chap{43}{\SH 耶和华回到圣殿
\par }{\PP \VerseOne{1}以后,他带我到一座门,就是朝东的门。
\VS{2}{\PN{以色列}} 神的荣光从东而来。他的声音如同多水的声音;地就因他的荣耀发光。
\VS{3}其状如从前他来灭城的时候我所见的异象,那异象如我在{\PN{迦巴鲁河}}边所见的异象,我就俯伏在地。
\VS{4}耶和华的荣光从朝东的门照入殿中。
\VS{5}灵将我举起,带入内院,不料,耶和华的荣光充满了殿。
\par }{\PP \VS{6}我听见有一位从殿中对我说话。有一人站在我旁边。
\VS{7}他对我说:「人子啊,这是我宝座之地,是我脚掌所踏之地。我要在这里住,在{\PN{以色列}}人中直到永远。{\PN{以色列}}家和他们的君王必不再玷污我的圣名,就是行邪淫、在{\ADD{
{\PN{锡安}} 的}}高处{\ADD{葬埋}}他们君王的尸首,
\VS{8}使他们的门槛挨近我的门槛,他们的门框挨近我的门框;他们与我中间仅隔一墙,并且行可憎的事,玷污了我的圣名,所以我发怒灭绝他们。
\VS{9}现在他们当从我面前远除邪淫和他们君王的尸首,我就住在他们中间直到永远。
\par }{\PP \VS{10}「人子啊,你要将这殿指示{\PN{以色列}}家,使他们因自己的罪孽惭愧,也要他们量殿的尺寸。
\VS{11}他们若因自己所行的一切事惭愧,你就将殿的规模、样式、出入之处,和一切形状、典章、礼仪、法则指示他们,在他们眼前写上,使他们遵照殿的一切规模典章去做。
\VS{12}殿的法则乃是如此:殿在山顶上,四围的全界要称为至圣。这就是殿的法则。」
\par }{\SH 祭坛
\par }{\PP \VS{13}以下量祭坛,是以肘为度(这肘是一肘零一掌)。底座高一肘,边宽一肘,四围起边高一掌,这是坛的座。
\VS{14}从底座到下层磴台,高二肘,边宽一肘。从小磴台到大磴台,高四肘,边宽一肘。
\VS{15}坛上的供台,高四肘。供台的四拐角上都有角。
\VS{16}供台长十二{\ADD{肘}},宽十二肘,四面见方。
\VS{17}磴台长十四{\ADD{肘}},宽十四肘,四面见方。四围起边高半肘,底座四围的边宽一肘。台阶朝东。
\par }{\SH 奉献祭坛
\par }{\PP \VS{18}他对我说:「人子啊,主耶和华如此说:建造祭坛,为要在其上献燔祭洒血,造成的时候典章如下:
\VS{19}主耶和华说,你要将一只公牛犊作为赎罪祭,给祭司{\PN{利未}}人{\PN{撒督}}的后裔,就是那亲近我、事奉我的。
\VS{20}你要取些公牛的血,抹在坛的四角和磴台的四拐角,并四围所起的边上。你这样洁净坛,坛就洁净了。
\VS{21}你又要将那作赎罪祭的公牛{\ADD{犊}}烧在殿外、圣地之外预定之处。
\VS{22}次日,要将无残疾的公山羊献为赎罪祭;要洁净坛,像用公牛{\ADD{犊}}洁净的一样。
\VS{23}洁净了坛,就要将一只无残疾的公牛犊和羊群中一只无残疾的公绵羊
\VS{24}奉到耶和华前。祭司要撒盐在其上,献与耶和华为燔祭。
\VS{25}七日内,每日要预备一只公山羊为赎罪祭,也要预备一只公牛犊和羊群中的一只公绵羊,都要没有残疾的。
\VS{26}七日祭司洁净坛,坛就洁净了;要这样把坛分别为圣。
\VS{27}满了七日,自八日以后,祭司要在坛上献你们的燔祭和平安祭;我必悦纳你们。这是主耶和华说的。」

\par }\Chap{44}{\SH 东门的使用
\par }{\PP \VerseOne{1}他又带我回到圣地朝东的外门;那门关闭了。
\VS{2}耶和华对我说:「这门必须关闭,不可敞开,谁也不可由其中进入;因为耶和华—以色列的 神已经由其中进入,所以必须关闭。
\VS{3}至于王,他必按王的位分,坐在其内,在耶和华面前吃饼。他必由这门的廊而入,也必由此而出。」
\par }{\SH 进入圣殿的规则
\par }{\PP \VS{4}他又带我由北门来到殿前。我观看,见耶和华的荣光充满耶和华的殿,我就俯伏在地。
\VS{5}耶和华对我说:「人子啊,我对你所说耶和华殿中的一切典章法则,你要放在心上,用眼看,用耳听,并要留心殿宇和圣地一切出入之处。
\VS{6}你要对那悖逆的{\PN{以色列}}家说,主耶和华如此说:{\PN{以色列}}家啊,你们行一切可憎的事,当够了吧!
\VS{7}你们把我的食物,就是脂油和血献上的时候,将身心未受割礼的外邦人领进我的圣地,玷污了我的殿;又背了我的约,在你们一切可憎的事上,加上{\ADD{这一层}}。
\VS{8}你们也没有看守我的圣物,却派别人在圣地替你们看守我所吩咐你们的。
\par }{\PP \VS{9}「主耶和华如此说:{\PN{以色列}}中的外邦人,就是身心未受割礼的,都不可入我的圣地。」
\par }{\SH 不准当祭司的利未人
\par }{\PP \VS{10}「当{\PN{以色列}}人走迷的时候,有{\PN{利未}}人远离我,就是走迷离开我、随从他们的偶像,他们必担当自己的罪孽。
\VS{11}然而他们必在我的圣地当仆役,照管殿门,在殿中供职;必为民宰杀燔祭牲和{\ADD{平安}}祭牲,必站在民前伺候他们。
\VS{12}因为这些{\PN{利未}}人曾在偶像前伺候这民,成了{\PN{以色列}}家罪孽的绊脚石,所以我向他们起誓:他们必担当自己的罪孽。这是主耶和华说的。
\VS{13}他们不可亲近我,给我供祭司的职分,也不可挨近我的一件圣物,就是至圣的物;他们却要担当自己的羞辱和所行可憎之事的报应。
\VS{14}然而我要使他们看守殿宇,办理其中的一切事,并做其内一切当做之工。」
\par }{\SH 祭司
\par }{\PP \VS{15}「{\PN{以色列}}人走迷离开我的时候,祭司{\PN{利未}}人{\PN{撒督}}的子孙仍看守我的圣所。他们必亲近我,事奉我,并且侍立在我面前,将脂油与血献给我。这是主耶和华说的。
\VS{16}他们必进入我的圣所,就近我的桌前事奉我,守我所吩咐的。
\VS{17}他们进内院门必穿细麻衣。在内院门和殿内供职的时候不可穿羊毛衣服。
\VS{18}他们头上要戴细麻布裹头巾,腰穿细麻布裤子;不可穿使身体出汗{\ADD{的衣服}}。
\VS{19}他们出到外院的民那里,当脱下供职的衣服,放在圣屋内,穿上别的衣服,免得因圣衣使民成圣。
\VS{20}不可剃头,也不可容发绺长长,只可剪发。
\VS{21}祭司进内院的时候都不可喝酒。
\VS{22}不可娶寡妇和被休的妇人为妻,只可娶{\PN{以色列}}后裔中的处女,或是祭司遗留的寡妇。
\VS{23}他们要使我的民知道圣俗的分别,又使他们分辨洁净的和不洁净的。
\VS{24}有争讼的事,他们应当站立判断,要按我的典章判断。在我一切的节期必守我的律法、条例,也必以我的安息日为圣日。
\VS{25}他们不可挨近死尸沾染自己,只可为父亲、母亲、儿子、女儿、弟兄,和未嫁人的姊妹沾染自己。
\VS{26}祭司洁净之后,必再计算七日。
\VS{27}当他进内院,进圣所,在圣所中事奉的日子,要为自己献赎罪祭。这是主耶和华说的。
\par }{\PP \VS{28}「祭司必有产业,我是他们的产业。不可在{\PN{以色列}}中给他们基业;我是他们的基业。
\VS{29}素祭、赎罪祭,和赎愆祭他们都可以吃,{\PN{以色列}}中一切永献的物都要归他们。
\VS{30}首先初熟之物和一切所献的供物都要归给祭司。你们也要用初熟的麦子磨面给祭司;这样,福气就必临到你们的家了。
\VS{31}无论是鸟是兽,凡自死的,或是撕裂的,祭司都不可吃。」

\par }\Chap{45}{\SH 归给耶和华的土地
\par }{\PP \VerseOne{1}你们拈阄分地为业,要献上一分给耶和华为圣供地,长二万五千{\ADD{肘}},宽一万{\ADD{肘}}。这分以内,四围都为圣地。
\VS{2}其中有作为圣所之地,{\ADD{长}}五百{\ADD{肘}},{\ADD{宽}}五百{\ADD{肘}},四面见方。四围再有五十肘为郊野之地。
\VS{3}要以肘为度量地,长二万五千{\ADD{肘}},宽一万{\ADD{肘}}。其中有圣所,是至圣的。
\VS{4}这是全地的一分圣地,要归与供圣所职事的祭司,就是亲近事奉耶和华的,作为他们房屋之地与圣所之圣地。
\VS{5}又有一分,长二万五千{\ADD{肘}},宽一万{\ADD{肘}},要归与在殿中供职的{\PN{利未}}人,作为二十间房屋之业。
\par }{\PP \VS{6}也要分定属城的地业,宽五千{\ADD{肘}},长二万五千{\ADD{肘}},挨着那分圣供地,要归{\PN{以色列}}全家。
\par }{\SH 君王的土地
\par }{\PP \VS{7}归王{\ADD{之地}}要在圣供地和属城之地的两旁,就是圣供地和属城之地的旁边,西至西头,东至东头,从西到东,其长与{\ADD{每支派的}}分一样。
\VS{8}这地在{\PN{以色列}}中必归王为业。我所立的王必不再欺压我的民,却要按支派将地分给{\PN{以色列}}家。
\par }{\SH 君王应守的规例
\par }{\PP \VS{9}主耶和华如此说:「{\PN{以色列}}的王啊,你们应当知足,要除掉强暴和抢夺的事,施行公平和公义,不再勒索我的民。这是主耶和华说的。
\par }{\PP \VS{10}「你们要用公道天平、公道伊法、公道罢特。
\VS{11}伊法与罢特大小要一样。罢特可盛贺梅珥十分之一,伊法也可盛贺梅珥十分之一,都以贺梅珥的大小为准。
\VS{12}舍客勒是二十季拉;二十舍客勒,二十五舍客勒,十五舍客勒,为你们的弥那。
\par }{\PP \VS{13}「你们当献的供物乃是这样:一贺梅珥麦子要献伊法六分之一;一贺梅珥大麦要献伊法六分之一。
\VS{14}你们献所分定的油,按油的罢特,一柯珥油要献罢特十分之一(原来十罢特就是一贺梅珥)。
\VS{15}从{\PN{以色列}}滋润的草场上每二百羊中,要献一只羊羔。这都可作素祭、燔祭、平安祭,为民赎罪。这是主耶和华说的。
\VS{16}此地的民都要奉上这供物给{\PN{以色列}}中的王。
\VS{17}王的本分是在节期、月朔、安息日,就是{\PN{以色列}}家一切的节期,奉上燔祭、素祭、奠祭。他要预备赎罪祭、素祭、燔祭,和平安祭,为{\PN{以色列}}家赎罪。」
\par }{\SH 节期
\par }{\PP \VS{18}主耶和华如此说:「正{\ADD{月}}初一{\ADD{日}},你要取无残疾的公牛犊,洁净圣所。
\VS{19}祭司要取些赎罪祭牲的血,抹在殿的门柱上和坛磴台的四角上,并内院的门框上。
\VS{20}本月初七日\FTNT{}{{\FR 45:20: }七十士译本是七月初一日}也要为误犯罪的和愚蒙犯罪的如此行,为殿赎罪。
\par }{\PP \VS{21}「正月十四日,你们要守逾越节,守节七日,要吃无酵饼。
\VS{22}当日,王要为自己和国内的众民预备一只公牛作赎罪祭。
\VS{23}这节的七日,每日他要为耶和华预备无残疾的公牛七只、公绵羊七只为燔祭。每日又要预备公山羊一只为赎罪祭。
\VS{24}他也要预备素祭,就是为一只公牛同献一伊法{\ADD{细面}},为一只公绵羊同献一伊法{\ADD{细面}},每一伊法{\ADD{细面}}加油一欣。
\VS{25}七月十五日守节的时候,七日他都要如此行,照{\ADD{逾越节的}}赎罪祭、燔祭、素祭,和油的条例一样。」

\par }\Chap{46}{\SH 君王和节期
\par }{\PP \VerseOne{1}主耶和华如此说:「内院朝东的门,在办理事务的六日内必须关闭;惟有安息日和月朔必须敞开。
\VS{2}王要从这门的廊进入,站在门框旁边。祭司要为他预备燔祭和平安祭,他就要在门槛那里敬拜,然后出去。这门直到晚上不可关闭。
\VS{3}在安息日和月朔,国内的居民要在这门口,耶和华面前敬拜。
\VS{4}安息日,王所献与耶和华的燔祭要用无残疾的羊羔六只,无残疾的公绵羊一只;
\VS{5}同献的素祭要为公绵羊献一伊法{\ADD{细面}},为羊羔照他的力量而献,一伊法{\ADD{细面}}加油一欣。
\VS{6}当月朔,要献无残疾的公牛犊一只,羊羔六只,公绵羊一只,都要无残疾的。
\VS{7}他也要预备素祭,为公牛献一伊法{\ADD{细面}},为公绵羊献一伊法{\ADD{细面}},为羊羔照他的力量而献,一伊法{\ADD{细面}}加油一欣。
\VS{8}王进入的时候必由这门的廊而入,也必由此而出。
\par }{\PP \VS{9}「在各节期,国内居民朝见耶和华的时候,从北门进入敬拜的,必由南门而出;从南门进入的,必由北门而出。不可从所入的门而出,必要直往前行,{\ADD{由对门}}而出。
\VS{10}民进入,王也要在民中进入;民出去,王也要一同出去。
\par }{\PP \VS{11}「在节期和圣会的日子同献的素祭,要为一只公牛献一伊法{\ADD{细面}},为一只公绵羊献一伊法{\ADD{细面}},为羊羔照他的力量而献,一伊法{\ADD{细面}}加油一欣。
\VS{12}王预备甘心献的燔祭或平安祭,就是向耶和华甘心献的,当有人为他开朝东的门。他就预备燔祭和平安祭,与安息日预备的一样,{\ADD{献毕}}就出去。他出去之后,当有人将门关闭。」
\par }{\SH 每日的祭
\par }{\PP \VS{13}「每日,你要预备无残疾一岁的羊羔一只,献与耶和华为燔祭;要每早晨预备。
\VS{14}每早晨也要预备同献的素祭,{\ADD{细面}}一伊法六分之一,并油一欣三分之一,调和细面。这素祭要常献与耶和华为永远的定例。
\VS{15}每早晨要这样预备羊羔、素祭,并油为常献的燔祭。」
\par }{\SH 王和土地
\par }{\PP \VS{16}主耶和华如此说:「王若将产业赐给他的儿子,就成了他儿子的产业,那是他们承受为业的。
\VS{17}倘若王将一分产业赐给他的臣仆,就成了他臣仆的产业;到自由之年仍要归与王。至于王的产业,必归与他的儿子。
\VS{18}王不可夺取民的产业,以致驱逐他们离开所承受的;他要从自己的地业中,将产业赐给他儿子,免得我的民分散,各人离开所承受的。」
\par }{\SH 圣殿的厨房
\par }{\PP \VS{19}那带我的,将我从门旁进入之处、领进为祭司预备的圣屋,是朝北的,见后头西边有一块地。
\VS{20}他对我说:「这是祭司煮赎愆祭、赎罪祭,烤素祭之地,免得带到外院,使民成圣。」
\par }{\PP \VS{21}他又带我到外院,使我经过院子的四拐角,见每拐角各有一个院子。
\VS{22}院子四拐角的院子,周围有墙,每院长四十{\ADD{肘}},宽三十{\ADD{肘}}。四拐角院子的尺寸都是一样,
\VS{23}其中周围有一排房子,房子内有煮{\ADD{肉}}的地方。
\VS{24}他对我说:「这都是煮{\ADD{肉}}的房子,殿内的仆役要在这里煮民的祭物。」

\par }\Chap{47}{\SH 从圣殿流出的水
\par }{\PP \VerseOne{1}他带我回到殿门,见殿的门槛下有水往东流出(原来殿面朝东)。这水从槛下,由殿的右边,在祭坛的南边往下流。
\VS{2}他带我出北门,又领我从外边转到朝东的外门,见水从右边流出。
\par }{\PP \VS{3}他手拿准绳往东出去的时候,量了一千肘,使我趟过水,水到踝子骨。
\VS{4}他又量了一千{\ADD{肘}},使我趟过水,水就到膝;再量了一千{\ADD{肘}},使我趟过{\ADD{水}},水便到腰;
\VS{5}又量了一千{\ADD{肘}},水便成了河,使我不能趟过。因为水势涨起,成为可洑的水,不可趟的河。
\VS{6}他对我说:「人子啊,你看见了什么?」
\par }{\PP 他就带我回到河边。
\VS{7}我回到河边的时候,见在河这边与那边的岸上有极多的树木。
\VS{8}他对我说:「这水往东方流去,必下到{\PN{亚拉巴}},直到海。所发出来的{\ADD{水}}必流入{\ADD{
{\PN{盐}}}}{\PN{海}},使水变甜\FTNT{}{{\FR 47:8: }原文是得医治;下同}。
\VS{9}这河水所到之处,凡滋生的动物都必生活,并且因这流来的水必有极多的鱼,{\ADD{海水}}也变甜了。这河水所到之处,百物都必生活。
\VS{10}必有渔夫站在河边,从{\PN{隐·基底}}直到{\PN{隐·以革莲}},都作晒\FTNT{}{{\FR 47:10: }或译:张}网之处。那鱼各从其类,好像大海的鱼甚多。
\VS{11}只是泥泞之地与洼湿之处不得治好,必为盐地。
\VS{12}在河这边与那边的岸上必生长各类的树木;{\ADD{其果}}可作食物,叶子不枯干,果子不断绝。每月必结新果子,因为这水是从圣所流出来的。树上的果子必作食物,叶子乃为治病。」
\par }{\SH 地界
\par }{\PP \VS{13}主耶和华如此说:「你们要照地的境界,按{\PN{以色列}}十二支派分地为业。{\PN{约瑟}}{\ADD{必得两}}分。
\VS{14}你们承受这地为业,要彼此均分;因为我曾起誓应许将这地赐与你们的列祖;这地必归你们为业。
\par }{\PP \VS{15}「地的四界乃是如此:北界从大海往{\PN{希特伦}},直到{\PN{西达达}}口。
\VS{16}又往{\PN{哈马}}、{\PN{比罗他}}、{\PN{西伯莲}}({\PN{西伯莲}}在{\PN{大马士革}}与{\PN{哈马}}两界中间),到{\PN{浩兰}}边界的{\PN{哈撒·哈提干}}。
\VS{17}这样,境界从海边往{\PN{大马士革}}地界上的{\PN{哈萨·以难}},北边以{\PN{哈马}}地为界。这是北界。
\par }{\PP \VS{18}「东界在{\PN{浩兰}}、{\PN{大马士革}}、{\PN{基列}},和{\PN{以色列}}地的中间,就是{\PN{约旦河}}。你们要从{\ADD{北}}界量到东海。这是东界。
\par }{\PP \VS{19}「南界是从{\PN{他玛}}到{\PN{米利巴·加低斯}}的水,延到{\ADD{
{\PN{埃及}}}}小河,直到大海。这是南界。
\par }{\PP \VS{20}「西界就是大海,从{\ADD{南}}界直到{\PN{哈马}}口对面之地。这是西界。
\par }{\PP \VS{21}「你们要按着{\PN{以色列}}的支派彼此分这地。
\VS{22}要拈阄分这地为业,归与自己和你们中间寄居的外人,就是在你们中间生养儿女的外人。你们要看他们如同{\PN{以色列}}人中所生的一样;他们在{\PN{以色列}}支派中要与你们同得地业。
\VS{23}外人寄居在哪支派中,你们就在那里分给他地业。这是主耶和华说的。」

\par }\Chap{48}{\SH 各支派分配土地
\par }{\PP \VerseOne{1}众支派按名所得{\ADD{之地}}记在下面:从北头,由{\PN{希特伦}}往{\PN{哈马}}口,到{\PN{大马士革}}地界上的{\PN{哈萨·以难}}。北边靠着{\PN{哈马}}地(各支派{\ADD{的地}}都有东西的边界),是{\PN{但}}的一{\ADD{分}}。
\VS{2}挨着{\PN{但}}的地界,从东到西,是{\PN{亚设}}的一{\ADD{分}}。
\VS{3}挨着{\PN{亚设}}的地界,从东到西,是{\PN{拿弗他利}}的一{\ADD{分}}。
\VS{4}挨着{\PN{拿弗他利}}的地界,从东到西,是{\PN{玛拿西}}的一{\ADD{分}}。
\VS{5}挨着{\PN{玛拿西}}的地界,从东到西,是{\PN{以法莲}}的一{\ADD{分}}。
\VS{6}挨着{\PN{以法莲}}的地界,从东到西,是{\PN{吕便}}的一{\ADD{分}}。
\VS{7}挨着{\PN{吕便}}的地界,从东到西,是{\PN{犹大}}的一{\ADD{分}}。
\par }{\SH 中部特区
\par }{\PP \VS{8}挨着{\PN{犹大}}的地界,从东到西,必有你们所当献的供地,宽二万五千{\ADD{肘}}。从东界到西界,长短与各分之地相同,圣地当在其中。
\VS{9}你们献与耶和华的供地要长二万五千{\ADD{肘}},宽一万{\ADD{肘}}。
\VS{10}这圣供地要归与祭司,北{\ADD{长}}二万五千{\ADD{肘}},西宽一万{\ADD{肘}},东宽一万{\ADD{肘}},南长二万五千{\ADD{肘}}。耶和华的圣地当在其中。
\VS{11}这地要归与{\PN{撒督}}的子孙中成为圣的祭司,就是那守我所吩咐的。当{\PN{以色列}}人走迷的时候,他们不像那些{\PN{利未}}人走迷了。
\VS{12}这要归与他们为供地,是全地中至圣的。供地挨着{\PN{利未}}人的地界。
\VS{13}{\PN{利未}}人所得的地要长二万五千{\ADD{肘}},宽一万{\ADD{肘}},与祭司的地界相等,都长二万五千{\ADD{肘}},宽一万{\ADD{肘}}。
\VS{14}这地不可卖,不可换,初熟之物也不可归与别人,因为是归耶和华为圣的。
\par }{\PP \VS{15}这二万五千{\ADD{肘}}前面所剩下五千{\ADD{肘}}宽之地要作俗用,作为造城盖房郊野之地。城要在当中。
\VS{16}城的尺寸乃是如此:北面四千五百{\ADD{肘}},南面四千五百{\ADD{肘}},东面四千五百{\ADD{肘}},西面四千五百{\ADD{肘}}。
\VS{17}城必有郊野,向北二百五十{\ADD{肘}},向南二百五十{\ADD{肘}},向东二百五十{\ADD{肘}},向西二百五十{\ADD{肘}}。
\VS{18}靠着圣供地的余地,东长一{\ADD{万}}肘,西长一万{\ADD{肘}},要与圣供地相等;其中的土产要作城内工人的食物。
\VS{19}所有{\PN{以色列}}支派中,在城内做工的,都要耕种这地。
\VS{20}你们所献的圣供地连归城之地,是四方的:长二万五千{\ADD{肘}},宽二万五千{\ADD{肘}}。
\par }{\PP \VS{21}圣供地连归城之地,两边的余地要归与王。供地东边,{\ADD{南北}}二万五千{\ADD{肘}},东至东界,西边{\ADD{南北}}二万五千{\ADD{肘}},西至西界,与各分之地相同,都要归王。圣供地和殿的圣地要在其中,
\VS{22}并且{\PN{利未}}人之地,与归城之地的东西两边延长之地(这两地在王地中间),就是在{\PN{犹大}}和{\PN{便雅悯}}两界中间,要归与王。
\par }{\SH 其他支派的土地
\par }{\PP \VS{23}论到其余的支派,从东到西,是{\PN{便雅悯}}的一{\ADD{分}}。
\VS{24}挨着{\PN{便雅悯}}的地界,从东到西,是{\PN{西缅}}的一{\ADD{分}}。
\VS{25}挨着{\PN{西缅}}的地界,从东到西,是{\PN{以萨迦}}的一{\ADD{分}}。
\VS{26}挨着{\PN{以萨迦}}的地界,从东到西,是{\PN{西布伦}}的一{\ADD{分}}。
\VS{27}挨着{\PN{西布伦}}的地界,从东到西,是{\PN{迦得}}的一{\ADD{分}}。
\VS{28}{\PN{迦得}}地的南界是从{\PN{他玛}}到{\PN{米利巴·加低斯}}的水,延到{\PN{埃及}}小河,直到大海。
\VS{29}这就是你们要拈阄分给{\PN{以色列}}支派为业之地,乃是他们各支派所得之分。这是主耶和华说的。
\par }{\SH 耶路撒冷的城门
\par }{\PP \VS{30}城的北面四千五百{\ADD{肘}}。出城之处如下;
\VS{31}城的各门要按{\PN{以色列}}支派的名字。北面有三门,一为{\PN{吕便}}门,一为{\PN{犹大}}门,一为{\PN{利未}}门;
\VS{32}东面四千五百{\ADD{肘}},有三门,一为{\PN{约瑟}}门,一为{\PN{便雅悯}}门,一为{\PN{但}}门;
\VS{33}南面四千五百{\ADD{肘}},有三门,一为{\PN{西缅}}门,一为{\PN{以萨迦}}门,一为{\PN{西布伦}}门;
\VS{34}西面四千五百{\ADD{肘}},有三门,一为{\PN{迦得}}门,一为{\PN{亚设}}门,一为{\PN{拿弗他利}}门。
\VS{35}城四围共一万八千{\ADD{肘}}。从此以后,这城的名字必称为「耶和华的所在」。
\par }