\NormalFont\ShortTitle{使徒行传}
{\MT 使徒行传

\par }\ChapOne{1}{\SH 赐圣灵的应许
\par }{\PP \VerseOne{1}{\PN{提阿非罗}}啊,我已经作了前书,论到耶稣开头一切所行所教训的,
\VS{2}直到他借着圣灵吩咐所拣选的使徒,以后被接上升的日子为止。
\VS{3}他受害之后,用许多的凭据将自己活活地显给使徒看,四十天之久向他们显现,讲说 神国的事。
\VS{4}耶稣和他们聚集的时候,嘱咐他们说:「不要离开{\PN{耶路撒冷}},要等候父所应许的,就是你们听见我说过的。
\VS{5}{\PN{约翰}}是用水施洗,但不多几日,你们要受圣灵的洗。」
\par }{\SH 耶稣升天
\par }{\PP \VS{6}他们聚集的时候,问耶稣说:「主啊,你复兴{\PN{以色列}}国就在这时候吗?」
\VS{7}耶稣对他们说:「父凭着自己的权柄所定的时候、日期,不是你们可以知道的。
\VS{8}但圣灵降临在你们身上,你们就必得着能力,并要在{\PN{耶路撒冷}}、{\PN{犹太}}全地,和{\PN{撒马利亚}},直到地极,作我的见证。」
\VS{9}说了这话,他们正看的时候,他就被取上升,有一朵云彩把他接去,便看不见他了。
\VS{10}当他往上去,他们定睛望天的时候,忽然有两个人身穿白衣,站在旁边,说:
\VS{11}「{\PN{加利利}}人哪,你们为什么站着望天呢?这离开你们被接升天的耶稣,你们见他怎样往天上去,他还要怎样来。」
\par }{\SH 接替犹大职分的人
\par }{\PP \VS{12}有一座山,名叫{\PN{橄榄山}},离{\PN{耶路撒冷}}不远,约有安息日可走的路程。当下,门徒从那里回{\PN{耶路撒冷}}去,
\VS{13}进了城,就上了所住的一间楼房;在那里有{\PN{彼得}}、{\PN{约翰}}、{\PN{雅各}}、{\PN{安得烈}}、{\PN{腓力}}、{\PN{多马}}、{\PN{巴多罗买}}、{\PN{马太}}、{\PN{亚勒腓}}的儿子{\PN{雅各}}、奋锐党的{\PN{西门}},和{\PN{雅各}}的儿子\FTNT{}{{\FR 1:13: }或译:兄弟}{\PN{犹大}}。
\VS{14}这些人同着几个妇人和耶稣的母亲{\PN{马利亚}},并耶稣的弟兄,都同心合意地恒切祷告。
\par }{\PP \VS{15}那时,有许多人{\ADD{聚会}},约有一百二十名,{\PN{彼得}}就在弟兄中间站起来,
\VS{16}说:「弟兄们!圣灵借{\PN{大卫}}的口,在圣经上预言领人捉拿耶稣的{\PN{犹大}},这话是必须应验的。
\VS{17}他本来列在我们数中,并且在{\ADD{使徒的}}职任上得了一分。
\VS{18}这人用他作恶的工价买了一块田,以后身子仆倒,肚腹崩裂,肠子都流出来。
\VS{19}住在{\PN{耶路撒冷}}的众人都知道这事,所以按着他们那里的话给那块田起名叫{\PN{亚革大马}},就是「血田」的意思。
\VS{20}因为诗篇上写着,说:
\par }{\Q 愿他的住处变为荒场,
\par }{\Q 无人在内居住;
\par }{\MM 又说:
\par }{\Q 愿别人得他的职分。
\par }{\MM \VS{21}所以,主耶稣在我们中间始终出入的时候,
\VS{22}就是从{\PN{约翰}}施洗起,直到主离开我们被接上升的日子为止,必须从那常与我们作伴的人中立一位与我们同作耶稣复活的见证。」
\VS{23}于是选举两个人,就是那叫做{\PN{巴撒巴}},又称呼{\PN{犹士都}}的{\PN{约瑟}},和{\PN{马提亚}}。
\VS{24-25}众人就祷告说:「主啊,你知道万人的心,求你从这两个人中,指明你所拣选的是谁,叫他得这使徒的位分。这位分{\PN{犹大}}已经丢弃,往自己的地方去了。」
\VS{26}于是众人为他们摇签,摇出{\PN{马提亚}}来;他就和十一个使徒同列。

\par }\Chap{2}{\SH 圣灵降临
\par }{\PP \VerseOne{1}五旬节到了,门徒都聚集在一处。
\VS{2}忽然,从天上有响声下来,好像一阵大风吹过,充满了他们所坐的屋子,
\VS{3}又有舌头如火焰显现出来,分开落在他们各人头上。
\VS{4}他们就都被圣灵充满,按着{\ADD{圣}}灵所赐的口才说起别国的话来。
\par }{\PP \VS{5}那时,有虔诚的{\PN{犹太}}人从天下各国来,住在{\PN{耶路撒冷}}。
\VS{6}这声音一响,众人都来聚集,各人听见门徒用众人的乡谈说话,就甚纳闷;
\VS{7}都惊讶希奇说:「看哪,这说话的不都是{\PN{加利利}}人吗?
\VS{8}我们各人怎么听见他们说我们生来所用的乡谈呢?
\VS{9}我们{\PN{帕提亚}}人、{\PN{米底亚}}人、{\PN{以拦}}人,和住在{\PN{美索不达米亚}}、{\PN{犹太}}、{\PN{加帕多家}}、{\PN{本都}}、{\PN{亚细亚}}、
\VS{10}{\PN{弗吕家}}、{\PN{旁非利亚}}、{\PN{埃及}}的人,并靠近{\PN{古利奈}}的{\PN{利比亚}}一带地方的人,从{\PN{罗马}}来的客旅中,或是{\PN{犹太}}人,或是进{\ADD{
{\PN{犹太}}}}教的人,
\VS{11}{\PN{克里特}}和{\PN{阿拉伯}}人,都听见他们用我们的乡谈,讲说 神的大作为。」
\VS{12}众人就都惊讶猜疑,彼此说:「这是什么意思呢?」
\VS{13}还有人讥诮说:「他们无非是新酒灌满了。」
\par }{\SH 彼得在五旬节的讲论
\par }{\PP \VS{14}{\PN{彼得}}和十一个使徒站起,高声说:「{\PN{犹太}}人和一切住在{\PN{耶路撒冷}}的人哪,这件事你们当知道,也当侧耳听我的话。
\VS{15}你们想这些人是醉了;其实不是醉了,因为时候刚到巳初。
\VS{16}这正是先知{\PN{约珥}}所说的:
\par }{\Q \VS{17}神说:在末后的日子,
\par }{\Q 我要将我的灵浇灌凡有血气的。
\par }{\Q 你们的儿女要说预言;
\par }{\Q 你们的少年人要见异象;
\par }{\Q 老年人要做异梦。
\par }{\Q \VS{18}在那些日子,
\par }{\Q 我要将我的灵浇灌我的仆人和使女,
\par }{\Q 他们就要说预言。
\par }{\Q \VS{19}在天上,我要显出奇事;
\par }{\Q 在地下,我要显出神迹;
\par }{\Q 有血,有火,有烟雾。
\par }{\Q \VS{20}日头要变为黑暗,
\par }{\Q 月亮要变为血;
\par }{\Q 这都在主大而明显的日子未到以前。
\par }{\Q \VS{21}到那时候,
\par }{\Q 凡求告主名的,就必得救。
\par }{\PP \VS{22}「{\PN{以色列}}人哪,请听我的话: 神借着{\PN{拿撒勒}}人耶稣在你们中间施行异能、奇事、神迹,将他证明出来,这是你们自己知道的。
\VS{23}他既按着 神的定旨先见被交与人,你们就借着无法之人的手,把他钉在十字架上,杀了。
\VS{24}神却将死的痛苦解释了,叫他复活,因为他原不能被死拘禁。
\VS{25}{\PN{大卫}}指着他说:
\par }{\Q 我看见主常在我眼前;
\par }{\Q 他在我右边,叫我不至于摇动。
\par }{\PP \VS{26}所以,我心里欢喜,
\par }{\Q 我的灵\FTNT{}{{\FR 2:26: }原文是舌}快乐;
\par }{\Q 并且我的肉身要安居在指望中。
\par }{\Q \VS{27}因你必不将我的灵魂撇在阴间,
\par }{\Q 也不叫你的圣者见朽坏。
\par }{\Q \VS{28}你已将生命的道路指示我,
\par }{\Q 必叫我因见你的面\FTNT{}{{\FR 2:28: }或译:叫我在你面前}
\par }{\Q 得着满足的快乐。
\par }{\PP \VS{29}「弟兄们!先祖{\PN{大卫}}的事,我可以明明地对你们说:他死了,也葬埋了,并且他的坟墓直到今日还在我们这里。
\VS{30}{\PN{大卫}}既是先知,又晓得 神曾向他起誓,要从他的后裔中立一位坐在他的宝座上,
\VS{31}就预先看明这事,讲论基督复活说:
\par }{\Q 他的灵魂不撇在阴间;
\par }{\Q 他的肉身也不见朽坏。
\par }{\MM \VS{32}这耶稣, 神已经叫他复活了,我们都为这事作见证。
\VS{33}他既被 神的右手高举\FTNT{}{{\FR 2:33: }或译:他既高举在 神的右边},又从父受了所应许的圣灵,就把你们所看见所听见的,浇灌下来。
\VS{34}{\PN{大卫}}并没有升到天上,但自己说:
\par }{\Q 主对我主说:
\par }{\Q 你坐在我的右边,
\par }{\PP \VS{35}等我使你仇敌作你的脚凳。
\par }{\PP \VS{36}「故此,{\PN{以色列}}全家当确实地知道,你们钉在十字架上的这位耶稣, 神已经立他为主,为基督了。」
\par }{\PP \VS{37}众人听见这话,觉得扎心,就对{\PN{彼得}}和其余的使徒说:「弟兄们,我们当怎样行?」
\VS{38}{\PN{彼得}}说:「你们各人要悔改,奉耶稣基督的名受洗,叫你们的罪得赦,就必领受所赐的圣灵;
\VS{39}因为这应许是给你们和你们的儿女,并一切在远方的人,就是主—我们 神所召来的。」
\VS{40}{\PN{彼得}}还用许多话作见证,劝勉他们说:「你们当救自己脱离这弯曲的世代。」
\VS{41}于是领受他话的人就受了洗。那一天,{\ADD{门徒}}约添了三千人,
\VS{42}都恒心遵守使徒的教训,彼此交接,擘饼,祈祷。
\par }{\SH 在圣徒中间的生活
\par }{\PP \VS{43}众人都惧怕;使徒又行了许多奇事神迹。
\VS{44}信的人都在一处,凡物公用,
\VS{45}并且卖了田产、家业,照各人所需用的分给各人。
\VS{46}他们天天同心合意恒切地在殿里,且在家中擘饼,存着欢喜、诚实的心用饭,
\VS{47}赞美 神,得众民的喜爱。主将得救的人天天加给他们。

\par }\Chap{3}{\SH 圣殿门口的瘸腿者得医治
\par }{\PP \VerseOne{1}申初祷告的时候,{\PN{彼得}}、{\PN{约翰}}上{\ADD{圣}}殿去。
\VS{2}有一个人,生来是瘸腿的,天天被人抬来,放在殿的一个门口(那门名叫美{\ADD{门}}),要求进殿的人周济。
\VS{3}他看见{\PN{彼得}}、{\PN{约翰}}将要进殿,就求他们周济。
\VS{4}{\PN{彼得}}、{\PN{约翰}}定睛看他;{\PN{彼得}}说:「你看我们!」
\VS{5}那人就留意看他们,指望得着什么。
\VS{6}{\PN{彼得}}说:「金银我都没有,只把我所有的给你:我奉{\PN{拿撒勒}}人耶稣基督的名,叫你起来行走!」
\VS{7}于是拉着他的右手,扶他起来;他的脚和踝子骨立刻健壮了,
\VS{8}就跳起来,站着,又行走,同他们进了殿,走着,跳着,赞美 神。
\VS{9}百姓都看见他行走,赞美 神;
\VS{10}认得他是那素常坐在殿的美门口求周济的,就因他所遇着的事满心希奇、惊讶。
\par }{\SH 彼得在所罗门廊下的讲论
\par }{\PP \VS{11}那人正在称为{\PN{所罗门}}的廊下,拉着{\PN{彼得}}、{\PN{约翰}};众百姓一齐跑到他们那里,很觉希奇。
\VS{12}{\PN{彼得}}看见,就对百姓说:「{\PN{以色列}}人哪,为什么把这事当作希奇呢?为什么定睛看我们,以为我们凭自己的能力和虔诚使这人行走呢?
\VS{13}{\PN{亚伯拉罕}}、{\PN{以撒}}、{\PN{雅各}}的 神,就是我们列祖的 神,已经荣耀了他的仆人\FTNT{}{{\FR 3:13: }或译:儿子}耶稣;你们却把他交付{\PN{彼拉多}}。{\PN{彼拉多}}定意要释放他,你们竟在{\PN{彼拉多}}面前弃绝了他。
\VS{14}你们弃绝了那圣洁公义者,反求着释放一个凶手给你们。
\VS{15}你们杀了那生命的主, 神却叫他从死里复活了;我们都是为这事作见证。
\VS{16}我们因信他的名,他的名便叫你们所看见所认识的这人健壮了;正是他所赐的信心,叫这人在你们众人面前全然好了。
\VS{17}弟兄们,我晓得你们做这事是出于不知,你们的官长也是如此。
\VS{18}但 神曾借众先知的口,预言基督将要受害,就这样应验了。
\VS{19}所以,你们当悔改归正,使你们的罪得以涂抹,
\VS{20}这样,那安舒的日子就必从主面前来到;主也必差遣所预定给你们的基督(耶稣)降临。
\VS{21}天必留他,等到万物复兴的时候,就是 神从创世以来、借着圣先知的口所说的。
\VS{22}{\PN{摩西}}曾说:『主— 神要从你们弟兄中间给你们兴起一位先知像我,凡他向你们所说的,你们都要听从。
\VS{23}凡不听从那先知的,必要从民中全然灭绝。』
\VS{24}从{\PN{撒母耳}}以来的众先知,凡说预言的,也都说到这些日子。
\VS{25}你们是先知的子孙,也承受 神与你们祖宗所立的约,就是对{\PN{亚伯拉罕}}说:『地上万族都要因你的后裔得福。』
\VS{26}神既兴起他的仆人\FTNT{}{{\FR 3:26: }或译:儿子},就先差他到你们这里来,赐福给你们,叫你们各人回转,离开罪恶。」

\par }\Chap{4}{\SH 彼得、约翰在公会被查问
\par }{\PP \VerseOne{1}使徒对百姓说话的时候,祭司们和守殿官,并撒都该人忽然来了。
\VS{2}因他们教训百姓,本着耶稣,传说死人复活,就很烦恼,
\VS{3}于是下手拿住他们;因为天已经晚了,就把他们押到第二天。
\VS{4}但听道之人有许多信的,男丁数目约到五千。
\par }{\PP \VS{5}第二天,官府、长老,和文士在{\PN{耶路撒冷}}聚会,
\VS{6}又有大祭司{\PN{亚那}}和{\PN{该亚法}}、{\PN{约翰}}、{\PN{亚历山大}},并大祭司的亲族都在那里,
\VS{7}叫使徒站在当中,就问他们说:「你们用什么能力,奉谁的名做这事呢?」
\VS{8}那时{\PN{彼得}}被圣灵充满,对他们说:
\VS{9}「治民的官府和长老啊,倘若今日因为在残疾人身上所行的善事查问我们他是怎么得了痊愈,
\VS{10}你们众人和{\PN{以色列}}百姓都当知道,站在你们面前的这人得痊愈是因你们所钉十字架、 神叫他从死里复活的{\PN{拿撒勒}}人耶稣基督的名。
\VS{11}他是
\par }{\Q 你们匠人所弃的石头,
\par }{\Q 已成了房角的头块石头。
\par }{\MM \VS{12}除他以外,别无拯救;因为在天下人间,没有赐下别的名,我们可以靠着得救。」
\VS{13}他们见{\PN{彼得}}、{\PN{约翰}}的胆量,又看出他们原是没有学问的小民,就希奇,认明他们是跟过耶稣的;
\VS{14}又看见那治好了的人和他们一同站着,就无话可驳。
\VS{15}于是吩咐他们从公会出去,就彼此商议说:
\VS{16}「我们当怎样办这两个人呢?因为他们诚然行了一件明显的神迹,凡住{\PN{耶路撒冷}}的人都知道,我们也不能说没有。
\VS{17}惟恐这事越发传扬在民间,我们必须恐吓他们,叫他们不再奉这名对人讲论。」
\VS{18}于是叫了他们来,禁止他们总不可奉耶稣的名讲论教训人。
\VS{19}{\PN{彼得}}、{\PN{约翰}}说:「听从你们,不听从 神,这在 神面前合理不合理,你们自己酌量吧!
\VS{20}我们所看见所听见的,不能不说。」
\VS{21}官长为百姓的缘故,想不出法子刑罚他们,又恐吓一番,把他们释放了。这是因众人为所行的奇事都归荣耀与 神。
\VS{22}原来借着神迹医好的那人有四十多岁了。
\par }{\SH 门徒求主赐胆量
\par }{\PP \VS{23}二人既被释放,就到会友那里去,把祭司长和长老所说的话都告诉他们。
\VS{24}他们听见了,就同心合意地高声向 神说:「主啊!你是造天、地、海,和其中万物的,
\VS{25}你曾借着圣灵,{\ADD{托}}你仆人—我们祖宗{\PN{大卫}}的口,说:
\par }{\Q 外邦为什么争闹?
\par }{\Q 万民为什么谋算虚妄的事?
\par }{\Q \VS{26}世上的君王一齐起来,
\par }{\Q 臣宰也聚集,
\par }{\Q 要敌挡主,
\par }{\Q 并主的受膏者\FTNT{}{{\FR 4:26: }或译:基督}。
\par }{\MM \VS{27}{\PN{希律}}和{\PN{本丢·彼拉多}},外邦人和{\PN{以色列}}民,果然在这城里聚集,要攻打你所膏的圣仆\FTNT{}{{\FR 4:27: }仆:或译子}耶稣,
\VS{28}成就你手和你意旨所预定必有的事。
\VS{29-30}他们恐吓我们,现在求主鉴察,一面叫你仆人大放胆量讲你的道,一面伸出你的手来医治疾病,并且使神迹奇事因着你圣仆\FTNT{}{{\FR 4:29-30: }仆:或译子}耶稣的名行出来。」
\VS{31}祷告完了,聚会的地方震动,他们就都被圣灵充满,放胆讲论 神的道。
\par }{\SH 凡物公用
\par }{\PP \VS{32}那许多信的人都是一心一意的,没有一人说他的东西有一样是自己的,都是大家公用。
\VS{33}使徒大有能力,见证主耶稣复活;众人也都蒙大恩。
\VS{34}内中也没有一个缺乏的;因为人人将田产房屋都卖了,把所卖的价银拿来,
\VS{35}放在使徒脚前,照各人所需用的,分给各人。
\VS{36}有一个{\PN{利未}}人,生在{\PN{塞浦路斯}},名叫{\PN{约瑟}},使徒称他为{\PN{巴拿巴}}({\PN{巴拿巴}}翻出来就是劝慰子)。
\VS{37}他有田地,也卖了,把价银拿来,放在使徒脚前。

\par }\Chap{5}{\SH 亚拿尼亚和撒非喇
\par }{\PP \VerseOne{1}有一个人,名叫{\PN{亚拿尼亚}},同他的妻子{\PN{撒非喇}}卖了田产,
\VS{2}把价银私自留下几分,他的妻子也知道,其余的几分拿来放在使徒脚前。
\VS{3}{\PN{彼得}}说:「{\PN{亚拿尼亚}}!为什么撒但充满了你的心,叫你欺哄圣灵,把田地的价银私自留下几分呢?
\VS{4}田地还没有卖,不是你自己的吗?既卖了,价银不是你作主吗?你怎么心里起这意念呢?你不是欺哄人,是欺哄 神了。」
\VS{5}{\PN{亚拿尼亚}}听见这话,就仆倒,断了气;听见的人都甚惧怕。
\VS{6}有些少年人起来,把他包裹,抬出去埋葬了。
\par }{\PP \VS{7}约过了三小时,他的妻子进来,还不知道这事。
\VS{8}{\PN{彼得}}对她说:「你告诉我,你们卖田地的价银就是这些吗?」她说:「就是这些。」
\VS{9}{\PN{彼得}}说:「你们为什么同心试探主的灵呢?埋葬你丈夫之人的脚已到门口,他们也要把你抬出去。」
\VS{10}妇人立刻仆倒在{\PN{彼得}}脚前,断了气。那些少年人进来,见她已经死了,就抬出去,埋在她丈夫旁边。
\VS{11}全教会和听见这事的人都甚惧怕。
\par }{\SH 主借使徒行了许多神迹奇事
\par }{\PP \VS{12}主借使徒的手在民间行了许多神迹奇事;他们\FTNT{}{{\FR 5:12: }或译:信的人}都同心合意地在{\PN{所罗门}}的廊下。
\VS{13}其余的人没有一个敢贴近他们,百姓却尊重他们。
\VS{14}信而归主的人越发增添,连男带女很多。
\VS{15}甚至有人将病人抬到街上,放在床上或褥子上,指望{\PN{彼得}}过来的时候,或者得他的影儿照在什么人身上。
\VS{16}还有许多人带着病人和被污鬼缠磨的,从{\PN{耶路撒冷}}四围的城邑来,全都得了医治。
\par }{\SH 使徒受迫害
\par }{\PP \VS{17}大祭司和他的一切同人,就是撒都该教门的人,都起来,满心忌恨,
\VS{18}就下手拿住使徒,收在外监。
\VS{19}但主的使者夜间开了监门,领他们出来,
\VS{20}说:「你们去站在殿里,把这生命的道都讲给百姓听。」
\VS{21}使徒听了这话,天将亮的时候就进殿里去教训人。大祭司和他的同人来了,叫齐公会的人和{\PN{以色列}}族的众长老,就差人到监里去,要把使徒提出来。
\VS{22}但差役到了,不见他们在监里,就回来禀报说:
\VS{23}「我们看见监牢关得极妥当,看守的人也站在门外;及至开了门,里面一个人都不见。」
\VS{24}守殿官和祭司长听见这话,心里犯难,不知这事将来如何。
\VS{25}有一个人来禀报说:「你们收在监里的人,现在站在殿里教训百姓。」
\VS{26}于是守殿官和差役去带使徒来,并没有用强暴,因为怕百姓用石头打他们。
\par }{\PP \VS{27}带到了,便叫使徒站在公会前;大祭司问他们说:
\VS{28}「我们不是严严地禁止你们,不可奉这名教训人吗?你们倒把你们的道理充满了{\PN{耶路撒冷}},想要叫这人的血归到我们身上!」
\VS{29}{\PN{彼得}}和众使徒回答说:「顺从 神,不顺从人,是应当的。
\VS{30}你们挂在木头上杀害的耶稣,我们祖宗的 神已经叫他复活。
\VS{31}神且用右手将他高举\FTNT{}{{\FR 5:31: }或译:他就是 神高举在自己的右边},叫他作君王,作救主,将悔改的心和赦罪的恩赐给{\PN{以色列}}人。
\VS{32}我们为这事作见证; 神赐给顺从之人的圣灵也为这事作见证。」
\par }{\PP \VS{33}公会的人听见就极其恼怒,想要杀他们。
\VS{34}但有一个法利赛人,名叫{\PN{迦玛列}},是众百姓所敬重的教法师,在公会中站起来,吩咐人把使徒暂且带到外面去,
\VS{35}就对众人说:「{\PN{以色列}}人哪,论到这些人,你们应当小心怎样办理。
\VS{36}从前{\PN{杜达}}起来,自夸为大,附从他的人约有四百,他被杀后,附从他的全都散了,归于无有。
\VS{37}此后,报名上册的时候,又有{\PN{加利利}}的{\PN{犹大}}起来,引诱些百姓跟从他;他也灭亡,附从他的人也都四散了。
\VS{38}现在,我劝你们不要管这些人,任凭他们吧!他们所谋的、所行的,若是出于人,必要败坏;
\VS{39}若是出于 神,你们就不能败坏他们,恐怕你们倒是攻击 神了。」
\VS{40}公会的人听从了他,便叫使徒来,把他们打了,又吩咐他们不可奉耶稣的名讲道,就把他们释放了。
\VS{41}他们离开公会,心里欢喜,因被算是配为这名受辱。
\VS{42}他们就每日在殿里、在家里不住地教训人,传耶稣是基督。

\par }\Chap{6}{\SH 拣选七人办理供给的事
\par }{\PP \VerseOne{1}那时,门徒增多,有说希腊话的{\PN{犹太}}人向{\PN{希伯来}}人发怨言,因为在天天的供给上忽略了他们的寡妇。
\VS{2}十二使徒叫众门徒来,对他们说:「我们撇下 神的道去管理饭食,原是不合宜的。
\VS{3}所以弟兄们,当从你们中间选出七个有好名声、被{\ADD{圣}}灵充满、智慧充足的人,我们就派他们管理这事。
\VS{4}但我们要专心以祈祷、传道为事。」
\VS{5}大众都喜悦这话,就拣选了{\PN{司提反}},乃是大有信心、圣灵充满的人,又拣选{\PN{腓利}}、{\PN{伯罗哥罗}}、{\PN{尼迦挪}}、{\PN{提门}}、{\PN{巴米拿}},并进{\ADD{
{\PN{犹太}}}}教的{\PN{安提阿}}人{\PN{尼哥拉}},
\VS{6}叫他们站在使徒面前。使徒祷告了,就按手在他们头上。
\par }{\PP \VS{7}神的道兴旺起来;在{\PN{耶路撒冷}}门徒数目加增的甚多,也有许多祭司信从了这道。
\par }{\SH 司提反被捕
\par }{\PP \VS{8}{\PN{司提反}}满得恩惠、能力,在民间行了大奇事和神迹。
\VS{9}当时有称{\PN{利百地拿}}会堂的几个人,并有{\PN{古利奈}}、{\PN{亚历山大}}、{\PN{基利家}}、{\PN{亚细亚}}各处会堂的几个人,都起来和{\PN{司提反}}辩论。
\VS{10}{\PN{司提反}}是以智慧和{\ADD{圣}}灵说话,众人敌挡不住,
\VS{11}就买出人来说:「我们听见他说谤
{\PN{摩西}}和 神的话。」
\VS{12}他们又耸动了百姓、长老,并文士,就忽然来捉拿他,把他带到公会去,
\VS{13}设下假见证,说:「这个人说话,不住地糟践圣所和律法。
\VS{14}我们曾听见他说,这{\PN{拿撒勒}}人耶稣要毁坏此地,也要改变{\PN{摩西}}所交给我们的规条。」
\VS{15}在公会里坐着的人都定睛看他,见他的面貌,好像天使的面貌。

\par }\Chap{7}{\SH 司提反当众申诉
\par }{\PP \VerseOne{1}大祭司就说:「这些事果然有吗?」
\VS{2}{\PN{司提反}}说:「诸位父兄请听!当日我们的祖宗{\PN{亚伯拉罕}}在{\PN{美索不达米亚}}还未住{\PN{哈兰}}的时候,荣耀的 神向他显现,
\VS{3}对他说:『你要离开本地和亲族,往我所要指示你的地方去。』
\VS{4}他就离开{\PN{迦勒底}}人之地,住在{\PN{哈兰}}。他父亲死了以后, {\ADD{神}}使他从那里搬到你们现在所住之地。
\VS{5}在这地方, 神并没有给他产业,连立足之地也没有给他;但应许要将这地赐给他和他的后裔为业;那时他还没有儿子。
\VS{6}神说:『他的后裔必寄居外邦,那里的人要叫他们作奴仆,苦待他们四百年。』
\VS{7}神又说:『使他们作奴仆的那国,我要惩罚。以后他们要出来,在这地方事奉我。』
\VS{8}神又赐他割礼的约。于是{\ADD{
{\PN{亚伯拉罕}}}}生了{\PN{以撒}},第八日给他行了割礼。{\PN{以撒}}生{\PN{雅各}};{\PN{雅各}}生十二位先祖。
\par }{\PP \VS{9}「先祖嫉妒{\PN{约瑟}},把他卖到{\PN{埃及}}去; 神却与他同在,
\VS{10}救他脱离一切苦难,又使他在{\PN{埃及}}王法老面前得恩典,有智慧。法老就派他作{\PN{埃及}}国的宰相兼管全家。
\VS{11}后来{\PN{埃及}}和{\PN{迦南}}全地遭遇饥荒,大受艰难,我们的祖宗就绝了粮。
\VS{12}{\PN{雅各}}听见在{\PN{埃及}}有粮,就打发我们的祖宗初次往那里去。
\VS{13}第二次{\PN{约瑟}}与弟兄们相认,他的亲族也被法老知道了。
\VS{14}{\PN{约瑟}}就打发弟兄请父亲{\PN{雅各}}和全家七十五个人都来。
\VS{15}于是{\PN{雅各}}下了{\PN{埃及}},后来他和我们的祖宗都死在那里;
\VS{16}又被带到{\PN{示剑}},葬于{\PN{亚伯拉罕}}在{\PN{示剑}}用银子从{\PN{哈抹}}子孙买来的坟墓里。
\par }{\PP \VS{17}「及至 神应许{\PN{亚伯拉罕}}的日期将到,{\PN{以色列}}民在{\PN{埃及}}兴盛众多,
\VS{18}直到有不晓得{\PN{约瑟}}的新王兴起。
\VS{19}他用诡计待我们的宗族,苦害我们的祖宗,叫他们丢弃婴孩,使婴孩不能存活。
\VS{20}那时,{\PN{摩西}}生下来,俊美非凡,在他父亲家里抚养了三个月。
\VS{21}他被丢弃的时候,法老的女儿拾了去,养为自己的儿子。
\VS{22}{\PN{摩西}}学了{\PN{埃及}}人一切的学问,说话行事都有才能。
\par }{\PP \VS{23}「他将到四十岁,心中起意去看望他的弟兄{\PN{以色列}}人;
\VS{24}{\ADD{到了那里}},见他们一个人受冤屈,就护庇他,为那受欺压的人报仇,打死了那{\PN{埃及}}人。
\VS{25}他以为弟兄必明白 神是借他的手搭救他们;他们却不明白。
\VS{26}第二天,遇见两个{\ADD{以色列}}人争斗,就劝他们和睦,说:『你们二位是弟兄,为什么彼此欺负呢?』
\VS{27}那欺负邻舍的把他推开,说:『谁立你作我们的首领和审判官呢?
\VS{28}难道你要杀我,像昨天杀那{\PN{埃及}}人吗?』
\VS{29}{\PN{摩西}}听见这话就逃走了,寄居于{\PN{米甸}};在那里生了两个儿子。
\par }{\PP \VS{30}「过了四十年,在{\PN{西奈山}}的旷野,有一位天使从荆棘火焰中向{\PN{摩西}}显现。
\VS{31}{\PN{摩西}}见了那异象,便觉希奇,正进前观看的时候,有主的声音说:
\VS{32}『我是你列祖的 神,就是{\PN{亚伯拉罕}}的 神,{\PN{以撒}}的 神,{\PN{雅各}}的 神。』{\PN{摩西}}战战兢兢,不敢观看。
\VS{33}主对他说:『把你脚上的鞋脱下来,因为你所站之地是圣地。
\VS{34}我的百姓在{\PN{埃及}}所受的困苦,我实在看见了,他们悲叹的声音,我也听见了。我下来要救他们。你来!我要差你往{\PN{埃及}}去。』
\VS{35}这{\PN{摩西}}就是百姓弃绝说『谁立你作我们的首领和审判官』的; 神却借那在荆棘中显现之使者的手差派他作首领、作救赎的。
\VS{36}这人领百姓出来,在{\PN{埃及}},在{\PN{红海}},在旷野,四十年间行了奇事神迹。
\VS{37}那曾对{\PN{以色列}}人说『 神要从你们弟兄中间给你们兴起一位先知像我』的,就是这位{\PN{摩西}}。
\VS{38}这人曾在旷野会中和{\PN{西奈山}}上,与那对他说话的天使同在,又与我们的祖宗同在,并且领受活泼的圣言传给我们。
\VS{39}我们的祖宗不肯听从,反弃绝他,心里归向{\PN{埃及}},
\VS{40}对{\PN{亚伦}}说:『你且为我们造些神像,在我们前面引路;因为领我们出{\PN{埃及}}地的那个{\PN{摩西}},我们不知道他遭了什么事。』
\VS{41}那时,他们造了一个牛犊,又拿祭物献给那像,欢喜自己手中的工作。
\VS{42}神就转脸不顾,任凭他们事奉天上的日月星辰,正如先知书上所写的说:
\par }{\Q {\PN{以色列}}家啊,
\par }{\Q 你们四十年间在旷野,
\par }{\Q 岂是将牺牲和祭物献给我吗?
\par }{\Q \VS{43}你们抬着摩洛的帐幕和理番神的星,
\par }{\Q 就是你们所造为要敬拜的像。
\par }{\Q {\ADD{因此}},我要把你们迁到{\PN{巴比伦}}外去。
\par }{\PP \VS{44}「我们的祖宗在旷野,有法柜的帐幕,是 {\ADD{神}}吩咐{\PN{摩西}}叫他照所看见的样式做的。
\VS{45}这帐幕,我们的祖宗相继承受。当 神在他们面前赶出外邦人去的时候,他们同{\PN{约书亚}}把帐幕搬进承受为业之地,直存到{\PN{大卫}}的日子。
\VS{46}{\PN{大卫}}在 神面前蒙恩,祈求为{\PN{雅各}}的 神预备居所;
\VS{47}却是{\PN{所罗门}}为 神造成殿宇。
\VS{48}其实,至高者并不住人手所造的,就如先知所言:
\par }{\PP \VS{49}主说:天是我的座位,
\par }{\Q 地是我的脚凳;
\par }{\Q 你们要为我造何等的殿宇?
\par }{\Q 哪里是我安息的地方呢?
\par }{\Q \VS{50}这一切不都是我手所造的吗?
\par }{\PP \VS{51}「你们这硬着颈项、心与耳未受割礼的人,常时抗拒圣灵!你们的祖宗怎样,你们也怎样。
\VS{52}哪一个先知不是你们祖宗逼迫呢?他们也把预先传说那义者要来的人杀了;如今你们又把那义者卖了,杀了。
\VS{53}你们受了天使所传的律法,竟不遵守。」
\par }{\SH 众人用石头打死司提反
\par }{\PP \VS{54}众人听见这话就极其恼怒,向{\PN{司提反}}咬牙切齿。
\VS{55}但{\PN{司提反}}被圣灵充满,定睛望天,看见 神的荣耀,又看见耶稣站在 神的右边,
\VS{56}就说:「我看见天开了,人子站在 神的右边。」
\VS{57}众人大声喊叫,捂着耳朵,齐心拥上前去,
\VS{58}把他推到城外,用石头打他。作见证的人把衣裳放在一个少年人名叫{\PN{扫罗}}的脚前。
\VS{59}他们正用石头打的时候,{\PN{司提反}}呼吁{\ADD{主}}说:「求主耶稣接收我的灵魂!」
\VS{60}又跪下大声喊着说:「主啊,不要将这罪归于他们!」说了这话,就睡了。{\PN{扫罗}}也喜悦他被害。

\par }\Chap{8}{\SH 扫罗逼迫教会
\par }{\PP \VerseOne{1}从这日起,{\PN{耶路撒冷}}的教会大遭逼迫,除了使徒以外,门徒都分散在{\PN{犹太}}和{\PN{撒马利亚}}各处。
\VS{2}有虔诚的人把{\PN{司提反}}埋葬了,为他捶胸大哭。
\VS{3}{\PN{扫罗}}却残害教会,进各人的家,拉着男女下在监里。
\par }{\SH 福音传到撒马利亚
\par }{\PP \VS{4}那些分散的人往各处去传道。
\VS{5}{\PN{腓利}}下{\PN{撒马利亚}}城去,宣讲基督。
\VS{6}众人听见了,又看见{\PN{腓利}}所行的神迹,就同心合意地听从他的话。
\VS{7}因为有许多人被污鬼附着,那些鬼大声呼叫,从他们身上出来;还有许多瘫痪的、瘸腿的,都得了医治。
\VS{8}在那城里,就大有欢喜。
\par }{\PP \VS{9}有一个人,名叫{\PN{西门}},向来在那城里行邪术,妄自尊大,使{\PN{撒马利亚}}的百姓惊奇;
\VS{10}无论大小都听从他,说:「这人就是那称为 神的大能者。」
\VS{11}他们听从他,因他久用邪术,使他们惊奇。
\VS{12}及至他们信了{\PN{腓利}}所传 神国的福音和耶稣基督的名,连男带女就受了洗。
\VS{13}{\PN{西门}}自己也信了;既受了洗,就常与{\PN{腓利}}在一处,看见他所行的神迹和大异能,就甚惊奇。
\par }{\PP \VS{14}使徒在{\PN{耶路撒冷}}听见{\PN{撒马利亚}}人领受了 神的道,就打发{\PN{彼得}}、{\PN{约翰}}往他们那里去。
\VS{15}两个人到了,就为他们祷告,要叫他们受圣灵。
\VS{16}因为圣灵还没有降在他们一个人身上,他们只奉主耶稣的名受了洗。
\VS{17}于是使徒按手在他们{\ADD{头}}上,他们就受了圣灵。
\VS{18}{\PN{西门}}看见使徒按手,便有圣灵赐下,就拿钱给使徒,
\VS{19}说:「把这权柄也给我,叫我手按着谁,谁就可以受圣灵。」
\VS{20}{\PN{彼得}}说:「你的银子和你一同灭亡吧!因你想 神的恩赐是可以用钱买的。
\VS{21}你在这道上无分无关;因为在 神面前,你的心不正。
\VS{22}你当懊悔你这罪恶,祈求主,或者你心里的意念可得赦免。
\VS{23}我看出你正在苦胆之中,被罪恶捆绑。」
\VS{24}{\PN{西门}}说:「愿你们为我求主,叫你们所说的,没有一样临到我身上。」
\par }{\PP \VS{25}使徒既证明主道,而且传讲,就回{\PN{耶路撒冷}}去,一路在{\PN{撒马利亚}}好些村庄传扬福音。
\par }{\SH 腓利和衣索匹亚的太监
\par }{\PP \VS{26}有主的一个使者对{\PN{腓利}}说:「起来!向南走,往那从{\PN{耶路撒冷}}下{\PN{迦萨}}的路上去。」那路是旷野。
\VS{27}{\PN{腓利}}就起身去了,不料,有一个{\PN{衣索匹亚}}\FTNT{}{{\FR 8:27: }就是古实,见以赛亚十八章一节}人,是个有大权的太监,在{\PN{衣索匹亚}}女王{\PN{甘大基}}的手下总管银库,他上{\PN{耶路撒冷}}礼拜去了。
\VS{28}现在回来,在车上坐着,念先知{\PN{以赛亚}}的书。
\VS{29}圣灵对{\PN{腓利}}说:「你去!贴近那车走。」
\VS{30}{\PN{腓利}}就跑到太监那里,听见他念先知{\PN{以赛亚}}的书,便问他说:「你所念的,你明白吗?」
\VS{31}他说:「没有人指教我,怎能明白呢?」于是请{\PN{腓利}}上车,与他同坐。
\VS{32}他所念的那段经,说:
\par }{\Q 他像羊被牵到宰杀之地,
\par }{\Q 又像羊羔在剪毛的人手下无声;
\par }{\Q 他也是这样不开口。
\par }{\Q \VS{33}他卑微的时候,
\par }{\Q 人不按公义审判他\FTNT{}{{\FR 8:33: }原文是他的审判被夺去};
\par }{\Q 谁能述说他的世代?
\par }{\Q 因为他的生命从地上夺去。
\par }{\PP \VS{34}太监对{\PN{腓利}}说:「请问,先知说这话是指着谁?是指着自己呢?是指着别人呢?」
\VS{35}{\PN{腓利}}就开口从这经上起,对他传讲耶稣。
\VS{36}二人正往前走,到了有水的地方,太监说:「看哪,这里有水,我受洗有什么妨碍呢?」\FTNT{}{{\FR 8:36: }有古卷加:37腓利说:「你若是一心相信,就可以。」他回答说:「我信耶稣基督是 神的儿子。」}
\VS{38}于是吩咐车站住,{\PN{腓利}}和太监二人同下水里去,{\PN{腓利}}就给他施洗。
\VS{39}从水里上来,主的灵把{\PN{腓利}}提了去,太监也不再见他了,就欢欢喜喜地走路。
\VS{40}后来有人在{\PN{亚锁都}}遇见{\PN{腓利}};他走遍那地方,在各城宣传福音,直到{\PN{凯撒利亚}}。

\par }\Chap{9}{\SH 扫罗的转变
\par }{\R (22·6—16;26·12—18)
\par }{\PP \VerseOne{1}{\PN{扫罗}}仍然向主的门徒口吐威吓凶杀的话,去见大祭司,
\VS{2}求文书给{\PN{大马士革}}的各会堂,若是找着信奉这道的人,无论男女,都准他捆绑带到{\PN{耶路撒冷}}。
\VS{3}{\PN{扫罗}}行路,将到{\PN{大马士革}},忽然从天上发光,四面照着他;
\VS{4}他就仆倒在地,听见有声音对他说:「{\PN{扫罗}}!{\PN{扫罗}}!你为什么逼迫我?」
\VS{5}他说:「主啊!你是谁?」主说:「我就是你所逼迫的耶稣。
\VS{6}起来!进城去,你所当做的事,必有人告诉你。」
\VS{7}同行的人站在那里,说不出话来,听见声音,却看不见人。
\VS{8}{\PN{扫罗}}从地上起来,睁开眼睛,竟不能看见什么。有人拉他的手,领他进了{\PN{大马士革}};
\VS{9}三日不能看见,也不吃也不喝。
\par }{\PP \VS{10}当下,在{\PN{大马士革}}有一个门徒,名叫{\PN{亚拿尼亚}}。主在异象中对他说:「{\PN{亚拿尼亚}}。」他说:「主,我在这里。」
\VS{11}主对他说:「起来!往直街去,在{\PN{犹大}}的家里,访问一个{\PN{大数}}人,名叫{\PN{扫罗}}。他正祷告,
\VS{12}又看见了一个人,名叫{\PN{亚拿尼亚}},进来按手在他身上,叫他能看见。」
\VS{13}{\PN{亚拿尼亚}}回答说:「主啊,我听见许多人说,这人怎样在{\PN{耶路撒冷}}多多苦害你的圣徒,
\VS{14}并且他在这里有从祭司长得来的权柄捆绑一切求告你名的人。」
\VS{15}主对{\PN{亚拿尼亚}}说:「你只管去!他是我所拣选的器皿,要在外邦人和君王,并{\PN{以色列}}人面前宣扬我的名。
\VS{16}我也要指示他,为我的名必须受许多的苦难。」
\VS{17}{\PN{亚拿尼亚}}就去了,进入那家,把手按在{\PN{扫罗}}身上,说:「兄弟{\PN{扫罗}},在你来的路上向你显现的主,就是耶稣,打发我来,叫你能看见,又被圣灵充满。」
\VS{18}{\PN{扫罗}}的眼睛上,好像有鳞立刻掉下来,他就能看见。于是起来受了洗;
\VS{19}吃过饭就健壮了。
\par }{\SH 扫罗在大马士革传道
\par }{\PP {\PN{扫罗}}和{\PN{大马士革}}的门徒同住了些日子,
\VS{20}就在各会堂里宣传耶稣,说他是 神的儿子。
\VS{21}凡听见的人都惊奇,说:「在{\PN{耶路撒冷}}残害求告这名的,不是这人吗?并且他到这里来,特要捆绑他们,带到祭司长那里。」
\VS{22}但{\PN{扫罗}}越发有能力,驳倒住{\PN{大马士革}}的{\PN{犹太}}人,证明耶稣是基督。
\par }{\SH 未受犹太人的谋害
\par }{\PP \VS{23}过了好些日子,{\PN{犹太}}人商议要杀{\PN{扫罗}},
\VS{24}但他们的计谋被{\PN{扫罗}}知道了。他们又昼夜在城门守候,要杀他。
\VS{25}他的门徒就在夜间用筐子把他从城墙上缒下去。
\par }{\SH 扫罗在耶路撒冷
\par }{\PP \VS{26}{\PN{扫罗}}到了{\PN{耶路撒冷}},想与门徒结交,他们却都怕他,不信他是门徒。
\VS{27}惟有{\PN{巴拿巴}}接待他,领去见使徒,把他在路上怎么看见主,主怎么向他说话,他在{\PN{大马士革}}怎么奉耶稣的名放胆传道,都述说出来。
\VS{28}于是{\PN{扫罗}}在{\PN{耶路撒冷}}和门徒出入来往,
\VS{29}奉主的名放胆传道,并与说希腊话的{\PN{犹太}}人讲论辩驳;他们却想法子要杀他。
\VS{30}弟兄们知道了就送他下{\PN{凯撒利亚}},打发他往{\PN{大数}}去。
\par }{\PP \VS{31}那时,{\PN{犹太}}、{\PN{加利利}}、{\PN{撒马利亚}}各处的教会都得平安,被建立;凡事敬畏主,蒙圣灵的安慰,人数就增多了。
\par }{\SH 以尼雅得医治
\par }{\PP \VS{32}{\PN{彼得}}周流四方的时候,也到了居住{\PN{吕大}}的圣徒那里;
\VS{33}遇见一个人,名叫{\PN{以尼雅}},得了瘫痪,在褥子上躺卧八年。
\VS{34}{\PN{彼得}}对他说:「{\PN{以尼雅}},耶稣基督医好你了,起来!收拾你的褥子。」他就立刻起来了。
\VS{35}凡住{\PN{吕大}}和{\PN{沙
}}的人都看见了他,就归服主。
\par }{\SH 多加复活
\par }{\PP \VS{36}在{\PN{约帕}}有一个女徒,名叫{\PN{大比大}},翻{\ADD{希腊}}话就是{\PN{多加}}\FTNT{}{{\FR 9:36: }就是羚羊的意思};她广行善事,多施周济。
\VS{37}当时,她患病而死,有人把她洗了,停在楼上。
\VS{38}{\PN{吕大}}原与{\PN{约帕}}相近;门徒听见{\PN{彼得}}在那里,就打发两个人去见他,央求他说:「快到我们那里去,不要耽延。」
\VS{39}{\PN{彼得}}就起身和他们同去;到了,便有人领他上楼。众寡妇都站在{\PN{彼得}}旁边哭,拿{\PN{多加}}与她们同在时所做的里衣外衣给他看。
\VS{40}{\PN{彼得}}叫她们都出去,就跪下祷告,转身对着死人说:「{\PN{大比大}},起来!」她就睁开眼睛,见了{\PN{彼得}},便坐起来。
\VS{41}{\PN{彼得}}伸手扶她起来,叫众圣徒和寡妇进去,把{\PN{多加}}活活地交给他们。
\VS{42}这事传遍了{\PN{约帕}},就有许多人信了主。
\VS{43}此后,{\PN{彼得}}在{\PN{约帕}}一个硝皮匠{\PN{西门}}的家里住了多日。

\par }\Chap{10}{\SH 彼得和哥尼流
\par }{\PP \VerseOne{1}在{\PN{凯撒利亚}}有一个人,名叫{\PN{哥尼流}},是「{\PN{意大利}}营」的百夫长。
\VS{2}他是个虔诚人,他和全家都敬畏 神,多多周济百姓,常常祷告 神。
\VS{3}有一天,约在申初,他在异象中明明看见 神的一个使者进去,到他那里,说:「{\PN{哥尼流}}。」
\VS{4}{\PN{哥尼流}}定睛看他,惊怕说:「主啊,什么事呢?」天使说:「你的祷告和你的周济达到 神面前,已蒙记念了。
\VS{5}现在你当打发人往{\PN{约帕}}去,请那称呼{\PN{彼得}}的{\PN{西门}}来。
\VS{6}他住在海边一个硝皮匠{\PN{西门}}的家里,房子在海边上。」
\VS{7}向他说话的天使去后,{\PN{哥尼流}}叫了两个家人和常伺候他的一个虔诚兵来,
\VS{8}把这事都述说给他们听,就打发他们往{\PN{约帕}}去。
\par }{\PP \VS{9}第二天,他们行路将近那城。{\PN{彼得}}约在午正,上房顶去祷告,
\VS{10}觉得饿了,想要吃。那家的人正预备饭的时候,{\PN{彼得}}魂游象外,
\VS{11}看见天开了,有一物降下,好像一块大布,系着四角,缒在地上,
\VS{12}里面有地上各样四足的走兽和昆虫,并天上的飞鸟;
\VS{13}又有声音向他说:「{\PN{彼得}},起来,宰了吃!」
\VS{14}{\PN{彼得}}却说:「主啊,这是不可的!凡俗物和不洁净的物,我从来没有吃过。」
\VS{15}第二次有声音向他说:「 神所洁净的,你不可当作俗物。」
\VS{16}这样一连三次,那物随即收回天上去了。
\par }{\PP \VS{17}{\PN{彼得}}心里正在猜疑之间,不知所看见的异象是什么意思。{\PN{哥尼流}}所差来的人已经访问到{\PN{西门}}的家,站在门外,
\VS{18}喊着问:「有称呼{\PN{彼得}}的{\PN{西门}}住在这里没有?」
\VS{19}{\PN{彼得}}还思想那异象的时候,{\ADD{圣}}灵向他说:「有三个人来找你。
\VS{20}起来,下去,和他们同往,不要疑惑,因为是我差他们来的。」
\VS{21}于是{\PN{彼得}}下去见那些人,说:「我就是你们所找的人。你们来是为什么缘故?」
\VS{22}他们说:「百夫长{\PN{哥尼流}}是个义人,敬畏 神,为{\PN{犹太}}通国所称赞。他蒙一位圣天使所指示,叫他请你到他家里去,听你的话。」
\VS{23}{\PN{彼得}}就请他们进去,住了一宿。
\par }{\PP 次日,起身和他们同去,还有{\PN{约帕}}的几个弟兄同着他去;
\VS{24}又次日,他们进入{\PN{凯撒利亚}},{\PN{哥尼流}}已经请了他的亲属密友等候他们。
\VS{25}{\PN{彼得}}一进去,{\PN{哥尼流}}就迎接他,俯伏在他脚前拜他。
\VS{26}{\PN{彼得}}却拉他,说:「你起来,我也是人。」
\VS{27}{\PN{彼得}}和他说着话进去,见有好些人在那里聚集,
\VS{28}就对他们说:「你们知道,{\PN{犹太}}人和别国的人亲近来往本是不合例的,但 神已经指示我,无论什么人都不可看作俗而不洁净的。
\VS{29}所以我被请的时候,就不推辞而来。现在请问:你们叫我来有什么意思呢?」
\VS{30}{\PN{哥尼流}}说:「前四天,这个时候,我在家中守着申初的祷告,忽然有一个人穿着光明的衣裳,站在我面前,
\VS{31}说:『{\PN{哥尼流}},你的祷告已蒙垂听,你的周济达到 神面前已蒙记念了。
\VS{32}你当打发人往{\PN{约帕}}去,请那称呼{\PN{彼得}}的{\PN{西门}}来,他住在海边一个硝皮匠{\PN{西门}}的家里。』
\VS{33}所以我立时打发人去请你。你来了很好;现今我们都在 神面前,要听主所吩咐你的一切话。」
\par }{\SH 在哥尼流家中讲道
\par }{\PP \VS{34}{\PN{彼得}}就开口说:「我真看出 神是不偏待人。
\VS{35}原来,各国中那敬畏主、行义的人都为主所悦纳。
\VS{36}神借着耶稣基督(他是万有的主)传和平的福音,将这道赐给{\PN{以色列}}人。
\VS{37}这话在{\PN{约翰}}宣传洗礼以后,从{\PN{加利利}}起,传遍了{\PN{犹太}}。
\VS{38}神怎样以圣灵和能力膏{\PN{拿撒勒}}人耶稣,这都是你们知道的。他周流四方,行善事,医好凡被魔鬼压制的人,因为 神与他同在。
\VS{39}他在{\PN{犹太}}人之地,并{\PN{耶路撒冷}}所行的一切事,有我们作见证。他们竟把他挂在木头上杀了。
\VS{40}第三日, 神叫他复活,显现出来;
\VS{41}不是显现给众人看,乃是显现给 神预先所拣选为他作见证的人看,就是我们这些在他从死里复活以后和他同吃同喝的人。
\VS{42}他吩咐我们传道给众人,证明他是 神所立定的,要作审判活人、死人的主。
\VS{43}众先知也为他作见证说:『凡信他的人必因他的名得蒙赦罪。』」
\par }{\SH 外邦人领受圣灵
\par }{\PP \VS{44}{\PN{彼得}}还说这话的时候,圣灵降在一切听道的人身上。
\VS{45}那些奉割礼、和{\PN{彼得}}同来的信徒,见圣灵的恩赐也浇在外邦人身上,就都希奇;
\VS{46}因听见他们说方言,称赞 神为大。
\VS{47}于是{\PN{彼得}}说:「这些人既受了圣灵,与我们一样,谁能禁止用水给他们施洗呢?」
\VS{48}就吩咐奉耶稣基督的名给他们施洗。他们又请{\PN{彼得}}住了几天。

\par }\Chap{11}{\SH 向耶路撒冷教会报告
\par }{\PP \VerseOne{1}使徒和在{\PN{犹太}}的众弟兄听说外邦人也领受了 神的道。
\VS{2}及至{\PN{彼得}}上了{\PN{耶路撒冷}},那些奉割礼的门徒和他争辩说:
\VS{3}「你进入未受割礼之人的家和他们一同吃饭了。」
\VS{4}{\PN{彼得}}就开口把这事挨次给他们讲解说:
\VS{5}「我在{\PN{约帕}}城里祷告的时候,魂游象外,看见异象,有一物降下,好像一块大布,系着四角,从天缒下,直来到我跟前。
\VS{6}我定睛观看,见内中有地上四足的牲畜和野兽、昆虫,并天上的飞鸟。
\VS{7}我且听见有声音向我说:『{\PN{彼得}},起来,宰了吃!』
\VS{8}我说:『主啊,这是不可的!凡俗而不洁净的物从来没有入过我的口。』
\VS{9}第二次,有声音从天上说:『 神所洁净的,你不可当作俗物。』
\VS{10}这样一连三次,就都收回天上去了。
\VS{11}正当那时,有三个人站在我们所住的房门前,是从{\PN{凯撒利亚}}差来见我的。
\VS{12}圣灵吩咐我和他们同去,不要疑惑\FTNT{}{{\FR 11:12: }或译:不要分别等类}。同着我去的,还有这六位弟兄,我们都进了那人的家。
\VS{13}那人就告诉我们,他如何看见一位天使站在他屋里,说:『你打发人往{\PN{约帕}}去,请那称呼{\PN{彼得}}的{\PN{西门}}来,
\VS{14}他有话告诉你,可以叫你和你的全家得救。』
\VS{15}我一开讲,圣灵便降在他们身上,正像当初降在我们身上一样。
\VS{16}我就想起主的话说:『{\PN{约翰}}是用水施洗,但你们要受圣灵的洗。』
\VS{17}神既然给他们恩赐,像在我们信主耶稣基督的时候给了我们一样;我是谁,能拦阻 神呢?」
\VS{18}众人听见这话,就不言语了,只归荣耀与 神,说:「这样看来, 神也赐恩给外邦人,叫他们悔改得生命了。」
\par }{\SH 安提阿的教会
\par }{\PP \VS{19}那些因{\PN{司提反}}的事遭患难四散的门徒直走到{\PN{腓尼基}}和{\PN{塞浦路斯}},并{\PN{安提阿}};他们不向别人讲道,只向{\PN{犹太}}人讲。
\VS{20}但内中有{\PN{塞浦路斯}}和{\PN{古利奈}}人,他们到了{\PN{安提阿}}也向{\PN{希腊}}人传讲主耶稣\FTNT{}{{\FR 11:20: }有古卷:也向说希腊话的犹太人传讲主耶稣}。
\VS{21}主与他们同在,信而归主的人就很多了。
\VS{22}这风声传到{\PN{耶路撒冷}}教会人的耳中,他们就打发{\PN{巴拿巴}}出去,走到{\PN{安提阿}}为止。
\VS{23}他到了那里,看见 神所赐的恩就欢喜,劝勉众人,立定心志,恒久靠主。
\VS{24}这{\PN{巴拿巴}}原是个好人,被圣灵充满,大有信心。于是有许多人归服了主。
\VS{25}他又往{\PN{大数}}去找{\PN{扫罗}},
\VS{26}找着了,就带他到{\PN{安提阿}}去。他们足有一年的工夫和教会一同聚集,教训了许多人。门徒称为「基督徒」是从{\PN{安提阿}}起首。
\par }{\PP \VS{27}当那些日子,有几位先知从{\PN{耶路撒冷}}下到{\PN{安提阿}}。
\VS{28}内中有一位,名叫{\PN{亚迦布}},站起来,借着{\ADD{圣}}灵指明天下将有大饥荒。(这事到{\PN{克劳第}}年间果然有了。)
\VS{29}于是门徒定意照各人的力量捐钱,送去供给住在{\PN{犹太}}的弟兄。
\VS{30}他们就这样行,把捐项托{\PN{巴拿巴}}和{\PN{扫罗}}送到众长老那里。

\par }\Chap{12}{\SH 雅各被杀,彼得被囚
\par }{\PP \VerseOne{1}那时,{\PN{希律}}王下手苦害教会中几个人,
\VS{2}用刀杀了{\PN{约翰}}的哥哥{\PN{雅各}}。
\VS{3}他见{\PN{犹太}}人喜欢这事,又去捉拿{\PN{彼得}}。那时正是除酵的日子。
\VS{4}{\PN{希律}}拿了{\PN{彼得}},收在监里,交付四班兵丁看守,每班四个人,意思要在逾越节后把他提出来,当着百姓{\ADD{办他}}。
\VS{5}于是{\PN{彼得}}被囚在监里;教会却为他切切地祷告 神。
\par }{\SH 彼得获救出狱
\par }{\PP \VS{6}{\PN{希律}}将要提他出来的前一夜,{\PN{彼得}}被两条铁链锁着,睡在两个兵丁当中;看守的人也在门外看守。
\VS{7}忽然,有主的一个使者站在旁边,屋里有光照耀,天使拍{\PN{彼得}}的肋旁,拍醒了他,说:「快快起来!」那铁链就从他手上脱落下来。
\VS{8}天使对他说:「束上带子,穿上鞋。」他就那样做。天使又说:「披上外衣,跟着我来。」
\VS{9}{\PN{彼得}}就出来跟着他,不知道天使所做是真的,只当见了异象。
\VS{10}过了第一层第二层监牢,就来到临街的铁门,那门自己开了。他们出来,走过一条街,天使便离开他去了。
\VS{11}{\PN{彼得}}醒悟过来,说:「我现在真知道主差遣他的使者,救我脱离{\PN{希律}}的手和{\PN{犹太}}百姓一切所盼望的。」
\VS{12}想了一想,就往那称呼{\PN{马可}}的{\PN{约翰}}、他母亲{\PN{马利亚}}家去,在那里有好些人聚集祷告。
\VS{13}{\PN{彼得}}敲外门,有一个使女,名叫{\PN{罗大}},出来探听,
\VS{14}听得是{\PN{彼得}}的声音,就欢喜的顾不得开门,跑进去告诉众人说:「{\PN{彼得}}站在门外。」
\VS{15}他们说:「你是疯了!」使女极力地说:「真是他!」他们说:「必是他的天使!」
\VS{16}{\PN{彼得}}不住地敲门。他们开了门,看见他,就甚惊奇。
\VS{17}{\PN{彼得}}摆手,不要他们作声,就告诉他们主怎样领他出监;又说:「你们把这事告诉{\PN{雅各}}和众弟兄。」于是出去,往别处去了。
\VS{18}到了天亮,兵丁扰乱得很,不知道{\PN{彼得}}往哪里去了。
\VS{19}{\PN{希律}}找他,找不着,就审问看守的人,吩咐把他们拉去杀了。后来{\PN{希律}}离开{\PN{犹太}},下{\PN{凯撒利亚}}去,住在那里。
\par }{\SH 希律的死
\par }{\PP \VS{20}{\PN{希律}}恼怒{\PN{泰尔}}、{\PN{西顿}}的人。他们那一带地方是从王的地土得粮,因此就托了王的内侍臣{\PN{伯拉斯都}}的情,一心来求和。
\VS{21}{\PN{希律}}在所定的日子,穿上朝服,坐在位上,对他们讲论一番。
\VS{22}百姓喊着说:「这是神的声音,不是人的声音。」
\VS{23}{\PN{希律}}不归荣耀给 神,所以主的使者立刻罚他,他被虫所咬,气就绝了。
\par }{\PP \VS{24}神的道日见兴旺,越发广传。
\VS{25}{\PN{巴拿巴}}和{\PN{扫罗}}办完了他们供给的事,就从{\PN{耶路撒冷}}回来,带着称呼{\PN{马可}}的{\PN{约翰}}同去。

\par }\Chap{13}{\SH 巴拿巴和扫罗奉差遣
\par }{\PP \VerseOne{1}在{\PN{安提阿}}的教会中,有几位先知和教师,就是{\PN{巴拿巴}}和称呼{\PN{尼结}}的{\PN{西面}}、{\PN{古利奈}}人{\PN{路求}},与分封之王{\PN{希律}}同养的{\PN{马念}},并{\PN{扫罗}}。
\VS{2}他们事奉主、禁食的时候,圣灵说:「要为我分派{\PN{巴拿巴}}和{\PN{扫罗}},去做我召他们所做的工。」
\VS{3}于是禁食祷告,按手在他们{\ADD{头}}上,就打发他们去了。
\par }{\SH 在塞浦路斯传道
\par }{\PP \VS{4}他们既被圣灵差遣,就下到{\PN{西流基}},从那里坐船往{\PN{塞浦路斯}}去。
\VS{5}到了{\PN{撒拉米}},就在{\PN{犹太}}人各会堂里传讲 神的道,也有{\PN{约翰}}作他们的帮手。
\VS{6}经过全岛,直到{\PN{帕弗}},在那里遇见一个有法术、假充先知的{\PN{犹太}}人,名叫{\PN{巴·耶稣}}。
\VS{7}这人常和方伯{\PN{士求·保罗}}同在。{\PN{士求·保罗}}是个通达人,他请了{\PN{巴拿巴}}和{\PN{扫罗}}来,要听 神的道。
\VS{8}只是那行法术的{\PN{以吕马}}(这名翻出来就是行法术的意思)敌挡使徒,要叫方伯不信真道。
\VS{9}{\PN{扫罗}}又名{\PN{保罗}},被圣灵充满,定睛看他,
\VS{10}说:「你这充满各样诡诈奸恶,魔鬼的儿子,众善的仇敌,你混乱主的正道还不止住吗?
\VS{11}现在主的手加在你身上,你要瞎眼,暂且不见日光。」他的眼睛立刻昏蒙黑暗,四下里求人拉着手领他。
\VS{12}方伯看见所做的事,很希奇主的道,就信了。
\par }{\SH 在彼西底的安提阿传道
\par }{\PP \VS{13}{\PN{保罗}}和他的同人从{\PN{帕弗}}开船,来到{\PN{旁非利亚}}的{\PN{别加}},{\PN{约翰}}就离开他们,回{\PN{耶路撒冷}}去。
\VS{14}他们离了{\PN{别加}}往前行,来到{\PN{彼西底}}的{\PN{安提阿}},在安息日进会堂坐下。
\VS{15}读完了律法和先知{\ADD{的书}},管会堂的叫人过去,对他们说:「二位兄台,若有什么劝勉众人的话,请说。」
\par }{\PP \VS{16}{\PN{保罗}}就站起来,举手,说:「{\PN{以色列}}人和一切敬畏 神的人,请听。
\VS{17}这{\PN{以色列}}民的 神拣选了我们的祖宗,当民寄居{\PN{埃及}}的时候抬举他们,用大能的手领他们出来;
\VS{18}又在旷野容忍\FTNT{}{{\FR 13:18: }或译:抚养}他们,约有四十年。
\VS{19}既灭了{\PN{迦南}}地七族的人,就把那地分给他们为业;
\VS{20}此后给他们设立士师,约有四百五十年,直到先知{\PN{撒母耳}}的时候。
\VS{21}后来他们求一个王, 神就将{\PN{便雅悯}}支派中{\PN{基士}}的儿子{\PN{扫罗}},给他们作王四十年。
\VS{22}既废了{\PN{扫罗}},就选立{\PN{大卫}}作他们的王,又为他作见证说:『我寻得{\PN{耶西}}的儿子{\PN{大卫}},他是合我心意的人,凡事要遵行我的旨意。』
\VS{23}从这人的后裔中, 神已经照着所应许的,为{\PN{以色列}}人立了一位救主,就是耶稣。
\VS{24}在他没有出来以先,{\PN{约翰}}向{\PN{以色列}}众民宣讲悔改的洗礼。
\VS{25}{\PN{约翰}}将行尽他的程途说:『你们以为我是谁?我不是{\ADD{基督}};只是有一位在我以后来的,我解他脚上的鞋带也是不配的。』
\par }{\PP \VS{26}「弟兄们,{\PN{亚伯拉罕}}的子孙和你们中间敬畏 神的人哪,这救世的道是传给我们的。
\VS{27}{\PN{耶路撒冷}}居住的人和他们的官长,因为不认识基督,也不明白每安息日所读众先知的书,就把基督定了死罪,正应了先知的预言;
\VS{28}虽然查不出他有当死的罪来,还是求{\PN{彼拉多}}杀他;
\VS{29}既成就了{\ADD{经上}}指着他所记的一切话,就把他从木头上取下来,放在坟墓里。
\VS{30}神却叫他从死里复活。
\VS{31}那从{\PN{加利利}}同他上{\PN{耶路撒冷}}的人多日看见他,这些人如今在民间是他的见证。
\VS{32}我们也报好信息给你们,就是那应许祖宗的话,
\VS{33}神已经向我们这作儿女的应验,叫耶稣复活了。正如诗篇第二篇上记着说:
\par }{\Q 你是我的儿子,
\par }{\Q 我今日生你。
\par }{\MM \VS{34}论到 神叫他从死里复活,不再归于朽坏,就这样说:
\par }{\Q 我必将{\ADD{所应许}}{\PN{大卫}}那圣洁、
\par }{\Q 可靠的{\ADD{恩典}}赐给你们。
\par }{\Q \VS{35}又有一篇上说:
\par }{\Q 你必不叫你的圣者见朽坏。
\par }{\PP \VS{36}「{\PN{大卫}}在世的时候遵行了 神的旨意,就睡了\FTNT{}{{\FR 13:36: }或译:大卫按 神的旨意服事了他那一世的人,就睡了},归到他祖宗那里,已见朽坏;
\VS{37}惟独 神所复活的,他并未见朽坏。
\VS{38}所以,弟兄们,你们当晓得:赦罪的道是由这人传给你们的。
\VS{39}你们靠{\PN{摩西}}的律法,在一切不得称义的事上信靠这人,就都得称义了。
\VS{40}所以,你们务要小心,免得先知书上所说的临到你们。
\VS{41}主{\ADD{说}}:
\par }{\Q 你们这轻慢的人要观看,要惊奇,要灭亡;
\par }{\Q 因为在你们的时候,我行一件事,
\par }{\Q 虽有人告诉你们,
\par }{\Q 你们总是不信。」
\par }{\PP \VS{42}他们出会堂的时候,众人请他们到下安息日再讲这话给他们听。
\VS{43}散会以后,{\PN{犹太}}人和敬虔进{\PN{犹太}}教的人多有跟从{\PN{保罗}}、{\PN{巴拿巴}}的。二人对他们讲道,劝他们务要恒久在 神的恩中。
\par }{\PP \VS{44}到下安息日,合城的人几乎都来聚集,要听 神的道。
\VS{45}但{\PN{犹太}}人看见人这样多,就满心嫉妒,硬驳{\PN{保罗}}所说的话,并且毁谤。
\VS{46}{\PN{保罗}}和{\PN{巴拿巴}}放胆说:「 神的道先讲给你们原是应当的;只因你们弃绝这道,断定自己不配得永生,我们就转向外邦人去。
\VS{47}因为主曾这样吩咐我们说:
\par }{\Q 我已经立你作外邦人的光,
\par }{\Q 叫你施行救恩,直到地极。」
\par }{\PP \VS{48}外邦人听见这话,就欢喜了,赞美 神的道;凡预定得永生的人都信了。
\VS{49}于是主的道传遍了那一带地方。
\VS{50}但{\PN{犹太}}人挑唆虔敬、尊贵的妇女和城内有名望的人,逼迫{\PN{保罗}}、{\PN{巴拿巴}},将他们赶出境外。
\VS{51}二人对着众人跺下脚上的尘土,就往{\PN{以哥念}}去了。
\VS{52}门徒满心喜乐,又被圣灵充满。

\par }\Chap{14}{\SH 在以哥念传道
\par }{\PP \VerseOne{1}二人在{\PN{以哥念}}同进{\PN{犹太}}人的会堂,在那里讲的,叫{\PN{犹太}}人和{\PN{希腊}}人信的很多。
\VS{2}但那不顺从的{\PN{犹太}}人耸动外邦人,叫他们心里恼恨弟兄。
\VS{3}二人在那里住了多日,倚靠主放胆讲道,主借他们的手施行神迹奇事,证明他的恩道。
\VS{4}城里的众人就分了党,有附从{\PN{犹太}}人的,有附从使徒的。
\VS{5}那时,外邦人和{\PN{犹太}}人,并他们的官长,一齐拥上来,要凌辱使徒,用石头打他们。
\VS{6}使徒知道了,就逃往{\PN{吕高尼}}的{\PN{路司得}}、{\PN{特庇}}两个城和周围地方去,
\VS{7}在那里传福音。
\par }{\SH 在路司得传道
\par }{\PP \VS{8}{\PN{路司得}}城里坐着一个两脚无力的人,生来是瘸腿的,从来没有走过。
\VS{9}他听{\PN{保罗}}讲道,{\PN{保罗}}定睛看他,见他有信心,可得痊愈,
\VS{10}就大声说:「你起来,两脚站直!」那人就跳起来,而且行走。
\VS{11}众人看见{\PN{保罗}}所做的事,就用{\PN{吕高尼}}的话大声说:「有神借着人形降临在我们中间了。」
\VS{12}于是称{\PN{巴拿巴}}为{\PN{宙斯}},称{\PN{保罗}}为{\PN{希耳米}},因为他说话领首。
\VS{13}有城外{\PN{宙斯}}{\ADD{庙}}的祭司牵着牛,拿着花圈,来到门前,要同众人{\ADD{向使徒}}献祭。
\VS{14}{\PN{巴拿巴}}、{\PN{保罗}}二使徒听见,就撕开衣裳,跳进众人中间,喊着说:
\VS{15}「诸君,为什么做这事呢?我们也是人,性情和你们一样。我们传福音给你们,是叫你们离弃这些虚妄,归向那创造天、地、海,和其中万物的永生 神。
\VS{16}他在从前的世代,任凭万国各行其道;
\VS{17}然而为自己未尝不显出证据来,就如常施恩惠,从天降雨,赏赐丰年,叫你们饮食饱足,满心喜乐。」
\VS{18}二人说了这些话,仅仅地拦住众人不献祭与他们。
\VS{19}但有些{\PN{犹太}}人从{\PN{安提阿}}和{\PN{以哥念}}来,挑唆众人,就用石头打{\PN{保罗}},以为他是死了,便拖到城外。
\VS{20}门徒正围着他,他就起来,走进城去。
\par }{\SH 回到叙利亚的安提阿
\par }{\PP 第二天,同{\PN{巴拿巴}}往{\PN{特庇}}去,
\VS{21}对那城里{\ADD{的人}}传了福音,使好些人作门徒,就回{\PN{路司得}}、{\PN{以哥念}}、{\PN{安提阿}}去,
\VS{22}坚固门徒的心,劝他们恒守所信的道;又说:「我们进入 神的国,必须经历许多艰难。」
\VS{23}二人在各教会中选立了长老,又禁食祷告,就把他们交托所信的主。
\VS{24}二人经过{\PN{彼西底}},来到{\PN{旁非利亚}}。
\VS{25}在{\PN{别加}}讲了道,就下{\PN{亚大利}}去,
\VS{26}从那里坐船,往{\PN{安提阿}}去。当初,他们被众人所托、蒙 神之恩、要办现在所做之工,就是在这地方。
\VS{27}到了那里,聚集了会众,就述说 神借他们所行的一切事,并 神怎样为外邦人开了信道的门。
\VS{28}二人就在那里同门徒住了多日。

\par }\Chap{15}{\SH 耶路撒冷会议
\par }{\PP \VerseOne{1}有几个人从{\PN{犹太}}下来,教训弟兄们说:「你们若不按{\PN{摩西}}的规条受割礼,不能得救。」
\VS{2}{\PN{保罗}}、{\PN{巴拿巴}}与他们大大地纷争辩论;{\ADD{众门徒}}就定规,叫{\PN{保罗}}、{\PN{巴拿巴}}和本会中几个人,为所辩论的,上{\PN{耶路撒冷}}去见使徒和长老。
\VS{3}于是教会送他们起行。他们经过{\PN{腓尼基}}、{\PN{撒马利亚}},随处传说外邦人归主的事,叫众弟兄都甚欢喜。
\VS{4}到了{\PN{耶路撒冷}},教会和使徒并长老都接待他们,他们就述说 神同他们所行的一切事。
\VS{5}惟有几个信徒是法利赛教门的人,起来说:「必须给外邦人行割礼,吩咐他们遵守{\PN{摩西}}的律法。」
\par }{\PP \VS{6}使徒和长老聚会商议这事;
\VS{7}辩论已经多了,{\PN{彼得}}就起来,说:「诸位弟兄,你们知道 神早已在你们中间拣选了我,叫外邦人从我口中得听福音之道,而且相信。
\VS{8}知道人心的 神也为他们作了见证,赐圣灵给他们,正如给我们一样;
\VS{9}又借着信洁净了他们的心,并不分他们我们。
\VS{10}现在为什么试探 神,要把我们祖宗和我们所不能负的轭放在门徒的颈项上呢?
\VS{11}我们得救乃是因主耶稣的恩,和他们一样,这是我们所信的。」
\par }{\PP \VS{12}众人都默默无声,听{\PN{巴拿巴}}和{\PN{保罗}}述说 神借他们在外邦人中所行的神迹奇事。
\VS{13}他们住了声,{\PN{雅各}}就说:「诸位弟兄,请听我的话。
\VS{14}方才{\PN{西门}}述说 神当初怎样眷顾外邦人,从他们中间选取百姓归于自己的名下;
\VS{15}众先知的话也与这意思相合。
\VS{16}正如{\ADD{经上}}所写的:
\par }{\Q 此后,我要回来,
\par }{\Q 重新修造{\PN{大卫}}倒塌的帐幕,
\par }{\Q 把那破坏的重新修造建立起来,
\par }{\Q \VS{17}叫余剩的人,
\par }{\Q 就是凡称为我名下的外邦人,
\par }{\Q 都寻求主。
\par }{\Q \VS{18}这话是从创世以来显明这事的主说的。
\par }{\PP \VS{19}「所以据我的意见,不可难为那归服 神的外邦人;
\VS{20}只要写信,吩咐他们禁戒偶像的污秽和奸淫,并勒死的牲畜和血。
\VS{21}因为从古以来,{\PN{摩西}}{\ADD{的书}}在各城有人传讲,每逢安息日,在会堂里诵读。」
\par }{\SH 会议的复函
\par }{\PP \VS{22}那时,使徒和长老并全教会定意从他们中间拣选人,差他们和{\PN{保罗}}、{\PN{巴拿巴}}同往{\PN{安提阿}}去;{\ADD{所拣选的}}就是称呼{\PN{巴撒巴}}的{\PN{犹大}}和{\PN{西拉}}。这两个人在弟兄中是作首领的。
\VS{23}于是写信交付他们,内中说:「使徒和作长老的弟兄们问{\PN{安提阿}}、{\PN{叙利亚}}、{\PN{基利家}}外邦众弟兄的安。
\VS{24}我们听说,有几个人从我们这里出去,用言语搅扰你们,惑乱你们的心。\FTNT{}{{\FR 15:24: }有古卷加:你们必须受割礼,守摩西的律法。}其实我们并没有吩咐他们。
\VS{25}所以,我们同心定意,拣选几个人,差他们同我们所亲爱的{\PN{巴拿巴}}和{\PN{保罗}}往你们那里去。
\VS{26}这二人是为我主耶稣基督的名不顾性命的。
\VS{27}我们就差了{\PN{犹大}}和{\PN{西拉}},他们也要亲口诉说这些事。
\VS{28}因为圣灵和我们定意不将别的重担放在你们身上,惟有几件事是不可少的,
\VS{29}就是禁戒祭偶像的物和血,并勒死的牲畜和奸淫。这几件你们若能自己禁戒不犯就好了。愿你们平安!」
\par }{\PP \VS{30}他们既奉了差遣,就下{\PN{安提阿}}去,聚集众人,交付书信。
\VS{31}众人念了,因为信上安慰的话就欢喜了。
\VS{32}{\PN{犹大}}和{\PN{西拉}}也是先知,就用许多话劝勉弟兄,坚固他们。
\VS{33}住了些日子,弟兄们打发他们平平安安地回到差遣他们的人那里去。\FTNT{}{{\FR 15:33: }有古卷加:34惟有西拉定意仍住在那里。}
\VS{35}但{\PN{保罗}}和{\PN{巴拿巴}}仍住在{\PN{安提阿}},和许多别人一同教训人,传主的道。
\par }{\SH 保罗和巴拿巴分手
\par }{\PP \VS{36}过了些日子,{\PN{保罗}}对{\PN{巴拿巴}}说:「我们可以回到从前宣传主道的各城,看望弟兄们景况如何。」
\VS{37}{\PN{巴拿巴}}有意要带称呼{\PN{马可}}的{\PN{约翰}}同去;
\VS{38}但{\PN{保罗}}因为{\PN{马可}}从前在{\PN{旁非利亚}}离开他们,不和他们同去做工,就以为不可带他去。
\VS{39}于是二人起了争论,甚至彼此分开。{\PN{巴拿巴}}带着{\PN{马可}},坐船往{\PN{塞浦路斯}}去;
\VS{40}{\PN{保罗}}拣选了{\PN{西拉}},也出去,蒙弟兄们把他交于主的恩中。
\VS{41}他就走遍{\PN{叙利亚}}、{\PN{基利家}},坚固众教会。

\par }\Chap{16}{\SH 提摩太跟保罗、西拉同工
\par }{\PP \VerseOne{1}{\PN{保罗}}来到{\PN{特庇}},又到{\PN{路司得}}。在那里有一个门徒,名叫{\PN{提摩太}},是信主之{\PN{犹太}}妇人的儿子,他父亲却是{\PN{希腊}}人。
\VS{2}{\PN{路司得}}和{\PN{以哥念}}的弟兄都称赞他。
\VS{3}{\PN{保罗}}要带他同去,只因那些地方的{\PN{犹太}}人都知道他父亲是{\PN{希腊}}人,就给他行了割礼。
\VS{4}他们经过各城,把{\PN{耶路撒冷}}使徒和长老所定的条规交给门徒遵守。
\VS{5}于是众教会信心越发坚固,人数天天加增。
\par }{\SH 保罗看见马其顿人的异象
\par }{\PP \VS{6}圣灵既然禁止他们在{\PN{亚细亚}}讲道,他们就经过{\PN{弗吕家}}、{\PN{加拉太}}一带地方。
\VS{7}到了{\PN{每西亚}}的边界,他们想要往{\PN{庇推尼}}去,耶稣的灵却不许。
\VS{8}他们就越过{\PN{每西亚}},下到{\PN{特罗亚}}去。
\VS{9}在夜间有异象现与{\PN{保罗}}。有一个{\PN{马其顿}}人站着求他说:「请你过到{\PN{马其顿}}来帮助我们。」
\VS{10}{\PN{保罗}}既看见这异象,我们随即想要往{\PN{马其顿}}去,以为 神召我们传福音给那里的人听。
\par }{\SH 吕底亚归主
\par }{\PP \VS{11}于是从{\PN{特罗亚}}开船,一直行到{\PN{撒摩特喇}},第二天到了{\PN{尼亚坡里}}。
\VS{12}从那里来到{\PN{腓立比}},就是{\PN{马其顿}}这一方的头一个城,也是{\ADD{
{\PN{罗马}}}}的驻防城。我们在这城里住了几天。
\VS{13}当安息日,我们出城门,到了河边,知道那里有一个祷告的地方,我们就坐下对那聚会的妇女讲道。
\VS{14}有一个卖紫色布疋的妇人,名叫{\PN{吕底亚}},是{\PN{推雅推喇}}城的人,素来敬拜 神。她听见了,主就开导她的心,叫她留心听{\PN{保罗}}所讲的话。
\VS{15}她和她一家既领了洗,便求我们说:「你们若以为我是真信主的\FTNT{}{{\FR 16:15: }或译:你们若以为我是忠心事主的},请到我家里来住」;于是强留我们。
\par }{\SH 在腓立比被囚
\par }{\PP \VS{16}后来,我们往那祷告的地方去。有一个使女迎着面来,她被巫鬼所附,用法术,叫她主人们大得财利。
\VS{17}她跟随{\PN{保罗}}和我们,喊着说:「这些人是至高 神的仆人,对你们传说救人的道。」
\VS{18}她一连多日这样喊叫,{\PN{保罗}}就心中厌烦,转身对那鬼说:「我奉耶稣基督的名,吩咐你从她身上出来!」那鬼当时就出来了。
\VS{19}使女的主人们见得利的指望没有了,便揪住{\PN{保罗}}和{\PN{西拉}},拉他们到市上去见首领;
\VS{20}又带到官长面前说:「这些人原是{\PN{犹太}}人,竟骚扰我们的城,
\VS{21}传我们{\PN{罗马}}人所不可受不可行的规矩。」
\VS{22}众人就一同起来攻击他们。官长吩咐剥了他们的衣裳,用棍打;
\VS{23}打了许多棍,便将他们下在监里,嘱咐禁卒严紧看守。
\VS{24}禁卒领了这样的命,就把他们下在内监里,两脚上了木狗。
\par }{\PP \VS{25}约在半夜,{\PN{保罗}}和{\PN{西拉}}祷告,唱诗赞美 神,众囚犯也侧耳而听。
\VS{26}忽然,地大震动,甚至监牢的地基都摇动了,监门立刻全开,众囚犯的锁链也都松开了。
\VS{27}禁卒一醒,看见监门全开,以为囚犯已经逃走,就拔刀要自杀。
\VS{28}{\PN{保罗}}大声呼叫说:「不要伤害自己!我们都在这里。」
\VS{29}禁卒叫人拿灯来,就跳进去,战战兢兢地俯伏在{\PN{保罗}}、{\PN{西拉}}面前;
\VS{30}又领他们出来,说:「二位先生,我当怎样行才可以得救?」
\VS{31}他们说:「当信主耶稣,你和你一家都必得救。」
\VS{32}他们就把主的道讲给他和他全家的人听。
\VS{33}当夜,就在那时候,禁卒把他们带去,洗他们的伤;他和属乎他的人立时都受了洗。
\VS{34}于是禁卒领他们上自己家里去,给他们摆上饭。他和全家,因为信了 神,都很喜乐。
\par }{\PP \VS{35}到了天亮,官长打发差役来,说:「释放那两个人吧。」
\VS{36}禁卒就把这话告诉{\PN{保罗}}说:「官长打发人来叫释放你们,如今可以出监,平平安安地去吧。」
\VS{37}{\PN{保罗}}却说:「我们是{\PN{罗马}}人,并没有定罪,他们就在众人面前打了我们,又把我们下在监里,现在要私下撵我们出去吗?这是不行的。叫他们自己来领我们出去吧!」
\VS{38}差役把这话回禀官长。官长听见他们是{\PN{罗马}}人,就害怕了,
\VS{39}于是来劝他们,领他们出来,请他们离开那城。
\VS{40}二人出了监,往{\PN{吕底亚}}家里去,见了弟兄们,劝慰他们一番,就走了。

\par }\Chap{17}{\SH 帖撒罗尼迦的骚动
\par }{\PP \VerseOne{1}{\PN{保罗}}和{\PN{西拉}}经过{\PN{暗妃坡里}}、{\PN{亚波罗尼亚}},来到{\PN{帖撒罗尼迦}},在那里有{\PN{犹太}}人的会堂。
\VS{2}{\PN{保罗}}照他素常的规矩进去,一连三个安息日,本着圣经与他们辩论,
\VS{3}讲解陈明基督必须受害,从死里复活;又说:「我所传与你们的这位耶稣就是基督。」
\VS{4}他们中间有些人听了劝,就附从{\PN{保罗}}和{\PN{西拉}},并有许多虔敬的{\PN{希腊}}人,尊贵的妇女也不少。
\VS{5}但那{\ADD{不信的}}{\PN{犹太}}人心里嫉妒,招聚了些市井匪类,搭伙成群,耸动合城的人闯进{\PN{耶孙}}的家,要将{\PN{保罗}}、{\PN{西拉}}带到百姓那里。
\VS{6}找不着他们,就把{\PN{耶孙}}和几个弟兄拉到地方官那里,喊叫说:「那搅乱天下的也到这里来了,
\VS{7}{\PN{耶孙}}收留他们。这些人都违背凯撒的命令,说另有一个王耶稣。」
\VS{8}众人和地方官听见这话,就惊慌了;
\VS{9}于是取了{\PN{耶孙}}和其余之人的保状,就释放了他们。
\par }{\SH 使徒们在庇哩亚传道
\par }{\PP \VS{10}弟兄们随即在夜间打发{\PN{保罗}}和{\PN{西拉}}往{\PN{庇哩亚}}去。二人到了,就进入{\PN{犹太}}人的会堂。
\VS{11}这地方的人贤于{\PN{帖撒罗尼迦}}的人,甘心领受这道,天天考查圣经,要晓得这道是与不是。
\VS{12}所以他们中间多有相信的,又有{\PN{希腊}}尊贵的妇女,男子也不少。
\VS{13}但{\PN{帖撒罗尼迦}}的{\PN{犹太}}人知道{\PN{保罗}}又在{\PN{庇哩亚}}传 神的道,也就往那里去,耸动搅扰众人。
\VS{14}当时弟兄们便打发{\PN{保罗}}往海边去,{\PN{西拉}}和{\PN{提摩太}}仍住在{\PN{庇哩亚}}。
\VS{15}送{\PN{保罗}}的人带他到了{\PN{雅典}},既领了{\PN{保罗}}的命,叫{\PN{西拉}}和{\PN{提摩太}}速速到他这里来,就回去了。
\par }{\SH 保罗在雅典
\par }{\PP \VS{16}{\PN{保罗}}在{\PN{雅典}}等候他们的时候,看见满城都是偶像,就心里着急;
\VS{17}于是在会堂里与{\PN{犹太}}人和虔敬的人,并每日在市上所遇见的人,辩论。
\VS{18}还有{\PN{伊壁鸠鲁}}和{\PN{斯多亚}}两门的学士,与他争论。有的说:「这胡言乱语的要说什么?」有的说:「他似乎是传说外邦鬼神的。」这话是因{\PN{保罗}}传讲耶稣与复活的道。
\VS{19}他们就把他带到{\PN{亚略·巴古}},说:「你所讲的这新道,我们也可以知道吗?
\VS{20}因为你有些奇怪的事传到我们耳中,我们愿意知道这些事是什么意思。」(
\VS{21}{\PN{雅典}}人和住在那里的客人都不顾别的事,只将新闻说说听听。)
\par }{\PP \VS{22}{\PN{保罗}}站在{\PN{亚略·巴古}}当中,说:「众位{\PN{雅典}}人哪,我看你们凡事很敬畏鬼神。
\VS{23}我游行的时候,观看你们所敬拜的,遇见一座坛,上面写着『未识之神』。你们所不认识而敬拜的,我现在告诉你们。
\VS{24}创造宇宙和其中万物的 神,既是天地的主,就不住人手所造的殿,
\VS{25}也不用人手服事,好像缺少什么;自己倒将生命、气息、万物,赐给万人。
\VS{26}他从一本\FTNT{}{{\FR 17:26: }本:有古卷是血脉}造出万族的人,住在全地上,并且预先定准他们的年限和所住的疆界,
\VS{27}要叫他们寻求 神,或者可以揣摩而得,其实他离我们各人不远;
\VS{28}我们生活、动作、存留,都在乎他。就如你们作诗的,有人说:『我们也是他所生的。』
\VS{29}我们既是 神所生的,就不当以为 神的神性像人用手艺、心思所雕刻的金、银、石。
\VS{30}世人蒙昧无知的时候, 神并不监察,如今却吩咐各处的人都要悔改。
\VS{31}因为他已经定了日子,要借着他所设立的人按公义审判天下,并且叫他从死里复活,给万人作可信的凭据。」
\par }{\PP \VS{32}众人听见从死里复活的话,就有讥诮他的;又有人说:「我们再听你讲这个吧!」
\VS{33}于是{\PN{保罗}}从他们当中出去了。
\VS{34}但有几个人贴近他,信了{\ADD{主}},其中有{\PN{亚略·巴古}}的官{\PN{丢尼修}},并一个妇人,名叫{\PN{大马哩}},还有别人一同信从。

\par }\Chap{18}{\SH 保罗在哥林多
\par }{\PP \VerseOne{1}这事以后,{\PN{保罗}}离了{\PN{雅典}},来到{\PN{哥林多}}。
\VS{2}遇见一个{\PN{犹太}}人,名叫{\PN{亚居拉}},他生在{\PN{本都}};因为{\PN{克劳第}}命{\PN{犹太}}人都离开{\PN{罗马}},新近带着妻{\PN{百基拉}},从{\PN{意大利}}来。{\PN{保罗}}就投奔了他们。
\VS{3}他们本是制造帐棚为业。{\PN{保罗}}因与他们同业,就和他们同住做工。
\VS{4}每逢安息日,{\PN{保罗}}在会堂里辩论,劝化{\PN{犹太}}人和{\PN{希腊}}人。
\par }{\PP \VS{5}{\PN{西拉}}和{\PN{提摩太}}从{\PN{马其顿}}来的时候,{\PN{保罗}}为道迫切,向{\PN{犹太}}人证明耶稣是基督。
\VS{6}他们既抗拒、毁谤,{\PN{保罗}}就抖着衣裳,说:「你们的罪\FTNT{}{{\FR 18:6: }原文是血}归到你们自己头上,与我无干\FTNT{}{{\FR 18:6: }原文是我却干净}。从今以后,我要往外邦人那里去。」
\VS{7}于是离开那里,到了一个人的家中;这人名叫{\PN{提多·犹士都}},是敬拜 神的,他的家靠近会堂。
\VS{8}管会堂的{\PN{基利司布}}和全家都信了主,还有许多{\PN{哥林多}}人听了,就相信受洗。
\VS{9}夜间,主在异象中对{\PN{保罗}}说:「不要怕,只管讲,不要闭口,
\VS{10}有我与你同在,必没有人下手害你,因为在这城里我有许多的百姓。」
\VS{11}{\PN{保罗}}在那里住了一年零六个月,将 神的道教训他们。
\par }{\PP \VS{12}到{\PN{迦流}}作{\PN{亚该亚}}方伯的时候,{\PN{犹太}}人同心起来攻击{\PN{保罗}},拉他到公堂,
\VS{13}说:「这个人劝人不按着律法敬拜 神。」
\VS{14}{\PN{保罗}}刚要开口,{\PN{迦流}}就对{\PN{犹太}}人说:「你们这些{\PN{犹太}}人!如果是为冤枉或奸恶的事,我理当耐性听你们。
\VS{15}但所争论的,若是关乎言语、名目,和你们的律法,你们自己去办吧!这样的事我不愿意审问」;
\VS{16}就把他们撵出公堂。
\VS{17}众人便揪住管会堂的{\PN{所提尼}},在堂前打他。这些事{\PN{迦流}}都不管。
\par }{\SH 保罗回到安提阿
\par }{\PP \VS{18}{\PN{保罗}}又住了多日,就辞别了弟兄,坐船往{\PN{叙利亚}}去;{\PN{百基拉}}、{\PN{亚居拉}}和他同去。他因为许过愿,就在{\PN{坚革哩}}剪了头发。
\VS{19}到了{\PN{以弗所}},{\PN{保罗}}就把他们留在那里,自己进了会堂,和{\PN{犹太}}人辩论。
\VS{20}众人请他多住些日子,他却不允,
\VS{21}就辞别他们,说:「 神若许我,我还要回到你们这里」;于是开船离了{\PN{以弗所}}。
\VS{22}在{\PN{凯撒利亚}}下了船,就上{\ADD{
{\PN{耶路撒冷}}}}去问教会安,随后下{\PN{安提阿}}去。
\VS{23}住了些日子,又离开那里,挨次经过{\PN{加拉太}}和{\PN{弗吕家}}地方,坚固众门徒。
\par }{\SH 亚波罗在以弗所讲道
\par }{\PP \VS{24}有一个{\PN{犹太}}人,名叫{\PN{亚波罗}},来到{\PN{以弗所}}。他生在{\PN{亚历山大}},是有学问\FTNT{}{{\FR 18:24: }或译:口才}的,最能讲解圣经。
\VS{25}这人已经在主的道上受了教训,心里火热,将耶稣的事详细讲论教训人;只是他单晓得{\PN{约翰}}的洗礼。
\VS{26}他在会堂里放胆讲道;{\PN{百基拉}}、{\PN{亚居拉}}听见,就接他来,将 神的道给他讲解更加详细。
\VS{27}他想要往{\PN{亚该亚}}去,弟兄们就勉励他,并写信请门徒接待他\FTNT{}{{\FR 18:27: }或译:弟兄们就写信劝门徒接待他}。他到了那里,多帮助那蒙恩信主的人,
\VS{28}在众人面前极有能力驳倒{\PN{犹太}}人,引圣经证明耶稣是基督。

\par }\Chap{19}{\SH 保罗在以弗所
\par }{\PP \VerseOne{1}{\PN{亚波罗}}在{\PN{哥林多}}的时候,{\PN{保罗}}经过了上边一带地方,就来到{\PN{以弗所}};在那里遇见几个门徒,
\VS{2}问他们说:「你们信的时候受了圣灵没有?」他们回答说:「没有,也未曾听见有圣灵{\ADD{赐下来}}。」
\VS{3}{\PN{保罗}}说:「这样,你们受的是什么洗呢?」他们说:「是{\PN{约翰}}的洗。」
\VS{4}{\PN{保罗}}说:「{\PN{约翰}}所行的是悔改的洗,告诉百姓当信那在他以后要来的,就是耶稣。」
\VS{5}他们听见这话,就奉主耶稣的名受洗。
\VS{6}{\PN{保罗}}按手在他们{\ADD{头}}上,圣灵便降在他们身上,他们就说方言,又说预言\FTNT{}{{\FR 19:6: }或译:又讲道}。
\VS{7}一共约有十二个人。
\par }{\PP \VS{8}{\PN{保罗}}进会堂,放胆讲道,一连三个月,辩论 神国的事,劝化众人。
\VS{9}后来,有些人心里刚硬不信,在众人面前毁谤这道,{\PN{保罗}}就离开他们,也叫门徒与他们分离,便在{\PN{推喇奴}}的学房天天辩论。
\VS{10}这样有两年之久,叫一切住在{\PN{亚细亚}}的,无论是{\PN{犹太}}人,是{\PN{希腊}}人,都听见主的道。
\par }{\SH 士基瓦的儿子们
\par }{\PP \VS{11}神借{\PN{保罗}}的手行了些非常的奇事;
\VS{12}甚至有人从{\PN{保罗}}身上拿手巾或围裙放在病人身上,病就退了,恶鬼也出去了。
\VS{13}那时,有几个游行各处、念咒赶鬼的{\PN{犹太}}人,向那被恶鬼附的人擅自称主耶稣的名,说:「我奉{\PN{保罗}}所传的耶稣敕令你们出来!」
\VS{14}做这事的,有{\PN{犹太}}祭司长{\PN{士基瓦}}的七个儿子。
\VS{15}恶鬼回答他们说:「耶稣我认识,{\PN{保罗}}我也知道。你们却是谁呢?」
\VS{16}恶鬼所附的人就跳在他们身上,胜了其中二人,制伏他们,叫他们赤着身子受了伤,从那房子里逃出去了。
\VS{17}凡住在{\PN{以弗所}}的,无论是{\PN{犹太}}人,是{\PN{希腊}}人,都知道这事,也都惧怕;主耶稣的名从此就尊大了。
\VS{18}那已经信的,多有人来承认诉说自己所行的事。
\VS{19}平素行邪术的,也有许多人把书拿来,堆积在众人面前焚烧。他们算计书价,便知道共合五万块钱。
\VS{20}主的道大大兴旺,而且得胜,就是这样。
\par }{\SH 以弗所的暴动
\par }{\PP \VS{21}这些事完了,{\PN{保罗}}心里定意经过了{\PN{马其顿}}、{\PN{亚该亚}},就往{\PN{耶路撒冷}}去;又说:「我到了那里以后,也必须往{\PN{罗马}}去看看。」
\VS{22}于是从帮助他的人中打发{\PN{提摩太}}、{\PN{以拉都}}二人往{\PN{马其顿}}去,自己暂时等在{\PN{亚细亚}}。
\par }{\PP \VS{23}那时,因为这道起的扰乱不小。
\VS{24}有一个银匠,名叫{\PN{底米丢}},是制造{\PN{亚底米}}神银龛的,他使这样手艺人生意发达。
\VS{25}他聚集他们和同行的工人,说:「众位,你们知道我们是倚靠这生意发财。
\VS{26}这{\PN{保罗}}不但在{\PN{以弗所}},也几乎在{\PN{亚细亚}}全地,引诱迷惑许多人,说:『人手所做的,不是神。』这是你们所看见所听见的。
\VS{27}这样,不独我们这事业被人藐视,就是大女神{\PN{亚底米}}的庙也要被人轻忽,连{\PN{亚细亚}}全地和普天下所敬拜的大女神之威荣也要消灭了。」
\par }{\PP \VS{28}众人听见,就怒气填胸,喊着说:「大哉,{\PN{以弗所}}人的{\PN{亚底米}}啊!」
\VS{29}满城都轰动起来。众人拿住与{\PN{保罗}}同行的{\PN{马其顿}}人{\PN{该犹}}和{\PN{亚里达古}},齐心拥进戏园里去。
\VS{30}{\PN{保罗}}想要进去,到百姓那里,门徒却不许他去。
\VS{31}还有{\PN{亚细亚}}几位首领,是{\PN{保罗}}的朋友,打发人来劝他,不要冒险到戏园里去。
\VS{32}聚集的人纷纷乱乱,有喊叫这个的,有喊叫那个的;大半不知道是为什么聚集。
\VS{33}有人把{\PN{亚历山大}}从众人中带出来,{\PN{犹太}}人推他往前,{\PN{亚历山大}}就摆手,要向百姓分诉;
\VS{34}只因他们认出他是{\PN{犹太}}人,就大家同声喊着说:「大哉!{\PN{以弗所}}人的{\PN{亚底米}}啊。」如此约有两小时。
\VS{35}那城里的书记安抚了众人,就说:「{\PN{以弗所}}人哪,谁不知道{\PN{以弗所}}人的城是看守大{\PN{亚底米}}的庙和从{\PN{宙斯}}那里落下来的{\ADD{像}}呢?
\VS{36}这事既是驳不倒的,你们就当安静,不可造次。
\VS{37}你们把这些人带来,他们并没有偷窃庙中之物,也没有谤 我们的女神。
\VS{38}若是{\PN{底米丢}}和他同行的人有控告人的事,自有放告的日子\FTNT{}{{\FR 19:38: }或译:自有公堂},也有方伯可以彼此对告。
\VS{39}你们若问别的事,就可以照常例聚集断定。
\VS{40}今日的扰乱本是无缘无故,我们难免被查问。论到这样聚众,我们也说不出所以然来。」
\VS{41}说了这话,便叫众人散去。

\par }\Chap{20}{\SH 保罗再访问马其顿和希腊
\par }{\PP \VerseOne{1}乱定之后,{\PN{保罗}}请门徒来,劝勉他们,就辞别起行,往{\PN{马其顿}}去。
\VS{2}走遍了那一带地方,用许多话劝勉门徒\FTNT{}{{\FR 20:2: }或译:众人},然后来到{\PN{希腊}}。
\VS{3}在那里住了三个月,将要坐船往{\PN{叙利亚}}去,{\PN{犹太}}人设计要害他,他就定意从{\PN{马其顿}}回去。
\VS{4}同他到{\PN{亚细亚}}去的,有{\PN{庇哩亚}}人{\PN{毕罗斯}}的儿子{\PN{所巴特}},{\PN{帖撒罗尼迦}}人{\PN{亚里达古}}和{\PN{西公都}},还有{\PN{特庇}}人{\PN{该犹}},并{\PN{提摩太}},又有{\PN{亚细亚}}人{\PN{推基古}}和{\PN{特罗非摩}}。
\VS{5}这些人先走,在{\PN{特罗亚}}等候我们。
\VS{6}过了除酵的日子,我们从{\PN{腓立比}}开船,五天到了{\PN{特罗亚}},和他们相会,在那里住了七天。
\par }{\SH 保罗最后一次访问特罗亚
\par }{\PP \VS{7}七日的第一日,我们聚会擘饼的时候,{\PN{保罗}}因为要次日起行,就与他们讲论,直讲到半夜。
\VS{8}我们聚会的那座楼上,有好些灯烛。
\VS{9}有一个少年人,名叫{\PN{犹推古}},坐在窗台上,困倦沉睡。{\PN{保罗}}讲了多时,少年人睡熟了,就从三层楼上掉下去;扶起他来,已经死了。
\VS{10}{\PN{保罗}}下去,伏在他身上,抱着他,说:「你们不要发慌,他的灵魂还在身上。」
\VS{11}{\PN{保罗}}又上去,擘饼,吃了,谈论许久,直到天亮,这才走了。
\VS{12}有人把那童子活活地领来,得的安慰不小。
\par }{\SH 从特罗亚到米利都
\par }{\PP \VS{13}我们先上船,开往{\PN{亚朔}}去,意思要在那里接{\PN{保罗}};因为他是这样安排的,他自己打算要步行。
\VS{14}他既在{\PN{亚朔}}与我们相会,我们就接他上船,来到{\PN{米推利尼}}。
\VS{15}从那里开船,次日到了{\PN{基阿}}的对面;又次日,在{\PN{撒摩}}靠岸;又次日,来到{\PN{米利都}}。
\VS{16}乃因{\PN{保罗}}早已定意越过{\PN{以弗所}},免得在{\PN{亚细亚}}耽延,他急忙前走,巴不得赶五旬节能到{\PN{耶路撒冷}}。
\par }{\SH 保罗向以弗所长老发表谈话
\par }{\PP \VS{17}{\PN{保罗}}从{\PN{米利都}}打发人往{\PN{以弗所}}去,请教会的长老来。
\VS{18}他们来了,{\PN{保罗}}就说:「你们知道,自从我到{\PN{亚细亚}}的日子以来,在你们中间始终为人如何,
\VS{19}服事主,凡事谦卑,眼中流泪,又因{\PN{犹太}}人的谋害,经历试炼。
\VS{20}你们也知道,凡与你们有益的,我没有一样避讳不说的,或在众人面前,或在各人家里,我都教导你们;
\VS{21}又对{\PN{犹太}}人和{\PN{希腊}}人证明当向 神悔改,信靠我主耶稣基督。
\VS{22}现在我往{\PN{耶路撒冷}}去,心甚迫切\FTNT{}{{\FR 20:22: }原文是心被捆绑},不知道在那里要遇见什么事;
\VS{23}但知道圣灵在各城里向我指证,说有捆锁与患难等待我。
\VS{24}我却不以性命为念,也不看为宝贵,只要行完我的路程,成就我从主耶稣所领受的职事,证明 神恩惠的福音。
\par }{\PP \VS{25}「我素常在你们中间来往,传讲 神国{\ADD{的道}};如今我晓得,你们以后都不得再见我的面了。
\VS{26}所以我今日向你们证明,你们中间无论何人死亡,罪不在我身上\FTNT{}{{\FR 20:26: }原文是我于众人的血是洁净的}。
\VS{27}因为 神的旨意,我并没有一样避讳不传给你们的。
\VS{28}圣灵立你们作全群的监督,你们就当为自己谨慎,也为全群谨慎,牧养 神的教会,就是他用自己血所买来的\FTNT{}{{\FR 20:28: }或译:救赎的}。
\VS{29}我知道,我去之后必有凶暴的豺狼进入你们中间,不爱惜羊群。
\VS{30}就是你们中间,也必有人起来说悖谬的话,要引诱门徒跟从他们。
\VS{31}所以你们应当警醒,记念我三年之久昼夜不住地流泪、劝戒你们各人。
\VS{32}如今我把你们交托 神和他恩惠的道;这道能建立你们,叫你们和一切成圣的人同得基业。
\VS{33}我未曾贪图一个人的金、银、衣服。
\VS{34}我这两只手常供给我和同人的需用,这是你们自己知道的。
\VS{35}我凡事给你们作榜样,叫你们知道应当这样劳苦,扶助软弱的人,又当记念主耶稣的话,说:『施比受更为有福。』」
\par }{\PP \VS{36}{\PN{保罗}}说完了这话,就跪下同众人祷告。
\VS{37}众人痛哭,抱着{\PN{保罗}}的颈项,和他亲嘴。
\VS{38}叫他们最伤心的,就是他说「以后不能再见我的面」那句话,于是送他上船去了。

\par }\Chap{21}{\SH 保罗上耶路撒冷
\par }{\PP \VerseOne{1}我们离别了众人,就开船一直行到{\PN{哥士}}。第二天到了{\PN{罗底}},从那里到{\PN{帕大喇}},
\VS{2}遇见一只船要往{\PN{腓尼基}}去,就上船起行。
\VS{3}望见{\PN{塞浦路斯}},就从南边行过,往{\PN{叙利亚}}去,我们就在{\PN{泰尔}}上岸,因为船要在那里卸货。
\VS{4}找着了门徒,就在那里住了七天。他们被圣灵感动,对{\PN{保罗}}说:「不要上{\PN{耶路撒冷}}去。」
\VS{5}过了这几天,我们就起身前行。他们众人同妻子儿女,送我们到城外,我们都跪在岸上祷告,彼此辞别。
\VS{6}我们上了船,他们就回家去了。
\par }{\PP \VS{7}我们从{\PN{泰尔}}行尽了水路,来到{\PN{多利买}},就问那里的弟兄安,和他们同住了一天。
\VS{8}第二天,我们离开那里,来到{\PN{凯撒利亚}},就进了传福音的{\PN{腓利}}家里,和他同住。他是那七个{\ADD{执事}}里的一个。
\VS{9}他有四个女儿,都是处女,是说预言的。
\VS{10}我们在那里多住了几天,有一个先知,名叫{\PN{亚迦布}},从{\PN{犹太}}下来,
\VS{11}到了我们这里,就拿{\PN{保罗}}的腰带捆上自己的手脚,说:「圣灵说:{\PN{犹太}}人在{\PN{耶路撒冷}},要如此捆绑这腰带的主人,把他交在外邦人手里。」
\VS{12}我们和那本地的人听见这话,都苦劝{\PN{保罗}}不要上{\PN{耶路撒冷}}去。
\VS{13}{\PN{保罗}}说:「你们为什么这样痛哭,使我心碎呢?我为主耶稣的名,不但被人捆绑,就是死在{\PN{耶路撒冷}}也是愿意的。」
\VS{14}{\PN{保罗}}既不听劝,我们便住了口,只说:「愿主的旨意成就」,便了。
\par }{\PP \VS{15}过了几日,我们收拾行李上{\PN{耶路撒冷}}去。
\VS{16}有{\PN{凯撒利亚}}的几个门徒和我们同去,带我们到一个久为\FTNT{}{{\FR 21:16: }久为:或译老}门徒的家里,叫我们与他同住;他名叫{\PN{拿孙}},是{\PN{塞浦路斯}}人。
\par }{\SH 保罗访问雅各
\par }{\PP \VS{17}到了{\PN{耶路撒冷}},弟兄们欢欢喜喜地接待我们。
\VS{18}第二天,{\PN{保罗}}同我们去见{\PN{雅各}};长老们也都在那里。
\VS{19}{\PN{保罗}}问了他们安,便将 神用他传教,在外邦人中间所行之事,一一地述说了。
\VS{20}他们听见,就归荣耀与 神,对{\PN{保罗}}说:「兄台,你看{\PN{犹太}}人中信主的有多少万,并且都为律法热心。
\VS{21}他们听见人说,你教训一切在外邦的{\PN{犹太}}人离弃{\PN{摩西}},对他们说,不要给孩子行割礼,也不要遵行条规。
\VS{22}众人必听见你来了,这可怎么办呢?
\VS{23}你就照着我们的话行吧!我们这里有四个人,都有愿在身。
\VS{24}你带他们去,与他们一同行洁净的礼,替他们拿出规费,叫他们得以剃头。这样,众人就可知道,先前所听见你的事都是虚的;并可知道,你自己为人,循规蹈矩,遵行律法。
\VS{25}至于信主的外邦人,我们已经写信拟定,叫他们谨忌那祭偶像之物,和血,并勒死的牲畜,与奸淫。」
\VS{26}于是{\PN{保罗}}带着那四个人,第二天与他们一同行了洁净的礼,进了殿,报明洁净的日期满足,只等{\ADD{祭司}}为他们各人献祭。
\par }{\SH 保罗在圣殿里被捕
\par }{\PP \VS{27}那七日将完,从{\PN{亚细亚}}来的{\PN{犹太}}人看见{\PN{保罗}}在殿里,就耸动了众人,下手拿他,
\VS{28}喊叫说:「{\PN{以色列}}人来帮助,这就是在各处教训众人糟践{\ADD{我们}}百姓和律法,并这地方的。他又带着{\PN{希腊}}人进殿,污秽了这圣地。」(
\VS{29}这话是因他们曾看见{\PN{以弗所}}人{\PN{特罗非摩}}同{\PN{保罗}}在城里,以为{\PN{保罗}}带他进了殿。)
\VS{30}合城都震动,百姓一齐跑来,拿住{\PN{保罗}},拉他出殿,殿门立刻都关了。
\VS{31}他们正想要杀他,有人报信给营里的千夫长说:「{\PN{耶路撒冷}}合城都乱了。」
\VS{32}千夫长立时带着兵丁和几个百夫长,跑下去到他们那里。他们见了千夫长和兵丁,就止住不打{\PN{保罗}}。
\VS{33}于是千夫长上前拿住他,吩咐用两条铁链捆锁;又问他是什么人,做的是什么事。
\VS{34}众人有喊叫这个的,有喊叫那个的;千夫长因为这样乱嚷,得不着实情,就吩咐人将{\PN{保罗}}带进营楼去。
\VS{35}到了台阶上,众人挤得凶猛,兵丁只得将{\PN{保罗}}抬起来。
\VS{36}众人跟在后面,喊着说:「除掉他!」
\par }{\SH 保罗为自己辩护
\par }{\PP \VS{37}将要带他进营楼,{\PN{保罗}}对千夫长说:「我对你说句话可以不可以?」他说:「你懂得希腊话吗?
\VS{38}你莫非是从前作乱、带领四千凶徒往旷野去的那{\PN{埃及}}人吗?」
\VS{39}{\PN{保罗}}说:「我本是{\PN{犹太}}人,生在{\PN{基利家}}的{\PN{大数}},并不是无名小城的人。求你准我对百姓说话。」
\VS{40}千夫长准了。{\PN{保罗}}就站在台阶上,向百姓摆手,他们都静默无声,{\PN{保罗}}便用希伯来话对他们说:

\par }\Chap{22}{\PP \VerseOne{1}「诸位父兄请听,我现在对你们分诉。」
\VS{2}众人听他说的是希伯来话,就更加安静了。
\VS{3}{\PN{保罗}}说:「我原是{\PN{犹太}}人,生在{\PN{基利家}}的{\PN{大数}},长在这城里,在{\PN{迦玛列}}门下,按着我们祖宗严紧的律法受教,热心事奉 神,像你们众人今日一样。
\VS{4}我也曾逼迫奉这道的人,直到死地,无论男女都锁拿下监。
\VS{5}这是大祭司和众长老都可以给我作见证的。我又领了他们达与弟兄的书信,往{\PN{大马士革}}去,要把在那里{\ADD{奉这道}}的人锁拿,带到{\PN{耶路撒冷}}受刑。」
\par }{\SH 保罗叙述归主的经过
\par }{\R (9·1—19;26·12—18)
\par }{\PP \VS{6}「我将到{\PN{大马士革}},正走的时候,约在晌午,忽然从天上发大光,四面照着我。
\VS{7}我就仆倒在地,听见有声音对我说:『{\PN{扫罗}}!{\PN{扫罗}}!你为什么逼迫我?』
\VS{8}我回答说:『主啊,你是谁?』他说:『我就是你所逼迫的{\PN{拿撒勒}}人耶稣。』
\VS{9}与我同行的人看见了那光,却没有听明那位对我说话的声音。
\VS{10}我说:『主啊,我当做什么?』主说:『起来,进{\PN{大马士革}}去,在那里,要将所派你做的一切事告诉你。』
\VS{11}我因那光的荣耀不能看见,同行的人就拉着我手进了{\PN{大马士革}}。
\VS{12}那{\ADD{里}}有一个人,名叫{\PN{亚拿尼亚}},按着律法是虔诚人,为一切住在那里的{\PN{犹太}}人所称赞。
\VS{13}他来见我,站在旁边,对我说:『兄弟{\PN{扫罗}},你可以看见。』我当时往上一看,就看见了他。
\VS{14}他又说:『我们祖宗的 神拣选了你,叫你明白他的旨意,又得见那义者,听他口中所出的声音。
\VS{15}因为你要将所看见的,所听见的,对着万人为他作见证。
\VS{16}现在你为什么耽延呢?起来,求告他的名受洗,洗去你的罪。』」
\par }{\SH 保罗奉召向外邦人传道
\par }{\PP \VS{17}「后来,我回到{\PN{耶路撒冷}},在殿里祷告的时候,魂游象外,
\VS{18}看见主向我说:『你赶紧地离开{\PN{耶路撒冷}},不可迟延;因你为我作的见证,这里的人必不领受。』
\VS{19}我就说:『主啊,他们知道我从前把信你的人收在监里,又在各会堂里鞭打他们。
\VS{20}并且你的见证人{\PN{司提反}}被害流血的时候,我也站在旁边欢喜;又看守害死他之人的衣裳。』
\VS{21}主向我说:『你去吧!我要差你远远地往外邦人那里去。』」
\par }{\SH 保罗与长官交涉
\par }{\PP \VS{22}众人听他说到这句话,就高声说:「这样的人,从世上除掉他吧!他是不当活着的。」
\VS{23}众人喧嚷,摔掉衣裳,把尘土向空中扬起来。
\VS{24}千夫长就吩咐将{\PN{保罗}}带进营楼去,叫人用鞭子拷问他,要知道他们向他这样喧嚷是为什么缘故。
\VS{25}刚用皮条捆上,{\PN{保罗}}对旁边站着的百夫长说:「人是{\PN{罗马}}人,又没有定罪,你们就鞭打他,有这个例吗?」
\VS{26}百夫长听见这话,就去见千夫长,告诉他说:「你要做什么?这人是{\PN{罗马}}人。」
\VS{27}千夫长就来问{\PN{保罗}}说:「你告诉我,你是{\PN{罗马}}人吗?」{\PN{保罗}}说:「是。」
\VS{28}千夫长说:「我用许多银子才入了{\PN{罗马}}的民籍。」{\PN{保罗}}说:「我生来就是。」
\VS{29}于是那些要拷问{\PN{保罗}}的人就离开他去了。千夫长既知道他是{\PN{罗马}}人,又因为捆绑了他,也害怕了。
\par }{\SH 保罗在公会前申诉
\par }{\PP \VS{30}第二天,千夫长为要知道{\PN{犹太}}人控告{\PN{保罗}}的实情,便解开他,吩咐祭司长和全公会的人都聚集,将{\PN{保罗}}带下来,叫他站在他们面前。

\par }\Chap{23}{\PP \VerseOne{1}{\PN{保罗}}定睛看着公会的人,说:「弟兄们,我在 神面前行事为人都是凭着良心,直到今日。」
\VS{2}大祭司{\PN{亚拿尼亚}}就吩咐旁边站着的人打他的嘴。
\VS{3}{\PN{保罗}}对他说:「你这粉饰的墙, 神要打你!你坐堂为的是按律法审问我,你竟违背律法,吩咐人打我吗?」
\VS{4}站在旁边的人说:「你辱骂 神的大祭司吗?」
\VS{5}{\PN{保罗}}说:「弟兄们,我不晓得他是大祭司;{\ADD{经上}}记着说:『不可毁谤你百姓的官长。』」
\par }{\PP \VS{6}{\PN{保罗}}看出大众一半是撒都该人,一半是法利赛人,就在公会中大声说:「弟兄们,我是法利赛人,也是法利赛人的子孙。我现在受审问,是为盼望死人复活。」
\VS{7}说了这话,法利赛人和撒都该人就争论起来,会众分为两党。
\VS{8}因为撒都该人说,没有复活,也没有天使和鬼魂;法利赛人却说,两样都有。
\VS{9}于是大大地喧嚷起来。有几个法利赛党的文士站起来争辩说:「我们看不出这人有什么恶处,倘若有鬼魂或是天使对他说过话,怎么样呢?」
\VS{10}那时大起争吵,千夫长恐怕{\PN{保罗}}被他们扯碎了,就吩咐兵丁下去,把他从众人当中抢出来,带进营楼去。
\par }{\PP \VS{11}当夜,主站在{\PN{保罗}}旁边,说:「放心吧!你怎样在{\PN{耶路撒冷}}为我作见证,也必怎样在{\PN{罗马}}{\ADD{为我}}作见证。」
\par }{\SH 杀害保罗的阴谋
\par }{\PP \VS{12}到了天亮,{\PN{犹太}}人同谋起誓,说:「若不先杀{\PN{保罗}}就不吃不喝。」
\VS{13}这样同心起誓的有四十多人。
\VS{14}他们来见祭司长和长老,说:「我们已经起了一个大誓,若不先杀{\PN{保罗}}就不吃什么。
\VS{15}现在你们和公会要知会千夫长,叫他带下{\PN{保罗}}到你们这里来,假作要详细察考他的事;我们已经预备好了,不等他来到跟前就杀他。」
\VS{16}{\PN{保罗}}的外甥听见他们设下埋伏,就来到营楼里告诉{\PN{保罗}}。
\VS{17}{\PN{保罗}}请一个百夫长来,说:「你领这少年人去见千夫长,他有事告诉他。」
\VS{18}于是把他领去见千夫长,说:「被囚的{\PN{保罗}}请我到他那里,求我领这少年人来见你;他有事告诉你。」
\VS{19}千夫长就拉着他的手,走到一旁,私下问他说:「你有什么事告诉我呢?」
\VS{20}他说:「{\PN{犹太}}人已经约定,要求你明天带下{\PN{保罗}}到公会里去,假作要详细查问他的事。
\VS{21}你切不要随从他们;因为他们有四十多人埋伏,已经起誓说,若不先杀{\PN{保罗}}就不吃不喝。现在预备好了,只等你应允。」
\VS{22}于是千夫长打发少年人走,嘱咐他说:「不要告诉人你将这事报给我了。」
\par }{\SH 保罗被解交腓力斯巡抚
\par }{\PP \VS{23}千夫长便叫了两个百夫长来,说:「预备步兵二百,马兵七十,长枪手二百,今夜亥初往{\PN{凯撒利亚}}去;
\VS{24}也要预备牲口叫{\PN{保罗}}骑上,护送到巡抚{\PN{腓力斯}}那里去。」
\VS{25}千夫长又写了文书,
\VS{26}大略说:「{\PN{克劳第·吕西亚}},请巡抚{\PN{腓力斯}}大人安。
\VS{27}这人被{\PN{犹太}}人拿住,将要杀害,我得知他是{\PN{罗马}}人,就带兵丁下去救他出来。
\VS{28}因要知道他们告他的缘故,我就带他下到他们的公会去,
\VS{29}便查知他被告是因他们律法的辩论,并没有什么该死该绑的罪名。
\VS{30}后来有人把要害他的计谋告诉我,我就立时解他到你那里去,又吩咐告他的人在你面前告他。\FTNT{}{{\FR 23:30: }有古卷加:愿你平安!}」
\par }{\PP \VS{31}于是,兵丁照所吩咐他们的,将{\PN{保罗}}夜里带到{\PN{安提帕底}}。
\VS{32}第二天,让马兵护送,他们就回营楼去。
\VS{33}马兵来到{\PN{凯撒利亚}},把文书呈给巡抚,便叫{\PN{保罗}}站在他面前。
\VS{34}巡抚看了文书,问{\PN{保罗}}是哪省的人,既晓得他是{\PN{基利家}}人,
\VS{35}就说:「等告你的人来到,我要细听你的事」;便吩咐人把他看守在{\PN{希律}}的衙门里。

\par }\Chap{24}{\SH 犹太人控告保罗
\par }{\PP \VerseOne{1}过了五天,大祭司{\PN{亚拿尼亚}}同几个长老,和一个辩士{\PN{帖土罗}}下来,向巡抚控告{\PN{保罗}}。
\VS{2}{\PN{保罗}}被提了来,{\PN{帖土罗}}就告他说:
\VS{3}「{\PN{腓力斯}}大人,我们因你得以大享太平,并且这一国的弊病,因着你的先见得以更正了;我们随时随地满心感谢不尽。
\VS{4}惟恐多说,你嫌烦絮,只求你宽容听我们说几句话。
\VS{5}我们看这个人,如同瘟疫一般,是鼓动普天下众{\PN{犹太}}人生乱的,又是{\PN{拿撒勒}}教党里的一个头目,
\VS{6}连{\ADD{圣}}殿他也想要污秽;我们把他捉住了。\FTNT{}{{\FR 24:6: }有古卷加:要按我们的律法审问, \VS{7}不料,千夫长吕西亚前来,甚是强横,从我们手中把他夺去,吩咐告他的人到你这里来。}
\VS{8}你自己究问他,就可以知道我们告他的一切事了。」
\VS{9}众{\PN{犹太}}人也随着告他说:「事情诚然是这样。」
\par }{\SH 保罗为自己辩护
\par }{\PP \VS{10}巡抚点头叫{\PN{保罗}}说话。他就说:「我知道你在这国里断事多年,所以我乐意为自己分诉。
\VS{11}你查问就可以知道,从我上{\PN{耶路撒冷}}礼拜到今日不过有十二天。
\VS{12}他们并没有看见我在殿里,或是在会堂里,或是在城里,和人辩论,耸动众人。
\VS{13}他们现在所告我的事并不能对你证实了。
\VS{14}但有一件事,我向你承认,就是他们所称为异端的道,我正按着那道事奉我祖宗的 神,又信合乎{\ADD{律法}}的和{\ADD{先知书}}上一切所记载的,
\VS{15}并且靠着 神,盼望死人,无论善恶,都要复活,就是他们自己也有这个盼望。
\VS{16}我因此自己勉励,对 神对人,常存无亏的良心。
\VS{17}过了几年,我带着周济本国的捐项和供献的物上去。
\VS{18}正献的时候,他们看见我在殿里已经洁净了,并没有聚众,也没有吵嚷,惟有几个从{\PN{亚细亚}}来的{\PN{犹太}}人。
\VS{19}他们若有告我的事,就应当到你面前来告我。
\VS{20}即或不然,这些人若看出我站在公会前,有妄为的地方,他们自己也可以说明。
\VS{21}纵然有,也不过一句话,就是我站在他们中间大声说:『我今日在你们面前受审,是为死人复活{\ADD{的道理}}。』」
\par }{\PP \VS{22}{\PN{腓力斯}}本是详细晓得这道,就支吾他们说:「且等千夫长{\PN{吕西亚}}下来,我要审断你们的事。」
\VS{23}于是吩咐百夫长看守{\PN{保罗}},并且宽待他,也不拦阻他的亲友来供给他。
\par }{\SH 腓力斯留保罗在监里
\par }{\PP \VS{24}过了几天,{\PN{腓力斯}}和他夫人—{\PN{犹太}}的女子{\PN{土西拉}}—一同来到,就叫了{\PN{保罗}}来,听他讲论信基督耶稣的道。
\VS{25}{\PN{保罗}}讲论公义、节制,和将来的审判。{\PN{腓力斯}}甚觉恐惧,说:「你暂且去吧,等我得便再叫你来。」
\VS{26}{\PN{腓力斯}}又指望{\PN{保罗}}送他银钱,所以屡次叫他来,和他谈论。
\VS{27}过了两年,{\PN{波求·非斯都}}接了{\PN{腓力斯}}的任;{\PN{腓力斯}}要讨{\PN{犹太}}人的喜欢,就留{\PN{保罗}}在监里。

\par }\Chap{25}{\SH 保罗要向凯撒上诉
\par }{\PP \VerseOne{1}{\PN{非斯都}}到了任,过了三天,就从{\PN{凯撒利亚}}上{\PN{耶路撒冷}}去。
\VS{2}祭司长和{\PN{犹太}}人的首领向他控告{\PN{保罗}},
\VS{3}又央告他,求他的情,将{\PN{保罗}}提到{\PN{耶路撒冷}}来,他们要在路上埋伏杀害他。
\VS{4}{\PN{非斯都}}却回答说:「{\PN{保罗}}押在{\PN{凯撒利亚}},我自己快要往那里去」;
\VS{5}又说:「你们中间有权势的人与我一同下去,那人若有什么不是,就可以告他。」
\par }{\PP \VS{6}{\PN{非斯都}}在他们那里住了不过十天八天,就下{\PN{凯撒利亚}}去;第二天坐堂,吩咐将{\PN{保罗}}提上来。
\VS{7}{\PN{保罗}}来了,那些从{\PN{耶路撒冷}}下来的{\PN{犹太}}人周围站着,将许多重大的事控告他,都是不能证实的。
\VS{8}{\PN{保罗}}分诉说:「无论{\PN{犹太}}人的律法,或是{\ADD{圣}}殿,或是凯撒,我都没有干犯。」
\VS{9}但{\PN{非斯都}}要讨{\PN{犹太}}人的喜欢,就问{\PN{保罗}}说:「你愿意上{\PN{耶路撒冷}}去,在那里听我审断这事吗?」
\VS{10}{\PN{保罗}}说:「我站在凯撒的堂前,这就是我应当受审的地方。我向{\PN{犹太}}人并没有行过什么不义的事,这也是你明明知道的。
\VS{11}我若行了不义的事,犯了什么该死的罪,就是死,我也不辞。他们所告我的事若都不实,就没有人可以把我交给他们。我要上告于凯撒。」
\VS{12}{\PN{非斯都}}和议会商量了,就说:「你既上告于凯撒,可以往凯撒那里去。」
\par }{\SH 保罗被带到亚基帕王面前
\par }{\PP \VS{13}过了些日子,{\PN{亚基帕}}王和{\PN{百妮基}}氏来到{\PN{凯撒利亚}},问{\PN{非斯都}}安。
\VS{14}在那里住了多日,{\PN{非斯都}}将{\PN{保罗}}的事告诉王,说:「这里有一个人,是{\PN{腓力斯}}留在监里的。
\VS{15}我在{\PN{耶路撒冷}}的时候,祭司长和{\PN{犹太}}的长老将他的事禀报了我,求我定他的罪。
\VS{16}我对他们说,无论什么人,被告还没有和原告对质,未得机会分诉所告他的事,就先定他的罪,这不是{\PN{罗马}}人的条例。
\VS{17}及至他们都来到这里,我就不耽延,第二天便坐堂,吩咐把那人提上来。
\VS{18}告他的人站着告他;所告的,并没有我所逆料的那等恶事。
\VS{19}不过是有几样辩论,为他们自己敬鬼神的事,又为一个人名叫耶稣,是已经死了,{\PN{保罗}}却说他是活着的。
\VS{20}这些事当怎样究问,我心里作难,所以问他说:『你愿意上{\PN{耶路撒冷}}去,在那里为这些事听审吗?』
\VS{21}但{\PN{保罗}}求我留下他,要听皇上审断,我就吩咐把他留下,等我解他到凯撒那里去。」
\VS{22}{\PN{亚基帕}}对{\PN{非斯都}}说:「我自己也愿听这人辩论。」{\PN{非斯都}}说:「明天你可以听。」
\par }{\PP \VS{23}第二天,{\PN{亚基帕}}和{\PN{百妮基}}大张威势而来,同着众千夫长和城里的尊贵人进了公厅。{\PN{非斯都}}吩咐一声,就有人将{\PN{保罗}}带进来。
\VS{24}{\PN{非斯都}}说:「{\PN{亚基帕}}王和在这里的诸位啊,你们看这人,就是一切{\PN{犹太}}人,在{\PN{耶路撒冷}}和这里,曾向我恳求、呼叫说:『不可容他再活着。』
\VS{25}但我查明他没有犯什么该死的罪,并且他自己上告于皇帝,所以我定意把他解去。
\VS{26}论到这人,我没有确实的事可以奏明主上。因此,我带他到你们面前,也特意带他到你{\PN{亚基帕}}王面前,为要在查问之后有所陈奏。
\VS{27}据我看来,解送囚犯,不指明他的罪案是不合理的。」

\par }\Chap{26}{\SH 保罗在亚基帕王面前申辩
\par }{\PP \VerseOne{1}{\PN{亚基帕}}对{\PN{保罗}}说:「准你为自己辩明。」于是{\PN{保罗}}伸手分诉,说:「
\VS{2}{\PN{亚基帕}}王啊,{\PN{犹太}}人所告我的一切事,今日得在你面前分诉,实为万幸;
\VS{3}更可幸的,是你熟悉{\PN{犹太}}人的规矩和他们的辩论;所以求你耐心听我。
\VS{4}我从起初在本国的民中,并在{\PN{耶路撒冷}},自幼为人如何,{\PN{犹太}}人都知道。
\VS{5}他们若肯作见证就晓得,我从起初是按着我们教中最严紧的教门作了法利赛人。
\VS{6}现在我站在这里受审,是因为指望 神向我们祖宗所应许的;
\VS{7}这应许,我们十二个支派,昼夜切切地事奉 {\ADD{神}},都指望得着。王啊,我被{\PN{犹太}}人控告,就是因这指望。
\VS{8}神叫死人复活,你们为什么看作不可信的呢?
\VS{9}从前我自己以为应当多方攻击{\PN{拿撒勒}}人耶稣的名,
\VS{10}我在{\PN{耶路撒冷}}也曾这样行了。既从祭司长得了权柄,我就把许多圣徒囚在监里。他们被杀,我也出名定案。
\VS{11}在各会堂,我屡次用刑强逼他们说亵渎的话,又分外恼恨他们,甚至追逼他们,直到外邦的城邑。」
\par }{\SH 保罗叙述归主经过
\par }{\R (9·1—19;22·6—16)
\par }{\PP \VS{12}「那时,我领了祭司长的权柄和命令,往{\PN{大马士革}}去。
\VS{13}王啊,我在路上,晌午的时候,看见从天发光,比日头还亮,四面照着我并与我同行的人。
\VS{14}我们都仆倒在地,我就听见有声音用希伯来话向我说:『{\PN{扫罗}}!{\PN{扫罗}}!为什么逼迫我?你用脚踢刺是难的!』
\VS{15}我说:『主啊,你是谁?』主说:『我就是你所逼迫的耶稣。
\VS{16}你起来站着,我特意向你显现,要派你作执事,作见证,将你所看见的事和我将要指示你的事证明出来。
\VS{17}我也要救你脱离百姓和外邦人的手。
\VS{18}我差你到他们那里去,要叫他们的眼睛得开,从黑暗中归向光明,从撒但权下归向 神;又因信我,得蒙赦罪,和一切成圣的人同得基业。』」
\par }{\SH 保罗向犹太和外邦人作见证
\par }{\PP \VS{19}「{\PN{亚基帕}}王啊,我故此没有违背那从天上来的异象;
\VS{20}先在{\PN{大马士革}},后在{\PN{耶路撒冷}}和{\PN{犹太}}全地,以及外邦,劝勉他们应当悔改归向 神,行事与悔改的心相称。
\VS{21}因此,{\PN{犹太}}人在殿里拿住我,想要杀我。
\VS{22}然而我蒙 神的帮助,直到今日还站得住,对着尊贵、卑贱、老幼作见证;所讲的并不外乎众先知和{\PN{摩西}}所说将来必成的事,
\VS{23}就是基督必须受害,并且因从死里复活,要首先把光明的道传给百姓和外邦人。」
\par }{\SH 保罗恳请亚基帕信主
\par }{\PP \VS{24}{\PN{保罗}}这样分诉,{\PN{非斯都}}大声说:「{\PN{保罗}},你癫狂了吧。你的学问太大,反叫你癫狂了!」
\VS{25}{\PN{保罗}}说:「{\PN{非斯都}}大人,我不是癫狂,我说的乃是真实明白话。
\VS{26}王也晓得这些事,所以我向王放胆直言,我深信这些事没有一件向王隐藏的,因都不是在背地里做的。
\VS{27}{\PN{亚基帕}}王啊,你信先知吗?我知道你是信的。」
\VS{28}{\PN{亚基帕}}对{\PN{保罗}}说:「你想少微一劝,便叫我作基督徒啊\FTNT{}{{\FR 26:28: }或译:你这样劝我,几乎叫我作基督徒了}!」
\VS{29}{\PN{保罗}}说:「无论是少劝是多劝,我向 神所求的,不但你一个人,就是今天一切听我的,都要像我一样,只是不要像我有这些锁链。」
\par }{\PP \VS{30}于是,王和巡抚并{\PN{百妮基}}与同坐的人都起来,
\VS{31}退到里面,彼此谈论说:「这人并没有犯什么该死该绑的罪。」
\VS{32}{\PN{亚基帕}}又对{\PN{非斯都}}说:「这人若没有上告于凯撒,就可以释放了。」

\par }\Chap{27}{\SH 保罗坐船往罗马
\par }{\PP \VerseOne{1}{\ADD{
{\PN{非斯都}}}}既然定规了,叫我们坐船往{\PN{意大利}}去,便将{\PN{保罗}}和别的囚犯交给御营里的一个百夫长,名叫{\PN{犹流}}。
\VS{2}有一只{\PN{亚大米田}}的船,要沿着{\PN{亚细亚}}一带地方的海边走,我们就上了那船开行;有{\PN{马其顿}}的{\PN{帖撒罗尼迦}}人{\PN{亚里达古}}和我们同去。
\VS{3}第二天,到了{\PN{西顿}};{\PN{犹流}}宽待{\PN{保罗}},准他往朋友那里去,受他们的照应。
\VS{4}从那里又开船,因为风不顺,就贴着{\PN{塞浦路斯}}背风岸行去。
\VS{5}过了{\PN{基利家}}、{\PN{旁非利亚}}前面的海,就到了{\PN{吕家}}的{\PN{每拉}}。
\VS{6}在那里,百夫长遇见一只{\PN{亚历山大}}的船,要往{\PN{意大利}}去,便叫我们上了那船。
\VS{7}一连多日,船行得慢,仅仅来到{\PN{革尼土}}的对面。因为被风拦阻,就贴着{\PN{克里特}}背风岸,从{\PN{撒摩尼}}对面行过。
\VS{8}我们沿岸行走,仅仅来到一个地方,名叫{\PN{佳澳}};离那里不远,有{\PN{拉西亚}}城。
\par }{\PP \VS{9}走的日子多了,已经过了禁食的节期,行船又危险,{\PN{保罗}}就劝众人说:
\VS{10}「众位,我看这次行船,不但货物和船要受损伤,大遭破坏,连我们的性命也难保。」
\VS{11}但百夫长信从掌船的和船主,不信从{\PN{保罗}}所说的。
\VS{12}且因在这海口过冬不便,船上的人就多半说,不如开船离开这地方,或者能到{\PN{菲尼基}}过冬。{\PN{菲尼基}}是{\PN{克里特}}的一个海口,一面朝东北,一面朝东南。
\par }{\SH 海上的风暴
\par }{\PP \VS{13}这时,微微起了南风,他们以为得意,就起了锚,贴近{\PN{克里特}}行去。
\VS{14}不多几时,狂风从岛上扑下来;那风名叫「友拉革罗」。
\VS{15}船被风抓住,敌不住风,我们就任风刮去。
\VS{16}贴着一个小岛的背风岸奔行,那岛名叫{\PN{高大}},在那里仅仅收住了小船。
\VS{17}既然把小船拉上来,就用缆索捆绑船底,又恐怕在{\PN{赛耳底}}沙滩上搁了浅,就落下篷来,任船飘去。
\VS{18}我们被风浪逼得甚急,第二天众人就把{\ADD{货物}}抛在海里。
\VS{19}到第三天,他们又亲手把船上的器具抛弃了。
\VS{20}太阳和星辰多日不显露,又有狂风大浪催逼,我们得救的指望就都绝了。
\par }{\PP \VS{21}众人多日没有吃什么,{\PN{保罗}}就出来站在他们中间,说:「众位,你们本该听我的话,不离开{\PN{克里特}},免得遭这样的伤损破坏。
\VS{22}现在我还劝你们放心,你们的性命一个也不失丧,惟独失丧这船。
\VS{23}因我所属所事奉的 神,他的使者昨夜站在我旁边,说:
\VS{24}『{\PN{保罗}},不要害怕,你必定站在凯撒面前,并且与你同船的人, 神都赐给你了。』
\VS{25}所以众位可以放心,我信 神他怎样对我说,事情也要怎样成就。
\VS{26}只是我们必要撞在一个岛上。」
\par }{\PP \VS{27}到了第十四天夜间,船在{\PN{亚得里亚海}}飘来飘去。约到半夜,水手以为渐近旱地,
\VS{28}就探深浅,探得有十二丈;稍往前行,又探深浅,探得有九丈。
\VS{29}恐怕撞在石头上,就从船尾抛下四个锚,盼望天亮。
\VS{30}水手想要逃出船去,把小船放在海里,假作要从船头抛锚的样子。
\VS{31}{\PN{保罗}}对百夫长和兵丁说:「这些人若不等在船上,你们必不能得救。」
\VS{32}于是兵丁砍断小船的绳子,由它飘去。
\par }{\PP \VS{33}天渐亮的时候,{\PN{保罗}}劝众人都吃饭,说:「你们悬望忍饿不吃什么,已经十四天了。
\VS{34}所以我劝你们吃饭,这是关乎你们救命的事;因为你们各人连一根头发也不至于损坏。」
\VS{35}{\PN{保罗}}说了这话,就拿着饼,在众人面前祝谢了 神,擘开吃。
\VS{36}于是他们都放下心,也就吃了。
\VS{37}我们在船上的共有二百七十六个人。
\VS{38}他们吃饱了,就把船上的麦子抛在海里,为要叫船轻一点。
\par }{\SH 船搁了浅
\par }{\PP \VS{39}到了天亮,他们不认识那地方,但见一个海湾,有岸可登,就商议能把船拢进去不能。
\VS{40}于是砍断缆索,弃锚在海里;同时也松开舵绳,拉起头篷,顺着风向岸行去。
\VS{41}但遇着两水夹流的地方,就把船搁了浅;船头胶住不动,船尾被浪的猛力冲坏。
\VS{42}兵丁的意思要把囚犯杀了,恐怕有洑水脱逃的。
\VS{43}但百夫长要救{\PN{保罗}},不准他们任意而行,就吩咐会洑水的,跳下水去先上岸;
\VS{44}其余的人可以用板子或船上的零碎东西上岸。这样,众人都得了救,上了岸。

\par }\Chap{28}{\SH 保罗在马耳他岛上
\par }{\PP \VerseOne{1}我们既已得救,才知道那岛名叫{\PN{马耳他}}。
\VS{2}土人看待我们,有非常的情分;因为当时下雨,天气又冷,就生火接待我们众人。
\VS{3}那时,{\PN{保罗}}拾起一捆柴,放在火上,有一条毒蛇,因为热了出来,咬住他的手。
\VS{4}土人看见那毒蛇悬在他手上,就彼此说:「这人必是个凶手,虽然从海里救上来,天理还不容他活着。」
\VS{5}{\PN{保罗}}竟把那毒蛇甩在火里,并没有受伤。
\VS{6}土人想他必要肿起来,或是忽然仆倒死了;看了多时,见他无害,就转念,说:「他是个神。」
\VS{7}离那地方不远,有田产是岛长{\PN{部百流}}的;他接纳我们,尽情款待三日。
\VS{8}当时,{\PN{部百流}}的父亲患热病和痢疾躺着。{\PN{保罗}}进去,{\ADD{为他}}祷告,按手在他身上,治好了他。
\VS{9}从此,岛上其余的病人也来,得了医治。
\VS{10}他们又多方地尊敬我们;到了开船的时候,也把我们所需用的送到船上。
\par }{\SH 保罗抵达罗马
\par }{\PP \VS{11}过了三个月,我们上了{\PN{亚历山大}}的船往前行;这船以「宙斯双子」为记,是在那海岛过了冬的。
\VS{12}到了{\PN{叙拉古}},我们停泊三日;
\VS{13}又从那里绕行,来到{\PN{利基翁}}。过了一天,起了南风,第二天就来到{\PN{部丢利}}。
\VS{14}在那里遇见弟兄们,请我们与他们同住了七天。这样,我们来到{\PN{罗马}}。
\VS{15}那里的弟兄们一听见我们的信息就出来,到{\PN{亚比乌}}市和{\PN{三馆}}地方迎接我们。{\PN{保罗}}见了他们,就感谢 神,放心壮胆。
\par }{\PP \VS{16}进了{\PN{罗马}}城,\FTNT{}{{\FR 28:16: }有古卷加:百夫长把众囚犯交给御营的统领,惟有}{\PN{保罗}}蒙准和一个看守他的兵另住在一处。
\par }{\SH 保罗在罗马传道
\par }{\PP \VS{17}过了三天,{\PN{保罗}}请{\PN{犹太}}人的首领来。他们来了,就对他们说:「弟兄们,我虽没有做什么事干犯本国的百姓和我们祖宗的规条,却被锁绑,从{\PN{耶路撒冷}}解在{\PN{罗马}}人的手里。
\VS{18}他们审问了我,就愿意释放我;因为在我身上,并没有该死的罪。
\VS{19}无奈{\PN{犹太}}人不服,我不得已,只好上告于凯撒,并非有什么事要控告我本国的百姓。
\VS{20}因此,我请你们来见面说话,我原为{\PN{以色列}}人所指望的,被这链子捆锁。」
\VS{21}他们说:「我们并没有接着从{\PN{犹太}}来论你的信,也没有弟兄到这里来报给我们说你有什么不好处。
\VS{22}但我们愿意听你的意见如何;因为这教门,我们晓得是到处被毁谤的。」
\par }{\PP \VS{23}他们和{\PN{保罗}}约定了日子,就有许多人到他的寓处来。{\PN{保罗}}从早到晚,对他们讲论这事,证明 神国{\ADD{的道}},引{\PN{摩西}}的律法和先知的书,以耶稣的事劝勉他们。
\VS{24}他所说的话,有信的,有不信的。
\VS{25}他们彼此不合,就散了;未散以先,{\PN{保罗}}说了一句话,说:「圣灵借先知{\PN{以赛亚}}向你们祖宗所说的话是不错的。
\VS{26}他说:
\par }{\Q 你去告诉这百姓说:
\par }{\Q 你们听是要听见,却不明白;
\par }{\Q 看是要看见,却不晓得;
\par }{\Q \VS{27}因为这百姓油蒙了心,
\par }{\Q 耳朵发沉,
\par }{\Q 眼睛闭着;
\par }{\Q 恐怕眼睛看见,
\par }{\Q 耳朵听见,
\par }{\Q 心里明白,回转过来,
\par }{\Q 我就医治他们。
\par }{\MM \VS{28}所以你们当知道, 神这救恩,如今传给外邦人,他们也必听受。」\FTNT{}{{\FR 28:28: }
有古卷加:29
{\PN{保罗}}说了这话,{\PN{犹太}}人议论纷纷地就走了。}
\par }{\PP \VS{30}{\PN{保罗}}在自己所租的房子里住了足足两年。凡来见他的人,他全都接待,
\VS{31}放胆传讲 神国{\ADD{的道}},将主耶稣基督的事教导人,并没有人禁止。
\par }