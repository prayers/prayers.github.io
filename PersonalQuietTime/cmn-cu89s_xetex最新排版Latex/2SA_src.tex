\NormalFont\ShortTitle{撒母耳记下}
{\MT 撒母耳记下

\par }\ChapOne{1}{\SH 大卫听到扫罗的死讯
\par }{\PP \VerseOne{1}{\PN{扫罗}}死后,{\PN{大卫}}击杀{\PN{亚玛力}}人回来,在{\PN{洗革拉}}住了两天。
\VS{2}第三天,有一人从{\PN{扫罗}}的营里出来,衣服撕裂,头蒙灰尘,到{\PN{大卫}}面前伏地叩拜。
\VS{3}{\PN{大卫}}问他说:「你从哪里来?」他说:「我从{\PN{以色列}}的营里逃来。」
\VS{4}{\PN{大卫}}又问他说:「事情怎样?请你告诉我。」他回答说:「百姓从阵上逃跑,也有许多人仆倒死亡;{\PN{扫罗}}和他儿子{\PN{约拿单}}也死了。」
\VS{5}{\PN{大卫}}问报信的少年人说:「你怎么知道{\PN{扫罗}}和他儿子{\PN{约拿单}}死了呢?」
\VS{6}报信的少年人说:「我偶然到{\PN{基利波山}},看见{\PN{扫罗}}伏在自己枪上,有战车、马兵紧紧地追他。
\VS{7}他回头看见我,就呼叫我。我说:『我在这里。』
\VS{8}他问我说:『你是什么人?』我说:『我是{\PN{亚玛力}}人。』
\VS{9}他说:『请你来,将我杀死;因为痛苦抓住我,我的生命尚存。』
\VS{10}我准知他仆倒必不能活,就去将他杀死,把他头上的冠冕、臂上的镯子拿到我主这里。」
\par }{\PP \VS{11}{\PN{大卫}}就撕裂衣服,跟随他的人也是如此,
\VS{12}而且悲哀哭号,禁食到晚上,是因{\PN{扫罗}}和他儿子{\PN{约拿单}},并耶和华的民{\PN{以色列}}家的人,倒在刀下。
\VS{13}{\PN{大卫}}问报信的少年人说:「你是哪里的人?」他说:「我是{\PN{亚玛力}}客人的儿子。」
\VS{14}{\PN{大卫}}说:「你伸手杀害耶和华的受膏者,怎么不畏惧呢?」
\VS{15}{\PN{大卫}}叫了一个少年人来,说:「你去杀他吧!」
\VS{16}{\PN{大卫}}对他说:「你流人血的罪归到自己的头上,因为你亲口作见证说:『我杀了耶和华的受膏者。』」少年人就把他杀了。
\par }{\SH 大卫为扫罗和约拿单作哀歌
\par }{\PP \VS{17}{\PN{大卫}}作哀歌,吊{\PN{扫罗}}和他儿子{\PN{约拿单}},
\VS{18}且吩咐将这歌教导{\PN{犹大}}人。这歌名叫「弓歌」,写在{\PN{雅煞珥}}书上。
\par }{\Q \VS{19}{\ADD{歌中说}}:{\PN{以色列}}啊,
\par }{\Q 你尊荣者在山上被杀!
\par }{\Q 大英雄何竟死亡!
\par }{\Q \VS{20}不要在{\PN{迦特}}报告;
\par }{\Q 不要在{\PN{亚实基伦}}街上传扬;
\par }{\Q 免得{\PN{非利士}}的女子欢乐;
\par }{\Q 免得未受割礼之人的女子矜夸。
\par }{\Q \VS{21}{\PN{基利波山}}哪,愿你那里没有雨露!
\par }{\Q 愿你田地无土产可作供物!
\par }{\Q 因为英雄的盾牌在那里被污丢弃;
\par }{\Q {\PN{扫罗}}的盾牌仿佛未曾抹油。
\par }{\BB \par }{\Q \VS{22}{\PN{约拿单}}的弓箭非流敌人的血不退缩;
\par }{\Q {\PN{扫罗}}的刀剑非剖勇士的油不收回。
\par }{\BB \par }{\Q \VS{23}{\PN{扫罗}}和{\PN{约拿单}}—
\par }{\Q 活时相悦相爱,死时也不分离
\par }{\Q —他们比鹰更快,比狮子还强。
\par }{\BB \par }{\Q \VS{24}{\PN{以色列}}的女子啊,当为{\PN{扫罗}}哭号!
\par }{\Q 他曾使你们穿朱红色的美衣,
\par }{\Q 使你们衣服有黄金的妆饰。
\par }{\BB \par }{\Q \VS{25}英雄何竟在阵上仆倒!
\par }{\Q {\PN{约拿单}}何竟在山上被杀!
\par }{\Q \VS{26}我兄{\PN{约拿单}}哪,我为你悲伤!
\par }{\Q 我甚喜悦你!
\par }{\Q 你向我发的爱情奇妙非常,
\par }{\Q 过于妇女的爱情。
\par }{\BB \par }{\Q \VS{27}英雄何竟仆倒!
\par }{\Q 战具何竟灭没!

\par }\Chap{2}{\SH 大卫受膏作犹大王
\par }{\PP \VerseOne{1}此后,{\PN{大卫}}问耶和华说:「我上{\PN{犹大}}的一个城去可以吗?」耶和华说:「可以。」{\PN{大卫}}说:「我上哪一个城去呢?」耶和华说:「上{\PN{希伯
}}去。」
\VS{2}于是{\PN{大卫}}和他的两个妻—一个是{\PN{耶斯列}}人{\PN{亚希暖}},一个是作过{\PN{迦密}}人{\PN{拿八}}妻的{\PN{亚比该}}—都上那里去了。
\VS{3}{\PN{大卫}}也将跟随他的人和他们各人的眷属一同带上去,住在{\PN{希伯
}}的城邑中。
\VS{4}{\PN{犹大}}人来到{\PN{希伯
}},在那里膏{\PN{大卫}}作{\PN{犹大}}家的王。
\par }{\PP 有人告诉{\PN{大卫}}说:「葬埋{\PN{扫罗}}的是{\PN{基列·雅比}}人。」
\VS{5}{\PN{大卫}}就差人去见{\PN{基列·雅比}}人,对他们说:「你们厚待你们的主—{\PN{扫罗}},将他葬埋。愿耶和华赐福与你们!
\VS{6}你们既行了这事,愿耶和华以慈爱诚实待你们,我也要为此厚待你们。
\VS{7}现在你们的主—{\PN{扫罗}}死了,{\PN{犹大}}家已经膏我作他们的王,所以你们要刚强奋勇。」
\par }{\SH 伊施波设作以色列王
\par }{\PP \VS{8}{\PN{扫罗}}的元帅{\PN{尼珥}}的儿子{\PN{押尼珥}},曾将{\PN{扫罗}}的儿子{\PN{伊施波设}}带过{\ADD{河}},到{\PN{玛哈念}},
\VS{9}立他作王,治理{\PN{基列}}、{\PN{亚书利}}、{\PN{耶斯列}}、{\PN{以法莲}}、{\PN{便雅悯}},和{\PN{以色列}}众人。
\VS{10}{\PN{扫罗}}的儿子{\PN{伊施波设}}登基的时候年四十岁,作{\PN{以色列}}王二年;惟独{\PN{犹大}}家归从{\PN{大卫}}。
\VS{11}{\PN{大卫}}在{\PN{希伯
}}作{\PN{犹大}}家的王,共七年零六个月。
\par }{\SH 扫罗家和大卫家争战
\par }{\PP \VS{12}{\PN{尼珥}}的儿子{\PN{押尼珥}}和{\PN{扫罗}}的儿子{\PN{伊施波设}}的仆人从{\PN{玛哈念}}出来,往{\PN{基遍}}去。
\VS{13}{\PN{洗鲁雅}}的儿子{\PN{约押}}和{\PN{大卫}}的仆人也出来,在{\PN{基遍}}池旁与他们相遇;一班坐在池这边,一班坐在池那边。
\VS{14}{\PN{押尼珥}}对{\PN{约押}}说:「让少年人起来,在我们面前戏耍吧!」{\PN{约押}}说:「可以。」
\VS{15}就按着定数起来:属{\PN{扫罗}}儿子{\PN{伊施波设}}的{\PN{便雅悯}}人过去十二名,{\PN{大卫}}的仆人也过去十二名,
\VS{16}彼此揪头,用刀刺肋,一同仆倒。所以,那地叫做{\PN{希利甲·哈素林}},就在{\PN{基遍}}。
\VS{17}那日的战事凶猛,{\PN{押尼珥}}和{\PN{以色列}}人败在{\PN{大卫}}的仆人面前。
\VS{18}在那里有{\PN{洗鲁雅}}的三个儿子:{\PN{约押}}、{\PN{亚比筛}}、{\PN{亚撒黑}}。{\PN{亚撒黑}}脚快如野鹿一般;
\VS{19}{\PN{亚撒黑}}追赶{\PN{押尼珥}},直追赶他不偏左右。
\VS{20}{\PN{押尼珥}}回头说:「你是{\PN{亚撒黑}}吗?」回答说:「是。」
\VS{21}{\PN{押尼珥}}对他说:「你或转向左转向右,拿住一个少年人,剥去他的战衣。」{\PN{亚撒黑}}却不肯转开不追赶他。
\VS{22}{\PN{押尼珥}}又对{\PN{亚撒黑}}说:「你转开不追赶我吧!我何必杀你呢?若杀你,有什么脸见你哥哥{\PN{约押}}呢?」
\VS{23}{\PN{亚撒黑}}仍不肯转开。故此,{\PN{押尼珥}}就用枪 刺入他的肚腹,甚至枪从背后透出,{\PN{亚撒黑}}就在那里仆倒而死。众人赶到{\PN{亚撒黑}}仆倒而死的地方,就都站住。
\par }{\PP \VS{24}{\PN{约押}}和{\PN{亚比筛}}追赶{\PN{押尼珥}},日落的时候,到了通{\PN{基遍}}旷野的路旁,{\PN{基亚}}对面的{\PN{亚玛山}}。
\VS{25}{\PN{便雅悯}}人聚集,跟随{\PN{押尼珥}}站在一个山顶上。
\VS{26}{\PN{押尼珥}}呼叫{\PN{约押}}说:「刀剑岂可永远杀人吗?你岂不知终久必有苦楚吗?你要等何时才叫百姓回去、不追赶弟兄呢?」
\VS{27}{\PN{约押}}说:「我指着永生的 神起誓:你若不说{\ADD{戏耍的}}那句话,今日早晨百姓就回去,不追赶弟兄了。」
\VS{28}于是{\PN{约押}}吹角,众民就站住,不再追赶{\PN{以色列}}人,也不再打仗了。
\par }{\PP \VS{29}{\PN{押尼珥}}和跟随他的人整夜经过{\PN{亚拉巴}},过{\PN{约旦河}},走过{\PN{毕伦}},到了{\PN{玛哈念}}。
\par }{\PP \VS{30}{\PN{约押}}追赶{\PN{押尼珥}}回来,聚集众民,见{\PN{大卫}}的仆人中缺少了十九个人和{\PN{亚撒黑}}。
\VS{31}但{\PN{大卫}}的仆人杀了{\PN{便雅悯}}人和跟随{\PN{押尼珥}}的人,共三百六十名。
\VS{32}众人将{\PN{亚撒黑}}送到{\PN{伯利恒}},葬在他父亲的坟墓里。{\PN{约押}}和跟随他的人走了一夜,天亮的时候到了{\PN{希伯
}}。

\par }\Chap{3}{\PP \VerseOne{1}{\PN{扫罗}}家和{\PN{大卫}}家争战许久。{\PN{大卫}}家日见强盛;{\PN{扫罗}}家日见衰弱。
\par }{\SH 大卫的儿子们
\par }{\PP \VS{2}{\PN{大卫}}在{\PN{希伯
}}得了几个儿子:长子{\PN{暗嫩}}是{\PN{耶斯列}}人{\PN{亚希暖}}所生的;
\VS{3}次子{\PN{基利押}}\FTNT{}{{\FR 3:3: }在历代上三章一节作但以利}是作过{\PN{迦密}}人{\PN{拿八}}的妻{\PN{亚比该}}所生的;三子{\PN{押沙龙}}是{\PN{基述}}王{\PN{达买}}的女儿{\PN{玛迦}}所生的;
\VS{4}四子{\PN{亚多尼雅}}是{\PN{哈及}}所生的;五子{\PN{示法提雅}}是{\PN{亚比她}}所生的;
\VS{5}六子{\PN{以特念}}是{\PN{大卫}}的妻{\PN{以格拉}}所生的。{\PN{大卫}}这六个儿子都是在{\PN{希伯
}}生的。
\par }{\SH 押尼珥投效大卫
\par }{\PP \VS{6}{\PN{扫罗}}家和{\PN{大卫}}家争战的时候,{\PN{押尼珥}}在{\PN{扫罗}}家大有权势。
\VS{7}{\PN{扫罗}}有一妃嫔,名叫{\PN{利斯巴}},是{\PN{爱亚}}的女儿。一日,{\PN{伊施波设}}对{\PN{押尼珥}}说:「你为什么与我父的妃嫔同房呢?」
\VS{8}{\PN{押尼珥}}因{\PN{伊施波设}}的话就甚发怒,说:「我岂是{\PN{犹大}}的狗头呢?我恩待你父{\PN{扫罗}}的家和他的弟兄、朋友,不将你交在{\PN{大卫}}手里,今日你竟为这妇人责备我吗?
\VS{9-10}我若不照着耶和华起誓应许{\PN{大卫}}的话行,废去{\PN{扫罗}}的位,建立{\PN{大卫}}的位,使他治理{\PN{以色列}}和{\PN{犹大}},从{\PN{但}}直到{\PN{别是巴}},愿 神重重地降罚与我!」
\VS{11}{\PN{伊施波设}}惧怕{\PN{押尼珥}},不敢回答一句。
\par }{\PP \VS{12}{\PN{押尼珥}}打发人去见{\PN{大卫}},替他说:「这国归谁呢?」又说:「你与我立约,我必帮助你,使{\PN{以色列}}人都归服你。」
\VS{13}{\PN{大卫}}说:「好!我与你立约。但有一件,你来见我面的时候,若不将{\PN{扫罗}}的女儿{\PN{米甲}}带来,必不得见我的面。」
\VS{14}{\PN{大卫}}就打发人去见{\PN{扫罗}}的儿子{\PN{伊施波设}},说:「你要将我的妻{\PN{米甲}}归还我;她是我从前用一百{\PN{非利士}}人的阳皮所聘定的。」
\VS{15}{\PN{伊施波设}}就打发人去,将{\PN{米甲}}从{\PN{拉亿}}的儿子、她丈夫{\PN{帕铁}}那里接回来。
\VS{16}{\PN{米甲}}的丈夫跟着她,一面走一面哭,直跟到{\PN{巴户琳}}。{\PN{押尼珥}}说:「你回去吧!」{\PN{帕铁}}就回去了。
\par }{\PP \VS{17}{\PN{押尼珥}}对{\PN{以色列}}长老说:「从前你们愿意{\PN{大卫}}作王治理你们,
\VS{18}现在你们可以照心愿而行。因为耶和华曾论到{\PN{大卫}}说:『我必借我仆人{\PN{大卫}}的手,救我民{\PN{以色列}}脱离{\PN{非利士}}人和众仇敌的手。』」
\VS{19}{\PN{押尼珥}}也用这话说给{\PN{便雅悯}}人听,又到{\PN{希伯
}},将{\PN{以色列}}人和{\PN{便雅悯}}全家一切所喜悦的事说给{\PN{大卫}}听。
\VS{20}{\PN{押尼珥}}带着二十个人来到{\PN{希伯
}}见{\PN{大卫}},{\PN{大卫}}就为{\PN{押尼珥}}和他带来的人设摆筵席。
\VS{21}{\PN{押尼珥}}对{\PN{大卫}}说:「我要起身去招聚{\PN{以色列}}众人来见我主我王,与你立约,你就可以照着心愿作王。」于是{\PN{大卫}}送{\PN{押尼珥}}去,{\PN{押尼珥}}就平平安安地去了。
\par }{\SH 押尼珥被杀
\par }{\PP \VS{22}{\PN{约押}}和{\PN{大卫}}的仆人攻击敌军,带回许多的掠物。那时{\PN{押尼珥}}不在{\PN{希伯
}}{\PN{大卫}}那里,因{\PN{大卫}}已经送他去,他也平平安安地去了。
\VS{23}{\PN{约押}}和跟随他的全军到了,就有人告诉{\PN{约押}}说:「{\PN{尼珥}}的儿子{\PN{押尼珥}}来见王,王送他去,他也平平安安地去了。」
\VS{24}{\PN{约押}}去见王说:「你这是做什么呢?{\PN{押尼珥}}来见你,你为何送他去,他就踪影不见了呢?
\VS{25}你当晓得,{\PN{尼珥}}的儿子{\PN{押尼珥}}来是要诓哄你,要知道你的出入和你一切所行的事。」
\par }{\PP \VS{26}{\PN{约押}}从{\PN{大卫}}那里出来,就打发人去追赶{\PN{押尼珥}},在{\PN{西拉井}}追上他,将他带回来,{\PN{大卫}}却不知道。
\VS{27}{\PN{押尼珥}}回到{\PN{希伯
}},{\PN{约押}}领他到城门的瓮洞,假作要与他说机密话,就在那里刺透他的肚腹,他便死了。这是报杀他兄弟{\PN{亚撒黑}}的仇。
\VS{28}{\PN{大卫}}听见了,就说:「流{\PN{尼珥}}的儿子{\PN{押尼珥}}的血,这罪在耶和华面前必永不归我和我的国。
\VS{29}愿流他血的罪归到{\PN{约押}}头上和他父的全家;又愿{\PN{约押}}家不断有患漏症的,长大麻风的,架拐而行的,被刀杀死的,缺乏饮食的。」
\VS{30}{\PN{约押}}和他兄弟{\PN{亚比筛}}杀了{\PN{押尼珥}},是因{\PN{押尼珥}}在{\PN{基遍}}争战的时候杀了他们的兄弟{\PN{亚撒黑}}。
\par }{\SH 安葬押尼珥
\par }{\PP \VS{31}{\PN{大卫}}吩咐{\PN{约押}}和跟随他的众人说:「你们当撕裂衣服,腰束麻布,在{\PN{押尼珥}}{\ADD{棺}}前哀哭。」{\PN{大卫}}王也跟在棺后。
\VS{32}他们将{\PN{押尼珥}}葬在{\PN{希伯
}}。王在{\PN{押尼珥}}的墓旁放声而哭,众民也都哭了。
\VS{33}王为{\PN{押尼珥}}举哀,说:
\par }{\Q {\PN{押尼珥}}何竟像愚顽人死呢?
\par }{\Q \VS{34}你手未曾捆绑,脚未曾锁住。
\par }{\Q 你死,如人死在罪孽之辈手下一样。
\par }{\MM 于是众民又为{\PN{押尼珥}}哀哭。
\VS{35}日头未落的时候,众民来劝{\PN{大卫}}吃饭,但{\PN{大卫}}起誓说:「我若在日头未落以前吃饭,或吃别物,愿 神重重地降罚与我!」
\VS{36}众民知道了就都喜悦。凡王所行的,众民无不喜悦。
\VS{37}那日,{\PN{以色列}}众民才知道杀{\PN{尼珥}}的儿子{\PN{押尼珥}}并非出于王意。
\VS{38}王对臣仆说:「你们岂不知今日{\PN{以色列}}人中死了一个作元帅的大丈夫吗?
\VS{39}我虽然受膏为王,今日还是软弱;这{\PN{洗鲁雅}}的两个儿子比我刚强。愿耶和华照着恶人所行的恶报应他。」

\par }\Chap{4}{\SH 伊施波设被杀
\par }{\PP \VerseOne{1}{\PN{扫罗}}的儿子{\PN{伊施波设}}听见{\PN{押尼珥}}死在{\PN{希伯
}},手就发软;{\PN{以色列}}众人也都惊惶。
\VS{2}{\PN{扫罗}}的儿子{\PN{伊施波设}}有两个军长,一名{\PN{巴拿}},一名{\PN{利甲}},是{\PN{便雅悯}}支派、{\PN{比录}}人{\PN{临门}}的儿子。{\PN{比录}}也属{\PN{便雅悯}}。
\VS{3}{\PN{比录}}人早先逃到{\PN{基他音}},在那里寄居,直到今日。
\par }{\PP \VS{4}{\PN{扫罗}}的儿子{\PN{约拿单}}有一个儿子名叫{\PN{米非波设}},是瘸腿的。{\PN{扫罗}}和{\PN{约拿单}}{\ADD{死亡}}的消息从{\PN{耶斯列}}传到的时候,他才五岁。他乳母抱着他逃跑;因为跑得太急,孩子掉在地上,腿就瘸了。
\par }{\PP \VS{5}一日,{\PN{比录}}人{\PN{临门}}的两个儿子{\PN{利甲}}和{\PN{巴拿}}出去,约在午热的时候到了{\PN{伊施波设}}的家;{\PN{伊施波设}}正睡午觉。
\VS{6}他们进了房子,假作要取麦子,就刺透{\PN{伊施波设}}的肚腹,逃跑了。
\VS{7}他们进房子的时候,{\PN{伊施波设}}正在卧房里躺在床上,他们将他杀死,割了他的首级,拿着首级在{\PN{亚拉巴}}走了一夜,
\VS{8}将{\PN{伊施波设}}的首级拿到{\PN{希伯
}}见{\PN{大卫}}王,说:「王的仇敌{\PN{扫罗}}曾寻索王的性命。看哪,这是他儿子{\PN{伊施波设}}的首级;耶和华今日为我主我王在{\PN{扫罗}}和他后裔的身上报了仇。」
\VS{9}{\PN{大卫}}对{\PN{比录}}人{\PN{临门}}的儿子{\PN{利甲}}和他兄弟{\PN{巴拿}}说:「我指着救我性命脱离一切苦难、永生的耶和华起誓:
\VS{10}从前有人报告我说,{\PN{扫罗}}死了,他自以为报好消息;我就拿住他,将他杀在{\PN{洗革拉}},这就作了他报消息的赏赐。
\VS{11}何况恶人将义人杀在他的床上,我岂不向你们讨流他血的罪、从世上除灭你们呢?」
\VS{12}于是{\PN{大卫}}吩咐少年人将他们杀了,砍断他们的手脚,挂在{\PN{希伯
}}的池旁,却将{\PN{伊施波设}}的首级葬在{\PN{希伯
}}、{\PN{押尼珥}}的坟墓里。

\par }\Chap{5}{\SH 大卫作犹大和以色列王
\par }{\R (代上11·1—9;14·1—7)
\par }{\PP \VerseOne{1}{\PN{以色列}}众支派来到{\PN{希伯
}}见{\PN{大卫}},说:「我们原是你的骨肉。
\VS{2}从前{\PN{扫罗}}作我们王的时候,率领{\PN{以色列}}人出入的是你;耶和华也曾应许你说:『你必牧养我的民{\PN{以色列}},作{\PN{以色列}}的君。』」
\VS{3}于是{\PN{以色列}}的长老都来到{\PN{希伯
}}见{\PN{大卫}}王,{\PN{大卫}}在{\PN{希伯
}}耶和华面前与他们立约,他们就膏{\PN{大卫}}作{\PN{以色列}}的王。
\VS{4}{\PN{大卫}}登基的时候年三十岁,在位四十年;
\VS{5}在{\PN{希伯
}}作{\PN{犹大}}王七年零六个月,在{\PN{耶路撒冷}}作{\PN{以色列}}和{\PN{犹大}}王三十三年。
\par }{\PP \VS{6}{\PN{大卫}}和跟随他的人到了{\PN{耶路撒冷}},要攻打住那地方的{\PN{耶布斯}}人。{\PN{耶布斯}}人对{\PN{大卫}}说:「你若不赶出瞎子、瘸子,必不能进这地方」;心里想{\PN{大卫}}决不能进去。
\VS{7}然而{\PN{大卫}}攻取{\PN{锡安}}的保障,就是{\PN{大卫}}的城。
\par }{\PP \VS{8}当日,{\PN{大卫}}说:「谁攻打{\PN{耶布斯}}人,当上水沟{\ADD{攻打}}我心里所恨恶的瘸子、瞎子。」从此有俗语说:「在那里有瞎子、瘸子,他不能进屋去。」
\par }{\PP \VS{9}{\PN{大卫}}住在保障里,给保障起名叫{\PN{大卫城}}。{\PN{大卫}}又从{\PN{米罗}}以里,周围筑墙。
\VS{10}{\PN{大卫}}日见强盛,因为耶和华—万军之 神与他同在。
\par }{\PP \VS{11}{\PN{泰尔}}王{\PN{希兰}}将香柏木运到{\PN{大卫}}那里,又差遣使者和木匠、石匠给{\PN{大卫}}建造宫殿。
\VS{12}{\PN{大卫}}就知道耶和华坚立他作{\PN{以色列}}王,又为自己的民{\PN{以色列}}使他的国兴旺。
\par }{\PP \VS{13}{\PN{大卫}}离开{\PN{希伯
}}之后,在{\PN{耶路撒冷}}又立后妃,又生儿女。
\VS{14}在{\PN{耶路撒冷}}所生的儿子是{\PN{沙母亚}}、{\PN{朔罢}}、{\PN{拿单}}、{\PN{所罗门}}、
\VS{15}{\PN{益辖}}、{\PN{以利书亚}}、{\PN{尼斐}}、{\PN{雅非亚}}、
\VS{16}{\PN{以利沙玛}}、{\PN{以利雅大}}、{\PN{以利法列}}。
\par }{\SH 战胜非利士人
\par }{\R (代上14·8—17)
\par }{\PP \VS{17}{\PN{非利士}}人听见人膏{\PN{大卫}}作{\PN{以色列}}王,{\PN{非利士}}众人就上来寻索{\PN{大卫}};{\PN{大卫}}听见,就下到保障。
\VS{18}{\PN{非利士}}人来了,布散在{\PN{利乏音谷}}。
\VS{19}{\PN{大卫}}求问耶和华说:「我可以上去攻打{\PN{非利士}}人吗?你将他们交在我手里吗?」耶和华说:「你可以上去,我必将{\PN{非利士}}人交在你手里。」
\par }{\PP \VS{20}{\PN{大卫}}来到{\PN{巴力·毗拉心}},在那里击杀{\PN{非利士}}人,说:「耶和华在我面前冲破敌人,如同水冲去一般。」因此称那地方为{\PN{巴力·毗拉心}}。
\VS{21}{\PN{非利士}}人将偶像撇在那里,{\PN{大卫}}和跟随他的人拿去了。
\par }{\PP \VS{22}{\PN{非利士}}人又上来,布散在{\PN{利乏音谷}}。
\VS{23}{\PN{大卫}}求问耶和华;耶和华说:「不要一直地上去,要转到他们后头,从桑林对面攻打他们。
\VS{24}你听见桑树梢上有脚步的声音,就要急速前去,因为那时耶和华已经在你前头去攻打{\PN{非利士}}人的军队。」
\VS{25}{\PN{大卫}}就遵着耶和华所吩咐的去行,攻打{\PN{非利士}}人,从{\PN{迦巴}}直到{\PN{基色}}。

\par }\Chap{6}{\SH 运约柜到耶路撒冷
\par }{\R (代上13·1—14;15·25—16·6,43)
\par }{\PP \VerseOne{1}{\PN{大卫}}又聚集{\PN{以色列}}中所有挑选的人三万。
\VS{2}{\PN{大卫}}起身,率领跟随他的众人前往,要从{\PN{巴拉·犹大}}将 神的{\ADD{约}}柜运来;这{\ADD{约}}柜就是坐在二基路伯上万军之耶和华留名的{\ADD{约}}柜。
\VS{3}他们将 神的{\ADD{约}}柜从冈上{\PN{亚比拿达}}的家里抬出来,放在新车上;{\PN{亚比拿达}}的两个儿子{\PN{乌撒}}和{\PN{亚希约}}赶这新车。
\VS{4}他们将 神的{\ADD{约}}柜从冈上{\PN{亚比拿达}}家里抬出来的时候,{\PN{亚希约}}在柜前行走。
\VS{5}{\PN{大卫}}和{\PN{以色列}}的全家在耶和华面前,用松木制造的各样{\ADD{乐器}}和琴、瑟、鼓、钹、锣,作乐跳舞。
\par }{\PP \VS{6}到了{\PN{拿艮}}的禾场,因为牛失前蹄\FTNT{}{{\FR 6:6: }或译:惊跳},{\PN{乌撒}}就伸{\ADD{手}}扶住 神的{\ADD{约}}柜。
\VS{7}神耶和华向{\PN{乌撒}}发怒,因这错误击杀他,他就死在 神的{\ADD{约}}柜旁。
\VS{8}{\PN{大卫}}因耶和华击杀\FTNT{}{{\FR 6:8: }原文是闯杀}{\PN{乌撒}},心里愁烦,就称那地方为{\PN{毗列斯·乌撒}},直到今日。
\par }{\PP \VS{9}那日,{\PN{大卫}}惧怕耶和华,说:「耶和华的{\ADD{约}}柜怎可运到我这里来?」
\VS{10}于是{\PN{大卫}}不肯将耶和华的{\ADD{约}}柜运进{\PN{大卫}}的城,却运到{\PN{迦特}}人{\PN{俄别·以东}}的家中。
\VS{11}耶和华的{\ADD{约}}柜在{\PN{迦特}}人{\PN{俄别·以东}}家中三个月;耶和华赐福给{\PN{俄别·以东}}和他的全家。
\par }{\PP \VS{12}有人告诉{\PN{大卫}}王说:「耶和华因为{\ADD{约}}柜赐福给{\PN{俄别·以东}}的家和一切属他的。」{\PN{大卫}}就去,欢欢喜喜地将 神的{\ADD{约}}柜从{\PN{俄别·以东}}家中抬到{\PN{大卫}}的城里。
\VS{13}抬耶和华{\ADD{约}}柜的人走了六步,{\PN{大卫}}就献牛与肥羊为祭。
\VS{14}{\PN{大卫}}穿着细麻布的以弗得,在耶和华面前极力跳舞。
\VS{15}这样,{\PN{大卫}}和{\PN{以色列}}的全家欢呼吹角,将耶和华的{\ADD{约}}柜抬上来。
\par }{\PP \VS{16}耶和华的{\ADD{约}}柜进了{\PN{大卫城}}的时候,{\PN{扫罗}}的女儿{\PN{米甲}}从窗户里观看,见{\PN{大卫}}王在耶和华面前踊跃跳舞,心里就轻视他。
\VS{17}众人将耶和华的{\ADD{约}}柜请进去,安放在所预备的地方,就是在{\PN{大卫}}所搭的帐幕里。{\PN{大卫}}在耶和华面前献燔祭和平安祭。
\VS{18}{\PN{大卫}}献完了燔祭和平安祭,就奉万军之耶和华的名给民祝福,
\VS{19}并且分给{\PN{以色列}}众人,无论男女,每人一个饼,一块{\ADD{肉}},一个葡萄饼;众人就各回各家去了。
\VS{20}{\PN{大卫}}回家要给眷属祝福;{\PN{扫罗}}的女儿{\PN{米甲}}出来迎接他,说:「{\PN{以色列}}王今日在臣仆的婢女眼前露体,如同一个轻贱人无耻露体一样,有好大的荣耀啊!」
\VS{21}{\PN{大卫}}对{\PN{米甲}}说:「这是在耶和华面前;耶和华已拣选我,废了你父和你父的全家,立我作耶和华民{\PN{以色列}}的君,所以我必在耶和华面前跳舞。
\VS{22}我也必更加卑微,自己看为轻贱。你所说的那些婢女,她们倒要尊敬我。」
\VS{23}{\PN{扫罗}}的女儿{\PN{米甲}},直到死日,没有生养儿女。

\par }\Chap{7}{\SH 拿单传信息给大卫
\par }{\R (代上17·1—15)
\par }{\PP \VerseOne{1}王住在自己宫中,耶和华使他安靖,不被四围的仇敌扰乱。
\VS{2}那时,王对先知{\PN{拿单}}说:「看哪,我住在香柏木的宫中, 神的{\ADD{约}}柜反在幔子里。」
\VS{3}{\PN{拿单}}对王说:「你可以照你的心意而行,因为耶和华与你同在。」
\VS{4}当夜,耶和华的话临到{\PN{拿单}}说:
\VS{5}「你去告诉我仆人{\PN{大卫}},说耶和华如此说:『你岂可建造殿宇给我居住呢?
\VS{6}自从我领{\PN{以色列}}人出{\PN{埃及}}直到今日,我未曾住过殿宇,常在会幕和帐幕中行走。
\VS{7}凡我同{\PN{以色列}}人所走的地方,我何曾向{\PN{以色列}}一支派{\ADD{的士师}},就是我吩咐牧养我民{\PN{以色列}}的说:你们为何不给我建造香柏木的殿宇呢?』
\par }{\PP \VS{8}「现在,你要告诉我仆人{\PN{大卫}},说万军之耶和华如此说:『我从羊圈中将你召来,叫你不再跟从羊群,立你作我民{\PN{以色列}}的君。
\VS{9}你无论往哪里去,我常与你同在,剪除你的一切仇敌。我必使你得大名,好像世上大大有名的人一样。
\VS{10}我必为我民{\PN{以色列}}选定一个地方,栽培他们,使他们住自己的地方,不再迁移;凶恶之子也不像从前扰害他们,
\VS{11}并不像我命士师治理我民{\PN{以色列}}的时候一样。我必使你安靖,不被一切仇敌扰乱,并且我—耶和华应许你,必为你建立家室。
\VS{12}你寿数满足、与你列祖同睡的时候,我必使你的后裔接续你的位;我也必坚定他的国。
\VS{13}他必为我的名建造殿宇;我必坚定他的国位,直到永远。
\VS{14}我要作他的父,他要作我的子;他若犯了罪,我必用人的杖责打他,用人的鞭责罚他。
\VS{15}但我的慈爱仍不离开他,像离开在你面前所废弃的{\PN{扫罗}}一样。
\VS{16}你的家和你的国必在我\FTNT{}{{\FR 7:16: }原文是你}面前永远坚立。你的国位也必坚定,直到永远。』」
\par }{\PP \VS{17}{\PN{拿单}}就按这一切话,照这默示,告诉{\PN{大卫}}。
\par }{\SH 大卫的祈祷和感恩
\par }{\R (代上17·16—27)
\par }{\PP \VS{18}于是{\PN{大卫}}王进去,坐在耶和华面前,说:「主耶和华啊,我是谁?我的家算什么?你竟使我到这地步呢?
\VS{19}主耶和华啊,这在你眼中还看为小,又应许你仆人的家至于久远。主耶和华啊,这岂是人所常遇的事吗?
\VS{20}主耶和华啊,我还有何言可以对你说呢?因为你知道你的仆人。
\VS{21}你行这大事使仆人知道,是因你所应许的话,也是照你的心意。
\VS{22}主耶和华啊,你本为大,照我们耳中听见,没有可比你的;除你以外再无 神。
\VS{23}世上有何民能比你的民{\PN{以色列}}呢?你从{\PN{埃及}}救赎他们作自己的子民,又在你赎出来的民面前行大而可畏的事,{\ADD{驱逐}}列邦人和他们的神,显出你的大名。
\VS{24}你曾坚立你的民{\PN{以色列}}作你的子民,直到永远;你—耶和华也作了他们的 神。
\par }{\PP \VS{25}「耶和华 神啊,你所应许仆人和仆人家的话,求你坚定,直到永远;照你所说的而行。
\VS{26}愿人永远尊你的名为大,说:『万军之耶和华是治理{\PN{以色列}}的 神。』这样,你仆人{\PN{大卫}}的家必在你面前坚立。
\VS{27}万军之耶和华—{\PN{以色列}}的 神啊,因你启示你的仆人说:『我必为你建立家室』,所以仆人大胆向你如此祈祷。
\par }{\PP \VS{28}「主耶和华啊,惟有你是 神。你的话是真实的;你也应许将这福气赐给仆人。
\VS{29}现在求你赐福与仆人的家,可以永存在你面前。主耶和华啊,这是你所应许的。愿你永远赐福与仆人的家!」

\par }\Chap{8}{\SH 耶和华使大卫得胜
\par }{\R (代上18·1—17)
\par }{\PP \VerseOne{1}此后,{\PN{大卫}}攻打{\PN{非利士}}人,把他们治服,从他们手下夺取了京城的权柄\FTNT{}{{\FR 8:1: }原文是母城的嚼环};
\VS{2}又攻打{\PN{摩押}}人,使他们躺卧在地上,用绳量一量:量二绳的杀了,量一绳的存留。{\PN{摩押}}人就归服{\PN{大卫}},给他进贡。
\par }{\PP \VS{3}{\PN{琐巴}}王{\PN{利合}}的儿子{\PN{哈大底谢}}往大河去,要夺回他的国权。{\PN{大卫}}就攻打他,
\VS{4}擒拿了他的马兵一千七百,步兵二万,将拉战车的马砍断蹄筋,但留下一百辆车的马。
\par }{\PP \VS{5}{\PN{大马士革}}的{\PN{亚兰}}人来帮助{\PN{琐巴}}王{\PN{哈大底谢}},{\PN{大卫}}就杀了{\PN{亚兰}}人二万二千。
\VS{6}于是{\PN{大卫}}在{\PN{大马士革}}的{\PN{亚兰}}地设立防营,{\PN{亚兰}}人就归服他,给他进贡。{\PN{大卫}}无论往哪里去,耶和华都使他得胜。
\VS{7}他夺了{\PN{哈大底谢}}臣仆所拿的金盾牌,带到{\PN{耶路撒冷}}。
\VS{8}{\PN{大卫}}王又从属{\PN{哈大底谢}}的{\PN{比他}}和{\PN{比罗他}}城中夺取了许多的铜。
\par }{\PP \VS{9}{\PN{哈马}}王{\PN{陀以}}听见{\PN{大卫}}杀败{\PN{哈大底谢}}的全军,
\VS{10}就打发他儿子{\PN{约兰}}去见{\PN{大卫}}王,问他的安,为他祝福,因为他杀败了{\PN{哈大底谢}}(原来{\PN{陀以}}与{\PN{哈大底谢}}常常争战)。{\PN{约兰}}带了金银铜的器皿来,
\VS{11}{\PN{大卫}}王将这些器皿和他治服各国所得来的金银都分别为圣,献给耶和华,
\VS{12}就是从{\PN{亚兰}}、{\PN{摩押}}、{\PN{亚扪}}、{\PN{非利士}}、{\PN{亚玛力}}人所得来的,以及从{\PN{琐巴}}王{\PN{利合}}的儿子{\PN{哈大底谢}}所掠之物。
\par }{\PP \VS{13}{\PN{大卫}}在{\PN{盐谷}}击杀了{\PN{亚兰}}\FTNT{}{{\FR 8:13: }或译:以东,见诗篇六十篇诗题}一万八千人回来,就得了大名;
\VS{14}又在{\PN{以东}}全地设立防营,{\PN{以东}}人就都归服{\PN{大卫}}。{\PN{大卫}}无论往哪里去,耶和华都使他得胜。
\par }{\PP \VS{15}{\PN{大卫}}作{\PN{以色列}}众人的王,又向众民秉公行义。
\VS{16}{\PN{洗鲁雅}}的儿子{\PN{约押}}作元帅;{\PN{亚希律}}的儿子{\PN{约沙法}}作史官;
\VS{17}{\PN{亚希突}}的儿子{\PN{撒督}}和{\PN{亚比亚他}}的儿子{\PN{亚希米勒}}作祭司{\ADD{长}};{\PN{西莱雅}}作书记;
\VS{18}{\PN{耶何耶大}}的儿子{\PN{比拿雅}}统辖{\PN{基利提}}人和{\PN{比利提}}人。{\PN{大卫}}的众子都作领袖。

\par }\Chap{9}{\SH 大卫和米非波设
\par }{\PP \VerseOne{1}{\PN{大卫}}问说:「{\PN{扫罗}}家还有剩下的人没有?我要因{\PN{约拿单}}的缘故向他施恩。」
\VS{2}{\PN{扫罗}}家有一个仆人,名叫{\PN{洗巴}},有人叫他来见{\PN{大卫}},王问他说:「你是{\PN{洗巴}}吗?」回答说:「仆人是。」
\VS{3}王说:「{\PN{扫罗}}家还有人没有?我要照 神的慈爱恩待他。」{\PN{洗巴}}对王说:「还有{\PN{约拿单}}的一个儿子,是瘸腿的。」
\VS{4}王说:「他在哪里?」{\PN{洗巴}}对王说:「他在{\PN{罗·底巴}},{\PN{亚米利}}的儿子{\PN{玛吉}}家里。」
\VS{5}于是{\PN{大卫}}王打发人去,从{\PN{罗·底巴}}、{\PN{亚米利}}的儿子{\PN{玛吉}}家里召了他来。
\VS{6}{\PN{扫罗}}的孙子、{\PN{约拿单}}的儿子{\PN{米非波设}}来见{\PN{大卫}},伏地叩拜。{\PN{大卫}}说:「{\PN{米非波设}}!」{\PN{米非波设}}说:「仆人在此。」
\VS{7}{\PN{大卫}}说:「你不要惧怕,我必因你父亲{\PN{约拿单}}的缘故施恩与你,将你祖父{\PN{扫罗}}的一切田地都归还你;你也可以常与我同席吃饭。」
\VS{8}{\PN{米非波设}}又叩拜,说:「仆人算什么,不过如死狗一般,竟蒙王这样眷顾!」
\par }{\PP \VS{9}王召了{\PN{扫罗}}的仆人{\PN{洗巴}}来,对他说:「我已将属{\PN{扫罗}}和他的一切家产都赐给你主人的儿子了。
\VS{10}你和你的众子、仆人要为你主人的儿子{\PN{米非波设}}耕种田地,把所产的拿来供他食用;他却要常与我同席吃饭。」{\PN{洗巴}}有十五个儿子,二十个仆人。
\VS{11}{\PN{洗巴}}对王说:「凡我主我王吩咐仆人的,仆人都必遵行。」王又说:「{\PN{米非波设}}必与我同席吃饭,如王的儿子一样。」
\VS{12}{\PN{米非波设}}有一个小儿子,名叫{\PN{米迦}}。凡住在{\PN{洗巴}}家里的人都作了{\PN{米非波设}}的仆人。
\VS{13}于是{\PN{米非波设}}住在{\PN{耶路撒冷}},常与王同席吃饭。他两腿都是瘸的。

\par }\Chap{10}{\SH 击败亚扪人和亚兰人
\par }{\R (代上19·1—19)
\par }{\PP \VerseOne{1}此后,{\PN{亚扪}}人的王死了,他儿子{\PN{哈嫩}}接续他作王。
\VS{2}{\PN{大卫}}说:「我要照{\PN{哈嫩}}的父亲{\PN{拿辖}}厚待我的恩典厚待{\PN{哈嫩}}。」于是{\PN{大卫}}差遣臣仆,为他丧父安慰他。{\PN{大卫}}的臣仆到了{\PN{亚扪}}人的境内。
\VS{3}但{\PN{亚扪}}人的首领对他们的主{\PN{哈嫩}}说:「{\PN{大卫}}差人来安慰你,你想他是尊敬你父亲吗?他差臣仆来不是详察窥探、要倾覆这城吗?」
\par }{\PP \VS{4}{\PN{哈嫩}}便将{\PN{大卫}}臣仆的胡须剃去一半,又割断他们下半截的衣服,使他们露出下体,打发他们回去。
\VS{5}有人告诉{\PN{大卫}},他就差人去迎接他们,(因为他们甚觉羞耻),告诉他们说:「可以住在{\PN{耶利哥}},等到胡须长起再回来。」
\par }{\PP \VS{6}{\PN{亚扪}}人知道{\PN{大卫}}憎恶他们,就打发人去,招募{\PN{伯·利合}}的{\PN{亚兰}}人和{\PN{琐巴}}的{\PN{亚兰}}人,步兵二万,与{\PN{玛迦}}王的人一千、{\PN{陀伯}}人一万二千。
\VS{7}{\PN{大卫}}听见了,就差派{\PN{约押}}统带勇猛的全军出去。
\VS{8}{\PN{亚扪}}人出来,在城门前摆阵;{\PN{琐巴}}与{\PN{利合}}的{\PN{亚兰}}人、{\PN{陀伯}}人,并{\PN{玛迦}}人,另在郊野摆阵。
\par }{\PP \VS{9}{\PN{约押}}看见敌人在他前后摆阵,就从{\PN{以色列}}军中挑选精兵,使他们对着{\PN{亚兰}}人摆阵。
\VS{10}其余的兵交与他兄弟{\PN{亚比筛}},对着{\PN{亚扪}}人摆阵。
\VS{11}{\PN{约押}}对{\PN{亚比筛}}说:「{\PN{亚兰}}人若强过我,你就来帮助我;{\PN{亚扪}}人若强过你,我就去帮助你。
\VS{12}我们都当刚强,为本国的民和 神的城邑作大丈夫。愿耶和华凭他的意旨而行!」
\par }{\PP \VS{13}于是,{\PN{约押}}和跟随他的人前进攻打{\PN{亚兰}}人;{\PN{亚兰}}人在{\PN{约押}}面前逃跑。
\VS{14}{\PN{亚扪}}人见{\PN{亚兰}}人逃跑,他们也在{\PN{亚比筛}}面前逃跑进城。{\PN{约押}}就离开{\PN{亚扪}}人那里,回{\PN{耶路撒冷}}去了。
\par }{\PP \VS{15}{\PN{亚兰}}人见自己被{\PN{以色列}}人打败,就又聚集。
\VS{16}{\PN{哈大底谢}}差遣人,将大河那边的{\PN{亚兰}}人调来;他们到了{\PN{希兰}},{\PN{哈大底谢}}的将军{\PN{朔法}}率领他们。
\VS{17}有人告诉{\PN{大卫}},他就聚集{\PN{以色列}}众人,过{\PN{约旦河}},来到{\PN{希兰}}。{\PN{亚兰}}人迎着{\PN{大卫}}摆阵,与他打仗。
\VS{18}{\PN{亚兰}}人在{\PN{以色列}}人面前逃跑;{\PN{大卫}}杀了{\PN{亚兰}}七百辆战车{\ADD{的人}},四万马兵,又杀了{\PN{亚兰}}的将军{\PN{朔法}}。
\VS{19}属{\PN{哈大底谢}}的诸王见自己被{\PN{以色列}}人打败,就与{\PN{以色列}}人和好,归服他们。于是{\PN{亚兰}}人不敢再帮助{\PN{亚扪}}人了。

\par }\Chap{11}{\SH 大卫和拔示巴
\par }{\PP \VerseOne{1}过了一年,到列王出战的时候,{\PN{大卫}}又差派{\PN{约押}},率领臣仆和{\PN{以色列}}众人出{\ADD{战}}。他们就打败{\PN{亚扪}}人,围攻{\PN{拉巴}}。{\PN{大卫}}仍住在{\PN{耶路撒冷}}。
\par }{\PP \VS{2}一日,太阳平西,{\PN{大卫}}从床上起来,在王宫的平顶上游行,看见一个妇人沐浴,容貌甚美,
\VS{3}{\PN{大卫}}就差人打听那妇人是谁。有人说:「她是{\PN{以连}}的女儿,{\PN{赫}}人{\PN{乌利亚}}的妻{\PN{拔示巴}}。」
\VS{4}{\PN{大卫}}差人去,将妇人接来;那时她的月经才得洁净。她来了,{\PN{大卫}}与她同房,她就回家去了。
\VS{5}于是她怀了孕,打发人去告诉{\PN{大卫}}说:「我怀了孕。」
\par }{\PP \VS{6}{\PN{大卫}}差人到{\PN{约押}}那里,{\ADD{说}}:「你打发{\PN{赫}}人{\PN{乌利亚}}到我这里来。」{\PN{约押}}就打发{\PN{乌利亚}}去见{\PN{大卫}}。
\VS{7}{\PN{乌利亚}}来了,{\PN{大卫}}问{\PN{约押}}好,也问兵好,又问争战的事怎样。
\VS{8}{\PN{大卫}}对{\PN{乌利亚}}说:「你回家去,洗洗脚吧!」{\PN{乌利亚}}出了王宫,随后王送他一分{\ADD{食}}物。
\VS{9}{\PN{乌利亚}}却和他主人的仆人一同睡在宫门外,没有回家去。
\VS{10}有人告诉{\PN{大卫}}说:「{\PN{乌利亚}}没有回家去。」{\PN{大卫}}就问{\PN{乌利亚}}说:「你从远路上来,为什么不回家去呢?」
\VS{11}{\PN{乌利亚}}对{\PN{大卫}}说:「{\ADD{约}}柜和{\PN{以色列}}与{\PN{犹大}}兵都住在棚里,我主{\PN{约押}}和我主\FTNT{}{{\FR 11:11: }或译:王}的仆人都在田野安营,我岂可回家吃喝、与妻子同寝呢?我敢在王面前起誓\FTNT{}{{\FR 11:11: }原文是我指着王和王的性命起誓}:我决不行这事!」
\VS{12}{\PN{大卫}}吩咐{\PN{乌利亚}}说:「你今日仍住在这里,明日我打发你去。」于是{\PN{乌利亚}}那日和次日住在{\PN{耶路撒冷}}。
\VS{13}{\PN{大卫}}召了{\PN{乌利亚}}来,叫他在自己面前吃喝,使他喝醉。到了晚上,{\PN{乌利亚}}出去与他主的仆人一同住宿,还没有回到家里去。
\par }{\PP \VS{14}次日早晨,{\PN{大卫}}写信与{\PN{约押}},交{\PN{乌利亚}}随手带去。
\VS{15}信内写着说:「要派{\PN{乌利亚}}前进,到阵势极险之处,你们便退后,使他被杀。」
\VS{16}{\PN{约押}}围城的时候,知道敌人那里有勇士,便将{\PN{乌利亚}}派在那里。
\VS{17}城里的人出来和{\PN{约押}}打仗;{\PN{大卫}}的仆人中有几个被杀的,{\PN{赫}}人{\PN{乌利亚}}也死了。
\par }{\PP \VS{18}于是,{\PN{约押}}差人去将争战的一切事告诉{\PN{大卫}},
\VS{19}又嘱咐使者说:「你把争战的一切事对王说完了,
\VS{20}王若发怒,问你说:『你们打仗为什么挨近城墙呢?岂不知敌人必从城上射箭吗?
\VS{21}从前打死{\PN{耶路·比设}}\FTNT{}{{\FR 11:21: }就是耶路·巴力,见士师记九章一节}儿子{\PN{亚比米勒}}的是谁呢?岂不是一个妇人从城上抛下一块上磨石来,打在他身上,他就死在{\PN{提备斯}}吗?你们为什么挨近城墙呢?』你就说:『王的仆人—{\PN{赫}}人{\PN{乌利亚}}也死了。』」
\par }{\PP \VS{22}使者起身,来见{\PN{大卫}},照着{\PN{约押}}所吩咐他的话奏告{\PN{大卫}}。
\VS{23}使者对{\PN{大卫}}说:「敌人强过我们,出到郊野与我们打仗,我们追杀他们,直到城门口。
\VS{24}射箭的从城上射王的仆人,射死几个,{\PN{赫}}人{\PN{乌利亚}}也死了。」
\VS{25}王向使者说:「你告诉{\PN{约押}}说:『不要因这事愁闷,刀剑或吞灭这人或吞灭那人,没有一定的;你只管竭力攻城,将城倾覆。』可以用这话勉励{\PN{约押}}。」
\par }{\PP \VS{26}{\PN{乌利亚}}的妻听见丈夫{\PN{乌利亚}}死了,就为他哀哭。
\VS{27}哀哭的日子过了,{\PN{大卫}}差人将她接到宫里,她就作了{\PN{大卫}}的妻,给{\PN{大卫}}生了一个儿子。但{\PN{大卫}}所行的这事,耶和华甚不喜悦。

\par }\Chap{12}{\SH 拿单的信息和大卫的悔改
\par }{\PP \VerseOne{1}耶和华差遣{\PN{拿单}}去见{\PN{大卫}}。{\PN{拿单}}到了{\PN{大卫}}那里,对他说:「在一座城里有两个人:一个是富户,一个是穷人。
\VS{2}富户有许多牛群羊群;
\VS{3}穷人除了所买来养活的一只小母羊羔之外,别无所有。羊羔在他家里和他儿女一同长大,吃他所吃的,喝他所喝的,睡在他怀中,在他看来如同女儿一样。
\VS{4}有一客人来到这富户家里;富户舍不得从自己的牛群羊群中取一只预备给客人吃,却取了那穷人的羊羔,预备给客人吃。」
\VS{5}{\PN{大卫}}就甚恼怒那人,对{\PN{拿单}}说:「我指着永生的耶和华起誓,行这事的人该死!
\VS{6}他必偿还羊羔四倍;因为他行这事,没有怜恤的心。」
\par }{\PP \VS{7}{\PN{拿单}}对{\PN{大卫}}说:「你就是那人!耶和华—{\PN{以色列}}的 神如此说:『我膏你作{\PN{以色列}}的王,救你脱离{\PN{扫罗}}的手。
\VS{8}我将你主人的家业赐给你,将你主人的妻交在你怀里,又将{\PN{以色列}}和{\PN{犹大}}家赐给你;你若还以为不足,我早就加倍地赐给你。
\VS{9}你为什么藐视耶和华的命令,行他眼中看为恶的事呢?你借{\PN{亚扪}}人的刀杀害{\PN{赫}}人{\PN{乌利亚}},又娶了他的妻为妻。
\VS{10}你既藐视我,娶了{\PN{赫}}人{\PN{乌利亚}}的妻为妻,所以刀剑必永不离开你的家。』
\VS{11}耶和华如此说:『我必从你家中兴起祸患攻击你;我必在你眼前把你的妃嫔赐给别人,他在日光之下就与她们同寝。
\VS{12}你在暗中行这事,我却要在{\PN{以色列}}众人面前,日光之下,报应你。』」
\VS{13}{\PN{大卫}}对{\PN{拿单}}说:「我得罪耶和华了!」{\PN{拿单}}说:「耶和华已经除掉你的罪,你必不至于死。
\VS{14}只是你行这事,叫耶和华的仇敌大得亵渎的机会,故此,你所得的孩子必定要死。」
\VS{15}{\PN{拿单}}就回家去了。
\par }{\SH 大卫的儿子夭折
\par }{\PP 耶和华击打{\PN{乌利亚}}妻给{\PN{大卫}}所生的孩子,使他得重病。
\VS{16}所以{\PN{大卫}}为这孩子恳求 神,而且禁食,进入{\ADD{内室}},终夜躺在地上。
\VS{17}他家中的老臣来到他旁边,要把他从地上扶起来,他却不肯起来,也不同他们吃饭。
\VS{18}到第七日,孩子死了。{\PN{大卫}}的臣仆不敢告诉他孩子死了,因他们说:「孩子还活着的时候,我们劝他,他尚且不肯听我们的话,若告诉他孩子死了,岂不更加忧伤吗?」
\VS{19}{\PN{大卫}}见臣仆彼此低声说话,就知道孩子死了,问臣仆说:「孩子死了吗?」他们说:「死了。」
\VS{20}{\PN{大卫}}就从地上起来,沐浴,抹膏,换了衣裳,进耶和华的殿敬拜;然后回宫,吩咐人摆饭,他便吃了。
\VS{21}臣仆问他说:「你所行的是什么意思?孩子活着的时候,你禁食哭泣;孩子死了,你倒起来吃饭。」
\VS{22}{\PN{大卫}}说:「孩子还活着,我禁食哭泣;因为我想,或者耶和华怜恤我,使孩子不死也未可知。
\VS{23}孩子死了,我何必禁食,我岂能使他返回呢?我必往他那里去,他却不能回我这里来。」
\par }{\SH 所罗门出生
\par }{\PP \VS{24}{\PN{大卫}}安慰他的妻{\PN{拔示巴}},与她同寝,她就生了儿子,给他起名叫{\PN{所罗门}}。耶和华也喜爱他,
\VS{25}就借先知{\PN{拿单}}赐他一个名字,叫{\PN{耶底底亚}},因为耶和华{\ADD{爱他}}。
\par }{\SH 大卫攻取拉巴
\par }{\R (代上20·1—3)
\par }{\PP \VS{26}{\PN{约押}}攻取{\PN{亚扪}}人的京城{\PN{拉巴}}。
\VS{27}{\PN{约押}}打发使者去见{\PN{大卫}},说:「我攻打{\PN{拉巴}},取其水城。
\VS{28}现在你要聚集其余的军兵来,安营围攻这城,恐怕我取了这城,人就以我的名叫这城。」
\VS{29}于是{\PN{大卫}}聚集众军,往{\PN{拉巴}}去攻城,就取了这城,
\VS{30}夺了{\PN{亚扪}}人之王所戴的金冠冕\FTNT{}{{\FR 12:30: }王:或译玛勒堪;玛勒堪就是米勒公,又名摩洛,亚扪族之神名},其上的金子重一他连得,又嵌着宝石。人将这冠冕戴在{\PN{大卫}}头上。{\PN{大卫}}从城里夺了许多财物,
\VS{31}将城里的人拉出来,放在锯下,或铁耙下,或铁斧下,或叫他经过砖窑\FTNT{}{{\FR 12:31: }或译:强他们用锯,或用打粮食的铁器,或用铁斧做工,或使在砖窑里服役};{\PN{大卫}}待{\PN{亚扪}}各城的居民都是如此。其后,{\PN{大卫}}和众军都回{\PN{耶路撒冷}}去了。

\par }\Chap{13}{\SH 暗嫩和她玛
\par }{\PP \VerseOne{1}{\PN{大卫}}的儿子{\PN{押沙龙}}有一个美貌的妹子,名叫{\PN{她玛}}。{\PN{大卫}}的儿子{\PN{暗嫩}}爱她。
\VS{2}{\PN{暗嫩}}为他妹子{\PN{她玛}}忧急成病。{\PN{她玛}}还是处女,{\PN{暗嫩}}以为难向她行事。
\VS{3}{\PN{暗嫩}}有一个朋友,名叫{\PN{约拿达}},是{\PN{大卫}}长兄{\PN{示米亚}}的儿子。这{\PN{约拿达}}为人极其狡猾;
\VS{4}他问{\PN{暗嫩}}说:「王的儿子啊,为何一天比一天瘦弱呢?请你告诉我。」{\PN{暗嫩}}回答说:「我爱我兄弟{\PN{押沙龙}}的妹子{\PN{她玛}}。」
\VS{5}{\PN{约拿达}}说:「你不如躺在床上装病;你父亲来看你,就对他说:『求父叫我妹子{\PN{她玛}}来,在我眼前预备食物,递给我吃,使我看见,好从她手里接过来吃。』」
\VS{6}于是{\PN{暗嫩}}躺卧装病。王来看他,他对王说:「求父叫我妹子{\PN{她玛}}来,在我眼前为我做两个饼,我好从她手里接过来吃。」
\par }{\PP \VS{7}{\PN{大卫}}就打发人到宫里,对{\PN{她玛}}说:「你往你哥哥{\PN{暗嫩}}的屋里去,为他预备食物。」
\VS{8}{\PN{她玛}}就到她哥哥{\PN{暗嫩}}的屋里;{\PN{暗嫩}}正躺卧。{\PN{她玛}}抟面,在他眼前做饼,且烤熟了,
\VS{9}在他面前将饼从锅里倒出来,他却不肯吃,便说:「众人离开我出去吧!」众人就都离开他,出去了。
\VS{10}{\PN{暗嫩}}对{\PN{她玛}}说:「你把食物拿进卧房,我好从你手里接过来吃。」{\PN{她玛}}就把所做的饼拿进卧房,到她哥哥{\PN{暗嫩}}那里,
\VS{11}拿着饼上前给他吃,他便拉住{\PN{她玛}},说:「我妹妹,你来与我同寝。」
\VS{12}{\PN{她玛}}说:「我哥哥,不要玷辱我。{\PN{以色列}}人中不当这样行,你不要做这丑事;
\VS{13}你{\ADD{玷辱了我}},我何以掩盖我的羞耻呢?你在{\PN{以色列}}中也成了愚妄人。你可以求王,他必不禁止我归你。」
\VS{14}但{\PN{暗嫩}}不肯听她的话,因比她力大,就玷辱她,与她同寝。
\par }{\PP \VS{15}随后,{\PN{暗嫩}}极其恨她,那恨她的心比先前爱她的心更甚,对她说:「你起来,去吧!」
\VS{16}{\PN{她玛}}说:「不要这样!你赶出我去的这罪比你才行的更重!」但{\PN{暗嫩}}不肯听她的话,
\VS{17}就叫伺候自己的仆人来,说:「将这个女子赶出去!她一出去,你就关门,上闩。」
\VS{18}那时{\PN{她玛}}穿着彩衣,因为没有出嫁的公主都是这样穿。{\PN{暗嫩}}的仆人就把她赶出去,关门上闩。
\VS{19}{\PN{她玛}}把灰尘撒在头上,撕裂所穿的彩衣,以手抱头,一面行走,一面哭喊。
\par }{\PP \VS{20}她胞兄{\PN{押沙龙}}问她说:「莫非你哥哥{\PN{暗嫩}}与你亲近了吗?我妹妹,暂且不要作声,他是你的哥哥,不要将这事放在心上。」{\PN{她玛}}就孤孤单单地住在她胞兄{\PN{押沙龙}}家里。
\par }{\PP \VS{21}{\PN{大卫}}王听见这事,就甚发怒。
\VS{22}{\PN{押沙龙}}并不和他哥哥{\PN{暗嫩}}说好说歹;因为{\PN{暗嫩}}玷辱他妹妹{\PN{她玛}},所以{\PN{押沙龙}}恨恶他。
\par }{\SH 押沙龙为妹报仇
\par }{\PP \VS{23}过了二年,在靠近{\PN{以法莲}}的{\PN{巴力·夏琐}}有人为{\PN{押沙龙}}剪羊毛;{\PN{押沙龙}}请王的众子与他同去。
\VS{24}{\PN{押沙龙}}来见王,说:「现在有人为仆人剪羊毛,请王和王的臣仆与仆人同去。」
\VS{25}王对{\PN{押沙龙}}说:「我儿,我们不必都去,恐怕使你耗费太多。」{\PN{押沙龙}}再三请王,王仍是不肯去,只为他祝福。
\VS{26}{\PN{押沙龙}}说:「王若不去,求王许我哥哥{\PN{暗嫩}}同去。」王说:「何必要他去呢?」
\VS{27}{\PN{押沙龙}}再三求王,王就许{\PN{暗嫩}}和王的众子与他同去。
\par }{\PP \VS{28}{\PN{押沙龙}}吩咐仆人说:「你们注意,看{\PN{暗嫩}}饮酒畅快的时候,我对你们说杀{\PN{暗嫩}},你们便杀他,不要惧怕。这不是我吩咐你们的吗?你们只管壮胆奋勇!」
\VS{29}{\PN{押沙龙}}的仆人就照{\PN{押沙龙}}所吩咐的,向{\PN{暗嫩}}行了。王的众子都起来,各人骑上骡子,逃跑了。
\par }{\PP \VS{30}他们还在路上,有风声传到{\PN{大卫}}那里,说:「{\PN{押沙龙}}将王的众子都杀了,没有留下一个。」
\VS{31}王就起来,撕裂衣服,躺在地上。王的臣仆也都撕裂衣服,站在旁边。
\VS{32}{\PN{大卫}}的长兄,{\PN{示米亚}}的儿子{\PN{约拿达}}说:「我主,不要以为王的众子—少年人都杀了,只有{\PN{暗嫩}}一个人死了。自从{\PN{暗嫩}}玷辱{\PN{押沙龙}}妹子{\PN{她玛}}的那日,{\PN{押沙龙}}就定意杀{\PN{暗嫩}}了。
\VS{33}现在,我主我王,不要把这事放在心上,以为王的众子都死了,只有{\PN{暗嫩}}一个人死了。」
\par }{\PP \VS{34}{\PN{押沙龙}}逃跑了。
\par }{\PP 守望的少年人举目观看,见有许多人从山坡的路上来。
\VS{35}{\PN{约拿达}}对王说:「看哪,王的众子都来了,果然与你仆人所说的相合。」
\VS{36}话才说完,王的众子都到了,放声大哭;王和臣仆也都哭得甚恸。
\par }{\PP \VS{37}{\PN{押沙龙}}逃到{\PN{基述}}王{\PN{亚米忽}}的儿子{\PN{达买}}那里去了。{\PN{大卫}}天天为他儿子悲哀。
\VS{38}{\PN{押沙龙}}逃到{\PN{基述}},在那里住了三年。
\VS{39}{\PN{暗嫩}}死了以后,{\PN{大卫}}王得了安慰,心里切切想念{\PN{押沙龙}}。

\par }\Chap{14}{\SH 约押设法让押沙龙回来
\par }{\PP \VerseOne{1}{\PN{洗鲁雅}}的儿子{\PN{约押}},知道王心里想念{\PN{押沙龙}},
\VS{2}就打发人往{\PN{提哥亚}}去,从那里叫了一个聪明的妇人来,对她说:「请你假装居丧的,穿上孝衣,不要用膏抹身,要装作为死者许久悲哀的妇人;
\VS{3}进去见王,对王如此如此说。」于是{\PN{约押}}将当说的话教导了妇人。
\par }{\PP \VS{4}{\PN{提哥亚}}妇人到王面前,伏地叩拜,说:「王啊,求你拯救!」
\VS{5}王问她说:「你有什么事呢?」回答说:「婢女实在是寡妇,我丈夫死了。
\VS{6}我有两个儿子,一日在田间争斗,没有人解劝,这个就打死那个。
\VS{7}现在全家的人都起来攻击婢女,说:『你将那打死兄弟的交出来,我们好治死他,偿他打死兄弟的命,灭绝那承受家业的。』这样,他们要将我剩下的炭火灭尽,不与我丈夫留名留后在世上。」
\VS{8}王对妇人说:「你回家去吧!我必为你下令。」
\VS{9}{\PN{提哥亚}}妇人又对王说:「我主我王,愿这罪归我和我父家,与王和王的位无干。」
\VS{10}王说:「凡难为你的,你就带他到我这里来,他必不再搅扰你。」
\VS{11}妇人说:「愿王记念耶和华—你的 神,不许报血仇的人施行灭绝,恐怕他们灭绝我的儿子。」王说:「我指着永生的耶和华起誓:你的儿子连一根头发也不致落在地上。」
\VS{12}妇人说:「求我主我王容婢女再说一句话。」王说:「你说吧!」
\VS{13}妇人说:「王为何也起意要害 神的民呢?王不使那逃亡的人回来,王的这话就是自证己错了!
\VS{14}我们都是必死的,如同水泼在地上,不能收回。 神并不夺取人的性命,乃设法使逃亡的人不致成为赶出、回不来的。
\VS{15}我来将这话告诉我主我王,是因百姓使我惧怕。婢女想,不如将这话告诉王,或者王成就婢女所求的。
\VS{16}人要将我和我儿子从 神的地业上一同除灭,王必应允救我脱离他的手。
\VS{17}婢女又想,我主我王的话必安慰我;因为我主我王能辨别是非,如同 神的使者一样。惟愿耶和华—你的 神与你同在!」
\VS{18}王对妇人说:「我要问你一句话,你一点不要瞒我。」妇人说:「愿我主我王说。」
\VS{19}王说:「你这些话莫非是{\PN{约押}}的主意吗?」妇人说:「我敢在我主我王面前起誓:王的话正对,不偏左右,是王的仆人{\PN{约押}}吩咐我的,这些话是他教导我的。
\VS{20}王的仆人{\PN{约押}}如此行,为要挽回这事。我主的智慧却如 神使者的智慧,能知世上一切事。」
\par }{\PP \VS{21}王对{\PN{约押}}说:「我应允你这事。你可以去,把那少年人{\PN{押沙龙}}带回来。」
\VS{22}{\PN{约押}}就面伏于地叩拜,祝谢于王,又说:「王既应允仆人所求的,仆人今日知道在我主我王眼前蒙恩了。」
\VS{23}于是{\PN{约押}}起身往{\PN{基述}}去,将{\PN{押沙龙}}带回{\PN{耶路撒冷}}。
\VS{24}王说:「使他回自己家里去,不要见我的面。」{\PN{押沙龙}}就回自己家里去,没有见王的面。
\par }{\SH 大卫饶恕押沙龙
\par }{\PP \VS{25}{\PN{以色列}}全地之中,无人像{\PN{押沙龙}}那样俊美,得人的称赞,从脚底到头顶毫无瑕疵。
\VS{26}他的头发甚重,每到年底剪发一次;所剪下来的,按王的平称一称,重二百舍客勒。
\VS{27}{\PN{押沙龙}}生了三个儿子,一个女儿。女儿名叫{\PN{她玛}},是个容貌俊美的女子。
\par }{\PP \VS{28}{\PN{押沙龙}}住在{\PN{耶路撒冷}}足有二年,没有见王的面。
\VS{29}{\PN{押沙龙}}打发人去叫{\PN{约押}}来,要托他去见王,{\PN{约押}}却不肯来。第二次打发人去叫他,他仍不肯来。
\VS{30}所以{\PN{押沙龙}}对仆人说:「你们看,{\PN{约押}}有一块田,与我的田相近,其中有大麦,你们去放火烧了。」{\PN{押沙龙}}的仆人就去放火烧了那田。
\par }{\PP \VS{31}于是{\PN{约押}}起来,到了{\PN{押沙龙}}家里,问他说:「你的仆人为何放火烧了我的田呢?」
\VS{32}{\PN{押沙龙}}回答{\PN{约押}}说:「我打发人去请你来,好托你去见王,替我说:『我为何从{\PN{基述}}回来呢?不如仍在那里。』现在要许我见王的面;我若有罪,任凭王杀我就是了。」
\VS{33}于是{\PN{约押}}去见王,将这话奏告王,王便叫{\PN{押沙龙}}来。{\PN{押沙龙}}来见王,在王面前俯伏于地,王就与{\PN{押沙龙}}亲嘴。

\par }\Chap{15}{\SH 押沙龙阴谋造反
\par }{\PP \VerseOne{1}此后,{\PN{押沙龙}}为自己预备车马,又派五十人在他前头奔走。
\VS{2}{\PN{押沙龙}}常常早晨起来,站在城门的道旁,凡有争讼要去求王判断的,{\PN{押沙龙}}就叫他过来,问他说:「你是哪一城的人?」回答说:「仆人是{\PN{以色列}}某支派的人。」
\VS{3}{\PN{押沙龙}}对他说:「你的事有情有理,无奈王没有委人听你伸诉。」
\VS{4}{\PN{押沙龙}}又说:「恨不得我作国中的士师!凡有争讼求审判的到我这里来,我必秉公判断。」
\VS{5}若有人近前来要拜{\PN{押沙龙}},{\PN{押沙龙}}就伸手拉住他,与他亲嘴。
\VS{6}{\PN{以色列}}人中,凡去见王求判断的,{\PN{押沙龙}}都是如此待他们。这样,{\PN{押沙龙}}暗中得了{\PN{以色列}}人的心。
\par }{\PP \VS{7}满了四十年\FTNT{}{{\FR 15:7: }有作四年的},{\PN{押沙龙}}对王说:「求你准我往{\PN{希伯
}}去,还我向耶和华所许的愿。
\VS{8}因为仆人住在{\PN{亚兰}}的{\PN{基述}},曾许愿说:『耶和华若使我再回{\PN{耶路撒冷}},我必事奉他。』」
\VS{9}王说:「你平平安安地去吧!」{\PN{押沙龙}}就起身,往{\PN{希伯
}}去了。
\VS{10}{\PN{押沙龙}}打发探子走遍{\PN{以色列}}各支派,说:「你们一听见角声就说:『{\PN{押沙龙}}在{\PN{希伯
}}作王了!』」
\VS{11}{\PN{押沙龙}}在{\PN{耶路撒冷}}请了二百人与他同去,都是诚诚实实去的,并不知道其中的真情。
\VS{12}{\PN{押沙龙}}献祭的时候,打发人去将{\PN{大卫}}的谋士、{\PN{基罗}}人{\PN{亚希多弗}}从他本城请了来。于是叛逆的势派甚大;因为随从{\PN{押沙龙}}的人民,日渐增多。
\par }{\SH 大卫逃离耶路撒冷
\par }{\PP \VS{13}有人报告{\PN{大卫}}说:「{\PN{以色列}}人的心都归向{\PN{押沙龙}}了!」
\VS{14}{\PN{大卫}}就对{\PN{耶路撒冷}}跟随他的臣仆说:「我们要起来逃走,不然都不能躲避{\PN{押沙龙}}了;要速速地去,恐怕他忽然来到,加害于我们,用刀杀尽合城的人。」
\VS{15}王的臣仆对王说:「我主我王所定的,仆人都愿遵行。」
\VS{16}于是王带着全家的人出去了,但留下十个妃嫔看守宫殿。
\par }{\PP \VS{17}王出去,众民都跟随他,到{\PN{伯·墨哈}},就住下了。
\VS{18}王的臣仆都在他面前过去。{\PN{基利提}}人、{\PN{比利提}}人,就是从{\PN{迦特}}跟随王来的六百人,也都在他面前过去。
\VS{19}王对{\PN{迦特}}人{\PN{以太}}说:「你是外邦逃来的人,为什么与我们同去呢?你可以回去与{\ADD{新}}王同住,{\ADD{或者回}}你本地去吧!
\VS{20}你来的日子不多,我今日怎好叫你与我们一同飘流、没有一定的住处呢?你不如带你的弟兄回去吧!{\ADD{愿耶和华}}用慈爱诚实待你。」
\par }{\PP \VS{21}{\PN{以太}}对王说:「我指着永生的耶和华起誓,又敢在王面前起誓:无论生死,王在哪里,仆人也必在那里。」
\VS{22}{\PN{大卫}}对{\PN{以太}}说:「你前去过{\ADD{河}}吧!」于是{\PN{迦特}}人{\PN{以太}}带着跟随他的人和所有的{\ADD{妇人}}孩子,就都过去了。
\VS{23}本地的人都放声大哭。众民尽都过去,王也过了{\PN{汲沦溪}};众民往旷野去了。
\par }{\PP \VS{24}{\PN{撒督}}和抬 神约柜的{\PN{利未}}人也一同来了,将 神的{\ADD{约}}柜放下。{\PN{亚比亚他}}上来,等着众民从城里出来过去。
\VS{25}王对{\PN{撒督}}说:「你将 神的{\ADD{约}}柜抬回城去。我若在耶和华眼前蒙恩,他必使我回来,再见{\ADD{约}}柜和他的居所。
\VS{26}倘若他说:『我不喜悦你』,看哪,我在这里,愿他凭自己的意旨待我!」
\VS{27}王又对祭司{\PN{撒督}}说:「你不是先见吗?你可以安然回城;你儿子{\PN{亚希玛斯}}和{\PN{亚比亚他}}的儿子{\PN{约拿单}}都可以与你同去。
\VS{28}我在旷野的渡口那里等你们报信给我。」
\VS{29}于是{\PN{撒督}}和{\PN{亚比亚他}}将 神的{\ADD{约}}柜抬回{\PN{耶路撒冷}},他们就住在那里。
\par }{\PP \VS{30}{\PN{大卫}}蒙头赤脚上{\PN{橄榄山}},一面上一面哭。跟随他的人也都蒙头哭着上去;
\VS{31}有人告诉{\PN{大卫}}说:「{\PN{亚希多弗}}也在叛党之中,随从{\PN{押沙龙}}。」{\PN{大卫}}祷告说:「耶和华啊,求你使{\PN{亚希多弗}}的计谋变为愚拙!」
\VS{32}{\PN{大卫}}到了{\ADD{山}}顶、敬拜 神的地方,见{\PN{亚基}}人{\PN{户筛}},衣服撕裂,头蒙灰尘来迎接他。
\VS{33}{\PN{大卫}}对他说:「你若与我同去,必累赘我;
\VS{34}你若回城去,对{\PN{押沙龙}}说:『王啊,我愿作你的仆人;我向来作你父亲的仆人,现在我也照样作你的仆人。』这样,你就可以为我破坏{\PN{亚希多弗}}的计谋。
\VS{35}祭司{\PN{撒督}}和{\PN{亚比亚他}}岂不都在那里吗?你在王宫里听见什么,就要告诉祭司{\PN{撒督}}和{\PN{亚比亚他}}。
\VS{36}{\PN{撒督}}的儿子{\PN{亚希玛斯}},{\PN{亚比亚他}}的儿子{\PN{约拿单}},也都在那里。凡你们所听见的可以托这二人来报告我。」
\par }{\PP \VS{37}于是,{\PN{大卫}}的朋友{\PN{户筛}}进了城;{\PN{押沙龙}}也进了{\PN{耶路撒冷}}。

\par }\Chap{16}{\SH 大卫和洗巴
\par }{\PP \VerseOne{1}{\PN{大卫}}刚过{\ADD{山}}顶,见{\PN{米非波设}}的仆人{\PN{洗巴}}拉着备好了的两匹驴,驴上驮着二百面饼,一百葡萄饼,一百个夏天的果饼,一皮袋酒来迎接他。
\VS{2}王问{\PN{洗巴}}说:「你带这些来是什么意思呢?」{\PN{洗巴}}说:「驴是给王的家眷骑的;面饼和夏天的果饼是给少年人吃的;酒是给在旷野疲乏人喝的。」
\VS{3}王问说:「你主人的儿子在哪里呢?」{\PN{洗巴}}回答王说:「他仍在{\PN{耶路撒冷}},因他说:『{\PN{以色列}}人今日必将我父的国归还我。』」
\VS{4}王对{\PN{洗巴}}说:「凡属{\PN{米非波设}}的都归你了。」{\PN{洗巴}}说:「我叩拜我主我王,愿我在你眼前蒙恩。」
\par }{\SH 大卫和示每
\par }{\PP \VS{5}{\PN{大卫}}王到了{\PN{巴户琳}},见有一个人出来,是{\PN{扫罗}}族{\PN{基拉}}的儿子,名叫{\PN{示每}}。他一面走一面咒骂,
\VS{6}又拿石头砍{\PN{大卫}}王和王的臣仆;众民和勇士都在王的左右。
\VS{7}{\PN{示每}}咒骂说:「你这流人血的坏人哪,去吧去吧!
\VS{8}你流{\PN{扫罗}}全家的血,接续他作王;耶和华把这罪归在你身上,将这国交给你儿子{\PN{押沙龙}}。现在你自取其祸,因为你是流人血的人。」
\par }{\PP \VS{9}{\PN{洗鲁雅}}的儿子{\PN{亚比筛}}对王说:「这死狗岂可咒骂我主我王呢?求你容我过去,割下他的头来。」
\VS{10}王说:「{\PN{洗鲁雅}}的儿子,我与你们有何关涉呢?他咒骂是因耶和华吩咐他说:『你要咒骂{\PN{大卫}}。』如此,谁敢说你为什么这样行呢?」
\VS{11}{\PN{大卫}}又对{\PN{亚比筛}}和众臣仆说:「我亲生的儿子尚且寻索我的性命,何况这{\PN{便雅悯}}人呢?由他咒骂吧!因为这是耶和华吩咐他的。
\VS{12}或者耶和华见我遭难,为我今日被这人咒骂,就施恩与我。」
\VS{13}于是{\PN{大卫}}和跟随他的人往前行走。{\PN{示每}}在{\PN{大卫}}对面山坡,一面行走一面咒骂,又拿石头砍他,拿土扬他。
\VS{14}王和跟随他的众人疲疲乏乏地到了一个地方,就在那里歇息歇息。
\par }{\SH 押沙龙在耶路撒冷
\par }{\PP \VS{15}{\PN{押沙龙}}和{\PN{以色列}}众人来到{\PN{耶路撒冷}};{\PN{亚希多弗}}也与他同来。
\VS{16}{\PN{大卫}}的朋友{\PN{亚基}}人{\PN{户筛}}去见{\PN{押沙龙}},对他说:「愿王万岁!愿王万岁!」
\VS{17}{\PN{押沙龙}}问{\PN{户筛}}说:「这是你恩待朋友吗?为什么不与你的朋友同去呢?」
\VS{18}{\PN{户筛}}对{\PN{押沙龙}}说:「不然,耶和华和这民,并{\PN{以色列}}众人所拣选的,我必归顺他,与他同住。
\VS{19}再者,我当服事谁呢?岂不是前王的儿子吗?我怎样服事你父亲,也必照样服事你。」
\par }{\PP \VS{20}{\PN{押沙龙}}对{\PN{亚希多弗}}说:「你们出个主意,我们怎样行才好?」
\VS{21}{\PN{亚希多弗}}对{\PN{押沙龙}}说:「你父所留下看守宫殿的妃嫔,你可以与她们亲近。{\PN{以色列}}众人听见你父亲憎恶你,凡归顺你人的手就更坚强。」
\VS{22}于是人为{\PN{押沙龙}}在宫殿的平顶上支搭帐棚;{\PN{押沙龙}}在{\PN{以色列}}众人眼前,与他父的妃嫔亲近。
\VS{23}那时{\PN{亚希多弗}}所出的主意好像人问 神的话一样;他昔日给{\PN{大卫}},今日给{\PN{押沙龙}}所出的主意,都是这样。

\par }\Chap{17}{\SH 户筛破坏亚希多弗的计谋
\par }{\PP \VerseOne{1}{\PN{亚希多弗}}又对{\PN{押沙龙}}说:「求你准我挑选一万二千人,今夜我就起身追赶{\PN{大卫}},
\VS{2}趁他疲乏手软,我忽然追上他,使他惊惶;跟随他的民必都逃跑,我就单杀王一人,
\VS{3}使众民都归顺你。你所寻找的人{\ADD{既然死了}},众民就如已经归顺你;这样,也都平安无事了。」
\VS{4}{\PN{押沙龙}}和{\PN{以色列}}的长老都以这话为美。
\par }{\PP \VS{5}{\PN{押沙龙}}说:「要召{\PN{亚基}}人{\PN{户筛}}来,我们也要听他怎样说。」
\VS{6}{\PN{户筛}}到了{\PN{押沙龙}}面前,{\PN{押沙龙}}向他说:「{\PN{亚希多弗}}是如此如此说的,我们照着他的话行可以不可以?若不可,你就说吧!」
\VS{7}{\PN{户筛}}对{\PN{押沙龙}}说:「{\PN{亚希多弗}}这次所定的谋不善。」
\VS{8}{\PN{户筛}}又说:「你知道,你父亲和跟随他的人都是勇士,现在他们心里恼怒,如同田野丢崽子的母熊一般,而且你父亲是个战士,必不和民一同住宿。
\VS{9}他现今或藏在坑中或在别处,若有人首先被杀,凡听见的必说:『跟随{\PN{押沙龙}}的民被杀了。』
\VS{10}虽有人胆大如狮子,他的心也必消化;因为{\PN{以色列}}人都知道你父亲是英雄,跟随他的人也都是勇士。
\VS{11}依我之计,不如将{\PN{以色列}}众人—从{\PN{但}}直到{\PN{别是巴}},如同海边的沙那样多—聚集到你这里来,你也亲自率领他们出战。
\VS{12}这样,我们在何处遇见他,就下到他那里,如同露水下在地上一般,连他带跟随他的人,一个也不留下。
\VS{13}他若进了哪一座城,{\PN{以色列}}众人必带绳子去,将那城拉到河里,甚至连一块小石头都不剩下。」
\VS{14}{\PN{押沙龙}}和{\PN{以色列}}众人说:「{\PN{亚基}}人{\PN{户筛}}的计谋比{\PN{亚希多弗}}的计谋更好!」这是因耶和华定意破坏{\PN{亚希多弗}}的良谋,为要降祸与{\PN{押沙龙}}。
\par }{\SH 大卫因接获警告而逃脱
\par }{\PP \VS{15}{\PN{户筛}}对祭司{\PN{撒督}}和{\PN{亚比亚他}}说: 「{\PN{亚希多弗}}为{\PN{押沙龙}}和{\PN{以色列}}的长老所定的计谋是如此如此,我所定的计谋是如此如此。
\VS{16}现在你们要急速打发人去,告诉{\PN{大卫}}说:『今夜不可住在旷野的渡口,务要过河,免得王和跟随他的人都被吞灭。』」
\par }{\PP \VS{17}那时,{\PN{约拿单}}和{\PN{亚希玛斯}}在{\PN{隐·罗结}}那里等候,不敢进城,恐怕被人看见。有一个使女出来,将这话告诉他们,他们就去报信给{\PN{大卫}}王。
\VS{18}然而有一个童子看见他们,就去告诉{\PN{押沙龙}}。他们急忙跑到{\PN{巴户琳}}某人的家里;那人院中有一口井,他们就下到井里。
\VS{19}那家的妇人用盖盖上井口,又在上头铺上碎麦,事就没有泄漏。
\VS{20}{\PN{押沙龙}}的仆人来到那家,问妇人说:「{\PN{亚希玛斯}}和{\PN{约拿单}}在哪里?」妇人说:「他们过了河了。」仆人找他们,找不着,就回{\PN{耶路撒冷}}去了。
\par }{\PP \VS{21}他们走后,二人从井里上来,去告诉{\PN{大卫}}王说:「{\PN{亚希多弗}}如此如此定计害你,你们务要起来,快快过河。」
\VS{22}于是{\PN{大卫}}和跟随他的人都起来,过{\PN{约旦河}}。到了天亮,无一人不过{\PN{约旦河}}的。
\VS{23}{\PN{亚希多弗}}见不依从他的计谋,就备上驴,归回本城;到了家,留下遗言,便吊死了,葬在他父亲的坟墓里。
\par }{\PP \VS{24}{\PN{大卫}}到了{\PN{玛哈念}},{\PN{押沙龙}}和跟随他的{\PN{以色列}}人也都过了{\PN{约旦河}}。
\VS{25}{\PN{押沙龙}}立{\PN{亚玛撒}}作元帅,代替{\PN{约押}}。{\PN{亚玛撒}}是{\ADD{
{\PN{以实玛利}}}}人\FTNT{}{{\FR 17:25: }又作以色列人}{\PN{以特拉}}的儿子。{\PN{以特拉}}曾与{\PN{拿辖}}的女儿{\PN{亚比该}}亲近;这{\PN{亚比该}}与{\PN{约押}}的母亲{\PN{洗鲁雅}}是姊妹。
\VS{26}{\PN{押沙龙}}和{\PN{以色列}}人都安营在{\PN{基列}}地。
\par }{\PP \VS{27}{\PN{大卫}}到了{\PN{玛哈念}},{\PN{亚扪}}族的{\PN{拉巴}}人{\PN{拿辖}}的儿子{\PN{朔比}},{\PN{罗·底巴}}人{\PN{亚米利}}的儿子{\PN{玛吉}},{\PN{基列}}的{\PN{罗基琳}}人{\PN{巴西莱}},
\VS{28}带着被、褥、盆、碗、瓦器、小麦、大麦、麦面、炒{\ADD{谷}}、豆子、红豆、炒{\ADD{豆}}、
\VS{29}蜂蜜、奶油、绵羊、奶饼,供给{\PN{大卫}}和跟随他的人吃;他们说:「民在旷野,必饥渴困乏了。」

\par }\Chap{18}{\SH 押沙龙败亡
\par }{\PP \VerseOne{1}{\PN{大卫}}数点跟随他的人,立千夫长、百夫长率领他们。
\VS{2}{\PN{大卫}}打发军兵出战,分为三队:一队在{\PN{约押}}手下,一队在{\PN{洗鲁雅}}的儿子、{\PN{约押}}兄弟{\PN{亚比筛}}手下,一队在{\PN{迦特}}人{\PN{以太}}手下。{\PN{大卫}}对军兵说:「我必与你们一同出战。」
\VS{3}军兵却说:「你不可出战。若是我们逃跑,敌人必不介意;我们阵亡一半,敌人也不介意。因为你一人强似我们万人,你不如在城里预备帮助我们。」
\VS{4}王向他们说:「你们以为怎样好,我就怎样行。」于是王站在城门旁,军兵或百或千地挨次出去了。
\VS{5}王嘱咐{\PN{约押}}、{\PN{亚比筛}}、{\PN{以太}}说:「你们要为我的缘故宽待那少年人{\PN{押沙龙}}。」王为{\PN{押沙龙}}嘱咐众将的话,兵都听见了。
\par }{\PP \VS{6}兵就出到田野迎着{\PN{以色列}}人,在{\PN{以法莲}}树林里交战。
\VS{7}{\PN{以色列}}人败在{\PN{大卫}}的仆人面前;那日阵亡的甚多,共有二万人。
\VS{8}因为在那里四面打仗,死于树林的比死于刀剑的更多。
\par }{\PP \VS{9}{\PN{押沙龙}}偶然遇见{\PN{大卫}}的仆人。{\PN{押沙龙}}骑着骡子,从大橡树密枝底下经过,他的头{\ADD{发}}被树枝绕住,就悬挂起来,所骑的骡子便离他去了。
\VS{10}有个人看见,就告诉{\PN{约押}}说:「我看见{\PN{押沙龙}}挂在橡树上了。」
\VS{11}{\PN{约押}}对报信的人说:「你既看见他,为什么不将他打死落在地上呢?{\ADD{你若打死他}},我就赏你十{\ADD{舍客勒}}银子,一条带子。」
\VS{12}那人对{\PN{约押}}说:「我就是得你一千{\ADD{舍客勒}}银子,我也不敢伸手害王的儿子;因为我们听见王嘱咐你和{\PN{亚比筛}}并{\PN{以太}}说:『你们要谨慎,不可害那少年人{\PN{押沙龙}}。』
\VS{13}我若妄为害了他的性命,就是你自己也必与我为敌(原来,无论何事都瞒不过王。)」
\VS{14}{\PN{约押}}说:「我不能与你留连。」{\PN{约押}}手拿三杆短枪,趁{\PN{押沙龙}}在橡树上还活着,就刺透他的心。
\VS{15}给{\PN{约押}}拿兵器的十个少年人围绕{\PN{押沙龙}},将他杀死。
\par }{\PP \VS{16}{\PN{约押}}吹角,拦阻众人,他们就回来,不再追赶{\PN{以色列}}人。
\VS{17}他们将{\PN{押沙龙}}丢在林中一个大坑里,上头堆起一大堆石头。{\PN{以色列}}众人都逃跑,各回各家去了。
\par }{\PP \VS{18}{\PN{押沙龙}}活着的时候,在{\PN{王谷}}立了一根{\ADD{石}}柱,因他说:「我没有儿子为我留名。」他就以自己的名称那{\ADD{石}}柱叫{\PN{押沙龙柱}},直到今日。
\par }{\SH 大卫听闻押沙龙死讯
\par }{\PP \VS{19}{\PN{撒督}}的儿子{\PN{亚希玛斯}}说:「容我跑去,将耶和华向仇敌给王报仇的信息报与王知。」
\VS{20}{\PN{约押}}对他说:「你今日不可去报信,改日可以报信;因为今日王的儿子死了,所以你不可去报信。」
\VS{21}{\PN{约押}}对{\PN{古示}}人说:「你去将你所看见的告诉王。」{\PN{古示}}人在{\PN{约押}}面前下拜,就跑去了。
\VS{22}{\PN{撒督}}的儿子{\PN{亚希玛斯}}又对{\PN{约押}}说:「无论怎样,求你容我随着{\PN{古示}}人跑去。」{\PN{约押}}说:「我儿,你报这信息,既不得赏赐,何必要跑去呢?」
\VS{23}他{\ADD{又说}}:「无论怎样,我要跑去。」{\PN{约押}}说:「你跑去吧!」{\PN{亚希玛斯}}就从平原往前跑,跑过{\PN{古示}}人去了。
\par }{\PP \VS{24}{\PN{大卫}}正坐在城瓮里。守望的人上城门楼的顶上,举目观看,见有一个人独自跑来。
\VS{25}守望的人就大声告诉王。王说:「他若独自来,必是报口信的。」那人跑得渐渐近了。
\VS{26}守望的人又见一人跑来,就对守城门的人说:「又有一人独自跑来。」王说:「这也必是报信的。」
\VS{27}守望的人说:「我看前头人的跑法,好像{\PN{撒督}}的儿子{\PN{亚希玛斯}}的跑法一样。」王说:「他是个好人,必是报好信息。」
\par }{\PP \VS{28}{\PN{亚希玛斯}}向王呼叫说:「平安了!」就在王面前脸伏于地叩拜,说:「耶和华—你的 神是应当称颂的,因他已将那举手攻击我主我王的人交给王了。」
\VS{29}王问说:「少年人{\PN{押沙龙}}平安不平安?」{\PN{亚希玛斯}}回答说:「{\PN{约押}}打发王的仆人,那时仆人听见众民大声喧哗,却不知道是什么事。」
\VS{30}王说:「你退去,站在旁边。」他就退去,站在旁边。
\par }{\PP \VS{31}{\PN{古示}}人也来到,说:「有信息报给我主我王!耶和华今日向一切兴起攻击你的人给你报仇了。」
\VS{32}王问{\PN{古示}}人说:「少年人{\PN{押沙龙}}平安不平安?」{\PN{古示}}人回答说:「愿我主我王的仇敌,和一切兴起要杀害你的人,都与那少年人一样。」
\VS{33}王就心里伤恸,上城门楼去哀哭,一面走一面说:「我儿{\PN{押沙龙}}啊!我儿,我儿{\PN{押沙龙}}啊!我恨不得替你死,{\PN{押沙龙}}啊,我儿!我儿!」

\par }\Chap{19}{\SH 约押劝谏王
\par }{\PP \VerseOne{1}有人告诉{\PN{约押}}说:「王为{\PN{押沙龙}}哭泣悲哀。」
\VS{2}众民听说王为他儿子忧愁,他们得胜的欢乐却变成悲哀。
\VS{3}那日众民暗暗地进城,就如败阵逃跑、惭愧的民一般。
\VS{4}王蒙着脸,大声哭号说:「我儿{\PN{押沙龙}}啊!{\PN{押沙龙}},我儿,我儿啊!」
\VS{5}{\PN{约押}}进去见王,说:「你今日使你一切仆人脸面惭愧了!他们今日救了你的性命和你儿女妻妾的性命,
\VS{6}你却爱那恨你的人,恨那爱你的人。你今日明明地不以将帅、仆人为念。我今日看明,若{\PN{押沙龙}}活着,我们都死亡,你就喜悦了。
\VS{7}现在你当出去,安慰你仆人的心。我指着耶和华起誓:你若不出去,今夜必无一人与你同在一处;这祸患就比你从幼年到如今所遭的更甚!」
\VS{8}于是王起来,坐在城门口。众民听说王坐在城门口,就都到王面前。
\par }{\SH 大卫回耶路撒冷
\par }{\PP {\PN{以色列}}人已经逃跑,各回各家去了。
\VS{9}{\PN{以色列}}众支派的人纷纷议论说:「王曾救我们脱离仇敌的手,又救我们脱离{\PN{非利士}}人的手,现在他躲避{\PN{押沙龙}}逃走了。
\VS{10}我们膏{\PN{押沙龙}}治理我们,他已经阵亡。现在为什么不出一言请王回来呢?」
\par }{\PP \VS{11}{\PN{大卫}}王差人去见祭司{\PN{撒督}}和{\PN{亚比亚他}},说:「你们当向{\PN{犹大}}长老说:『{\PN{以色列}}众人已经有话请王回宫,你们为什么落在他们后头呢?
\VS{12}你们是我的弟兄,是我的骨肉,为什么在人后头请王回来呢?』
\VS{13}也要对{\PN{亚玛撒}}说:『你不是我的骨肉吗?我若不立你替{\PN{约押}}常作元帅,愿 神重重地降罚与我!』」
\VS{14}如此就挽回{\PN{犹大}}众人的心,如同一人的心。他们便打发人去见王,说:「请王和王的一切臣仆回来。」
\par }{\PP \VS{15}王就回来,到了{\PN{约旦河}}。{\PN{犹大}}人来到{\PN{吉甲}},要去迎接王,请他过{\PN{约旦河}}。
\VS{16}{\PN{巴户琳}}的{\PN{便雅悯}}人、{\PN{基拉}}的儿子{\PN{示每}}急忙与{\PN{犹大}}人一同下去迎接{\PN{大卫}}王。
\VS{17}跟从{\PN{示每}}的有一千{\PN{便雅悯}}人,还有{\PN{扫罗}}家的仆人{\PN{洗巴}}和他十五个儿子,二十个仆人;他们都趟过{\PN{约旦河}}迎接王。
\VS{18}有摆渡船过去,渡王的家眷,任王使用。
\par }{\SH 大卫宽恕示每
\par }{\PP 王要过{\PN{约旦河}}的时候,{\PN{基拉}}的儿子{\PN{示每}}就俯伏在王面前,
\VS{19}对王说:「我主我王出{\PN{耶路撒冷}}的时候,仆人行悖逆的事,现在求我主不要因此加罪与仆人,不要记念,也不要放在心上。
\VS{20}仆人明知自己有罪,所以{\PN{约瑟}}全家之中,今日我首先下来迎接我主我王。」
\par }{\PP \VS{21}{\PN{洗鲁雅}}的儿子{\PN{亚比筛}}说:「{\PN{示每}}既咒骂耶和华的受膏者,不应当治死他吗?」
\VS{22}{\PN{大卫}}说:「{\PN{洗鲁雅}}的儿子,我与你们有何关涉,使你们今日与我反对呢?今日在{\PN{以色列}}中岂可治死人呢?我岂不知今日我作{\PN{以色列}}的王吗?」
\VS{23}于是王对{\PN{示每}}说:「你必不死。」王就向他起誓。
\par }{\SH 大卫善待米非波设
\par }{\PP \VS{24}{\PN{扫罗}}的孙子{\PN{米非波设}}也下去迎接王。他自从王去的日子,直到王平平安安地回来,没有修脚,没有剃胡须,也没有洗衣服。
\VS{25}他来到{\PN{耶路撒冷}}迎接王的时候,王问他说:「{\PN{米非波设}},你为什么没有与我同去呢?」
\VS{26}他回答说:「我主我王,仆人是瘸腿的。那日我想要备驴骑上,与王同去,无奈我的仆人欺哄了我,
\VS{27}又在我主我王面前谗毁我。然而我主我王如同 神的使者一般,你看怎样好,就怎样行吧!
\VS{28}因为我祖全家的人,在我主我王面前都算为死人,王却使仆人在王的席上同人吃饭,我现在向王还能辨理诉冤吗?」
\VS{29}王对他说:「你何必再提你的事呢?我说,你与{\PN{洗巴}}均分地土。」
\VS{30}{\PN{米非波设}}对王说:「我主我王既平平安安地回宫,就任凭{\PN{洗巴}}都取了也可以。」
\par }{\SH 大卫善待巴西莱
\par }{\PP \VS{31}{\PN{基列}}人{\PN{巴西莱}}从{\PN{罗基琳}}下来,要送王过{\PN{约旦河}},就与王一同过了{\PN{约旦河}}。
\VS{32}{\PN{巴西莱}}年纪老迈,已经八十岁了。王住在{\PN{玛哈念}}的时候,他就拿食物来供给王;他原是大富户。
\VS{33}王对{\PN{巴西莱}}说:「你与我同去,我要在{\PN{耶路撒冷}}那里养你的老。」
\VS{34}{\PN{巴西莱}}对王说:「我在世的年日还能有多少,使我与王同上{\PN{耶路撒冷}}呢?
\VS{35}仆人现在八十岁了,还能尝出饮食的滋味、辨别美恶吗?还能听男女歌唱的声音吗?仆人何必累赘我主我王呢?
\VS{36}仆人只要送王过{\PN{约旦河}},王何必赐我这样的恩典呢?
\VS{37}求你准我回去,好死在我本城,{\ADD{葬在}}我父母的墓旁。这里有王的仆人{\PN{金罕}},让他同我主我王过去,可以随意待他。」
\VS{38}王说:「{\PN{金罕}}可以与我同去,我必照你的心愿待他。你向我求什么,我都必为你成就。」
\VS{39}于是众民过{\PN{约旦河}},王也过去。王与{\PN{巴西莱}}亲嘴,为他祝福,{\PN{巴西莱}}就回本地去了。
\par }{\SH 以色列人和犹大人为王起争论
\par }{\PP \VS{40}王过去,到了{\PN{吉甲}},{\PN{金罕}}也跟他过去。{\PN{犹大}}众民和{\PN{以色列}}民的一半也都送王过去。
\VS{41}{\PN{以色列}}众人来见王,对他说:「我们弟兄{\PN{犹大}}人为什么暗暗送王和王的家眷,并跟随王的人过{\PN{约旦河}}?」
\VS{42}{\PN{犹大}}众人回答{\PN{以色列}}人说:「因为王与我们是亲属,你们为何因这事发怒呢?我们吃了王的什么呢?王赏赐了我们什么呢?」
\VS{43}{\PN{以色列}}人回答{\PN{犹大}}人说:「按支派,我们与王有十分的情分;在{\PN{大卫}}身上,我们也比你们更有情分。你们为何藐视我们,请王回来不先与我们商量呢?」
\par }{\PP 但{\PN{犹大}}人的话比{\PN{以色列}}人的话更硬。

\par }\Chap{20}{\SH 示巴反叛
\par }{\PP \VerseOne{1}在那里恰巧有一个匪徒,名叫{\PN{示巴}},是{\PN{便雅悯}}人{\PN{比基利}}的儿子。他吹角,说:「我们与{\PN{大卫}}无分,与{\PN{耶西}}的儿子无涉。{\PN{以色列}}人哪,你们各回各家去吧!」
\VS{2}于是{\PN{以色列}}人都离开{\PN{大卫}},跟随{\PN{比基利}}的儿子{\PN{示巴}}。但{\PN{犹大}}人从{\PN{约旦河}}直到{\PN{耶路撒冷}},都紧紧跟随他们的王。
\par }{\PP \VS{3}{\PN{大卫}}王来到{\PN{耶路撒冷}},进了宫殿,就把从前留下看守宫殿的十个妃嫔禁闭在{\ADD{冷宫}},养活她们,不与她们亲近。她们如同寡妇被禁,直到死的日子。
\par }{\PP \VS{4}王对{\PN{亚玛撒}}说:「你要在三日之内将{\PN{犹大}}人招聚了来,你也回到这里来。」
\VS{5}{\PN{亚玛撒}}就去招聚{\PN{犹大}}人,却耽延过了王所限的日期。
\VS{6}{\PN{大卫}}对{\PN{亚比筛}}说:「现在恐怕{\PN{比基利}}的儿子{\PN{示巴}}加害于我们比{\PN{押沙龙}}更甚。你要带领你主的仆人追赶他,免得他得了坚固城,躲避我们。」
\VS{7}{\PN{约押}}的人和{\PN{基利提}}人、{\PN{比利提}}人,并所有的勇士,都跟着{\PN{亚比筛}},从{\PN{耶路撒冷}}出去追赶{\PN{比基利}}的儿子{\PN{示巴}}。
\VS{8}他们到了{\PN{基遍}}的大磐石那里,{\PN{亚玛撒}}来迎接他们。那时{\PN{约押}}穿着战衣,腰束佩刀的带子,刀在鞘内;{\PN{约押}}前行,刀从鞘内掉出来。
\VS{9}{\PN{约押}}{\ADD{左手拾起刀来}},对{\PN{亚玛撒}}说:「我兄弟,你好啊!」就用右手抓住{\PN{亚玛撒}}的胡子,要与他亲嘴。
\VS{10}{\PN{亚玛撒}}没有防备{\PN{约押}}手里所拿的刀;{\PN{约押}}用刀刺入他的肚腹,他的肠子流在地上,没有再刺他,就死了。
{\PN{约押}}和他兄弟{\PN{亚比筛}}往前追赶{\PN{比基利}}的儿子{\PN{示巴}}。
\par }{\PP \VS{11}有{\PN{约押}}的一个少年人站在{\PN{亚玛撒}}尸身旁边,{\ADD{对众人}}说:「谁喜悦{\PN{约押}},谁归顺{\PN{大卫}},就当跟随{\PN{约押}}去。」
\VS{12}{\PN{亚玛撒}}在道路上滚在自己的血里。那人见众民{\ADD{经过}}都站住,就把{\PN{亚玛撒}}的尸身从路上挪到田间,用衣服遮盖。
\VS{13}尸身从路上挪移之后,众民就都跟随{\PN{约押}}去追赶{\PN{比基利}}的儿子{\PN{示巴}}。
\par }{\PP \VS{14}他走遍{\PN{以色列}}各支派,直到{\PN{伯·玛迦}}的{\PN{亚比拉}},并{\PN{比利}}人的全地;那些地方的人也都聚集跟随他。
\VS{15}{\PN{约押}}和跟随的人到了{\PN{伯·玛迦}}的{\PN{亚比拉}},围困{\PN{示巴}},就对着城筑垒;跟随{\PN{约押}}的众民{\ADD{用锤}}撞城,要使城塌陷。
\VS{16}有一个聪明妇人从城上呼叫说:「听啊,听啊,请{\PN{约押}}近前来,我好与他说话。」
\VS{17}{\PN{约押}}就近前来,妇人问他说:「你是{\PN{约押}}不是?」他说:「我是。」妇人说:「求你听婢女的话。」{\PN{约押}}说:「我听。」
\VS{18}妇人说:「古时有话说,当先在{\PN{亚比拉}}求问,然后事就定妥。
\VS{19}我们这城的人在{\PN{以色列}}人中是和平、忠厚的。你为何要毁坏{\PN{以色列}}中的大城,吞灭耶和华的产业呢?」
\VS{20}{\PN{约押}}回答说:「我决不吞灭毁坏,
\VS{21}乃因{\PN{以法莲}}山地的一个人—{\PN{比基利}}的儿子{\PN{示巴}}—举手攻击{\PN{大卫}}王,你们若将他一人交出来,我便离城而去。」妇人对{\PN{约押}}说:「那人的首级必从城墙上丢给你。」
\VS{22}妇人就凭她的智慧去劝众人。他们便割下{\PN{比基利}}的儿子{\PN{示巴}}的首级,丢给{\PN{约押}}。{\PN{约押}}吹角,众人就离城而散,各归各家去了。{\PN{约押}}回{\PN{耶路撒冷}},到王那里。
\par }{\PP \VS{23}{\PN{约押}}作{\PN{以色列}}全军的元帅;{\PN{耶何耶大}}的儿子{\PN{比拿雅}}统辖{\PN{基利提}}人和{\PN{比利提}}人;
\VS{24}{\PN{亚多兰}}掌管服苦的人;{\PN{亚希律}}的儿子{\PN{约沙法}}作史官;
\VS{25}{\PN{示法}}作书记;{\PN{撒督}}和{\PN{亚比亚他}}作祭司{\ADD{长}};
\VS{26}{\PN{睚珥}}人{\PN{以拉}}作{\PN{大卫}}的宰相。

\par }\Chap{21}{\SH 扫罗的后代被处死
\par }{\PP \VerseOne{1}{\PN{大卫}}年间有饥荒,一连三年,{\PN{大卫}}就求问耶和华。耶和华说:「这饥荒是因{\PN{扫罗}}和他流人血之家杀死{\PN{基遍}}人。」
\VS{2}原来这{\PN{基遍}}人不是{\PN{以色列}}人,乃是{\PN{亚摩利}}人中所剩的;{\PN{以色列}}人曾向他们起誓,{\ADD{不杀灭他们}},{\PN{扫罗}}却为{\PN{以色列}}人和{\PN{犹大}}人发热心,想要杀灭他们。{\PN{大卫}}王召了他们来,
\VS{3}问他们说:「我当为你们怎样行呢?可用什么赎这罪,使你们为耶和华的产业祝福呢?」
\VS{4}{\PN{基遍}}人回答说:「我们和{\PN{扫罗}}与他家的事并不关乎金银,也不要因我们的缘故杀一个{\PN{以色列}}人。」{\PN{大卫}}说:「你们怎样说,我就为你们怎样行。」
\VS{5}他们对王说:「那从前谋害我们、要灭我们、使我们不得再住{\PN{以色列}}境内的人,
\VS{6}现在愿将他的子孙七人交给我们,我们好在耶和华面前,将他们悬挂在耶和华拣选{\PN{扫罗}}的{\PN{基比亚}}。」王说:「我必交给你们。」
\par }{\PP \VS{7}王因为曾与{\PN{扫罗}}的儿子{\PN{约拿单}}指着耶和华起誓{\ADD{结盟}},就爱惜{\PN{扫罗}}的孙子、{\PN{约拿单}}的儿子{\PN{米非波设}},{\ADD{不交出来}},
\VS{8}却把{\PN{爱雅}}的女儿{\PN{利斯巴}}给{\PN{扫罗}}所生的两个儿子{\PN{亚摩尼}}、{\PN{米非波设}},和{\PN{扫罗}}女儿{\PN{米甲}}{\ADD{的姊姊}}给{\PN{米何拉}}人{\PN{巴西莱}}儿子{\PN{亚得列}}所生的五个儿子
\VS{9}交在{\PN{基遍}}人的手里。{\PN{基遍}}人就把他们,在耶和华面前,悬挂在山上,这七人就一同死亡。被杀的时候正是收割的日子,就是动手割大麦的时候。
\par }{\PP \VS{10}{\PN{爱雅}}的女儿{\PN{利斯巴}}用麻布在磐石上搭棚,从动手收割的时候直到天降雨在尸身上的时候,日间不容空中的雀鸟落在尸身上,夜间不让田野的走兽前来糟践。
\par }{\PP \VS{11}有人将{\PN{扫罗}}的妃嫔{\PN{爱雅}}女儿{\PN{利斯巴}}所行的这事告诉{\PN{大卫}}。
\VS{12}{\PN{大卫}}就去,从{\PN{基列·雅比}}人那里将{\PN{扫罗}}和他儿子{\PN{约拿单}}的骸骨搬了来(是因{\PN{非利士}}人从前在{\PN{基利波}}杀{\PN{扫罗}},将尸身悬挂在{\PN{伯·珊}}的街市上,{\PN{基列·雅比}}人把尸身偷了去。)
\VS{13}{\PN{大卫}}将{\PN{扫罗}}和他儿子{\PN{约拿单}}的骸骨从那里搬了来,又收殓被悬挂七人的骸骨,
\VS{14}将{\PN{扫罗}}和他儿子{\PN{约拿单}}的骸骨葬在{\PN{便雅悯}}的{\PN{洗拉}},在{\PN{扫罗}}父亲{\PN{基士}}的坟墓里;众人行了王所吩咐的。此后 神垂听国民所求的。
\par }{\SH 攻打非利士巨人
\par }{\R (代上20·4—8)
\par }{\PP \VS{15}{\PN{非利士}}人与{\PN{以色列}}人打仗;{\PN{大卫}}带领仆人下去,与{\PN{非利士}}人接战,{\PN{大卫}}就疲乏了。
\VS{16}伟人的一个儿子{\PN{以实·比诺}}要杀{\PN{大卫}};他的铜枪重三百{\ADD{舍客勒}},又佩着新{\ADD{刀}}。
\VS{17}但{\PN{洗鲁雅}}的儿子{\PN{亚比筛}}帮助{\PN{大卫}},攻打{\PN{非利士}}人,将他杀死。当日,跟随{\PN{大卫}}的人向{\PN{大卫}}起誓说:「以后你不可再与我们一同出战,恐怕熄灭{\PN{以色列}}的灯。」
\par }{\PP \VS{18}后来,{\PN{以色列}}人在{\PN{歌伯}}与{\PN{非利士}}人打仗,{\PN{户沙}}人{\PN{西比该}}杀了伟人的一个儿子{\PN{撒弗}}。
\VS{19}又在{\PN{歌伯}}与{\PN{非利士}}人打仗,{\PN{伯利恒}}人{\PN{雅雷俄珥金}}的儿子{\PN{伊勒哈难}}杀了{\PN{迦特}}人{\PN{歌利亚}}。这人的枪杆粗如织布的机轴。
\VS{20}又在{\PN{迦特}}打仗,那里有一个身量高大的人,手脚都是六指,共有二十四个指头;他也是伟人的儿子。
\VS{21}这人向{\PN{以色列}}人骂阵,{\PN{大卫}}的哥哥{\PN{示米亚}}的儿子{\PN{约拿单}}就杀了他。
\VS{22}这四个人是{\PN{迦特}}伟人的儿子,都死在{\PN{大卫}}和他仆人的手下。

\par }\Chap{22}{\SH 大卫的凯歌
\par }{\R (诗18)
\par }{\PP \VerseOne{1}当耶和华救{\PN{大卫}}脱离一切仇敌和{\PN{扫罗}}之手的日子,他向耶和华念这诗,
\VS{2}说:
\par }{\Q 耶和华是我的岩石,
\par }{\Q 我的山寨,我的救主,
\par }{\Q \VS{3}我的 神,我的磐石,我所投靠的。
\par }{\Q 他是我的盾牌,是拯救我的角,
\par }{\Q 是我的高台,是我的避难所。
\par }{\Q 我的救主啊,你是救我脱离强暴的。
\par }{\Q \VS{4}我要求告当赞美的耶和华,
\par }{\Q 这样,我必从仇敌手中被救出来。
\par }{\BB \par }{\Q \VS{5}曾有死亡的波浪环绕我,
\par }{\Q 匪类的急流使我惊惧,
\par }{\Q \VS{6}阴间的绳索缠绕我,
\par }{\Q 死亡的网罗临到我。
\par }{\BB \par }{\Q \VS{7}我在急难中求告耶和华,
\par }{\Q 向我的 神呼求。
\par }{\Q 他从殿中听了我的声音;
\par }{\Q 我的呼求入了他的耳中。
\par }{\BB \par }{\Q \VS{8}那时因他发怒,地就摇撼战抖;
\par }{\Q 天的根基也震动摇撼。
\par }{\Q \VS{9}从他鼻孔冒烟上腾;
\par }{\Q 从他口中发火焚烧,连炭也着了。
\par }{\Q \VS{10}他又使天下垂,亲自降临;
\par }{\Q 有黑云在他脚下。
\par }{\Q \VS{11}他坐着基路伯飞行,
\par }{\Q 在风的翅膀上显现。
\par }{\Q \VS{12}他以黑暗和聚集的水、
\par }{\Q 天空的厚云为他四围的行宫。
\par }{\Q \VS{13}因他面前的光辉炭都着了。
\par }{\Q \VS{14}耶和华从天上打雷;
\par }{\Q 至高者发出声音。
\par }{\Q \VS{15}他射出箭来,使仇敌四散,
\par }{\Q 发出闪电,使他们扰乱。
\par }{\Q \VS{16}耶和华的斥责一发,鼻孔的气一出,
\par }{\Q 海底就出现,大地的根基也显露。
\par }{\BB \par }{\Q \VS{17}他从高天伸手抓住我,
\par }{\Q 把我从大水中拉上来。
\par }{\Q \VS{18}他救我脱离我的劲敌和那些恨我的人,
\par }{\Q 因为他们比我强盛。
\par }{\Q \VS{19}我遭遇灾难的日子,他们来攻击我;
\par }{\Q 但耶和华是我的倚靠。
\par }{\Q \VS{20}他又领我到宽阔之处;
\par }{\Q 他救拔我,因他喜悦我。
\par }{\BB \par }{\Q \VS{21}耶和华按着我的公义报答我,
\par }{\Q 按着我手中的清洁赏赐我。
\par }{\Q \VS{22}因为我遵守了耶和华的道,
\par }{\Q 未曾作恶离开我的 神。
\par }{\BB \par }{\Q \VS{23}他的一切典章常在我面前;
\par }{\Q 他的律例,我也未曾离弃。
\par }{\Q \VS{24}我在他面前作了完全人;
\par }{\Q 我也保守自己远离我的罪孽。
\par }{\Q \VS{25}所以耶和华按我的公义,
\par }{\Q 按我在他眼前的清洁赏赐我。
\par }{\BB \par }{\Q \VS{26}慈爱的人,你以慈爱待他;
\par }{\Q 完全的人,你以完全待他;
\par }{\Q \VS{27}清洁的人,你以清洁待他;
\par }{\Q 乖僻的人,你以弯曲待他。
\par }{\Q \VS{28}困苦的百姓,你必拯救;
\par }{\Q 但你的眼目察看高傲的人,使他降卑。
\par }{\Q \VS{29}耶和华啊,你是我的灯;
\par }{\Q 耶和华必照明我的黑暗。
\par }{\Q \VS{30}我借着你冲入敌军,
\par }{\Q 借着我的 神跳过墙垣。
\par }{\Q \VS{31}至于 神,他的道是完全的;
\par }{\Q 耶和华的话是炼净的。
\par }{\Q 凡投靠他的,他便作他们的盾牌。
\par }{\BB \par }{\Q \VS{32}除了耶和华,谁是 神呢?
\par }{\Q 除了我们的 神,谁是磐石呢?
\par }{\Q \VS{33}神是我坚固的保障;
\par }{\Q 他引导完全人行他的路。
\par }{\Q \VS{34}他使我的脚快如母鹿的{\ADD{蹄}},
\par }{\Q 又使我在高处安稳。
\par }{\Q \VS{35}他教导我的手能以争战,
\par }{\Q 甚至我的膀臂能开铜弓。
\par }{\Q \VS{36}你把你的救恩给我作盾牌;
\par }{\Q 你的温和使我为大。
\par }{\Q \VS{37}你使我脚下的地步宽阔;
\par }{\Q 我的脚未曾滑跌。
\par }{\Q \VS{38}我追赶我的仇敌,灭绝了他们,
\par }{\Q 未灭以先,我没有归回。
\par }{\Q \VS{39}我灭绝了他们,
\par }{\Q 打伤了他们,使他们不能起来;
\par }{\Q 他们都倒在我的脚下。
\par }{\Q \VS{40}因为你曾以力量束我的腰,使我能争战;
\par }{\Q 你也使那起来攻击我的都服在我以下。
\par }{\Q \VS{41}你又使我的仇敌在我面前转背逃跑,
\par }{\Q 叫我能以剪除那恨我的人。
\par }{\Q \VS{42}他们仰望,却无人拯救;
\par }{\Q 就是呼求耶和华,他也不应允。
\par }{\Q \VS{43}我捣碎他们,如同地上的灰尘,
\par }{\Q 践踏他们,四散在地,如同街上的泥土。
\par }{\Q \VS{44}你救我脱离我百姓的争竞,
\par }{\Q 保护我作列国的元首;
\par }{\Q 我素不认识的民必事奉我。
\par }{\Q \VS{45}外邦人要投降我,
\par }{\Q 一听见我的名声就必顺从我。
\par }{\Q \VS{46}外邦人要衰残,
\par }{\Q 战战兢兢地出他们的营寨。
\par }{\BB \par }{\Q \VS{47}耶和华是活神,愿我的磐石被人称颂!
\par }{\Q 愿 神—那拯救我的磐石被人尊崇!
\par }{\Q \VS{48}这位 神就是那为我伸冤、
\par }{\Q 使众民服在我以下的。
\par }{\Q \VS{49}你救我脱离仇敌,
\par }{\Q 又把我举起,高过那些起来攻击我的;
\par }{\Q 你救我脱离强暴的人。
\par }{\BB \par }{\Q \VS{50}耶和华啊,因此我要在外邦中称谢你,
\par }{\Q 歌颂你的名。
\par }{\Q \VS{51}耶和华赐极大的救恩给他{\ADD{所立}}的王,
\par }{\Q 施慈爱给他的受膏者,
\par }{\Q 就是给{\PN{大卫}}和他的后裔,
\par }{\Q 直到永远!

\par }\Chap{23}{\SH 大卫的遗言
\par }{\PP \VerseOne{1}以下是{\PN{大卫}}末了的话。{\PN{耶西}}的儿子{\PN{大卫}}得居高位,是{\PN{雅各}} 神所膏的,作{\PN{以色列}}的美歌者,说:
\par }{\Q \VS{2}耶和华的灵借着我说:
\par }{\Q 他的话在我口中。
\par }{\Q \VS{3}{\PN{以色列}}的 神、
\par }{\Q {\PN{以色列}}的磐石晓谕我说:
\par }{\Q 那以公义治理人民的,
\par }{\Q 敬畏 神执掌权柄,
\par }{\Q \VS{4}他必像日出的晨光,
\par }{\Q 如无云的清晨,
\par }{\Q 雨后的晴光,
\par }{\Q 使地{\ADD{发生}}嫩草。
\par }{\Q \VS{5}我家在 神面前并非如此;
\par }{\Q  神却与我立永远的约。
\par }{\Q 这约凡事坚稳,
\par }{\Q 关乎我的一切救恩和我一切所想望的,
\par }{\Q 他岂不为我成就吗?
\par }{\Q \VS{6}但匪类都必像荆棘被丢弃;
\par }{\Q 人不敢用手拿它;
\par }{\Q \VS{7}拿它的人必带铁器和枪杆,
\par }{\Q 终久它必被火焚烧。
\par }{\SH 大卫的勇士
\par }{\R (代上11·10—47)
\par }{\PP \VS{8}{\PN{大卫}}勇士的名字记在下面:{\PN{他革扪}}人{\PN{约设·巴设}},又称{\PN{伊斯尼}}人{\PN{亚底挪}},他是军长的统领,一时击杀了八百人。
\par }{\PP \VS{9}其次是{\PN{亚合}}人{\PN{朵多}}的儿子{\PN{以利亚撒}}。从前{\PN{非利士}}人聚集要打仗,{\PN{以色列}}人迎着上去,有跟随{\PN{大卫}}的三个勇士向{\PN{非利士}}人骂阵,其中有{\PN{以利亚撒}}。
\VS{10}他起来击杀{\PN{非利士}}人,直到手臂疲乏,手黏住刀把。那日耶和华使{\ADD{
{\PN{以色列}}}}人大获全胜;众民在{\PN{以利亚撒}}后头专夺财物。
\par }{\PP \VS{11}其次是{\PN{哈拉}}人{\PN{亚基}}的儿子{\PN{沙玛}}。一日,{\PN{非利士}}人聚集成群,在一块长满红豆的田里,众民就在{\PN{非利士}}人面前逃跑。
\VS{12}{\PN{沙玛}}却站在那田间击杀{\PN{非利士}}人,救护了那田。耶和华使{\ADD{
{\PN{以色列}}}}人大获全胜。
\par }{\PP \VS{13}收割的时候,有三十个勇士中的三个人下到{\PN{亚杜兰洞}}见{\PN{大卫}}。{\PN{非利士}}的军兵在{\PN{利乏音谷}}安营。
\VS{14}那时{\PN{大卫}}在山寨,{\PN{非利士}}人的防营在{\PN{伯利恒}}。
\VS{15}{\PN{大卫}}渴想,说:「甚愿有人将{\PN{伯利恒}}城门旁、井里的水打来给我喝。」
\VS{16}这三个勇士就闯过{\PN{非利士}}人的营盘,从{\PN{伯利恒}}城门旁的井里打水,拿来奉给{\PN{大卫}}。他却不肯喝,将水奠在耶和华面前,
\VS{17}说:「耶和华啊,这三个人冒死{\ADD{去打水}};{\ADD{这水}}好像他们的血一般,我断不敢喝。」如此,{\PN{大卫}}不肯喝。这是三个勇士所做的事。
\par }{\PP \VS{18}{\PN{洗鲁雅}}的儿子、{\PN{约押}}的兄弟{\PN{亚比筛}}是这三个勇士的首领;他举枪杀了三百人,就在三个勇士里得了名。
\VS{19}他在这三个勇士里是最尊贵的,所以作他们的首领,只是不及前三个勇士。
\par }{\PP \VS{20}有{\PN{甲薛}}勇士{\PN{耶何耶大}}的儿子{\PN{比拿雅}}行过大能的事;他杀了{\PN{摩押}}人{\PN{亚利伊勒}}的两个儿子,又在下雪的时候下坑里去,杀了一个狮子,
\VS{21}又杀了一个强壮的{\PN{埃及}}人;{\PN{埃及}}人手里拿着枪,{\PN{比拿雅}}只拿着棍子下去,从{\PN{埃及}}人手里夺过枪来,用那枪将他杀死。
\VS{22}这是{\PN{耶何耶大}}的儿子{\PN{比拿雅}}所行的事,就在三个勇士里得了名。
\VS{23}他比那三十个勇士都尊贵,只是不及前三个勇士。{\PN{大卫}}立他作护卫长。
\par }{\PP \VS{24}三十个勇士里有{\PN{约押}}的兄弟{\PN{亚撒黑}},{\PN{伯利恒}}人{\PN{朵多}}的儿子{\PN{伊勒哈难}},
\VS{25}{\PN{哈律}}人{\PN{沙玛}},{\PN{哈律}}人{\PN{以利加}},
\VS{26}{\PN{帕勒提}}人{\PN{希利斯}},{\PN{提哥亚}}人{\PN{益吉}}的儿子{\PN{以拉}},
\VS{27}{\PN{亚拿突}}人{\PN{亚比以谢}},{\PN{户沙}}人{\PN{米本乃}},
\VS{28}{\PN{亚合}}人{\PN{撒们}},{\PN{尼陀法}}人{\PN{玛哈莱}},
\VS{29}{\PN{尼陀法}}人{\PN{巴拿}}的儿子{\PN{希立}},{\PN{便雅悯}}族、{\PN{基比亚}}人{\PN{利拜}}的儿子{\PN{以太}},
\VS{30}{\PN{比拉顿}}人{\PN{比拿雅}},{\PN{迦实溪}}人{\PN{希太}},
\VS{31}{\PN{伯·亚拉巴}}人{\PN{亚比亚本}},{\PN{巴鲁米}}人{\PN{押斯玛弗}},
\VS{32}{\PN{沙本}}人{\PN{以利雅哈巴}},{\PN{雅善}}儿子中的{\PN{约拿单}},
\VS{33}{\PN{哈拉}}人{\PN{沙玛}},{\PN{哈拉}}人{\PN{沙拉}}的儿子{\PN{亚希暗}},
\VS{34}{\PN{玛迦}}人{\PN{亚哈拜}}的儿子{\PN{以利法列}},{\PN{基罗}}人{\PN{亚希多弗}}的儿子{\PN{以连}},
\VS{35}{\PN{迦密}}人{\PN{希斯莱}},{\PN{亚巴}}人{\PN{帕莱}},
\VS{36}{\PN{琐巴}}人{\PN{拿单}}的儿子{\PN{以甲}},{\PN{迦得}}人{\PN{巴尼}},
\VS{37}{\PN{亚扪}}人{\PN{洗勒}},{\PN{比录}}人{\PN{拿哈莱}}(是给{\PN{洗鲁雅}}的儿子{\PN{约押}}拿兵器的),
\VS{38}{\PN{以帖}}人{\PN{以拉}},{\PN{以帖}}人{\PN{迦立}},
\VS{39}{\PN{赫}}人{\PN{乌利亚}},共有三十七人。

\par }\Chap{24}{\SH 大卫调查人口
\par }{\R (代上21·1—27)
\par }{\PP \VerseOne{1}耶和华又向{\PN{以色列}}人发怒,就激动{\PN{大卫}},使他吩咐人去数点{\PN{以色列}}人和{\PN{犹大}}人。
\VS{2}{\PN{大卫}}就吩咐跟随他的元帅{\PN{约押}}说:「你去走遍{\PN{以色列}}众支派,从{\PN{但}}直到{\PN{别是巴}},数点百姓,我好知道他们的数目。」
\VS{3}{\PN{约押}}对王说:「无论百姓多少,愿耶和华—你的 神再加增百倍,使我主我王亲眼得见。我主我王何必喜悦行这事呢?」
\VS{4}但王的命令胜过{\PN{约押}}和众军长。{\PN{约押}}和众军长就从王面前出去,数点{\PN{以色列}}的百姓。
\VS{5}他们过了{\PN{约旦河}},在{\PN{迦得谷}}中、城的右边{\PN{亚罗珥}}安营,与{\PN{雅谢}}相对,
\VS{6}又到了{\PN{基列}}和{\PN{他停·合示}}地,又到了{\PN{但·雅安}},绕到{\PN{西顿}},
\VS{7}来到{\PN{泰尔}}的保障,并{\PN{希未}}人和{\PN{迦南}}人的各城,又到{\PN{犹大}}南方的{\PN{别是巴}}。
\VS{8}他们走遍全地,过了九个月零二十天,就回到{\PN{耶路撒冷}}。
\VS{9}{\PN{约押}}将百姓的总数奏告于王:{\PN{以色列}}拿刀的勇士有八十万;{\PN{犹大}}有五十万。
\par }{\PP \VS{10}{\PN{大卫}}数点百姓以后,就心中自责,祷告耶和华说:「我行这事大有罪了。耶和华啊,求你除掉仆人的罪孽,因我所行的甚是愚昧。」
\VS{11}{\PN{大卫}}早晨起来,耶和华的话临到先知{\PN{迦得}},就是{\PN{大卫}}的先见,说:
\VS{12}「你去告诉{\PN{大卫}},说耶和华如此说:『我有三样灾,随你选择一样,我好降与你。』」
\VS{13}于是{\PN{迦得}}来见{\PN{大卫}},对他说:「你愿意国中有七年的饥荒呢?是在你敌人面前逃跑,被追赶三个月呢?是在你国中有三日的瘟疫呢?现在你要揣摩思想,我好回复那差我来的。」
\VS{14}{\PN{大卫}}对{\PN{迦得}}说:「我甚为难!我愿落在耶和华的手里,因为他有丰盛的怜悯。我不愿落在人的手里。」
\par }{\PP \VS{15}于是,耶和华降瘟疫与{\PN{以色列}}人,自早晨到所定的时候;从{\PN{但}}直到{\PN{别是巴}},民间死了七万人。
\VS{16}天使向{\PN{耶路撒冷}}伸手要灭城的时候,耶和华后悔,就不降这灾了,吩咐灭民的天使说:「够了!住手吧!」那时耶和华的使者在{\PN{耶布斯}}人{\PN{亚劳拿}}的禾场那里。
\VS{17}{\PN{大卫}}看见灭民的天使,就祷告耶和华说:「我犯了罪,行了恶;但这群羊做了什么呢?愿你的手攻击我和我的父家。」
\par }{\PP \VS{18}当日,{\PN{迦得}}来见{\PN{大卫}},对他说:「你上去,在{\PN{耶布斯}}人{\PN{亚劳拿}}的禾场上为耶和华筑一座坛。」
\VS{19}{\PN{大卫}}就照着{\PN{迦得}}奉耶和华名所说的话上去了。
\VS{20}{\PN{亚劳拿}}观看,见王和他臣仆前来,就迎接出去,脸伏于地,向王下拜,
\VS{21}说:「我主我王为何来到仆人这里呢?」{\PN{大卫}}说:「我要买你这禾场,为耶和华筑一座坛,使民间的瘟疫止住。」
\VS{22}{\PN{亚劳拿}}对{\PN{大卫}}说:「我主我王,你喜悦用什么,就拿去献祭。看哪,这里有牛可以作燔祭,有打粮的器具和套牛的轭可以当柴烧。
\VS{23}王啊,这一切,我{\PN{亚劳拿}}都奉给你」;又对王说:「愿耶和华—你的 神悦纳你。」
\VS{24}王对{\PN{亚劳拿}}说:「不然。我必要按着价值向你买;我不肯用白得之物作燔祭献给耶和华—我的 神。」{\PN{大卫}}就用五十舍客勒银子买了那禾场与牛。
\VS{25}{\PN{大卫}}在那里为耶和华筑了一座坛,献燔祭和平安祭。如此,耶和华垂听国民所求的,瘟疫在{\PN{以色列}}人中就止住了。
\par }