\NormalFont\ShortTitle{提摩太前书}
{\MT 提摩太前书

\par }\ChapOne{1}{\SH 问候
\par }{\PP \VerseOne{1}奉我们救主 神和我们的盼望基督耶稣之命,作基督耶稣使徒的{\PN{保罗}}
\VS{2}写信给那因信主作我真儿子的{\PN{提摩太}}。愿恩惠、怜悯、平安从父 神和我们主基督耶稣归与你!
\par }{\SH 警告虚伪的道理
\par }{\PP \VS{3}我往{\PN{马其顿}}去的时候,曾劝你仍住在{\PN{以弗所}},好嘱咐那几个人不可传异教,
\VS{4}也不可听从荒渺无凭的话语和无穷的家谱;这等事只生辩论,并不发明 神在信上所立的章程。
\VS{5}但命令的总归就是爱;这爱是从清洁的心和无亏的良心,无伪的信心生出来的。
\VS{6}有人偏离这些,反去讲虚浮的话,
\VS{7}想要作教法师,却不明白自己所讲说的所论定的。
\VS{8}我们知道律法原是好的,只要人用得合宜;
\VS{9}因为律法不是为义人设立的,乃是为不法和不服的,不虔诚和犯罪的,不圣洁和恋世俗的,弑父母和杀人的,
\VS{10}行淫和亲男色的,抢人口和说谎话的,并起假誓的,或是为别样敌正道的事设立的。
\VS{11}这是照着可称颂之 神交托我荣耀福音说的。
\par }{\SH 感谢 神的怜悯
\par }{\PP \VS{12}我感谢那给我力量的我们主基督耶稣,因他以我有忠心,派我服事他。
\VS{13}我从前是亵渎 {\ADD{神}}的,逼迫{\ADD{人}}的,侮慢{\ADD{人}}的;然而我还蒙了怜悯,因我是不信不明白的时候而做的。
\VS{14}并且我主的恩是格外丰盛,使我在基督耶稣里有信心和爱心。
\VS{15}「基督耶稣降世,为要拯救罪人。」这话是可信的,是十分可佩服的。在罪人中我是个罪魁。
\VS{16}然而,我蒙了怜悯,是因耶稣基督要在我这罪魁身上显明他一切的忍耐,给后来信他得永生的人作榜样。
\VS{17}但愿尊贵、荣耀归与那不能朽坏、不能看见、永世的君王、独一的 神,直到永永远远。阿们!
\par }{\PP \VS{18}我儿{\PN{提摩太}}啊,我照从前指着你的预言,将这命令交托你,叫你因此可以打那美好的仗。
\VS{19}常存信心和无亏的良心。有人丢弃良心,就在真道上如同船破坏了一般。
\VS{20}其中有{\PN{许米乃}}和{\PN{亚历山大}};我已经把他们交给撒但,使他们受责罚就不再谤渎了。

\par }\Chap{2}{\SH 有关祷告的指示
\par }{\PP \VerseOne{1}我劝你,第一要为万人恳求、祷告、代求、祝谢;
\VS{2}为君王和一切在位的,{\ADD{也该如此}},使我们可以敬虔、端正、平安无事地度日。
\VS{3}这是好的,在 神我们救主面前可蒙悦纳。
\VS{4}他愿意万人得救,明白真道。
\VS{5}因为只有一位 神,在 神和人中间,只有一位中保,乃是{\ADD{降世}}为人的基督耶稣;
\VS{6}他舍自己作万人的赎价,到了时候,这事必证明出来。
\VS{7}我为此奉派作传道的,作使徒,作外邦人的师傅,教导他们相信,学习真道。我说的是真话,并不是谎言。
\par }{\PP \VS{8}我愿男人无忿怒,无争论\FTNT{}{{\FR 2:8: }或译:疑惑},举起圣洁的手,随处祷告;
\VS{9}又愿女人廉耻、自守,以正派衣裳为妆饰,不以编发、黄金、珍珠,和贵价的衣裳为妆饰,
\VS{10}只要有善行,这才与自称是敬 神的女人相宜。
\VS{11}女人要沉静学道,一味地顺服。
\VS{12}我不许女人讲道,也不许她辖管男人,只要沉静。
\VS{13}因为先造的是{\PN{亚当}},后造的是{\PN{夏娃}},
\VS{14}且不是{\PN{亚当}}被引诱,乃是女人被引诱,陷在罪里。
\VS{15}然而,女人若常存信心、爱心,又圣洁自守,就必在生产上得救。

\par }\Chap{3}{\SH 监督的资格
\par }{\PP \VerseOne{1}「人若想要得监督的职分,就是羡慕善工。」这话是可信的。
\VS{2}作监督的,必须无可指责,只作一个妇人的丈夫,有节制,自守,端正,乐意接待远人,善于教导;
\VS{3}不因酒滋事,不打人,只要温和,不争竞,不贪财;
\VS{4}好好管理自己的家,使儿女凡事端庄顺服\FTNT{}{{\FR 3:4: }或译:端端庄庄地使儿女顺服}。
\VS{5}人若不知道管理自己的家,焉能照管 神的教会呢?
\VS{6}初入教的不可作监督,恐怕他自高自大,就落在魔鬼{\ADD{所受}}的刑罚里。
\VS{7}监督也必须在教外有好名声,恐怕被人毁谤,落在魔鬼的网罗里。
\par }{\SH 执事的资格
\par }{\PP \VS{8}作执事的,也是如此:必须端庄,不一口两舌,不好喝酒,不贪不义之财;
\VS{9}要存清洁的良心,固守真道的奥秘。
\VS{10}这等人也要先受试验,若没有可责之处,然后叫他们作执事。
\VS{11}女执事\FTNT{}{{\FR 3:11: }原文是女人}也是如此:必须端庄,不说谗言,有节制,凡事忠心。
\VS{12}执事只要作一个妇人的丈夫,好好管理儿女和自己的家。
\VS{13}因为善作执事的,自己就得到美好的地步,并且在基督耶稣里的真道上大有胆量。
\par }{\SH 敬虔的奥秘
\par }{\PP \VS{14}我指望快到你那里去,所以先将这些事写给你。
\VS{15}倘若我耽延日久,你也可以知道在 神的家中当怎样行。这家就是永生 神的教会,真理的柱石和根基。
\par }{\Q \VS{16}大哉,敬虔的奥秘,无人不以为然!
\par }{\Q 就是 神在肉身显现,
\par }{\Q 被圣灵称义\FTNT{}{{\FR 3:16: }或译:在灵性称义},
\par }{\Q 被天使看见,
\par }{\Q 被传于外邦,
\par }{\Q 被世人信服,
\par }{\Q 被接在荣耀里。

\par }\Chap{4}{\SH 预言有人离弃真道
\par }{\PP \VerseOne{1}圣灵明说,在后来的时候,必有人离弃真道,听从那引诱人的{\ADD{邪}}灵和鬼魔的道理。
\VS{2}这是因为说谎之人的假冒;这等人的良心如同被热铁烙惯了一般。
\VS{3}他们禁止嫁娶,又禁戒食物\FTNT{}{{\FR 4:3: }或译:又叫人戒荤},就是 神所造、叫那信而明白真道的人感谢着领受的。
\VS{4}凡 神所造的物都是好的,若感谢着领受,就没有一样可弃的,
\VS{5}都因 神的道和人的祈求成为圣洁了。
\par }{\SH 基督耶稣的好执事
\par }{\PP \VS{6}你若将这些事提醒弟兄们,便是基督耶稣的好执事,在真道的话语和你向来所服从的善道上得了教育。
\VS{7}只是要弃绝那世俗的言语和老妇荒谬的话,在敬虔上操练自己。
\VS{8}「操练身体,益处还少;惟独敬虔,凡事都有益处,因有今生和来生的应许。」
\VS{9}这话是可信的,是十分可佩服的。
\VS{10}我们劳苦努力,正是为此,因我们的指望在乎永生的 神;他是万人的救主,更是信徒的救主。
\par }{\PP \VS{11}这些事,你要吩咐人,也要教导人。
\VS{12}不可叫人小看你年轻,总要在言语、行为、爱心、信心、清洁上,都作信徒的榜样。
\VS{13}你要以宣读、劝勉、教导为念,直等到我来。
\VS{14}你不要轻忽所得的恩赐,就是从前借着预言、在众长老按手的时候赐给你的。
\VS{15}这些事你要殷勤去做,并要在此专心,使众人看出你的长进来。
\VS{16}你要谨慎自己和自己的教训,要在这些事上恒心;因为这样行,又能救自己,又能救听你的人。

\par }\Chap{5}{\SH 对别人的本分
\par }{\PP \VerseOne{1}不可严责老年人,只要劝他如同父亲;劝少年人如同弟兄;
\VS{2}劝老年妇女如同母亲;劝少年妇女如同姊妹;总要清清洁洁的。
\par }{\PP \VS{3}要尊敬那真为寡妇的。
\VS{4}若寡妇有儿女,或有孙子孙女,便叫他们先在自己家中学着行孝,报答亲恩,因为这在 神面前是可悦纳的。
\VS{5}那独居无靠、真为寡妇的,是仰赖 神,昼夜不住地祈求祷告。
\VS{6}但那好宴乐的寡妇正活着的时候也是死的。
\VS{7}这些事你要嘱咐她们,叫她们无可指责。
\VS{8}人若不看顾亲属,就是背了真道,比不信的人还不好,不看顾自己家里的人,更是如此。
\VS{9}寡妇记在册子上,必须年纪到六十岁,从来只作一个丈夫的妻子,
\VS{10}又有行善的名声,就如养育儿女,接待远人,洗圣徒的脚,救济遭难的人,竭力行各样善事。
\VS{11}至于年轻的寡妇,就可以辞她;因为她们的情欲发动,违背基督的时候就想要嫁人。
\VS{12}她们被定罪,是因废弃了当初所许的愿;
\VS{13}并且她们又习惯懒惰,挨家闲游;不但是懒惰,又说长道短,好管闲事,说些不当说的话。
\VS{14}所以我愿意年轻的{\ADD{寡妇}}嫁人,生养儿女,治理家务,不给敌人辱骂的把柄。
\VS{15}因为已经有转去随从撒但的。
\VS{16}信{\ADD{主}}的妇女,若家中有寡妇,自己就当救济她们,不可累着教会,好使教会能救济那真{\ADD{无倚靠的}}寡妇。
\par }{\PP \VS{17}那善于管理教会的长老,当以为配受加倍的敬奉;那劳苦传道教导人的,更当如此。
\VS{18}因为经上说:「牛{\ADD{在场上}}踹谷的时候,不可笼住它的嘴」;又说:「工人得工价是应当的。」
\VS{19}控告长老的呈子,非有两三个见证就不要收。
\VS{20}犯罪的人,当在众人面前责备他,叫其余的人也可以惧怕。
\VS{21}我在 神和基督耶稣并蒙拣选的天使面前嘱咐你:要遵守这些话,不可存成见,行事也不可有偏心。
\VS{22}给人行按手的礼,不可急促;不要在别人的罪上有分,要保守自己清洁。
\VS{23}因你胃口不清,屡次患病,再不要照常喝水,可以稍微用点酒。
\par }{\PP \VS{24}有些人的罪是明显的,{\ADD{如同}}先到审判案前;有些人的罪是随后跟了去的。
\VS{25}这样,善行也有明显的,那不明显的也不能隐藏。

\par }\Chap{6}{\PP \VerseOne{1}凡在轭下作仆人的,当以自己主人配受十分的恭敬,免得 神的名和道理被人亵渎。
\VS{2}仆人有信道的主人,不可因为与他是弟兄就轻看他;更要加意服事他;因为得服事之益处的,是信道蒙爱的。
\par }{\SH 假道理和真财富
\par }{\PP 你要以此教训人,劝勉人。
\VS{3}若有人传异教,不服从我们主耶稣基督纯正的话与那合乎敬虔的道理,
\VS{4}他是自高自大,一无所知,专好问难,争辩言词,从此就生出嫉妒、纷争、毁谤、妄疑,
\VS{5}并那坏了心术、失丧真理之人的争竞。他们以敬虔为得利的门路。
\VS{6}然而,敬虔加上知足的心便是大利了;
\VS{7}因为我们没有带什么到世上来,也不能带什么去。
\VS{8}只要有衣有食,就当知足。
\VS{9}但那些想要发财的人,就陷在迷惑、落在网罗和许多无知有害的私欲里,叫人沉在败坏和灭亡中。
\VS{10}贪财是万恶之根。有人贪恋钱财,就被引诱离了真道,用许多愁苦把自己刺透了。
\par }{\SH 美好的仗
\par }{\PP \VS{11}但你这属 神的人要逃避这些事,追求公义、敬虔、信心、爱心、忍耐、温柔。
\VS{12}你要为真道打那美好的仗,持定永生。你为此被召,也在许多见证人面前,已经作了那美好的见证。
\VS{13}我在叫万物生活的 神面前,并在向{\PN{本丢·彼拉多}}作过那美好见证的基督耶稣面前嘱咐你:
\VS{14}要守这命令,毫不玷污,无可指责,直到我们的主耶稣基督显现。
\VS{15}到了日期,那可称颂、独有权能的万王之王、万主之主,
\VS{16}就是那独一不死、住在人不能靠近的光里,是人未曾看见、也是不能看见的,要将他显明出来。但愿尊贵和永远的权能都归给他。阿们!
\par }{\PP \VS{17}你要嘱咐那些今世富足的人,不要自高,也不要倚靠无定的钱财;只要倚靠那厚赐百物给我们享受的 神。
\VS{18}又要嘱咐他们行善,在好事上富足,甘心施舍,乐意供给\FTNT{}{{\FR 6:18: }或译:体贴}人,
\VS{19}为自己积成美好的根基,预备将来,叫他们持定那真正的生命。
\par }{\PP \VS{20}{\PN{提摩太}}啊,你要保守所托付你的,躲避世俗的虚谈和那敌真道、似是而非的学问。
\VS{21}已经有人自称有这学问,就偏离了真道。
\par }{\PP 愿恩惠{\ADD{常}}与你们同在!
\par }