\NormalFont\ShortTitle{约伯记}
{\MT 约伯记

\par }\ChapOne{1}{\SH 撒但试探约伯
\par }{\PP \VerseOne{1}{\PN{乌斯}}地有一个人名叫{\PN{约伯}};那人完全正直,敬畏 神,远离恶事。
\VS{2}他生了七个儿子,三个女儿。
\VS{3}他的家产有七千羊,三千骆驼,五百对牛,五百母驴,并有许多仆婢。这人在东方人中就为至大。
\VS{4}他的儿子按着日子各在自己家里设摆筵宴,就打发人去,请了他们的三个姊妹来,与他们一同吃喝。
\VS{5}筵宴的日子过了,{\PN{约伯}}打发人去叫他们自洁。他清早起来,按着他们众人的数目献燔祭;因为他说:「恐怕我儿子犯了罪,心中弃掉 神。」{\PN{约伯}}常常这样行。
\par }{\PP \VS{6}有一天, 神的众子来侍立在耶和华面前,撒但也来在其中。
\VS{7}耶和华问撒但说:「你从哪里来?」撒但回答说:「我从地上走来走去,往返而来。」
\VS{8}耶和华问撒但说:「你曾用心察看我的仆人{\PN{约伯}}没有?地上再没有人像他完全正直,敬畏 神,远离恶事。」
\VS{9}撒但回答耶和华说:「{\PN{约伯}}敬畏 神,岂是无故呢?
\VS{10}你岂不是四面圈上篱笆围护他和他的家,并他一切所有的吗?他手所做的都蒙你赐福;他的家产也在地上增多。
\VS{11}你且伸手毁他一切所有的;他必当面弃掉你。」
\VS{12}耶和华对撒但说:「凡他所有的都在你手中;只是不可伸手加害于他。」于是撒但从耶和华面前退去。
\par }{\SH 约伯丧失儿女和财产
\par }{\PP \VS{13}有一天,{\PN{约伯}}的儿女正在他们长兄的家里吃饭喝酒,
\VS{14}有报信的来见{\PN{约伯}},说:「牛正耕地,驴在旁边吃草,
\VS{15}{\PN{示巴}}人忽然闯来,把牲畜掳去,并用刀杀了仆人;惟有我一人逃脱,来报信给你。」
\VS{16}他还说话的时候,又有人来说:「 神从天上降下火来,将群羊和仆人都烧灭了;惟有我一人逃脱,来报信给你。」
\VS{17}他还说话的时候,又有人来说:「{\PN{迦勒底}}人分作三队忽然闯来,把骆驼掳去,并用刀杀了仆人;惟有我一人逃脱,来报信给你。」
\VS{18}他还说话的时候,又有人来说:「你的儿女正在他们长兄的家里吃饭喝酒,
\VS{19}不料,有狂风从旷野刮来,击打房屋的四角,房屋倒塌在少年人身上,他们就都死了;惟有我一人逃脱,来报信给你。」
\par }{\PP \VS{20}{\PN{约伯}}便起来,撕裂外袍,剃了头,伏在地上下拜,
\VS{21}说:「我赤身出于母胎,也必赤身归回;赏赐的是耶和华,收取的也是耶和华。耶和华的名是应当称颂的。」
\par }{\PP \VS{22}在这一切的事上{\PN{约伯}}并不犯罪,也不以 神为愚妄\FTNT{}{{\FR 1:22: }或译:也不妄评 神}。

\par }\Chap{2}{\SH 撒但再试探约伯
\par }{\PP \VerseOne{1}又有一天, 神的众子来侍立在耶和华面前,撒但也来在其中。
\VS{2}耶和华问撒但说:「你从哪里来?」撒但回答说:「我从地上走来走去,往返而来。」
\VS{3}耶和华问撒但说:「你曾用心察看我的仆人{\PN{约伯}}没有?地上再没有人像他完全正直,敬畏 神,远离恶事。你虽激动我攻击他,无故地毁灭他,他仍然持守他的纯正。」
\VS{4}撒但回答耶和华说:「人以皮代皮,情愿舍去一切所有的,保全性命。
\VS{5}你且伸手伤他的骨头和他的肉,他必当面弃掉你。」
\VS{6}耶和华对撒但说:「他在你手中,只要存留他的性命。」
\par }{\PP \VS{7}于是撒但从耶和华面前退去,击打{\PN{约伯}},使他从脚掌到头顶长毒疮。
\VS{8}{\PN{约伯}}就坐在炉灰中,拿瓦片刮身体。
\par }{\PP \VS{9}他的妻子对他说:「你仍然持守你的纯正吗?你弃掉 神,死了吧!」
\VS{10}{\PN{约伯}}却对她说:「你说话像愚顽的妇人一样。嗳!难道我们从 神手里得福,不也受祸吗?」在这一切的事上{\PN{约伯}}并不以口犯罪。
\par }{\SH 三个朋友来安慰约伯
\par }{\PP \VS{11}{\PN{约伯}}的三个朋友—{\PN{提幔}}人{\PN{以利法}}、{\PN{书亚}}人{\PN{比勒达}}、{\PN{拿玛}}人{\PN{琐法}}—听说有这一切的灾祸临到他身上,各人就从本处约会同来,为他悲伤,安慰他。
\VS{12}他们远远地举目观看,认不出他来,就放声大哭。各人撕裂外袍,把尘土向天扬起来,落在自己的头上。
\VS{13}他们就同他七天七夜坐在地上,一个人也不向他说句话,因为他极其痛苦。

\par }\Chap{3}{\SH 约伯咒诅自己
\par }{\PP \VerseOne{1}此后,{\PN{约伯}}开口咒诅自己的{\ADD{生}}日,
\VS{2}说:
\par }{\Q \VS{3}愿我生的那日
\par }{\Q 和说怀了男胎的那夜都灭没。
\par }{\Q \VS{4}愿那日变为黑暗;
\par }{\Q 愿 神不从上面寻找它;
\par }{\Q 愿亮光不照于其上。
\par }{\Q \VS{5}愿黑暗和死荫索取那日;
\par }{\Q 愿密云停在其上;
\par }{\Q 愿日蚀恐吓它。
\par }{\Q \VS{6}愿那夜被幽暗夺取,
\par }{\Q 不在年中的日子同乐,
\par }{\Q 也不入月中的数目。
\par }{\Q \VS{7}愿那夜没有生育,
\par }{\Q 其间也没有欢乐的声音。
\par }{\Q \VS{8}愿那咒诅日子且能惹动鳄鱼的
\par }{\Q 咒诅那夜。
\par }{\Q \VS{9}愿那夜黎明的星宿变为黑暗,
\par }{\Q 盼亮却不亮,
\par }{\Q 也不见早晨的光线\FTNT{}{{\FR 3:9: }原文是眼皮};
\par }{\Q \VS{10}因没有把怀我胎的门关闭,
\par }{\Q 也没有将患难对我的眼隐藏。
\par }{\BB \par }{\Q \VS{11}我为何不出{\ADD{母}}胎而死?
\par }{\Q 为何不出{\ADD{母}}腹绝气?
\par }{\Q \VS{12}为何有膝接收我?
\par }{\Q 为何有奶哺养我?
\par }{\Q \VS{13}不然,我就早已躺卧安睡,
\par }{\Q \VS{14}和地上为自己重造荒邱的君王、谋士,
\par }{\Q \VS{15}或与有金子、将银子装满了房屋的王子
\par }{\Q 一同安息;
\par }{\Q \VS{16}或像隐而未现、不到期而落的胎,
\par }{\Q 归于无有,如同未见光的婴孩。
\par }{\Q \VS{17}在那里恶人止息搅扰,
\par }{\Q 困乏人得享安息,
\par }{\Q \VS{18}被囚的人同得安逸,
\par }{\Q 不听见督工的声音。
\par }{\Q \VS{19}大小都在那里;
\par }{\Q 奴仆脱离主人的辖制。
\par }{\BB \par }{\Q \VS{20}受患难的人为何有光赐给他呢?
\par }{\Q 心中愁苦的人为何有生命赐给他呢?
\par }{\Q \VS{21}他们切望死,却不得死;
\par }{\Q 求死,胜于求隐藏的珍宝。
\par }{\Q \VS{22}他们寻见坟墓就快乐,
\par }{\Q 极其欢喜。
\par }{\Q \VS{23}人的道路既然遮隐,
\par }{\Q  神又把他四面围困,
\par }{\Q {\ADD{为何有光赐给}}他呢?
\par }{\Q \VS{24}我未曾吃饭就发出叹息;
\par }{\Q 我唉哼{\ADD{的声音}}涌出如水。
\par }{\Q \VS{25}因我所恐惧的临到我身,
\par }{\Q 我所惧怕的迎我而来。
\par }{\Q \VS{26}我不得安逸,不得平静,
\par }{\Q 也不得安息,却有患难来到。

\par }\Chap{4}{\SH 第一次对话
\par }{\R (4·1—14·22)
\par }{\Q \VerseOne{1}{\PN{提幔}}人{\PN{以利法}}回答说:
\par }{\Q \VS{2}人若想与你说话,你就厌烦吗?
\par }{\Q 但谁能忍住不说呢?
\par }{\Q \VS{3}你素来教导许多的人,
\par }{\Q 又坚固软弱的手。
\par }{\Q \VS{4}你的言语曾扶助那将要跌倒的人;
\par }{\Q 你又使软弱的膝稳固。
\par }{\Q \VS{5}但现在祸患临到你,你就昏迷,
\par }{\Q 挨近你,你便惊惶。
\par }{\Q \VS{6}你的倚靠不是在你敬畏 {\ADD{神}}吗?
\par }{\Q 你的盼望不是在你行事纯正吗?
\par }{\Q \VS{7}请你追想:无辜的人有谁灭亡?
\par }{\Q 正直的人在何处剪除?
\par }{\Q \VS{8}按我所见,耕罪孽、种毒害的人
\par }{\Q 都照样收割。
\par }{\Q \VS{9}神一出气,他们就灭亡;
\par }{\Q  神一发怒,他们就消没。
\par }{\Q \VS{10}狮子的吼叫和猛狮的声音{\ADD{尽都止息}};
\par }{\Q 少壮狮子的牙齿也都敲掉。
\par }{\Q \VS{11}老狮子因绝食而死;
\par }{\Q 母狮之子也都离散。
\par }{\BB \par }{\Q \VS{12}我暗暗地得了默示;
\par }{\Q 我耳朵也听其细微的声音。
\par }{\Q \VS{13}在思念夜中、异象之间,
\par }{\Q 世人沉睡的时候,
\par }{\Q \VS{14}恐惧、战兢临到我身,
\par }{\Q 使我百骨打战。
\par }{\Q \VS{15}有灵从我面前经过,
\par }{\Q 我身上的毫毛直立。
\par }{\Q \VS{16}那灵停住,
\par }{\Q 我却不能辨其形状;
\par }{\Q 有影像在我眼前。
\par }{\Q 我在静默中听见有声音{\ADD{说}}:
\par }{\Q \VS{17}必死的人岂能比 神公义吗?
\par }{\Q 人岂能比造他的主洁净吗?
\par }{\Q \VS{18}主不信靠他的臣仆,
\par }{\Q 并且指他的使者为愚昧;
\par }{\Q \VS{19}何况那住在土房、根基在尘土里、
\par }{\Q 被蠹虫所毁坏的人呢?
\par }{\Q \VS{20}早晚之间,就被毁灭,
\par }{\Q 永归无有,无人理会。
\par }{\Q \VS{21}他帐棚的绳索岂不从中抽出来呢?
\par }{\Q 他死,且是无智慧而死。

\par }\PoetryChap{5}{\Q \VerseOne{1}你且呼求,有谁答应你?
\par }{\Q 诸圣者之中,你转向哪一位呢?
\par }{\Q \VS{2}忿怒害死愚妄人;
\par }{\Q 嫉妒杀死痴迷人。
\par }{\Q \VS{3}我曾见愚妄人扎下根,
\par }{\Q 但我忽然咒诅他的住处。
\par }{\Q \VS{4}他的儿女远离稳妥的地步,
\par }{\Q 在城门口被压,并无人搭救。
\par }{\Q \VS{5}他的庄稼有饥饿的人吃尽了,
\par }{\Q 就是在荆棘里的也抢去了;
\par }{\Q 他的财宝有网罗张口吞灭了。
\par }{\Q \VS{6}祸患原不是从土中出来;
\par }{\Q 患难也不是从地里发生。
\par }{\Q \VS{7}人生在世必遇患难,
\par }{\Q 如同火星飞腾。
\par }{\BB \par }{\Q \VS{8}至于我,我必仰望 神,
\par }{\Q 把我的事情托付他。
\par }{\Q \VS{9}他行大事不可测度,
\par }{\Q 行奇事不可胜数:
\par }{\Q \VS{10}降雨在地上,
\par }{\Q 赐水于田里;
\par }{\Q \VS{11}将卑微的安置在高处,
\par }{\Q 将哀痛的举到稳妥之地;
\par }{\Q \VS{12}破坏狡猾人的计谋,
\par }{\Q 使他们所谋的不得成就。
\par }{\Q \VS{13}他叫有智慧的中了自己的诡计,
\par }{\Q 使狡诈人的计谋速速灭亡。
\par }{\Q \VS{14}他们白昼遇见黑暗,
\par }{\Q 午间摸索如在夜间。
\par }{\Q \VS{15}神拯救穷乏人
\par }{\Q 脱离他们口中的刀和强暴人的手。
\par }{\Q \VS{16}这样,贫寒的人有指望,
\par }{\Q 罪孽之辈必塞口无言。
\par }{\BB \par }{\Q \VS{17}神所惩治的人是有福的!
\par }{\Q 所以你不可轻看全能者的管教。
\par }{\Q \VS{18}因为他打破,又缠裹;
\par }{\Q 他击伤,用手医治。
\par }{\Q \VS{19}你六次遭难,他必救你;
\par }{\Q 就是七次,灾祸也无法害你。
\par }{\Q \VS{20}在饥荒中,他必救你脱离死亡;
\par }{\Q 在争战中,他必救你脱离刀剑的权力。
\par }{\Q \VS{21}你必被隐藏,不受口舌之害;
\par }{\Q 灾殃临到,你也不惧怕。
\par }{\Q \VS{22}你遇见灾害饥馑,就必嬉笑;
\par }{\Q 地上的野兽,你也不惧怕。
\par }{\Q \VS{23}因为你必与田间的石头立约;
\par }{\Q 田里的野兽也必与你和好。
\par }{\Q \VS{24}你必知道你帐棚平安,
\par }{\Q 要查看你的羊圈,一无所失;
\par }{\Q \VS{25}也必知道你的后裔将来发达,
\par }{\Q 你的子孙像地上的青草。
\par }{\Q \VS{26}你必寿高年迈才归坟墓,
\par }{\Q 好像禾捆到时收藏。
\par }{\Q \VS{27}这理,我们已经考察,本是如此。
\par }{\Q 你须要听,要知道是与自己有益。

\par }\PoetryChap{6}{\Q \VerseOne{1}{\PN{约伯}}回答说:
\par }{\Q \VS{2}惟愿我的烦恼称一称,
\par }{\Q 我一切的灾害放在天平里;
\par }{\Q \VS{3}现今都比海沙更重,
\par }{\Q 所以我的言语急躁。
\par }{\Q \VS{4}因全能者的箭射入我身;
\par }{\Q 其毒,我的灵喝尽了;
\par }{\Q  神的惊吓摆阵攻击我。
\par }{\Q \VS{5}野驴有草岂能叫唤?
\par }{\Q 牛有料岂能吼叫?
\par }{\Q \VS{6}物淡而无盐岂可吃吗?
\par }{\Q 蛋青有什么滋味呢?
\par }{\Q \VS{7}看为可厌的食物,
\par }{\Q 我心不肯挨近。
\par }{\BB \par }{\Q \VS{8}惟愿我得着所求的,
\par }{\Q 愿 神赐我所切望的;
\par }{\Q \VS{9}就是愿 神把我压碎,
\par }{\Q 伸手将我剪除。
\par }{\Q \VS{10}我因没有违弃那圣者的言语,
\par }{\Q 就仍以此为安慰,
\par }{\Q 在不止息的痛苦中还可踊跃。
\par }{\Q \VS{11}我有什么气力使我等候?
\par }{\Q 我有什么结局使我忍耐?
\par }{\Q \VS{12}我的气力岂是石头的气力?
\par }{\Q 我的肉身岂是铜的呢?
\par }{\Q \VS{13}在我岂不是毫无帮助吗?
\par }{\Q 智慧岂不是从我心中赶出净尽吗?
\par }{\BB \par }{\Q \VS{14}那将要灰心、离弃全能者、
\par }{\Q 不敬畏 神的人,
\par }{\Q 他的朋友当以慈爱待他。
\par }{\Q \VS{15}我的弟兄诡诈,好像溪水,
\par }{\Q 又像溪水流干的河道。
\par }{\Q \VS{16}这河因结冰发黑,
\par }{\Q 有雪藏在其中;
\par }{\Q \VS{17}天气渐暖就随时消化,
\par }{\Q 日头炎热便从原处干涸。
\par }{\Q \VS{18}结伴的客旅{\ADD{离弃大道}},
\par }{\Q 顺河偏行,到荒野之地死亡。
\par }{\Q \VS{19}{\PN{提玛}}结伴的客旅瞻望;
\par }{\Q {\PN{示巴}}同伙的人等候。
\par }{\Q \VS{20}他们因失了盼望就抱愧,
\par }{\Q 来到那里便蒙羞。
\par }{\Q \VS{21}现在你们正是这样,
\par }{\Q 看见惊吓的事便惧怕。
\par }{\Q \VS{22}我岂说:请你们供给我,
\par }{\Q 从你们的财物中送礼物给我?
\par }{\Q \VS{23}岂说:拯救我脱离敌人的手吗?
\par }{\Q 救赎我脱离强暴人的手吗?
\par }{\BB \par }{\Q \VS{24}请你们教导我,我便不作声;
\par }{\Q 使我明白在何事上有错。
\par }{\Q \VS{25}正直的言语力量何其大!
\par }{\Q 但你们责备是责备什么呢?
\par }{\Q \VS{26}绝望人的讲论既然如风,
\par }{\Q 你们还想要驳正言语吗?
\par }{\Q \VS{27}你们想为孤儿拈阄,
\par }{\Q 以朋友当货物。
\par }{\BB \par }{\Q \VS{28}现在请你们看看我,
\par }{\Q 我决不当面说谎。
\par }{\Q \VS{29}请你们转意,不要不公;
\par }{\Q 请再转意,我的事有理。
\par }{\Q \VS{30}我的舌上岂有不义吗?
\par }{\Q 我的口里岂不辨奸恶吗?

\par }\PoetryChap{7}{\Q \VerseOne{1}人在世上岂无争战吗?
\par }{\Q 他的日子不像雇工人的日子吗?
\par }{\Q \VS{2}像奴仆切慕黑影,
\par }{\Q 像雇工人盼望工价;
\par }{\Q \VS{3}我也照样经过困苦的日月,
\par }{\Q 夜间的疲乏为我而定。
\par }{\Q \VS{4}我躺卧的时候便说:
\par }{\Q 我何时起来,黑夜就过去呢?
\par }{\Q 我尽是反来复去,直到天亮。
\par }{\Q \VS{5}我的肉体以虫子和尘土为衣;
\par }{\Q 我的皮肤才收了口又重新破裂。
\par }{\Q \VS{6}我的日子比梭更快,
\par }{\Q 都消耗在无指望之中。
\par }{\BB \par }{\Q \VS{7}求你想念,我的生命不过是一口气;
\par }{\Q 我的眼睛必不再见福乐。
\par }{\Q \VS{8}观看我的人,他的眼必不再见我;
\par }{\Q 你的眼目要看我,我却不在了。
\par }{\Q \VS{9}云彩消散而过;
\par }{\Q 照样,人下阴间也不再上来。
\par }{\Q \VS{10}他不再回自己的家;
\par }{\Q 故土也不再认识他。
\par }{\BB \par }{\Q \VS{11}我不禁止我口;
\par }{\Q 我灵愁苦,要发出言语;
\par }{\Q 我心苦恼,要吐露哀情。
\par }{\Q \VS{12}我{\ADD{对 神说}}:我岂是洋海,
\par }{\Q 岂是大鱼,你竟防守我呢?
\par }{\Q \VS{13}若说:我的床必安慰我,
\par }{\Q 我的榻必解释我的苦情,
\par }{\Q \VS{14}你就用梦惊骇我,
\par }{\Q 用异象恐吓我,
\par }{\Q \VS{15}甚至我宁肯噎死,宁肯死亡,
\par }{\Q 胜似留我{\ADD{这一身的}}骨头。
\par }{\Q \VS{16}我厌弃{\ADD{性命}},不愿永活。
\par }{\Q 你任凭我吧,因我的日子都是虚空。
\par }{\Q \VS{17}人算什么,你竟看他为大,
\par }{\Q 将他放在心上?
\par }{\Q \VS{18}每早鉴察他,
\par }{\Q 时刻试验他?
\par }{\Q \VS{19}你到何时才转眼不看我,
\par }{\Q 才任凭我咽下唾沫呢?
\par }{\Q \VS{20}鉴察人的主啊,我若有罪,于你何妨?
\par }{\Q 为何以我当你的箭靶子,
\par }{\Q 使我厌弃自己的性命?
\par }{\Q \VS{21}为何不赦免我的过犯,
\par }{\Q 除掉我的罪孽?
\par }{\Q 我现今要躺卧在尘土中;
\par }{\Q 你要殷勤地寻找我,我却不在了。

\par }\PoetryChap{8}{\Q \VerseOne{1}{\PN{书亚}}人{\PN{比勒达}}回答说:
\par }{\Q \VS{2}这些话你要说到几时?
\par }{\Q 口中的言语{\ADD{如}}狂风{\ADD{要到几时呢}}?
\par }{\Q \VS{3}神岂能偏离公平?
\par }{\Q 全能者岂能偏离公义?
\par }{\Q \VS{4}或者你的儿女得罪了他;
\par }{\Q 他使他们受报应。
\par }{\Q \VS{5}你若殷勤地寻求 神,
\par }{\Q 向全能者恳求;
\par }{\Q \VS{6}你若清洁正直,
\par }{\Q 他必定为你起来,
\par }{\Q 使你公义的居所兴旺。
\par }{\Q \VS{7}你起初虽然微小,
\par }{\Q 终久必甚发达。
\par }{\BB \par }{\Q \VS{8}请你考问前代,
\par }{\Q 追念他们的列祖所查究的。
\par }{\Q \VS{9}我们不过从昨日才有,一无所知;
\par }{\Q 我们在世的日子好像影儿。
\par }{\Q \VS{10}他们岂不指教你,告诉你,
\par }{\Q 从心里发出言语来呢?
\par }{\BB \par }{\Q \VS{11}蒲草没有泥岂能发长?
\par }{\Q 芦荻没有水岂能生发?
\par }{\Q \VS{12}尚青的时候,还没有割下,
\par }{\Q 比百样的草先枯槁。
\par }{\Q \VS{13}凡忘记 神的人,景况也是这样;
\par }{\Q 不虔敬人的指望要灭没。
\par }{\Q \VS{14}他所仰赖的必折断;
\par }{\Q 他所倚靠的是蜘蛛网。
\par }{\Q \VS{15}他要倚靠房屋,房屋却站立不住;
\par }{\Q 他要抓住房屋,房屋却不能存留。
\par }{\Q \VS{16}他在日光之下发青,
\par }{\Q 蔓子爬满了园子;
\par }{\Q \VS{17}他的根盘绕{\ADD{石}}堆,
\par }{\Q 扎入石地。
\par }{\Q \VS{18}他若从本地被拔出,
\par }{\Q 那地就不认识他,{\ADD{说}}:
\par }{\Q 我没有见过你。
\par }{\Q \VS{19}看哪,这就是他道中之乐;
\par }{\Q 以后必另有人从地而生。
\par }{\Q \VS{20}神必不丢弃完全人,
\par }{\Q 也不扶助邪恶人。
\par }{\Q \VS{21}他还要以喜笑充满你的口,
\par }{\Q 以欢呼充满你的嘴。
\par }{\Q \VS{22}恨恶你的要披戴惭愧;
\par }{\Q 恶人的帐棚必归于无有。

\par }\PoetryChap{9}{\Q \VerseOne{1}{\PN{约伯}}回答说:
\par }{\Q \VS{2}我真知道是这样;
\par }{\Q 但人在 神面前怎能成为义呢?
\par }{\Q \VS{3}若愿意与他争辩,
\par }{\Q 千中之一也不能回答。
\par }{\Q \VS{4}他心里有智慧,且大有能力。
\par }{\Q 谁向 神刚硬而得亨通呢?
\par }{\Q \VS{5}他发怒,把山翻倒挪移,
\par }{\Q 山并不知觉。
\par }{\Q \VS{6}他使地震动,离其本位,
\par }{\Q 地的柱子就摇撼。
\par }{\Q \VS{7}他吩咐日头不出来,就不出来,
\par }{\Q 又封闭众星。
\par }{\Q \VS{8}他独自铺张苍天,
\par }{\Q 步行在海浪之上。
\par }{\Q \VS{9}他造北斗、参星、昴星,
\par }{\Q 并南方的密宫;
\par }{\Q \VS{10}他行大事,不可测度,
\par }{\Q 行奇事,不可胜数。
\par }{\Q \VS{11}他从我旁边经过,我却不看见;
\par }{\Q 他在我面前行走,我倒不知觉。
\par }{\Q \VS{12}他夺取,谁能阻挡?
\par }{\Q 谁敢问他:你做什么?
\par }{\BB \par }{\Q \VS{13}神必不收回他的怒气;
\par }{\Q 扶助{\PN{拉哈伯}}的,屈身在他以下。
\par }{\Q \VS{14}既是这样,我怎敢回答他,
\par }{\Q 怎敢选择言语与他{\ADD{辩论呢}}?
\par }{\Q \VS{15}我虽有义,也不回答他,
\par }{\Q 只要向那审判我的恳求。
\par }{\Q \VS{16}我若呼吁,他应允我;
\par }{\Q 我仍不信他真听我的声音。
\par }{\Q \VS{17}他用暴风折断我,
\par }{\Q 无故地加增我的损伤。
\par }{\Q \VS{18}我就是喘一口气,他都不容,
\par }{\Q 倒使我满心苦恼。
\par }{\Q \VS{19}{\ADD{若论}}力量,{\ADD{他真有}}能力!
\par }{\Q 若论审判,{\ADD{他说}}谁能将我传来呢?
\par }{\Q \VS{20}我虽有义,自己的口要定我为有罪;
\par }{\Q 我虽完全,我口必显我为弯曲。
\par }{\Q \VS{21}我本完全,不顾自己;
\par }{\Q 我厌恶我的性命。
\par }{\Q \VS{22}{\ADD{善恶无分}},都是一样;
\par }{\Q 所以我说,完全人和恶人,他都灭绝。
\par }{\Q \VS{23}若忽然遭杀害之祸,
\par }{\Q 他必戏笑无辜的人遇难。
\par }{\Q \VS{24}世界交在恶人手中;
\par }{\Q 蒙蔽世界审判官的脸,
\par }{\Q 若不是他,是谁呢?
\par }{\BB \par }{\Q \VS{25}我的日子比跑信的更快,
\par }{\Q 急速过去,不见福乐。
\par }{\Q \VS{26}我的日子过去如快船,
\par }{\Q 如急落抓食的鹰。
\par }{\Q \VS{27}我若说:我要忘记我的哀情,
\par }{\Q 除去我的{\ADD{愁容}},心中畅快;
\par }{\Q \VS{28}我因愁苦而惧怕,
\par }{\Q 知道你必不以我为无辜。
\par }{\Q \VS{29}我必被你定为有罪,
\par }{\Q 我何必徒然劳苦呢?
\par }{\Q \VS{30}我若用雪水洗身,
\par }{\Q 用硷洁净我的手,
\par }{\Q \VS{31}你还要扔我在坑里,
\par }{\Q 我的衣服都憎恶我。
\par }{\Q \VS{32}他本不像我是人,使我可以回答他,
\par }{\Q 又使我们可以同听审判。
\par }{\Q \VS{33}我们中间没有听讼的人
\par }{\Q 可以向我们两造按手。
\par }{\Q \VS{34}愿他把杖离开我,
\par }{\Q 不使惊惶威吓我。
\par }{\Q \VS{35}我就说话,也不惧怕他,
\par }{\Q 现在我却不是那样。

\par }\PoetryChap{10}{\Q \VerseOne{1}我厌烦我的性命,
\par }{\Q 必由着自己述说我的哀情;
\par }{\Q 因心里苦恼,我要说话,
\par }{\Q \VS{2}对 神说:不要定我有罪,
\par }{\Q 要指示我,你为何与我争辩?
\par }{\Q \VS{3}你手所造的,
\par }{\Q 你又欺压,又藐视,
\par }{\Q 却光照恶人的计谋。
\par }{\Q 这事你以为美吗?
\par }{\Q \VS{4}你的眼岂是肉眼?
\par }{\Q 你查看岂像人查看吗?
\par }{\Q \VS{5}你的日子岂像人的日子,
\par }{\Q 你的年岁岂像人的年岁,
\par }{\Q \VS{6}就追问我的罪孽,
\par }{\Q 寻察我的罪过吗?
\par }{\Q \VS{7}其实,你知道我没有罪恶,
\par }{\Q 并没有能救{\ADD{我}}脱离你手的。
\par }{\Q \VS{8}你的手创造我,造就我的四肢百体,
\par }{\Q 你还要毁灭我。
\par }{\Q \VS{9}求你记念—制造我如抟泥一般,
\par }{\Q 你还要使我归于尘土吗?
\par }{\Q \VS{10}你不是倒出我来好像奶,
\par }{\Q 使我凝结如同奶饼吗?
\par }{\Q \VS{11}你以皮和肉为衣给我穿上,
\par }{\Q 用骨与筋把我全体联络。
\par }{\Q \VS{12}你将生命和慈爱赐给我;
\par }{\Q 你也眷顾保全我的心灵。
\par }{\Q \VS{13}然而,你{\ADD{待我的}}这些事早已藏在你心里;
\par }{\Q 我知道你久有此意。
\par }{\Q \VS{14}我若犯罪,你就察看我,
\par }{\Q 并不赦免我的罪孽。
\par }{\Q \VS{15}我若行恶,便有了祸;
\par }{\Q 我若为义,也不敢抬头,
\par }{\Q 正是满心羞愧,
\par }{\Q 眼见我的苦情。
\par }{\Q \VS{16}我若昂首自得,你就追捕我如狮子,
\par }{\Q 又在我身上显出奇能。
\par }{\Q \VS{17}你重立见证攻击我,
\par }{\Q 向我加增恼怒,
\par }{\Q 如军兵更换着攻击我。
\par }{\BB \par }{\Q \VS{18}你为何使我出母胎呢?
\par }{\Q 不如我当时气绝,无人得见我;
\par }{\Q \VS{19}这样,就如没有我一般,
\par }{\Q 一出母胎就被送入坟墓。
\par }{\Q \VS{20-21}我的日子不是甚少吗?
\par }{\Q 求你停手宽容我,
\par }{\Q 叫我在往而不返之先—
\par }{\Q 就是往黑暗和死荫之地以先—
\par }{\Q 可以稍得畅快。
\par }{\Q \VS{22}那地甚是幽暗,是死荫混沌{\ADD{之地}};
\par }{\Q 那里的光好像幽暗。

\par }\PoetryChap{11}{\Q \VerseOne{1}{\PN{拿玛}}人{\PN{琐法}}回答说:
\par }{\Q \VS{2}这许多的言语岂不该回答吗?
\par }{\Q 多嘴多舌的人岂可称为义吗?
\par }{\Q \VS{3}你夸大的话岂能使人不作声吗?
\par }{\Q 你戏笑的时候岂没有人叫你害羞吗?
\par }{\Q \VS{4}你说:我的道理纯全;
\par }{\Q 我在你眼前洁净。
\par }{\Q \VS{5}惟愿 神说话;
\par }{\Q 愿他开口攻击你,
\par }{\Q \VS{6}并将智慧的奥秘指示你;
\par }{\Q 他有诸般的智识。
\par }{\Q 所以当知道 神追讨你
\par }{\Q 比你罪孽该得的还少。
\par }{\Q \VS{7}你考察就能测透 神吗?
\par }{\Q 你岂能尽情测透全能者吗?
\par }{\Q \VS{8}他{\ADD{的智慧}}高于天,你还能做什么?
\par }{\Q 深于阴间,你还能知道什么?
\par }{\Q \VS{9}其量比地长,
\par }{\Q 比海宽。
\par }{\Q \VS{10}他若经过,将人拘禁,
\par }{\Q 招人受审,谁能阻挡他呢?
\par }{\Q \VS{11}他本知道虚妄的人;
\par }{\Q 人的罪孽,他虽不留意,还是无所不见。
\par }{\Q \VS{12}空虚的人却毫无知识;
\par }{\Q 人生在世{\ADD{好像}}野驴的驹子。
\par }{\BB \par }{\Q \VS{13}你若将心安正,
\par }{\Q 又向主举手;
\par }{\Q \VS{14}你手里若有罪孽,
\par }{\Q 就当远远地除掉,
\par }{\Q 也不容非义住在你帐棚之中。
\par }{\Q \VS{15}那时,你必仰起脸来毫无斑点;
\par }{\Q 你也必坚固,无所惧怕。
\par }{\Q \VS{16}你必忘记你的苦楚,
\par }{\Q 就是想起也如流过去的水一样。
\par }{\Q \VS{17}你在世的日子要比正午更明,
\par }{\Q 虽有黑暗仍像早晨。
\par }{\Q \VS{18}你因有指望就必稳固,
\par }{\Q 也必{\ADD{四围}}巡查,坦然安息。
\par }{\Q \VS{19}你躺卧,无人惊吓,
\par }{\Q 且有许多人向你求恩。
\par }{\Q \VS{20}但恶人的眼目必要失明。
\par }{\Q 他们无路可逃;
\par }{\Q 他们的指望就是气绝。

\par }\PoetryChap{12}{\Q \VerseOne{1}{\PN{约伯}}回答说:
\par }{\Q \VS{2}你们真是子民哪,
\par }{\Q 你们死亡,智慧也就灭没了。
\par }{\Q \VS{3}但我也有聪明,与你们一样,
\par }{\Q 并非不及你们。
\par }{\Q 你们所说的,谁不知道呢?
\par }{\Q \VS{4}我这求告 神、蒙他应允的人
\par }{\Q 竟成了朋友所讥笑的;
\par }{\Q 公义完全人竟受了人的讥笑。
\par }{\Q \VS{5}安逸的人心里藐视灾祸;
\par }{\Q 这灾祸常常等待滑脚的人。
\par }{\Q \VS{6}强盗的帐棚兴旺,
\par }{\Q 惹 神的人稳固,
\par }{\Q  神{\ADD{多将财物}}送到他们手中。
\par }{\BB \par }{\Q \VS{7}你且问走兽,走兽必指教你;
\par }{\Q 又问空中的飞鸟,飞鸟必告诉你;
\par }{\Q \VS{8}或与地说话,地必指教你;
\par }{\Q 海中的鱼也必向你说明。
\par }{\Q \VS{9}看这一切,
\par }{\Q 谁不知道是耶和华的手做成的呢?
\par }{\Q \VS{10}凡活物的生命和人类的气息都在他手中。
\par }{\Q \VS{11}耳朵岂不试验言语,
\par }{\Q 正如上膛尝食物吗?
\par }{\Q \VS{12}年老的有智慧;
\par }{\Q 寿高的有知识。
\par }{\BB \par }{\Q \VS{13}在 神有智慧和能力;
\par }{\Q 他有谋略和知识。
\par }{\Q \VS{14}他拆毁的,就不能再建造;
\par }{\Q 他捆住人,便不得开释。
\par }{\Q \VS{15}他把水留住,水便枯干;
\par }{\Q 他再发出水来,水就翻地。
\par }{\Q \VS{16}在他有能力和智慧,
\par }{\Q 被诱惑的与诱惑人的都是属他。
\par }{\Q \VS{17}他把谋士剥衣掳去,
\par }{\Q 又使审判官变成愚人。
\par }{\Q \VS{18}他放松君王的绑,
\par }{\Q 又用带子捆他们的腰。
\par }{\Q \VS{19}他把祭司剥衣掳去,
\par }{\Q 又使有能的人倾败。
\par }{\Q \VS{20}他废去忠信人的讲论,
\par }{\Q 又夺去老人的聪明。
\par }{\Q \VS{21}他使君王蒙羞被辱,
\par }{\Q 放松有力之人的腰带。
\par }{\Q \VS{22}他将深奥的事从黑暗中彰显,
\par }{\Q 使死荫显为光明。
\par }{\Q \VS{23}他使邦国兴旺而又毁灭;
\par }{\Q 他使邦国开广而又掳去。
\par }{\Q \VS{24}他将地上民中首领的聪明夺去,
\par }{\Q 使他们在荒废无路之地漂流;
\par }{\Q \VS{25}他们无光,在黑暗中摸索,
\par }{\Q 又使他们东倒西歪,像醉酒的人一样。

\par }\PoetryChap{13}{\Q \VerseOne{1}这一切,我眼都见过;
\par }{\Q 我耳都听过,而且明白。
\par }{\Q \VS{2}你们所知道的,我也知道,
\par }{\Q 并非不及你们。
\par }{\Q \VS{3}我真要对全能者说话;
\par }{\Q 我愿与 神理论。
\par }{\Q \VS{4}你们是编造谎言的,
\par }{\Q 都是无用的医生。
\par }{\Q \VS{5}惟愿你们全然不作声;
\par }{\Q 这就算为你们的智慧!
\par }{\Q \VS{6}请你们听我的辩论,
\par }{\Q 留心听我口中的分诉。
\par }{\Q \VS{7}你们要为 神说不义的话吗?
\par }{\Q 为他说诡诈的言语吗?
\par }{\Q \VS{8}你们要为 神徇情吗?
\par }{\Q 要为他争论吗?
\par }{\Q \VS{9}他查出你们来,这岂是好吗?
\par }{\Q 人欺哄人,你们也要照样欺哄他吗?
\par }{\Q \VS{10}你们若暗中徇情,
\par }{\Q 他必要责备你们。
\par }{\Q \VS{11}他的尊荣岂不叫你们惧怕吗?
\par }{\Q 他的惊吓岂不临到你们吗?
\par }{\Q \VS{12}你们以为可记念的箴言是炉灰的箴言;
\par }{\Q 你们以为可靠的坚垒是淤泥的坚垒。
\par }{\BB \par }{\Q \VS{13}你们不要作声,任凭我吧!
\par }{\Q 让我说话,无论如何我都承当。
\par }{\Q \VS{14}我何必把我的肉挂在牙上,
\par }{\Q 将我的命放在手中。
\par }{\Q \VS{15}他必杀我;我虽无指望,
\par }{\Q 然而我在他面前还要辩明我所行的。
\par }{\Q \VS{16}这要成为我的拯救,
\par }{\Q 因为不虔诚的人不得到他面前。
\par }{\Q \VS{17}你们要细听我的言语,
\par }{\Q 使我所辩论的入你们的耳中。
\par }{\Q \VS{18}我已陈明我的案,
\par }{\Q 知道自己有义。
\par }{\Q \VS{19}有谁与我争论,
\par }{\Q 我就情愿缄默不言,气绝而亡。
\par }{\Q \VS{20}惟有两件不要向我施行,
\par }{\Q 我就不躲开你的面:
\par }{\Q \VS{21}就是把你的手缩回,远离我身;
\par }{\Q 又不使你的惊惶威吓我。
\par }{\Q \VS{22}这样,你呼叫,我就回答;
\par }{\Q 或是让我说话,你回答我。
\par }{\Q \VS{23}我的罪孽和罪过有多少呢?
\par }{\Q 求你叫我知道我的过犯与罪愆。
\par }{\Q \VS{24}你为何掩面、
\par }{\Q 拿我当仇敌呢?
\par }{\Q \VS{25}你要惊动被风吹的叶子吗?
\par }{\Q 要追赶枯干的碎秸吗?
\par }{\Q \VS{26}你按罪状刑罚我,
\par }{\Q 又使我担当幼年的罪孽;
\par }{\Q \VS{27}也把我的脚上了木狗,
\par }{\Q 并窥察我一切的道路,
\par }{\Q 为我的脚掌划定界限。
\par }{\Q \VS{28}我已经像灭绝的烂物,
\par }{\Q 像虫蛀的衣裳。

\par }\PoetryChap{14}{\Q \VerseOne{1}人为妇人所生,
\par }{\Q 日子短少,多有患难;
\par }{\Q \VS{2}出来如花,又被割下,
\par }{\Q 飞去如影,不能存留。
\par }{\Q \VS{3}这样的人你岂睁眼看他吗?
\par }{\Q 又叫我来受审吗?
\par }{\Q \VS{4}谁能使洁净之物出于污秽之中呢?
\par }{\Q 无论谁也不能!
\par }{\Q \VS{5}人的日子既然限定,
\par }{\Q 他的月数在你那里,
\par }{\Q 你也派定他的界限,使他不能越过,
\par }{\Q \VS{6}便求你转眼不看他,使他得歇息,
\par }{\Q 直等他像雇工人完毕他的日子。
\par }{\BB \par }{\Q \VS{7}树若被砍下,
\par }{\Q 还可指望发芽,
\par }{\Q 嫩枝生长不息;
\par }{\Q \VS{8}其根虽然衰老在地里,
\par }{\Q 干也死在土中,
\par }{\Q \VS{9}及至得了水气,还要发芽,
\par }{\Q 又长枝条,像新栽的树一样。
\par }{\Q \VS{10}但人死亡而消灭;
\par }{\Q 他气绝,竟在何处呢?
\par }{\Q \VS{11}海中的水绝尽,
\par }{\Q 江河消散干涸。
\par }{\Q \VS{12}人也是如此,躺下不再起来,
\par }{\Q 等到天没有了,仍不得复醒,
\par }{\Q 也不得从睡中唤醒。
\par }{\Q \VS{13}惟愿你把我藏在阴间,
\par }{\Q 存于隐密处,等你的忿怒过去;
\par }{\Q 愿你为我定了日期,记念我。
\par }{\Q \VS{14}人若死了岂能{\ADD{再}}活呢?
\par }{\Q 我只要在我一切争战的日子,
\par }{\Q 等我被释放\FTNT{}{{\FR 14:14: }或译:改变}的时候来到。
\par }{\Q \VS{15}你呼叫,我便回答;
\par }{\Q 你手所做的,你必羡慕。
\par }{\Q \VS{16}但如今你数点我的脚步,
\par }{\Q 岂不窥察我的罪过吗?
\par }{\Q \VS{17}我的过犯被你封在囊中,
\par }{\Q 也缝严了我的罪孽。
\par }{\BB \par }{\Q \VS{18}山崩变为无有;
\par }{\Q 磐石挪开原处。
\par }{\Q \VS{19}水流消磨石头,
\par }{\Q 所流溢的洗去地上的尘土;
\par }{\Q 你也照样灭绝人的指望。
\par }{\Q \VS{20}你攻击人常常得胜,使他去世;
\par }{\Q 你改变他的容貌,叫他往而不回。
\par }{\Q \VS{21}他儿子得尊荣,他也不知道,
\par }{\Q 降为卑,他也不觉得。
\par }{\Q \VS{22}但知身上疼痛,
\par }{\Q 心中悲哀。

\par }\Chap{15}{\SH 第二次对话
\par }{\R (15·1—21·34)
\par }{\Q \VerseOne{1}{\PN{提幔}}人{\PN{以利法}}回答说:
\par }{\Q \VS{2}智慧人岂可用虚空的知识回答,
\par }{\Q 用东风充满肚腹呢?
\par }{\Q \VS{3}他岂可用无益的话
\par }{\Q 和无济于事的言语理论呢?
\par }{\Q \VS{4}你是废弃敬畏的意,
\par }{\Q 在 神面前阻止敬虔的心。
\par }{\Q \VS{5}你的罪孽指教你的口;
\par }{\Q 你选用诡诈人的舌头。
\par }{\Q \VS{6}你自己的口定你有罪,并非是我;
\par }{\Q 你自己的嘴见证你的不是。
\par }{\BB \par }{\Q \VS{7}你岂是头一个被生的人吗?
\par }{\Q 你受造在诸山之先吗?
\par }{\Q \VS{8}你曾听见 神的密旨吗?
\par }{\Q 你还将智慧独自得尽吗?
\par }{\Q \VS{9}你知道什么是我们不知道的呢?
\par }{\Q 你明白什么是我们不明白的呢?
\par }{\Q \VS{10}我们这里有白发的和年纪老迈的,
\par }{\Q 比你父亲还老。
\par }{\Q \VS{11}神用温和的话安慰你,
\par }{\Q 你以为太小吗?
\par }{\Q \VS{12}你的心为何将你逼去?
\par }{\Q 你的眼为何冒出火星,
\par }{\Q \VS{13}使你的灵反对 神,
\par }{\Q 也任你的口发这言语?
\par }{\Q \VS{14}人是什么,竟算为洁净呢?
\par }{\Q 妇人所生的是什么,竟算为义呢?
\par }{\Q \VS{15}神不信靠他的众圣者;
\par }{\Q 在他眼前,天也不洁净,
\par }{\Q \VS{16}何况那污秽可憎、
\par }{\Q 喝罪孽如水的世人呢!
\par }{\BB \par }{\Q \VS{17}我指示你,你要听;
\par }{\Q 我要述说所看见的,
\par }{\Q \VS{18}就是智慧人从列祖所受,
\par }{\Q 传说而不隐瞒的。
\par }{\Q \VS{19}这地惟独赐给他们,
\par }{\Q 并没有外人从他们中间经过。
\par }{\Q \VS{20}恶人一生之日劬劳痛苦;
\par }{\Q 强暴人一生的年数也是如此。
\par }{\Q \VS{21}惊吓的声音常在他耳中;
\par }{\Q 在平安时,抢夺的必临到他那里。
\par }{\Q \VS{22}他不信自己能从黑暗中转回;
\par }{\Q 他被刀剑等候。
\par }{\Q \VS{23}他漂流在外求食,{\ADD{说}}:哪里有食物呢?
\par }{\Q 他知道黑暗的日子在他手边预备好了。
\par }{\Q \VS{24}急难困苦叫他害怕,
\par }{\Q 而且胜了他,好像君王预备上阵一样。
\par }{\Q \VS{25}他伸手攻击 神,
\par }{\Q 以骄傲攻击全能者,
\par }{\Q \VS{26}挺{\ADD{着}}颈项,
\par }{\Q 用盾牌的厚凸面向全能者直闯;
\par }{\Q \VS{27}是因他的脸蒙上脂油,
\par }{\Q 腰积成肥肉。
\par }{\Q \VS{28}他曾住在荒凉城邑,
\par }{\Q 无人居住、将成乱堆的房屋。
\par }{\Q \VS{29}他不得富足,财物不得常存,
\par }{\Q 产业在地上也不加增。
\par }{\Q \VS{30}他不得出离黑暗。
\par }{\Q 火焰要将他的枝子烧干;
\par }{\Q 因 神口中的气,他要灭亡
\FTNT{}{{\FR 15:30: }原文是走去}。
\par }{\Q \VS{31}他不用倚靠虚假欺哄自己,
\par }{\Q 因虚假必成为他的报应。
\par }{\Q \VS{32}他的日期未到之先,这事必成就;
\par }{\Q 他的枝子不得青绿。
\par }{\Q \VS{33}他必像葡萄树的葡萄,未熟而落;
\par }{\Q 又像橄榄树的花,一开而谢。
\par }{\Q \VS{34}原来不敬虔之辈必无生育;
\par }{\Q 受贿赂之人的帐棚必被火烧。
\par }{\Q \VS{35}他们所怀的是毒害,所生的是罪孽;
\par }{\Q 心里所预备的是诡诈。

\par }\PoetryChap{16}{\Q \VerseOne{1}{\PN{约伯}}回答说:
\par }{\Q \VS{2}这样的话我听了许多;
\par }{\Q 你们安慰人,反叫人愁烦。
\par }{\Q \VS{3}虚空的言语有穷尽吗?
\par }{\Q 有什么话惹动你回答呢?
\par }{\Q \VS{4}我也能说你们那样的话;
\par }{\Q 你们若处在我的境遇,
\par }{\Q 我也会联络言语攻击你们,
\par }{\Q 又能向你们摇头。
\par }{\Q \VS{5}但我必用口坚固你们,
\par }{\Q 用嘴消解{\ADD{你们的忧愁}}。
\par }{\BB \par }{\Q \VS{6}我虽说话,忧愁仍不得消解;
\par }{\Q 我虽停住不说,忧愁就离开我吗?
\par }{\Q \VS{7}但现在 神使我困倦,
\par }{\Q 使亲友远离我,
\par }{\Q \VS{8}又抓住我,作见证{\ADD{攻击我}};
\par }{\Q 我身体的枯瘦也当面见证我的不是。
\par }{\Q \VS{9}主发怒撕裂我,逼迫我,
\par }{\Q 向我切齿;
\par }{\Q 我的敌人怒目看我。
\par }{\Q \VS{10}他们向我开口,
\par }{\Q 打我的脸羞辱我,
\par }{\Q 聚会攻击我。
\par }{\Q \VS{11}神把我交给不敬虔的人,
\par }{\Q 把我扔到恶人的手中。
\par }{\Q \VS{12}我素来安逸,他折断我,
\par }{\Q 掐住我的颈项,把我摔碎,
\par }{\Q 又立我为他的箭靶子。
\par }{\Q \VS{13}他的弓箭手四面围绕我;
\par }{\Q 他破裂我的肺腑,并不留情,
\par }{\Q 把我的胆倾倒在地上,
\par }{\Q \VS{14}将我破裂又破裂,
\par }{\Q 如同勇士向我直闯。
\par }{\BB \par }{\Q \VS{15}我缝麻布在我皮肤上,
\par }{\Q 把我的角放在尘土中。
\par }{\Q \VS{16}我的脸因哭泣发紫,
\par }{\Q 在我的眼皮上有死荫。
\par }{\Q \VS{17}我的手中却无强暴;
\par }{\Q 我的祈祷也是清洁。
\par }{\BB \par }{\Q \VS{18}地啊,不要遮盖我的血!
\par }{\Q 不要{\ADD{阻挡}}我的哀求!
\par }{\Q \VS{19}现今,在天有我的见证,
\par }{\Q 在上有我的中保。
\par }{\Q \VS{20}我的朋友讥诮我,
\par }{\Q 我却向 神眼泪汪汪。
\par }{\Q \VS{21}愿人得与 神辩白,
\par }{\Q 如同人与朋友辩白一样;
\par }{\Q \VS{22}因为再过几年,
\par }{\Q 我必走那往而不返之路。

\par }\PoetryChap{17}{\Q \VerseOne{1}我的心灵消耗,我的日子灭尽;
\par }{\Q 坟墓为我{\ADD{预备}}好了。
\par }{\Q \VS{2}真有戏笑我的在我这里,
\par }{\Q 我眼常见他们惹动我。
\par }{\BB \par }{\Q \VS{3}愿主拿凭据给我,自己为我作保。
\par }{\Q {\ADD{在你以外}}谁肯与我击掌呢?
\par }{\Q \VS{4}因你使他们心不明理,
\par }{\Q 所以你必不高举他们。
\par }{\Q \VS{5}控告他的朋友、以朋友为可抢夺的,
\par }{\Q 连他儿女的眼睛也要失明。
\par }{\BB \par }{\Q \VS{6}神使我作了民中的笑谈;
\par }{\Q 他们也吐唾沫在我脸上。
\par }{\Q \VS{7}我的眼睛因忧愁昏花;
\par }{\Q 我的百体好像影儿。
\par }{\Q \VS{8}正直人因此必惊奇;
\par }{\Q 无辜的人要兴起攻击不敬虔之辈。
\par }{\Q \VS{9}然而,义人要持守所行的道;
\par }{\Q 手洁的人要力上加力。
\par }{\Q \VS{10}至于你们众人,可以再来{\ADD{辩论}}吧!
\par }{\Q 你们中间,我找不着一个智慧人。
\par }{\Q \VS{11}我的日子已经过了;
\par }{\Q 我的谋算、我心所想望的已经断绝。
\par }{\Q \VS{12}他们以黑夜为白昼,
\par }{\Q {\ADD{说}}:亮光近乎黑暗。
\par }{\Q \VS{13}我若盼望阴间为我的房屋,
\par }{\Q 若下榻在黑暗中,
\par }{\Q \VS{14}若对朽坏说:你是我的父;
\par }{\Q 对虫说:你是我的母亲姊妹;
\par }{\Q \VS{15}这样,我的指望在哪里呢?
\par }{\Q 我所指望的谁能看见呢?
\par }{\Q \VS{16}等到安息在尘土中,
\par }{\Q 这指望必下到阴间的门闩那里了。

\par }\PoetryChap{18}{\Q \VerseOne{1}{\PN{书亚}}人{\PN{比勒达}}回答说:
\par }{\Q \VS{2}你寻索言语要到几时呢?
\par }{\Q 你可以揣摩思想,然后我们就说话。
\par }{\Q \VS{3}我们为何算为畜生,
\par }{\Q 在你眼中看作污秽呢?
\par }{\Q \VS{4}你这恼怒将自己撕裂的,
\par }{\Q 难道大地为你见弃、
\par }{\Q 磐石挪开原处吗?
\par }{\BB \par }{\Q \VS{5}恶人的亮光必要熄灭;
\par }{\Q 他的火焰必不照耀。
\par }{\Q \VS{6}他帐棚中的亮光要变为黑暗;
\par }{\Q 他以上的灯也必熄灭。
\par }{\Q \VS{7}他坚强的脚步必见狭窄;
\par }{\Q 自己的计谋必将他绊倒。
\par }{\Q \VS{8}因为他被自己的脚陷入网中,
\par }{\Q 走在缠人的网罗上。
\par }{\Q \VS{9}圈套必抓住他的脚跟;
\par }{\Q 机关必擒获他。
\par }{\Q \VS{10}活扣为他藏在土内;
\par }{\Q 羁绊为他藏在路上。
\par }{\Q \VS{11}四面的惊吓要使他害怕,
\par }{\Q 并且追赶他的脚跟。
\par }{\Q \VS{12}他的力量必因饥饿衰败;
\par }{\Q 祸患要在他旁边等候。
\par }{\Q \VS{13}他本身的肢体要被吞吃;
\par }{\Q 死亡的长子要吞吃他的肢体。
\par }{\Q \VS{14}他要从所倚靠的帐棚被拔出来,
\par }{\Q 带到惊吓的王那里。
\par }{\Q \VS{15}不属他的必住在他的帐棚里;
\par }{\Q 硫磺必撒在他所住之处。
\par }{\Q \VS{16}下边,他的根本要枯干;
\par }{\Q 上边,他的枝子要剪除。
\par }{\Q \VS{17}他的记念在地上必然灭亡;
\par }{\Q 他的名字在街上也不存留。
\par }{\Q \VS{18}他必从光明中被撵到黑暗里,
\par }{\Q 必被赶出世界。
\par }{\Q \VS{19}在本民中必无子无孙;
\par }{\Q 在寄居之地也无一人存留。
\par }{\Q \VS{20}以后来的要惊奇他的日子,
\par }{\Q 好像以前去的受了惊骇。
\par }{\Q \VS{21}不义之人的住处总是这样;
\par }{\Q 此乃不认识 神之人的地步。

\par }\PoetryChap{19}{\Q \VerseOne{1}{\PN{约伯}}回答说:
\par }{\Q \VS{2}你们搅扰我的心,
\par }{\Q 用言语压碎我要到几时呢?
\par }{\Q \VS{3}你们这十次羞辱我;
\par }{\Q 你们苦待我也不以为耻。
\par }{\Q \VS{4}果真我有错,
\par }{\Q 这错乃是在我。
\par }{\Q \VS{5}你们果然要向我夸大,
\par }{\Q 以我的羞辱为证指责我,
\par }{\Q \VS{6}就该知道是 神倾覆我,
\par }{\Q 用网罗围绕我。
\par }{\Q \VS{7}我因委曲呼叫,却不蒙应允;
\par }{\Q 我呼求,却不得公断。
\par }{\Q \VS{8}神用篱笆拦住我的道路,使我不得经过;
\par }{\Q 又使我的路径黑暗。
\par }{\Q \VS{9}他剥去我的荣光,
\par }{\Q 摘去我头上的冠冕。
\par }{\Q \VS{10}他在四围攻击我,我便归于死亡,
\par }{\Q 将我的指望如树拔出来。
\par }{\Q \VS{11}他的忿怒向我发作,
\par }{\Q 以我为敌人。
\par }{\Q \VS{12}他的军旅一齐上来,
\par }{\Q 修筑战路攻击我,
\par }{\Q 在我帐棚的四围安营。
\par }{\BB \par }{\Q \VS{13}他把我的弟兄隔在远处,
\par }{\Q 使我所认识的全然与我生疏。
\par }{\Q \VS{14}我的亲戚与我断绝;
\par }{\Q 我的密友都忘记我。
\par }{\Q \VS{15}在我家寄居的,
\par }{\Q 和我的使女都以我为外人;
\par }{\Q 我在他们眼中看为外邦人。
\par }{\Q \VS{16}我呼唤仆人,
\par }{\Q 虽用口求他,他还是不回答。
\par }{\Q \VS{17}我口的气味,我妻子厌恶;
\par }{\Q 我的恳求,我同胞也憎嫌。
\par }{\Q \VS{18}连小孩子也藐视我;
\par }{\Q 我若起来,他们都嘲笑我。
\par }{\Q \VS{19}我的密友都憎恶我;
\par }{\Q 我平日所爱的人向我翻脸。
\par }{\Q \VS{20}我的皮肉紧贴骨头;
\par }{\Q 我只剩牙皮逃脱了。
\par }{\Q \VS{21}我朋友啊,可怜我!可怜我!
\par }{\Q 因为 神的手攻击我。
\par }{\Q \VS{22}你们为什么仿佛 神逼迫我,
\par }{\Q 吃我的肉还以为不足呢?
\par }{\BB \par }{\Q \VS{23}惟愿我的言语现在写上,
\par }{\Q 都记录在书上;
\par }{\Q \VS{24}用铁笔镌刻,
\par }{\Q 用铅灌在磐石上,直存到永远。
\par }{\Q \VS{25}我知道我的救赎主活着,
\par }{\Q 末了必站立在地上。
\par }{\Q \VS{26}我这皮肉灭绝之后,
\par }{\Q 我必在肉体之外得见 神。
\par }{\Q \VS{27}我自己要见他,
\par }{\Q 亲眼要看他,并不像外人。
\par }{\Q 我的心肠在我里面消灭了!
\par }{\Q \VS{28}你们若说:我们逼迫他要何等地重呢?
\par }{\Q {\ADD{惹}}事的根乃在乎他;
\par }{\Q \VS{29}你们就当惧怕刀剑;
\par }{\Q 因为忿怒{\ADD{惹动}}刀剑的刑罚,
\par }{\Q 使你们知道有报应\FTNT{}{{\FR 19:29: }原文是审判}。

\par }\PoetryChap{20}{\Q \VerseOne{1}{\PN{拿玛}}人{\PN{琐法}}回答说:
\par }{\Q \VS{2}我心中急躁,
\par }{\Q 所以我的思念叫我回答。
\par }{\Q \VS{3}我已听见那羞辱我,责备我的话;
\par }{\Q 我的悟性叫我回答。
\par }{\Q \VS{4}你岂不知亘古以来,
\par }{\Q 自从人生在地,
\par }{\Q \VS{5}恶人夸胜是暂时的,
\par }{\Q 不敬虔人的喜乐不过转眼之间吗?
\par }{\Q \VS{6}他的尊荣虽达到天上,
\par }{\Q 头虽顶到云中,
\par }{\Q \VS{7}他终必灭亡,像自己的粪一样;
\par }{\Q 素来见他的人要说:他在哪里呢?
\par }{\Q \VS{8}他必飞去如梦,不再寻见,
\par }{\Q 速被赶去,如夜间的异象。
\par }{\Q \VS{9}亲眼见过他的,必不再见他;
\par }{\Q 他的本处也再见不着他。
\par }{\Q \VS{10}他的儿女要求穷人的恩;
\par }{\Q 他的手要赔还{\ADD{不义}}之财。
\par }{\Q \VS{11}他的骨头虽然有青年{\ADD{之力}},
\par }{\Q 却要和他一同躺卧在尘土中。
\par }{\BB \par }{\Q \VS{12}他口内虽以恶为甘甜,
\par }{\Q 藏在舌头底下,
\par }{\Q \VS{13}爱恋不舍,含在口中;
\par }{\Q \VS{14}他的食物在肚里却要化为酸,
\par }{\Q 在他里面成为虺蛇的恶毒。
\par }{\Q \VS{15}他吞了财宝,还要吐出;
\par }{\Q  神要从他腹中掏出来。
\par }{\Q \VS{16}他必吸饮虺蛇的毒气;
\par }{\Q 蝮蛇的舌头也必杀他。
\par }{\Q \VS{17}流奶与蜜之河,
\par }{\Q 他不得再见。
\par }{\Q \VS{18}他劳碌得来的要赔还,不得享用\FTNT{}{{\FR 20:18: }原文是吞下};
\par }{\Q 不能照所得的财货欢乐。
\par }{\Q \VS{19}他欺压穷人,且又离弃;
\par }{\Q 强取非自己所盖的房屋
\FTNT{}{{\FR 20:19: }或译:强取房屋不得再建造}。
\par }{\Q \VS{20}他因贪而无厌,
\par }{\Q 所喜悦的连一样也不能保守。
\par }{\Q \VS{21}其余的没有一样他不吞灭,
\par }{\Q 所以他的福乐不能长久。
\par }{\Q \VS{22}他在满足有余的时候,必到狭窄的地步;
\par }{\Q 凡受苦楚的人都必加手在他身上。
\par }{\Q \VS{23}他正要充满肚腹的时候,
\par }{\Q  神必将猛烈的忿怒降在他身上;
\par }{\Q 正在他吃饭的时候,
\par }{\Q 要将这忿怒像雨降在他身上。
\par }{\Q \VS{24}他要躲避铁器;
\par }{\Q 铜弓{\ADD{的箭}}要将他射透。
\par }{\Q \VS{25}他把箭一抽,就从他身上出来;
\par }{\Q 发光的箭头从他胆中出来,
\par }{\Q 有惊惶临在他身上。
\par }{\Q \VS{26}他的财宝归于黑暗;
\par }{\Q {\ADD{人}}所不吹的火要把他烧灭,
\par }{\Q 要把他帐棚中所剩下的烧毁。
\par }{\Q \VS{27}天要显明他的罪孽;
\par }{\Q 地要兴起攻击他。
\par }{\Q \VS{28}他的家产必然过去;
\par }{\Q  神发怒的日子,{\ADD{他的货物}}都要消灭。
\par }{\Q \VS{29}这是恶人从 神所得的分,
\par }{\Q 是 神为他所定的产业。

\par }\PoetryChap{21}{\Q \VerseOne{1}{\PN{约伯}}回答说:
\par }{\Q \VS{2}你们要细听我的言语,
\par }{\Q 就算是你们安慰我。
\par }{\Q \VS{3}请宽容我,我又要说话;
\par }{\Q 说了以后,任凭你们嗤笑吧!
\par }{\Q \VS{4}我岂是向人诉冤?
\par }{\Q 为何不焦急呢?
\par }{\Q \VS{5}你们要看着我而惊奇,
\par }{\Q 用手捂口。
\par }{\Q \VS{6}我每逢思想,心就惊惶,
\par }{\Q 浑身战兢。
\par }{\Q \VS{7}恶人为何存活,
\par }{\Q 享大寿数,势力强盛呢?
\par }{\Q \VS{8}他们眼见儿孙,
\par }{\Q 和他们一同坚立。
\par }{\Q \VS{9}他们的家宅平安无惧;
\par }{\Q  神的杖也不加在他们身上。
\par }{\Q \VS{10}他们的公牛孳生而不断绝;
\par }{\Q 母牛下犊而不掉胎。
\par }{\Q \VS{11}他们打发小孩子出去,{\ADD{多}}如羊群;
\par }{\Q 他们的儿女踊跃跳舞。
\par }{\Q \VS{12}他们随着琴鼓歌唱,
\par }{\Q 又因箫声欢喜。
\par }{\Q \VS{13}他们度日诸事亨通,
\par }{\Q 转眼下入阴间。
\par }{\Q \VS{14}他们对 神说:离开我们吧!
\par }{\Q 我们不愿晓得你的道。
\par }{\Q \VS{15}全能者是谁,我们何必事奉他呢?
\par }{\Q 求告他有什么益处呢?
\par }{\Q \VS{16}看哪,他们亨通不在乎自己;
\par }{\Q 恶人所谋定的离我好远。
\par }{\BB \par }{\Q \VS{17}恶人的灯何尝熄灭?
\par }{\Q 患难何尝临到他们呢?
\par }{\Q  神何尝发怒,{\ADD{向他们}}分散灾祸呢?
\par }{\Q \VS{18}他们何尝像风前的碎秸,
\par }{\Q 如暴风刮去的糠秕呢?
\par }{\Q \VS{19}{\ADD{你们说}}: 神为恶人的儿女积蓄罪孽;
\par }{\Q {\ADD{我说}}:不如本人受报,好使他亲自知道。
\par }{\Q \VS{20}愿他亲眼看见自己败亡,
\par }{\Q 亲自饮全能者的忿怒。
\par }{\Q \VS{21}他的岁月既尽,
\par }{\Q 他还顾他本家吗?
\par }{\Q \VS{22}神既审判那在高位的,
\par }{\Q 谁能将知识教训他呢?
\par }{\Q \VS{23}有人至死身体强壮,
\par }{\Q 尽得平靖安逸;
\par }{\Q \VS{24}他的奶桶充满,
\par }{\Q 他的骨髓滋润。
\par }{\Q \VS{25}有人至死心中痛苦,
\par }{\Q 终身未尝福乐的滋味;
\par }{\Q \VS{26}他们一样躺卧在尘土中,
\par }{\Q 都被虫子遮盖。
\par }{\BB \par }{\Q \VS{27}我知道你们的意思,
\par }{\Q 并诬害我的计谋。
\par }{\Q \VS{28}你们说:霸者的房屋在哪里?
\par }{\Q 恶人住过的帐棚在哪里?
\par }{\Q \VS{29}你们岂没有询问过路的人吗?
\par }{\Q 不知道他们所引的证据吗?
\par }{\Q \VS{30}就是恶人在祸患的日子得存留,
\par }{\Q 在发怒的日子得逃脱。
\par }{\Q \VS{31}他所行的,有谁当面给他说明?
\par }{\Q 他所做的,有谁报应他呢?
\par }{\Q \VS{32}然而他要被抬到茔地;
\par }{\Q 并有人看守坟墓。
\par }{\Q \VS{33}他要以谷中的土块为甘甜;
\par }{\Q 在他以先去的无数,
\par }{\Q 在他以后去的更多。
\par }{\Q \VS{34}你们对答的话中既都错谬,
\par }{\Q 怎么徒然安慰我呢?

\par }\Chap{22}{\SH 第三次对话
\par }{\R (22·1—27·23)
\par }{\Q \VerseOne{1}{\PN{提幔}}人{\PN{以利法}}回答说:
\par }{\Q \VS{2}人岂能使 神有益呢?
\par }{\Q 智慧人但能有益于己。
\par }{\Q \VS{3}你为人公义,岂叫全能者喜悦呢?
\par }{\Q 你行为完全,岂能使他得利呢?
\par }{\Q \VS{4}岂是因你敬畏他
\par }{\Q 就责备你、审判你吗?
\par }{\Q \VS{5}你的罪恶岂不是大吗?
\par }{\Q 你的罪孽也没有穷尽;
\par }{\Q \VS{6}因你无故强取弟兄的物为当头,
\par }{\Q 剥去贫寒人的衣服。
\par }{\Q \VS{7}困乏的人,你没有给他水喝;
\par }{\Q 饥饿的人,你没有给他食物。
\par }{\Q \VS{8}有能力的人就得地土;
\par }{\Q 尊贵的人也住在其中。
\par }{\Q \VS{9}你打发寡妇空手回去,
\par }{\Q 折断孤儿的膀臂。
\par }{\Q \VS{10}因此,有网罗环绕你,
\par }{\Q 有恐惧忽然使你惊惶;
\par }{\Q \VS{11}或有黑暗蒙蔽你,
\par }{\Q 并有洪水淹没你。
\par }{\BB \par }{\Q \VS{12}神岂不是在高天吗?
\par }{\Q 你看星宿何其高呢!
\par }{\Q \VS{13}你说: 神知道什么?
\par }{\Q 他岂能看透幽暗施行审判呢?
\par }{\Q \VS{14}密云将他遮盖,使他不能看见;
\par }{\Q 他周游穹苍。
\par }{\Q \VS{15}你要依从上古的道吗?
\par }{\Q 这道是恶人所行的。
\par }{\Q \VS{16}他们未到死期,忽然除灭;
\par }{\Q 根基毁坏,好像被江河冲去。
\par }{\Q \VS{17}他们向 神说:离开我们吧!
\par }{\Q 又{\ADD{说}}:全能者能把我们怎么样呢?
\par }{\Q \VS{18}哪知 神以美物充满他们的房屋;
\par }{\Q 但恶人所谋定的离我好远。
\par }{\Q \VS{19}义人看见他们的{\ADD{结局}}就欢喜;
\par }{\Q 无辜的人嗤笑他们,
\par }{\Q \VS{20}说:那起来攻击我们的果然被剪除,
\par }{\Q 其余的都被火烧灭。
\par }{\BB \par }{\Q \VS{21}你要认识 神,就得平安;
\par }{\Q 福气也必临到你。
\par }{\Q \VS{22}你当领受他口中的教训,
\par }{\Q 将他的言语存在心里。
\par }{\Q \VS{23}你若归向全能者,从你帐棚中远除不义,
\par }{\Q 就必得建立。
\par }{\Q \VS{24}要将{\ADD{你的}}珍宝丢在尘土里,
\par }{\Q 将{\PN{俄斐}}的{\ADD{黄金}}丢在溪河石头之间;
\par }{\Q \VS{25}全能者就必为你的珍宝,
\par }{\Q 作你的宝银。
\par }{\Q \VS{26}你就要以全能者为喜乐,
\par }{\Q 向 神仰起脸来。
\par }{\Q \VS{27}你要祷告他,他就听你;
\par }{\Q 你也要还你的愿。
\par }{\Q \VS{28}你定意要做何事,必然给你成就;
\par }{\Q 亮光也必照耀你的路。
\par }{\Q \VS{29}人使{\ADD{你}}降卑,你仍可说:{\ADD{必得}}高升;
\par }{\Q 谦卑的人, 神必然拯救。
\par }{\Q \VS{30}人非无辜, 神且要搭救他;
\par }{\Q 他因你手中清洁,必蒙拯救。

\par }\PoetryChap{23}{\Q \VerseOne{1}{\PN{约伯}}回答说:
\par }{\Q \VS{2}如今我的哀告还算为悖逆;
\par }{\Q 我的责罚比我的唉哼还重。
\par }{\Q \VS{3}惟愿我能知道在哪里可以寻见 神,
\par }{\Q 能到他的台前,
\par }{\Q \VS{4}我就在他面前将我的案件陈明,
\par }{\Q 满口辩白。
\par }{\Q \VS{5}我必知道他回答我的言语,
\par }{\Q 明白他向我所说的话。
\par }{\Q \VS{6}他岂用大能与我争辩吗?
\par }{\Q 必不这样!他必理会我。
\par }{\Q \VS{7}在他那里正直人可以与他辩论;
\par }{\Q 这样,我必永远脱离那审判我的。
\par }{\BB \par }{\Q \VS{8}只是,我往前行,他不{\ADD{在那里}},
\par }{\Q 往后退,也不能见他。
\par }{\Q \VS{9}他在左边行事,我却不能看见,
\par }{\Q 在右边隐藏,我也不能见他。
\par }{\Q \VS{10}然而他知道我所行的路;
\par }{\Q 他试炼我之后,我必如精金。
\par }{\Q \VS{11}我脚追随他的步履;
\par }{\Q 我谨守他的道,并不偏离。
\par }{\Q \VS{12}他嘴唇的命令,我未曾背弃;
\par }{\Q 我看重他口中的言语,过于我需用的饮食。
\par }{\Q \VS{13}只是他{\ADD{心志}}已定,谁能使他转意呢?
\par }{\Q 他心里所愿的,就行出来。
\par }{\Q \VS{14}他向我所定的,就必做成;
\par }{\Q 这类的事他还有许多。
\par }{\Q \VS{15}所以我在他面前惊惶;
\par }{\Q 我思念{\ADD{这事}}便惧怕他。
\par }{\Q \VS{16}神使我丧胆;
\par }{\Q 全能者使我惊惶。
\par }{\Q \VS{17}我的恐惧不是因为黑暗,
\par }{\Q 也不是因为幽暗蒙蔽我的脸。

\par }\PoetryChap{24}{\Q \VerseOne{1}全能者既定期{\ADD{罚恶}},
\par }{\Q 为何不使认识他的人看见那日子呢?
\par }{\Q \VS{2}有人挪移地界,
\par }{\Q 抢夺群畜而牧养。
\par }{\Q \VS{3}他们拉去孤儿的驴,
\par }{\Q 强取寡妇的牛为当头。
\par }{\Q \VS{4}他们使穷人离开正道;
\par }{\Q 世上的贫民尽都隐藏。
\par }{\Q \VS{5}这些贫穷人如同野驴出到旷野,殷勤寻找食物;
\par }{\Q 他们{\ADD{靠着}}野地给儿女糊口,
\par }{\Q \VS{6}收割{\ADD{别人}}田间的禾稼,
\par }{\Q 摘取恶人余剩的葡萄,
\par }{\Q \VS{7}终夜赤身无衣,
\par }{\Q 天气寒冷毫无遮盖,
\par }{\Q \VS{8}在山上被大雨淋湿,
\par }{\Q 因没有避身之处就挨近磐石。
\par }{\Q \VS{9}又有人从母怀中抢夺孤儿,
\par }{\Q 强取穷人的{\ADD{衣服}}为当头,
\par }{\Q \VS{10}使人赤身无衣,到处流行,
\par }{\Q 且因饥饿扛抬禾捆,
\par }{\Q \VS{11}在那些人的围墙内造油,榨酒,
\par }{\Q 自己还口渴。
\par }{\Q \VS{12}在多民的城内有人唉哼,
\par }{\Q 受伤的人哀号;
\par }{\Q  神却不理会{\ADD{那恶人}}的愚妄。
\par }{\BB \par }{\Q \VS{13}又有人背弃光明,
\par }{\Q 不认识光明的道,
\par }{\Q 不住在光明的路上。
\par }{\Q \VS{14}杀人的黎明起来,
\par }{\Q 杀害困苦穷乏人,
\par }{\Q 夜间又作盗贼。
\par }{\Q \VS{15}奸夫等候黄昏,
\par }{\Q 说:必无眼能见我,
\par }{\Q 就把脸蒙蔽。
\par }{\Q \VS{16}盗贼黑夜挖窟窿;
\par }{\Q 白日躲藏,
\par }{\Q 并不认识光明。
\par }{\Q \VS{17}他们看早晨如幽暗,
\par }{\Q 因为他们晓得幽暗的惊骇。
\par }{\BB \par }{\Q \VS{18}{\ADD{这些恶人犹如浮萍}}快快{\ADD{飘去}}。
\par }{\Q 他们所得的分在世上被咒诅;
\par }{\Q 他们不得再走葡萄园的路。
\par }{\Q \VS{19}干旱炎热消没雪水;
\par }{\Q 阴间也如此{\ADD{消没}}犯罪之辈。
\par }{\Q \VS{20}怀他的母\FTNT{}{{\FR 24:20: }原文是胎}要忘记他;
\par }{\Q 虫子要吃他,觉得甘甜;
\par }{\Q 他不再被人记念。
\par }{\Q 不义{\ADD{的人}}必如树折断。
\par }{\BB \par }{\Q \VS{21}他恶待\FTNT{}{{\FR 24:21: }或译:他吞灭}不怀孕不生养的妇人,
\par }{\Q 不善待寡妇。
\par }{\Q \VS{22}然而 神用能力保全有势力的人;
\par }{\Q 那性命难保的人仍然兴起。
\par }{\Q \VS{23}神使他们安稳,他们就有所倚靠;
\par }{\Q  神的眼目也看顾他们的道路。
\par }{\Q \VS{24}他们被高举,不过片时就没有了;
\par }{\Q 他们降为卑,被除灭,与众人一样,
\par }{\Q 又如谷穗被割。
\par }{\Q \VS{25}若不是这样,谁能证实我是说谎的,
\par }{\Q 将我的言语驳为虚空呢?

\par }\PoetryChap{25}{\Q \VerseOne{1}{\PN{书亚}}人{\PN{比勒达}}回答说:
\par }{\Q \VS{2}神有治理之权,有威严可畏;
\par }{\Q 他在高处施行和平。
\par }{\Q \VS{3}他的诸军岂能数算?
\par }{\Q 他的光亮一发,谁不蒙照呢?
\par }{\Q \VS{4}这样在 神面前,人怎能称义?
\par }{\Q 妇人所生的怎能洁净?
\par }{\Q \VS{5}在 神眼前,月亮也无光亮,
\par }{\Q 星宿也不清洁。
\par }{\Q \VS{6}何况如虫的人,
\par }{\Q 如蛆的世人呢!

\par }\PoetryChap{26}{\Q \VerseOne{1}{\PN{约伯}}回答说:
\par }{\Q \VS{2}无能的人蒙你何等的帮助!
\par }{\Q 膀臂无力的人蒙你何等的拯救!
\par }{\Q \VS{3}无智慧的人蒙你何等的指教!
\par }{\Q 你向他多显大知识!
\par }{\Q \VS{4}你向谁发出言语来?
\par }{\Q 谁的灵从你而出?
\par }{\Q \VS{5}在大水和水族以下的阴魂战兢。
\par }{\Q \VS{6}在 神面前,阴间显露;
\par }{\Q 灭亡也不得遮掩。
\par }{\Q \VS{7}神将北极铺在空中,
\par }{\Q 将大地悬在虚空;
\par }{\Q \VS{8}将水包在密云中,
\par }{\Q 云却不破裂;
\par }{\Q \VS{9}遮蔽他的宝座,
\par }{\Q 将云铺在其上;
\par }{\Q \VS{10}在水面的周围划出界限,
\par }{\Q 直到光明黑暗的交界。
\par }{\Q \VS{11}天的柱子因他的斥责震动惊奇。
\par }{\Q \VS{12}他以能力搅动\FTNT{}{{\FR 26:12: }或译:平静}大海;
\par }{\Q 他借知识打伤{\PN{拉哈伯}},
\par }{\Q \VS{13}借他的灵使天有妆饰;
\par }{\Q 他的手刺杀快蛇。
\par }{\Q \VS{14}看哪,这不过是 神工作的些微;
\par }{\Q 我们所听于他的是何等细微的声音!
\par }{\Q 他大能的雷声谁能明透呢?

\par }\PoetryChap{27}{\Q \VerseOne{1}{\PN{约伯}}接着说:
\par }{\Q \VS{2}神夺去我的理,全能者使我心中愁苦。
\par }{\Q 我指着永生的 神起誓:
\par }{\Q \VS{3}我的生命尚在我里面;
\par }{\Q  神所赐呼吸之气仍在我的鼻孔内。
\par }{\Q \VS{4}我的嘴决不说非义之言;
\par }{\Q 我的舌也不说诡诈之语。
\par }{\Q \VS{5}我断不以你们为是;
\par }{\Q 我至死必不以自己为不正!
\par }{\Q \VS{6}我持定我的义,必不放松;
\par }{\Q 在世的日子,我心必不责备我。
\par }{\BB \par }{\Q \VS{7}愿我的仇敌如恶人一样;
\par }{\Q 愿那起来攻击我的,如不义之人一般。
\par }{\Q \VS{8}不敬虔的人虽然得利,
\par }{\Q  神夺取其命的时候还有什么指望呢?
\par }{\Q \VS{9}患难临到他,
\par }{\Q  神岂能听他的呼求?
\par }{\Q \VS{10}他岂以全能者为乐,
\par }{\Q 随时求告 神呢?
\par }{\Q \VS{11}神的作为,我要指教你们;
\par }{\Q 全能者所行的,我也不隐瞒。
\par }{\Q \VS{12}你们自己也都见过,
\par }{\Q 为何全然变为虚妄呢?
\par }{\BB \par }{\Q \VS{13}神为恶人所定的分,
\par }{\Q 强暴人从全能者所得的报\FTNT{}{{\FR 27:13: }原文是产业}乃是这样:
\par }{\Q \VS{14}倘或他的儿女增多,还是被刀所杀;
\par }{\Q 他的子孙必不得饱食。
\par }{\Q \VS{15}他所遗留的人必死而埋葬;
\par }{\Q 他的寡妇也不哀哭。
\par }{\Q \VS{16}他虽积蓄银子如尘沙,
\par }{\Q 预备衣服如泥土;
\par }{\Q \VS{17}他只管预备,义人却要穿上;
\par }{\Q 他的银子,无辜的人要分取。
\par }{\Q \VS{18}他建造房屋如虫做窝,
\par }{\Q 又如守望者所搭的棚。
\par }{\Q \VS{19}他虽富足躺卧,却不得收殓,
\par }{\Q 转眼之间就不在了。
\par }{\Q \VS{20}惊恐如波涛将他追上;
\par }{\Q 暴风在夜间将他刮去。
\par }{\Q \VS{21}东风把他飘去,
\par }{\Q 又刮他离开本处。
\par }{\Q \VS{22}神要向他射箭,并不留情;
\par }{\Q 他恨不得逃脱 神的手。
\par }{\Q \VS{23}人要向他拍掌,
\par }{\Q 并要发叱声,使他离开本处。

\par }\Chap{28}{\SH 歌颂智慧
\par }{\Q \VerseOne{1}银子有矿;
\par }{\Q 炼金有方。
\par }{\Q \VS{2}铁从地里挖出;
\par }{\Q 铜从石中熔化。
\par }{\Q \VS{3}人为黑暗定界限,
\par }{\Q 查究幽暗阴翳的石头,直到极处,
\par }{\Q \VS{4}在无人居住之处刨开矿穴,
\par }{\Q {\ADD{过路的人}}也想不到他们;
\par }{\Q 又与人远离,悬在空中摇来摇去。
\par }{\Q \VS{5}至于地,能出粮食,
\par }{\Q 地内好像被火翻起来。
\par }{\Q \VS{6}地中的石头有蓝宝石,
\par }{\Q 并有金沙。
\par }{\Q \VS{7}矿中的路鸷鸟不得知道;
\par }{\Q 鹰眼也未见过。
\par }{\Q \VS{8}狂傲的野兽未曾行过;
\par }{\Q 猛烈的狮子也未曾经过。
\par }{\BB \par }{\Q \VS{9}人伸手凿开坚石,
\par }{\Q 倾倒山根,
\par }{\Q \VS{10}在磐石中凿出水道,
\par }{\Q 亲眼看见各样宝物。
\par }{\Q \VS{11}他封闭水不得滴流,
\par }{\Q 使隐藏的物显露出来。
\par }{\BB \par }{\Q \VS{12}然而,智慧有何处可寻?
\par }{\Q 聪明之处在哪里呢?
\par }{\Q \VS{13}智慧的价值无人能知,
\par }{\Q 在活人之地也无处可寻。
\par }{\Q \VS{14}深渊说:不在我内;
\par }{\Q 沧海说:不在我中。
\par }{\Q \VS{15}智慧非用黄金可得,
\par }{\Q 也不能平白银为它的价值。
\par }{\Q \VS{16}{\PN{俄斐}}金和贵重的红玛瑙,
\par }{\Q 并蓝宝石,不足与较量;
\par }{\Q \VS{17}黄金和玻璃不足与比较;
\par }{\Q 精金的器皿不足与兑换。
\par }{\Q \VS{18}珊瑚、水晶都不足论;
\par }{\Q 智慧的价值胜过珍珠\FTNT{}{{\FR 28:18: }或译:红宝石}。
\par }{\Q \VS{19}{\PN{古实}}的红璧玺不足与比较;
\par }{\Q 精金也不足与较量。
\par }{\BB \par }{\Q \VS{20}智慧从何处来呢?
\par }{\Q 聪明之处在哪里呢?
\par }{\Q \VS{21}是向一切有生命的眼目隐藏,
\par }{\Q 向空中的飞鸟掩蔽。
\par }{\Q \VS{22}灭没和死亡说:
\par }{\Q 我们风闻其名。
\par }{\BB \par }{\Q \VS{23}神明白智慧的道路,
\par }{\Q 晓得智慧的所在。
\par }{\Q \VS{24}因他鉴察直到地极,
\par }{\Q 遍观普天之下,
\par }{\Q \VS{25}要为风定轻重,
\par }{\Q 又度量诸水;
\par }{\Q \VS{26}他为雨露定命令,
\par }{\Q 为雷电定道路。
\par }{\Q \VS{27}那时他看见智慧,而且述说;
\par }{\Q 他坚定,并且查究。
\par }{\Q \VS{28}他对人说:敬畏主就是智慧;
\par }{\Q 远离恶便是聪明。

\par }\Chap{29}{\SH 约伯最后的申诉
\par }{\Q \VerseOne{1}{\PN{约伯}}又接着说:
\par }{\Q \VS{2}惟愿我{\ADD{的景况}}如从前的月份,
\par }{\Q 如 神保守我的日子。
\par }{\Q \VS{3}那时他的灯照在我头上;
\par }{\Q 我借他的光行过黑暗。
\par }{\Q \VS{4}我愿如壮年的时候:
\par }{\Q 那时我在帐棚中,
\par }{\Q  神待我有密友之情;
\par }{\Q \VS{5}全能者仍与我同在;
\par }{\Q 我的儿女都环绕我。
\par }{\Q \VS{6}奶多可洗我的脚;
\par }{\Q 磐石为我出油成河。
\par }{\Q \VS{7}我出到城门,
\par }{\Q 在街上设立座位;
\par }{\Q \VS{8}少年人见我而回避,
\par }{\Q 老年人也起身站立;
\par }{\Q \VS{9}王子都停止说话,
\par }{\Q 用手捂口;
\par }{\Q \VS{10}首领静默无声,
\par }{\Q 舌头贴住上膛。
\par }{\Q \VS{11}耳朵听我的,就称我有福;
\par }{\Q 眼睛看我的,便称赞我;
\par }{\Q \VS{12}因我拯救哀求的困苦人
\par }{\Q 和无人帮助的孤儿。
\par }{\Q \VS{13}将要灭亡的为我祝福;
\par }{\Q 我也使寡妇心中欢乐。
\par }{\Q \VS{14}我以公义为衣服,
\par }{\Q 以公平为外袍和冠冕。
\par }{\Q \VS{15}我为瞎子的眼,
\par }{\Q 瘸子的脚。
\par }{\Q \VS{16}我为穷乏人的父;
\par }{\Q 素不认识的人,我查明他的案件。
\par }{\Q \VS{17}我打破不义之人的牙床,
\par }{\Q 从他牙齿中夺了所抢的。
\par }{\Q \VS{18}我便说:我必死在家中\FTNT{}{{\FR 29:18: }原文是窝中},
\par }{\Q 必增添我的日子,多如尘沙。
\par }{\Q \VS{19}我的根长到水边;
\par }{\Q 露水终夜沾在我的枝上。
\par }{\Q \VS{20}我的荣耀在身上增新;
\par }{\Q 我的弓在手中日强。
\par }{\BB \par }{\Q \VS{21}人听见我而仰望,
\par }{\Q 静默等候我的指教。
\par }{\Q \VS{22}我说话之后,他们就不再说;
\par }{\Q 我的言语{\ADD{像雨露}}滴在他们身上。
\par }{\Q \VS{23}他们仰望我如仰望雨,
\par }{\Q 又张开口如切慕春雨。
\par }{\Q \VS{24}他们不敢自信,我就向他们含笑;
\par }{\Q 他们不使我脸上的光改变。
\par }{\Q \VS{25}我为他们选择道路,又坐首位;
\par }{\Q 我如君王在军队中居住,
\par }{\Q 又如吊丧的安慰伤心的人。

\par }\PoetryChap{30}{\Q \VerseOne{1}但如今,比我年少的人戏笑我;
\par }{\Q 其人之父我曾藐视,
\par }{\Q 不肯安在看守我羊群的狗中。
\par }{\Q \VS{2}他们壮年的气力既已衰败,
\par }{\Q 其手之力与我何益呢?
\par }{\Q \VS{3}他们因穷乏饥饿,身体枯瘦,
\par }{\Q 在荒废凄凉的幽暗中啃干燥之地,
\par }{\Q \VS{4}在草丛之中采咸草,
\par }{\Q 罗腾\FTNT{}{{\FR 30:4: }小树名,松类}的根为他们的食物。
\par }{\Q \VS{5}他们从人中被赶出;
\par }{\Q 人追喊他们如贼一般,
\par }{\Q \VS{6}以致他们住在荒谷之间,
\par }{\Q 在地洞和岩穴中;
\par }{\Q \VS{7}在草丛中叫唤,
\par }{\Q 在荆棘下聚集。
\par }{\Q \VS{8}这都是愚顽下贱人的儿女;
\par }{\Q 他们被鞭打,赶出境外。
\par }{\BB \par }{\Q \VS{9}现在这些人以我为歌曲,
\par }{\Q 以我为笑谈。
\par }{\Q \VS{10}他们厌恶我,躲在旁边站着,
\par }{\Q 不住地吐唾沫在我脸上。
\par }{\Q \VS{11}松开他们的绳索苦待我,
\par }{\Q 在我面前脱去辔头。
\par }{\Q \VS{12}这等下流人在我右边起来,
\par }{\Q 推开我的脚,筑成战路来攻击我。
\par }{\Q \VS{13}这些无人帮助的,
\par }{\Q 毁坏我的道,加增我的灾。
\par }{\Q \VS{14}他们来如同闯进大破口,
\par }{\Q 在毁坏之间滚{\ADD{在我身上}}。
\par }{\Q \VS{15}惊恐临到我,
\par }{\Q 驱逐我的尊荣如风;
\par }{\Q 我的福禄如云过去。
\par }{\BB \par }{\Q \VS{16}现在我心极其悲伤;
\par }{\Q 困苦的日子将我抓住。
\par }{\Q \VS{17}夜间,我里面的骨头刺我,
\par }{\Q {\ADD{疼痛}}不止,好像啃我。
\par }{\Q \VS{18}因 {\ADD{神}}的大力,我的外衣污秽不堪,
\par }{\Q 又如里衣的领子将我缠住。
\par }{\Q \VS{19}神把我扔在淤泥中,
\par }{\Q 我就像尘土和炉灰一般。
\par }{\Q \VS{20}{\ADD{主啊}},我呼求你,你不应允我;
\par }{\Q 我站起来,你就定睛看我。
\par }{\Q \VS{21}你向我变心,待我残忍,
\par }{\Q 又用大能追逼我,
\par }{\Q \VS{22}把我提在风中,使我驾风而行,
\par }{\Q 又使我消灭在烈风中。
\par }{\Q \VS{23}我知道要使我临到死地,
\par }{\Q 到那为众生所定的阴宅。
\par }{\BB \par }{\Q \VS{24}然而,人仆倒岂不伸手?
\par }{\Q 遇灾难岂不求救呢?
\par }{\Q \VS{25}人遭难,我岂不为他哭泣呢?
\par }{\Q 人穷乏,我岂不为他忧愁呢?
\par }{\Q \VS{26}我仰望得好处,灾祸就到了;
\par }{\Q 我等待光明,黑暗便来了。
\par }{\Q \VS{27}我心里烦扰不安,
\par }{\Q 困苦的日子临到我身。
\par }{\Q \VS{28}我没有日光就哀哭行去
\FTNT{}{{\FR 30:28: }或译:我面发黑并非因日晒};
\par }{\Q 我在会中站着求救。
\par }{\Q \VS{29}我与野狗为弟兄,
\par }{\Q 与鸵鸟为同伴。
\par }{\Q \VS{30}我的皮肤黑{\ADD{而脱落}};
\par }{\Q 我的骨头因热烧焦。
\par }{\Q \VS{31}所以,我的琴音{\ADD{变}}为悲音;
\par }{\Q 我的箫声{\ADD{变}}为哭声。

\par }\PoetryChap{31}{\Q \VerseOne{1}我与眼睛立约,
\par }{\Q 怎能恋恋瞻望处女呢?
\par }{\Q \VS{2}从至上的 神所得之分,
\par }{\Q 从至高全能者所得之业是什么呢?
\par }{\Q \VS{3}岂不是祸患临到不义的,
\par }{\Q 灾害临到作孽的呢?
\par }{\Q \VS{4}神岂不是察看我的道路,
\par }{\Q 数点我的脚步呢?
\par }{\BB \par }{\Q \VS{5}我若与虚谎同行,
\par }{\Q 脚若追随诡诈;
\par }{\Q (
\VS{6}我若被公道的天平称度,
\par }{\Q 使 神可以知道我的纯正;)
\par }{\Q \VS{7}我的脚步若偏离正路,
\par }{\Q 我的心若随着我的眼目,
\par }{\Q 若有玷污粘在我手上;
\par }{\Q \VS{8}就愿我所种的有别人吃,
\par }{\Q 我田所产的被拔出来。
\par }{\BB \par }{\Q \VS{9}我若受迷惑,向妇人{\ADD{起淫念}},
\par }{\Q 在邻舍的门外蹲伏,
\par }{\Q \VS{10}就愿我的妻子给别人推磨,
\par }{\Q 别人也与她同室。
\par }{\Q \VS{11}因为这是大罪,
\par }{\Q 是审判官{\ADD{当罚的}}罪孽。
\par }{\Q \VS{12}这本是火焚烧,直到毁灭,
\par }{\Q 必拔除我所有的家产。
\par }{\BB \par }{\Q \VS{13}我的仆婢与我争辩的时候,
\par }{\Q 我若藐视不听他们的情节;
\par }{\Q \VS{14}神兴起,我怎样行呢?
\par }{\Q 他察问,我怎样回答呢?
\par }{\Q \VS{15}造我在腹中的,不也是造他吗?
\par }{\Q 将他与我抟在腹中的岂不是一位吗?
\par }{\BB \par }{\Q \VS{16}我若不容贫寒人得其所愿,
\par }{\Q 或叫寡妇眼中失望,
\par }{\Q \VS{17}或独自吃我一点食物,
\par }{\Q 孤儿没有与我同吃;
\par }{\Q (
\VS{18}从幼年时孤儿与我同长,好像父子一样;
\par }{\Q 我从出母腹就扶助\FTNT{}{{\FR 31:18: }原文是引领}寡妇。)
\par }{\Q \VS{19}我若见人因无衣死亡,
\par }{\Q 或见穷乏人身无遮盖;
\par }{\Q \VS{20}我若不使他因我羊的毛得暖,
\par }{\Q 为我祝福;
\par }{\Q \VS{21}我若在城门口见有帮助我的,
\par }{\Q 举手攻击孤儿;
\par }{\Q \VS{22}情愿我的肩头从缺盆骨脱落,
\par }{\Q 我的膀臂从羊矢骨折断。
\par }{\Q \VS{23}因 神降的灾祸使我恐惧;
\par }{\Q 因他的威严,我不能妄为。
\par }{\BB \par }{\Q \VS{24}我若以黄金为指望,
\par }{\Q 对精金说:{\ADD{你是}}我的倚靠;
\par }{\Q \VS{25}我若因财物丰裕,
\par }{\Q 因我手多得资财而欢喜;
\par }{\Q \VS{26}我若见太阳发光,
\par }{\Q 明月行在空中,
\par }{\Q \VS{27}心就暗暗被引诱,
\par }{\Q 口便亲手;
\par }{\Q \VS{28}这也是审判官{\ADD{当罚的}}罪孽,
\par }{\Q 又是我背弃在上的 神。
\par }{\BB \par }{\Q \VS{29}我若见恨我的遭报就欢喜,
\par }{\Q 见他遭灾便高兴;
\par }{\Q (
\VS{30}我没有容口犯罪,
\par }{\Q 咒诅他的生命;)
\par }{\Q \VS{31}若我帐棚的人未尝说,
\par }{\Q 谁不以主人的食物吃饱呢?
\par }{\Q (
\VS{32}从来我没有容客旅在街上住宿,
\par }{\Q 却开门迎接行路的人;)
\par }{\Q \VS{33}我若像{\PN{亚当}}\FTNT{}{{\FR 31:33: }或译:别人}遮掩我的过犯,
\par }{\Q 将罪孽藏在怀中;
\par }{\Q \VS{34}因惧怕大众,
\par }{\Q 又因宗族藐视我使我惊恐,
\par }{\Q 以致闭口无言,杜门不出;
\par }{\Q \VS{35}惟愿有一位肯听我!
\par }{\Q (看哪,在这里有我所划的押,
\par }{\Q 愿全能者回答我!)
\par }{\Q \VS{36}愿那敌我者所写的状词{\ADD{在我这里}}!
\par }{\Q 我必带在肩上,又绑在头上为冠冕。
\par }{\Q \VS{37}我必向他述说我脚步的数目,
\par }{\Q 必如君王进到他面前。
\par }{\BB \par }{\Q \VS{38}我若{\ADD{夺取}}田地,这地向我喊冤,
\par }{\Q 犁沟一同哭泣;
\par }{\Q \VS{39}我若吃地的出产不给价值,
\par }{\Q 或叫原主丧命;
\par }{\Q \VS{40}愿这地长蒺藜代替麦子,
\par }{\Q 长恶草代替大麦。
\par }{\BB \par }{\Q {\PN{约伯}}的话说完了。

\par }\Chap{32}{\SH 以利户的话
\par }{\R (32·1—37·24)
\par }{\PP \VerseOne{1}于是这三个人,因{\PN{约伯}}自以为义就不再回答他。
\VS{2}那时有{\PN{布西}}人{\PN{兰}}族{\PN{巴拉迦}}的儿子{\PN{以利户}}向{\PN{约伯}}发怒;因{\PN{约伯}}自以为义,不以 神为义。
\VS{3}他又向{\PN{约伯}}的三个朋友发怒;因为他们想不出回答的话来,仍以{\PN{约伯}}为有罪。
\VS{4}{\PN{以利户}}要与{\PN{约伯}}说话,就等候他们,因为他们比自己年老。
\VS{5}{\PN{以利户}}见这三个人口中无话回答,就怒气发作。
\par }{\PP \VS{6}{\PN{布西}}人{\PN{巴拉迦}}的儿子{\PN{以利户}}回答说:
\par }{\BB \par }{\Q 我年轻,你们老迈;
\par }{\Q 因此我退让,不敢向你们陈说我的意见。
\par }{\Q \VS{7}我说,年老的当先说话;
\par }{\Q 寿高的当以智慧教训人。
\par }{\Q \VS{8}但在人里面有灵;
\par }{\Q 全能者的气使人有聪明。
\par }{\Q \VS{9}尊贵的不都有智慧;
\par }{\Q 寿高的不都能明白公平。
\par }{\Q \VS{10}因此我说:你们要听我言;
\par }{\Q 我也要陈说我的意见。
\par }{\BB \par }{\Q \VS{11}你们查究所要说的话;
\par }{\Q 那时我等候你们的话,
\par }{\Q 侧耳听你们的辩论,
\par }{\Q \VS{12}留心听你们;
\par }{\Q 谁知你们中间无一人折服{\PN{约伯}},
\par }{\Q 驳倒他的话。
\par }{\Q \VS{13}你们切不可说:我们寻得智慧;
\par }{\Q  神能胜他,人却不能。
\par }{\Q \VS{14}{\PN{约伯}}没有向我争辩;
\par }{\Q 我也不用你们的话回答他。
\par }{\BB \par }{\Q \VS{15}他们惊奇不再回答,
\par }{\Q 一言不发。
\par }{\Q \VS{16}我岂因他们不说话,
\par }{\Q 站住不再回答,仍旧等候呢?
\par }{\Q \VS{17}我也要回答我的一分话,
\par }{\Q 陈说我的意见。
\par }{\Q \VS{18}因为我的言语满怀;
\par }{\Q 我里面的灵激动我。
\par }{\Q \VS{19}我的胸怀如盛酒{\ADD{之囊}}没有出气之缝,
\par }{\Q 又如新皮袋快要破裂。
\par }{\Q \VS{20}我要说话,使我舒畅;
\par }{\Q 我要开口回答。
\par }{\Q \VS{21}我必不看人的情面,
\par }{\Q 也不奉承人。
\par }{\Q \VS{22}我不晓得奉承;
\par }{\Q {\ADD{若奉承}},造我的主必快快除灭我。

\par }\PoetryChap{33}{\Q \VerseOne{1}{\PN{约伯}}啊,请听我的话,
\par }{\Q 留心听我一切的言语。
\par }{\Q \VS{2}我现在开口,
\par }{\Q 用舌发言。
\par }{\Q \VS{3}我的言语{\ADD{要发明}}心中所存的正直;
\par }{\Q 我所知道的,我嘴唇要诚实地说出。
\par }{\Q \VS{4}神的灵造我;
\par }{\Q 全能者的气使我得生。
\par }{\Q \VS{5}你若回答我,
\par }{\Q 就站起来,在我面前陈明。
\par }{\Q \VS{6}我在 神面前与你一样,
\par }{\Q 也是用土造成。
\par }{\Q \VS{7}我不用威严惊吓你,
\par }{\Q 也不用势力重压你。
\par }{\BB \par }{\Q \VS{8}你所说的,我听见了,
\par }{\Q 也听见你的言语,{\ADD{说}}:
\par }{\Q \VS{9}我是清洁无过的,我是无辜的;
\par }{\Q 在我里面也没有罪孽。
\par }{\Q \VS{10}神找机会攻击我,
\par }{\Q 以我为仇敌,
\par }{\Q \VS{11}把我的脚上了木狗,
\par }{\Q 窥察我一切的道路。
\par }{\BB \par }{\Q \VS{12}我要回答你说:你这话无理,
\par }{\Q 因 神比世人更大。
\par }{\Q \VS{13}你为何与他争论呢?
\par }{\Q 因他的事都不对人解说?
\par }{\Q \VS{14}神说一次、两次,
\par }{\Q {\ADD{世人}}却不理会。
\par }{\Q \VS{15}人躺在床上沉睡的时候,
\par }{\Q  神就用梦和夜间的异象,
\par }{\Q \VS{16}开通他们的耳朵,
\par }{\Q 将当受的教训印在他们心上,
\par }{\Q \VS{17}好叫人不从自己的谋算,
\par }{\Q 不行骄傲的事\FTNT{}{{\FR 33:17: }原文是将骄傲向人隐藏},
\par }{\Q \VS{18}拦阻人不陷于坑里,
\par }{\Q 不死在刀下。
\par }{\BB \par }{\Q \VS{19}人在床上被惩治,
\par }{\Q 骨头中不住地疼痛,
\par }{\Q \VS{20}以致他的口厌弃食物,
\par }{\Q 心厌恶美味。
\par }{\Q \VS{21}他的肉消瘦,不得再见;
\par }{\Q 先前不见的骨头都凸出来。
\par }{\Q \VS{22}他的灵魂临近深坑;
\par }{\Q 他的生命近于灭命的。
\par }{\Q \VS{23}一千天使中,
\par }{\Q 若有一个作传话的与 神同在,
\par }{\Q 指示人所当行的事,
\par }{\Q \VS{24}神就给他开恩,
\par }{\Q 说:救赎他免得下坑;
\par }{\Q 我已经得了赎价。
\par }{\Q \VS{25}他的肉要比孩童的肉更嫩;
\par }{\Q 他就返老还童。
\par }{\Q \VS{26}他祷告 神,
\par }{\Q  神就喜悦他,
\par }{\Q 使他欢呼朝见 神的面;
\par }{\Q  神又看他为义。
\par }{\Q \VS{27}他在人前歌唱说:
\par }{\Q 我犯了罪,颠倒是非,
\par }{\Q 这竟与我无益。
\par }{\Q \VS{28}神救赎我的灵魂免入深坑;
\par }{\Q 我的生命也必见光。
\par }{\BB \par }{\Q \VS{29}神两次、三次向人行这一切的事,
\par }{\Q \VS{30}为要从深坑救回人的灵魂,
\par }{\Q 使他被光照耀,与活人一样。
\par }{\Q \VS{31}{\PN{约伯}}啊,你当侧耳听我的话,
\par }{\Q 不要作声,等我讲说。
\par }{\Q \VS{32}你若有话说,就可以回答我;
\par }{\Q 你只管说,因我愿以你为是。
\par }{\Q \VS{33}若不然,你就听我说;
\par }{\Q 你不要作声,我便将智慧教训你。

\par }\PoetryChap{34}{\Q \VerseOne{1}{\PN{以利户}}又说:
\par }{\Q \VS{2}你们智慧人要听我的话;
\par }{\Q 有知识的人要留心听我说。
\par }{\Q \VS{3}因为耳朵试验话语,
\par }{\Q 好像上膛尝食物。
\par }{\Q \VS{4}我们当选择何为是,
\par }{\Q 彼此知道何为善。
\par }{\Q \VS{5}{\PN{约伯}}曾说:我是公义,
\par }{\Q  神夺去我的理;
\par }{\Q \VS{6}我虽有理,还{\ADD{算为}}说谎言的;
\par }{\Q 我虽无过,受的伤还不能医治。
\par }{\Q \VS{7}谁像{\PN{约伯}},
\par }{\Q 喝讥诮如同喝水呢?
\par }{\Q \VS{8}他与作孽的结伴,
\par }{\Q 和恶人同行。
\par }{\Q \VS{9}他说:人以 神为乐,
\par }{\Q 总是无益。
\par }{\BB \par }{\Q \VS{10}所以,你们明理的人要听我的话。
\par }{\Q  神断不致行恶;
\par }{\Q 全能者断不致作孽。
\par }{\Q \VS{11}他必按人所做的报应人,
\par }{\Q 使各人照所行的得报。
\par }{\Q \VS{12}神必不作恶;
\par }{\Q 全能者也不偏离公平。
\par }{\Q \VS{13}谁派他治理地,
\par }{\Q 安定全世界呢?
\par }{\Q \VS{14}他若专心为己,
\par }{\Q 将灵和气收归自己,
\par }{\Q \VS{15}凡有血气的就必一同死亡;
\par }{\Q 世人必仍归尘土。
\par }{\BB \par }{\Q \VS{16}你若明理,就当听我的话,
\par }{\Q 留心听我言语的声音。
\par }{\Q \VS{17}难道恨恶公平的可以掌权吗?
\par }{\Q 那有公义的、有大能的,岂可定他有罪吗?
\par }{\Q \VS{18}他对君王说:{\ADD{你是}}鄙陋的;
\par }{\Q 对贵臣说:{\ADD{你是}}邪恶的。
\par }{\Q \VS{19}他待王子不徇情面,
\par }{\Q 也不看重富足的过于贫穷的,
\par }{\Q 因为都是他手所造。
\par }{\Q \VS{20}在转眼之间,半夜之中,
\par }{\Q 他们就死亡。
\par }{\Q 百姓被震动而去世;
\par }{\Q 有权力的被夺去非借人手。
\par }{\BB \par }{\Q \VS{21}神注目观看人的道路,
\par }{\Q 看明人的脚步。
\par }{\Q \VS{22}没有黑暗、阴翳能给作孽的藏身。
\par }{\Q \VS{23}神审判人,不必使人到他面前再三鉴察。
\par }{\Q \VS{24}他用难测之法打破有能力的人,
\par }{\Q 设立别人代替他们。
\par }{\Q \VS{25}他原知道他们的行为,
\par }{\Q 使他们在夜间倾倒灭亡。
\par }{\Q \VS{26}他在众人眼前击打他们,
\par }{\Q 如同击打恶人一样。
\par }{\Q \VS{27}因为他们偏行不跟从他,
\par }{\Q 也不留心他的道,
\par }{\Q \VS{28}甚至使贫穷人的哀声达到他那里;
\par }{\Q 他也听了困苦人的哀声。
\par }{\BB \par }{\Q \VS{29}他使人安静,谁能扰乱\FTNT{}{{\FR 34:29: }或译:定罪}呢?
\par }{\Q 他掩面,谁能见他呢?
\par }{\Q 无论待一国或一人都是如此—
\par }{\Q \VS{30}使不虔敬的人不得作王,
\par }{\Q 免得有人牢笼百姓。
\par }{\BB \par }{\Q \VS{31}有谁对 神说:
\par }{\Q 我受了{\ADD{责罚}},不再犯罪;
\par }{\Q \VS{32}我所看不明的,求你指教我;
\par }{\Q 我若作了孽,必不再作?
\par }{\Q \VS{33}他施行报应,
\par }{\Q 岂要随你的心愿、叫你推辞不受吗?
\par }{\Q 选定的是你,不是我。
\par }{\Q 你所知道的只管说吧!
\par }{\Q \VS{34}明理的人和听我话的智慧人必对我说:
\par }{\Q \VS{35}{\PN{约伯}}说话没有知识,
\par }{\Q 言语中毫无智慧。
\par }{\Q \VS{36}愿{\PN{约伯}}被试验到底,
\par }{\Q 因他回答像恶人一样。
\par }{\Q \VS{37}他在罪上又加悖逆;
\par }{\Q 在我们中间拍手,
\par }{\Q 用许多言语轻慢 神。

\par }\PoetryChap{35}{\Q \VerseOne{1}{\PN{以利户}}又说:
\par }{\Q \VS{2}你以为有理,
\par }{\Q 或以为你的公义胜于 神的公义,
\par }{\Q \VS{3}才说这与我有什么益处?
\par }{\Q {\ADD{我不犯罪}}比犯罪有什么好处呢?
\par }{\Q \VS{4}我要回答你和在你这里的朋友。
\par }{\Q \VS{5}你要向天观看,
\par }{\Q 瞻望那高于你的穹苍。
\par }{\Q \VS{6}你若犯罪,能使 神受何害呢?
\par }{\Q 你的过犯加增,能使 神受何损呢?
\par }{\Q \VS{7}你若是公义,还能加增他什么呢?
\par }{\Q 他从你手里还接受什么呢?
\par }{\Q \VS{8}你的过恶{\ADD{或能害}}你这类的人;
\par }{\Q 你的公义{\ADD{或能叫}}世人{\ADD{得益处}}。
\par }{\BB \par }{\Q \VS{9}人因多受欺压就哀求,
\par }{\Q 因受能者的辖制\FTNT{}{{\FR 35:9: }原文是膀臂}便求救,
\par }{\Q \VS{10}却无人说:造我的 神在哪里?
\par }{\Q 他使人夜间歌唱,
\par }{\Q \VS{11}教训我们胜于地上的走兽,
\par }{\Q 使我们有聪明胜于空中的飞鸟。
\par }{\Q \VS{12}他们在那里,
\par }{\Q 因恶人的骄傲呼求,却无人答应。
\par }{\Q \VS{13}虚妄的{\ADD{呼求}}, 神必不垂听;
\par }{\Q 全能者也必不眷顾。
\par }{\Q \VS{14}何况你说,你不得见他;
\par }{\Q 你的案件在他面前,你等候他吧。
\par }{\Q \VS{15}但如今因他未曾发怒降罚,
\par }{\Q 也不甚理会狂傲,
\par }{\Q \VS{16}所以{\PN{约伯}}开口说虚妄的话,
\par }{\Q 多发无知识的言语。

\par }\PoetryChap{36}{\Q \VerseOne{1}{\PN{以利户}}又接着说:
\par }{\Q \VS{2}你再容我片时,我就指示你,
\par }{\Q 因我还有话为 神说。
\par }{\Q \VS{3}我要将所知道的从远处引来,
\par }{\Q 将公义归给造我的主。
\par }{\Q \VS{4}我的言语真不虚谎;
\par }{\Q 有知识全备的与你同在。
\par }{\BB \par }{\Q \VS{5}神有大能,并不藐视人;
\par }{\Q 他的智慧甚广。
\par }{\Q \VS{6}他不保护恶人的性命,
\par }{\Q 却为困苦人伸冤。
\par }{\Q \VS{7}他时常看顾义人,
\par }{\Q 使他们和君王同坐宝座,
\par }{\Q 永远要被高举。
\par }{\Q \VS{8}他们若被锁链捆住,
\par }{\Q 被苦难的绳索缠住,
\par }{\Q \VS{9}他就把他们的作为和过犯指示他们,
\par }{\Q 叫他们知道有骄傲的行动。
\par }{\Q \VS{10}他也开通他们的耳朵得受教训,
\par }{\Q 吩咐他们离开罪孽转回。
\par }{\Q \VS{11}他们若听从事奉他,
\par }{\Q 就必度日亨通,历年福乐;
\par }{\Q \VS{12}若不听从,
\par }{\Q 就要被刀杀灭,无知无识而死。
\par }{\BB \par }{\Q \VS{13}那心中不敬虔的人积蓄怒气;
\par }{\Q  神捆绑他们,他们竟不求救;
\par }{\Q \VS{14}必在青年时死亡,
\par }{\Q 与污秽人一样{\ADD{丧命}}。
\par }{\Q \VS{15}神借着困苦救拔困苦人,
\par }{\Q 趁他们受欺压开通他们的耳朵。
\par }{\Q \VS{16}神也必引你出离患难,
\par }{\Q 进入宽阔不狭窄之地;
\par }{\Q 摆在你席上的必满有肥甘。
\par }{\BB \par }{\Q \VS{17}但你满口有恶人批评的言语;
\par }{\Q 判断和刑罚抓住你。
\par }{\Q \VS{18}不可容忿怒触动你,使你不服责罚;
\par }{\Q 也不可因赎价大就偏行。
\par }{\Q \VS{19}你的呼求\FTNT{}{{\FR 36:19: }或译:资财},或是你一切的势力,
\par }{\Q 果有灵验,叫你不受患难吗?
\par }{\Q \VS{20}不要切慕黑夜,
\par }{\Q 就是众民在本处被除灭的时候。
\par }{\Q \VS{21}你要谨慎,不可重看罪孽,
\par }{\Q 因你选择罪孽过于选择苦难。
\par }{\Q \VS{22}神行事有高大的能力;
\par }{\Q 教训人的有谁像他呢?
\par }{\Q \VS{23}谁派定他的道路?
\par }{\Q 谁能说:你所行的不义?
\par }{\BB \par }{\Q \VS{24}你不可忘记称赞他所行的为大,
\par }{\Q 就是人所歌颂的。
\par }{\Q \VS{25}他所行的,万人都看见;
\par }{\Q 世人也从远处观看。
\par }{\Q \VS{26}神为大,我们不能全知;
\par }{\Q 他的年数不能测度。
\par }{\Q \VS{27}他吸取水点,
\par }{\Q 这水点从云雾中就变成雨;
\par }{\Q \VS{28}云彩将雨落下,沛然降与世人。
\par }{\Q \VS{29}谁能明白云彩如何铺张,
\par }{\Q 和 神行宫的雷声呢?
\par }{\Q \VS{30}他将亮光普照在自己的四围;
\par }{\Q 他又遮覆海底。
\par }{\Q \VS{31}他用这些审判众民,
\par }{\Q 且赐丰富的粮食。
\par }{\Q \VS{32}他以电光遮手,
\par }{\Q 命闪电击中敌人\FTNT{}{{\FR 36:32: }或译:中了靶子}。
\par }{\Q \VS{33}所发的雷声显明他的{\ADD{作为}},
\par }{\Q 又向牲畜指明要起{\ADD{暴风}}。

\par }\PoetryChap{37}{\Q \VerseOne{1}因此我心战兢,
\par }{\Q 从原处移动。
\par }{\Q \VS{2}听啊, 神轰轰的声音
\par }{\Q 是他口中所发的响声。
\par }{\Q \VS{3}他发响声震遍天下,
\par }{\Q 发电光闪到地极。
\par }{\Q \VS{4}随后人听见有雷声轰轰,大发威严,
\par }{\Q 雷电接连不断。
\par }{\Q \VS{5}神发出奇妙的雷声;
\par }{\Q 他行大事,我们不能测透。
\par }{\Q \VS{6}他对雪说:要降在地上;
\par }{\Q 对大雨和暴雨也是这样说。
\par }{\Q \VS{7}他封住各人的手,
\par }{\Q 叫所造的万人都晓得他的作为。
\par }{\Q \VS{8}百兽进入穴中,
\par }{\Q 卧在洞内。
\par }{\Q \VS{9}暴风出于{\ADD{南}}宫;
\par }{\Q 寒冷出于北方。
\par }{\Q \VS{10}神嘘气成冰;
\par }{\Q 宽阔之水也都凝结。
\par }{\Q \VS{11}他使密云盛满水气,
\par }{\Q 布散电光之云;
\par }{\Q \VS{12}这云是借他的指引游行旋转,
\par }{\Q 得以在全地面上行他一切所吩咐的,
\par }{\Q \VS{13}或为责罚,或为{\ADD{润}}地,
\par }{\Q 或为施行慈爱。
\par }{\BB \par }{\Q \VS{14}{\PN{约伯}}啊,你要留心听,
\par }{\Q 要站立思想 神奇妙的作为。
\par }{\Q \VS{15}神如何{\ADD{吩咐}}这些,
\par }{\Q 如何使云中的电光照耀,你知道吗?
\par }{\Q \VS{16}云彩如何浮于空中,
\par }{\Q 那知识全备者奇妙的作为,你知道吗?
\par }{\Q \VS{17}南{\ADD{风}}使地寂静,
\par }{\Q 你的衣服就如火热,你知道吗?
\par }{\Q \VS{18}你岂能与 神同铺穹苍吗?
\par }{\Q 这穹苍坚硬,如同铸成的镜子。
\par }{\Q \VS{19}我们愚昧不能陈说;
\par }{\Q 请你指教我们该对他说什么话。
\par }{\Q \VS{20}人岂可说:我愿与他说话?
\par }{\Q 岂有人自愿灭亡吗?
\par }{\BB \par }{\Q \VS{21}现在{\ADD{有云遮蔽}},人不得见穹苍的光亮;
\par }{\Q 但风吹过,天又发晴。
\par }{\Q \VS{22}金光出于北方,
\par }{\Q 在 神那里有可怕的威严。
\par }{\Q \VS{23}论到全能者,我们不能测度;
\par }{\Q 他大有能力,有公平和大义,
\par }{\Q 必不苦待人。
\par }{\Q \VS{24}所以,人敬畏他;
\par }{\Q 凡自{\ADD{以为}}心中有智慧的人,他都不顾念。

\par }\Chap{38}{\SH 耶和华回答约伯
\par }{\Q \VerseOne{1}那时,耶和华从旋风中回答{\PN{约伯}}说:
\par }{\Q \VS{2}谁用无知的言语使{\ADD{我的}}旨意暗昧不明?
\par }{\Q \VS{3}你要如勇士束腰;
\par }{\Q 我问你,你可以指示我。
\par }{\BB \par }{\Q \VS{4}我立大地根基的时候,你在哪里呢?
\par }{\Q 你若有聪明,只管说吧!
\par }{\Q \VS{5}你若晓得{\ADD{就说}},是谁定地的尺度?
\par }{\Q 是谁把准绳拉在其上?
\par }{\Q \VS{6}地的根基安置在何处?
\par }{\Q 地的角石是谁安放的?
\par }{\Q \VS{7}那时,晨星一同歌唱;
\par }{\Q  神的众子也都欢呼。
\par }{\BB \par }{\Q \VS{8}海水冲出,如出胎胞,
\par }{\Q 那时谁将它关闭呢?
\par }{\Q \VS{9}是我用云彩当海的衣服,
\par }{\Q 用幽暗当包裹它的布,
\par }{\Q \VS{10}为它定界限,
\par }{\Q 又安门和闩,
\par }{\Q \VS{11}说:你只可到这里,不可越过;
\par }{\Q 你狂傲的浪要到此止住。
\par }{\BB \par }{\Q \VS{12}你自生以来,曾命定晨光,
\par }{\Q 使清晨的日光知道本位,
\par }{\Q \VS{13}叫这光普照地的四极,
\par }{\Q 将恶人从其中驱逐出来吗?
\par }{\Q \VS{14}因{\ADD{这光}},地面改变如泥上印印,
\par }{\Q {\ADD{万物}}出现如衣服一样。
\par }{\Q \VS{15}亮光不照恶人;
\par }{\Q 强横的膀臂也必折断。
\par }{\BB \par }{\Q \VS{16}你曾进到海源,
\par }{\Q 或在深渊的隐密处行走吗?
\par }{\Q \VS{17}死亡的门曾向你显露吗?
\par }{\Q 死荫的门你曾见过吗?
\par }{\Q \VS{18}地的广大你能明透吗?
\par }{\Q 你若全知道,只管说吧!
\par }{\BB \par }{\Q \VS{19}光明的居所从何而至?
\par }{\Q 黑暗的本位在于何处?
\par }{\Q \VS{20}你能带到本境,
\par }{\Q 能看明其室之路吗?
\par }{\Q \VS{21}你总知道,
\par }{\Q 因为你早已生在世上,
\par }{\Q 你日子的数目也多。
\par }{\BB \par }{\Q \VS{22}你曾进入雪库,
\par }{\Q 或见过雹仓吗?
\par }{\Q \VS{23}这雪雹乃是我为降灾,
\par }{\Q 并打仗和争战的日子所预备的。
\par }{\Q \VS{24}光亮从何路分开?
\par }{\Q 东风从何路分散遍地?
\par }{\BB \par }{\Q \VS{25}谁为雨水分道?
\par }{\Q 谁为雷电开路?
\par }{\Q \VS{26}使雨降在无人之地、
\par }{\Q 无人居住的旷野?
\par }{\Q \VS{27}使荒废凄凉{\ADD{之地}}得以丰足,
\par }{\Q 青草得以发生?
\par }{\Q \VS{28}雨有父吗?
\par }{\Q 露水珠是谁生的呢?
\par }{\Q \VS{29}冰出于谁的胎?
\par }{\Q 天上的霜是谁生的呢?
\par }{\Q \VS{30}诸水坚硬\FTNT{}{{\FR 38:30: }或译:隐藏}如石头;
\par }{\Q 深渊之面凝结成冰。
\par }{\BB \par }{\Q \VS{31}你能系住昴星的结吗?
\par }{\Q 能解开参星的带吗?
\par }{\Q \VS{32}你能按时领出十二宫吗?
\par }{\Q 能引导北斗和随它的众星\FTNT{}{{\FR 38:32: }星:原文是子}吗?
\par }{\Q \VS{33}你知道天的定例吗?
\par }{\Q 能使地归在天的权下吗?
\par }{\BB \par }{\Q \VS{34}你能向云彩扬起声来,
\par }{\Q 使倾盆的雨遮盖你吗?
\par }{\Q \VS{35}你能发出闪电,叫它行去,
\par }{\Q 使它对你说:我们在这里?
\par }{\Q \VS{36}谁将智慧放在怀中?
\par }{\Q 谁将聪明赐于心内?
\par }{\Q \VS{37-38}谁能用智慧数算云彩呢?
\par }{\Q 尘土聚集成团,土块紧紧结连;
\par }{\Q 那时,谁能倾倒天上的瓶呢?
\par }{\BB \par }{\Q \VS{39-40}母狮子在洞中蹲伏,
\par }{\Q 少壮狮子在隐密处埋伏;
\par }{\Q 你能为它们抓取食物,
\par }{\Q 使它们饱足吗?
\par }{\Q \VS{41}乌鸦之雏因无食物飞来飞去,哀告 神;
\par }{\Q 那时,谁为它预备食物呢?

\par }\PoetryChap{39}{\Q \VerseOne{1}山岩间的野山羊几时生产,你知道吗?
\par }{\Q 母鹿下犊之期,你能察定吗?
\par }{\Q \VS{2}它们怀胎的月数,你能数算吗?
\par }{\Q 它们几时生产,你能晓得吗?
\par }{\Q \VS{3}它们屈身,将子生下,
\par }{\Q 就除掉疼痛。
\par }{\Q \VS{4}这子渐渐肥壮,在荒野长大,
\par }{\Q 去而不回。
\par }{\BB \par }{\Q \VS{5}谁放野驴出去自由?
\par }{\Q 谁解开快驴的绳索?
\par }{\Q \VS{6}我使旷野作它的住处,
\par }{\Q 使咸地当它的居所。
\par }{\Q \VS{7}它嗤笑城内的喧嚷,
\par }{\Q 不听赶牲口的喝声。
\par }{\Q \VS{8}遍山是它的草场;
\par }{\Q 它寻找各样青绿之物。
\par }{\BB \par }{\Q \VS{9}野牛岂肯服事你?
\par }{\Q 岂肯住在你的槽旁?
\par }{\Q \VS{10}你岂能用套绳将野牛笼在犁沟之间?
\par }{\Q 它岂肯随你耙山谷之地?
\par }{\Q \VS{11}岂可因它的力大就倚靠它?
\par }{\Q 岂可把你的工交给它做吗?
\par }{\Q \VS{12}岂可信靠它把你的粮食运到家,
\par }{\Q 又收聚你禾场上的谷吗?
\par }{\BB \par }{\Q \VS{13}鸵鸟的翅膀欢然搧展,
\par }{\Q 岂是显慈爱的翎毛和羽毛吗?
\par }{\Q \VS{14}因它把蛋留在地上,
\par }{\Q 在尘土中使得温暖;
\par }{\Q \VS{15}却想不到被脚踹碎,
\par }{\Q 或被野兽践踏。
\par }{\Q \VS{16}它忍心待雏,似乎不是自己的;
\par }{\Q 虽然徒受劳苦,也不{\ADD{为雏}}惧怕;
\par }{\Q \VS{17}因为 神使它没有智慧,
\par }{\Q 也未将悟性赐给它。
\par }{\Q \VS{18}它几时挺身展开翅膀,
\par }{\Q 就嗤笑马和骑马的人。
\par }{\BB \par }{\Q \VS{19}马的大力是你所赐的吗?
\par }{\Q 它颈项上挓挲的鬃是你给它披上的吗?
\par }{\Q \VS{20}是你叫它跳跃像蝗虫吗?
\par }{\Q 它喷气之威使人惊惶。
\par }{\Q \VS{21}它在谷中刨地,自喜其力;
\par }{\Q 它出去迎接佩带兵器的人。
\par }{\Q \VS{22}它嗤笑可怕的事并不惊惶,
\par }{\Q 也不因刀剑退回。
\par }{\Q \VS{23}箭袋和发亮的枪,
\par }{\Q 并短枪在它身上铮铮有声。
\par }{\Q \VS{24}它发猛烈的怒气将地吞下;
\par }{\Q 一听角声就不耐站立。
\par }{\Q \VS{25}角每发声,它说呵哈;
\par }{\Q 它从远处闻着战气,
\par }{\Q 又听见军长大发雷声和{\ADD{兵丁}}呐喊。
\par }{\BB \par }{\Q \VS{26}鹰雀飞翔,展开翅膀一直向南,
\par }{\Q 岂是借你的智慧吗?
\par }{\Q \VS{27}大鹰上腾在高处搭窝,
\par }{\Q 岂是听你的吩咐吗?
\par }{\Q \VS{28}它住在山岩,
\par }{\Q 以山峰和坚固之所为家,
\par }{\Q \VS{29}从那里窥看食物,
\par }{\Q 眼睛远远观望。
\par }{\Q \VS{30}它的雏也咂血;
\par }{\Q 被杀的人在哪里,它也在那里。

\par }\PoetryChap{40}{\Q \VerseOne{1}耶和华又对{\PN{约伯}}说:
\par }{\Q \VS{2}强辩的岂可与全能者争论吗?
\par }{\Q 与 神辩驳的可以回答这些吧!
\par }{\BB \par }{\Q \VS{3}于是,{\PN{约伯}}回答耶和华说:
\par }{\Q \VS{4}我是卑贱的!我用什么回答你呢?
\par }{\Q 只好用手捂口。
\par }{\Q \VS{5}我说了一次,再不回答;
\par }{\Q 说了两次,就不再说。
\par }{\BB \par }{\Q \VS{6}于是,耶和华从旋风中回答{\PN{约伯}}说:
\par }{\Q \VS{7}你要如勇士束腰;
\par }{\Q 我问你,你可以指示我。
\par }{\Q \VS{8}你岂可废弃我所拟定的?
\par }{\Q 岂可定我有罪,好显自己为义吗?
\par }{\Q \VS{9}你有 神那样的膀臂吗?
\par }{\Q 你能像他发雷声吗?
\par }{\BB \par }{\Q \VS{10}你要以荣耀庄严为妆饰,
\par }{\Q 以尊荣威严为衣服;
\par }{\Q \VS{11}要发出你满溢的怒气,
\par }{\Q 见一切骄傲的人,使他降卑;
\par }{\Q \VS{12}见一切骄傲的人,将他制伏,
\par }{\Q 把恶人践踏在本处;
\par }{\Q \VS{13}将他们一同隐藏在尘土中,
\par }{\Q 把他们的脸蒙蔽在隐密处;
\par }{\Q \VS{14}我就认你右手能以救自己。
\par }{\BB \par }{\Q \VS{15}你且观看河马;
\par }{\Q 我造你也造它。
\par }{\Q 它吃草与牛一样;
\par }{\Q \VS{16}它的气力在腰间,
\par }{\Q 能力在肚腹的筋上。
\par }{\Q \VS{17}它摇动尾巴如香柏树;
\par }{\Q 它大腿的筋互相联络。
\par }{\Q \VS{18}它的骨头{\ADD{好像}}铜管;
\par }{\Q 它的肢体仿佛铁棍。
\par }{\BB \par }{\Q \VS{19}它在 神所造的物中为首;
\par }{\Q 创造它的给它刀剑。
\par }{\Q \VS{20}诸山给它出食物,
\par }{\Q 也是百兽游玩之处。
\par }{\Q \VS{21}它伏在莲叶之下,
\par }{\Q 卧在芦苇隐密处和水洼子里。
\par }{\Q \VS{22}莲叶的阴凉遮蔽它;
\par }{\Q 溪旁的柳树环绕它。
\par }{\Q \VS{23}河水泛滥,它不发战;
\par }{\Q 就是{\PN{约旦河}}的水涨到它口边,也是安然。
\par }{\Q \VS{24}在它防备的时候,谁能捉拿它?
\par }{\Q 谁能牢笼它穿它的鼻子呢?

\par }\PoetryChap{41}{\Q \VerseOne{1}你能用鱼钩钓上鳄鱼吗?
\par }{\Q 能用绳子压下它的舌头吗?
\par }{\Q \VS{2}你能用绳索穿它的鼻子吗?
\par }{\Q 能用钩穿它的腮骨吗?
\par }{\Q \VS{3}它岂向你连连恳求,
\par }{\Q 说柔和的话吗?
\par }{\Q \VS{4}岂肯与你立约,
\par }{\Q 使你拿它永远作奴仆吗?
\par }{\Q \VS{5}你岂可拿它当雀鸟玩耍吗?
\par }{\Q 岂可为你的幼女将它拴住吗?
\par }{\Q \VS{6}搭伙的{\ADD{渔夫}}岂可拿它当货物吗?
\par }{\Q 能把它分给商人吗?
\par }{\Q \VS{7}你能用倒钩枪扎满它的皮,
\par }{\Q 能用鱼叉叉满它的头吗?
\par }{\Q \VS{8}你按手在它身上,想与它争战,
\par }{\Q 就不再这样行吧!
\par }{\Q \VS{9}人指望捉拿它是徒然的;
\par }{\Q 一见它,岂不丧胆吗?
\par }{\Q \VS{10}没有那么凶猛的人敢惹它。
\par }{\Q 这样,谁能在我面前站立得住呢?
\par }{\Q \VS{11}谁先给我什么,使我偿还呢?
\par }{\Q 天下{\ADD{万物}}都是我的。
\par }{\BB \par }{\Q \VS{12}论到鳄鱼的肢体和其大力,并美好的骨骼,
\par }{\Q 我不能缄默不言。
\par }{\Q \VS{13}谁能剥它的外衣?
\par }{\Q 谁能进它上下牙骨之间呢?
\par }{\Q \VS{14}谁能开它的腮颊?
\par }{\Q 它牙齿四围是可畏的。
\par }{\Q \VS{15}它以坚固的鳞甲为可夸,
\par }{\Q 紧紧合闭,封得严密。
\par }{\Q \VS{16}这鳞甲一一相连,
\par }{\Q 甚至气不得透入其间,
\par }{\Q \VS{17}都是互相联络、胶结,
\par }{\Q 不能分离。
\par }{\Q \VS{18}它打喷嚏就发出光来;
\par }{\Q 它眼睛好像早晨的光线\FTNT{}{{\FR 41:18: }原文是眼皮}。
\par }{\Q \VS{19}从它口中发出烧着的火把,
\par }{\Q 与飞迸的火星;
\par }{\Q \VS{20}从它鼻孔冒出烟来,
\par }{\Q 如烧开的锅和{\ADD{点着}}的芦苇。
\par }{\Q \VS{21}它的气点着煤炭,
\par }{\Q 有火焰从它口中发出。
\par }{\Q \VS{22}它颈项中存着劲力;
\par }{\Q 在它面前的都恐吓蹦跳。
\par }{\Q \VS{23}它的肉块互相联络,
\par }{\Q 紧贴其身,不能摇动。
\par }{\Q \VS{24}它的心结实如石头,
\par }{\Q 如下磨石那样结实。
\par }{\Q \VS{25}它一起来,勇士都惊恐,
\par }{\Q 心里慌乱,便都昏迷。
\par }{\Q \VS{26}人若用刀,用枪,用标枪,
\par }{\Q 用尖枪扎它,都是无用。
\par }{\Q \VS{27}它以铁为干草,
\par }{\Q 以铜为烂木。
\par }{\Q \VS{28}箭不能{\ADD{恐吓它}}使它逃避;
\par }{\Q 弹石在它看为碎秸。
\par }{\Q \VS{29}棍棒算为禾秸;
\par }{\Q 它嗤笑短枪飕的响声。
\par }{\Q \VS{30}它肚腹下{\ADD{如}}尖瓦片;
\par }{\Q 它如钉耙经过淤泥。
\par }{\Q \VS{31}它使深渊开滚如锅,
\par }{\Q 使洋海如锅中的膏油。
\par }{\Q \VS{32}它行的路随后发光,
\par }{\Q 令人想深渊如同白发。
\par }{\Q \VS{33}在地上没有像它造的那样,
\par }{\Q 无所惧怕。
\par }{\Q \VS{34}凡高大的,它无不藐视;
\par }{\Q 它在骄傲的水族上作王。

\par }\PoetryChap{42}{\Q \VerseOne{1}{\PN{约伯}}回答耶和华说:
\par }{\Q \VS{2}我知道,你万事都能做;
\par }{\Q 你的旨意不能拦阻。
\par }{\Q \VS{3}谁用无知的言语使{\ADD{你的}}旨意隐藏呢?
\par }{\Q 我所说的是我不明白的;
\par }{\Q 这些事太奇妙,是我不知道的。
\par }{\Q \VS{4}求你听我,我要说话;
\par }{\Q 我问你,求你指示我。
\par }{\Q \VS{5}我从前风闻有你,
\par }{\Q 现在亲眼看见你。
\par }{\Q \VS{6}因此我厌恶自己\FTNT{}{{\FR 42:6: }或译:我的言语},
\par }{\Q 在尘土和炉灰中懊悔。
\par }{\SH 结尾
\par }{\PP \VS{7}耶和华对{\PN{约伯}}说话以后,就对{\PN{提幔}}人{\PN{以利法}}说:「我的怒气向你和你两个朋友发作,因为你们议论我不如我的仆人{\PN{约伯}}说的是。
\VS{8}现在你们要取七只公牛,七只公羊,到我仆人{\PN{约伯}}那里去,为自己献上燔祭,我的仆人{\PN{约伯}}就为你们祈祷。我因悦纳他,就不按你们的愚妄办你们。你们议论我,不如我的仆人{\PN{约伯}}说的是。」
\VS{9}于是{\PN{提幔}}人{\PN{以利法}}、{\PN{书亚}}人{\PN{比勒达}}、{\PN{拿玛}}人{\PN{琐法}}照着耶和华所吩咐的去行;耶和华就悦纳{\PN{约伯}}。
\par }{\PP \VS{10}{\PN{约伯}}为他的朋友祈祷。耶和华就使{\PN{约伯}}从苦境\FTNT{}{{\FR 42:10: }原文是掳掠}转回,并且耶和华赐给他的比他从前所有的加倍。
\VS{11}{\PN{约伯}}的弟兄、姊妹,和以先所认识的人都来见他,在他家里一同吃饭;又论到耶和华所降与他的一切灾祸,都为他悲伤安慰他。每人也送他一块{\ADD{银子}}和一个金环。
\VS{12}这样,耶和华后来赐福给{\PN{约伯}}比先前更多。他有一万四千羊,六千骆驼,一千对牛,一千母驴。
\VS{13}他也有七个儿子,三个女儿。
\VS{14}他给长女起名叫{\PN{耶米玛}},次女叫{\PN{基洗亚}},三女叫{\PN{基连·哈朴}}。
\VS{15}在那全地的妇女中找不着像{\PN{约伯}}的女儿那样美貌。她们的父亲使她们在弟兄中得产业。
\VS{16}此后,{\PN{约伯}}又活了一百四十年,得见他的儿孙,直到四代。
\VS{17}这样,{\PN{约伯}}年纪老迈,日子满足而死。
\par }