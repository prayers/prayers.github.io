\NormalFont\ShortTitle{以赛亚书}
{\MT 以赛亚书

\par }\ChapOne{1}{\PP \VerseOne{1}当{\PN{乌西雅}}、{\PN{约坦}}、{\PN{亚哈斯}}、{\PN{希西家}}作{\PN{犹大}}王的时候,{\PN{亚摩斯}}的儿子{\PN{以赛亚}}得默示,论到{\PN{犹大}}和{\PN{耶路撒冷}}。
\par }{\SH  神责备以色列
\par }{\Q \VS{2}天哪,要听!地啊,侧耳而听!
\par }{\Q 因为耶和华说:
\par }{\Q 我养育儿女,将他们养大,
\par }{\Q 他们竟悖逆我。
\par }{\Q \VS{3}牛认识主人,
\par }{\Q 驴认识主人的槽,
\par }{\Q {\PN{以色列}}却不认识;
\par }{\Q 我的民却不留意。
\par }{\BB \par }{\Q \VS{4}嗐!犯罪的国民,
\par }{\Q 担着罪孽的百姓;
\par }{\Q 行恶的种类,
\par }{\Q 败坏的儿女!
\par }{\Q 他们离弃耶和华,
\par }{\Q 藐视{\PN{以色列}}的圣者,
\par }{\Q 与他生疏,往后退步。
\par }{\BB \par }{\Q \VS{5}你们为什么屡次悖逆,
\par }{\Q 还要受责打吗?
\par }{\Q 你们已经满头疼痛,
\par }{\Q 全心发昏。
\par }{\Q \VS{6}从脚掌到头顶,
\par }{\Q 没有一处完全的,
\par }{\Q 尽是伤口、青肿,与新打的伤痕,
\par }{\Q 都没有收口,没有缠裹,
\par }{\Q 也没有用膏滋润。
\par }{\BB \par }{\Q \VS{7}你们的地土已经荒凉;
\par }{\Q 你们的城邑被火焚毁。
\par }{\Q 你们的田地在你们眼前为外邦人所侵吞,
\par }{\Q 既被外邦人倾覆就成为荒凉。
\par }{\Q \VS{8}仅存{\PN{锡安}}城\FTNT{}{{\FR 1:8: }原文是女子},
\par }{\Q 好像葡萄园的草棚,
\par }{\Q 瓜田的茅屋,
\par }{\Q 被围困的城邑。
\par }{\Q \VS{9}若不是万军之耶和华给我们稍留余种,
\par }{\Q 我们早已像{\PN{所多玛}}、{\PN{蛾摩拉}}的样子了。
\par }{\BB \par }{\Q \VS{10}你们这{\PN{所多玛}}的官长啊,
\par }{\Q 要听耶和华的话!
\par }{\Q 你们这{\PN{蛾摩拉}}的百姓啊,
\par }{\Q 要侧耳听我们 神的训诲!
\par }{\Q \VS{11}耶和华说:
\par }{\Q 你们所献的许多祭物与我何益呢?
\par }{\Q 公绵羊的燔祭和肥畜的脂油,
\par }{\Q 我已经够了;
\par }{\Q 公牛的血,羊羔的血,公山羊的血,
\par }{\Q 我都不喜悦。
\par }{\BB \par }{\Q \VS{12}你们来朝见我,
\par }{\Q 谁向你们讨这些,
\par }{\Q 使你们践踏我的院宇呢?
\par }{\Q \VS{13}你们不要再献虚浮的供物。
\par }{\Q 香品是我所憎恶的;
\par }{\Q 月朔和安息日,并宣召的大会,
\par }{\Q 也是我所憎恶的;
\par }{\Q 作罪孽,又守严肃会,
\par }{\Q 我也不能容忍。
\par }{\Q \VS{14}你们的月朔和节期,我心里恨恶,
\par }{\Q 我都以为麻烦;
\par }{\Q 我担当,便不耐烦。
\par }{\Q \VS{15}你们举手祷告,我必遮眼不看;
\par }{\Q 就是你们多多地祈祷,我也不听。
\par }{\Q 你们的手都满了{\ADD{杀人的}}血。
\par }{\Q \VS{16}你们要洗濯、自洁,
\par }{\Q 从我眼前除掉你们的恶行,
\par }{\Q 要止住作恶,
\par }{\Q \VS{17}学习行善,
\par }{\Q 寻求公平,
\par }{\Q 解救受欺压的;
\par }{\Q 给孤儿伸冤,
\par }{\Q 为寡妇辨屈。
\par }{\BB \par }{\Q \VS{18}耶和华说:
\par }{\Q 你们来,我们彼此辩论。
\par }{\Q 你们的罪虽像朱红,必变成雪白;
\par }{\Q 虽红如丹颜,必白如羊毛。
\par }{\Q \VS{19}你们若甘心听从,
\par }{\Q 必吃地上的美物,
\par }{\Q \VS{20}若不听从,反倒悖逆,
\par }{\Q 必被刀剑吞灭。
\par }{\Q 这是耶和华亲口说的。
\par }{\SH 罪大恶极的城
\par }{\Q \VS{21}可叹,忠信的城变为妓女!
\par }{\Q 从前充满了公平,
\par }{\Q 公义居在其中,
\par }{\Q 现今却有凶手居住。
\par }{\Q \VS{22}你的银子变为渣滓;
\par }{\Q 你的酒用水搀对。
\par }{\Q \VS{23}你的官长居心悖逆,
\par }{\Q 与盗贼作伴,
\par }{\Q 各都喜爱贿赂,
\par }{\Q 追求赃私。
\par }{\Q 他们不为孤儿伸冤;
\par }{\Q 寡妇的案件也不得呈到他们面前。
\par }{\BB \par }{\Q \VS{24}因此,主—万军之耶和华、
\par }{\Q {\PN{以色列}}的大能者说:
\par }{\Q 哎!我要向我的对头雪恨,
\par }{\Q 向我的敌人报仇。
\par }{\Q \VS{25}我必反手加在你身上,
\par }{\Q 炼尽你的渣滓,
\par }{\Q 除净你的杂质。
\par }{\Q \VS{26}我也必复还你的审判官,像起初一样,
\par }{\Q 复还你的谋士,像起先一般。
\par }{\Q 然后,你必称为公义之城,
\par }{\Q 忠信之邑。
\par }{\BB \par }{\Q \VS{27}{\PN{锡安}}必因公平得蒙救赎;
\par }{\Q 其中归正的人必因公义得蒙救赎。
\par }{\Q \VS{28}但悖逆的和犯罪的必一同败亡;
\par }{\Q 离弃耶和华的必致消灭。
\par }{\Q \VS{29}那等人必因你们所喜爱的橡树抱愧;
\par }{\Q 你们必因所选择的园子蒙羞。
\par }{\Q \VS{30}因为,你们必如叶子枯干的橡树,
\par }{\Q 好像无水{\ADD{浇灌}}的园子。
\par }{\Q \VS{31}有权势的必如麻瓤;
\par }{\Q 他的工作好像火星,
\par }{\Q 都要一同焚毁,无人扑灭。

\par }\Chap{2}{\SH 永恒的和平
\par }{\R (弥4·1—3)
\par }{\PP \VerseOne{1}{\PN{亚摩斯}}的儿子{\PN{以赛亚}}得默示,论到{\PN{犹大}}和{\PN{耶路撒冷}}。
\par }{\Q \VS{2}末后的日子,耶和华殿的山必坚立,
\par }{\Q 超乎诸山,高举过于万岭;
\par }{\Q 万民都要流归这山。
\par }{\Q \VS{3}必有许多国的民前往,说:
\par }{\Q 来吧,我们登耶和华的山,
\par }{\Q 奔{\PN{雅各}} 神的殿。
\par }{\Q 主必将他的道教训我们;
\par }{\Q 我们也要行他的路。
\par }{\Q 因为训诲必出于{\PN{锡安}};
\par }{\Q 耶和华的言语必出于{\PN{耶路撒冷}}。
\par }{\Q \VS{4}他必在列国中施行审判,
\par }{\Q 为许多国民断定是非。
\par }{\Q 他们要将刀打成犁头,
\par }{\Q 把枪打成镰刀。
\par }{\Q 这国不举刀攻击那国;
\par }{\Q 他们也不再学习战事。
\par }{\Q \VS{5}{\PN{雅各}}家啊,来吧!
\par }{\Q 我们在耶和华的光明中行走。
\par }{\SH 骄者必败
\par }{\Q \VS{6}{\ADD{耶和华}},你离弃了你百姓{\PN{雅各}}家,
\par }{\Q 是因他们充满了东方的{\ADD{风俗}},
\par }{\Q 作观兆的,像{\PN{非利士}}人一样,
\par }{\Q 并与外邦人击掌。
\par }{\Q \VS{7}他们的国满了金银,
\par }{\Q 财宝也无穷;
\par }{\Q 他们的地满了马匹,
\par }{\Q 车辆也无数。
\par }{\Q \VS{8}他们的地满了偶像;
\par }{\Q 他们跪拜自己手所造的,
\par }{\Q 就是自己指头所做的。
\par }{\Q \VS{9}卑贱人屈膝;
\par }{\Q 尊贵人下跪;
\par }{\Q 所以不可饶恕他们。
\par }{\Q \VS{10}你当进入岩穴,藏在土中,
\par }{\Q 躲避耶和华的惊吓和他威严的荣光。
\par }{\Q \VS{11}到那日,眼目高傲的必降为卑;
\par }{\Q 性情狂傲的都必屈膝;
\par }{\Q 惟独耶和华被尊崇。
\par }{\BB \par }{\Q \VS{12}必有万军耶和华{\ADD{降罚}}的一个日子,
\par }{\Q 要临到骄傲狂妄的;
\par }{\Q 一切自高的都必降为卑;
\par }{\Q \VS{13}又临到{\PN{黎巴嫩}}高大的香柏树和{\PN{巴珊}}的橡树;
\par }{\Q \VS{14}又临到一切高山的峻岭;
\par }{\Q \VS{15}又临到高台和坚固城墙;
\par }{\Q \VS{16}又临到{\PN{他施}}的船只并一切可爱的美物。
\par }{\Q \VS{17}骄傲的必屈膝;
\par }{\Q 狂妄的必降卑。
\par }{\Q 在那日,惟独耶和华被尊崇;
\par }{\Q \VS{18}偶像必全然废弃。
\par }{\Q \VS{19}耶和华兴起,使地大震动的时候,
\par }{\Q 人就进入石洞,进入土穴,
\par }{\Q 躲避耶和华的惊吓和他威严的荣光。
\par }{\Q \VS{20}到那日,人必将为拜而造的金偶像、银偶像
\par }{\Q 抛给田鼠和蝙蝠。
\par }{\Q \VS{21}到耶和华兴起,使地大震动的时候,
\par }{\Q 人好进入磐石洞中和岩石穴里,
\par }{\Q 躲避耶和华的惊吓和他威严的荣光。
\par }{\Q \VS{22}你们休要{\ADD{倚靠}}世人。
\par }{\Q 他鼻孔里不过有气息;
\par }{\Q 他在一切事上可算什么呢?

\par }\Chap{3}{\SH 耶路撒冷的混乱
\par }{\Q \VerseOne{1}主—万军之耶和华从{\PN{耶路撒冷}}和{\PN{犹大}},
\par }{\Q 除掉众人所倚靠的,所仗赖的,
\par }{\Q 就是所倚靠的粮,所仗赖的水;
\par }{\Q \VS{2}除掉勇士和战士,
\par }{\Q 审判官和先知,
\par }{\Q 占卜的和长老,
\par }{\Q \VS{3}五十夫长和尊贵人,
\par }{\Q 谋士和有巧艺的,
\par }{\Q 以及妙行法术的。
\par }{\Q \VS{4}{\ADD{主说}}:我必使孩童作他们的首领,
\par }{\Q 使婴孩辖管他们。
\par }{\Q \VS{5}百姓要彼此欺压;
\par }{\Q 各人受邻舍的欺压。
\par }{\Q 少年人必侮慢老年人;
\par }{\Q 卑贱人必侮慢尊贵人。
\par }{\BB \par }{\Q \VS{6}人在父家拉住弟兄,{\ADD{说}}:
\par }{\Q 你有衣服,可以作我们的官长。
\par }{\Q 这败落的事归在你手下吧!
\par }{\Q \VS{7}那时,他必扬声说:
\par }{\Q 我不作医治你们的人;
\par }{\Q 因我家中没有粮食,也没有衣服,
\par }{\Q 你们不可立我作百姓的官长。
\par }{\Q \VS{8}{\PN{耶路撒冷}}败落,
\par }{\Q {\PN{犹大}}倾倒;
\par }{\Q 因为他们的舌头和行为与耶和华反对,
\par }{\Q 惹了他荣光的眼目。
\par }{\BB \par }{\Q \VS{9}他们的面色证明自己的不正;
\par }{\Q 他们述说自己的罪恶,并不隐瞒,
\par }{\Q 好像{\PN{所多玛}}一样。
\par }{\Q 他们有祸了!因为作恶自害。
\par }{\Q \VS{10}你们要论义人说:他必享福乐,
\par }{\Q 因为要吃自己行为所结的果子。
\par }{\Q \VS{11}恶人有祸了!他必遭灾难!
\par }{\Q 因为要照自己手所行的受报应。
\par }{\Q \VS{12}至于我的百姓,
\par }{\Q 孩童欺压他们,
\par }{\Q 妇女辖管他们。
\par }{\Q 我的百姓啊,引导你的使你走错,
\par }{\Q 并毁坏你所行的道路。
\par }{\SH 耶和华审判他的人民
\par }{\Q \VS{13}耶和华起来辩论,
\par }{\Q 站着审判众民。
\par }{\Q \VS{14}耶和华必审问他民中的长老和首领,{\ADD{说}}:
\par }{\Q 吃尽葡萄园果子的就是你们;
\par }{\Q 向贫穷人所夺的都在你们家中。
\par }{\Q \VS{15}主—万军之耶和华说:
\par }{\Q 你们为何压制我的百姓,
\par }{\Q 搓磨贫穷人的脸呢?
\par }{\SH 对耶路撒冷女子的警告
\par }{\Q \VS{16}耶和华又说:
\par }{\Q 因为{\PN{锡安}}的女子狂傲,
\par }{\Q 行走挺项,卖弄眼目,
\par }{\Q 俏步徐行,脚下玎珰,
\par }{\Q \VS{17}所以,主必使{\PN{锡安}}的女子头长秃疮;
\par }{\Q 耶和华又使她们赤露下体。
\par }{\PP \VS{18}到那日,主必除掉她们华美的脚钏、发网、月牙圈、
\VS{19}耳环、手镯、蒙脸的帕子、
\VS{20}华冠、足链、华带、香盒、符囊、
\VS{21}戒指、鼻环、
\VS{22}吉服、外套、云肩、荷包、
\VS{23}手镜、细麻衣、裹头巾、蒙身的帕子。
\par }{\Q \VS{24}必有臭烂代替馨香,
\par }{\Q 绳子代替腰带,
\par }{\Q 光秃代替美发,
\par }{\Q 麻衣系腰代替华服,
\par }{\Q 烙伤代替美容。
\par }{\Q \VS{25}你的男丁必倒在刀下;
\par }{\Q 你的勇士必死在阵上。
\par }{\Q \VS{26}{\PN{锡安}}\FTNT{}{{\FR 3:26: }原文是她}的城门必悲伤、哀号;
\par }{\Q 她必荒凉坐在地上。

\par }\Chap{4}{\PP \VerseOne{1}在那日,七个女人必拉住一个男人,说:「我们吃自己的食物,穿自己的衣服,但求你许我们归你名下;求你除掉我们的羞耻。」
\par }{\SH 耶路撒冷的重建
\par }{\PP \VS{2}到那日,耶和华发生的苗必华美尊荣,地的出产必为{\PN{以色列}}逃脱的人显为荣华茂盛。
\VS{3-4}主以公义的灵和焚烧的灵,将{\PN{锡安}}女子的污秽洗去,又将{\PN{耶路撒冷}}中{\ADD{杀人}}的血除净。那时,剩在{\PN{锡安}}、留在{\PN{耶路撒冷}}的,就是一切住{\PN{耶路撒冷}}、在生命册上记名的,必称为圣。
\VS{5}耶和华也必在{\PN{锡安}}全山,并各会众以上,使白日有烟云,黑夜有火焰的光。因为在全荣耀之上必有遮蔽。
\VS{6}必有亭子,白日可以得荫避暑,也可以作为藏身之处,躲避狂风暴雨。

\par }\Chap{5}{\SH 葡萄园之歌
\par }{\Q \VerseOne{1}我要为我所亲爱的唱歌,
\par }{\Q 是我所爱者的歌,论他葡萄园的事:
\par }{\Q 我所亲爱的有葡萄园在肥美的山冈上。
\par }{\Q \VS{2}他刨挖园子,捡去石头,
\par }{\Q 栽种上等的葡萄树,
\par }{\Q 在园中盖了一座楼,
\par }{\Q 又凿出压酒池;
\par }{\Q 指望结{\ADD{好}}葡萄,
\par }{\Q 反倒结了野葡萄。
\par }{\BB \par }{\Q \VS{3}{\PN{耶路撒冷}}的居民和{\PN{犹大}}人哪,
\par }{\Q 请你们现今在我与我的葡萄园中,断定是非。
\par }{\Q \VS{4}我为我葡萄园所做之外,
\par }{\Q 还有什么可做的呢?
\par }{\Q 我指望结{\ADD{好}}葡萄,
\par }{\Q 怎么倒结了野葡萄呢?
\par }{\BB \par }{\Q \VS{5}现在我告诉你们,
\par }{\Q 我要向我葡萄园怎样行:
\par }{\Q 我必撤去篱笆,使它被吞灭,
\par }{\Q 拆毁墙垣,使它被践踏。
\par }{\Q \VS{6}我必使它荒废,不再修理,
\par }{\Q 不再锄刨,荆棘蒺藜倒要生长。
\par }{\Q 我也必命云不降雨在其上。
\par }{\BB \par }{\Q \VS{7}万军之耶和华的葡萄园就是{\PN{以色列}}家;
\par }{\Q 他所喜爱的树就是{\PN{犹大}}人。
\par }{\Q 他指望的是公平,
\par }{\Q 谁知倒有暴虐\FTNT{}{{\FR 5:7: }或译:倒流人血};
\par }{\Q 指望的是公义,
\par }{\Q 谁知倒有冤声。
\par }{\SH 人的恶行
\par }{\Q \VS{8}祸哉!那些以房接房,
\par }{\Q 以地连地,
\par }{\Q 以致不留余地的,
\par }{\Q 只顾自己独居境内。
\par }{\Q \VS{9}我耳闻万军之耶和华{\ADD{说}}:
\par }{\Q 必有许多又大又美的房屋
\par }{\Q 成为荒凉,无人居住。
\par }{\Q \VS{10}三十亩葡萄园只出一罢特{\ADD{酒}};
\par }{\Q 一贺梅珥谷种只结一伊法{\ADD{粮食}}。
\par }{\BB \par }{\Q \VS{11}祸哉!那些清早起来追求浓酒,
\par }{\Q 留连到夜深,甚至因酒发烧的人。
\par }{\Q \VS{12}他们在筵席上
\par }{\Q 弹琴,鼓瑟,击鼓,吹笛,饮酒,
\par }{\Q 却不顾念耶和华的作为,
\par }{\Q 也不留心他手所做的。
\par }{\BB \par }{\Q \VS{13}所以,我的百姓因无知就被掳去;
\par }{\Q 他们的尊贵人甚是饥饿,
\par }{\Q 群众极其干渴。
\par }{\Q \VS{14}故此,阴间扩张其欲,
\par }{\Q 开了无限量的口;
\par }{\Q 他们的荣耀、群众、繁华,
\par }{\Q 并快乐的人都落{\ADD{在其中}}。
\par }{\Q \VS{15}卑贱人被压服;
\par }{\Q 尊贵人降为卑;
\par }{\Q 眼目高傲的人也降为卑。
\par }{\Q \VS{16}惟有万军之耶和华因公平而崇高;
\par }{\Q 圣者 神因公义显为圣。
\par }{\Q \VS{17}那时,羊羔必来吃草,如同在自己的草场;
\par }{\Q 丰肥人的荒场被游行的人吃尽。
\par }{\BB \par }{\Q \VS{18}祸哉!那些以虚假之细绳牵罪孽的人!
\par }{\Q 他们又像以套绳拉罪恶,
\par }{\Q \VS{19}说:任他急速行,赶快成就他的作为,
\par }{\Q 使我们看看;
\par }{\Q 任{\PN{以色列}}圣者所谋划的临近成就,
\par }{\Q 使我们知道。
\par }{\Q \VS{20}祸哉!那些称恶为善,称善为恶,
\par }{\Q 以暗为光,以光为暗,
\par }{\Q 以苦为甜,以甜为苦的人。
\par }{\Q \VS{21}祸哉!那些自以为有智慧,
\par }{\Q 自看为通达的人。
\par }{\Q \VS{22}祸哉!那些勇于饮酒,
\par }{\Q 以能力调浓酒的人。
\par }{\Q \VS{23}他们因受贿赂,就称恶人为义,
\par }{\Q 将义人的义夺去。
\par }{\BB \par }{\Q \VS{24}火苗怎样吞灭碎秸,
\par }{\Q 干草怎样落在火焰之中,
\par }{\Q 照样,他们的根必像朽{\ADD{物}},
\par }{\Q 他们的花必像灰尘飞腾;
\par }{\Q 因为他们厌弃万军之耶和华的训诲,
\par }{\Q 藐视{\PN{以色列}}圣者的言语。
\par }{\Q \VS{25}所以,耶和华的怒气向他的百姓发作。
\par }{\Q 他的手伸出攻击他们,山岭就震动;
\par }{\Q 他们的尸首在街市上好像粪土。
\par }{\Q 虽然如此,他的怒气还未转消;
\par }{\Q 他的手仍伸不缩。
\par }{\Q \VS{26}他必竖立大旗,招远方的国民,
\par }{\Q 发嘶声叫他们从地极而来;
\par }{\Q 看哪,他们必急速奔来。
\par }{\Q \VS{27}其中没有疲倦的,绊跌的;
\par }{\Q 没有打盹的,睡觉的;
\par }{\Q 腰带并不放松,
\par }{\Q 鞋带也不折断。
\par }{\Q \VS{28}他们的箭快利,
\par }{\Q 弓也上了弦。
\par }{\Q 马蹄算如坚石,
\par }{\Q 车轮好像旋风。
\par }{\Q \VS{29}他们要吼叫,像母狮子,
\par }{\Q 咆哮,像少壮狮子;
\par }{\Q 他们要咆哮抓食,
\par }{\Q 坦然叼去,无人救回。
\par }{\Q \VS{30}那日,他们要向{\PN{以色列}}人吼叫,
\par }{\Q 像海浪匉訇;
\par }{\Q 人若望地,只见黑暗艰难,
\par }{\Q 光明在云中变为昏暗。

\par }\Chap{6}{\SH 主呼召以赛亚作先知
\par }{\PP \VerseOne{1}当{\PN{乌西雅}}王崩的那年,我见主坐在高高的宝座上。他的衣裳垂下,遮满圣殿。
\VS{2}其上有撒拉弗侍立,各有六个翅膀:用两个翅膀遮脸,两个翅膀遮脚,两个翅膀飞翔;
\VS{3}彼此呼喊说:
\par }{\Q 圣哉!圣哉!圣哉!万军之耶和华;
\par }{\Q 他的荣光充满全地!
\par }{\PP \VS{4}因呼喊者的声音,门槛的根基震动,殿充满了烟云。
\VS{5}那时我说:「祸哉!我灭亡了!因为我是嘴唇不洁的人,又住在嘴唇不洁的民中,又因我眼见大君王—万军之耶和华。」
\par }{\PP \VS{6}有一撒拉弗飞到我跟前,手里拿着红炭,是用火剪从坛上取下来的,
\VS{7}将炭沾我的口,说:「看哪,这炭沾了你的嘴,你的罪孽便除掉,你的罪恶就赦免了。」
\VS{8}我又听见主的声音说:「我可以差遣谁呢?谁肯为我们去呢?」我说:「我在这里,请差遣我!」
\VS{9}他说:「你去告诉这百姓说:
\par }{\Q 你们听是要听见,却不明白;
\par }{\Q 看是要看见,却不晓得。
\par }{\Q \VS{10}要使这百姓心蒙脂油,
\par }{\Q 耳朵发沉,
\par }{\Q 眼睛昏迷;
\par }{\Q 恐怕眼睛看见,
\par }{\Q 耳朵听见,
\par }{\Q 心里明白,
\par }{\Q 回转过来,便得医治。」
\par }{\MM \VS{11}我就说:「主啊,这到几时为止呢?」他说:
\par }{\Q 直到城邑荒凉,无人居住,
\par }{\Q 房屋{\ADD{空闲}}无人,地土极其荒凉。
\par }{\Q \VS{12}并且耶和华将人迁到远方,
\par }{\Q 在这境内撇下的地土很多。
\par }{\Q \VS{13}境内剩下的{\ADD{人}}若还有十分之一,
\par }{\Q 也必被吞灭,
\par }{\Q 像栗树、橡树虽被砍伐,
\par }{\Q 树ⶍ子却仍存留。
\par }{\Q 这圣洁的种类在国中也是如此。

\par }\Chap{7}{\SH 传给亚哈斯王的信息
\par }{\PP \VerseOne{1}{\PN{乌西雅}}的孙子、{\PN{约坦}}的儿子、{\PN{犹大}}王{\PN{亚哈斯}}{\ADD{在位}}的时候,{\PN{亚兰}}王{\PN{利汛}}和{\PN{利玛利}}的儿子、{\PN{以色列}}王{\PN{比加}}上来攻打{\PN{耶路撒冷}},却不能攻取。
\VS{2}有人告诉{\PN{大卫}}家说:「{\PN{亚兰}}与{\PN{以法莲}}已经同盟。」王的心和百姓的心就都跳动,好像林中的树被风吹动一样。
\par }{\PP \VS{3}耶和华对{\PN{以赛亚}}说:「你和你的儿子{\PN{施亚雅述}}出去,到上池的水沟头,在漂布地的大路上,去迎接{\PN{亚哈斯}},
\VS{4}对他说:『你要谨慎安静,不要因{\PN{亚兰}}{\ADD{王}}{\PN{利汛}}和{\PN{利玛利}}的儿子这两个冒烟的火把头所发的烈怒害怕,也不要心里胆怯。
\VS{5}因为{\PN{亚兰}}和{\PN{以法莲}},并{\PN{利玛利}}的儿子,设恶谋害你,
\VS{6}说:我们可以上去攻击{\PN{犹大}},扰乱它,攻破它,在其中立{\PN{他比勒}}的儿子为王。
\VS{7}所以主耶和华如此说:
\par }{\Q 这所谋的必立不住,
\par }{\Q 也不得成就。
\par }{\Q \VS{8}原来{\PN{亚兰}}的首城是{\PN{大马士革}};
\par }{\Q {\PN{大马士革}}的首领是{\PN{利汛}}。
\par }{\PP (六十五年之内,{\PN{以法莲}}必然破坏,不再成为国民。)
\par }{\Q \VS{9}{\PN{以法莲}}的首城是{\PN{撒马利亚}};
\par }{\Q {\PN{撒马利亚}}的首领是{\PN{利玛利}}的儿子。
\par }{\Q 你们若是不信,
\par }{\Q 定然不得立稳。』」
\par }{\SH 以马内利的记号
\par }{\PP \VS{10}耶和华又晓谕{\PN{亚哈斯}}说:
\VS{11}「你向耶和华—你的 神求一个兆头:或求{\ADD{显}}在深处,或求{\ADD{显}}在高处。」
\VS{12}{\PN{亚哈斯}}说:「我不求;我不试探耶和华。」
\VS{13}{\PN{以赛亚}}说:「{\PN{大卫}}家啊,你们当听!你们使人厌烦岂算小事,还要使我的 神厌烦吗?
\VS{14}因此,主自己要给你们一个兆头,必有童女怀孕生子,给他起名叫{\PN{以马内利}}\FTNT{}{{\FR 7:14: }就是 神与我们同在的意思}。
\VS{15}到他晓得弃恶择善的时候,他必吃奶油与蜂蜜。
\VS{16}因为在这孩子还不晓得弃恶择善之先,你所憎恶的那二王之地必致见弃。
\VS{17}耶和华必使{\PN{亚述}}王{\ADD{攻击你}}的日子临到你和你的百姓,并你的父家,自从{\PN{以法莲}}离开{\PN{犹大}}以来,未曾有这样的日子。
\par }{\PP \VS{18}「那时,耶和华要发嘶声,使{\PN{埃及}}江河源头的苍蝇和{\PN{亚述}}地的蜂子飞来;
\VS{19}都必飞来,落在荒凉的谷内、磐石的穴里,和一切荆棘篱笆中,并一切的草场上。
\par }{\PP \VS{20}「那时,主必用大河外赁的剃头刀,就是{\PN{亚述}}王,剃去头发和脚上的毛,并要剃净胡须。
\par }{\PP \VS{21}「那时,一个人要养活一只母牛犊,两只母绵羊;
\VS{22}因为出的奶多,他就得吃奶油,在境内所剩的人都要吃奶油与蜂蜜。
\par }{\PP \VS{23}「从前,凡种一千棵葡萄树、值银一千{\ADD{舍客勒}}的地方,到那时必长荆棘和蒺藜。
\VS{24}人上那里去,必带弓箭,因为遍地满了荆棘和蒺藜。
\VS{25}所有用锄刨挖的山地,你因怕荆棘和蒺藜,不敢上那里去;只可成了放牛之处,为羊践踏之地。」

\par }\Chap{8}{\SH 用以赛亚的儿子作为记号
\par }{\PP \VerseOne{1}耶和华对我说:「你取一个大牌,拿人所用的笔\FTNT{}{{\FR 8:1: }或译:人常用的字},写上『玛黑珥·沙拉勒·哈施·罢斯』\FTNT{}{{\FR 8:1: }就是掳掠速临、抢夺快到的意思}。
\VS{2}我要用诚实的见证人,祭司{\PN{乌利亚}}和{\PN{耶比利家}}的儿子{\PN{撒迦利亚}}记录这事。」
\par }{\PP \VS{3}我—{\ADD{
{\PN{以赛亚}}}}与妻子\FTNT{}{{\FR 8:3: }原文是女先知}同室;她怀孕生子,耶和华就对我说:「给他起名叫{\PN{玛黑珥·沙拉勒·哈施·罢斯}};
\VS{4}因为在这小孩子不晓得叫父叫母之先,{\PN{大马士革}}的财宝和{\PN{撒马利亚}}的掳物必在{\PN{亚述}}王面前搬了去。」
\par }{\SH 亚述王快来了
\par }{\PP \VS{5}耶和华又晓谕我说:
\VS{6}「这百姓既厌弃{\PN{西罗亚}}缓流的水,喜悦{\PN{利汛}}和{\PN{利玛利}}的儿子;
\VS{7}因此,主必使大河翻腾的水猛然冲来,就是{\PN{亚述}}王和他所有的威势,必漫过一切的水道,涨过两岸;
\VS{8}必冲入{\PN{犹大}},涨溢泛滥,直到颈项。{\PN{以马内利}}啊,他展开翅膀,遍满你的地。」
\par }{\Q \VS{9}列国的人民哪,任凭你们喧嚷,终必破坏;
\par }{\Q 远方的众人哪,当侧耳而听!
\par }{\Q 任凭你们束起腰来,终必破坏;
\par }{\Q 你们束起腰来,终必破坏。
\par }{\Q \VS{10}任凭你们同谋,终归无有;
\par }{\Q 任凭你们言定,终不成立;
\par }{\Q 因为 神与我们同在。
\par }{\SH 耶和华警告先知
\par }{\PP \VS{11}耶和华以大能的手,指教我不可行这百姓所行的道,对我这样说:
\VS{12}「这百姓说同谋背叛,你们不要说同谋背叛。他们所怕的,你们不要怕,也不要畏惧。
\VS{13}但要尊万军之耶和华为圣,以他为你们所当怕的,所当畏惧的。
\VS{14}他必作为圣所,却向{\PN{以色列}}两家作绊脚的石头,跌人的磐石;向{\PN{耶路撒冷}}的居民作为圈套和网罗。
\VS{15}许多人必在其上绊脚跌倒,而且跌碎,并陷入网罗,被缠住。」
\par }{\SH 不可求问死人
\par }{\PP \VS{16}你要卷起律法{\ADD{书}},在我门徒中间封住训诲。
\VS{17}我要等候那掩面不顾{\PN{雅各}}家的耶和华;我也要仰望他。
\VS{18}看哪,我与耶和华所给我的儿女,就是从住在{\PN{锡安山}}万军之耶和华来的,在{\PN{以色列}}中作为预兆和奇迹。
\VS{19}有人对你们说:「当求问那些交鬼的和行巫术的,就是声音绵蛮,言语微细的。」{\ADD{你们便回答说}}:「百姓不当求问自己的 神吗?岂可为活人{\ADD{求问}}死人呢?」
\VS{20}人当以训诲和法度为标准;他们所说的,若不与此相符,必不得见晨光。
\par }{\SH 遭难的时候
\par }{\PP \VS{21}他们必经过这地,受艰难,受饥饿;饥饿的时候,心中焦躁,咒骂自己的君王和自己的 神。
\VS{22}仰观上天,俯察下地,不料,尽是艰难、黑暗,和幽暗的痛苦。{\ADD{他们必}}被赶入乌黑的黑暗中去。

\par }\Chap{9}{\PP \VerseOne{1}但那受过痛苦的必不再见幽暗。
\par }{\SH 未来的君王
\par }{\PP 从前 神使{\PN{西布伦}}地和{\PN{拿弗他利}}地被藐视,末后却使这沿海的路,{\PN{约旦河}}外,外邦人的{\PN{加利利}}地得着荣耀。
\par }{\Q \VS{2}在黑暗中行走的百姓看见了大光;
\par }{\Q 住在死荫之地的人有光照耀他们。
\par }{\Q \VS{3}你使这国民繁多,
\par }{\Q 加增他们的喜乐;
\par }{\Q 他们在你面前欢喜,
\par }{\Q 好像收割的欢喜,
\par }{\Q 像人分掳物那样的快乐。
\par }{\Q \VS{4}因为他们所负的重轭
\par }{\Q 和肩头上的杖,
\par }{\Q 并欺压他们人的棍,
\par }{\Q 你都已经折断,
\par }{\Q 好像在{\PN{米甸}}的日子一样。
\par }{\Q \VS{5}战士在乱杀之间所穿戴的盔甲,
\par }{\Q 并那滚在血中的衣服,
\par }{\Q 都必作为可烧的,
\par }{\Q 当作火柴。
\par }{\Q \VS{6}因有一婴孩为我们而生;
\par }{\Q 有一子赐给我们。
\par }{\Q 政权必担在他的肩头上;
\par }{\Q 他名称为「奇妙策士、全能的 神、永在的父、和平的君」。
\par }{\Q \VS{7}他的政权与平安必加增无穷。
\par }{\Q 他必在{\PN{大卫}}的宝座上治理他的国,
\par }{\Q 以公平公义使国坚定稳固,
\par }{\Q 从今直到永远。
\par }{\Q 万军之耶和华的热心必成就这事。
\par }{\BB \par }{\SH 耶和华要惩罚以色列
\par }{\Q \VS{8}主使一言入于{\PN{雅各}}{\ADD{家}},
\par }{\Q 落于{\PN{以色列}}{\ADD{家}}。
\par }{\Q \VS{9}这众百姓,就是{\PN{以法莲}}
\par }{\Q 和{\PN{撒马利亚}}的居民,都要知道;
\par }{\Q 他们凭骄傲自大的心说:
\par }{\Q \VS{10}砖{\ADD{墙}}塌了,我们却要凿石头建筑;
\par }{\Q 桑树砍了,我们却要换香柏树。
\par }{\Q \VS{11}因此,耶和华要高举{\PN{利汛}}的敌人
\par }{\Q 来攻击{\PN{以色列}},
\par }{\Q 并要激动{\PN{以色列}}的仇敌。
\par }{\Q \VS{12}东有{\PN{亚兰}}人,西有{\PN{非利士}}人;
\par }{\Q 他们张口要吞吃{\PN{以色列}}。
\par }{\Q 虽然如此,耶和华的怒气还未转消;
\par }{\Q 他的手仍伸不缩。
\par }{\BB \par }{\Q \VS{13}这百姓还没有归向击打他们的主,
\par }{\Q 也没有寻求万军之耶和华。
\par }{\Q \VS{14}因此,耶和华一日之间
\par }{\Q 必从{\PN{以色列}}中剪除头与尾,
\par }{\Q 棕枝与芦苇—
\par }{\Q \VS{15}长老和尊贵人就是头,
\par }{\Q 以谎言教人的先知就是尾。
\par }{\Q \VS{16}因为,引导这百姓的使他们走错了路;
\par }{\Q 被引导的都必败亡。
\par }{\Q \VS{17}所以,主必不喜悦他们的少年人,
\par }{\Q 也不怜恤他们的孤儿寡妇;
\par }{\Q 因为,各人是亵渎的,是行恶的,
\par }{\Q 并且各人的口都说愚妄的话。
\par }{\Q 虽然如此,耶和华的怒气还未转消;
\par }{\Q 他的手仍伸不缩。
\par }{\BB \par }{\Q \VS{18}邪恶像火焚烧,
\par }{\Q 烧灭荆棘和蒺藜,
\par }{\Q 在稠密的树林中着起来,
\par }{\Q 就成为烟柱,旋转上腾。
\par }{\Q \VS{19}因万军之耶和华的烈怒,地都烧遍;
\par }{\Q 百姓成为火柴;
\par }{\Q 无人怜爱弟兄。
\par }{\Q \VS{20}有人右边抢夺,仍受饥饿,
\par }{\Q 左边吞吃,仍不饱足;
\par }{\Q 各人吃自己膀臂上的肉。
\par }{\Q \VS{21}{\PN{玛拿西}}{\ADD{吞吃}}\FTNT{}{{\FR 9:21: }或译:攻击;下同}{\PN{以法莲}};
\par }{\Q {\PN{以法莲}}{\ADD{吞吃}}{\PN{玛拿西}},
\par }{\Q 又一同攻击{\PN{犹大}}。
\par }{\Q 虽然如此,耶和华的怒气还未转消;
\par }{\Q 他的手仍伸不缩。

\par }\PoetryChap{10}{\Q \VerseOne{1}祸哉!那些设立不义之律例的
\par }{\Q 和记录奸诈之判语的,
\par }{\Q \VS{2}为要屈枉穷乏人,
\par }{\Q 夺去我民中困苦人的理,
\par }{\Q 以寡妇当作掳物,
\par }{\Q 以孤儿当作掠物。
\par }{\Q \VS{3}到降罚的日子,有灾祸从远方临到,
\par }{\Q 那时,你们怎样行呢?
\par }{\Q 你们向谁逃奔求救呢?
\par }{\Q 你们的荣耀\FTNT{}{{\FR 10:3: }或译:财宝}存留何处呢?
\par }{\Q \VS{4}他们只得屈身在被掳的人以下,
\par }{\Q 仆倒在被杀的人以下。
\par }{\Q 虽然如此,耶和华的怒气还未转消;
\par }{\Q 他的手仍伸不缩。
\par }{\SH  神用亚述王为工具
\par }{\Q \VS{5}{\PN{亚述}}是我怒气的棍,
\par }{\Q 手中拿我恼恨的杖。
\par }{\Q \VS{6}我要打发他攻击亵渎的国民,
\par }{\Q 吩咐他攻击我所恼怒的百姓,
\par }{\Q 抢财为掳物,夺货为掠物,
\par }{\Q 将他们践踏,像街上的泥土一样。
\par }{\Q \VS{7}然而,他不是这样的意思;
\par }{\Q 他心也不这样打算。
\par }{\Q 他心里倒想毁灭,
\par }{\Q 剪除不少的国。
\par }{\Q \VS{8}他说:我的臣仆岂不都是王吗?
\par }{\Q \VS{9}{\PN{迦勒挪}}岂不像{\PN{迦基米施}}吗?
\par }{\Q {\PN{哈马}}岂不像{\PN{亚珥拔}}吗?
\par }{\Q {\PN{撒马利亚}}岂不像{\PN{大马士革}}吗?
\par }{\Q \VS{10}我手已经搆到有偶像的国;
\par }{\Q 这些国雕刻的偶像
\par }{\Q 过于{\PN{耶路撒冷}}和{\PN{撒马利亚}}的偶像。
\par }{\Q \VS{11}我怎样待{\PN{撒马利亚}}和其中的偶像,
\par }{\Q 岂不照样待{\PN{耶路撒冷}}和其中的偶像吗?
\par }{\PP \VS{12}主在{\PN{锡安山}}和{\PN{耶路撒冷}}成就他一切工作的时候,{\ADD{主说}}:「我必罚{\PN{亚述}}王自大的心和他高傲眼目的荣耀。」
\par }{\Q \VS{13}因为他说:
\par }{\Q 我所成就的事是靠我手的能力
\par }{\Q 和我的智慧,
\par }{\Q 我本有聪明。
\par }{\Q 我挪移列国的地界,
\par }{\Q 抢夺他们所积蓄的财宝;
\par }{\Q 并且我像勇士,使坐宝座的降为卑。
\par }{\Q \VS{14}我的手搆到列国的财宝,
\par }{\Q 好像人搆到鸟窝;
\par }{\Q 我也得了全地,
\par }{\Q 好像人拾起所弃的雀蛋。
\par }{\Q 没有动翅膀的;
\par }{\Q 没有张嘴的,也没有鸣叫的。
\par }{\BB \par }{\Q \VS{15}斧岂可向用斧砍木的自夸呢?
\par }{\Q 锯岂可向用锯的自大呢?
\par }{\Q 好比棍抡起那举棍的,
\par }{\Q 好比杖举起那非木的人。
\par }{\Q \VS{16}因此,主—万军之耶和华
\par }{\Q 必使{\PN{亚述}}王的肥壮人变为瘦弱,
\par }{\Q 在他的荣华之下必有火着起,
\par }{\Q 如同焚烧一样。
\par }{\Q \VS{17}{\PN{以色列}}的光必如火;
\par }{\Q 他的圣者必如火焰。
\par }{\Q 在一日之间,将{\PN{亚述}}王的荆棘
\par }{\Q 和蒺藜焚烧净尽;
\par }{\Q \VS{18}又将他树林和肥田的荣耀全然烧尽,
\par }{\Q 好像拿军旗的昏过去一样。
\par }{\Q \VS{19}他林中剩下的树必稀少,
\par }{\Q 就是孩子也能写其数。
\par }{\SH 残存之民归回
\par }{\PP \VS{20}到那日,{\PN{以色列}}所剩下的和{\PN{雅各}}家所逃脱的,不再倚靠那击打他们的,却要诚实倚靠耶和华—{\PN{以色列}}的圣者。
\VS{21}所剩下的,就是{\PN{雅各}}{\ADD{家}}所剩下的,必归回全能的 神。
\VS{22}{\PN{以色列}}啊,你的百姓虽多如海沙,{\ADD{惟有}}剩下的归回。原来灭绝的事已定,必有公义施行,如水涨溢。
\VS{23}因为主—万军之耶和华在全地之中必成就所定规的结局。
\par }{\SH 耶和华要惩罚亚述
\par }{\PP \VS{24}所以,主—万军之耶和华如此说:「住{\PN{锡安}}我的百姓啊,{\PN{亚述}}王虽然用棍击打你,又照{\PN{埃及}}的样子举杖攻击你,你却不要怕他。
\VS{25}因为还有一点点时候,{\ADD{向你们}}发的忿恨就要完毕,我的怒气要向他{\ADD{发作}},使他灭亡。
\VS{26}万军之耶和华要兴起鞭来攻击他,好像在{\PN{俄立}}磐石那里杀戮{\PN{米甸}}人一样。耶和华的杖要向海伸出,把杖举起,像在{\PN{埃及}}一样。
\VS{27}到那日,{\PN{亚述}}王的重担必离开你的肩头;他的轭必离开你的颈项;那轭也必因肥壮的缘故撑断\FTNT{}{{\FR 10:27: }或译:因膏油的缘故毁坏}。」
\par }{\SH 侵略者进攻
\par }{\Q \VS{28}{\PN{亚述}}王来到{\PN{亚叶}},
\par }{\Q 经过{\PN{米矶
}},
\par }{\Q 在{\PN{密抹}}安放辎重。
\par }{\Q \VS{29}他们过了隘口,
\par }{\Q 在{\PN{迦巴}}住宿。
\par }{\Q {\PN{拉玛}}人战兢;
\par }{\Q {\PN{扫罗}}的{\PN{基比亚}}人逃跑。
\par }{\Q \VS{30}{\PN{迦琳}}的居民\FTNT{}{{\FR 10:30: }原文是女子}哪,要高声呼喊!
\par }{\Q {\PN{莱煞}}人哪,须听!
\par }{\Q 哀哉!困苦的{\PN{亚拿突}}啊。
\par }{\Q \VS{31}{\PN{玛得米那}}人躲避;
\par }{\Q {\PN{基柄}}的居民逃遁。
\par }{\Q \VS{32}当那日,{\PN{亚述}}王要在{\PN{挪伯}}歇兵,
\par }{\Q 向{\PN{锡安}}女子的山—
\par }{\Q 就是{\PN{耶路撒冷}}的山—抡手{\ADD{攻他}}。
\par }{\BB \par }{\Q \VS{33}看哪,主—万军之耶和华
\par }{\Q 以惊吓削去树枝;
\par }{\Q 长高的必被砍下,
\par }{\Q 高大的必被伐倒。
\par }{\Q \VS{34}稠密的树林,他要用铁器砍下;
\par }{\Q {\PN{黎巴嫩}}{\ADD{的树木}}必被大能者伐倒。

\par }\Chap{11}{\SH 和平的国度
\par }{\Q \VerseOne{1}从{\PN{耶西}}的本\FTNT{}{{\FR 11:1: }原文是ⶍ}必发一条;
\par }{\Q 从他根生的枝子必结果实。
\par }{\Q \VS{2}耶和华的灵必住在他身上,
\par }{\Q 就是使他有智慧和聪明的灵,
\par }{\Q 谋略和能力的灵,
\par }{\Q 知识和敬畏耶和华的灵。
\par }{\Q \VS{3}他必以敬畏耶和华为乐;
\par }{\Q 行审判不凭眼见,
\par }{\Q 断是非也不凭耳闻;
\par }{\Q \VS{4}却要以公义审判贫穷人,
\par }{\Q 以正直判断世上的谦卑人,
\par }{\Q 以口中的杖击打世界,
\par }{\Q 以嘴里的气杀戮恶人。
\par }{\Q \VS{5}公义必当他的腰带;
\par }{\Q 信实必当他胁下的带子。
\par }{\BB \par }{\Q \VS{6}豺狼必与绵羊羔同居,
\par }{\Q 豹子与山羊羔同卧;
\par }{\Q 少壮狮子与牛犊并肥畜同群;
\par }{\Q 小孩子要牵引它们。
\par }{\Q \VS{7}牛必与熊同食;
\par }{\Q 牛犊必与小熊同卧;
\par }{\Q 狮子必吃草,与牛一样。
\par }{\Q \VS{8}吃奶的孩子必玩耍在虺蛇的洞口;
\par }{\Q 断奶的婴儿必按手在毒蛇的穴上。
\par }{\Q \VS{9}在我圣山的遍处,
\par }{\Q 这一切都不伤人,不害物;
\par }{\Q 因为认识耶和华的知识要充满遍地,
\par }{\Q 好像水充满洋海一般。
\par }{\SH 流亡的人回乡
\par }{\PP \VS{10}到那日,{\PN{耶西}}的根立作万民的大旗;外邦人必寻求他,他安息之所大有荣耀。
\VS{11}当那日,主必二次伸手救回自己百姓中所余剩的,就是在{\PN{亚述}}、{\PN{埃及}}、{\PN{巴忒罗}}、{\PN{古实}}、{\PN{以拦}}、{\PN{示拿}}、{\PN{哈马}},并众海岛所剩下的。
\par }{\Q \VS{12}他必向列国竖立大旗,
\par }{\Q 招回{\PN{以色列}}被赶散的人,
\par }{\Q 又从地的四方聚集分散的{\PN{犹大}}人。
\par }{\Q \VS{13}{\PN{以法莲}}的嫉妒就必消散;
\par }{\Q 扰害{\PN{犹大}}的必被剪除。
\par }{\Q {\PN{以法莲}}必不嫉妒{\PN{犹大}},
\par }{\Q {\PN{犹大}}也不扰害{\PN{以法莲}}。
\par }{\Q \VS{14}他们要向西飞,
\par }{\Q 扑在{\PN{非利士}}人的肩头上\FTNT{}{{\FR 11:14: }肩头上:或译西界},
\par }{\Q 一同掳掠东方人,
\par }{\Q 伸手按住{\PN{以东}}和{\PN{摩押}};
\par }{\Q {\PN{亚扪}}人也必顺服他们。
\par }{\Q \VS{15}耶和华必使{\PN{埃及}}海汊枯干,
\par }{\Q 抡手用暴热的风使大河分为七条,
\par }{\Q 令人过去不致湿脚。
\par }{\Q \VS{16}为主余剩的百姓,
\par }{\Q 就是从{\PN{亚述}}剩下回来的,
\par }{\Q 必有一条大道,
\par }{\Q 如当日{\PN{以色列}}从{\PN{埃及}}地上来一样。

\par }\Chap{12}{\SH 感恩的诗歌
\par }{\PP \VerseOne{1}到那日,你必说:
\par }{\Q 耶和华啊,我要称谢你!
\par }{\Q 因为你虽然向我发怒,
\par }{\Q 你的怒气却已转消;
\par }{\Q 你又安慰了我。
\par }{\BB \par }{\Q \VS{2}看哪! 神是我的拯救;
\par }{\Q 我要倚靠他,并不惧怕。
\par }{\Q 因为主耶和华是我的力量,
\par }{\Q 是我的诗歌,
\par }{\Q 他也成了我的拯救。
\par }{\BB \par }{\PP \VS{3}所以,你们必从救恩的泉源欢然取水。
\VS{4}在那日,你们要说:
\par }{\Q 当称谢耶和华,求告他的名;
\par }{\Q 将他所行的传扬在万民中,
\par }{\Q 提说他的名已被尊崇。
\par }{\BB \par }{\Q \VS{5}你们要向耶和华唱歌,
\par }{\Q 因他所行的甚是美好;
\par }{\Q 但愿这事普传天下。
\par }{\Q \VS{6}{\PN{锡安}}的居民哪,当扬声欢呼,
\par }{\Q 因为在你们中间的{\PN{以色列}}圣者乃为至大。

\par }\Chap{13}{\SH  神要惩罚巴比伦
\par }{\PP \VerseOne{1}{\PN{亚摩斯}}的儿子{\PN{以赛亚}}得默示,论{\PN{巴比伦}}。
\par }{\Q \VS{2}应当在净光的山竖立大旗,
\par }{\Q 向群众扬声招手,
\par }{\Q 使他们进入贵胄的门。
\par }{\Q \VS{3}我吩咐我所挑出来的人;
\par }{\Q 我招呼我的勇士—
\par }{\Q 就是那矜夸高傲之辈,
\par }{\Q 为要成就我怒中所定的。
\par }{\BB \par }{\Q \VS{4}山间有多人的声音,
\par }{\Q 好像是大国人民。
\par }{\Q 有许多国的民聚集哄嚷的声音;
\par }{\Q 这是万军之耶和华点齐军队,预备打仗。
\par }{\Q \VS{5}他们从远方来,
\par }{\Q 从天边来,
\par }{\Q 就是耶和华并他恼恨的兵器
\par }{\Q 要毁灭这全地。
\par }{\BB \par }{\Q \VS{6}你们要哀号,
\par }{\Q 因为耶和华的日子临近了!
\par }{\Q 这日来到,
\par }{\Q 好像毁灭从全能者来到。
\par }{\Q \VS{7}所以,人手都必软弱;
\par }{\Q 人心都必消化。
\par }{\Q \VS{8}他们必惊惶悲痛;
\par }{\Q 愁苦必将他们抓住。
\par }{\Q 他们疼痛,好像产难的妇人一样,
\par }{\Q 彼此惊奇相看,脸如火焰。
\par }{\BB \par }{\Q \VS{9}耶和华的日子临到,
\par }{\Q 必有残忍、忿恨、烈怒,
\par }{\Q 使这地荒凉,
\par }{\Q 从其中除灭罪人。
\par }{\Q \VS{10}天上的众星群宿都不发光;
\par }{\Q 日头一出就变黑暗;
\par }{\Q 月亮也不放光。
\par }{\Q \VS{11}我必因邪恶刑罚世界,
\par }{\Q 因罪孽刑罚恶人,
\par }{\Q 使骄傲人的狂妄止息,
\par }{\Q 制伏强暴人的狂傲。
\par }{\Q \VS{12}我必使人比精金还少,
\par }{\Q 使人比{\PN{俄斐}}纯金更少。
\par }{\Q \VS{13}我—万军之耶和华在忿恨中发烈怒的日子,
\par }{\Q 必使天震动,
\par }{\Q 使地摇撼,离其本位。
\par }{\Q \VS{14}人必像被追赶的鹿,
\par }{\Q 像无人收聚的羊,
\par }{\Q 各归回本族,
\par }{\Q 各逃到本土。
\par }{\Q \VS{15}凡被{\ADD{仇敌}}追上的必被刺死;
\par }{\Q 凡被捉住的必被刀杀。
\par }{\Q \VS{16}他们的婴孩必在他们眼前摔碎;
\par }{\Q 他们的房屋必被抢夺;
\par }{\Q 他们的妻子必被玷污。
\par }{\BB \par }{\Q \VS{17}我必激动{\PN{米底亚}}人来攻击他们。
\par }{\Q {\PN{米底亚}}人不注重银子,
\par }{\Q 也不喜爱金子。
\par }{\Q \VS{18}他们必用弓击碎少年人,
\par }{\Q 不怜悯妇人所生的,
\par }{\Q 眼也不顾惜孩子。
\par }{\Q \VS{19}{\PN{巴比伦}}素来为列国的荣耀,
\par }{\Q 为{\PN{迦勒底}}人所矜夸的华美,
\par }{\Q 必像 神所倾覆的{\PN{所多玛}}、{\PN{蛾摩拉}}一样。
\par }{\Q \VS{20}其内必永无人烟,
\par }{\Q 世世代代无人居住。
\par }{\Q {\PN{阿拉伯}}人也不在那里支搭帐棚;
\par }{\Q 牧羊的人也不使羊群卧在那里。
\par }{\Q \VS{21}只有旷野的走兽卧在那里;
\par }{\Q 咆哮的兽满了房屋。
\par }{\Q 鸵鸟住在那里;
\par }{\Q 野山羊在那里跳舞。
\par }{\Q \VS{22}豺狼必在它宫中呼号;
\par }{\Q 野狗必在它华美殿内吼叫。
\par }{\Q {\PN{巴比伦}}{\ADD{受罚}}的时候临近;
\par }{\Q 它的日子必不长久。

\par }\Chap{14}{\SH 从流亡中归回
\par }{\PP \VerseOne{1}耶和华要怜恤{\PN{雅各}},必再拣选{\PN{以色列}},将他们安置在本地。寄居的必与他们联合,紧贴{\PN{雅各}}家。
\VS{2}外邦人必将他们带回本土;{\PN{以色列}}家必在耶和华的地上得外邦人为仆婢,也要掳掠先前掳掠他们的,辖制先前欺压他们的。
\par }{\SH 巴比伦王下到阴间
\par }{\PP \VS{3}当耶和华使你脱离愁苦、烦恼,并人勉强你做的苦工,得享安息的日子,
\VS{4}你必题这诗歌论{\PN{巴比伦}}王说:
\par }{\Q 欺压人的何竟息灭?
\par }{\Q 强暴的何竟止息?
\par }{\Q \VS{5}耶和华折断了恶人的杖,
\par }{\Q 辖制人的圭,
\par }{\Q \VS{6}就是在忿怒中连连攻击众民的,
\par }{\Q 在怒气中辖制列国,
\par }{\Q 行逼迫无人阻止的。
\par }{\Q \VS{7}现在全地得安息,享平静;
\par }{\Q 人皆发声欢呼。
\par }{\Q \VS{8}松树和{\PN{黎巴嫩}}的香柏树
\par }{\Q 都因你欢乐,{\ADD{说}}:
\par }{\Q 自从你仆倒,
\par }{\Q 再无人上来砍伐我们。
\par }{\Q \VS{9}你下到阴间,
\par }{\Q 阴间就因你震动来迎接你,
\par }{\Q 又因你惊动在世曾为首领的阴魂,
\par }{\Q 并使那曾为列国君王的,
\par }{\Q 都离位站起。
\par }{\Q \VS{10}他们都要发言对你说:
\par }{\Q 你也变为软弱像我们一样吗?
\par }{\Q 你也成了我们的样子吗?
\par }{\Q \VS{11}你的威势和你琴瑟的声音都下到阴间。
\par }{\Q 你下铺的是虫,上盖的是蛆。
\par }{\BB \par }{\Q \VS{12}明亮之星,早晨之子啊,
\par }{\Q 你何竟从天坠落?
\par }{\Q 你这攻败列国的何竟被砍倒在地上?
\par }{\Q \VS{13}你心里曾说:
\par }{\Q 我要升到天上;
\par }{\Q 我要高举我的宝座在 神众星以上;
\par }{\Q 我要坐在聚会的山上,在北方的极处。
\par }{\Q \VS{14}我要升到高云之上;
\par }{\Q 我要与至上者同等。
\par }{\Q \VS{15}然而,你必坠落阴间,
\par }{\Q 到坑中极深之处。
\par }{\Q \VS{16}凡看见你的都要定睛看你,
\par }{\Q 留意看你,{\ADD{说}}:
\par }{\Q 使大地战抖,
\par }{\Q 使列国震动,
\par }{\Q \VS{17}使世界如同荒野,
\par }{\Q 使城邑倾覆,
\par }{\Q 不释放被掳的人归家,
\par }{\Q 是这个人吗?
\par }{\Q \VS{18}列国的君王
\par }{\Q 俱各在自己阴宅的荣耀中安睡。
\par }{\Q \VS{19}惟独你被抛弃,
\par }{\Q 不得入你的坟墓,
\par }{\Q 好像可憎的枝子,
\par }{\Q 以被杀的人为衣,
\par }{\Q 就是被刀刺透,
\par }{\Q 坠落坑中石头那里的;
\par }{\Q 你又像被践踏的尸首一样。
\par }{\Q \VS{20}你不得与君王同葬;
\par }{\Q 因为你败坏你的国,杀戮你的民。
\par }{\BB \par }{\Q 恶人后裔的名,必永不提说。
\par }{\Q \VS{21}先人既有罪孽,
\par }{\Q 就要预备杀戮他的子孙,
\par }{\Q 免得他们兴起来,得了遍地,
\par }{\Q 在世上修满城邑。
\par }{\SH  神要毁灭巴比伦
\par }{\Q \VS{22}万军之耶和华说:
\par }{\Q 我必兴起攻击他们,
\par }{\Q 将{\PN{巴比伦}}的名号和所余剩的人,
\par }{\Q 连子带孙一并剪除。
\par }{\Q 这是耶和华说的。
\par }{\Q \VS{23}我必使{\PN{巴比伦}}为箭猪所得,
\par }{\Q 又变为水池;
\par }{\Q 我要用灭亡的扫帚扫净它。
\par }{\Q 这是万军之耶和华说的。
\par }{\SH  神要毁灭亚述
\par }{\Q \VS{24}万军之耶和华起誓说:
\par }{\Q 我怎样思想,必照样成就;
\par }{\Q 我怎样定意,必照样成立,
\par }{\Q \VS{25}就是在我地上打折{\PN{亚述}}人,
\par }{\Q 在我山上将他践踏。
\par }{\Q 他加的轭必离开{\PN{以色列}}人;
\par }{\Q 他加的重担必离开他们的肩头。
\par }{\Q \VS{26}这是向全地所定的旨意;
\par }{\Q 这是向万国所伸出的手。
\par }{\Q \VS{27}万军之耶和华既然定意,谁能废弃呢?
\par }{\Q 他的手已经伸出,谁能转回呢?
\par }{\SH  神要毁灭非利士
\par }{\Q \VS{28}{\PN{亚哈斯}}王崩的那年,就有以下的默示:
\par }{\Q \VS{29}{\PN{非利士}}全地啊,
\par }{\Q 不要因击打你的杖折断就喜乐。
\par }{\Q 因为从蛇的根必生出毒蛇;
\par }{\Q 它所生的是火焰的飞龙。
\par }{\Q \VS{30}贫寒人的长子必有所食;
\par }{\Q 穷乏人必安然躺卧。
\par }{\Q 我必以饥荒治死你的根;
\par }{\Q 你所余剩的人必被杀戮。
\par }{\Q \VS{31}门哪,应当哀号!
\par }{\Q 城啊,应当呼喊!
\par }{\Q {\PN{非利士}}全地啊,你都消化了!
\par }{\Q 因为有烟从北方出来,
\par }{\Q 他行伍中并无乱队的。
\par }{\BB \par }{\Q \VS{32}可怎样回答外邦\FTNT{}{{\FR 14:32: }或指非利士}的使者呢?
\par }{\Q 必说:耶和华建立了{\PN{锡安}};
\par }{\Q 他百姓中的困苦人必投奔在其中。

\par }\Chap{15}{\SH  神要毁灭摩押
\par }{\PP \VerseOne{1}论{\PN{摩押}}的默示:
\par }{\Q 一夜之间,{\PN{摩押}}的{\PN{亚珥}}变为荒废,
\par }{\Q 归于无有;
\par }{\Q 一夜之间,{\PN{摩押}}的{\PN{基珥}}变为荒废,
\par }{\Q 归于无有。
\par }{\Q \VS{2}他们上{\PN{巴益}},又往{\PN{底本}},
\par }{\Q 到高处去哭泣。
\par }{\Q {\PN{摩押}}人因{\PN{尼波}}和{\PN{米底巴}}哀号,
\par }{\Q 各人头上光秃,
\par }{\Q 胡须剃净。
\par }{\Q \VS{3}他们在街市上都腰束麻布,
\par }{\Q 在房顶上和宽阔处俱各哀号,
\par }{\Q 眼泪汪汪。
\par }{\Q \VS{4}{\PN{希实本}}和{\PN{以利亚利}}悲哀的声音达到{\PN{雅杂}},
\par }{\Q 所以{\PN{摩押}}带兵器的高声喊嚷,
\par }{\Q 人心战兢。
\par }{\Q \VS{5}我心为{\PN{摩押}}悲哀;
\par }{\Q 他的贵胄\FTNT{}{{\FR 15:5: }或译:逃民}{\ADD{逃}}到{\PN{琐珥}},
\par }{\Q 到{\PN{伊基拉·施利施亚}}。
\par }{\Q 他们上{\PN{鲁希坡}},随走随哭;
\par }{\Q 在{\PN{何罗念}}的路上,因毁灭举起哀声。
\par }{\Q \VS{6}因为{\PN{宁林}}的水成为干涸,
\par }{\Q 青草枯干,嫩草灭没,
\par }{\Q 青绿之物,一无所有。
\par }{\Q \VS{7}因此,{\PN{摩押}}人所得的财物
\par }{\Q 和所积蓄的都要运过{\PN{柳树河}}。
\par }{\Q \VS{8}哀声遍闻{\PN{摩押}}的四境;
\par }{\Q 哀号的声音达到{\PN{以基莲}};
\par }{\Q 哀号的声音达到{\PN{比珥·以琳}}。
\par }{\Q \VS{9}{\PN{底们}}的水充满了血;
\par }{\Q 我还要加增{\PN{底们}}的{\ADD{灾难}},
\par }{\Q 叫狮子来追上{\PN{摩押}}逃脱的民
\par }{\Q 和那地上所余剩的人。

\par }\Chap{16}{\SH 摩押的难民逃到犹大
\par }{\Q \VerseOne{1}你们当将羊羔奉给那地掌权的,
\par }{\Q 从{\PN{西拉}}往旷野,送到{\PN{锡安}}城\FTNT{}{{\FR 16:1: }原文是女子}的山。
\par }{\Q \VS{2}{\PN{摩押}}的居民\FTNT{}{{\FR 16:2: }原文是女子}在{\PN{亚嫩}}渡口,
\par }{\Q 必像游飞的鸟,如拆窝的{\ADD{雏}}。
\par }{\Q \VS{3}求你献谋略,行公平,
\par }{\Q 使你的影子在午间如黑夜,
\par }{\Q 隐藏被赶散的人,不可显露逃民。
\par }{\Q \VS{4}求你容我这被赶散的人和你同居。
\par }{\Q 至于{\PN{摩押}},求你作他的隐密处,
\par }{\Q 脱离灭命者的面。
\par }{\Q 勒索人的归于无有,
\par }{\Q 毁灭的事止息了,
\par }{\Q 欺压人的从国中除灭了,
\par }{\Q \VS{5}必有宝座因慈爱坚立;
\par }{\Q 必有一位诚诚实实坐在其上,
\par }{\Q 在{\PN{大卫}}帐幕中施行审判,
\par }{\Q 寻求公平,速行公义。
\par }{\BB \par }{\Q \VS{6}我们听说{\PN{摩押}}人骄傲,
\par }{\Q 是极其骄傲;
\par }{\Q 听说他狂妄、骄傲、忿怒;
\par }{\Q 他夸大的话是虚空的。
\par }{\Q \VS{7}因此,{\PN{摩押}}人必为{\PN{摩押}}哀号;
\par }{\Q 人人都要哀号。
\par }{\Q 你们{\PN{摩押}}人要为{\PN{吉珥·哈列设}}的葡萄饼哀叹,
\par }{\Q 极其忧伤。
\par }{\BB \par }{\Q \VS{8}因为{\PN{希实本}}的田地
\par }{\Q 和{\PN{西比玛}}的葡萄树都衰残了。
\par }{\Q 列国的君主折断其上美好的枝子;
\par }{\Q 这枝子长到{\PN{雅谢}}延到旷野,
\par }{\Q 嫩枝向外探出,直探过{\PN{盐海}}。
\par }{\Q \VS{9}因此,我要为{\PN{西比玛}}的葡萄树哀哭,
\par }{\Q 与{\PN{雅谢}}人哀哭一样。
\par }{\Q {\PN{希实本}}、{\PN{以利亚利}}啊,
\par }{\Q 我要以眼泪浇灌你;
\par }{\Q 因为有{\ADD{交战}}呐喊的声音
\par }{\Q 临到你夏天的果子,
\par }{\Q 并你收割的庄稼。
\par }{\Q \VS{10}从肥美的田中夺去了欢喜快乐;
\par }{\Q 在葡萄园里必无歌唱,也无欢呼的声音。
\par }{\Q 踹酒的在酒榨中不得踹出酒来;
\par }{\Q 我使他欢呼的声音止息。
\par }{\Q \VS{11}因此,我心腹为{\PN{摩押}}哀鸣如琴;
\par }{\Q 我心肠为{\PN{吉珥·哈列设}}也是如此。
\par }{\PP \VS{12}{\PN{摩押}}人朝见的时候,在高处疲乏,又到他圣所祈祷,也不蒙应允。
\par }{\PP \VS{13}这是耶和华从前论{\PN{摩押}}的话。
\VS{14}但现在耶和华说:「三年之内,照雇工的年数,{\PN{摩押}}的荣耀与他的群众必被藐视,余剩的人甚少无几。」

\par }\Chap{17}{\SH  神要刑罚亚兰和以色列
\par }{\PP \VerseOne{1}论{\PN{大马士革}}的默示:
\par }{\Q 看哪,{\PN{大马士革}}已被废弃,不再为城,
\par }{\Q 必变作乱堆。
\par }{\Q \VS{2}{\PN{亚罗珥}}的城邑已被撇弃,
\par }{\Q 必成为牧羊之处;
\par }{\Q 羊在那里躺卧,无人惊吓。
\par }{\Q \VS{3}{\PN{以法莲}}不再有保障;
\par }{\Q {\PN{大马士革}}不再有国权;
\par }{\Q {\PN{亚兰}}所剩下的
\par }{\Q 必像{\PN{以色列}}人的荣耀消灭一样。
\par }{\Q 这是万军之耶和华说的。
\par }{\BB \par }{\Q \VS{4}到那日,{\PN{雅各}}的荣耀必至枵薄;
\par }{\Q 他肥胖的身体必渐瘦弱。
\par }{\Q \VS{5}就必像收割的人收敛禾稼,
\par }{\Q 用手割取穗子,
\par }{\Q 又像人在{\PN{利乏音谷}}拾取遗落的穗子。
\par }{\Q \VS{6}其间所剩下的不多,好像人打橄榄树—
\par }{\Q 在尽上的枝梢上只剩两三个果子;
\par }{\Q 在多果树的旁枝上只剩四五个果子。
\par }{\Q 这是耶和华—{\PN{以色列}}的 神说的。
\par }{\PP \VS{7}当那日,人必仰望造他们的主,眼目重看{\PN{以色列}}的圣者。
\VS{8}他们必不仰望祭坛,就是自己手所筑的,也不重看自己指头所做的,无论是木偶是日像。
\par }{\PP \VS{9}在那日,他们的坚固城必像树林中和山顶上所撇弃的地方,就是从前在{\PN{以色列}}人面前被人撇弃的。这样,地就荒凉了。
\par }{\Q \VS{10}因你忘记救你的 神,
\par }{\Q 不记念你能力的磐石;
\par }{\Q 所以,你栽上佳美的树秧子,
\par }{\Q 插上异样的栽子。
\par }{\Q \VS{11}栽种的日子,你周围圈上篱笆,
\par }{\Q 又到早晨使你所种的开花;
\par }{\Q 但在愁苦极其伤痛的日子,
\par }{\Q 所收割的都飞去了。
\par }{\SH 敌国都被击败
\par }{\Q \VS{12}唉!多民哄嚷,好像海浪匉訇;
\par }{\Q 列邦奔腾,好像猛水滔滔;
\par }{\Q \VS{13}列邦奔腾,好像多水滔滔;
\par }{\Q 但 神斥责他们,他们就远远逃避,
\par }{\Q 又被追赶,如同山上的风前糠,
\par }{\Q 又如暴风前的旋风土。
\par }{\PP \VS{14}到晚上有惊吓,未到早晨他们就没有了。这是掳掠我们之人所得的分,是抢夺我们之人的报应。

\par }\Chap{18}{\SH  神要责罚古实
\par }{\Q \VerseOne{1}唉!{\PN{古实}}河外翅膀刷刷响声之地,
\par }{\Q \VS{2}差遣使者在水面上,
\par }{\Q 坐蒲草船过海。
\par }{\Q {\ADD{先知说}}:你们快行的使者,
\par }{\Q 要到高大光滑的民那里去。
\par }{\Q 自从开国以来,那民极其可畏,
\par }{\Q 是分{\ADD{地界}}践踏{\ADD{人}}的;
\par }{\Q 他们的地有江河分开。
\par }{\BB \par }{\Q \VS{3}世上一切的居民和地上所住的人哪,
\par }{\Q 山上竖立大旗的时候你们要看;
\par }{\Q 吹角的时候你们要听。
\par }{\Q \VS{4}耶和华对我这样说:
\par }{\Q 我要安静,在我的居所观看,
\par }{\Q 如同日光中的清热,
\par }{\Q 又如露水的云雾在收割的热天。
\par }{\Q \VS{5}收割之先,花开已谢,
\par }{\Q 花也成了将熟的葡萄;
\par }{\Q 他必用镰刀削去嫩枝,
\par }{\Q 又砍掉蔓延的枝条,
\par }{\Q \VS{6}都要撇给山间的鸷鸟和地上的野兽。
\par }{\Q 夏天,鸷鸟要宿在其上;
\par }{\Q 冬天,野兽都卧在其中。
\par }{\BB \par }{\Q \VS{7}到那时,这高大光滑的民,
\par }{\Q 就是从开国以来极其可畏、
\par }{\Q 分{\ADD{地界}}践踏{\ADD{人}}的,
\par }{\Q 他们的地有江河分开;
\par }{\Q 他们必将礼物奉给万军之耶和华,
\par }{\Q 就是奉到{\PN{锡安山}}—
\par }{\Q 耶和华安置他名的地方。

\par }\Chap{19}{\SH  神要惩罚埃及
\par }{\PP \VerseOne{1}论{\PN{埃及}}的默示:
\par }{\Q 看哪,耶和华乘驾快云,
\par }{\Q 临到{\PN{埃及}}。
\par }{\Q {\PN{埃及}}的偶像在他面前战兢;
\par }{\Q {\PN{埃及}}人的心在里面消化。
\par }{\Q \VS{2}我必激动{\PN{埃及}}人攻击{\PN{埃及}}人—
\par }{\Q 弟兄攻击弟兄,
\par }{\Q 邻舍攻击邻舍;
\par }{\Q 这城攻击那城,
\par }{\Q 这国攻击那国。
\par }{\Q \VS{3}{\PN{埃及}}人的心神必在里面耗尽;
\par }{\Q 我必败坏他们的谋略。
\par }{\Q 他们必求问偶像和念咒的、
\par }{\Q 交鬼的、行巫术的。
\par }{\Q \VS{4}我必将{\PN{埃及}}人交在残忍主的手中;
\par }{\Q 强暴王必辖制他们。
\par }{\Q 这是主—万军之耶和华说的。
\par }{\BB \par }{\Q \VS{5}海中的水必绝尽,
\par }{\Q 河也消没干涸。
\par }{\Q \VS{6}江河要变臭;
\par }{\Q {\PN{埃及}}的河水都必减少枯干。
\par }{\Q 苇子和芦荻都必衰残;
\par }{\Q \VS{7}靠{\PN{尼罗河}}旁的草田,
\par }{\Q 并沿{\PN{尼罗河}}所种的田,都必枯干。
\par }{\Q {\ADD{庄稼}}被{\ADD{风}}吹去,归于无有。
\par }{\Q \VS{8}打鱼的必哀哭。
\par }{\Q 在{\PN{尼罗河}}一切钓鱼的必悲伤;
\par }{\Q 在水上撒网的必都衰弱。
\par }{\Q \VS{9}用梳好的麻造物的
\par }{\Q 和织白布的都必羞愧;
\par }{\Q \VS{10}国柱必被打碎,
\par }{\Q 所有佣工的,心必愁烦。
\par }{\BB \par }{\Q \VS{11}{\PN{琐安}}的首领极其愚昧;
\par }{\Q 法老大有智慧的谋士,
\par }{\Q 所筹划的成为愚谋。
\par }{\Q 你们怎敢对法老说:
\par }{\Q 我是智慧人的子孙,
\par }{\Q 我是古王的后裔?
\par }{\Q \VS{12}你的智慧人在哪里呢?
\par }{\Q 万军之耶和华向{\PN{埃及}}所定的旨意,
\par }{\Q 他们可以知道,可以告诉你吧!
\par }{\Q \VS{13}{\PN{琐安}}的首领都变为愚昧;
\par }{\Q {\PN{挪弗}}的首领都受了迷惑。
\par }{\Q 当{\PN{埃及}}支派房角石的,
\par }{\Q 使{\PN{埃及}}人走错了路。
\par }{\Q \VS{14}耶和华使乖谬的灵搀入{\PN{埃及}}中间;
\par }{\Q 首领使{\PN{埃及}}一切所做的都有差错,
\par }{\Q 好像醉酒之人呕吐的时候东倒西歪一样。
\par }{\Q \VS{15}{\PN{埃及}}中,无论是头与尾,
\par }{\Q 棕枝与芦苇,所做之工都不成就。
\par }{\PP \VS{16}到那日,{\PN{埃及}}人必像妇人一样,他们必因万军之耶和华在{\PN{埃及}}以上所抡的手,战兢惧怕。
\VS{17}{\PN{犹大}}地必使{\PN{埃及}}惊恐,向谁提起{\PN{犹大}}地,谁就惧怕。这是因万军之耶和华向{\PN{埃及}}所定的旨意。
\par }{\PP \VS{18}当那日,{\PN{埃及}}地必有五城的人说{\PN{迦南}}的方言,又指着万军之耶和华起誓。有一城必称为「灭亡城」。
\par }{\PP \VS{19}当那日,在{\PN{埃及}}地中必有为耶和华筑的一座坛;在{\PN{埃及}}的边界上必有为耶和华{\ADD{立}}的一根柱。
\VS{20}这都要在{\PN{埃及}}地为万军之耶和华作记号和证据。{\PN{埃及}}人因为受人的欺压哀求耶和华,他就差遣一位救主作护卫者,拯救他们。
\VS{21}耶和华必被{\PN{埃及}}人所认识。在那日,{\PN{埃及}}人必认识耶和华,也要献祭物和供物敬拜他,并向耶和华许愿还愿。
\VS{22}耶和华必击打{\PN{埃及}},又击打又医治,{\PN{埃及}}人就归向耶和华。他必应允他们的祷告,医治他们。
\par }{\PP \VS{23}当那日,必有从{\PN{埃及}}通{\PN{亚述}}去的大道。{\PN{亚述}}人要进入{\PN{埃及}},{\PN{埃及}}人也进入{\PN{亚述}};{\PN{埃及}}人要与{\PN{亚述}}人一同敬拜{\ADD{耶和华}}。
\VS{24}当那日,{\PN{以色列}}必与{\PN{埃及}}、{\PN{亚述}}三国一律,使地上的人得福;
\VS{25}因为万军之耶和华赐福给他们,说:「{\PN{埃及}}—我的百姓,{\PN{亚述}}—我手的工作,{\PN{以色列}}—我的产业,都有福了!」

\par }\Chap{20}{\SH 露身赤脚的先知
\par }{\PP \VerseOne{1}{\PN{亚述}}王{\PN{撒珥根}}打发他珥探到{\PN{亚实突}}的那年,他珥探就攻打{\PN{亚实突}},将城攻取。
\VS{2}那时,耶和华晓谕{\PN{亚摩斯}}的儿子{\PN{以赛亚}}说:「你去解掉你腰间的麻布,脱下你脚上的鞋。」{\PN{以赛亚}}就这样做,露身赤脚行走。
\VS{3}耶和华说:「我仆人{\PN{以赛亚}}怎样露身赤脚行走三年,作为关乎{\PN{埃及}}和{\PN{古实}}的预兆奇迹。
\VS{4}照样,{\PN{亚述}}王也必掳去{\PN{埃及}}人,掠去{\PN{古实}}人,无论老少,都露身赤脚,现出下体,使{\PN{埃及}}蒙羞。
\VS{5}{\PN{以色列}}人必因所仰望的{\PN{古实}},所夸耀的{\PN{埃及}},惊惶羞愧。
\par }{\PP \VS{6}「那时,这沿海一带的居民必说:『看哪,我们素所仰望的,就是我们为脱离{\PN{亚述}}王逃往求救的,不过是如此!我们怎能逃脱呢?』」

\par }\Chap{21}{\SH 巴比伦的覆灭
\par }{\PP \VerseOne{1}论海旁旷野的默示:
\par }{\Q 有{\ADD{仇敌}}从旷野,从可怕之地而来,
\par }{\Q 好像南方的旋风,猛然扫过。
\par }{\Q \VS{2}令人凄惨的异象已默示于我。
\par }{\Q 诡诈的行诡诈,毁灭的行毁灭。
\par }{\Q {\PN{以拦}}哪,你要上去!
\par }{\Q {\PN{米底亚}}啊,你要围困!
\par }{\Q {\ADD{主说}}:我使一切叹息止住。
\par }{\Q \VS{3}所以,我满腰疼痛;
\par }{\Q 痛苦将我抓住,
\par }{\Q 好像产难的妇人一样。
\par }{\Q 我疼痛甚至不能听;
\par }{\Q 我惊惶甚至不能看。
\par }{\Q \VS{4}我心慌张,惊恐威吓我。
\par }{\Q 我所羡慕的黄昏,变为我的战兢。
\par }{\Q \VS{5}他们摆设筵席,
\par }{\Q 派人守望,又吃又喝。
\par }{\Q 首领啊,你们起来,
\par }{\Q 用油抹盾牌。
\par }{\Q \VS{6}主对我如此说:
\par }{\Q 你去设立守望的,
\par }{\Q 使他将所看见的述说。
\par }{\Q \VS{7}他看见军队,
\par }{\Q 就是骑马的一对一对地来,
\par }{\Q 又看见驴队,骆驼队,
\par }{\Q 就要侧耳细听。
\par }{\Q \VS{8}他像狮子吼叫,{\ADD{说}}:
\par }{\Q 主啊,我白日常站在望楼上,
\par }{\Q 整夜立在我守望所。
\par }{\Q \VS{9}看哪,有一队军兵骑着马,
\par }{\Q 一对一对地来。
\par }{\Q 他就说:{\PN{巴比伦}}倾倒了!倾倒了!
\par }{\Q 他一切雕刻的神像都打碎于地。
\par }{\Q \VS{10}我被打的禾稼,我场上的谷啊,
\par }{\Q 我从万军之耶和华—
\par }{\Q {\PN{以色列}}的 神那里所听见的,都告诉你们了。
\par }{\SH 论以东的信息
\par }{\Q \VS{11}论{\PN{度玛}}的默示:
\par }{\Q 有人声从{\PN{西珥}}呼问我{\ADD{说}}:
\par }{\Q 守望的啊,夜里如何?
\par }{\Q 守望的啊,夜里如何?
\par }{\Q \VS{12}守望的说:
\par }{\Q 早晨将到,黑夜也来。
\par }{\Q 你们若要问就可以问,
\par }{\Q 可以回头再来。
\par }{\SH 论阿拉伯的信息
\par }{\Q \VS{13}论{\PN{阿拉伯}}的默示:
\par }{\Q {\PN{底但}}结伴的客旅啊,
\par }{\Q 你们必在{\PN{阿拉伯}}的树林中住宿。
\par }{\Q \VS{14}{\PN{提玛}}地的居民拿水来,送给口渴的,
\par }{\Q 拿饼来迎接逃避的。
\par }{\Q \VS{15}因为他们逃避刀剑和出了鞘的刀,
\par }{\Q 并上了弦的弓与刀兵的重灾。
\par }{\PP \VS{16}主对我这样说:「一年之内,照雇工的年数,{\PN{基达}}的一切荣耀必归于无有。
\VS{17}弓箭手所余剩的,就是{\PN{基达}}人的勇士,必然稀少,因为这是耶和华—{\PN{以色列}}的 神说的。」

\par }\Chap{22}{\SH 论耶路撒冷的信息
\par }{\PP \VerseOne{1}论异象谷的默示:
\par }{\Q 有什么事使你这{\ADD{满城的人}}都上房顶呢?
\par }{\Q \VS{2}你这满处呐喊、大有喧哗的城,
\par }{\Q 欢乐的邑啊,
\par }{\Q 你中间被杀的并不是被刀杀,
\par }{\Q 也不是因打仗死亡。
\par }{\Q \VS{3}你所有的官长一同逃跑,
\par }{\Q 都为弓箭手所捆绑。
\par }{\Q 你中间一切被找到的都一同被捆绑;
\par }{\Q 他们本是逃往远方的。
\par }{\Q \VS{4}所以我说:你们转眼不看我,
\par }{\Q 我要痛哭。
\par }{\Q 不要因我众民\FTNT{}{{\FR 22:4: }原文是民女}的毁灭,
\par }{\Q 就竭力安慰我。
\par }{\Q \VS{5}因为主—万军之耶和华使「异象谷」
\par }{\Q 有溃乱、践踏、烦扰的日子。
\par }{\Q 城被攻破,
\par }{\Q 哀声达到山间。
\par }{\Q \VS{6}{\PN{以拦}}带着箭袋,
\par }{\Q 还有坐战车的和马兵;
\par }{\Q {\PN{吉珥}}揭开盾牌。
\par }{\Q \VS{7}你嘉美的谷遍满战车,
\par }{\Q 也有马兵在城门前排列。
\par }{\Q \VS{8}他去掉{\PN{犹大}}的遮盖。
\par }{\PP 那日,你就仰望林库内的军器。
\VS{9}你们看见{\PN{大卫城}}的破口很多,便聚积下池的水,
\VS{10}又数点{\PN{耶路撒冷}}的房屋,将房屋拆毁,修补城墙,
\VS{11}又在两道城墙中间挖一个聚水池可盛旧池的水,却不仰望做这事的主,也不顾念从古定这事的。
\par }{\BB \par }{\Q \VS{12}当那日,主—万军之耶和华叫人哭泣哀号,
\par }{\Q 头上光秃,身披麻布。
\par }{\Q \VS{13}谁知,人倒欢喜快乐,
\par }{\Q 宰牛杀羊,吃肉喝酒,{\ADD{说}}:
\par }{\Q 我们吃喝吧!因为明天要死了。
\par }{\Q \VS{14}万军之耶和华亲自默示我{\ADD{说}}:
\par }{\Q 这罪孽直到你们死,断不得赦免!
\par }{\Q 这是主—万军之耶和华说的。
\par }{\SH 对舍伯那的警告
\par }{\PP \VS{15}主—万军之耶和华这样说:「你去见掌银库的,就是家宰{\PN{舍伯那}},{\ADD{对他说}}:
\VS{16}『你在这里做什么呢?有什么人竟在这里凿坟墓,就是在高处为自己凿坟墓,在磐石中为自己凿出安身之所?
\VS{17}看哪,耶和华必像{\ADD{大有力的}}人,将你紧紧缠裹,竭力抛去。
\VS{18}他必将你滚成一团,{\ADD{抛}}在宽阔之地,好像抛球一样。你这主人家的羞辱,必在那里坐你荣耀的车,也必在那里死亡。
\VS{19}我必赶逐你离开官职;你必从你的原位撤下。』
\par }{\PP \VS{20}「到那日,我必召我仆人{\PN{希勒家}}的儿子{\PN{以利亚敬}}来,
\VS{21}将你的外袍给他穿上,将你的腰带给他系紧,将你的政权交在他手中。他必作{\PN{耶路撒冷}}居民和{\PN{犹大}}家的父。
\VS{22}我必将{\PN{大卫}}家的钥匙放在他肩头上。他开,无人能关;他关,无人能开。
\VS{23}我必将他{\ADD{安稳}},像钉子钉在坚固处;他必作为他父家荣耀的宝座。
\VS{24}他父家所有的荣耀,连儿女带子孙,都挂在他身上,好像一切小器皿,从杯子到酒瓶挂上一样。
\VS{25}万军之耶和华说:当那日,钉在坚固处的钉子必压斜,被砍断落地;挂在其上的重担必被剪断。因为这是耶和华说的。」

\par }\Chap{23}{\SH  神要责罚泰尔的海港
\par }{\PP \VerseOne{1}论{\PN{泰尔}}的默示:
\par }{\Q {\PN{他施}}的船只都要哀号;
\par }{\Q 因为{\PN{泰尔}}变为荒场,
\par }{\Q 甚至没有房屋,没有可进之路。
\par }{\Q 这消息是从{\PN{基提}}地得来的。
\par }{\Q \VS{2}沿海的居民,
\par }{\Q 就是素来靠航海{\PN{西顿}}的商家得丰盛的,
\par }{\Q 你们当静默无言。
\par }{\Q \VS{3}在大水之上,
\par }{\Q {\PN{西曷}}的粮食、
\par }{\Q {\PN{尼罗河}}的庄稼是{\PN{泰尔}}的进项;
\par }{\Q 她作列国的大码头。
\par }{\Q \VS{4}{\PN{西顿}}哪,你当惭愧;
\par }{\Q 因为大海说,
\par }{\Q 就是海中的保障说:
\par }{\Q 我没有劬劳,也没有生产,
\par }{\Q 没有养育男子,也没有抚养童女。
\par }{\Q \VS{5}这风声传到{\PN{埃及}};
\par }{\Q {\PN{埃及}}人为{\PN{泰尔}}的风声极其疼痛。
\par }{\Q \VS{6}{\PN{泰尔}}人哪,你们当过到{\PN{他施}}去;
\par }{\Q 沿海的居民哪,你们都当哀号。
\par }{\Q \VS{7}这是你们欢乐的城,
\par }{\Q 从上古而有的吗?
\par }{\Q 其中的居民往远方寄居。
\par }{\Q \VS{8}{\PN{泰尔}}本是赐冠冕的。
\par }{\Q 她的商家是王子;
\par }{\Q 她的买卖人是世上的尊贵人。
\par }{\Q 遭遇如此是谁定的呢?
\par }{\Q \VS{9}是万军之耶和华所定的!
\par }{\Q 为要污辱一切高傲的荣耀,
\par }{\Q 使地上一切的尊贵人被藐视。
\par }{\Q \VS{10}{\PN{他施}}的民\FTNT{}{{\FR 23:10: }原文是女}哪,
\par }{\Q 可以流行你的地,好像{\PN{尼罗河}};
\par }{\Q 不再有腰带拘紧你。
\par }{\Q \VS{11}耶和华已经向海伸手,
\par }{\Q 震动列国。
\par }{\Q 至于{\PN{迦南}},
\par }{\Q 他已经吩咐拆毁其中的保障。
\par }{\Q \VS{12}他又说:受欺压{\PN{西顿}}的居民\FTNT{}{{\FR 23:12: }原文是处女}哪,
\par }{\Q 你必不得再欢乐。
\par }{\Q 起来!过到{\PN{基提}}去;
\par }{\Q 就是在那里也不得安歇。
\par }{\BB \par }{\PP (
\VS{13}看哪,{\PN{迦勒底}}人之地向来没有这民,这国是{\PN{亚述}}人为住旷野的人所立的。现在他们建筑戍楼,拆毁{\PN{泰尔}}的宫殿,使她成为荒凉。)
\par }{\Q \VS{14}{\PN{他施}}的船只都要哀号,
\par }{\Q 因为你们的保障变为荒场。
\par }{\MM \VS{15}到那时,{\PN{泰尔}}必被忘记七十年,照着一王的年日。七十年后,{\PN{泰尔}}的景况必像妓女所唱的歌:
\par }{\Q \VS{16}你这被忘记的妓女啊,
\par }{\Q 拿琴周流城内,
\par }{\Q 巧弹多唱,使人再想念你。
\par }{\PP \VS{17}七十年后,耶和华必眷顾{\PN{泰尔}},她就仍得利息\FTNT{}{{\FR 23:17: }原文是雇价;下同},与地上的万国交易\FTNT{}{{\FR 23:17: }原文是行淫}。
\VS{18}她的货财和利息要归耶和华为圣,必不积攒存留;因为她的货财必为住在耶和华面前的人所得,使他们吃饱,穿耐久的衣服。

\par }\Chap{24}{\SH 耶和华要责罚世界
\par }{\Q \VerseOne{1}看哪,耶和华使地空虚,变为荒凉;
\par }{\Q 又翻转大地,将居民分散。
\par }{\Q \VS{2}那时百姓怎样,祭司也怎样;
\par }{\Q 仆人怎样,主人也怎样;
\par }{\Q 婢女怎样,主母也怎样;
\par }{\Q 买物的怎样,卖物的也怎样;
\par }{\Q 放债的怎样,借债的也怎样;
\par }{\Q 取利的怎样,出利的也怎样。
\par }{\Q \VS{3}地必全然空虚,尽都荒凉;
\par }{\Q 因为这话是耶和华说的。
\par }{\BB \par }{\Q \VS{4}地上悲哀衰残,
\par }{\Q 世界败落衰残;
\par }{\Q 地上居高位的人也败落了。
\par }{\Q \VS{5}地被其上的居民污秽;
\par }{\Q 因为他们犯了律法,
\par }{\Q 废了律例,背了永约。
\par }{\Q \VS{6}所以,地被咒诅吞灭;
\par }{\Q 住在其上的显为有罪。
\par }{\Q 地上的居民被火焚烧,
\par }{\Q 剩下的人稀少。
\par }{\Q \VS{7}新酒悲哀,葡萄树衰残;
\par }{\Q 心中欢乐的俱都叹息。
\par }{\Q \VS{8}击鼓之乐止息;
\par }{\Q 宴乐人的声音完毕,
\par }{\Q 弹琴之乐也止息了。
\par }{\Q \VS{9}人必不得饮酒唱歌;
\par }{\Q 喝浓酒的,必以为苦。
\par }{\Q \VS{10}荒凉的城拆毁了;
\par }{\Q 各家关门闭户,使人都不得进去。
\par }{\Q \VS{11}在街上因酒有悲叹的声音;
\par }{\Q 一切喜乐变为昏暗;
\par }{\Q 地上的欢乐归于无有。
\par }{\Q \VS{12}城中只有荒凉;
\par }{\Q 城门拆毁净尽。
\par }{\Q \VS{13}在地上的万民中,
\par }{\Q 必像打过的橄榄树,
\par }{\Q 又像已摘的葡萄所剩无几。
\par }{\BB \par }{\Q \VS{14}这些人要高声欢呼;
\par }{\Q 他们为耶和华的威严,从海那里扬起声来。
\par }{\Q \VS{15}因此,你们要在东方荣耀耶和华,
\par }{\Q 在众海岛荣耀耶和华—{\PN{以色列}} 神的名。
\par }{\Q \VS{16}我们听见从地极有人歌唱,{\ADD{说}}:
\par }{\Q 荣耀归于义人。
\par }{\Q 我却说:我消灭了!
\par }{\Q 我消灭了,我有祸了!
\par }{\Q 诡诈的行诡诈;
\par }{\Q 诡诈的大行诡诈。
\par }{\BB \par }{\Q \VS{17}地上的居民哪,
\par }{\Q 恐惧、陷坑、网罗都临近你。
\par }{\Q \VS{18}躲避恐惧声音的必坠入陷坑;
\par }{\Q 从陷坑上来的必被网罗缠住;
\par }{\Q 因为{\ADD{天}}上的窗户都开了,
\par }{\Q 地的根基也震动了。
\par }{\Q \VS{19}地全然破坏,尽都崩裂,
\par }{\Q 大大地震动了。
\par }{\Q \VS{20}地要东倒西歪,好像醉酒的人;
\par }{\Q 又摇来摇去,好像吊床。
\par }{\Q 罪过在其上沉重,
\par }{\Q 必然塌陷,不能复起。
\par }{\BB \par }{\Q \VS{21}到那日,耶和华在高处必惩罚高处的众军,
\par }{\Q 在地上必惩罚地上的列王。
\par }{\Q \VS{22}他们必被聚集,像囚犯被聚在牢狱中,
\par }{\Q 并要囚在监牢里,
\par }{\Q 多日之后便被讨罪\FTNT{}{{\FR 24:22: }或译:眷顾}。
\par }{\Q \VS{23}那时,月亮要蒙羞,日头要惭愧;
\par }{\Q 因为万军之耶和华必在{\PN{锡安山}},
\par }{\Q 在{\PN{耶路撒冷}}作王;
\par }{\Q 在{\ADD{敬畏}}他的长老面前,必有荣耀。

\par }\Chap{25}{\SH 一首赞美诗
\par }{\Q \VerseOne{1}耶和华啊,你是我的 神;
\par }{\Q 我要尊崇你,我要称赞你的名。
\par }{\Q 因为你以忠信诚实行过奇妙的事,
\par }{\Q 成就你古时所定的。
\par }{\Q \VS{2}你使城变为乱堆,
\par }{\Q 使坚固城变为荒场,
\par }{\Q 使外邦人宫殿的城不再为城,
\par }{\Q 永远不再建造。
\par }{\Q \VS{3}所以,刚强的民必荣耀你;
\par }{\Q 强暴之国的城必敬畏你。
\par }{\Q \VS{4}因为当强暴人催逼人的时候,
\par }{\Q 如同暴风直吹墙壁,
\par }{\Q 你就作贫穷人的保障,
\par }{\Q 作困乏人急难中的保障,
\par }{\Q 作躲暴风之处,
\par }{\Q 作避炎热的阴凉。
\par }{\Q \VS{5}你要压制外邦人的喧哗,
\par }{\Q 好像干燥地的热气下落;
\par }{\Q 禁止强暴人的凯歌,
\par }{\Q 好像热气被云影消化。
\par }{\SH  神为万民设摆筵席
\par }{\PP \VS{6}在这山上,万军之耶和华必为万民用肥甘设摆筵席,用陈酒和满髓的肥甘,并澄清的陈酒,设摆筵席。
\VS{7}他又必在这山上除灭遮盖万民之物和遮蔽万国蒙脸的帕子。
\VS{8}他已经吞灭死亡直到永远。主耶和华必擦去各人脸上的眼泪,又除掉普天下他百姓的羞辱,因为这是耶和华说的。
\par }{\PP \VS{9}到那日,人必说:「看哪,这是我们的 神;我们素来等候他,他必拯救我们。这是耶和华,我们素来等候他,我们必因他的救恩欢喜快乐。」
\par }{\SH  神要刑罚摩押
\par }{\PP \VS{10}耶和华的手必按在这山上;{\PN{摩押}}人在所居之地必被践踏,好像干草被践踏在粪池的水中。
\VS{11}他必在其中伸开手,好像洑水的伸开手洑水一样;但{\ADD{耶和华}}必使他的骄傲和他手所行的诡计一并败落。
\VS{12}耶和华使你城上的坚固高台倾倒,拆平,直到尘埃。

\par }\Chap{26}{\SH  神使他的百姓得胜
\par }{\PP \VerseOne{1}当那日,在{\PN{犹大}}地人必唱这歌{\ADD{说}}:
\par }{\Q 我们有坚固的城。
\par }{\Q 耶和华要将救恩定为城墙,为外郭。
\par }{\Q \VS{2}敞开城门,
\par }{\Q 使守信的义民得以进入。
\par }{\Q \VS{3}坚心倚赖{\ADD{你的}},
\par }{\Q 你必保守他十分平安,
\par }{\Q 因为他倚靠你。
\par }{\Q \VS{4}你们当倚靠耶和华直到永远,
\par }{\Q 因为耶和华是永久的磐石。
\par }{\Q \VS{5}他使住高处的与高城一并败落,
\par }{\Q 将城拆毁,拆平,直到尘埃,
\par }{\Q \VS{6}要被脚践踏,
\par }{\Q 就是被困苦人的脚和穷乏人的脚践踏。
\par }{\BB \par }{\Q \VS{7}义人的道是正直的;
\par }{\Q 你为正直的{\ADD{主}},必修平义人的路。
\par }{\Q \VS{8}耶和华啊,我们在你行审判的路上等候你;
\par }{\Q 我们心里所羡慕的是你的名,
\par }{\Q 就是你那可记念的{\ADD{名}}。
\par }{\Q \VS{9}夜间,我心中羡慕你;
\par }{\Q 我里面的灵切切寻求你。
\par }{\Q 因为你在世上行审判的时候,
\par }{\Q 地上的居民就学习公义。
\par }{\Q \VS{10}以恩惠待恶人,
\par }{\Q 他仍不学习公义;
\par }{\Q 在正直的地上,他必行事不义,
\par }{\Q 也不注意耶和华的威严。
\par }{\Q \VS{11}耶和华啊,你的手高举,
\par }{\Q 他们仍然不看;
\par }{\Q 却要看你为百姓发的热心,因而抱愧,
\par }{\Q 并且有火烧灭你的敌人。
\par }{\Q \VS{12}耶和华啊,你必派定我们得平安,
\par }{\Q 因为我们所做的事都是你给我们成就的。
\par }{\Q \VS{13}耶和华—我们的 神啊,
\par }{\Q 在你以外曾有别的主管辖我们,
\par }{\Q 但我们专要倚靠你,提你的名。
\par }{\Q \VS{14}他们死了,必不能再活;
\par }{\Q 他们去世,必不能再起;
\par }{\Q 因为你刑罚他们,毁灭他们,
\par }{\Q 他们的名号就全然消灭。
\par }{\Q \VS{15}耶和华啊,你增添国民,
\par }{\Q 你增添国民;
\par }{\Q 你得了荣耀,
\par }{\Q 又扩张地的四境。
\par }{\BB \par }{\Q \VS{16}耶和华啊,他们在急难中寻求你;
\par }{\Q 你的惩罚临到他们身上,
\par }{\Q 他们就倾心吐胆祷告你。
\par }{\Q \VS{17}妇人怀孕,临产疼痛,
\par }{\Q 在痛苦之中喊叫;
\par }{\Q 耶和华啊,我们在你面前也是如此。
\par }{\Q \VS{18}我们也曾怀孕疼痛,
\par }{\Q 所产的竟像风一样。
\par }{\Q 我们在地上未曾行什么拯救的事;
\par }{\Q 世上的居民也未曾败落。
\par }{\Q \VS{19}死人\FTNT{}{{\FR 26:19: }原文是你的死人}要复活,
\par }{\Q 尸首\FTNT{}{{\FR 26:19: }原文是我的尸首}要兴起。
\par }{\Q 睡在尘埃的啊,要醒起歌唱!
\par }{\Q 因你的甘露{\ADD{好像}}菜蔬上的甘露,
\par }{\Q 地也要交出死人来。
\par }{\SH 审判和复兴
\par }{\Q \VS{20}我的百姓啊,你们要来进入内室,
\par }{\Q 关上门,隐藏片时,
\par }{\Q 等到忿怒过去。
\par }{\Q \VS{21}因为耶和华从他的居所出来,
\par }{\Q 要刑罚地上居民的罪孽。
\par }{\Q 地也必露出其中的血,
\par }{\Q 不再掩盖被杀的人。

\par }\Chap{27}{\PP \VerseOne{1}到那日,耶和华必用他刚硬有力的大刀刑罚鳄鱼—就是那快行的蛇,刑罚鳄鱼—就是那曲行的蛇,并杀海中的大鱼。
\par }{\Q \VS{2}当那日,有出酒的葡萄园,
\par }{\Q 你们要指这园唱歌{\ADD{说}}:
\par }{\Q \VS{3}我—耶和华是看守葡萄园的;
\par }{\Q 我必时刻浇灌,
\par }{\Q 昼夜看守,
\par }{\Q 免得有人损害。
\par }{\Q \VS{4}我心中不存忿怒。
\par }{\Q 惟愿荆棘蒺藜与我交战,
\par }{\Q 我就勇往直前,
\par }{\Q 把它一同焚烧。
\par }{\Q \VS{5}不然,让它持住我的能力,
\par }{\Q 使它与我和好,
\par }{\Q 愿它与我和好。
\par }{\BB \par }{\Q \VS{6}将来{\PN{雅各}}要扎根,
\par }{\Q {\PN{以色列}}要发芽开花;
\par }{\Q 他们的果实必充满世界。
\par }{\Q \VS{7}主击打他们,
\par }{\Q 岂像击打那些击打他们的人吗?
\par }{\Q 他们被杀戮,
\par }{\Q 岂像被他们所杀戮的吗?
\par }{\Q \VS{8}你打发他们去,
\par }{\Q 是相机宜与他们相争;
\par }{\Q 刮东风的日子,
\par }{\Q 就用暴风将他们逐去。
\par }{\Q \VS{9}所以,{\PN{雅各}}的罪孽得赦免,
\par }{\Q 他的罪过得除掉的果效,全在乎此:
\par }{\Q 就是他叫祭坛的石头变为打碎的灰石,
\par }{\Q 以致木偶和日像不再立起。
\par }{\Q \VS{10}因为坚固城变为凄凉,
\par }{\Q 成了撇下离弃的居所,像旷野一样;
\par }{\Q 牛犊必在那里吃草,
\par }{\Q 在那里躺卧,并吃尽其中的树枝。
\par }{\Q \VS{11}枝条枯干,必被折断;
\par }{\Q 妇女要来点火烧着。
\par }{\Q 因为这百姓蒙昧无知,
\par }{\Q 所以,创造他们的必不怜恤他们;
\par }{\Q 造成他们的也不施恩与他们。
\par }{\PP \VS{12}{\PN{以色列}}人哪,到那日,耶和华必从大河,直到{\PN{埃及}}小河,将你们一一地收集,如同人打树拾果一样。
\VS{13}当那日,必大发角声,在{\PN{亚述}}地将要灭亡的,并在{\PN{埃及}}地被赶散的,都要来,他们就在{\PN{耶路撒冷}}圣山上敬拜耶和华。

\par }\Chap{28}{\SH 对北国的警告
\par }{\Q \VerseOne{1}祸哉!{\PN{以法莲}}的酒徒,
\par }{\Q 住在肥美谷的山上,
\par }{\Q 他们心里高傲,
\par }{\Q 以所夸的为冠冕,
\par }{\Q {\ADD{犹如}}将残之花。
\par }{\Q \VS{2}看哪,主有一大能大力者,
\par }{\Q 像一阵冰雹,
\par }{\Q 像毁灭的暴风,
\par }{\Q 像涨溢的大水,
\par }{\Q 他必用手将{\ADD{冠冕}}摔落于地。
\par }{\Q \VS{3}{\PN{以法莲}}高傲的酒徒,
\par }{\Q 他的冠冕必被踏在脚下。
\par }{\Q \VS{4}那荣美将残之花,
\par }{\Q 就是在肥美谷山上的,
\par }{\Q 必像夏令以前初熟的无花果;
\par }{\Q 看见这果的就注意,
\par }{\Q 一到手中就吞吃了。
\par }{\BB \par }{\Q \VS{5}到那日,万军之耶和华
\par }{\Q 必作他余剩之民的荣冠华冕,
\par }{\Q \VS{6}也作了在位上行审判者公平之灵,
\par }{\Q 并城门口打退仇敌者的力量。
\par }{\SH 以赛亚和犹大酒醉的先知
\par }{\Q \VS{7}就是这{\ADD{地的}}人也因酒摇摇晃晃,
\par }{\Q 因浓酒东倒西歪。
\par }{\Q 祭司和先知因浓酒摇摇晃晃,
\par }{\Q 被酒所困,
\par }{\Q 因浓酒东倒西歪。
\par }{\Q 他们错解默示,
\par }{\Q 谬行审判。
\par }{\Q \VS{8}因为各席上满了呕吐的污秽,
\par }{\Q 无一处{\ADD{干净}}。
\par }{\BB \par }{\Q \VS{9}{\ADD{讥诮先知的说}}:
\par }{\Q 他要将知识指教谁呢?
\par }{\Q 要使谁明白传言呢?
\par }{\Q 是那刚断奶离怀的吗?
\par }{\Q \VS{10}他竟命上加命,令上加令,
\par }{\Q 律上加律,例上加例,
\par }{\Q 这里一点,那里一点。
\par }{\BB \par }{\Q \VS{11}{\ADD{先知说}}:不然,主要借异邦人的嘴唇
\par }{\Q 和外邦人的舌头对这百姓说话。
\par }{\Q \VS{12}他曾对他们说:
\par }{\Q 你们要使疲乏人得安息,
\par }{\Q 这样才得安息,才得舒畅,
\par }{\Q 他们却不肯听。
\par }{\Q \VS{13}所以,耶和华向他们说的话是
\par }{\Q 命上加命,令上加令,
\par }{\Q 律上加律,例上加例,
\par }{\Q 这里一点,那里一点,
\par }{\Q 以致他们前行仰面跌倒,
\par }{\Q 而且跌碎,并陷入网罗被缠住。
\par }{\SH 锡安的房角石
\par }{\Q \VS{14}所以,你们这些亵慢的人,
\par }{\Q 就是辖管住在{\PN{耶路撒冷}}这百姓的,
\par }{\Q 要听耶和华的话。
\par }{\Q \VS{15}你们曾说:
\par }{\Q 我们与死亡立约,
\par }{\Q 与阴间结盟;
\par }{\Q 敌军\FTNT{}{{\FR 28:15: }原文是鞭子}{\ADD{如水}}涨漫经过的时候,
\par }{\Q 必不临到我们;
\par }{\Q 因我们以谎言为避所,
\par }{\Q 在虚假以下藏身。
\par }{\Q \VS{16}所以,主耶和华如此说:
\par }{\Q 看哪,我在{\PN{锡安}}放一块石头作为根基,
\par }{\Q 是试验过的石头,
\par }{\Q 是稳固根基,宝贵的房角{\ADD{石}};
\par }{\Q 信靠的人必不着急。
\par }{\Q \VS{17}我必以公平为准绳,
\par }{\Q 以公义为线铊。
\par }{\Q 冰雹必冲去谎言的避所;
\par }{\Q 大水必漫过藏身之处。
\par }{\Q \VS{18}你们与死亡所立的约必然废掉,
\par }{\Q 与阴间所结的盟必立不住。
\par }{\Q 敌军\FTNT{}{{\FR 28:18: }原文是鞭子}{\ADD{如水}}涨漫经过的时候,
\par }{\Q 你们必被他践踏;
\par }{\Q \VS{19}每逢经过必将你们掳去。
\par }{\Q 因为每早晨他必经过,
\par }{\Q 白昼黑夜都必如此。
\par }{\Q 明白传言的必受惊恐。
\par }{\Q \VS{20}原来,床榻短,使人不能舒身;
\par }{\Q 被窝窄,使人不能遮体。
\par }{\Q \VS{21}耶和华必兴起,像在{\PN{毗拉心山}};
\par }{\Q 他必发怒,像在{\PN{基遍谷}},
\par }{\Q 好做成他的工,就是非常的工;
\par }{\Q 成就他的事,就是奇异的事。
\par }{\Q \VS{22}现在你们不可亵慢,
\par }{\Q 恐怕捆你们的绑索更结实了;
\par }{\Q 因为我从主—万军之耶和华那里听见,
\par }{\Q 已经决定在全地上施行灭绝的事。
\par }{\SH  神的智慧
\par }{\Q \VS{23}你们当侧耳听我的声音,
\par }{\Q 留心听我的言语。
\par }{\Q \VS{24}那耕地为要撒种的,
\par }{\Q 岂是常常耕地呢?
\par }{\Q 岂是常常开垦耙地呢?
\par }{\Q \VS{25}他拉平了地面,
\par }{\Q 岂不就撒种小茴香,
\par }{\Q 播种大茴香,
\par }{\Q 按行列种小麦,
\par }{\Q 在定处种大麦,
\par }{\Q 在田边种粗麦呢?
\par }{\Q \VS{26}因为他的 神教导他务农相宜,
\par }{\Q 并且指教他。
\par }{\BB \par }{\Q \VS{27}原来打小茴香,不用尖利的器具,
\par }{\Q 轧大茴香,也不用碌碡\FTNT{}{{\FR 28:27: }原文是车轮;下同};
\par }{\Q 但用杖打小茴香,
\par }{\Q 用棍打大茴香。
\par }{\Q \VS{28}做饼的{\ADD{粮食}}是用磨磨碎,
\par }{\Q 因它不必常打;
\par }{\Q 虽用碌碡和马打散,
\par }{\Q 却不磨它。
\par }{\Q \VS{29}这也是出于万军之耶和华—
\par }{\Q 他的谋略奇妙;
\par }{\Q 他的智慧广大。

\par }\Chap{29}{\SH 耶路撒冷的命运
\par }{\Q \VerseOne{1}唉!{\PN{亚利伊勒}},{\PN{亚利伊勒}},
\par }{\Q {\PN{大卫}}安营的城,
\par }{\Q 任凭你年上加年,
\par }{\Q 节期照常周流。
\par }{\Q \VS{2}我终必使{\PN{亚利伊勒}}困难;
\par }{\Q 她必悲伤哀号,
\par }{\Q 我却仍以她为{\PN{亚利伊勒}}。
\par }{\Q \VS{3}我必四围安营攻击你,
\par }{\Q 屯兵围困你,
\par }{\Q 筑垒攻击你。
\par }{\Q \VS{4}你必败落,从地中说话;
\par }{\Q 你的言语必微细出于尘埃。
\par }{\Q 你的声音必像那交鬼者的声音出于地;
\par }{\Q 你的言语低低微微出于尘埃。
\par }{\Q \VS{5}你仇敌的群众,却要像细尘;
\par }{\Q 强暴人的群众,也要像飞糠。
\par }{\Q 这事必顷刻之间忽然临到。
\par }{\Q \VS{6}万军之耶和华必用雷轰、地震、大声、旋风、暴风,
\par }{\Q 并吞灭的火焰,向她讨罪。
\par }{\Q \VS{7}那时,攻击{\PN{亚利伊勒}}列国的群众,
\par }{\Q 就是一切攻击{\PN{亚利伊勒}}和她的保障,
\par }{\Q 并使她困难的,
\par }{\Q 必如梦景,如夜间的异象;
\par }{\Q \VS{8}又必像饥饿的人梦中吃饭,
\par }{\Q 醒了仍觉腹空;
\par }{\Q 或像口渴的人梦中喝水,
\par }{\Q 醒了仍觉发昏,心里想喝。
\par }{\Q 攻击{\PN{锡安山}}列国的群众也必如此。
\par }{\SH 没有人听警告
\par }{\Q \VS{9}你们等候惊奇吧!
\par }{\Q 你们宴乐昏迷吧!
\par }{\Q 他们醉了,却非因酒;
\par }{\Q 他们东倒西歪,却非因浓酒。
\par }{\Q \VS{10}因为耶和华将沉睡的灵浇灌你们,
\par }{\Q 封闭你们的眼,
\par }{\Q 蒙盖你们的头。
\par }{\Q 你们的眼就是先知;
\par }{\Q 你们的头就是先见。
\par }{\PP \VS{11}所有的默示,你们看如封住的书卷,人将这书卷交给识字的,说:「请念吧!」他说:「我不能念,因为是封住了。」
\VS{12}又将这书卷交给不识字的人,说:「请念吧!」他说:「我不识字。」
\par }{\Q \VS{13}主说:因为这百姓亲近我,
\par }{\Q 用嘴唇尊敬我,
\par }{\Q 心却远离我;
\par }{\Q 他们敬畏我,
\par }{\Q 不过是领受人的吩咐。
\par }{\Q \VS{14}所以,我在这百姓中要行奇妙的事,
\par }{\Q 就是奇妙又奇妙的事。
\par }{\Q 他们智慧人的智慧必然消灭,
\par }{\Q 聪明人的聪明必然隐藏。
\par }{\SH 将来的希望
\par }{\Q \VS{15}祸哉!那些向耶和华深藏谋略的,
\par }{\Q 又在暗中行事,说:
\par }{\Q 谁看见我们呢?
\par }{\Q 谁知道我们呢?
\par }{\Q \VS{16}你们把事颠倒了,
\par }{\Q 岂可看窑匠如泥吗?
\par }{\Q 被制作的物岂可论制作物的说:
\par }{\Q 他没有制作我?
\par }{\Q 或是被创造的物论造物的说:
\par }{\Q 他没有聪明?
\par }{\BB \par }{\Q \VS{17}{\PN{黎巴嫩}}变为肥田,
\par }{\Q 肥田看如树林,
\par }{\Q 不是只有一点点时候吗?
\par }{\Q \VS{18}那时,聋子必听见这书上的话;
\par }{\Q 瞎子的眼必从迷蒙黑暗中得以看见。
\par }{\Q \VS{19}谦卑人必因耶和华增添欢喜;
\par }{\Q 人间贫穷的必因{\PN{以色列}}的圣者快乐。
\par }{\Q \VS{20}因为,强暴人已归无有,
\par }{\Q 亵慢人已经灭绝,
\par }{\Q 一切找机会作孽的都被剪除。
\par }{\Q \VS{21}他们在争讼的事上定{\ADD{无罪的}}为有罪,
\par }{\Q 为城门口责备人的设下网罗,
\par }{\Q 用虚无的事屈枉义人。
\par }{\BB \par }{\Q \VS{22}所以,救赎{\PN{亚伯拉罕}}的耶和华
\par }{\Q 论{\PN{雅各}}家如此说:
\par }{\Q {\PN{雅各}}必不再羞愧,
\par }{\Q 面容也不致变色。
\par }{\Q \VS{23}但他看见他的众子,
\par }{\Q 就是我手的工作在他那里,
\par }{\Q 他们必尊我的名为圣,
\par }{\Q 必尊{\PN{雅各}}的圣者为圣,
\par }{\Q 必敬畏{\PN{以色列}}的 神。
\par }{\Q \VS{24}心中迷糊的必得明白;
\par }{\Q 发怨言的必受训诲。

\par }\Chap{30}{\SH 跟埃及订立无用的条约
\par }{\Q \VerseOne{1}耶和华说:
\par }{\Q 祸哉!这悖逆的儿女。
\par }{\Q 他们同谋,却不由于我,
\par }{\Q 结盟,却不由于我的灵,
\par }{\Q 以致罪上加罪;
\par }{\Q \VS{2}起身下{\PN{埃及}}去,并没有求问我;
\par }{\Q 要靠法老的力量加添自己的力量,
\par }{\Q 并投在{\PN{埃及}}的荫下。
\par }{\Q \VS{3}所以,法老的力量必作你们的羞辱;
\par }{\Q 投在{\PN{埃及}}的荫下,要为你们的惭愧。
\par }{\Q \VS{4}他们的首领已在{\PN{琐安}};
\par }{\Q 他们的使臣到了{\PN{哈内斯}}。
\par }{\Q \VS{5}他们必因那不利于他们的民蒙羞。
\par }{\Q 那民并非帮助,也非利益,
\par }{\Q 只作羞耻凌辱。
\par }{\Q \VS{6}论南方牲畜的默示:
\par }{\Q 他们把财物驮在驴驹的脊背上,
\par }{\Q 将宝物驮在骆驼的肉鞍上,
\par }{\Q 经过艰难困苦之地,
\par }{\Q 就是公狮、母狮、蝮蛇、火焰的飞龙之地,
\par }{\Q 往那不利于他们的民那里去。
\par }{\Q \VS{7}{\PN{埃及}}的帮助是徒然无益的;
\par }{\Q 所以我称她为「坐而不动的{\PN{拉哈伯}}」。
\par }{\SH 不顺服的百姓
\par }{\Q \VS{8}现今你去,
\par }{\Q 在他们面前将这话刻在版上,
\par }{\Q 写在书上,
\par }{\Q 以便传留后世,直到永永远远。
\par }{\Q \VS{9}因为他们是悖逆的百姓、说谎的儿女,
\par }{\Q 不肯听从耶和华训诲的儿女。
\par }{\Q \VS{10}他们对先见说:不要望见{\ADD{不吉利的事}},
\par }{\Q 对先知说:不要向我们讲正直的话;
\par }{\Q 要向我们说柔和的话,
\par }{\Q 言虚幻的事。
\par }{\Q \VS{11}你们要离弃正道,偏离直路,
\par }{\Q 不要在我们面前再提说{\PN{以色列}}的圣者。
\par }{\Q \VS{12}所以,{\PN{以色列}}的圣者如此说:
\par }{\Q 因为你们藐视这{\ADD{训诲的}}话,
\par }{\Q 倚赖欺压和乖僻,以此为可靠的,
\par }{\Q \VS{13}故此,这罪孽在你们身上,
\par }{\Q 好像将要破裂凸出来的高墙,
\par }{\Q 顷刻之间忽然坍塌;
\par }{\Q \VS{14}要被打碎,好像把窑匠的瓦器打碎,
\par }{\Q 毫不顾惜,
\par }{\Q 甚至碎块中找不到一片可用以从炉内取火,
\par }{\Q 从池中舀水。
\par }{\BB \par }{\Q \VS{15}主耶和华—{\PN{以色列}}的圣者曾如此说:
\par }{\Q 你们得救在乎归回安息;
\par }{\Q 你们得力在乎平静安稳;
\par }{\Q 你们竟自不肯。
\par }{\Q \VS{16}你们却说:不然,我们要骑马奔走。
\par }{\Q 所以你们必然奔走;
\par }{\Q 又说:我们要骑飞快的{\ADD{牲口}}。
\par }{\Q 所以追赶你们的,也必飞快。
\par }{\Q \VS{17}一人叱喝,必令千人{\ADD{逃跑}};
\par }{\Q 五人叱喝,你们都必逃跑;
\par }{\Q 以致剩下的,好像山顶的旗杆,
\par }{\Q 冈上的大旗。
\par }{\Q \VS{18}耶和华必然等候,要施恩给你们;
\par }{\Q 必然兴起,好怜悯你们。
\par }{\Q 因为耶和华是公平的 神;
\par }{\Q 凡等候他的都是有福的!
\par }{\SH  神要赐福给他的子民
\par }{\PP \VS{19}百姓必在{\PN{锡安}}、在{\PN{耶路撒冷}}居住;你不再哭泣。主必因你哀求的声音施恩给你;他听见的时候就必应允你。
\VS{20}主虽然以艰难给你当饼,以困苦给你当水,你的教师却不再隐藏;你眼必看见你的教师。
\VS{21}你或向左或向右,你必听见后边有声音说:「这是正路,要行在其间。」
\VS{22}你雕刻偶像所包的银子和铸造偶像所镀的金子,你要玷污,要抛弃,好像污秽之物,对偶像说:「去吧!」
\par }{\PP \VS{23}你将种子撒在地里,主必降雨在其上,并使地所出的粮肥美丰盛。到那时,你的牲畜必在宽阔的草场吃草。
\VS{24}耕地的牛和驴驹必吃加盐的料;这料是用木杴和杈子扬净的。
\VS{25}在大行杀戮的日子,高台倒塌的时候,各高山冈陵必有川流河涌。
\VS{26}当耶和华缠裹他百姓的损处,医治他民鞭伤的日子,月光必像日光,日光必加七倍,像七日的光一样。
\par }{\SH  神要惩罚亚述
\par }{\Q \VS{27}看哪,耶和华的名从远方来,
\par }{\Q 怒气烧起,密烟上腾。
\par }{\Q 他的嘴唇满有忿恨;
\par }{\Q 他的舌头像吞灭的火。
\par }{\Q \VS{28}他的气如涨溢的河水,直涨到颈项,
\par }{\Q 要用毁灭的筛箩筛净列国,
\par }{\Q 并且在众民的口中必有使人错行的嚼环。
\par }{\PP \VS{29}你们必唱歌,像守圣节的夜间一样,并且心中喜乐,像人吹笛,上耶和华的山,到{\PN{以色列}}的磐石那里。
\VS{30}耶和华必使人听他威严的声音,又显他降罚的膀臂和他怒中的忿恨,并吞灭的火焰与霹雷、暴风、冰雹。
\VS{31}{\PN{亚述}}人必因耶和华的声音惊惶;耶和华必用杖击打他。
\VS{32}耶和华必将命定的杖加在他身上;每打一下,人必击鼓弹琴。打杖的时候,耶和华必抡起{\ADD{手}}来,与他交战。
\VS{33}原来{\PN{陀斐特}}又深又宽,早已为王预备好了;其中堆的是火与许多木柴。耶和华的气如一股硫磺火使他着起来。

\par }\Chap{31}{\SH  神要保护耶路撒冷
\par }{\Q \VerseOne{1}祸哉!那些下{\PN{埃及}}求帮助的,
\par }{\Q 是因仗赖马匹,倚靠甚多的车辆,
\par }{\Q 并倚靠强壮的马兵,
\par }{\Q 却不仰望{\PN{以色列}}的圣者,
\par }{\Q 也不求问耶和华。
\par }{\Q \VS{2}其实,耶和华有智慧;
\par }{\Q 他必降灾祸,
\par }{\Q 并不反悔自己的话,
\par }{\Q 却要兴起攻击那作恶之家,
\par }{\Q 又攻击那作孽帮助人的。
\par }{\Q \VS{3}{\PN{埃及}}人不过是人,并不是 神;
\par }{\Q 他们的马不过是血肉,并不是灵。
\par }{\Q 耶和华一伸手,那帮助人的必绊跌,
\par }{\Q 那受帮助的也必跌倒,都一同灭亡。
\par }{\BB \par }{\Q \VS{4}耶和华对我如此说:
\par }{\Q 狮子和少壮狮子护食咆哮,
\par }{\Q 就是喊许多牧人来攻击它,
\par }{\Q 它总不因他们的声音惊惶,
\par }{\Q 也不因他们的喧哗缩伏。
\par }{\Q 如此,万军之耶和华
\par }{\Q 也必降临在{\PN{锡安山}}冈上争战。
\par }{\Q \VS{5}雀鸟怎样搧翅{\ADD{覆雏}},
\par }{\Q 万军之耶和华也要照样保护{\PN{耶路撒冷}}。
\par }{\Q 他必保护拯救,
\par }{\Q 要越门保守。
\par }{\PP \VS{6}{\PN{以色列}}人哪,你们深深地悖逆耶和华,现今要归向他。
\VS{7}到那日,各人必将他金偶像银偶像,就是亲手所造、陷自己在罪中的,都抛弃了。
\par }{\Q \VS{8}{\PN{亚述}}人必倒在刀下,并非人的刀;
\par }{\Q 有刀要将他吞灭,并非人的刀。
\par }{\Q 他必逃避这刀;
\par }{\Q 他的少年人必成为服苦的。
\par }{\Q \VS{9}他的磐石必因惊吓挪去;
\par }{\Q 他的首领必因大旗惊惶。
\par }{\Q 这是那有火在{\PN{锡安}}、
\par }{\Q 有炉在{\PN{耶路撒冷}}的耶和华说的。

\par }\Chap{32}{\SH 公正的王
\par }{\Q \VerseOne{1}看哪,必有一王凭公义行政;
\par }{\Q 必有首领借公平掌权。
\par }{\Q \VS{2}必有一人像避风所和避暴雨的隐密处,
\par }{\Q 又像河流在干旱之地,
\par }{\Q 像大磐石的影子在疲乏之地。
\par }{\Q \VS{3}那能看的人,眼不再昏迷;
\par }{\Q 能听的人,耳必得听闻。
\par }{\Q \VS{4}冒失人的心必明白知识;
\par }{\Q 结巴人的舌必说话通快。
\par }{\Q \VS{5}愚顽人不再称为高明;
\par }{\Q 吝啬人不再称为大方。
\par }{\Q \VS{6}因为愚顽人必说愚顽话,
\par }{\Q 心里想作罪孽,
\par }{\Q 惯行亵渎的事,
\par }{\Q 说错谬的话攻击耶和华,
\par }{\Q 使饥饿的人无食可吃,
\par }{\Q 使口渴的人无水可喝。
\par }{\Q \VS{7}吝啬人所用的法子是恶的;
\par }{\Q 他图谋恶计,
\par }{\Q 用谎言毁灭谦卑人;
\par }{\Q 穷乏人讲公理的时候,
\par }{\Q 他也是这样行。
\par }{\Q \VS{8}高明人却谋高明事,
\par }{\Q 在高明事上也必永存。
\par }{\SH 审判和复兴
\par }{\Q \VS{9}安逸的妇女啊,起来听我的声音!
\par }{\Q 无虑的女子啊,侧耳听我的言语!
\par }{\Q \VS{10}无虑的女子啊,再过一年多,必受骚扰;
\par }{\Q 因为无葡萄可摘,
\par }{\Q 无果子\FTNT{}{{\FR 32:10: }或译:禾稼}可收。
\par }{\Q \VS{11}安逸的妇女啊,要战兢;
\par }{\Q 无虑的女子啊,要受骚扰。
\par }{\Q 脱去衣服,赤着身体,
\par }{\Q 腰束{\ADD{麻布}}。
\par }{\Q \VS{12}她们必为美好的田地
\par }{\Q 和多结果的葡萄树,捶胸哀哭。
\par }{\Q \VS{13}荆棘蒺藜必长在我百姓的地上,
\par }{\Q 又长在欢乐的城中和一切快乐的房屋上。
\par }{\Q \VS{14}因为宫殿必被撇下,
\par }{\Q 多民的城必被离弃;
\par }{\Q 山冈望楼永为洞穴,
\par }{\Q 作野驴所喜乐的,
\par }{\Q 为羊群的草场。
\par }{\Q \VS{15}等到{\ADD{圣}}灵从上浇灌我们,
\par }{\Q 旷野就变为肥田,
\par }{\Q 肥田看如树林。
\par }{\Q \VS{16}那时,公平要居在旷野;
\par }{\Q 公义要居在肥田。
\par }{\Q \VS{17}公义的果效必是平安;
\par }{\Q 公义的效验必是平稳,直到永远。
\par }{\Q \VS{18}我的百姓必住在平安的居所,
\par }{\Q 安稳的住处,平静的安歇所。
\par }{\Q \VS{19}(但要降冰雹打倒树林;
\par }{\Q 城必全然拆平。)
\par }{\Q \VS{20}你们在各水边撒种、
\par }{\Q 牧放牛驴的有福了!

\par }\Chap{33}{\SH 求救的祈祷
\par }{\Q \VerseOne{1}祸哉!你这毁灭人的,
\par }{\Q 自己倒不被毁灭;
\par }{\Q 行事诡诈的,
\par }{\Q 人倒不以诡诈待你。
\par }{\Q 你毁灭罢休了,
\par }{\Q 自己必被毁灭;
\par }{\Q 你行完了诡诈,
\par }{\Q 人必以诡诈待你。
\par }{\BB \par }{\Q \VS{2}耶和华啊,求你施恩于我们;
\par }{\Q 我们等候你。
\par }{\Q 求你每早晨作我们的膀臂,
\par }{\Q 遭难的时候为我们的拯救。
\par }{\Q \VS{3}喧嚷的响声一发,众民奔逃;
\par }{\Q 你一兴起,列国四散。
\par }{\Q \VS{4}你们所掳的必被敛尽,
\par }{\Q 好像蚂蚱吃\FTNT{}{{\FR 33:4: }原文是敛}尽{\ADD{禾稼}}。
\par }{\Q 人要蹦在其上,好像蝗虫一样。
\par }{\BB \par }{\Q \VS{5}耶和华被尊崇,因他居在高处;
\par }{\Q 他以公平公义充满{\PN{锡安}}。
\par }{\Q \VS{6}你一生一世必得安稳—
\par }{\Q 有丰盛的救恩,
\par }{\Q 并智慧和知识;
\par }{\Q 你以敬畏耶和华为至宝。
\par }{\BB \par }{\Q \VS{7}看哪,他们的豪杰在外头哀号;
\par }{\Q 求和的使臣痛痛哭泣。
\par }{\Q \VS{8}大路荒凉,行人止息;
\par }{\Q {\ADD{敌人}}背约,
\par }{\Q 藐视城邑,
\par }{\Q 不顾人民。
\par }{\Q \VS{9}地上悲哀衰残;
\par }{\Q {\PN{黎巴嫩}}羞愧枯干;
\par }{\Q {\PN{沙
}}像旷野;
\par }{\Q {\PN{巴珊}}和{\PN{迦密}}{\ADD{的树林}}凋残。
\par }{\SH 耶和华对敌人的警告
\par }{\Q \VS{10}耶和华说:
\par }{\Q 现在我要起来;
\par }{\Q 我要兴起;
\par }{\Q 我要勃然而兴。
\par }{\Q \VS{11}你们要怀的是糠秕,
\par }{\Q 要生的是碎秸;
\par }{\Q 你们的气就是吞灭自己的火。
\par }{\Q \VS{12}列邦必像已烧的石灰,
\par }{\Q 像已割的荆棘在火中焚烧。
\par }{\Q \VS{13}你们远方的人当听我所行的;
\par }{\Q 你们近处的人当承认我的大能。
\par }{\Q \VS{14}{\PN{锡安}}中的罪人都惧怕;
\par }{\Q 不敬虔的人被战兢抓住。
\par }{\Q 我们中间谁能与吞灭的火同住?
\par }{\Q 我们中间谁能与永火同住呢?
\par }{\Q \VS{15}行事公义、说话正直、
\par }{\Q 憎恶欺压的财利、
\par }{\Q 摆手不受贿赂、
\par }{\Q 塞耳不听流血的话,
\par }{\Q 闭眼不看邪恶事的,
\par }{\Q \VS{16}他必居高处;
\par }{\Q 他的保障是磐石的坚垒;
\par }{\Q 他的粮必不缺乏\FTNT{}{{\FR 33:16: }原文是赐给};
\par }{\Q 他的水必不断绝。
\par }{\SH 光荣的将来
\par }{\Q \VS{17}你的眼必见王的荣美,
\par }{\Q 必见辽阔之地。
\par }{\Q \VS{18}你的心必思想那惊吓的事,
\par }{\Q {\ADD{自问说}}:记数目的在哪里呢?
\par }{\Q 平{\ADD{贡银}}的在哪里呢?
\par }{\Q 数戍楼的在哪里呢?
\par }{\Q \VS{19}你必不见那强暴的民,
\par }{\Q 就是说话深奥,你不能明白,
\par }{\Q 言语呢喃,你不能懂得的。
\par }{\Q \VS{20}你要看{\PN{锡安}}—我们守圣节的城!
\par }{\Q 你的眼必见{\PN{耶路撒冷}}为安静的居所,
\par }{\Q 为不挪移的帐幕,
\par }{\Q 橛子永不拔出,
\par }{\Q 绳索一根也不折断。
\par }{\Q \VS{21}在那里,耶和华必显威严与我们同在,
\par }{\Q 当作江河宽阔之地;
\par }{\Q 其中必没有荡桨摇橹的船来往,
\par }{\Q 也没有威武的船经过。
\par }{\Q \VS{22}因为,耶和华是审判我们的;
\par }{\Q 耶和华是给我们设律法的;
\par }{\Q 耶和华是我们的王;
\par }{\Q 他必拯救我们。
\par }{\BB \par }{\Q \VS{23}你的绳索松开:
\par }{\Q 不能栽稳桅杆,
\par }{\Q 也不能扬起篷来。
\par }{\Q 那时许多掳来的物被分了;
\par }{\Q 瘸腿的把掠物夺去了。
\par }{\Q \VS{24}城内居民必不说:我病了;
\par }{\Q 其中居住的百姓,罪孽都赦免了。

\par }\Chap{34}{\SH  神要惩罚仇敌
\par }{\Q \VerseOne{1}列国啊,要近前来听!
\par }{\Q 众民哪,要侧耳而听!
\par }{\Q 地和其上所充满的,
\par }{\Q 世界和其中一切所出的都应当听!
\par }{\Q \VS{2}因为耶和华向万国发忿恨,
\par }{\Q 向他们的全军发烈怒,
\par }{\Q 将他们灭尽,交出他们受杀戮。
\par }{\Q \VS{3}被杀的必然抛弃,
\par }{\Q 尸首臭气上腾;
\par }{\Q 诸山被他们的血融化。
\par }{\Q \VS{4}天上的万象都要消没;
\par }{\Q 天被卷起,好像书卷。
\par }{\Q 其上的万象要残败,
\par }{\Q 像葡萄树的叶子残败,
\par }{\Q 又像无花果树的{\ADD{叶子}}残败一样。
\par }{\Q \VS{5}因为我的刀在天上已经喝足;
\par }{\Q 这刀必临到{\PN{以东}}和我所咒诅的民,
\par }{\Q 要施行审判。
\par }{\Q \VS{6}耶和华的刀满了血,
\par }{\Q 用脂油和羊羔、公山羊的血,
\par }{\Q 并公绵羊腰子的脂油滋润的;
\par }{\Q 因为耶和华在{\PN{波斯拉}}有献祭的事,
\par }{\Q 在{\PN{以东}}地大行杀戮。
\par }{\Q \VS{7}野牛、牛犊,和公牛要一同下来。
\par }{\Q 他们的地喝醉了血;
\par }{\Q 他们的尘土因脂油肥润。
\par }{\Q \VS{8}因耶和华有报仇之日,
\par }{\Q 为{\PN{锡安}}的争辩有报应之年。
\par }{\Q \VS{9}{\PN{以东}}的河水要变为石油,
\par }{\Q 尘埃要变为硫磺;
\par }{\Q 地土成为烧着的石油,
\par }{\Q \VS{10}昼夜总不熄灭,
\par }{\Q 烟气永远上腾,
\par }{\Q 必世世代代成为荒废,
\par }{\Q 永永远远无人经过。
\par }{\Q \VS{11}鹈鹕、箭猪却要得为业;
\par }{\Q 猫头鹰、乌鸦要住在其间。
\par }{\Q 耶和华必将空虚的准绳,
\par }{\Q 混沌的线铊,拉在其上。
\par }{\Q \VS{12}{\PN{以东}}人要召贵胄来治国;
\par }{\Q 那里却无一个,
\par }{\Q 首领也都归于无有。
\par }{\BB \par }{\Q \VS{13}{\PN{以东}}的宫殿要长荆棘;
\par }{\Q 保障要长蒺藜和刺草;
\par }{\Q 要作野狗的住处,
\par }{\Q 鸵鸟的居所。
\par }{\Q \VS{14}旷野的走兽要和豺狼相遇;
\par }{\Q 野山羊要与伴偶对叫。
\par }{\Q 夜间的怪物必在那里栖身,
\par }{\Q 自找安歇之处。
\par }{\Q \VS{15}箭蛇要在那里做窝,
\par }{\Q 下蛋,抱蛋,生子,
\par }{\Q 聚子在其影下;
\par }{\Q 鹞鹰各与伴偶聚集在那里。
\par }{\BB \par }{\Q \VS{16}你们要查考宣读耶和华的书。
\par }{\Q 这都无一缺少,
\par }{\Q 无一没有伴偶;
\par }{\Q 因为我的口已经吩咐,
\par }{\Q 他的灵将它们聚集。
\par }{\Q \VS{17}他也为它们拈阄,
\par }{\Q 又亲手用准绳给它们分地;
\par }{\Q 它们必永得为业,
\par }{\Q 世世代代住在其间。

\par }\Chap{35}{\SH 神圣之路
\par }{\Q \VerseOne{1}旷野和干旱之地必然欢喜;
\par }{\Q 沙漠也必快乐;
\par }{\Q 又像玫瑰开花,
\par }{\Q \VS{2}必开花繁盛,
\par }{\Q 乐上加乐,而且欢呼。
\par }{\Q {\PN{黎巴嫩}}的荣耀,
\par }{\Q 并{\PN{迦密}}与{\PN{沙
}}的华美,必赐给它。
\par }{\Q 人必看见耶和华的荣耀,
\par }{\Q 我们 神的华美。
\par }{\BB \par }{\Q \VS{3}你们要使软弱的手坚壮,
\par }{\Q 无力的膝稳固。
\par }{\Q \VS{4}对胆怯的人说:
\par }{\Q 你们要刚强,不要惧怕。
\par }{\Q 看哪,你们的 神必来报仇,
\par }{\Q 必来施行极大的报应;
\par }{\Q 他必来拯救你们。
\par }{\BB \par }{\Q \VS{5}那时,瞎子的眼必睁开;
\par }{\Q 聋子的耳必开通。
\par }{\Q \VS{6}那时,瘸子必跳跃像鹿;
\par }{\Q 哑巴的舌头必能歌唱。
\par }{\Q 在旷野必有水发出;
\par }{\Q 在沙漠必有河涌流。
\par }{\Q \VS{7}发光的沙\FTNT{}{{\FR 35:7: }或译:蜃楼}要变为水池;
\par }{\Q 干渴之地要变为泉源。
\par }{\Q 在野狗躺卧之处,
\par }{\Q 必有青草、芦苇,和蒲草。
\par }{\BB \par }{\Q \VS{8}在那里必有一条大道,
\par }{\Q 称为圣路。
\par }{\Q 污秽人不得经过,
\par }{\Q 必专为赎民行走;
\par }{\Q 行路的人虽愚昧,
\par }{\Q 也不致失迷。
\par }{\Q \VS{9}在那里必没有狮子,
\par }{\Q 猛兽也不登这路;
\par }{\Q 在那里都遇不见,
\par }{\Q 只有赎民在那里行走。
\par }{\Q \VS{10}并且耶和华救赎的民必归回,
\par }{\Q 歌唱来到{\PN{锡安}};
\par }{\Q 永乐必归到他们的头上;
\par }{\Q 他们必得着欢喜快乐,
\par }{\Q 忧愁叹息尽都逃避。

\par }\Chap{36}{\SH 亚述人威胁耶路撒冷
\par }{\R (王下18·13—37;代下32·1—19)
\par }{\PP \VerseOne{1}{\PN{希西家}}王十四年,{\PN{亚述}}王{\PN{西拿基立}}上来攻击{\PN{犹大}}的一切坚固城,将城攻取。
\VS{2}{\PN{亚述}}王从{\PN{拉吉}}差遣拉伯沙基率领大军往{\PN{耶路撒冷}},到{\PN{希西家}}王那里去。他就站在上池的水沟旁,在漂布地的大路上。
\VS{3}于是{\PN{希勒家}}的儿子家宰{\PN{以利亚敬}},并书记{\PN{舍伯那}}和{\PN{亚萨}}的儿子史官{\PN{约亚}},出来见拉伯沙基。
\par }{\PP \VS{4}拉伯沙基对他们说:「你们去告诉{\PN{希西家}}说,{\PN{亚述}}大王如此说:『你所倚靠的有什么可仗赖的呢?
\VS{5}你说,有打仗的计谋和能力,我看不过是虚话。你到底倚靠谁才背叛我呢?
\VS{6}看哪,你所倚靠的{\PN{埃及}}是那压伤的苇杖,人若靠这杖,就必刺透他的手。{\PN{埃及}}王法老向一切倚靠他的人也是这样。
\VS{7}你若对我说:我们倚靠耶和华—我们的 神。{\PN{希西家}}岂不是将 神的邱坛和祭坛废去,且对{\PN{犹大}}和{\PN{耶路撒冷}}的人说:你们当在这坛前敬拜吗?
\VS{8}现在你把当头给我主{\PN{亚述}}王,我给你二千匹马,看你这一面骑马的人够不够。
\VS{9}若不然,怎能打败我主臣仆中最小的军长呢?你竟倚靠{\PN{埃及}}的战车马兵吗?
\VS{10}现在我上来攻击毁灭这地,岂没有耶和华的意思吗?耶和华吩咐我说,你上去攻击毁灭这地吧!』」
\par }{\PP \VS{11}{\PN{以利亚敬}}、{\PN{舍伯那}}、{\PN{约亚}}对拉伯沙基说:「求你用{\PN{亚兰}}言语和仆人说话,因为我们懂得;不要用{\PN{犹大}}言语和我们说话,达到城上百姓的耳中。」
\VS{12}拉伯沙基说:「我主差遣我来,岂是单对你和你的主说这些话吗?不也是对这些坐在城上、要与你们一同吃自己粪喝自己尿的人说吗?」
\par }{\PP \VS{13}于是,拉伯沙基站着,用{\PN{犹大}}言语大声喊着说:「你们当听{\PN{亚述}}大王的话。
\VS{14}王如此说:『你们不要被{\PN{希西家}}欺哄了,因他不能拯救你们。
\VS{15}也不要听{\PN{希西家}}使你们倚靠耶和华说:耶和华必要拯救我们,这城必不交在{\PN{亚述}}王的手中。
\VS{16}不要听{\PN{希西家}}的话,因{\PN{亚述}}王如此说:你们要与我和好。出来投降我,各人就可以吃自己葡萄树和无花果树的果子,喝自己井里的水。
\VS{17}等我来领你们到一个地方,与你们本地一样,就是有五谷和新酒之地,有粮食和葡萄园之地。
\VS{18}你们要谨防,恐怕{\PN{希西家}}劝导你们说:耶和华必拯救我们。列国的神有哪一个救他本国脱离{\PN{亚述}}王的手呢?
\VS{19}{\PN{哈马}}和{\PN{亚珥拔}}的神在哪里呢?{\PN{西法瓦音}}的神在哪里呢?他们曾救{\PN{撒马利亚}}脱离我的手吗?
\VS{20}这些国的神有谁曾救自己的国脱离我的手呢?难道耶和华能救{\PN{耶路撒冷}}脱离我的手吗?』」
\par }{\PP \VS{21}百姓静默不言,并不回答一句,因为王曾吩咐说:「不要回答他。」
\VS{22}当下{\PN{希勒家}}的儿子家宰{\PN{以利亚敬}}和书记{\PN{舍伯那}},并{\PN{亚萨}}的儿子史官{\PN{约亚}},都撕裂衣服,来到{\PN{希西家}}那里,将拉伯沙基的话告诉了他。

\par }\Chap{37}{\SH 国王下令征求以赛亚的意见
\par }{\R (王下19·1—7)
\par }{\PP \VerseOne{1}{\PN{希西家}}王听见就撕裂衣服,披上麻布,进了耶和华的殿,
\VS{2}使家宰{\PN{以利亚敬}}和书记{\PN{舍伯那}},并祭司中的长老,都披上麻布,去见{\PN{亚摩斯}}的儿子先知{\PN{以赛亚}},
\VS{3}对他说:「{\PN{希西家}}如此说:『今日是急难、责罚、凌辱的日子,就如妇人将要生产婴孩,却没有力量生产。
\VS{4}或者耶和华—你的 神听见拉伯沙基的话,就是他主人{\PN{亚述}}王打发他来辱骂永生 神的话;耶和华—你的 神听见这话就发斥责。故此,求你为余剩的民扬声祷告。』」
\par }{\PP \VS{5}{\PN{希西家}}王的臣仆就去见{\PN{以赛亚}}。
\VS{6}{\PN{以赛亚}}对他们说:「要这样对你们的主人说,耶和华如此说:『你听见{\PN{亚述}}王的仆人亵渎我的话,不要惧怕。
\VS{7}我必惊动\FTNT{}{{\FR 37:7: }原文是使灵进入}他的心;他要听见风声就归回本地,我必使他在那里倒在刀下。』」
\par }{\SH 亚述人再来威胁
\par }{\R (王下19·8—19)
\par }{\PP \VS{8}拉伯沙基回去,正遇见{\PN{亚述}}王攻打{\PN{立拿}};原来他早听见{\PN{亚述}}王拔营离开{\PN{拉吉}}。
\VS{9}{\PN{亚述}}王听见人论{\PN{古实}}王{\PN{特哈加}}说:「他出来要与你争战。」{\PN{亚述}}王一听见,就打发使者去见{\PN{希西家}},吩咐他们说:
\VS{10}「你们对{\PN{犹大}}王{\PN{希西家}}如此说:『不要听你所倚靠的 神欺哄你说:{\PN{耶路撒冷}}必不交在{\PN{亚述}}王的手中。
\VS{11}你总听说{\PN{亚述}}诸王向列国所行的乃是尽行灭绝,难道你还能得救吗?
\VS{12}我列祖所毁灭的,就是{\PN{歌散}}、{\PN{哈兰}}、{\PN{利色}},和属{\PN{提·拉撒}}的{\PN{伊甸}}人;这些国的神何曾拯救这些国呢?
\VS{13}{\PN{哈马}}的王,{\PN{亚珥拔}}的王,{\PN{西法瓦音}}城的王,{\PN{希拿}}和{\PN{以瓦}}的王,都在哪里呢?』」
\par }{\PP \VS{14}{\PN{希西家}}从使者手里接过书信来,看完了,就上耶和华的殿,将书信在耶和华面前展开。
\VS{15}{\PN{希西家}}向耶和华祷告说:
\VS{16}「坐在二基路伯上万军之耶和华—{\PN{以色列}}的 神啊,你—惟有你是天下万国的 神,你曾创造天地。
\VS{17}耶和华啊,求你侧耳而听;耶和华啊,求你睁眼而看,要听{\PN{西拿基立}}的一切话,他是打发使者来辱骂永生 神的。
\VS{18}耶和华啊,{\PN{亚述}}诸王果然使列国和列国之地变为荒凉,
\VS{19}将列国的神像都扔在火里;因为他本不是神,乃是人手所造的,是木头、石头的,所以灭绝他。
\VS{20}耶和华—我们的 神啊,现在求你救我们脱离{\PN{亚述}}王的手,使天下万国都知道惟有你是耶和华。」
\par }{\SH 以赛亚给王的信息
\par }{\R (王下19·20—37)
\par }{\PP \VS{21}{\PN{亚摩斯}}的儿子{\PN{以赛亚}}就打发人去见{\PN{希西家}},说:「耶和华—{\PN{以色列}}的 神如此说,你既然求我攻击{\PN{亚述}}王{\PN{西拿基立}},
\VS{22}所以耶和华论他这样说:
\par }{\Q {\PN{锡安}}的处女藐视你,嗤笑你;
\par }{\Q {\PN{耶路撒冷}}的女子向你摇头。
\par }{\BB \par }{\Q \VS{23}你辱骂谁,亵渎谁?
\par }{\Q 扬起声来,高举眼目攻击谁呢?
\par }{\Q 乃是攻击{\PN{以色列}}的圣者。
\par }{\Q \VS{24}你借你的臣仆辱骂主说:
\par }{\Q 我率领许多战车上山顶,
\par }{\Q 到{\PN{黎巴嫩}}极深之处;
\par }{\Q 我要砍伐其中高大的香柏树
\par }{\Q 和佳美的松树。
\par }{\Q 我必上极高之处,
\par }{\Q 进入肥田的树林。
\par }{\Q \VS{25}我已经挖井喝水;
\par }{\Q 我必用脚掌踏干{\PN{埃及}}的一切河。
\par }{\BB \par }{\Q \VS{26}{\ADD{耶和华说}}:你岂没有听见
\par }{\Q 我早先所做的、古时所立的吗?
\par }{\Q 现在借你使坚固城荒废,变为乱堆。
\par }{\Q \VS{27}所以其中的居民力量甚小,
\par }{\Q 惊惶羞愧。
\par }{\Q 他们像野草,像青菜,
\par }{\Q 如房顶上的草,
\par }{\Q 又如田间未长成的{\ADD{禾稼}}。
\par }{\BB \par }{\Q \VS{28}你坐下,你出去,你进来,
\par }{\Q 你向我发烈怒,我都知道。
\par }{\Q \VS{29}因你向我发烈怒,
\par }{\Q 又因你狂傲{\ADD{的话}}达到我耳中,
\par }{\Q 我就要用钩子钩上你的鼻子,
\par }{\Q 把嚼环放在你口里,
\par }{\Q 使你从原路转回去。
\par }{\PP \VS{30}「{\PN{以色列}}人哪,我赐你们一个证据:你们今年要吃自生的,明年也要吃自长的,至于后年,你们要耕种收割,栽植葡萄园,吃其中的果子。
\VS{31}{\PN{犹大}}家所逃脱余剩的,仍要往下扎根,向上结果。
\VS{32}必有余剩的民从{\PN{耶路撒冷}}而出;必有逃脱的人从{\PN{锡安山}}而来。万军之耶和华的热心必成就这事。
\par }{\PP \VS{33}「所以耶和华论{\PN{亚述}}王如此说:他必不得来到这城,也不在这里射箭,不得拿盾牌到城前,也不筑垒攻城。
\VS{34}他从哪条路来,必从那条路回去,必不得来到这城。这是耶和华说的。
\VS{35}因我为自己的缘故,又为我仆人{\PN{大卫}}的缘故,必保护拯救这城。」
\par }{\PP \VS{36}耶和华的使者出去,在{\PN{亚述}}营中杀了十八万五千人。清早有人起来一看,都是死尸了。
\VS{37}{\PN{亚述}}王{\PN{西拿基立}}就拔营回去,住在{\PN{尼尼微}}。
\VS{38}一日在他的神—{\PN{尼斯洛}}庙里叩拜,他儿子{\PN{亚得米勒}}和{\PN{沙利色}}用刀杀了他,就逃到{\PN{亚拉腊}}地。他儿子{\PN{以撒哈顿}}接续他作王。

\par }\Chap{38}{\SH 希西家王的疾病和痊愈
\par }{\R (王下20·1—11;代下32·24—26)
\par }{\PP \VerseOne{1}那时{\PN{希西家}}病得要死,{\PN{亚摩斯}}的儿子先知{\PN{以赛亚}}去见他,对他说:「耶和华如此说:你当留遗命与你的家,因为你必死不能活了。」
\VS{2}{\PN{希西家}}就转脸朝墙,祷告耶和华说:
\VS{3}「耶和华啊,求你记念我在你面前怎样存完全的心,按诚实行事,又做你眼中所看为善的。」{\PN{希西家}}就痛哭了。
\VS{4}耶和华的话临到{\PN{以赛亚}}说:
\VS{5}「你去告诉{\PN{希西家}}说,耶和华—你祖{\PN{大卫}}的 神如此说:我听见了你的祷告,看见了你的眼泪。我必加增你十五年的寿数;
\VS{6}并且我要救你和这城脱离{\PN{亚述}}王的手,也要保护这城。
\par }{\PP \VS{7}「我—耶和华必成就我所说的。我先给你一个兆头,
\VS{8}就是叫{\PN{亚哈斯}}的日晷,向前进的日影往后退十度。」于是,前进的日影果然在日晷上往后退了十度。
\par }{\Q \VS{9}{\PN{犹大}}王{\PN{希西家}}患病已经痊愈,就作诗说:
\par }{\Q \VS{10}我说:正在我中年\FTNT{}{{\FR 38:10: }或译:晌午}之日
\par }{\Q 必进入阴间的门;
\par }{\Q 我余剩的年岁不得享受。
\par }{\Q \VS{11}我说:我必不得见耶和华,
\par }{\Q 就是在活人之地不见耶和华;
\par }{\Q 我与世上的居民不再见面。
\par }{\Q \VS{12}我的住处被迁去离开我,
\par }{\Q 好像牧人的帐棚一样;
\par }{\Q 我将性命卷起,
\par }{\Q 像织布的卷布一样。
\par }{\Q 耶和华必将我从机头剪断,
\par }{\Q 从早到晚,他要使我完结。
\par }{\Q \VS{13}我使自己安静直到天亮;
\par }{\Q 他像狮子折断我一切的骨头,
\par }{\Q 从早到晚,他要使我完结。
\par }{\BB \par }{\Q \VS{14}我像燕子呢喃,
\par }{\Q 像白鹤鸣叫,
\par }{\Q 又像鸽子哀鸣;
\par }{\Q 我因仰观,眼睛困倦。
\par }{\Q 耶和华啊,我受欺压,
\par }{\Q 求你为我作保。
\par }{\Q \VS{15}我可说什么呢?
\par }{\Q 他应许我的,也给我成就了。
\par }{\Q 我因心里的苦楚,
\par }{\Q 在一生的年日必悄悄而行。
\par }{\BB \par }{\Q \VS{16}主啊,人得存活乃在乎此。
\par }{\Q 我灵存活也全在此。
\par }{\Q 所以求你使我痊愈,仍然存活。
\par }{\Q \VS{17}看哪,我受大苦,本为使我得平安;
\par }{\Q 你因爱我的灵魂\FTNT{}{{\FR 38:17: }或译:生命}便救我脱离败坏的坑,
\par }{\Q 因为你将我一切的罪扔在你的背后。
\par }{\Q \VS{18}原来,阴间不能称谢你,
\par }{\Q 死亡不能颂扬你;
\par }{\Q 下坑的人不能盼望你的诚实。
\par }{\Q \VS{19}只有活人,活人必称谢你,
\par }{\Q 像我今日称谢你一样。
\par }{\Q 为父的,必使儿女知道你的诚实。
\par }{\BB \par }{\Q \VS{20}耶和华{\ADD{肯}}救我,
\par }{\Q 所以,我们要一生一世
\par }{\Q 在耶和华殿中
\par }{\Q 用丝弦的乐器唱我的诗歌。
\par }{\PP \VS{21}{\PN{以赛亚}}说:「当取一块无花果饼来,贴在疮上,王必痊愈。」
\VS{22}{\PN{希西家}}问说:「我能上耶和华的殿,有什么兆头呢?」

\par }\Chap{39}{\SH 巴比伦王的使节
\par }{\R (王下20·12—19)
\par }{\PP \VerseOne{1}那时,{\PN{巴比伦}}王{\PN{巴拉但}}的儿子{\PN{米罗达·巴拉但}}听见{\PN{希西家}}病而痊愈,就送书信和礼物给他。
\VS{2}{\PN{希西家}}喜欢见使者,就把自己宝库的金子、银子、香料、贵重的膏油,和他武库的一切军器,并所有的财宝都给他们看;他家中和全国之内,{\PN{希西家}}没有一样不给他们看的。
\VS{3}于是先知{\PN{以赛亚}}来见{\PN{希西家}}王,问他说:「这些人说什么?他们从哪里来见你?」{\PN{希西家}}说:「他们从远方的{\PN{巴比伦}}来见我。」
\VS{4}{\PN{以赛亚}}说:「他们在你家里看见了什么?」{\PN{希西家}}说:「凡我家中所有的,他们都看见了;我财宝中没有一样不给他们看的。」
\par }{\PP \VS{5}{\PN{以赛亚}}对{\PN{希西家}}说:「你要听万军之耶和华的话:
\VS{6}日子必到,凡你家里所有的,并你列祖积蓄到如今的,都要被掳到{\PN{巴比伦}}去,不留下一样;这是耶和华说的。
\VS{7}并且从你本身所生的众子,其中必有被掳去、在{\PN{巴比伦}}王宫里当太监的。」
\VS{8}{\PN{希西家}}对{\PN{以赛亚}}说:「你所说耶和华的话甚好,因为在我的年日中必有太平和稳固的景况。」

\par }\Chap{40}{\SH 安慰的话
\par }{\Q \VerseOne{1}你们的 神说:
\par }{\Q 你们要安慰,安慰我的百姓。
\par }{\Q \VS{2}要对{\PN{耶路撒冷}}说安慰的话,
\par }{\Q 又向她宣告{\ADD{说}},
\par }{\Q 她争战的日子已满了;
\par }{\Q 她的罪孽赦免了;
\par }{\Q 她为自己的一切罪,
\par }{\Q 从耶和华手中加倍受罚。
\par }{\BB \par }{\Q \VS{3}有人声喊着说:
\par }{\Q 在旷野预备耶和华的路\FTNT{}{{\FR 40:3: }或译:在旷野,有人声喊着说:当预备耶和华的路},
\par }{\Q 在沙漠地修平我们 神的道。
\par }{\Q \VS{4}一切山洼都要填满,
\par }{\Q 大小山冈都要削平;
\par }{\Q 高高低低的要改为平坦,
\par }{\Q 崎崎岖岖的必成为平原。
\par }{\Q \VS{5}耶和华的荣耀必然显现;
\par }{\Q 凡有血气的必一同看见;
\par }{\Q 因为这是耶和华亲口说的。
\par }{\BB \par }{\Q \VS{6}有人声说:你喊叫吧!
\par }{\Q 有一个说:我喊叫什么呢?
\par }{\Q {\ADD{说}}:凡有血气的尽都如草;
\par }{\Q 他的美容都像野地的花。
\par }{\Q \VS{7}草必枯干,花必凋残,
\par }{\Q 因为耶和华的气吹在其上;
\par }{\Q 百姓诚然是草。
\par }{\Q \VS{8}草必枯干,花必凋残,
\par }{\Q 惟有我们 神的话必永远立定。
\par }{\BB \par }{\Q \VS{9}报好信息给{\PN{锡安}}的啊,
\par }{\Q 你要登高山;
\par }{\Q 报好信息给{\PN{耶路撒冷}}的啊,
\par }{\Q 你要极力扬声。
\par }{\Q 扬声不要惧怕,
\par }{\Q 对{\PN{犹大}}的城邑说:
\par }{\Q 看哪,你们的 神!
\par }{\Q \VS{10}主耶和华必像大能者临到;
\par }{\Q 他的膀臂必为他掌权。
\par }{\Q 他的赏赐在他那里;
\par }{\Q 他的报应在他面前。
\par }{\Q \VS{11}他必像牧人牧养自己的羊群,
\par }{\Q 用膀臂聚集羊羔抱在怀中,
\par }{\Q 慢慢引导那乳养小羊的。
\par }{\SH 以色列的 神无可比拟
\par }{\Q \VS{12}谁曾用手心量诸水,
\par }{\Q 用手虎口量苍天,
\par }{\Q 用升斗盛大地的尘土,
\par }{\Q 用秤称山岭,
\par }{\Q 用天平平冈陵呢?
\par }{\Q \VS{13}谁曾测度耶和华的心\FTNT{}{{\FR 40:13: }或译:谁曾指示耶和华的灵},
\par }{\Q 或作他的谋士指教他呢?
\par }{\Q \VS{14}他与谁商议,谁教导他,
\par }{\Q 谁将公平的路指示他,
\par }{\Q 又将知识教训他,
\par }{\Q 将通达的道指教他呢?
\par }{\Q \VS{15}看哪,万民都像水桶的一滴,
\par }{\Q 又算如天平上的微尘;
\par }{\Q 他举起众海岛,好像极微之物。
\par }{\Q \VS{16}{\PN{黎巴嫩}}{\ADD{的树林}}不够当柴烧;
\par }{\Q 其中的走兽也不够作燔祭。
\par }{\Q \VS{17}万民在他面前好像虚无,
\par }{\Q 被他看为不及虚无,乃为虚空。
\par }{\BB \par }{\Q \VS{18}你们究竟将谁比 神,
\par }{\Q 用什么形象与 神比较呢?
\par }{\Q \VS{19}偶像是匠人铸造,
\par }{\Q 银匠用金包裹,
\par }{\Q {\ADD{为它}}铸造银链。
\par }{\Q \VS{20}穷乏献不起{\ADD{这样}}供物的,
\par }{\Q 就拣选不能朽坏的树木,
\par }{\Q 为自己寻找巧匠,
\par }{\Q 立起不能摇动的偶像。
\par }{\BB \par }{\Q \VS{21}你们岂不曾知道吗?
\par }{\Q 你们岂不曾听见吗?
\par }{\Q 从起初岂没有人告诉你们吗?
\par }{\Q 自从立地的根基,
\par }{\Q 你们岂没有明白吗?
\par }{\Q \VS{22}神坐在地球大圈之上;
\par }{\Q 地上的居民好像蝗虫。
\par }{\Q 他铺张穹苍如幔子,
\par }{\Q 展开诸天如可住的帐棚。
\par }{\Q \VS{23}他使君王归于虚无,
\par }{\Q 使地上的审判官成为虚空。
\par }{\BB \par }{\Q \VS{24}他们是刚才\FTNT{}{{\FR 40:24: }或译:不曾;下同}栽上,
\par }{\Q 刚才种上,
\par }{\Q 根也刚才扎在地里,
\par }{\Q 他一吹在其上,便都枯干;
\par }{\Q 旋风将他们吹去,像碎秸一样。
\par }{\BB \par }{\Q \VS{25}那圣者说:你们将谁比我,
\par }{\Q 叫他与我相等呢?
\par }{\Q \VS{26}你们向上举目,
\par }{\Q 看谁创造这万象,
\par }{\Q 按数目领出,
\par }{\Q 他一一称其名;
\par }{\Q 因他的权能,
\par }{\Q 又因他的大能大力,
\par }{\Q 连一个都不缺。
\par }{\BB \par }{\Q \VS{27}{\PN{雅各}}啊,你为何说,
\par }{\Q 我的道路向耶和华隐藏?
\par }{\Q {\PN{以色列}}啊,你为何言,
\par }{\Q 我的冤屈 神并不查问?
\par }{\Q \VS{28}你岂不曾知道吗?
\par }{\Q 你岂不曾听见吗?
\par }{\Q 永在的 神耶和华,创造地极的{\ADD{主}},
\par }{\Q 并不疲乏,也不困倦;
\par }{\Q 他的智慧无法测度。
\par }{\Q \VS{29}疲乏的,他赐能力;
\par }{\Q 软弱的,他加力量。
\par }{\Q \VS{30}就是少年人也要疲乏困倦;
\par }{\Q 强壮的也必全然跌倒。
\par }{\Q \VS{31}但那等候耶和华的必从新得力。
\par }{\Q 他们必如鹰展翅上腾;
\par }{\Q 他们奔跑却不困倦,
\par }{\Q 行走却不疲乏。

\par }\Chap{41}{\SH  神对以色列的保证
\par }{\Q \VerseOne{1}众海岛啊,当在我面前静默;
\par }{\Q 众民当从新得力,
\par }{\Q 都要近前来才可以说话,
\par }{\Q 我们可以彼此辩论。
\par }{\BB \par }{\Q \VS{2}谁从东方兴起一人,
\par }{\Q 凭公义召他来到脚前呢?
\par }{\Q 耶和华将列国交给他,
\par }{\Q 使他管辖君王,
\par }{\Q 把他们如灰尘交与他的刀,
\par }{\Q 如风吹的碎秸交与他的弓。
\par }{\Q \VS{3}他追赶他们,
\par }{\Q 走他所未走的道,
\par }{\Q 坦然前行。
\par }{\Q \VS{4}谁行做成就这事,
\par }{\Q 从起初宣召历代呢?
\par }{\Q 就是我—耶和华!
\par }{\Q 我是首先的,
\par }{\Q 也与末后的同在。
\par }{\BB \par }{\Q \VS{5}海岛看见就都害怕;
\par }{\Q 地极也都战兢,
\par }{\Q 就近前来。
\par }{\Q \VS{6}他们各人帮助邻舍,
\par }{\Q {\ADD{各人}}对弟兄说:壮胆吧!
\par }{\Q \VS{7}木匠勉励银匠,
\par }{\Q 用锤打光的勉励打砧的,
\par }{\Q 论焊工说,焊得好;
\par }{\Q 又用钉子钉稳,免得{\ADD{偶像}}动摇。
\par }{\BB \par }{\Q \VS{8}惟你{\PN{以色列}}—我的仆人,
\par }{\Q {\PN{雅各}}—我所拣选的,
\par }{\Q 我朋友{\PN{亚伯拉罕}}的后裔,
\par }{\Q \VS{9}你是我从地极所领\FTNT{}{{\FR 41:9: }原文是抓}来的,
\par }{\Q 从地角所召来的,
\par }{\Q 且对你说:你是我的仆人;
\par }{\Q 我拣选你,并不弃绝你。
\par }{\Q \VS{10}你不要害怕,因为我与你同在;
\par }{\Q 不要惊惶,因为我是你的 神。
\par }{\Q 我必坚固你,我必帮助你;
\par }{\Q 我必用我公义的右手扶持你。
\par }{\BB \par }{\Q \VS{11}凡向你发怒的必都抱愧蒙羞;
\par }{\Q 与你相争的必如无有,并要灭亡。
\par }{\Q \VS{12}与你争竞的,你要找他们也找不着;
\par }{\Q 与你争战的必如无有,成为虚无。
\par }{\Q \VS{13}因为我耶和华—
\par }{\Q 你的 神必搀扶你的右手,
\par }{\Q 对你说:不要害怕!
\par }{\Q 我必帮助你。
\par }{\BB \par }{\Q \VS{14}你这虫{\PN{雅各}}和你们{\PN{以色列}}人,
\par }{\Q 不要害怕!
\par }{\Q 耶和华说:我必帮助你。
\par }{\Q 你的救赎主就是{\PN{以色列}}的圣者。
\par }{\Q \VS{15}看哪,我已使你成为
\par }{\Q 有快齿打粮的新器具;
\par }{\Q 你要把山岭打得粉碎,
\par }{\Q 使冈陵如同糠秕。
\par }{\Q \VS{16}你要把它簸扬,风要吹去;
\par }{\Q 旋风要把它刮散。
\par }{\Q 你倒要以耶和华为喜乐,
\par }{\Q 以{\PN{以色列}}的圣者为夸耀。
\par }{\BB \par }{\Q \VS{17}困苦穷乏人寻求水却没有;
\par }{\Q 他们因口渴,舌头干燥。
\par }{\Q 我—耶和华必应允他们;
\par }{\Q 我—{\PN{以色列}}的 神必不离弃他们。
\par }{\Q \VS{18}我要在净光的高处开江河,
\par }{\Q 在谷中开泉源;
\par }{\Q 我要使沙漠变为水池,
\par }{\Q 使干地变为涌泉。
\par }{\Q \VS{19}我要在旷野种上香柏树、
\par }{\Q 皂荚树、番石榴树,和野橄榄树。
\par }{\Q 我在沙漠要把松树、杉树,
\par }{\Q 并黄杨树一同栽植;
\par }{\Q \VS{20}好叫人看见、知道、
\par }{\Q 思想、明白;
\par }{\Q 这是耶和华的手所做的,
\par }{\Q 是{\PN{以色列}}的圣者所造的。
\par }{\SH 耶和华向假神挑战
\par }{\Q \VS{21}耶和华{\ADD{对假神}}说:
\par }{\Q 你们要呈上你们的案件;
\par }{\Q {\PN{雅各}}的君说:
\par }{\Q 你们要声明你们确实的理由。
\par }{\Q \VS{22}可以声明,指示我们将来必遇的事,
\par }{\Q 说明先前的是什么事,
\par }{\Q 好叫我们思索,得知事的结局,
\par }{\Q 或者把将来的事指示我们。
\par }{\Q \VS{23}要说明后来的事,
\par }{\Q 好叫我们知道你们是神。
\par }{\Q 你们或降福,或降祸,
\par }{\Q 使我们惊奇,一同观看。
\par }{\Q \VS{24}看哪,你们属乎虚无;
\par }{\Q 你们的作为也属乎虚空。
\par }{\Q 那选择你们的是可憎恶的。
\par }{\BB \par }{\Q \VS{25}我从北方兴起一人;
\par }{\Q 他是求告我名的,
\par }{\Q 从日出之地而来。
\par }{\Q 他必临到掌权的,
\par }{\Q 好像临到灰泥,
\par }{\Q 仿佛窑匠踹泥一样。
\par }{\Q \VS{26}谁从起初指明这事,使我们知道呢?
\par }{\Q 谁从先前说明,使我们说{\ADD{他}}不错呢?
\par }{\Q 谁也没有指明;
\par }{\Q 谁也没有说明;
\par }{\Q 谁也没有听见你们的话。
\par }{\Q \VS{27}我首先对{\PN{锡安}}{\ADD{说}}:
\par }{\Q 看哪,我要将一位报好信息的赐给{\PN{耶路撒冷}}。
\par }{\Q \VS{28}我看的时候并没有人;
\par }{\Q 我问的时候,
\par }{\Q 他们中间也没有谋士可以回答一句。
\par }{\Q \VS{29}看哪,他们和他们的工作都是虚空,
\par }{\Q 且是虚无。
\par }{\Q 他们所铸的偶像都是风,
\par }{\Q 都是虚的。

\par }\Chap{42}{\SH 耶和华的仆人
\par }{\Q \VerseOne{1}看哪,我的仆人—
\par }{\Q 我所扶持所拣选、心里所喜悦的!
\par }{\Q 我已将我的灵赐给他;
\par }{\Q 他必将公理传给外邦。
\par }{\Q \VS{2}他不喧嚷,不扬声,
\par }{\Q 也不使街上听见他的声音。
\par }{\Q \VS{3}压伤的芦苇,他不折断;
\par }{\Q 将残的灯火,他不吹灭。
\par }{\Q 他凭真实将公理传开。
\par }{\Q \VS{4}他不灰心,也不丧胆,
\par }{\Q 直到他在地上设立公理;
\par }{\Q 海岛都等候他的训诲。
\par }{\BB \par }{\Q \VS{5}创造诸天,铺张穹苍,
\par }{\Q 将地和地所出的一并铺开,
\par }{\Q 赐气息给地上的众人,
\par }{\Q 又赐灵性给行在其上之人的 神耶和华,
\par }{\Q 他如此说:
\par }{\Q \VS{6}我—耶和华凭公义召你,
\par }{\Q 必搀扶你的手,保守你,
\par }{\Q 使你作众民的中保\FTNT{}{{\FR 42:6: }中保:原文是约},
\par }{\Q 作外邦人的光,
\par }{\Q \VS{7}开瞎子的眼,
\par }{\Q 领被囚的出牢狱,
\par }{\Q 领坐黑暗的出监牢。
\par }{\Q \VS{8}我是耶和华,这是我的名;
\par }{\Q 我必不将我的荣耀归给假神,
\par }{\Q 也不将我的称赞归给雕刻的偶像。
\par }{\Q \VS{9}看哪,先前的事已经成就,
\par }{\Q 现在我将新事说明,
\par }{\Q 这事未发以先,我就说给你们听。
\par }{\SH 颂赞之歌
\par }{\Q \VS{10}航海的和海中所有的,
\par }{\Q 海岛和其上的居民,
\par }{\Q 都当向耶和华唱新歌,
\par }{\Q 从地极赞美他。
\par }{\Q \VS{11}旷野和其中的城邑,
\par }{\Q 并{\PN{基达}}人居住的村庄都当扬声;
\par }{\Q {\PN{西拉}}的居民当欢呼,
\par }{\Q 在山顶上呐喊。
\par }{\Q \VS{12}他们当将荣耀归给耶和华,
\par }{\Q 在海岛中传扬他的颂赞。
\par }{\Q \VS{13}耶和华必像勇士出去,
\par }{\Q 必像战士激动热心,
\par }{\Q 要喊叫,大声呐喊,
\par }{\Q 要用大力攻击仇敌。
\par }{\SH  神应许帮助他的子民
\par }{\Q \VS{14}我许久闭口不言,静默不语;
\par }{\Q {\ADD{现在}}我要喊叫,像产难的妇人;
\par }{\Q 我要急气而喘哮。
\par }{\Q \VS{15}我要使大山小冈变为荒场,
\par }{\Q 使其上的花草都枯干;
\par }{\Q 我要使江河变为洲岛,
\par }{\Q 使水池都干涸。
\par }{\Q \VS{16}我要引瞎子行不认识的道,
\par }{\Q 领他们走不知道的路;
\par }{\Q 在他们面前使黑暗变为光明,
\par }{\Q 使弯曲变为平直。
\par }{\Q 这些事我都要行,
\par }{\Q 并不离弃他们。
\par }{\Q \VS{17}倚靠雕刻的偶像,
\par }{\Q 对铸造的偶像说:
\par }{\Q 你是我们的神;
\par }{\Q 这等人要退后,全然蒙羞。
\par }{\SH 以色列人不受教益
\par }{\Q \VS{18}你们这耳聋的,听吧!
\par }{\Q 你们这眼瞎的,看吧!
\par }{\Q 使你们能看见。
\par }{\Q \VS{19}谁比我的仆人眼瞎呢?
\par }{\Q 谁比我差遣的使者耳聋呢?
\par }{\Q 谁瞎眼像那{\ADD{与我}}和好的?
\par }{\Q 谁瞎眼像耶和华的仆人呢?
\par }{\Q \VS{20}你看见许多事却不领会,
\par }{\Q 耳朵开通却不听见。
\par }{\Q \VS{21}耶和华因自己公义的缘故,
\par }{\Q 喜欢使律法\FTNT{}{{\FR 42:21: }或译:训诲}为大,为尊。
\par }{\Q \VS{22}但这百姓是被抢被夺的,
\par }{\Q 都牢笼在坑中,隐藏在狱里;
\par }{\Q 他们作掠物,无人拯救,
\par }{\Q 作掳物,无人说交还。
\par }{\Q \VS{23}你们中间谁肯侧耳听此,
\par }{\Q 谁肯留心而听,以防将来呢?
\par }{\Q \VS{24}谁将{\PN{雅各}}交出当作掳物,
\par }{\Q 将{\PN{以色列}}交给抢夺的呢?
\par }{\Q 岂不是耶和华吗?
\par }{\Q 就是我们所得罪的那位。
\par }{\Q 他们不肯遵行他的道,
\par }{\Q 也不听从他的训诲。
\par }{\Q \VS{25}所以,他将猛烈的怒气和争战的勇力
\par }{\Q 倾倒在{\PN{以色列}}的身上。
\par }{\Q 在他四围如火着起,他还不知道,
\par }{\Q 烧着他,他也不介意。

\par }\Chap{43}{\SH  神应许拯救他的子民
\par }{\Q \VerseOne{1}{\PN{雅各}}啊,创造你的耶和华,
\par }{\Q {\PN{以色列}}啊,造成你的那位,
\par }{\Q 现在如此说:
\par }{\Q 你不要害怕!因为我救赎了你。
\par }{\Q 我曾提你的名召你,你是属我的。
\par }{\Q \VS{2}你从水中经过,我必与你同在;
\par }{\Q 你趟过江河,水必不漫过你;
\par }{\Q 你从火中行过,必不被烧,
\par }{\Q 火焰也不着在你身上。
\par }{\Q \VS{3}因为我是耶和华—你的 神,
\par }{\Q 是{\PN{以色列}}的圣者—你的救主;
\par }{\Q 我已经使{\PN{埃及}}作你的赎价,
\par }{\Q 使{\PN{古实}}和{\PN{西巴}}代替你。
\par }{\Q \VS{4}因我看你为宝为尊;
\par }{\Q 又因我爱你,
\par }{\Q 所以我使人代替你,
\par }{\Q 使列邦人替换你的生命。
\par }{\Q \VS{5}不要害怕,因我与你同在;
\par }{\Q 我必领你的后裔从东方来,
\par }{\Q 又从西方招聚你。
\par }{\Q \VS{6}我要对北方说,交出来!
\par }{\Q 对南方说,不要拘留!
\par }{\Q 将我的众子从远方带来,
\par }{\Q 将我的众女从地极领回,
\par }{\Q \VS{7}就是凡称为我名下的人,
\par }{\Q 是我为自己的荣耀创造的,
\par }{\Q 是我所做成,所造作的。
\par }{\SH 以色列是耶和华的见证人
\par }{\Q \VS{8}你要将有眼而瞎、
\par }{\Q 有耳而聋的民都带出来!
\par }{\Q \VS{9}任凭万国聚集;
\par }{\Q 任凭众民会合。
\par }{\Q 其中谁能将此声明,
\par }{\Q 并将先前的事说给我们听呢?
\par }{\Q 他们可以带出见证来,自显为是;
\par }{\Q 或者他们听见便说:这是真的。
\par }{\Q \VS{10}耶和华说:你们是我的见证,
\par }{\Q 我所拣选的仆人。
\par }{\Q 既是这样,便可以知道,且信服我,
\par }{\Q 又明白我就是耶和华。
\par }{\Q 在我以前没有真神\FTNT{}{{\FR 43:10: }真:原文是造作的};
\par }{\Q 在我以后也必没有。
\par }{\Q \VS{11}惟有我是耶和华;
\par }{\Q 除我以外没有救主。
\par }{\Q \VS{12}我曾指示,我曾拯救,我曾说明,
\par }{\Q 并且在你们中间没有别神。
\par }{\Q 所以耶和华说:
\par }{\Q 你们是我的见证。
\par }{\Q 我也是 神;
\par }{\Q \VS{13}自从有日子以来,我就是 神;
\par }{\Q 谁也不能救人脱离我手。
\par }{\Q 我要行事谁能阻止呢?
\par }{\SH 逃离巴比伦
\par }{\Q \VS{14}耶和华—你们的救赎主、
\par }{\Q {\PN{以色列}}的圣者如此说:
\par }{\Q 因你们的缘故,
\par }{\Q 我已经打发人到{\PN{巴比伦}}去;
\par }{\Q 并且我要使{\PN{迦勒底}}人如逃民,
\par }{\Q 都坐自己喜乐的船下来。
\par }{\Q \VS{15}我是耶和华—你们的圣者,
\par }{\Q 是创造{\PN{以色列}}的,是你们的君王。
\par }{\Q \VS{16}耶和华在沧海中开道,
\par }{\Q 在大水中开路,
\par }{\Q \VS{17}使车辆、马匹、军兵、勇士都出来,
\par }{\Q 一同躺下,不再起来;
\par }{\Q 他们灭没,好像熄灭的灯火。
\par }{\Q \VS{18}耶和华如此说:
\par }{\Q 你们不要记念从前的事,
\par }{\Q 也不要思想古时的事。
\par }{\Q \VS{19}看哪,我要做一件新事;
\par }{\Q 如今要发现,你们岂不知道吗?
\par }{\Q 我必在旷野开道路,
\par }{\Q 在沙漠开江河。
\par }{\Q \VS{20}野地的走兽必尊重我;
\par }{\Q 野狗和鸵鸟也必如此。
\par }{\Q 因我使旷野有水,
\par }{\Q 使沙漠有河,
\par }{\Q 好赐给我的百姓、我的选民喝。
\par }{\Q \VS{21}这百姓是我为自己所造的,
\par }{\Q 好述说我的美德。
\par }{\SH 以色列人的罪恶
\par }{\Q \VS{22}{\PN{雅各}}啊,你并没有求告我;
\par }{\Q {\PN{以色列}}啊,你倒厌烦我。
\par }{\Q \VS{23}你没有将你的羊带来给我作燔祭,
\par }{\Q 也没有用祭物尊敬我;
\par }{\Q 我没有因供物使你服劳,
\par }{\Q 也没有因乳香使你厌烦。
\par }{\Q \VS{24}你没有用银子为我买菖蒲,
\par }{\Q 也没有用祭物的脂油使我饱足;
\par }{\Q 倒使我因你的罪恶服劳,
\par }{\Q 使我因你的罪孽厌烦。
\par }{\BB \par }{\Q \VS{25}惟有我为自己的缘故涂抹你的过犯;
\par }{\Q 我也不记念你的罪恶。
\par }{\Q \VS{26}你要提醒我,你我可以一同辩论;
\par }{\Q 你可以将{\ADD{你的理}}陈明,自显为义。
\par }{\Q \VS{27}你的始祖犯罪;
\par }{\Q 你的师傅违背我。
\par }{\Q \VS{28}所以,我要辱没圣所的首领,
\par }{\Q 使{\PN{雅各}}成为咒诅,
\par }{\Q 使{\PN{以色列}}成为辱骂。

\par }\Chap{44}{\SH 耶和华是独一的 神
\par }{\Q \VerseOne{1}我的仆人{\PN{雅各}},
\par }{\Q 我所拣选的{\PN{以色列}}啊,
\par }{\Q 现在你当听。
\par }{\Q \VS{2}造作你,又从你出胎造就你,
\par }{\Q 并要帮助你的耶和华如此说:
\par }{\Q 我的仆人{\PN{雅各}},
\par }{\Q 我所拣选的{\PN{耶书
}}哪,
\par }{\Q 不要害怕!
\par }{\Q \VS{3}因为我要将水浇灌口渴的人,
\par }{\Q 将河浇灌干旱之地。
\par }{\Q 我要将我的灵浇灌你的后裔,
\par }{\Q 将我的福浇灌你的子孙。
\par }{\Q \VS{4}他们要发生在草中,
\par }{\Q 像溪水旁的柳树。
\par }{\Q \VS{5}这个要说:我是属耶和华的;
\par }{\Q 那个要以{\PN{雅各}}的名自称;
\par }{\Q 又一个要亲手写:归耶和华的\FTNT{}{{\FR 44:5: }或译:在手上写归耶和华},
\par }{\Q 并自称为{\PN{以色列}}。
\par }{\BB \par }{\Q \VS{6}耶和华—{\PN{以色列}}的君,
\par }{\Q {\PN{以色列}}的救赎主—万军之耶和华如此说:
\par }{\Q 我是首先的,我是末后的;
\par }{\Q 除我以外再没有{\ADD{真}}神。
\par }{\Q \VS{7}自从我设立古时的民,
\par }{\Q 谁能像我宣告,并且指明,又为自己陈说呢?
\par }{\Q 让他将未来的事和必成的事说明。
\par }{\Q \VS{8}你们不要恐惧,也不要害怕。
\par }{\Q 我岂不是从上古就说明指示你们吗?
\par }{\Q 并且你们是我的见证!
\par }{\Q 除我以外,岂有{\ADD{真}}神吗?
\par }{\Q 诚然没有磐石,我不知道一个!
\par }{\SH 虚无的偶像
\par }{\PP \VS{9}制造雕刻偶像的尽都虚空;他们所喜悦的都无益处;他们的见证无所看见,无所知晓,他们便觉羞愧。
\VS{10}谁制造神像,铸造无益的偶像?
\VS{11}看哪,他的同伴都必羞愧。工匠也不过是人,任他们聚会,任他们站立,都必惧怕,一同羞愧。
\par }{\PP \VS{12}铁匠把铁在火炭中烧热,用锤打铁器,用他有力的膀臂锤成;他饥饿而无力,不喝水而发倦。
\VS{13}木匠拉线,用笔划出样子,用刨子刨成形状,用圆尺划了模样,仿照人的体态,做成人形,好住在房屋中。
\VS{14}他砍伐香柏树,又取柞\FTNT{}{{\FR 44:14: }或译:青桐}树和橡树,在树林中选定了一棵。他栽种松树,得雨长养。
\VS{15}这树,人可用以烧火;他自己取些烤火,又烧着烤饼,而且做神像跪拜,做雕刻的偶像向它叩拜。
\VS{16}他把一分烧在火中,把一分烤肉吃饱。自己烤火说:「啊哈,我暖和了,我见火了。」
\VS{17}他用剩下的做了一神,就是雕刻的偶像。他向这偶像俯伏叩拜,祷告它说:「求你拯救我,因你是我的神。」
\par }{\PP \VS{18}他们不知道,也不思想;因为耶和华闭住他们的眼,不能看见,塞住他们的心,不能明白。
\VS{19}谁心里也不醒悟,也没有知识,没有聪明,能说:「我曾拿一分在火中烧了,在炭火上烤过饼;我也烤过肉吃。这剩下的,我岂要作可憎的物吗?我岂可向木ⶍ子叩拜呢?」
\VS{20}他以灰为食,心中昏迷,使他偏邪,他不能自救,也不能说:「我右手中岂不是有虚谎吗?」
\par }{\SH 耶和华是创造主也是救赎主
\par }{\Q \VS{21}{\PN{雅各}},{\PN{以色列}}啊,
\par }{\Q 你是我的仆人,要记念这些事。
\par }{\Q {\PN{以色列}}啊,你是我的仆人,
\par }{\Q 我造就你必不忘记你。
\par }{\Q \VS{22}我涂抹了你的过犯,像厚云{\ADD{消散}};
\par }{\Q 我涂抹了你的罪恶,如薄云{\ADD{灭没}}。
\par }{\Q 你当归向我,因我救赎了你。
\par }{\Q \VS{23}诸天哪,应当歌唱,
\par }{\Q 因为耶和华做成这事。
\par }{\Q 地的深处啊,应当欢呼;
\par }{\Q 众山应当发声歌唱;
\par }{\Q 树林和其中所有的树都当如此!
\par }{\Q 因为耶和华救赎了{\PN{雅各}},
\par }{\Q 并要因{\PN{以色列}}荣耀自己。
\par }{\BB \par }{\Q \VS{24}从你出胎,造就你的救赎主—耶和华如此说:
\par }{\Q 我—耶和华是创造万物的,
\par }{\Q 是独自铺张诸天、铺开大地的。
\par }{\Q 谁与我同在呢?
\par }{\Q \VS{25}使说假话的兆头失效,
\par }{\Q 使占卜的癫狂,
\par }{\Q 使智慧人退后,
\par }{\Q 使他的知识变为愚拙;
\par }{\Q \VS{26}使我仆人的话语立定,
\par }{\Q 我使者的谋算成就。
\par }{\Q 论到{\PN{耶路撒冷}}说:必有人居住;
\par }{\Q 论到{\PN{犹大}}的城邑说:必被建造,
\par }{\Q 其中的荒场我也必兴起。
\par }{\Q \VS{27}对深渊说:你干了吧!
\par }{\Q 我也要使你的江河干涸。
\par }{\Q \VS{28}论{\PN{塞鲁士}}说:{\ADD{他是}}我的牧人,
\par }{\Q 必成就我所喜悦的,
\par }{\Q 必下令建造{\PN{耶路撒冷}},
\par }{\Q 发命立稳圣殿的根基。

\par }\Chap{45}{\SH 耶和华立塞鲁士为王
\par }{\Q \VerseOne{1}我—耶和华所膏的{\PN{塞鲁士}};
\par }{\Q 我搀扶他的右手,
\par }{\Q 使列国降伏在他面前。
\par }{\Q 我也要放松列王的腰带,
\par }{\Q 使城门在他面前敞开,
\par }{\Q 不得关闭。
\par }{\Q 我对他如此说:
\par }{\Q \VS{2}我必在你前面行,
\par }{\Q 修平崎岖之地。
\par }{\Q 我必打破铜门,
\par }{\Q 砍断铁闩。
\par }{\Q \VS{3}我要将暗中的宝物和隐密的财宝赐给你,
\par }{\Q 使你知道提名召你的,
\par }{\Q 就是我—耶和华、{\PN{以色列}}的 神。
\par }{\Q \VS{4}因我仆人{\PN{雅各}},
\par }{\Q 我所拣选{\PN{以色列}}的缘故,
\par }{\Q 我就提名召你;
\par }{\Q 你虽不认识我,
\par }{\Q 我也加给你名号。
\par }{\Q \VS{5}我是耶和华,在我以外并没有别神;
\par }{\Q 除了我以外再没有 神。
\par }{\Q 你虽不认识我,
\par }{\Q 我必给你束腰。
\par }{\Q \VS{6}从日出之地到日落之处
\par }{\Q 使人都知道除了我以外,没有别神。
\par }{\Q 我是耶和华;
\par }{\Q 在我以外并没有别神。
\par }{\Q \VS{7}我造光,又造暗;
\par }{\Q 我施平安,又降灾祸;
\par }{\Q 造作这一切的是我—耶和华。
\par }{\BB \par }{\Q \VS{8}诸天哪,自上而滴,
\par }{\Q 穹苍降下公义;
\par }{\Q 地面开裂,产出救恩,
\par }{\Q 使公义一同发生;
\par }{\Q 这都是我—耶和华所造的。
\par }{\SH 创造和历史的主宰
\par }{\Q \VS{9}祸哉,那与造他的主争论的!
\par }{\Q 他不过是地上瓦片中的一块瓦片。
\par }{\Q 泥土岂可对抟弄他的说:你做什么呢?
\par }{\Q 所做的物岂可说:你没有手呢?
\par }{\Q \VS{10}祸哉,那对父亲说:
\par }{\Q 你生的是什么呢?
\par }{\Q 或对母亲\FTNT{}{{\FR 45:10: }原文是妇人}说:
\par }{\Q 你产的是什么呢?
\par }{\Q \VS{11}耶和华—{\PN{以色列}}的圣者,
\par }{\Q 就是造就{\PN{以色列}}的如此说:
\par }{\Q 将来的事,你们可以问我;
\par }{\Q 至于我的众子,并我手的工作,
\par }{\Q 你们可以求我命定\FTNT{}{{\FR 45:11: }原文是吩咐我}。
\par }{\Q \VS{12}我造地,又造人在地上。
\par }{\Q 我亲手铺张诸天;
\par }{\Q 天上万象也是我所命定的。
\par }{\Q \VS{13}我凭公义兴起{\PN{塞鲁士}}\FTNT{}{{\FR 45:13: }原文是他},
\par }{\Q 又要修直他一切道路。
\par }{\Q 他必建造我的城,
\par }{\Q 释放我被掳的民;
\par }{\Q 不是为工价,也不是为赏赐。
\par }{\Q 这是万军之耶和华说的。
\par }{\BB \par }{\Q \VS{14}耶和华如此说:
\par }{\Q {\PN{埃及}}劳碌得来的和{\PN{古实}}的货物必归你;
\par }{\Q 身量高大的{\PN{西巴}}人必投降你,也要属你。
\par }{\Q 他们必带着锁链过来随从你,
\par }{\Q 又向你下拜,祈求你{\ADD{说}}:
\par }{\Q  神真在你们中间,此外再没有别神;
\par }{\Q 再没有别的 神。
\par }{\Q \VS{15}救主—{\PN{以色列}}的 神啊,
\par }{\Q 你实在是自隐的 神。
\par }{\Q \VS{16}凡制造偶像的都必抱愧蒙羞,
\par }{\Q 都要一同归于惭愧。
\par }{\Q \VS{17}惟有{\PN{以色列}}必蒙耶和华的拯救,
\par }{\Q 得永远的救恩。
\par }{\Q 你们必不蒙羞,也不抱愧,
\par }{\Q 直到永世无尽。
\par }{\BB \par }{\Q \VS{18}创造诸天的耶和华,
\par }{\Q 制造成全大地的 神,
\par }{\Q 他创造坚定大地,
\par }{\Q 并非使地荒凉,
\par }{\Q 是要给人居住。
\par }{\Q 他如此说:
\par }{\Q 我是耶和华,再没有别神。
\par }{\Q \VS{19}我没有在隐密黑暗之地说话;
\par }{\Q 我没有对{\PN{雅各}}的后裔说:
\par }{\Q 你们寻求我是徒然的。
\par }{\Q 我—耶和华所讲的是公义,
\par }{\Q 所说的是正直。
\par }{\SH 天地之主和巴比伦的偶像
\par }{\Q \VS{20}你们从列国逃脱的人,
\par }{\Q 要一同聚集前来。
\par }{\Q 那些抬着雕刻木偶、
\par }{\Q 祷告不能救人之神的,
\par }{\Q 毫无知识。
\par }{\Q \VS{21}你们要述说陈明{\ADD{你们的理}},
\par }{\Q 让他们彼此商议。
\par }{\Q 谁从古时指明?
\par }{\Q 谁从上古述说?
\par }{\Q 不是我—耶和华吗?
\par }{\Q 除了我以外,再没有 神;
\par }{\Q 我是公义的 神,又是救主;
\par }{\Q 除了我以外,再没有别神。
\par }{\Q \VS{22}地极{\ADD{的人}}都当仰望我,
\par }{\Q 就必得救;
\par }{\Q 因为我是 神,再没有别神。
\par }{\Q \VS{23}我指着自己起誓,
\par }{\Q 我口所出的话是凭公义,并不反回:
\par }{\Q 万膝必向我跪拜;
\par }{\Q 万口必凭我起誓。
\par }{\BB \par }{\Q \VS{24}人论我说,
\par }{\Q 公义、能力,惟独在乎耶和华;
\par }{\Q 人都必归向他。
\par }{\Q 凡向他发怒的必至蒙羞。
\par }{\Q \VS{25}{\PN{以色列}}的后裔都必因耶和华得称为义,
\par }{\Q 并要夸耀。

\par }\PoetryChap{46}{\Q \VerseOne{1}{\PN{彼勒}}屈身,{\PN{尼波}}弯腰;
\par }{\Q {\PN{巴比伦}}的偶像驮在兽和牲畜上。
\par }{\Q 他们所抬的如今成了重驮,
\par }{\Q 使{\ADD{牲畜}}疲乏,
\par }{\Q \VS{2}都一同弯腰屈身,
\par }{\Q 不能保全重驮,
\par }{\Q 自己倒被掳去。
\par }{\BB \par }{\Q \VS{3}{\PN{雅各}}家,{\PN{以色列}}家一切余剩的要听我言:
\par }{\Q 你们自从生下,就蒙{\ADD{我}}保抱,
\par }{\Q 自从出胎,便蒙{\ADD{我}}怀搋。
\par }{\Q \VS{4}直到你们年老,我仍这样;
\par }{\Q 直到你们发白,我仍怀搋。
\par }{\Q 我已造作,也必保抱;
\par }{\Q 我必怀抱,也必拯救。
\par }{\BB \par }{\Q \VS{5}你们将谁与我相比,与我同等,
\par }{\Q 可以与我比较,使我们相同呢?
\par }{\Q \VS{6}那从囊中抓金子,
\par }{\Q 用天平平银子的人,
\par }{\Q 雇银匠制造神像,
\par }{\Q 他们又俯伏,又叩拜。
\par }{\Q \VS{7}他们将神像抬起,扛在肩上,
\par }{\Q 安置在定处,它就站立,
\par }{\Q 不离本位;
\par }{\Q 人呼求它,它不能答应,
\par }{\Q 也不能救人脱离患难。
\par }{\BB \par }{\Q \VS{8}你们当想念这事,自己作大丈夫。
\par }{\Q 悖逆的人哪,要心里思想。
\par }{\Q \VS{9}你们要追念上古的事。
\par }{\Q 因为我是 神,并无别神;
\par }{\Q {\ADD{我是}} 神,再没有能比我的。
\par }{\Q \VS{10}我从起初指明末后的事,
\par }{\Q 从古时言明未成的事,
\par }{\Q 说:我的筹算必立定;
\par }{\Q 凡我所喜悦的,我必成就。
\par }{\Q \VS{11}我召鸷鸟从东方来,
\par }{\Q 召那成就我筹算的人从远方来。
\par }{\Q 我已说出,也必成就;
\par }{\Q 我已谋定,也必做成。
\par }{\Q \VS{12}你们这些心中顽梗、
\par }{\Q 远离公义的,当听我言。
\par }{\Q \VS{13}我使我的公义临近,必不远离。
\par }{\Q 我的救恩必不迟延;
\par }{\Q 我要为{\PN{以色列}}—我的荣耀,
\par }{\Q 在{\PN{锡安}}施行救恩。

\par }\Chap{47}{\SH 审判巴比伦
\par }{\Q \VerseOne{1}{\PN{巴比伦}}的处女啊,
\par }{\Q 下来坐在尘埃;
\par }{\Q {\PN{迦勒底}}的闺女啊,
\par }{\Q 没有宝座,要坐在地上;
\par }{\Q 因为你不再称为柔弱娇嫩的。
\par }{\Q \VS{2}要用磨磨面,
\par }{\Q 揭去帕子,
\par }{\Q 脱去长衣,露腿趟河。
\par }{\Q \VS{3}你的下体必被露出;
\par }{\Q 你的丑陋必被看见。
\par }{\Q 我要报仇,
\par }{\Q 谁也不宽容。
\par }{\Q \VS{4}我们救赎主的名是
\par }{\Q 万军之耶和华—{\PN{以色列}}的圣者。
\par }{\BB \par }{\Q \VS{5}{\PN{迦勒底}}的闺女啊,
\par }{\Q 你要默然静坐,进入暗中,
\par }{\Q 因为你不再称为列国的主母。
\par }{\Q \VS{6}我向我的百姓发怒,
\par }{\Q 使我的产业被亵渎,
\par }{\Q 将他们交在你手中,
\par }{\Q 你毫不怜悯他们,
\par }{\Q 把极重的轭加在老年人身上。
\par }{\Q \VS{7}你自己说:我必永为主母,
\par }{\Q 所以你不将这事放在心上,
\par }{\Q 也不思想这事的结局。
\par }{\BB \par }{\Q \VS{8}你这专好宴乐、安然居住的,
\par }{\Q 现在当听这话。
\par }{\Q 你心中说:惟有我,
\par }{\Q 除我以外再没有别的。
\par }{\Q 我必不致寡居,
\par }{\Q 也不遭丧子之事。
\par }{\Q \VS{9}哪知,丧子、寡居这两件事
\par }{\Q 在一日转眼之间必临到你;
\par }{\Q 正在你多行邪术、广施符咒的时候,
\par }{\Q 这两件事必全然临到你身上。
\par }{\BB \par }{\Q \VS{10}你素来倚仗自己的恶行,说:
\par }{\Q 无人看见我。
\par }{\Q 你的智慧聪明使你偏邪,
\par }{\Q 并且你心里说:惟有我,
\par }{\Q 除我以外再没有别的。
\par }{\Q \VS{11}因此,祸患要临到你身;
\par }{\Q 你不知何时发现\FTNT{}{{\FR 47:11: }或译:如何驱逐}
\par }{\Q 灾害落在你身上,
\par }{\Q 你也不能除掉;
\par }{\Q 所不知道的毁灭也必忽然临到你身。
\par }{\BB \par }{\Q \VS{12}站起来吧!
\par }{\Q 用你从幼年劳神施行的符咒和你许多的邪术;
\par }{\Q 或者可得益处,
\par }{\Q 或者可得强胜。
\par }{\Q \VS{13}你筹划太多,以致疲倦。
\par }{\Q 让那些观天象的,看星宿的,
\par }{\Q 在月朔说预言的,都站起来,
\par }{\Q 救你脱离所要临到你的事。
\par }{\BB \par }{\Q \VS{14}他们要像碎秸被火焚烧,
\par }{\Q 不能救自己脱离火焰之力;
\par }{\Q 这火并非可烤的炭火,
\par }{\Q 也不是可以坐在其前的火。
\par }{\Q \VS{15}你所劳神的事都要这样与你{\ADD{无益}};
\par }{\Q 从幼年与你贸易的也都各奔各乡,无人救你。

\par }\Chap{48}{\SH  神掌管未来
\par }{\Q \VerseOne{1}{\PN{雅各}}家,称为{\PN{以色列}}名下,
\par }{\Q 从{\PN{犹大}}水源出来的,当听我言!
\par }{\Q 你们指着耶和华的名起誓,
\par }{\Q 提说{\PN{以色列}}的 神,
\par }{\Q 却不凭诚实,不凭公义。
\par }{\Q \VS{2}他们自称为圣城的人,
\par }{\Q 所倚靠的是
\par }{\Q 名为万军之耶和华—{\PN{以色列}}的 神。
\par }{\BB \par }{\Q \VS{3}主说:早先的事,我从古时说明,
\par }{\Q 已经出了我的口,
\par }{\Q 也是我所指示的;
\par }{\Q 我忽然行做,事便成就。
\par }{\Q \VS{4}因为我素来知道你是顽梗的—
\par }{\Q 你的颈项是铁的;
\par }{\Q 你的额是铜的。
\par }{\Q \VS{5}所以,我从古时将这事给你说明,
\par }{\Q 在未成以先指示你,
\par }{\Q 免得你说:这些事是我的偶像所行的,
\par }{\Q 是我雕刻的偶像和我铸造的偶像所命定的。
\par }{\BB \par }{\Q \VS{6}你已经听见,现在要看见这一切;
\par }{\Q 你不说明吗?
\par }{\Q 从今以后,我将新事,
\par }{\Q 就是你所不知道的隐密事指示你。
\par }{\Q \VS{7}这事是现今造的,并非从古就有;
\par }{\Q 在今日以先,你也未曾听见,
\par }{\Q 免得你说:这事我早已知道了。
\par }{\Q \VS{8}你未曾听见,未曾知道;
\par }{\Q 你的耳朵从来未曾开通。
\par }{\Q 我原知道你行事极其诡诈,
\par }{\Q 你自从出胎以来,
\par }{\Q 便称为悖逆的。
\par }{\BB \par }{\Q \VS{9}我为我的名暂且忍怒,
\par }{\Q 为我的颂赞向你容忍,
\par }{\Q 不将你剪除。
\par }{\Q \VS{10}我熬炼你,却不像熬炼银子;
\par }{\Q 你在苦难的炉中,我拣选你。
\par }{\Q \VS{11}我为自己的缘故必行这事,
\par }{\Q 我焉能使{\ADD{我的名}}被亵渎?
\par }{\Q 我必不将我的荣耀归给假神。
\par }{\SH 耶和华拣选塞鲁士为领导者
\par }{\Q \VS{12}{\PN{雅各}}—我所选召的{\PN{以色列}}啊,
\par }{\Q 当听我言:
\par }{\Q 我是耶和华,
\par }{\Q 我是首先的,也是末后的。
\par }{\Q \VS{13}我手立了地的根基;
\par }{\Q 我右手铺张诸天;
\par }{\Q 我一招呼便都立住。
\par }{\BB \par }{\Q \VS{14}你们都当聚集而听,
\par }{\Q 他们\FTNT{}{{\FR 48:14: }或译:偶像}内中谁说过这些事?
\par }{\Q 耶和华所爱的人必向{\PN{巴比伦}}行他所喜悦的事;
\par }{\Q 他的膀臂也要{\ADD{加在}}{\PN{迦勒底}}人身上。
\par }{\Q \VS{15}惟有我曾说过,我又选召他,
\par }{\Q 领他来,他的道路就必亨通。
\par }{\Q \VS{16}你们要就近我来听这话:
\par }{\Q 我从起头并未曾在隐密处说话;
\par }{\Q 自从有这事,我就在那里。
\par }{\Q 现在,主耶和华差遣我和他的灵来\FTNT{}{{\FR 48:16: }或译:耶和华和他的灵差遣我来}。
\par }{\SH 耶和华对他子民的计划
\par }{\Q \VS{17}耶和华—你的救赎主,
\par }{\Q {\PN{以色列}}的圣者如此说:
\par }{\Q 我是耶和华—你的 神,
\par }{\Q 教训你,使你得益处,
\par }{\Q 引导你所当行的路。
\par }{\Q \VS{18}甚愿你素来听从我的命令!
\par }{\Q 你的平安就如河水;
\par }{\Q 你的公义就如海浪。
\par }{\Q \VS{19}你的后裔也必多如{\ADD{海}}沙;
\par }{\Q 你腹中所生的也必多如沙粒。
\par }{\Q 他的名在我面前必不剪除,
\par }{\Q 也不灭绝。
\par }{\BB \par }{\Q \VS{20}你们要从{\PN{巴比伦}}出来,
\par }{\Q 从{\PN{迦勒底}}人中逃脱,
\par }{\Q 以欢呼的声音传扬说:
\par }{\Q 耶和华救赎了他的仆人{\PN{雅各}}!
\par }{\Q 你们要将这事宣扬到地极。
\par }{\Q \VS{21}耶和华引导他们经过沙漠。
\par }{\Q 他们并不干渴;
\par }{\Q 他为他们使水从磐石而流,
\par }{\Q 分裂磐石,水就涌出。
\par }{\Q \VS{22}耶和华说:
\par }{\Q 恶人必不得平安!

\par }\Chap{49}{\SH 外邦之光
\par }{\Q \VerseOne{1}众海岛啊,当听我言!
\par }{\Q 远方的众民哪,留心而听!
\par }{\Q 自我出胎,耶和华就选召我;
\par }{\Q 自出母腹,他就提我的名。
\par }{\Q \VS{2}他使我的口如快刀,
\par }{\Q 将我藏在他手荫之下;
\par }{\Q 又使我成为磨亮的箭,
\par }{\Q 将我藏在他箭袋之中;
\par }{\Q \VS{3}对我说:你是我的仆人{\PN{以色列}};
\par }{\Q 我必因你得荣耀。
\par }{\Q \VS{4}我却说:我劳碌是徒然;
\par }{\Q 我尽力是虚无虚空。
\par }{\Q 然而,我{\ADD{当得}}的理必在耶和华那里;
\par }{\Q 我的赏赐必在我 神那里。
\par }{\BB \par }{\Q \VS{5}耶和华从我出胎,造就我作他的仆人,
\par }{\Q 要使{\PN{雅各}}归向他,
\par }{\Q 使{\PN{以色列}}到他那里聚集。
\par }{\Q 原来耶和华看我为尊贵;
\par }{\Q 我的 神也成为我的力量。
\par }{\Q \VS{6}现在他说:你作我的仆人,
\par }{\Q 使{\PN{雅各}}众支派复兴,
\par }{\Q 使{\PN{以色列}}中得保全的归回尚为小事,
\par }{\Q 我还要使你作外邦人的光,
\par }{\Q 叫你施行我的救恩,直到地极。
\par }{\BB \par }{\Q \VS{7}救赎主—{\PN{以色列}}的圣者耶和华
\par }{\Q 对那被人所藐视、本国所憎恶、
\par }{\Q 官长所虐待的如此说:
\par }{\Q 君王要看见就站起,
\par }{\Q 首领也要下拜;
\par }{\Q 都因信实的耶和华,
\par }{\Q 就是拣选你—{\PN{以色列}}的圣者。
\par }{\SH 耶路撒冷的复兴
\par }{\Q \VS{8}耶和华如此说:
\par }{\Q 在悦纳的时候,我应允了你;
\par }{\Q 在拯救的日子,我济助了你。
\par }{\Q 我要保护你,
\par }{\Q 使你作众民的中保\FTNT{}{{\FR 49:8: }中保:原文是约};
\par }{\Q 复兴遍地,
\par }{\Q 使人承受荒凉之地为业。
\par }{\Q \VS{9}对那被捆绑的人说:出来吧!
\par }{\Q 对那在黑暗的人说:显露吧!
\par }{\Q 他们在路上必得饮食,
\par }{\Q 在一切净光的高处必有食物。
\par }{\Q \VS{10}不饥不渴,
\par }{\Q 炎热和烈日必不伤害他们;
\par }{\Q 因为怜恤他们的必引导他们,
\par }{\Q 领他们到水泉旁边。
\par }{\Q \VS{11}我必使我的众山成为大道;
\par }{\Q 我的大路也被修高。
\par }{\Q \VS{12}看哪,这些从远方来;
\par }{\Q 这些从北方、从西方来;
\par }{\Q 这些从{\PN{秦}}\FTNT{}{{\FR 49:12: }原文是希尼}国来。
\par }{\Q \VS{13}诸天哪,应当欢呼!
\par }{\Q 大地啊,应当快乐!
\par }{\Q 众山哪,应当发声歌唱!
\par }{\Q 因为耶和华已经安慰他的百姓,
\par }{\Q 也要怜恤他困苦之民。
\par }{\BB \par }{\Q \VS{14}{\PN{锡安}}说:耶和华离弃了我;
\par }{\Q 主忘记了我。
\par }{\Q \VS{15}妇人焉能忘记她吃奶的婴孩,
\par }{\Q 不怜恤她所生的儿子?
\par }{\Q 即或有忘记的,
\par }{\Q 我却不忘记你。
\par }{\Q \VS{16}看哪,我将你铭刻在我掌上;
\par }{\Q 你的墙垣常在我眼前。
\par }{\Q \VS{17}你的儿女必急速{\ADD{归回}};
\par }{\Q 毁坏你的,使你荒废的,必都离你出去,
\par }{\Q \VS{18}你举目向四方观看;
\par }{\Q 他们都聚集来到你这里。
\par }{\Q 耶和华说:我指着我的永生起誓:
\par }{\Q 你必要以他们为妆饰佩戴,
\par }{\Q 以他们为华带束腰,像新妇一样。
\par }{\BB \par }{\Q \VS{19}至于你荒废凄凉之处,
\par }{\Q 并你被毁坏之地,
\par }{\Q 现今众民居住必显为太窄;
\par }{\Q 吞灭你的必离你遥远。
\par }{\Q \VS{20}你必听见丧子之后所生的儿女说:
\par }{\Q 这地方我居住太窄,
\par }{\Q 求你给我地方居住。
\par }{\Q \VS{21}那时你心里必说:我既丧子独居,
\par }{\Q 是被掳的,漂流在外。
\par }{\Q 谁给我生这些?
\par }{\Q 谁将这些养大呢?
\par }{\Q 撇下我一人独居的时候,
\par }{\Q 这些在哪里呢?
\par }{\BB \par }{\Q \VS{22}主耶和华如此说:
\par }{\Q 我必向列国举手,
\par }{\Q 向万民竖立大旗;
\par }{\Q 他们必将你的众子怀中抱来,
\par }{\Q 将你的众女肩上扛来。
\par }{\Q \VS{23}列王必作你的养父;
\par }{\Q 王后必作你的乳母。
\par }{\Q 他们必将脸伏地,向你下拜,
\par }{\Q 并舔你脚上的尘土。
\par }{\Q 你便知道我是耶和华;
\par }{\Q 等候我的必不致羞愧。
\par }{\BB \par }{\Q \VS{24}勇士抢去的岂能夺回?
\par }{\Q 该掳掠的岂能解救吗?
\par }{\Q \VS{25}但耶和华如此说:
\par }{\Q 就是勇士所掳掠的,也可以夺回;
\par }{\Q 强暴人所抢的,也可以解救。
\par }{\Q 与你相争的,我必与他相争;
\par }{\Q 我要拯救你的儿女。
\par }{\Q \VS{26}并且我必使那欺压你的吃自己的肉,
\par }{\Q 也要以自己的血喝醉,好像喝甜酒一样。
\par }{\Q 凡有血气的必都知道我—耶和华是你的救主,
\par }{\Q 是你的救赎主,是{\PN{雅各}}的大能者。

\par }\PoetryChap{50}{\Q \VerseOne{1}耶和华如此说:
\par }{\Q 我休你们的母亲,
\par }{\Q 休书在哪里呢?
\par }{\Q 我将你们卖给我哪一个债主呢?
\par }{\Q 你们被卖,是因你们的罪孽;
\par }{\Q 你们的母亲被休,是因你们的过犯。
\par }{\Q \VS{2}我来的时候,为何无人{\ADD{等候}}呢?
\par }{\Q 我呼唤的时候,为何无人答应呢?
\par }{\Q 我的膀臂岂是缩短、不能救赎吗?
\par }{\Q 我岂无拯救之力吗?
\par }{\Q 看哪,我一斥责,海就干了;
\par }{\Q 我使江河变为旷野;
\par }{\Q 其中的鱼因无水腥臭,干渴而死。
\par }{\Q \VS{3}我使诸天以黑暗为衣服,
\par }{\Q 以麻布为遮盖。
\par }{\SH 耶和华仆人的顺服
\par }{\Q \VS{4}主耶和华赐我受教者的舌头,
\par }{\Q 使我知道怎样用言语扶助疲乏的人。
\par }{\Q 主每早晨提醒,
\par }{\Q 提醒我的耳朵,
\par }{\Q 使我能听,像受教者一样。
\par }{\Q \VS{5}主耶和华开通我的耳朵;
\par }{\Q 我并没有违背,也没有退后。
\par }{\Q \VS{6}人打我的背,我任他打;
\par }{\Q 人拔我腮颊的胡须,我由他拔;
\par }{\Q 人辱我,吐我,我并不掩面。
\par }{\BB \par }{\Q \VS{7}主耶和华必帮助我,
\par }{\Q 所以我不抱愧。
\par }{\Q 我硬着脸面好像坚石;
\par }{\Q 我也知道我必不致蒙羞。
\par }{\Q \VS{8}称我为义的与我相近;
\par }{\Q 谁与我争论,
\par }{\Q 可以与我一同站立;
\par }{\Q 谁与我作对,
\par }{\Q 可以就近我来。
\par }{\Q \VS{9}主耶和华要帮助我;
\par }{\Q 谁能定我有罪呢?
\par }{\Q 他们都像衣服渐渐旧了,
\par }{\Q 为蛀虫所咬。
\par }{\BB \par }{\Q \VS{10}你们中间谁是敬畏耶和华、
\par }{\Q 听从他仆人之话的?
\par }{\Q 这人行在暗中,没有亮光。
\par }{\Q 当倚靠耶和华的名,
\par }{\Q 仗赖自己的 神。
\par }{\Q \VS{11}凡你们点火,用火把围绕自己的
\par }{\Q 可以行在你们的火焰里,
\par }{\Q 并你们所点的火把中。
\par }{\Q 这是我手所定的:
\par }{\Q 你们必躺在悲惨之中。

\par }\Chap{51}{\SH 安慰耶路撒冷的话
\par }{\Q \VerseOne{1}你们这追求公义、
\par }{\Q 寻求耶和华的,当听我言!
\par }{\Q 你们要追想被凿而出的磐石,
\par }{\Q 被挖而出的岩穴。
\par }{\Q \VS{2}要追想你们的祖宗{\PN{亚伯拉罕}}
\par }{\Q 和生养你们的{\PN{撒拉}};
\par }{\Q 因为{\PN{亚伯拉罕}}独自一人的时候,
\par }{\Q 我选召他,赐福与他,
\par }{\Q 使他{\ADD{人数}}增多。
\par }{\Q \VS{3}耶和华已经安慰{\PN{锡安}}
\par }{\Q 和{\PN{锡安}}一切的荒场,
\par }{\Q 使旷野像{\PN{伊甸}},
\par }{\Q 使沙漠像耶和华的园囿;
\par }{\Q 在其中必有欢喜、快乐、感谢,
\par }{\Q 和歌唱的声音。
\par }{\BB \par }{\Q \VS{4}我的百姓啊,要向我留心;
\par }{\Q 我的国民哪,要向我侧耳;
\par }{\Q 因为训诲必从我而出;
\par }{\Q 我必坚定我的公理为万民之光。
\par }{\Q \VS{5}我的公义临近;
\par }{\Q 我的救恩发出。
\par }{\Q 我的膀臂要审判万民;
\par }{\Q 海岛都要等候我,倚赖我的膀臂。
\par }{\Q \VS{6}你们要向天举目,
\par }{\Q 观看下地;
\par }{\Q 因为天必像烟云消散,
\par }{\Q 地必如衣服渐渐旧了;
\par }{\Q 其上的居民也要如此死亡\FTNT{}{{\FR 51:6: }如此死亡:或译像蠓虫死亡}。
\par }{\Q 惟有我的救恩永远长存;
\par }{\Q 我的公义也不废掉。
\par }{\Q \VS{7}知道公义、将我训诲存在心中的民,
\par }{\Q 要听我言!
\par }{\Q 不要怕人的辱骂,
\par }{\Q 也不要因人的毁谤惊惶。
\par }{\Q \VS{8}因为蛀虫必咬他们,好像咬衣服;
\par }{\Q 虫子必咬他们,如同咬羊绒。
\par }{\Q 惟有我的公义永远长存,
\par }{\Q 我的救恩直到万代。
\par }{\BB \par }{\Q \VS{9}耶和华的膀臂啊,兴起!兴起!
\par }{\Q 以能力为衣穿上,
\par }{\Q 像古时的年日、上古的世代兴起一样。
\par }{\Q 从前砍碎{\PN{拉哈伯}}、
\par }{\Q 刺透大鱼的,不是你吗?
\par }{\Q \VS{10}使海与深渊的水干涸、
\par }{\Q 使海的深处变为赎民经过之路的,
\par }{\Q 不是你吗?
\par }{\Q \VS{11}耶和华救赎的民必归回,
\par }{\Q 歌唱来到{\PN{锡安}};
\par }{\Q 永乐必归到他们的头上。
\par }{\Q 他们必得着欢喜快乐;
\par }{\Q 忧愁叹息尽都逃避。
\par }{\BB \par }{\Q \VS{12}惟有我,是安慰你们的。
\par }{\Q 你是谁,竟怕那必死的人?
\par }{\Q 怕那要变如草的世人?
\par }{\Q \VS{13}却忘记铺张诸天、立定地基、
\par }{\Q 创造你的耶和华?
\par }{\Q 又因欺压者图谋毁灭要发的暴怒,
\par }{\Q 整天害怕,
\par }{\Q 其实那欺压者的暴怒在哪里呢?
\par }{\Q \VS{14}被掳去的快得释放,
\par }{\Q 必不死而下坑;
\par }{\Q 他的食物也不致缺乏。
\par }{\Q \VS{15}我是耶和华—你的 神—
\par }{\Q 搅动大海,使海中的波浪匉訇—
\par }{\Q 万军之耶和华是我的名。
\par }{\Q \VS{16}我将我的话传给你,
\par }{\Q 用我的手影遮蔽你,
\par }{\Q 为要栽定诸天,立定地基,
\par }{\Q 又对{\PN{锡安}}说:你是我的百姓。
\par }{\SH 耶路撒冷苦难的终结
\par }{\Q \VS{17}{\PN{耶路撒冷}}啊,兴起!
\par }{\Q 兴起!站起来!
\par }{\Q 你从耶和华手中喝了他忿怒之杯,
\par }{\Q 喝了那使人东倒西歪的爵,以致喝尽。
\par }{\Q \VS{18}她所生育的诸子中,没有一个引导她的;
\par }{\Q 她所养大的诸子中,没有一个搀扶她的。
\par }{\Q \VS{19}荒凉、毁灭、饥荒、刀兵,
\par }{\Q 这几样临到你,
\par }{\Q 谁为你举哀?
\par }{\Q 我如何能安慰你呢?
\par }{\Q \VS{20}你的众子发昏,
\par }{\Q 在各市口上躺卧,
\par }{\Q 好像黄羊在网罗之中,
\par }{\Q 都满了耶和华的忿怒
\par }{\Q —你 神的斥责。
\par }{\Q \VS{21}因此,你这困苦却非因酒而醉的,
\par }{\Q 要听我言。
\par }{\Q \VS{22}你的主耶和华—
\par }{\Q 就是为他百姓辨屈的 神如此说:
\par }{\Q 看哪,我已将那使人东倒西歪的杯,
\par }{\Q 就是我忿怒的爵,
\par }{\Q 从你手中接过来;
\par }{\Q 你必不致再喝。
\par }{\BB \par }{\Q \VS{23}我必将这杯递在苦待你的人手中;
\par }{\Q 他们曾对你说:你屈身,
\par }{\Q 由我们践踏过去吧!
\par }{\Q 你便以背为地,
\par }{\Q 好像街市,任人经过。

\par }\Chap{52}{\SH  神要拯救耶路撒冷
\par }{\Q \VerseOne{1}{\PN{锡安}}哪,兴起!兴起!
\par }{\Q 披上你的能力!
\par }{\Q 圣城{\PN{耶路撒冷}}啊,穿上你华美的衣服!
\par }{\Q 因为从今以后,
\par }{\Q 未受割礼、不洁净的必不再进入你中间。
\par }{\Q \VS{2}{\PN{耶路撒冷}}啊,要抖下尘土!
\par }{\Q 起来坐{\ADD{在位上}}!
\par }{\Q {\PN{锡安}}被掳的居民\FTNT{}{{\FR 52:2: }原文是女子}哪,
\par }{\Q 要解开你颈项的锁链!
\par }{\BB \par }{\PP \VS{3}耶和华如此说:「你们是无价被卖的,也必无银被赎。
\VS{4}主耶和华如此说:起先我的百姓下到{\PN{埃及}},在那里寄居,又有{\PN{亚述}}人无故欺压他们。
\VS{5}耶和华说:我的百姓既是无价被掳去,如今我在这里做什么呢?耶和华说:辖制他们的人呼叫,我的名整天受亵渎。
\VS{6}所以,我的百姓必知道我的名;到那日{\ADD{他们必知道}}说这话的就是我。看哪,是我!」
\par }{\BB \par }{\Q \VS{7}那报佳音,传平安,
\par }{\Q 报好信,传救恩的,
\par }{\Q 对{\PN{锡安}}说:你的 神作王了!
\par }{\Q 这人的脚登山何等佳美!
\par }{\Q \VS{8}听啊,你守望之人的声音,
\par }{\Q 他们扬起声来,一同歌唱;
\par }{\Q 因为耶和华归回{\PN{锡安}}的时候,
\par }{\Q 他们必亲眼看见。
\par }{\Q \VS{9}{\PN{耶路撒冷}}的荒场啊,
\par }{\Q 要发起欢声,一同歌唱;
\par }{\Q 因为耶和华安慰了他的百姓,
\par }{\Q 救赎了{\PN{耶路撒冷}}。
\par }{\Q \VS{10}耶和华在万国眼前露出圣臂;
\par }{\Q 地极的人都看见我们 神的救恩了。
\par }{\BB \par }{\Q \VS{11}你们离开吧!离开吧!
\par }{\Q 从{\PN{巴比伦}}出来。
\par }{\Q 不要沾不洁净的物;
\par }{\Q 要从其中出来。
\par }{\Q 你们扛抬耶和华器皿的人哪,
\par }{\Q 务要自洁。
\par }{\Q \VS{12}你们出来必不致急忙,
\par }{\Q 也不致奔逃。
\par }{\Q 因为,耶和华必在你们前头行;
\par }{\Q {\PN{以色列}}的 神必作你们的后盾。
\par }{\SH 受苦的仆人
\par }{\Q \VS{13}我的仆人行事必有智慧\FTNT{}{{\FR 52:13: }或译:行事通达},
\par }{\Q 必被高举上升,
\par }{\Q 且成为至高。
\par }{\Q \VS{14}许多人因他\FTNT{}{{\FR 52:14: }原文是你}惊奇;
\par }{\Q 他的面貌比别人憔悴;
\par }{\Q 他的形容比世人枯槁。
\par }{\Q \VS{15}这样,他必洗净\FTNT{}{{\FR 52:15: }或译:鼓动}许多国民;
\par }{\Q 君王要向他闭口。
\par }{\Q 因所未曾传与他们的,他们必看见;
\par }{\Q 未曾听见的,他们要明白。

\par }\PoetryChap{53}{\Q \VerseOne{1}我们所传的\FTNT{}{{\FR 53:1: }或译:所传与我们的}有谁信呢?
\par }{\Q 耶和华的膀臂向谁显露呢?
\par }{\Q \VS{2}他在耶和华面前生长如嫩芽,
\par }{\Q 像根出于干地。
\par }{\Q 他无佳形美容;
\par }{\Q 我们看见他的时候,
\par }{\Q 也无美貌使我们羡慕他。
\par }{\Q \VS{3}他被藐视,被人厌弃;
\par }{\Q 多受痛苦,常经忧患。
\par }{\Q 他被藐视,
\par }{\Q 好像被人掩面不看的一样;
\par }{\Q 我们也不尊重他。
\par }{\BB \par }{\Q \VS{4}他诚然担当我们的忧患,
\par }{\Q 背负我们的痛苦;
\par }{\Q 我们却以为他受责罚,
\par }{\Q 被 神击打苦待了。
\par }{\Q \VS{5}哪知他为我们的过犯受害,
\par }{\Q 为我们的罪孽压伤。
\par }{\Q 因他受的刑罚,我们得平安;
\par }{\Q 因他受的鞭伤,我们得医治。
\par }{\Q \VS{6}我们都如羊走迷;
\par }{\Q 各人偏行己路;
\par }{\Q 耶和华使我们众人的罪孽都归在他身上。
\par }{\BB \par }{\Q \VS{7}他被欺压,
\par }{\Q 在受苦的时候却不开口
\FTNT{}{{\FR 53:7: }或译:他受欺压,却自卑不开口};
\par }{\Q 他像羊羔被牵到宰杀之地,
\par }{\Q 又像羊在剪毛的人手下无声,
\par }{\Q 他也是这样不开口。
\par }{\Q \VS{8}因受欺压和审判,他被夺去,
\par }{\Q 至于他同世的人,谁想他受鞭打、
\par }{\Q 从活人之地被剪除,
\par }{\Q 是因我百姓的罪过呢?
\par }{\Q \VS{9}他虽然未行强暴,
\par }{\Q 口中也没有诡诈,
\par }{\Q 人还使他与恶人同埋;
\par }{\Q 谁知死的时候与财主同葬。
\par }{\BB \par }{\Q \VS{10}耶和华却定意\FTNT{}{{\FR 53:10: }或译:喜悦}将他压伤,
\par }{\Q 使他受痛苦。
\par }{\Q 耶和华以他为赎罪祭\FTNT{}{{\FR 53:10: }或译:他献本身为赎罪祭}。
\par }{\Q 他必看见后裔,并且延长年日。
\par }{\Q 耶和华所喜悦的事必在他手中亨通。
\par }{\Q \VS{11}他必看见自己劳苦的功效,
\par }{\Q 便心满意足。
\par }{\Q 有许多人因认识我的义仆得称为义;
\par }{\Q 并且他要担当他们的罪孽。
\par }{\Q \VS{12}所以,我要使他与位大的同分,
\par }{\Q 与强盛的均分掳物。
\par }{\Q 因为他将命倾倒,以致于死;
\par }{\Q 他也被列在罪犯之中。
\par }{\Q 他却担当多人的罪,
\par }{\Q 又为罪犯代求。

\par }\Chap{54}{\SH 耶和华对以色列的爱
\par }{\Q \VerseOne{1}你这不怀孕、不生养的要歌唱;
\par }{\Q 你这未曾经过产难的要发声歌唱,扬声欢呼;
\par }{\Q 因为没有丈夫的比有丈夫的儿女更多。
\par }{\Q 这是耶和华说的。
\par }{\Q \VS{2}要扩张你帐幕之地,
\par }{\Q 张大你居所的幔子,不要限止;
\par }{\Q 要放长你的绳子,
\par }{\Q 坚固你的橛子。
\par }{\Q \VS{3}因为你要向左向右开展;
\par }{\Q 你的后裔必得多国为业,
\par }{\Q 又使荒凉的城邑有人居住。
\par }{\BB \par }{\Q \VS{4}不要惧怕,因你必不致蒙羞;
\par }{\Q 也不要抱愧,因你必不致受辱。
\par }{\Q 你必忘记幼年的羞愧,
\par }{\Q 不再记念你寡居的羞辱。
\par }{\Q \VS{5}因为造你的是你的丈夫;
\par }{\Q 万军之耶和华是他的名。
\par }{\Q 救赎你的是{\PN{以色列}}的圣者;
\par }{\Q 他必称为全地之 神。
\par }{\Q \VS{6}耶和华召你,
\par }{\Q 如召被离弃心中忧伤的妻,
\par }{\Q 就是幼年所娶被弃的妻。
\par }{\Q 这是你 神所说的。
\par }{\Q \VS{7}我离弃你不过片时,
\par }{\Q 却要施大恩将你收回。
\par }{\Q \VS{8}我的怒气涨溢,
\par }{\Q 顷刻之间向你掩面,
\par }{\Q 却要以永远的慈爱怜恤你。
\par }{\Q 这是耶和华—你的救赎主说的。
\par }{\BB \par }{\Q \VS{9}这事在我好像{\PN{挪亚}}的洪水。
\par }{\Q 我怎样起誓不再使{\PN{挪亚}}的洪水漫过遍地,
\par }{\Q 我也照样起誓不再向你发怒,
\par }{\Q 也不斥责你。
\par }{\Q \VS{10}大山可以挪开,
\par }{\Q 小山可以迁移;
\par }{\Q 但我的慈爱必不离开你;
\par }{\Q 我平安的约也不迁移。
\par }{\Q 这是怜恤你的耶和华说的。
\par }{\SH 耶路撒冷的将来
\par }{\Q \VS{11}你这受困苦、被风飘荡不得安慰的人哪,
\par }{\Q 我必以彩色安置你的石头,
\par }{\Q 以蓝宝石立定你的根基;
\par }{\Q \VS{12}又以红宝石造你的女墙,
\par }{\Q 以红玉造你的城门,
\par }{\Q 以宝石造你四围的边界\FTNT{}{{\FR 54:12: }或译:外郭}。
\par }{\Q \VS{13}你的儿女都要受耶和华的教训;
\par }{\Q 你的儿女必大享平安。
\par }{\Q \VS{14}你必因公义得坚立,
\par }{\Q 必远离欺压,不致害怕;
\par }{\Q 你必远离惊吓,惊吓必不临近你。
\par }{\Q \VS{15}即或有人聚集,却不由于我;
\par }{\Q 凡聚集攻击你的,必因你仆倒\FTNT{}{{\FR 54:15: }或译:投降你}。
\par }{\Q \VS{16}吹嘘炭火、打造合用器械的铁匠是我所造;
\par }{\Q 残害人、行毁灭的也是我所造。
\par }{\Q \VS{17}凡为攻击你造成的器械必不利用;
\par }{\Q 凡在审判时兴起用舌攻击你的,
\par }{\Q 你必定他为有罪。
\par }{\Q 这是耶和华仆人的产业,
\par }{\Q 是他们从我所得的义。
\par }{\Q 这是耶和华说的。

\par }\Chap{55}{\SH  神要施怜悯
\par }{\Q \VerseOne{1}你们一切干渴的都当就近水来;
\par }{\Q 没有银钱的也可以来。
\par }{\Q 你们都来,买了吃;
\par }{\Q 不用银钱,不用价值,
\par }{\Q 也来买酒和奶。
\par }{\Q \VS{2}你们为何花钱\FTNT{}{{\FR 55:2: }原文是平银}买那不足为食物的?
\par }{\Q 用劳碌得来的买那不使人饱足的呢?
\par }{\Q 你们要留意听我的话就能吃那美物,
\par }{\Q 得享肥甘,心中喜乐。
\par }{\Q \VS{3}你们当就近我来;
\par }{\Q 侧耳而听,就必得活。
\par }{\Q 我必与你们立永约,
\par }{\Q 就是{\ADD{应许}}{\PN{大卫}}那可靠的恩典。
\par }{\Q \VS{4}我已立他作万民的见证,
\par }{\Q 为万民的君王和司令。
\par }{\Q \VS{5}你素不认识的国民,你也必召来;
\par }{\Q 素不认识你的国民也必向你奔跑,
\par }{\Q 都因耶和华—你的 神{\PN{以色列}}的圣者,
\par }{\Q 因为他已经荣耀你。
\par }{\BB \par }{\Q \VS{6}当趁耶和华可寻找的时候寻找他,
\par }{\Q 相近的时候求告他。
\par }{\Q \VS{7}恶人当离弃自己的道路;
\par }{\Q 不义的人当除掉自己的意念。
\par }{\Q 归向耶和华,耶和华就必怜恤他;
\par }{\Q 当归向我们的 神,因为 神必广行赦免。
\par }{\Q \VS{8}耶和华说:我的意念非同你们的意念;
\par }{\Q 我的道路非同你们的道路。
\par }{\Q \VS{9}天怎样高过地,
\par }{\Q 照样,我的道路高过你们的道路;
\par }{\Q 我的意念高过你们的意念。
\par }{\BB \par }{\PP \VS{10}雨雪从天而降,并不返回,
\par }{\Q 却滋润地土,使地上发芽结实,
\par }{\Q 使撒种的有种,使要吃的有粮。
\par }{\Q \VS{11}我口所出的话也必如此,
\par }{\Q 决不徒然返回,
\par }{\Q 却要成就我所喜悦的,
\par }{\Q 在我发他去成就\FTNT{}{{\FR 55:11: }发他去成就:或译所命定}的事上必然亨通。
\par }{\Q \VS{12}你们必欢欢喜喜而出来,
\par }{\Q 平平安安蒙引导。
\par }{\Q 大山小山必在你们面前发声歌唱;
\par }{\Q 田野的树木也都拍掌。
\par }{\Q \VS{13}松树长出,代替荆棘;
\par }{\Q 番石榴长出,代替蒺藜。
\par }{\Q 这要为耶和华留名,
\par }{\Q 作为永远的证据,不能剪除。

\par }\Chap{56}{\SH  神的子民将包括各国的百姓
\par }{\Q \VerseOne{1}耶和华如此说:
\par }{\Q 你们当守公平,行公义;
\par }{\Q 因我的救恩临近,
\par }{\Q 我的公义将要显现。
\par }{\Q \VS{2}谨守安息日而不干犯,
\par }{\Q 禁止己手而不作恶;
\par }{\Q 如此行、如此持守的人便为有福。
\par }{\BB \par }{\Q \VS{3}与耶和华联合的外邦人不要说:
\par }{\Q 耶和华必定将我从他民中分别出来。
\par }{\Q 太监也不要说:我是枯树。
\par }{\Q \VS{4}因为耶和华如此说:
\par }{\Q 那些谨守我的安息日,
\par }{\Q 拣选我所喜悦的事,
\par }{\Q 持守我约的太监,
\par }{\Q \VS{5}我必使他们在我殿中,
\par }{\Q 在我墙内,有记念,有名号,
\par }{\Q 比有儿女的更美。
\par }{\Q 我必赐他们永远的名,不能剪除。
\par }{\Q \VS{6}还有那些与耶和华联合的外邦人,
\par }{\Q 要事奉他,要爱耶和华的名,
\par }{\Q 要作他的仆人—
\par }{\Q 就是凡守安息日不干犯,
\par }{\Q 又持守他\FTNT{}{{\FR 56:6: }原文是我}约的人。
\par }{\Q \VS{7}我必领他们到我的圣山,
\par }{\Q 使他们在祷告我的殿中喜乐。
\par }{\Q 他们的燔祭和{\ADD{平安}}祭,
\par }{\Q 在我坛上必蒙悦纳,
\par }{\Q 因我的殿必称为万民祷告的殿。
\par }{\Q \VS{8}主耶和华,
\par }{\Q 就是招聚{\PN{以色列}}被赶散的,说:
\par }{\Q 在这被招聚的人以外,
\par }{\Q 我还要招聚{\ADD{别人}}归并他们。
\par }{\SH 以色列的领袖被定罪
\par }{\Q \VS{9}田野的诸兽都来吞吃吧!
\par }{\Q 林中的诸兽也要如此。
\par }{\Q \VS{10}他看守的人是瞎眼的,
\par }{\Q 都没有知识,
\par }{\Q 都是哑巴狗,不能叫唤;
\par }{\Q 但知做梦,躺卧,贪睡,
\par }{\Q \VS{11}这些狗贪食,不知饱足。
\par }{\Q 这些牧人不能明白—
\par }{\Q 各人偏行己路,
\par }{\Q 各从各方求自己的利益。
\par }{\Q \VS{12}他们说:来吧!我去拿酒,
\par }{\Q 我们饱饮浓酒;
\par }{\Q 明日必和今日一样,
\par }{\Q 就是{\ADD{宴乐}}无量极大之日。

\par }\Chap{57}{\SH 以色列拜偶像被定罪
\par }{\Q \VerseOne{1}义人死亡,
\par }{\Q 无人放在心上;
\par }{\Q 虔诚人被收去,
\par }{\Q 无人思念。
\par }{\Q 这义人被收去是免了{\ADD{将来的}}祸患;
\par }{\Q \VS{2}他们得享\FTNT{}{{\FR 57:2: }原文是进入}平安。
\par }{\Q 素行正直的,各人在坟里\FTNT{}{{\FR 57:2: }原文是床上}安歇。
\par }{\Q \VS{3}你们这些巫婆的儿子,
\par }{\Q 奸夫和妓女的种子,
\par }{\Q 都要前来!
\par }{\Q \VS{4}你们向谁戏笑?
\par }{\Q 向谁张口吐舌呢?
\par }{\Q 你们岂不是悖逆的儿女,
\par }{\Q 虚谎的种类呢?
\par }{\Q \VS{5}你们在橡树中间,在各青翠树下欲火攻心;
\par }{\Q 在山谷间,在石穴下杀了儿女;
\par }{\Q \VS{6}在谷中光滑{\ADD{石头}}里有你的分。
\par }{\Q 这些就是你所得的分;
\par }{\Q 你也向他浇了奠祭,献了供物,
\par }{\Q 因这事我岂能容忍吗?
\par }{\Q \VS{7}你在高而又高的山上安设床榻,
\par }{\Q 也上那里去献祭。
\par }{\Q \VS{8}你在门后,在门框后,
\par }{\Q 立起你的纪念;
\par }{\Q 向外人赤露,又上去扩张床榻,
\par }{\Q 与他们立约;
\par }{\Q 你在那里看见他们的床就甚喜爱。
\par }{\Q \VS{9}你把油带到王那里,
\par }{\Q 又多加香料,
\par }{\Q 打发使者往远方去,
\par }{\Q 自卑自贱直到阴间,
\par }{\Q \VS{10}你因路远疲倦,
\par }{\Q 却不说这是枉然;
\par }{\Q 你{\ADD{以为}}有复兴之力,
\par }{\Q 所以不觉疲惫。
\par }{\BB \par }{\Q \VS{11}你怕谁?因谁恐惧?
\par }{\Q 竟说谎,不记念我,
\par }{\Q 又不将这事放在心上。
\par }{\Q 我不是许久闭口不言,
\par }{\Q 你仍不怕我吗?
\par }{\Q \VS{12}我要指明你的公义;
\par }{\Q 至于你所行的都必与你无益。
\par }{\Q \VS{13}你哀求的时候,
\par }{\Q 让你所聚集的拯救你吧!
\par }{\Q 风要把他们刮散,
\par }{\Q 一口气要把他们都吹去。
\par }{\Q 但那投靠我的必得地土,
\par }{\Q 必承受我的圣山为业。
\par }{\SH  神应许帮助和医治
\par }{\Q \VS{14}耶和华要说:
\par }{\Q 你们修筑修筑,预备道路,
\par }{\Q 将绊脚石从我百姓的路中除掉。
\par }{\Q \VS{15}因为那至高至上、永远长存\FTNT{}{{\FR 57:15: }原文是住在永远}
\par }{\Q 名为圣者的如此说:
\par }{\Q 我住在至高至圣的所在,
\par }{\Q 也与心灵痛悔谦卑的人同居;
\par }{\Q 要使谦卑人的灵苏醒,
\par }{\Q 也使痛悔人的心苏醒。
\par }{\Q \VS{16}我必不永远相争,也不长久发怒,
\par }{\Q 恐怕我所造的人与灵性都必发昏。
\par }{\Q \VS{17}因他贪婪的罪孽,我就发怒击打他;
\par }{\Q 我向他掩面发怒,
\par }{\Q 他却仍然随心背道。
\par }{\Q \VS{18}我看见他所行的道,也要医治他;
\par }{\Q 又要引导他,
\par }{\Q 使他和那一同伤心的人再得安慰。
\par }{\Q \VS{19}我造就嘴唇的果子;
\par }{\Q 愿平安康泰归与远处的人,
\par }{\Q 也归与近处的人;
\par }{\Q 并且我要医治他。
\par }{\Q 这是耶和华说的。
\par }{\Q \VS{20}惟独恶人,好像翻腾的海,
\par }{\Q 不得平静;
\par }{\Q 其中的水常涌出污秽和淤泥来。
\par }{\Q \VS{21}我的 神说:恶人必不得平安!

\par }\Chap{58}{\SH 真正的禁食
\par }{\Q \VerseOne{1}你要大声喊叫,不可止息;
\par }{\Q 扬起声来,好像吹角。
\par }{\Q 向我百姓说明他们的过犯;
\par }{\Q 向{\PN{雅各}}家说明他们的罪恶。
\par }{\Q \VS{2}他们天天寻求我,
\par }{\Q 乐意明白我的道,
\par }{\Q 好像行义的国民,
\par }{\Q 不离弃他们 神的典章,
\par }{\Q 向我求问公义的判语,
\par }{\Q 喜悦亲近 神。
\par }{\Q \VS{3}他们说:我们禁食,你为何不看见呢?
\par }{\Q 我们刻苦己心,你为何不理会呢?
\par }{\Q 看哪,你们禁食的日子仍求利益,
\par }{\Q 勒逼人为你们做苦工。
\par }{\Q \VS{4}你们禁食,却互相争竞,
\par }{\Q 以凶恶的拳头打人。
\par }{\Q 你们今日禁食,
\par }{\Q 不得使你们的声音听闻于上。
\par }{\Q \VS{5}这样禁食岂是我所拣选、
\par }{\Q 使人刻苦己心的日子吗?
\par }{\Q 岂是叫人垂头像苇子,
\par }{\Q 用麻布和炉灰铺在他以下吗?
\par }{\Q 你这可称为禁食、为耶和华所悦纳的日子吗?
\par }{\BB \par }{\Q \VS{6}我所拣选的禁食不是要松开凶恶的绳,
\par }{\Q 解下轭上的索,
\par }{\Q 使被欺压的得自由,
\par }{\Q 折断一切的轭吗?
\par }{\Q \VS{7}不是要把你的饼分给饥饿的人,
\par }{\Q 将飘流的穷人接到你家中,
\par }{\Q 见赤身的给他{\ADD{衣服}}遮体,
\par }{\Q 顾恤自己的骨肉而不掩藏吗?
\par }{\Q \VS{8}这样,你的光就必发现如早晨的光;
\par }{\Q 你所得的医治要速速发明。
\par }{\Q 你的公义必在你前面行;
\par }{\Q 耶和华的荣光必作你的后盾。
\par }{\Q \VS{9}那时你求告,耶和华必应允;
\par }{\Q 你呼求,他必说:我在这里。
\par }{\BB \par }{\Q 你若从你中间除掉{\ADD{重}}轭
\par }{\Q 和指摘人的指头,并发恶言的事,
\par }{\Q \VS{10}你心若向饥饿的人发怜悯,
\par }{\Q 使困苦的人得满足,
\par }{\Q 你的光就必在黑暗中发现;
\par }{\Q 你的幽暗必变如正午。
\par }{\Q \VS{11}耶和华也必时常引导你,
\par }{\Q 在干旱之地使你心满意足,
\par }{\Q 骨头强壮。
\par }{\Q 你必像浇灌的园子,
\par }{\Q 又像水流不绝的泉源。
\par }{\Q \VS{12}那些出于你的人必修造久已荒废之处;
\par }{\Q 你要建立{\ADD{拆毁}}累代的根基。
\par }{\Q 你必称为补破口的,
\par }{\Q 和重修路径与人居住的。
\par }{\SH 守安息日者的报赏
\par }{\Q \VS{13}你若在安息日掉转\FTNT{}{{\FR 58:13: }或译:谨慎}你的脚步,
\par }{\Q 在我圣日不以{\ADD{操作}}为喜乐,
\par }{\Q 称安息日为可喜乐的,
\par }{\Q 称耶和华的圣日为可尊重的;
\par }{\Q 而且尊敬这日,
\par }{\Q 不办自己的私事,
\par }{\Q 不随自己的私意,
\par }{\Q 不说自己的私话,
\par }{\Q \VS{14}你就以耶和华为乐。
\par }{\Q 耶和华要使你乘驾地的高处,
\par }{\Q 又以你祖{\PN{雅各}}的产业养育你。
\par }{\Q 这是耶和华亲口说的。

\par }\Chap{59}{\SH 先知指责百姓的罪恶
\par }{\Q \VerseOne{1}耶和华的膀臂并非缩短,不能拯救,
\par }{\Q 耳朵并非发沉,不能听见,
\par }{\Q \VS{2}但你们的罪孽使你们与 神隔绝;
\par }{\Q 你们的罪恶使他掩面不听你们。
\par }{\Q \VS{3}因你们的手被血沾染,
\par }{\Q 你们的指头被罪孽沾污,
\par }{\Q 你们的嘴唇说谎言,
\par }{\Q 你们的舌头出恶语。
\par }{\Q \VS{4}无一人按公义告状,
\par }{\Q 无一人凭诚实辨白;
\par }{\Q 都倚靠虚妄,说谎言。
\par }{\Q 所怀的是毒害;
\par }{\Q 所生的是罪孽。
\par }{\Q \VS{5}他们抱毒蛇蛋,
\par }{\Q 结蜘蛛网;
\par }{\Q 人吃这蛋必死。
\par }{\Q 这蛋被踏,必出蝮蛇。
\par }{\Q \VS{6}所结的网不能成为衣服;
\par }{\Q 所做的也不能遮盖自己。
\par }{\Q 他们的行为都是罪孽;
\par }{\Q 手所做的都是强暴。
\par }{\Q \VS{7}他们的脚奔跑行恶;
\par }{\Q 他们急速流无辜人的血;
\par }{\Q 意念都是罪孽,
\par }{\Q 所经过的路都荒凉毁灭。
\par }{\Q \VS{8}平安的路,他们不知道;
\par }{\Q 所行的事没有公平。
\par }{\Q 他们为自己修弯曲的路;
\par }{\Q 凡行此路的都不知道平安。
\par }{\SH 百姓悔罪
\par }{\Q \VS{9}因此,公平离我们远,
\par }{\Q 公义追不上我们。
\par }{\Q 我们指望光亮,却是黑暗,
\par }{\Q 指望光明,却行幽暗。
\par }{\Q \VS{10}我们摸索墙壁,好像瞎子;
\par }{\Q 我们摸索,如同无目之人。
\par }{\Q 我们晌午绊脚,如在黄昏一样;
\par }{\Q 我们在肥壮人中,像死人一般。
\par }{\Q \VS{11}我们咆哮如熊,
\par }{\Q 哀鸣如鸽;
\par }{\Q 指望公平,却是没有;
\par }{\Q 指望救恩,却远离我们。
\par }{\Q \VS{12}我们的过犯在你面前增多,
\par }{\Q 罪恶作见证告我们;
\par }{\Q 过犯与我们同在。
\par }{\Q 至于我们的罪孽,我们都知道:
\par }{\Q \VS{13}就是悖逆、不认识耶和华,
\par }{\Q 转去不跟从我们的 神,
\par }{\Q 说欺压和叛逆的话,
\par }{\Q 心怀谎言,随即说出。
\par }{\Q \VS{14}并且公平转而退后,
\par }{\Q 公义站在远处;
\par }{\Q 诚实在街上仆倒,
\par }{\Q 正直也不得进入。
\par }{\Q \VS{15}诚实少见;
\par }{\Q 离恶的人反成掠物。
\par }{\SH 耶和华准备拯救他的子民
\par }{\Q 那时,耶和华看见没有公平,
\par }{\Q 甚不喜悦。
\par }{\Q \VS{16}他见无人{\ADD{拯救}},
\par }{\Q 无人代求,甚为诧异,
\par }{\Q 就用自己的膀臂施行拯救,
\par }{\Q 以公义扶持自己。
\par }{\Q \VS{17}他以公义为铠甲\FTNT{}{{\FR 59:17: }或译:护心镜},
\par }{\Q 以拯救为头盔,
\par }{\Q 以报仇为衣服,
\par }{\Q 以热心为外袍。
\par }{\Q \VS{18}他必按人的行为施报,
\par }{\Q 恼怒他的敌人,
\par }{\Q 报复他的仇敌
\par }{\Q 向众海岛施行报应。
\par }{\PP \VS{19}如此,人从日落之处必敬畏耶和华的名,
\par }{\Q 从日出之地也必敬畏他的荣耀;
\par }{\Q 因为仇敌好像急流的河水冲来,
\par }{\Q 是耶和华之气所驱逐的。
\par }{\BB \par }{\Q \VS{20}必有一位救赎主来到{\PN{锡安}}—
\par }{\Q {\PN{雅各}}族中转离过犯的人那里。
\par }{\Q 这是耶和华说的。
\par }{\PP \VS{21}耶和华说:「至于我与他们所立的约乃是这样:我加给你的灵,传给你的话,必不离你的口,也不离你后裔与你后裔之后裔的口,从今直到永远;这是耶和华说的。」

\par }\Chap{60}{\SH 耶路撒冷未来的荣耀
\par }{\Q \VerseOne{1}兴起,发光!因为你的光已经来到!
\par }{\Q 耶和华的荣耀发现照耀你。
\par }{\Q \VS{2}看哪,黑暗遮盖大地,
\par }{\Q 幽暗遮盖万民,
\par }{\Q 耶和华却要显现照耀你;
\par }{\Q 他的荣耀要现在你身上。
\par }{\Q \VS{3}万国要来就你的光;
\par }{\Q 君王要来就你发现的光辉。
\par }{\BB \par }{\Q \VS{4}你举目向四方观看;
\par }{\Q 众人都聚集来到你这里。
\par }{\Q 你的众子从远方而来;
\par }{\Q 你的众女也被怀抱而来。
\par }{\Q \VS{5}那时,你看见就有光荣;
\par }{\Q 你心又跳动又宽畅;
\par }{\Q 因为大海丰盛的{\ADD{货物}}必转来归你;
\par }{\Q 列国的财宝也必来归你。
\par }{\Q \VS{6}成群的骆驼,
\par }{\Q 并{\PN{米甸}}和{\PN{以法}}的独峰驼必遮满你;
\par }{\Q {\PN{示巴}}的众人都必来到;
\par }{\Q 要奉上黄金乳香,
\par }{\Q 又要传说耶和华的赞美。
\par }{\Q \VS{7}{\PN{基达}}的羊群都必聚集到你这里,
\par }{\Q {\PN{尼拜约}}的公羊要供你使用,
\par }{\Q 在我坛上必蒙悦纳;
\par }{\Q 我必荣耀我荣耀的殿。
\par }{\BB \par }{\Q \VS{8}那些飞来如云、
\par }{\Q 又如鸽子向窗户飞回的是谁呢?
\par }{\Q \VS{9}众海岛必等候我,
\par }{\Q 首先是{\PN{他施}}的船只,
\par }{\Q 将你的众子连他们的金银从远方一同带来,
\par }{\Q 都为耶和华—你 神的名,
\par }{\Q 又为{\PN{以色列}}的圣者,
\par }{\Q 因为他已经荣耀了你。
\par }{\BB \par }{\Q \VS{10}外邦人必建筑你的城墙;
\par }{\Q 他们的王必服事你。
\par }{\Q 我曾发怒击打你,
\par }{\Q 现今却施恩怜恤你。
\par }{\Q \VS{11}你的城门必时常开放,
\par }{\Q 昼夜不关;
\par }{\Q 使人把列国的财物带来归你,
\par }{\Q 并将他们的君王牵引而来。
\par }{\Q \VS{12}哪一邦哪一国不事奉你,就必灭亡,
\par }{\Q 也必全然荒废。
\par }{\Q \VS{13}{\PN{黎巴嫩}}的荣耀,
\par }{\Q 就是松树、杉树、黄杨树,
\par }{\Q 都必一同归你,
\par }{\Q 为要修饰我圣所之地;
\par }{\Q 我也要使我脚踏之处得荣耀。
\par }{\Q \VS{14}素来苦待你的,他的子孙都必屈身来就你;
\par }{\Q 藐视你的,都要在你脚下跪拜。
\par }{\Q 他们要称你为「耶和华的城」,
\par }{\Q 为「{\PN{以色列}}圣者的{\PN{锡安}}」。
\par }{\BB \par }{\Q \VS{15}你虽然被撇弃被厌恶,
\par }{\Q 甚至无人经过,
\par }{\Q 我却使你变为永远的荣华,
\par }{\Q 成为累代的喜乐。
\par }{\Q \VS{16}你也必吃万国的奶,
\par }{\Q 又吃君王的奶。
\par }{\Q 你便知道我—耶和华是你的救主,
\par }{\Q 是你的救赎主,{\PN{雅各}}的大能者。
\par }{\BB \par }{\Q \VS{17}我要拿金子代替铜,
\par }{\Q 拿银子代替铁,
\par }{\Q 拿铜代替木头,
\par }{\Q 拿铁代替石头;
\par }{\Q 并要以和平为你的官长,
\par }{\Q 以公义为你的监督。
\par }{\Q \VS{18}你地上不再听见强暴的事,
\par }{\Q 境内不再听见荒凉毁灭的事。
\par }{\Q 你必称你的墙为「拯救」,
\par }{\Q 称你的门为「赞美」。
\par }{\BB \par }{\Q \VS{19}日头不再作你白昼的光;
\par }{\Q 月亮也不再发光照耀你。
\par }{\Q 耶和华却要作你永远的光;
\par }{\Q 你 神要为你的荣耀。
\par }{\Q \VS{20}你的日头不再下落;
\par }{\Q 你的月亮也不退缩;
\par }{\Q 因为耶和华必作你永远的光。
\par }{\Q 你悲哀的日子也完毕了。
\par }{\Q \VS{21}你的居民都成为义人,
\par }{\Q 永远得地为业;
\par }{\Q 是我种的栽子,我手的工作,
\par }{\Q 使我得荣耀。
\par }{\Q \VS{22}至小的{\ADD{族}}要加增千倍;
\par }{\Q 微弱的国必成为强盛。
\par }{\Q 我—耶和华要按定期速成这事。

\par }\Chap{61}{\SH 拯救的好消息
\par }{\Q \VerseOne{1}主耶和华的灵在我身上;
\par }{\Q 因为耶和华用膏膏我,
\par }{\Q 叫我传好信息给谦卑的人\FTNT{}{{\FR 61:1: }或译:传福音给贫穷的人},
\par }{\Q 差遣我医好伤心的人,
\par }{\Q 报告被掳的得释放,
\par }{\Q 被囚的出{\ADD{监牢}};
\par }{\Q \VS{2}报告耶和华的恩年,
\par }{\Q 和我们 神报仇的日子;
\par }{\Q 安慰一切悲哀的人,
\par }{\Q \VS{3}赐华冠与{\PN{锡安}}悲哀的人,代替灰尘;
\par }{\Q 喜乐油代替悲哀;
\par }{\Q 赞美衣代替忧伤之灵;
\par }{\Q 使他们称为「公义树」,
\par }{\Q 是耶和华所栽的,叫他得荣耀。
\par }{\Q \VS{4}他们必修造已久的荒场,
\par }{\Q 建立先前凄凉之处,
\par }{\Q 重修历代荒凉之城。
\par }{\BB \par }{\Q \VS{5}那时,外人必起来牧放你们的羊群;
\par }{\Q 外邦人必作你们耕种田地的,
\par }{\Q 修理葡萄园的。
\par }{\Q \VS{6}你们倒要称为耶和华的祭司;
\par }{\Q 人必称你们为我们 神的仆役。
\par }{\Q 你们必吃用列国的财物,
\par }{\Q 因得他们的荣耀自夸。
\par }{\Q \VS{7}你们{\ADD{必得}}加倍{\ADD{的好处}},代替所受的羞辱;
\par }{\Q 分中所得的喜乐,必代替所受的凌辱。
\par }{\Q 在境内必得加倍{\ADD{的产业}};
\par }{\Q 永远之乐必归与你们\FTNT{}{{\FR 61:7: }原文是他们}。
\par }{\BB \par }{\Q \VS{8}因为我—耶和华喜爱公平,
\par }{\Q 恨恶抢夺和罪孽;
\par }{\Q 我要凭诚实施行报应,
\par }{\Q 并要与我的百姓立永约。
\par }{\Q \VS{9}他们的后裔必在列国中被人认识;
\par }{\Q 他们的子孙在众民中也是如此。
\par }{\Q 凡看见他们的必认他们是耶和华赐福的后裔。
\par }{\BB \par }{\Q \VS{10}我因耶和华大大欢喜;
\par }{\Q 我的心靠 神快乐。
\par }{\Q 因他以拯救为衣给我穿上,
\par }{\Q 以公义为袍给我披上,
\par }{\Q 好像新郎戴上华冠,
\par }{\Q 又像新妇佩戴妆饰。
\par }{\Q \VS{11}田地怎样使百谷发芽,
\par }{\Q 园子怎样使所种的发生,
\par }{\Q 主耶和华必照样
\par }{\Q 使公义和赞美在万民中发出。

\par }\PoetryChap{62}{\Q \VerseOne{1}我因{\PN{锡安}}必不静默,
\par }{\Q 为{\PN{耶路撒冷}}必不息声,
\par }{\Q 直到他的公义如光辉发出,
\par }{\Q 他的救恩如明灯发亮。
\par }{\Q \VS{2}列国必见你的公义;
\par }{\Q 列王必见你的荣耀。
\par }{\Q 你必得新名的称呼,
\par }{\Q 是耶和华亲口所起的。
\par }{\Q \VS{3}你在耶和华的手中要作为华冠,
\par }{\Q 在你 神的掌上必作为冕旒。
\par }{\Q \VS{4}你必不再称为「撇弃的」;
\par }{\Q 你的地也不再称为「荒凉的」。
\par }{\Q 你却要称为「我所喜悦的」;
\par }{\Q 你的地也必称为「有夫之妇」。
\par }{\Q 因为耶和华喜悦你,
\par }{\Q 你的地也必归他。
\par }{\Q \VS{5}少年人怎样娶处女,
\par }{\Q 你的众民\FTNT{}{{\FR 62:5: }民:原文是子}也要照样娶你;
\par }{\Q 新郎怎样喜悦新妇,
\par }{\Q 你的 神也要照样喜悦你。
\par }{\BB \par }{\Q \VS{6}{\PN{耶路撒冷}}啊,
\par }{\Q 我在你城上设立守望的,
\par }{\Q 他们昼夜必不静默。
\par }{\Q 呼吁耶和华的,你们不要歇息,
\par }{\Q \VS{7}也不要使他歇息,
\par }{\Q 直等他建立{\PN{耶路撒冷}},
\par }{\Q 使{\PN{耶路撒冷}}在地上成为可赞美的。
\par }{\Q \VS{8}耶和华指着自己的右手和大能的膀臂起誓说:
\par }{\Q 我必不再将你的五谷给你仇敌作食物;
\par }{\Q 外邦人也不再喝你劳碌得来的新酒。
\par }{\Q \VS{9}惟有那收割的要吃,并赞美耶和华;
\par }{\Q 那聚敛的要在我圣所的院内喝。
\par }{\BB \par }{\Q \VS{10}你们当从门经过经过,
\par }{\Q 预备百姓的路;
\par }{\Q 修筑修筑大道,
\par }{\Q 捡去石头,
\par }{\Q 为万民竖立大旗,
\par }{\Q \VS{11}看哪,耶和华曾宣告到地极,
\par }{\Q 对{\PN{锡安}}的居民\FTNT{}{{\FR 62:11: }原文是女子}说:
\par }{\Q 你的拯救者来到。
\par }{\Q 他的赏赐在他那里;
\par }{\Q 他的报应在他面前。
\par }{\Q \VS{12}人必称他们为「圣民」,为「耶和华的赎民」;
\par }{\Q 你也必称为「被眷顾、不撇弃的城」。

\par }\Chap{63}{\SH 耶和华恩待以色列
\par }{\Q \VerseOne{1}这从{\PN{以东}}的{\PN{波斯拉}}来,
\par }{\Q 穿红衣服,
\par }{\Q 装扮华美,
\par }{\Q 能力广大,
\par }{\Q 大步行走的是谁呢?
\par }{\BB \par }{\Q 就是我,
\par }{\Q 是凭公义说话,
\par }{\Q 以大能施行拯救。
\par }{\BB \par }{\Q \VS{2}你的装扮为何有红色?
\par }{\Q 你的衣服为何像踹酒榨的呢?
\par }{\BB \par }{\Q \VS{3}我独自踹酒榨;
\par }{\Q 众民中无一人与我同在。
\par }{\Q 我发怒将他们踹下,
\par }{\Q 发烈怒将他们践踏。
\par }{\Q 他们的血溅在我衣服上,
\par }{\Q 并且污染了我一切的衣裳。
\par }{\Q \VS{4}因为,报仇之日在我心中;
\par }{\Q 救赎我民之年已经来到。
\par }{\Q \VS{5}我仰望,见无人帮助;
\par }{\Q 我诧异,没有人扶持。
\par }{\Q 所以,我自己的膀臂为我施行拯救;
\par }{\Q 我的烈怒将我扶持。
\par }{\Q \VS{6}我发怒,踹下众民;
\par }{\Q 发烈怒,使他们沉醉,
\par }{\Q 又将他们的血倒在地上。
\par }{\SH 耶和华对以色列的良善
\par }{\Q \VS{7}我要照耶和华一切所赐给我们的,
\par }{\Q 提起他的慈爱和美德,
\par }{\Q 并他向{\PN{以色列}}家所施的大恩;
\par }{\Q 这恩是照他的怜恤
\par }{\Q 和丰盛的慈爱赐给他们的。
\par }{\Q \VS{8}他说:他们诚然是我的百姓,
\par }{\Q 不行虚假的子民;
\par }{\Q 这样,他就作了他们的救主。
\par }{\Q \VS{9}他们在一切苦难中,
\par }{\Q 他也同受苦难;
\par }{\Q 并且他面前的使者拯救他们;
\par }{\Q 他以慈爱和怜悯救赎他们;
\par }{\Q 在古时的日子常保抱他们,怀搋他们。
\par }{\BB \par }{\Q \VS{10}他们竟悖逆,使主的圣灵担忧。
\par }{\Q 他就转作他们的仇敌,
\par }{\Q 亲自攻击他们。
\par }{\Q \VS{11}那时,他们\FTNT{}{{\FR 63:11: }原文是他}想起古时的日子—
\par }{\Q {\PN{摩西}}和他百姓,{\ADD{说}}:
\par }{\Q 将百姓和牧养他全群的人
\par }{\Q 从海里领上来的在哪里呢?
\par }{\Q 将他的圣灵降在他们中间的在哪里呢?
\par }{\Q \VS{12}使他荣耀的膀臂在{\PN{摩西}}的右手边行动,
\par }{\Q 在他们前面将水分开,
\par }{\Q 要建立自己永远的名,
\par }{\Q \VS{13}带领他们经过深处,
\par }{\Q 如马行走旷野,
\par }{\Q 使他们不致绊跌的在哪里呢?
\par }{\Q \VS{14}耶和华的灵使他们得安息,
\par }{\Q 仿佛牲畜下到山谷;
\par }{\Q 照样,你也引导你的百姓,
\par }{\Q 要建立自己荣耀的名。
\par }{\SH 求主怜悯帮助
\par }{\Q \VS{15}求你从天上垂顾,
\par }{\Q 从你圣洁荣耀的居所观看。
\par }{\Q 你的热心和你大能的作为在哪里呢?
\par }{\Q 你爱慕的心肠和怜悯向我们止住了。
\par }{\Q \VS{16}{\PN{亚伯拉罕}}虽然不认识我们,
\par }{\Q {\PN{以色列}}也不承认我们,
\par }{\Q 你却是我们的父。
\par }{\Q 耶和华啊,你是我们的父;
\par }{\Q 从万古以来,你名称为「我们的救赎主」。
\par }{\Q \VS{17}耶和华啊,你为何使我们走差离开你的道,
\par }{\Q 使我们心里刚硬、不敬畏你呢?
\par }{\Q 求你为你仆人,
\par }{\Q 为你产业支派的缘故,转回来。
\par }{\Q \VS{18}你的圣民不过暂时得这产业;
\par }{\Q 我们的敌人已经践踏你的圣所。
\par }{\Q \VS{19}我们好像你未曾治理的人,
\par }{\Q 又像未曾得称你名下的人。

\par }\PoetryChap{64}{\Q \VerseOne{1}愿你裂天而降;
\par }{\Q 愿山在你面前震动—
\par }{\Q \VS{2}好像火烧干柴,
\par }{\Q 又像火将水烧开,
\par }{\Q 使你敌人知道你的名,
\par }{\Q 使列国在你面前发颤!
\par }{\Q \VS{3}你曾行我们不能逆料可畏的事。
\par }{\Q 那时你降临,山岭在你面前震动。
\par }{\Q \VS{4}从古以来,人未曾听见、未曾耳闻、未曾眼见
\par }{\Q 在你以外有什么神为等候他的人行事。
\par }{\Q \VS{5}你迎接那欢喜行义、记念你道的人;
\par }{\Q 你曾发怒,我们{\ADD{仍}}犯罪;
\par }{\Q 这景况已久,我们还能得救吗?
\par }{\Q \VS{6}我们都像不洁净的人;
\par }{\Q 所有的义都像污秽的衣服。
\par }{\Q 我们都像叶子渐渐枯干;
\par }{\Q 我们的罪孽好像风把我们吹去。
\par }{\Q \VS{7}并且无人求告你的名;
\par }{\Q 无人奋力抓住你。
\par }{\Q 原来你掩面不顾我们,
\par }{\Q 使我们因罪孽消化。
\par }{\BB \par }{\Q \VS{8}耶和华啊,现在你仍是我们的父!
\par }{\Q 我们是泥,你是窑匠;
\par }{\Q 我们都是你手的工作。
\par }{\Q \VS{9}耶和华啊,求你不要大发震怒,
\par }{\Q 也不要永远记念罪孽。
\par }{\Q 求你垂顾我们,
\par }{\Q 我们都是你的百姓。
\par }{\Q \VS{10}你的圣邑变为旷野。
\par }{\Q {\PN{锡安}}变为旷野;
\par }{\Q {\PN{耶路撒冷}}成为荒场。
\par }{\Q \VS{11}我们圣洁华美的殿—
\par }{\Q 就是我们列祖赞美你的所在被火焚烧;
\par }{\Q 我们所羡慕的美地尽都荒废。
\par }{\Q \VS{12}耶和华啊,有这些事,你还忍得住吗?
\par }{\Q 你仍静默使我们深受苦难吗?

\par }\Chap{65}{\SH  神对悖逆者的惩罚
\par }{\Q \VerseOne{1}素来没有访问{\ADD{我的}},现在求问我;
\par }{\Q 没有寻找我的,我叫他们遇见;
\par }{\Q 没有称为我名下的,我对他们说:
\par }{\Q 我在这里!我在这里!
\par }{\Q \VS{2}我整天伸手招呼那悖逆的百姓;
\par }{\Q 他们随自己的意念行不善之道。
\par }{\Q \VS{3}这百姓时常当面惹我发怒;
\par }{\Q 在园中献祭,
\par }{\Q 在坛\FTNT{}{{\FR 65:3: }原文是砖}上烧香;
\par }{\Q \VS{4}在坟墓间坐着,
\par }{\Q 在隐密处住宿,
\par }{\Q 吃猪肉;
\par }{\Q 他们器皿中有可憎之物做的汤;
\par }{\Q \VS{5}且对人说:你站开吧!
\par }{\Q 不要挨近我,因为我比你圣洁。
\par }{\Q {\ADD{主说}}:这些人是我鼻中的烟,
\par }{\Q 是整天烧着的火。
\par }{\Q \VS{6-7}看哪,这都写在我面前。
\par }{\Q 我必不静默,必施行报应,
\par }{\Q 必将你们的罪孽和你们列祖的罪孽,
\par }{\Q 就是在山上烧香,
\par }{\Q 在冈上亵渎我的罪孽,
\par }{\Q 一同报应在他们后人怀中,
\par }{\Q 我先要把他们所行的量给他们;
\par }{\Q 这是耶和华说的。
\par }{\BB \par }{\Q \VS{8}耶和华如此说:葡萄中寻得新酒,
\par }{\Q 人就说:不要毁坏,
\par }{\Q 因为福在其中。
\par }{\Q 我因我仆人的缘故也必照样而行,
\par }{\Q 不将他们全然毁灭。
\par }{\BB \par }{\Q \VS{9}我必从{\PN{雅各}}中领出后裔,
\par }{\Q 从{\PN{犹大}}中领出承受我众山的。
\par }{\Q 我的选民必承受;
\par }{\Q 我的仆人要在那里居住。
\par }{\Q \VS{10}{\PN{沙
}}{\ADD{平原}}必成为羊群的圈;
\par }{\Q {\PN{亚割谷}}必成为牛群躺卧之处,
\par }{\Q 都为寻求我的民所得。
\par }{\Q \VS{11}但你们这些离弃耶和华、
\par }{\Q 忘记我的圣山、给时运摆筵席\FTNT{}{{\FR 65:11: }原文是桌子}、
\par }{\Q 给天命盛满调和酒的,
\par }{\Q \VS{12}我要命定你们归在刀下,
\par }{\Q 都必屈身被杀;
\par }{\Q 因为我呼唤,你们没有答应;
\par }{\Q 我说话,你们没有听从;
\par }{\Q 反倒行我眼中看为恶的,
\par }{\Q 拣选我所不喜悦的。
\par }{\BB \par }{\Q \VS{13}所以,主耶和华如此说:
\par }{\Q 我的仆人必得吃,你们却饥饿;
\par }{\Q 我的仆人必得喝,你们却干渴;
\par }{\Q 我的仆人必欢喜,你们却蒙羞。
\par }{\Q \VS{14}我的仆人因心中高兴欢呼,
\par }{\Q 你们却因心中忧愁哀哭,
\par }{\Q 又因心里忧伤哀号。
\par }{\Q \VS{15}你们必留下自己的名,
\par }{\Q 为我选民指着赌咒。
\par }{\Q 主耶和华必杀你们,
\par }{\Q 另起别名称呼他的仆人。
\par }{\Q \VS{16}这样,在地上为自己求福的,
\par }{\Q 必凭真实的 神求福;
\par }{\Q 在地上起誓的,
\par }{\Q 必指真实的 神起誓。
\par }{\Q 因为,从前的患难已经忘记,
\par }{\Q 也从我眼前隐藏了。
\par }{\SH 新天新地
\par }{\Q \VS{17}看哪!我造新天新地;
\par }{\Q 从前的事不再被记念,也不再追想。
\par }{\Q \VS{18}你们当因我所造的永远欢喜快乐;
\par }{\Q 因我造{\PN{耶路撒冷}}为人所喜,
\par }{\Q 造其中的居民为人所乐。
\par }{\Q \VS{19}我必因{\PN{耶路撒冷}}欢喜,
\par }{\Q 因我的百姓快乐;
\par }{\Q 其中必不再听见哭泣的声音和哀号的声音。
\par }{\Q \VS{20}其中必没有数日{\ADD{夭亡}}的婴孩,
\par }{\Q 也没有寿数不满的老者;
\par }{\Q 因为百岁死的仍算孩童,
\par }{\Q 有百岁{\ADD{死}}的罪人算被咒诅。
\par }{\Q \VS{21}他们要建造房屋,自己居住;
\par }{\Q 栽种葡萄园,吃其中的果子。
\par }{\Q \VS{22}他们建造的,别人不得住;
\par }{\Q 他们栽种的,别人不得吃;
\par }{\Q 因为我民的日子必像树木的日子;
\par }{\Q 我选民亲手劳碌得来的必长久享用。
\par }{\Q \VS{23}他们必不徒然劳碌,
\par }{\Q 所生产的,也不遭灾害,
\par }{\Q 因为都是蒙耶和华赐福的后裔;
\par }{\Q 他们的子孙也是如此。
\par }{\Q \VS{24}他们尚未求告,我就应允;
\par }{\Q 正说话的时候,我就垂听。
\par }{\Q \VS{25}豺狼必与羊羔同食;
\par }{\Q 狮子必吃草与牛一样;
\par }{\Q 尘土必作蛇的食物。
\par }{\Q 在我圣山的遍处,
\par }{\Q 这一切都不伤人,不害物。
\par }{\Q 这是耶和华说的。

\par }\Chap{66}{\SH 耶和华审判万民
\par }{\Q \VerseOne{1}耶和华如此说:
\par }{\Q 天是我的座位;
\par }{\Q 地是我的脚凳。
\par }{\Q 你们要为我造何等的殿宇?
\par }{\Q 哪里是我安息的地方呢?
\par }{\Q \VS{2}耶和华说:这一切都是我手所造的,
\par }{\Q 所以就都有了。
\par }{\Q 但我所看顾的,
\par }{\Q 就是虚心\FTNT{}{{\FR 66:2: }原文是贫穷}痛悔、
\par }{\Q 因我话而战兢的人。
\par }{\BB \par }{\Q \VS{3}{\ADD{假冒为善的}}宰牛,好像杀人,
\par }{\Q 献羊羔,好像打折狗项,
\par }{\Q 献供物,好像献猪血,
\par }{\Q 烧乳香,好像称颂偶像。
\par }{\Q 这等人拣选自己的道路,
\par }{\Q 心里喜悦行可憎恶的事。
\par }{\Q \VS{4}我也必拣选迷惑他们的事,
\par }{\Q 使他们所惧怕的临到他们;
\par }{\Q 因为我呼唤,无人答应;
\par }{\Q 我说话,他们不听从;
\par }{\Q 反倒行我眼中看为恶的,
\par }{\Q 拣选我所不喜悦的。
\par }{\BB \par }{\Q \VS{5}你们因耶和华言语战兢的人当听他的话:
\par }{\Q 你们的弟兄—就是恨恶你们,
\par }{\Q 因我名赶出你们的,曾说:
\par }{\Q 愿耶和华得荣耀,
\par }{\Q 使我们得见你们的喜乐;
\par }{\Q 但蒙羞的究竟是他们!
\par }{\BB \par }{\Q \VS{6}有喧哗的声音出自城中!
\par }{\Q 有声音出于殿中!
\par }{\Q 是耶和华向仇敌施行报应的声音!
\par }{\BB \par }{\Q \VS{7}{\PN{锡安}}未曾劬劳就生产,
\par }{\Q 未觉疼痛就生出男孩。
\par }{\Q \VS{8}国岂能一日而生?
\par }{\Q 民岂能一时而产?
\par }{\Q 因为{\PN{锡安}}一劬劳便生下儿女,
\par }{\Q 这样的事谁曾听见?
\par }{\Q 谁曾看见呢?
\par }{\Q \VS{9}耶和华说:我既使她临产,
\par }{\Q 岂不使她生产呢?
\par }{\Q 你的 神说:我既使她生产,
\par }{\Q 岂能使她闭{\ADD{胎}}不生呢?
\par }{\BB \par }{\Q \VS{10}你们爱慕{\PN{耶路撒冷}}的
\par }{\Q 都要与她一同欢喜快乐;
\par }{\Q 你们为她悲哀的
\par }{\Q 都要与她一同乐上加乐;
\par }{\Q \VS{11}使你们在她安慰的怀中吃奶得饱,
\par }{\Q 使他们得她丰盛的荣耀,
\par }{\Q 犹如挤奶,满心喜乐。
\par }{\BB \par }{\Q \VS{12}耶和华如此说:
\par }{\Q 我要使平安延及她,好像江河,
\par }{\Q 使列国的荣耀延及她,如同涨溢的河。
\par }{\Q 你们要{\ADD{从中}}享受\FTNT{}{{\FR 66:12: }原文是咂};
\par }{\Q 你们必蒙抱在肋旁,摇弄在膝上。
\par }{\Q \VS{13}母亲怎样安慰儿子,我就照样安慰你们;
\par }{\Q 你们也必因\FTNT{}{{\FR 66:13: }或译:在}{\PN{耶路撒冷}}得安慰。
\par }{\Q \VS{14}你们看见,就心中快乐;
\par }{\Q 你们的骨头必得滋润像嫩草一样;
\par }{\Q 而且耶和华的手向他仆人所行的必被人知道;
\par }{\Q 他也要向仇敌发恼恨。
\par }{\BB \par }{\Q \VS{15}看哪,耶和华必在火中降临;
\par }{\Q 他的车辇像旋风,
\par }{\Q 以烈怒施行报应,
\par }{\Q 以火焰施行责罚;
\par }{\Q \VS{16}因为耶和华在一切有血气的人身上,
\par }{\Q 必以火与刀施行审判;
\par }{\Q 被耶和华所杀的必多。
\par }{\PP \VS{17}「那些分别为圣、洁净自己的,进入园内跟在其中一个人的后头,吃猪肉和仓鼠并可憎之物,他们必一同灭绝;这是耶和华说的。
\par }{\PP \VS{18}「{\ADD{我知道}}他们的行为和他们的意念。{\ADD{时候}}将到,我必将万民万族\FTNT{}{{\FR 66:18: }族:原文是舌}聚来,看见我的荣耀,
\VS{19}我要显神迹\FTNT{}{{\FR 66:19: }或译:记号}在他们中间。逃脱的,我要差到列国去,就是到{\PN{他施}}、{\PN{普勒}}、拉弓的{\PN{路德}}和{\PN{土巴}}、{\PN{雅完}},并素来没有听见我名声、没有看见我荣耀辽远的海岛;他们必将我的荣耀传扬在列国中。
\VS{20}他们必将你们的弟兄从列国中送回,使他们或骑马,或坐车,坐轿,骑骡子,骑独峰驼,到我的圣山{\PN{耶路撒冷}},作为供物献给耶和华,好像{\PN{以色列}}人用洁净的器皿盛供物奉到耶和华的殿中;这是耶和华说的。
\VS{21}耶和华说:我也必从他们中间取人为祭司,为{\PN{利未}}人。」
\par }{\BB \par }{\Q \VS{22}耶和华说:我所要造的新天新地,
\par }{\Q 怎样在我面前长存;
\par }{\Q 你们的后裔和你们的名字也必照样长存。
\par }{\Q \VS{23}每逢月朔、安息日,
\par }{\Q 凡有血气的必来在我面前下拜。
\par }{\Q 这是耶和华说的。
\par }{\Q \VS{24}他们必出去观看那些违背我人的尸首;
\par }{\Q 因为他们的虫是不死的;
\par }{\Q 他们的火是不灭的;
\par }{\Q 凡有血气的都必憎恶他们。
\par }