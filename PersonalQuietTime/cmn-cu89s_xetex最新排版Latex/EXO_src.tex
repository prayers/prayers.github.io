\NormalFont\ShortTitle{出埃及记}
{\MT 出埃及记

\par }\ChapOne{1}{\SH 以色列人在埃及受虐待
\par }{\PP \VerseOne{1}{\PN{以色列}}的众子,各带家眷,和{\PN{雅各}}一同来到{\PN{埃及}}。他们的名字记在下面。
\VS{2}有{\PN{吕便}}、{\PN{西缅}}、{\PN{利未}}、{\PN{犹大}}、
\VS{3}{\PN{以萨迦}}、{\PN{西布伦}}、{\PN{便雅悯}}、
\VS{4}{\PN{但}}、{\PN{拿弗他利}}、{\PN{迦得}}、{\PN{亚设}}。
\VS{5}凡从{\PN{雅各}}而生的,共有七十人。{\PN{约瑟}}已经在{\PN{埃及}}。
\VS{6}{\PN{约瑟}}和他的弟兄,并那一代的人,都死了。
\VS{7}{\PN{以色列}}人生养众多,并且繁茂,极其强盛,满了那地。
\par }{\PP \VS{8}有不认识{\PN{约瑟}}的新王起来,治理{\PN{埃及}},
\VS{9}对他的百姓说:「看哪,这{\PN{以色列}}民比我们还多,又比我们强盛。
\VS{10}来吧,我们不如用巧计待他们,恐怕他们多起来,日后若遇什么争战的事,就连合我们的仇敌攻击我们,离开这地去了。」
\VS{11}于是{\PN{埃及}}人派督工的辖制他们,加重担苦害他们。他们为法老建造两座积货城,就是{\PN{比东}}和{\PN{兰塞}}。
\VS{12}只是越发苦害他们,他们越发多起来,越发蔓延;{\PN{埃及}}人就因{\PN{以色列}}人愁烦。
\VS{13}{\PN{埃及}}人严严地使{\PN{以色列}}人做工,
\VS{14}使他们因做苦工觉得命苦;无论是和泥,是做砖,是做田间各样的工,在一切的工上都严严地待他们。
\par }{\PP \VS{15}有{\PN{希伯来}}的两个收生婆,一名{\PN{施弗拉}},一名{\PN{普阿}};{\PN{埃及}}王对她们说:
\VS{16}「你们为{\PN{希伯来}}妇人收生,看她们临盆的时候,若是男孩,就把他杀了;若是女孩,就留她存活。」
\VS{17}但是收生婆敬畏 神,不照{\PN{埃及}}王的吩咐行,竟存留男孩的性命。
\VS{18}{\PN{埃及}}王召了收生婆来,说:「你们为什么做这事,存留男孩的性命呢?」
\VS{19}收生婆对法老说:「因为{\PN{希伯来}}妇人与{\PN{埃及}}妇人不同;{\PN{希伯来}}妇人本是健壮的\FTNT{}{{\FR 1:19: }原文是活泼的},收生婆还没有到,她们已经生产了。」
\VS{20}神厚待收生婆。{\PN{以色列}}人多起来,极其强盛。
\VS{21}收生婆因为敬畏 神, 神便叫她们成立家室。
\VS{22}法老吩咐他的众民说:「{\PN{以色列}}人所生的男孩,你们都要丢在河里;一切的女孩,你们要存留她的性命。」

\par }\Chap{2}{\SH 摩西的出生
\par }{\PP \VerseOne{1}有一个{\PN{利未}}家的人娶了一个{\PN{利未}}女子为妻。
\VS{2}那女人怀孕,生一个儿子,见他俊美,就藏了他三个月,
\VS{3}后来不能再藏,就取了一个蒲草箱,抹上石漆和石油,将孩子放在里头,把箱子搁在河边的芦荻中。
\VS{4}孩子的姊姊远远站着,要知道他究竟怎么样。
\par }{\PP \VS{5}法老的女儿来到河边洗澡,她的使女们在河边行走。她看见箱子在芦荻中,就打发一个婢女拿来。
\VS{6}她打开箱子,看见那孩子。孩子哭了,她就可怜他,说:「这是{\PN{希伯来}}人的一个孩子。」
\VS{7}孩子的姊姊对法老的女儿说:「我去在{\PN{希伯来}}妇人中叫一个奶妈来,为你奶这孩子,可以不可以?」
\VS{8}法老的女儿说:「可以。」童女就去叫了孩子的母亲来。
\VS{9}法老的女儿对她说:「你把这孩子抱去,为我奶他,我必给你工价。」妇人就抱了孩子去奶他。
\VS{10}孩子渐长,妇人把他带到法老的女儿那里,就作了她的儿子。她给孩子起名叫{\PN{摩西}},意思说:「因我把他从水里拉出来。」
\par }{\SH 摩西逃往米甸
\par }{\PP \VS{11}后来,{\PN{摩西}}长大,他出去到他弟兄那里,看他们的重担,见一个{\PN{埃及}}人打{\PN{希伯来}}人的一个弟兄。
\VS{12}他左右观看,见没有人,就把{\PN{埃及}}人打死了,藏在沙土里。
\VS{13}第二天他出去,见有两个{\PN{希伯来}}人争斗,就对那欺负人的说:「你为什么打你同族的人呢?」
\VS{14}那人说:「谁立你作我们的首领和审判官呢?难道你要杀我,像杀那{\PN{埃及}}人吗?」{\PN{摩西}}便惧怕,说:「这事必是被人知道了。」
\VS{15}法老听见这事,就想杀{\PN{摩西}},但{\PN{摩西}}躲避法老,逃往{\PN{米甸}}地居住。
\par }{\PP \VS{16}一日,他在井旁坐下。{\PN{米甸}}的祭司有七个女儿;她们来打水,打满了槽,要饮父亲的群羊。
\VS{17}有牧羊的人来,把她们赶走了,{\PN{摩西}}却起来帮助她们,又饮了她们的群羊。
\VS{18}她们来到父亲{\PN{流珥}}那里;他说:「今日你们为何来得这么快呢?」
\VS{19}她们说:「有一个{\PN{埃及}}人救我们脱离牧羊人的手,并且为我们打水饮了群羊。」
\VS{20}他对女儿们说:「那个人在哪里?你们为什么撇下他呢?你们去请他来吃饭。」
\VS{21}{\PN{摩西}}甘心和那人同住;那人把他的女儿{\PN{西坡拉}}给{\PN{摩西}}为妻。
\VS{22}{\PN{西坡拉}}生了一个儿子,{\PN{摩西}}给他起名叫{\PN{革舜}},意思说:「因我在外邦作了寄居的。」
\par }{\PP \VS{23}过了多年,{\PN{埃及}}王死了。{\PN{以色列}}人因做苦工,就叹息哀求,他们的哀声达于 神。
\VS{24}神听见他们的哀声,就记念他与{\PN{亚伯拉罕}}、{\PN{以撒}}、{\PN{雅各}}所立的约。
\VS{25}神看顾{\PN{以色列}}人,也知道他们的{\ADD{苦情}}。

\par }\Chap{3}{\SH  神呼召摩西
\par }{\PP \VerseOne{1}{\PN{摩西}}牧养他岳父{\PN{米甸}}祭司{\PN{叶忒罗}}的羊群;一日领羊群往野外去,到了 神的山,就是{\PN{何烈山}}。
\VS{2}耶和华的使者从荆棘里火焰中向{\PN{摩西}}显现。{\PN{摩西}}观看,不料,荆棘被火烧着,却没有烧毁。
\VS{3}{\PN{摩西}}说:「我要过去看这大异象,这荆棘为何没有烧坏呢?」
\VS{4}耶和华 神见他过去要看,就从荆棘里呼叫说:「{\PN{摩西}}!{\PN{摩西}}!」他说:「我在这里。」
\VS{5}神说:「不要近前来。当把你脚上的鞋脱下来,因为你所站之地是圣地」;
\VS{6}又说:「我是你父亲的 神,是{\PN{亚伯拉罕}}的 神,{\PN{以撒}}的 神,{\PN{雅各}}的 神。」{\PN{摩西}}蒙上脸,因为怕看 神。
\par }{\PP \VS{7}耶和华说:「我的百姓在{\PN{埃及}}所受的困苦,我实在看见了;他们因受督工的辖制所发的哀声,我也听见了。我原知道他们的痛苦,
\VS{8}我下来是要救他们脱离{\PN{埃及}}人的手,领他们出了那地,到美好、宽阔、流奶与蜜之地,就是到{\PN{迦南}}人、{\PN{赫}}人、{\PN{亚摩利}}人、{\PN{比利洗}}人、{\PN{希未}}人、{\PN{耶布斯}}人之地。
\VS{9}现在{\PN{以色列}}人的哀声达到我耳中,我也看见{\PN{埃及}}人怎样欺压他们。
\VS{10}故此,我要打发你去见法老,使你可以将我的百姓{\PN{以色列}}人从{\PN{埃及}}领出来。」
\VS{11}{\PN{摩西}}对 神说:「我是什么人,竟能去见法老,将{\PN{以色列}}人从{\PN{埃及}}领出来呢?」
\VS{12}神说:「我必与你同在。你将百姓从{\PN{埃及}}领出来之后,你们必在这山上事奉我;这就是我打发你去的证据。」
\par }{\PP \VS{13}{\PN{摩西}}对 神说:「我到{\PN{以色列}}人那里,对他们说:『你们祖宗的 神打发我到你们这里来。』他们若问我说:『他叫什么名字?』我要对他们说什么呢?」
\VS{14}神对{\PN{摩西}}说:「我是自有永有的」;又说:「你要对{\PN{以色列}}人这样说:『那自有的打发我到你们这里来。』」
\VS{15}神又对{\PN{摩西}}说:「你要对{\PN{以色列}}人这样说:『耶和华—你们祖宗的 神,就是{\PN{亚伯拉罕}}的 神,{\PN{以撒}}的 神,{\PN{雅各}}的 神,打发我到你们这里来。』耶和华是我的名,直到永远;这也是我的纪念,直到万代。
\VS{16}你去招聚{\PN{以色列}}的长老,对他们说:『耶和华—你们祖宗的 神,就是{\PN{亚伯拉罕}}的 神,{\PN{以撒}}的 神,{\PN{雅各}}的 神,向我显现,说:我实在眷顾了你们,我也{\ADD{看见}}{\PN{埃及}}人怎样待你们。
\VS{17}我也说:要将你们从{\PN{埃及}}的困苦中领出来,往{\PN{迦南}}人、{\PN{赫}}人、{\PN{亚摩利}}人、{\PN{比利洗}}人、{\PN{希未}}人、{\PN{耶布斯}}人的地去,就是到流奶与蜜之地。』
\VS{18}他们必听你的话。你和{\PN{以色列}}的长老要去见{\PN{埃及}}王,对他说:『耶和华—{\PN{希伯来}}人的 神遇见了我们,现在求你容我们往旷野去,走三天的路程,为要祭祀耶和华—我们的 神。』
\VS{19}我知道虽用大能的手,{\PN{埃及}}王也不容你们去。
\VS{20}我必伸手在{\PN{埃及}}中间施行我一切的奇事,攻击那地,然后他才容你们去。
\VS{21}我必叫你们在{\PN{埃及}}人眼前蒙恩,你们去的时候就不至于空手而去。
\VS{22}但各妇女必向她的邻舍,并居住在她家里的女人,要金器银器和衣裳,好给你们的儿女穿戴。这样你们就把{\PN{埃及}}人的{\ADD{财物}}夺去了。」

\par }\Chap{4}{\SH  神赐摩西行神迹的权能
\par }{\PP \VerseOne{1}{\PN{摩西}}回答说:「他们必不信我,也不听我的话,必说:『耶和华并没有向你显现。』」
\VS{2}耶和华对{\PN{摩西}}说:「你手里是什么?」他说:「是杖。」
\VS{3}耶和华说:「丢在地上。」他一丢下去,就变作蛇;{\PN{摩西}}便跑开。
\VS{4}耶和华对{\PN{摩西}}说:「伸出手来,拿住它的尾巴,它必在你手中仍变为杖;
\VS{5}如此好叫他们信耶和华—他们祖宗的 神,就是{\PN{亚伯拉罕}}的 神,{\PN{以撒}}的 神,{\PN{雅各}}的 神,是向你显现了。」
\par }{\PP \VS{6}耶和华又对他说:「把手放在怀里。」他就把手放在怀里,及至抽出来,不料,手长了大麻风,有雪{\ADD{那样}}白。
\VS{7}耶和华说:「再把手放在怀里。」他就再把手放在怀里,及至从怀里抽出来,不料,手已经复原,与周身的肉一样;
\VS{8}又说:「倘或他们不听你的话,也不信头一个神迹,他们必信第二个神迹。
\VS{9}这两个神迹若都不信,也不听你的话,你就从河里取些水,倒在旱地上,你从河里取的水必在旱地上变作血。」
\par }{\PP \VS{10}{\PN{摩西}}对耶和华说:「主啊,我素日不是能言的人,就是从你对仆人说话以后,也是这样。我本是拙口笨舌的。」
\VS{11}耶和华对他说:「谁造人的口呢?谁使人口哑、耳聋、目明、眼瞎呢?岂不是我—耶和华吗?
\VS{12}现在去吧,我必赐你口才,指教你所当说的话。」
\VS{13}{\PN{摩西}}说:「主啊,你愿意打发谁,就打发谁去吧!」
\VS{14}耶和华向{\PN{摩西}}发怒说:「不是有你的哥哥{\PN{利未}}人{\PN{亚伦}}吗?我知道他是能言的;现在他出来迎接你,他一见你,心里就欢喜。
\VS{15}你要将当说的话传给他;我也要赐你和他口才,又要指教你们所当行的事。
\VS{16}他要替你对百姓说话;你要以他当作口,他要以你当作 神。
\VS{17}你手里要拿这杖,好行神迹。」
\par }{\SH 摩西返回埃及
\par }{\PP \VS{18}于是,{\PN{摩西}}回到他岳父{\PN{叶忒罗}}那里,对他说:「求你容我回去见我在{\PN{埃及}}的弟兄,看他们还在不在。」{\PN{叶忒罗}}对{\PN{摩西}}说:「你可以平平安安地去吧!」
\VS{19}耶和华在{\PN{米甸}}对{\PN{摩西}}说:「你要回{\PN{埃及}}去,因为寻索你命的人都死了。」
\VS{20}{\PN{摩西}}就带着妻子和两个儿子,叫他们骑上驴,回{\PN{埃及}}地去。{\PN{摩西}}手里拿着 神的杖。
\VS{21}耶和华对{\PN{摩西}}说:「你回到{\PN{埃及}}的时候,要留意将我指示你的一切奇事行在法老面前。但我要使\FTNT{}{{\FR 4:21: }或译:任凭;下同}他的心刚硬,他必不容百姓去。
\VS{22}你要对法老说:『耶和华这样说:{\PN{以色列}}是我的儿子,我的长子。
\VS{23}我对你说过:容我的儿子去,好事奉我。你还是不肯容他去。看哪,我要杀你的长子。』」
\par }{\PP \VS{24}{\PN{摩西}}在路上住宿的地方,耶和华遇见他,想要杀他。
\VS{25}{\PN{西坡拉}}就拿一块火石,割下他儿子的阳皮,丢在{\PN{摩西}}脚前,说:「你真是我的血郎了。」
\VS{26}这样,耶和华才放了他。{\PN{西坡拉}}说:「你因割礼就是血郎了。」
\par }{\PP \VS{27}耶和华对{\PN{亚伦}}说:「你往旷野去迎接{\PN{摩西}}。」他就去,在 神的山遇见{\PN{摩西}},和他亲嘴。
\VS{28}{\PN{摩西}}将耶和华打发他所说的言语和嘱咐他所行的神迹都告诉了{\PN{亚伦}}。
\VS{29}{\PN{摩西}}、{\PN{亚伦}}就去招聚{\PN{以色列}}的众长老。
\VS{30}{\PN{亚伦}}将耶和华对{\PN{摩西}}所说的一切话述说了一遍,又在百姓眼前行了那些神迹,
\VS{31}百姓就信了。{\PN{以色列}}人听见耶和华眷顾他们,鉴察他们的困苦,就低头下拜。

\par }\Chap{5}{\SH 摩西和亚伦见埃及王
\par }{\PP \VerseOne{1}后来{\PN{摩西}}、{\PN{亚伦}}去对法老说:「耶和华—{\PN{以色列}}的 神这样说:『容我的百姓去,在旷野向我守节。』」
\VS{2}法老说:「耶和华是谁,使我听他的话,容{\PN{以色列}}人去呢?我不认识耶和华,也不容{\PN{以色列}}人去!」
\VS{3}他们说:「{\PN{希伯来}}人的 神遇见了我们。求你容我们往旷野去,走三天的路程,祭祀耶和华—我们的 神,免得他用瘟疫、刀兵攻击我们。」
\VS{4}{\PN{埃及}}王对他们说:「{\PN{摩西}}、{\PN{亚伦}}!你们为什么叫百姓旷工呢?你们去担你们的担子吧!」
\VS{5}又说:「看哪,这地的{\PN{以色列}}人如今众多,你们竟叫他们歇下担子!」
\VS{6}当天,法老吩咐督工的和官长说:
\VS{7}「你们不可照常把草给百姓做砖,叫他们自己去捡草。
\VS{8}他们素常做砖的数目,你们仍旧向他们要,一点不可减少;因为他们是懒惰的,所以呼求说:『容我们去祭祀我们的 神。』
\VS{9}你们要把更重的工夫加在这些人身上,叫他们劳碌,不听虚谎的言语。」
\par }{\PP \VS{10}督工的和官长出来对百姓说:「法老这样说:『我不给你们草。
\VS{11}你们自己在哪里能找草,就往那里去找吧!但你们的工一点不可减少。』」
\VS{12}于是百姓散在{\PN{埃及}}遍地,捡碎秸当作草。
\VS{13}督工的催着说:「你们一天当完一天的工,与先前有草一样。」
\VS{14}法老督工的,责打他所派{\PN{以色列}}人的官长,说:「你们昨天今天为什么没有照向来的数目做砖、完你们的工作呢?」
\par }{\PP \VS{15}{\PN{以色列}}人的官长就来哀求法老说:「为什么这样待你的仆人?
\VS{16}督工的不把草给仆人,并且对我们说:『做砖吧!』看哪,你仆人挨了打,其实是你百姓的错。」
\VS{17}但法老说:「你们是懒惰的!你们是懒惰的!所以说:『容我们去祭祀耶和华。』
\VS{18}现在你们去做工吧!草是不给你们的,砖却要如数交纳。」
\VS{19}{\PN{以色列}}人的官长听说「你们每天做砖的工作一点不可减少」,就知道是遭遇祸患了。
\VS{20}他们离了法老出来,正遇见{\PN{摩西}}、{\PN{亚伦}}站在对面,
\VS{21}就向他们说:「愿耶和华鉴察你们,施行判断;因你们使我们在法老和他臣仆面前有了臭名,把刀递在他们手中杀我们。」
\par }{\SH 摩西向耶和华诉苦
\par }{\PP \VS{22}{\PN{摩西}}回到耶和华那里,说:「主啊,你为什么苦待这百姓呢?为什么打发我去呢?
\VS{23}自从我去见法老,奉你的名说话,他就苦待这百姓,你一点也没有拯救他们。」

\par }\Chap{6}{\PP \VerseOne{1}耶和华对{\PN{摩西}}说:「现在你必看见我向法老所行的事,使他因我大能的手容{\PN{以色列}}人去,且把他们赶出他的地。」
\par }{\SH  神呼召摩西
\par }{\PP \VS{2}神晓谕{\PN{摩西}}说:「我是耶和华。
\VS{3}我从前向{\PN{亚伯拉罕}}、{\PN{以撒}}、{\PN{雅各}}显现为全能的 神;至于我名耶和华,他们未曾知道。
\VS{4}我与他们坚定所立的约,要把他们寄居的{\PN{迦南}}地赐给他们。
\VS{5}我也听见{\PN{以色列}}人被{\PN{埃及}}人苦待的哀声,我也记念我的约。
\VS{6}所以你要对{\PN{以色列}}人说:『我是耶和华;我要用伸出来的膀臂重重地刑罚{\PN{埃及}}人,救赎你们脱离他们的重担,不做他们的苦工。
\VS{7}我要以你们为我的百姓,我也要作你们的 神。你们要知道我是耶和华—你们的 神,是救你们脱离{\PN{埃及}}人之重担的。
\VS{8}我起誓应许给{\PN{亚伯拉罕}}、{\PN{以撒}}、{\PN{雅各}}的那地,我要把你们领进去,将那地赐给你们为业。我是耶和华。』」
\VS{9}{\PN{摩西}}将这话告诉{\PN{以色列}}人,只是他们因苦工愁烦,不肯听他的话。
\par }{\PP \VS{10}耶和华晓谕{\PN{摩西}}说:
\VS{11}「你进去对{\PN{埃及}}王法老说,要容{\PN{以色列}}人出他的地。」
\VS{12}{\PN{摩西}}在耶和华面前说:「{\PN{以色列}}人尚且不听我的话,法老怎肯听我这拙口笨舌的人呢?」
\VS{13}耶和华吩咐{\PN{摩西}}、{\PN{亚伦}}往{\PN{以色列}}人和{\PN{埃及}}王法老那里去,把{\PN{以色列}}人从{\PN{埃及}}地领出来。
\par }{\SH 摩西和亚伦的族谱
\par }{\PP \VS{14}{\PN{以色列}}人家长的名字记在下面。{\PN{以色列}}长子{\PN{吕便}}的儿子是{\PN{哈诺}}、{\PN{法路}}、{\PN{希斯伦}}、{\PN{迦米}};这是{\PN{吕便}}的各家。
\VS{15}{\PN{西缅}}的儿子是{\PN{耶母利}}、{\PN{雅悯}}、{\PN{阿辖}}、{\PN{雅斤}}、{\PN{琐辖}},和{\PN{迦南}}女子的儿子{\PN{扫罗}};这是{\PN{西缅}}的各家。
\VS{16}{\PN{利未}}众子的名字按着他们的后代记在下面:就是{\PN{革顺}}、{\PN{哥辖}}、{\PN{米拉利}}。{\PN{利未}}一生的岁数是一百三十七岁。
\VS{17}{\PN{革顺}}的儿子按着家室是{\PN{立尼}}、{\PN{示每}}。
\VS{18}{\PN{哥辖}}的儿子是{\PN{暗兰}}、{\PN{以斯哈}}、{\PN{希伯伦}}、{\PN{乌薛}}。{\PN{哥辖}}一生的岁数是一百三十三岁。
\VS{19}{\PN{米拉利}}的儿子是{\PN{抹利}}和{\PN{母示}};这是{\PN{利未}}的家,都按着他们的后代。
\VS{20}{\PN{暗兰}}娶了他父亲的妹妹{\PN{约基别}}为妻,她给他生了{\PN{亚伦}}和{\PN{摩西}}。{\PN{暗兰}}一生的岁数是一百三十七岁。
\VS{21}{\PN{以斯哈}}的儿子是{\PN{可拉}}、{\PN{尼斐}}、{\PN{细基利}}。
\VS{22}{\PN{乌薛}}的儿子是{\PN{米沙利}}、{\PN{以利撒反}}、{\PN{西提利}}。
\VS{23}{\PN{亚伦}}娶了{\PN{亚米拿达}}的女儿,{\PN{拿顺}}的妹妹,{\PN{以利沙巴}}为妻,她给他生了{\PN{拿答}}、{\PN{亚比户}}、{\PN{以利亚撒}}、{\PN{以他玛}}。
\VS{24}{\PN{可拉}}的儿子是{\PN{亚惜}}、{\PN{以利加拿}}、{\PN{亚比亚撒}};这是{\PN{可拉}}的各家。
\VS{25}{\PN{亚伦}}的儿子{\PN{以利亚撒}}娶了{\PN{普铁}}的一个女儿为妻,她给他生了{\PN{非尼哈}}。这是{\PN{利未}}人的家长,都按着他们的家。
\par }{\PP \VS{26}耶和华说:「将{\PN{以色列}}人按着他们的军队从{\PN{埃及}}地领出来。」这是对那{\PN{亚伦}}、{\PN{摩西}}说的。
\VS{27}对{\PN{埃及}}王法老说要将{\PN{以色列}}人从{\PN{埃及}}领出来的,就是这{\PN{摩西}}、{\PN{亚伦}}。
\par }{\SH 耶和华吩咐摩西和亚伦
\par }{\PP \VS{28}当耶和华在{\PN{埃及}}地对{\PN{摩西}}说话的日子,
\VS{29}他向{\PN{摩西}}说:「我是耶和华;我对你说的一切话,你都要告诉{\PN{埃及}}王法老。」
\VS{30}{\PN{摩西}}在耶和华面前说:「看哪,我是拙口笨舌的人,法老怎肯听我呢?」

\par }\Chap{7}{\PP \VerseOne{1}耶和华对{\PN{摩西}}说:「我使你在法老面前代替 神,你的哥哥{\PN{亚伦}}是替你说话的。
\VS{2}凡我所吩咐你的,你都要说。你的哥哥{\PN{亚伦}}要对法老说,容{\PN{以色列}}人出他的地。
\VS{3}我要使法老的心刚硬,也要在{\PN{埃及}}地多行神迹奇事。
\VS{4}但法老必不听你们;我要伸手重重地刑罚{\PN{埃及}},将我的军队{\PN{以色列}}民从{\PN{埃及}}地领出来。
\VS{5}我伸手攻击{\PN{埃及}},将{\PN{以色列}}人从他们中间领出来的时候,{\PN{埃及}}人就要知道我是耶和华。」
\VS{6}{\PN{摩西}}、{\PN{亚伦}}这样行;耶和华怎样吩咐他们,他们就照样行了。
\VS{7}{\PN{摩西}}、{\PN{亚伦}}与法老说话的时候,{\PN{摩西}}八十岁,{\PN{亚伦}}八十三岁。
\par }{\SH 亚伦的杖
\par }{\PP \VS{8}耶和华晓谕{\PN{摩西}}、{\PN{亚伦}}说:
\VS{9}「法老若对你们说:『你们行件奇事吧!』你就吩咐{\PN{亚伦}}说:『把杖丢在法老面前,使杖变作蛇。』」
\VS{10}{\PN{摩西}}、{\PN{亚伦}}进去见法老,就照耶和华所吩咐的行。{\PN{亚伦}}把杖丢在法老和臣仆面前,杖就变作蛇。
\VS{11}于是法老召了博士和术士来;他们是{\PN{埃及}}行法术的,也用邪术照样而行。
\VS{12}他们各人丢下自己的杖,杖就变作蛇;但{\PN{亚伦}}的杖吞了他们的杖。
\VS{13}法老心里刚硬,不肯听从{\PN{摩西}}、{\PN{亚伦}},正如耶和华所说的。
\par }{\SH 埃及遭灾 血灾
\par }{\PP \VS{14}耶和华对{\PN{摩西}}说:「法老心里固执,不肯容百姓去。
\VS{15}明日早晨,他出来往水边去,你要往河边迎接他,手里要拿着那变过蛇的杖,
\VS{16}对他说:『耶和华—{\PN{希伯来}}人的 神打发我来见你,说:容我的百姓去,好在旷野事奉我。到如今你还是不听。
\VS{17}耶和华这样说:我要用我手里的杖击打河中的水,水就变作血;因此,你必知道我是耶和华。
\VS{18}河里的鱼必死,河也要腥臭,{\PN{埃及}}人就要厌恶吃这河里的水。』」
\VS{19}耶和华晓谕{\PN{摩西}}说:「你对{\PN{亚伦}}说:『把你的杖伸在{\PN{埃及}}所有的水以上,就是在他们的江、河、池、塘以上,叫水都变作血。在{\PN{埃及}}遍地,无论在木器中,石器中,都必有血。』」
\par }{\PP \VS{20}{\PN{摩西}}、{\PN{亚伦}}就照耶和华所吩咐的行。{\PN{亚伦}}在法老和臣仆眼前举杖击打河里的水,河里的水都变作血了。
\VS{21}河里的鱼死了,河也腥臭了,{\PN{埃及}}人就不能吃这河里的水;{\PN{埃及}}遍地都有了血。
\VS{22}{\PN{埃及}}行法术的,也用邪术照样而行。法老心里刚硬,不肯听{\PN{摩西}}、{\PN{亚伦}},正如耶和华所说的。
\VS{23}法老转身进宫,也不把这事放在心上。
\VS{24}{\PN{埃及}}人都在河的两边挖地,要得水喝,因为他们不能喝这河里的水。
\par }{\PP \VS{25}耶和华击打河以后满了七天。

\par }\Chap{8}{\SH —蛙灾
\par }{\PP \VerseOne{1}耶和华吩咐{\PN{摩西}}说:「你进去见法老,对他说:『耶和华这样说:容我的百姓去,好事奉我。
\VS{2}你若不肯容他们去,我必使青蛙糟蹋你的四境。
\VS{3}河里要滋生青蛙;这青蛙要上来进你的宫殿和你的卧房,上你的床榻,进你臣仆的房屋,上你百姓的身上,进你的炉灶和你的抟面盆,
\VS{4}又要上你和你百姓并你众臣仆的身上。』」
\VS{5}耶和华晓谕{\PN{摩西}}说:「你对{\PN{亚伦}}说:『把你的杖伸在江、河、池以上,使青蛙到{\PN{埃及}}地上来。』」
\VS{6}{\PN{亚伦}}便伸杖在{\PN{埃及}}的诸水以上,青蛙就上来,遮满了{\PN{埃及}}地。
\VS{7}行法术的也用他们的邪术照样而行,叫青蛙上了{\PN{埃及}}地。
\par }{\PP \VS{8}法老召了{\PN{摩西}}、{\PN{亚伦}}来,说:「请你们求耶和华使这青蛙离开我和我的民,我就容百姓去祭祀耶和华。」
\VS{9}{\PN{摩西}}对法老说:「任凭你吧,我要何时为你和你的臣仆并你的百姓祈求,除灭青蛙离开你和你的宫殿只留在河里呢?」
\VS{10}他说:「明天。」{\PN{摩西}}说:「可以照你的话吧,好叫你知道没有像耶和华—我们 神的。
\VS{11}青蛙要离开你和你的宫殿,并你的臣仆与你的百姓,只留在河里。」
\VS{12}于是{\PN{摩西}}、{\PN{亚伦}}离开法老出去。{\PN{摩西}}为扰害法老的青蛙呼求耶和华。
\VS{13}耶和华就照{\PN{摩西}}的话行。凡在房里、院中、田间的青蛙都死了。
\VS{14}众人把青蛙聚拢成堆,遍地就都腥臭。
\VS{15}但法老见灾祸松缓,就硬着心,不肯听他们,正如耶和华所说的。
\par }{\SH 虱灾
\par }{\PP \VS{16}耶和华吩咐{\PN{摩西}}说:「你对{\PN{亚伦}}说:『伸出你的杖击打地上的尘土,使尘土在{\PN{埃及}}遍地变作虱子\FTNT{}{{\FR 8:16: }或译:虼蚤;下同}。』」
\VS{17}他们就这样行。{\PN{亚伦}}伸杖击打地上的尘土,就在人身上和牲畜身上有了虱子;{\PN{埃及}}遍地的尘土都变成虱子了。
\VS{18}行法术的也用邪术要生出虱子来,却是不能。于是在人身上和牲畜身上都有了虱子。
\VS{19}行法术的就对法老说:「这是 神的手段。」法老心里刚硬,不肯听{\PN{摩西}}、{\PN{亚伦}},正如耶和华所说的。
\par }{\SH 蝇灾
\par }{\PP \VS{20}耶和华对{\PN{摩西}}说:「你清早起来,法老来到水边,你站在他面前,对他说:『耶和华这样说:容我的百姓去,好事奉我。
\VS{21}你若不容我的百姓去,我要叫成群的苍蝇到你和你臣仆并你百姓的身上,进你的房屋,并且{\PN{埃及}}人的房屋和他们所住的地都要满了成群的苍蝇。
\VS{22}当那日,我必分别我百姓所住的{\PN{歌珊}}地,使那里没有成群的苍蝇,好叫你知道我是天下的耶和华。
\VS{23}我要将我的百姓和你的百姓分别出来。明天必有这神迹。』」
\VS{24}耶和华就这样行。苍蝇成了大群,进入法老的宫殿,和他臣仆的房屋;{\PN{埃及}}遍地就因这成群的苍蝇败坏了。
\par }{\PP \VS{25}法老召了{\PN{摩西}}、{\PN{亚伦}}来,说:「你们去,在这地祭祀你们的 神吧!」
\VS{26}{\PN{摩西}}说:「这样行本不相宜,因为我们要把{\PN{埃及}}人所厌恶的祭祀耶和华—我们的 神;若把{\PN{埃及}}人所厌恶的在他们眼前献为祭,他们岂不拿石头打死我们吗?
\VS{27}我们要往旷野去,走三天的路程,照着耶和华—我们 神所要吩咐我们的祭祀他。」
\VS{28}法老说:「我容你们去,在旷野祭祀耶和华—你们的 神;只是不要走得很远。求你们为我祈祷。」
\VS{29}{\PN{摩西}}说:「我要出去求耶和华,使成群的苍蝇明天离开法老和法老的臣仆并法老的百姓;法老却不可再行诡诈,不容百姓去祭祀耶和华。」
\VS{30}于是{\PN{摩西}}离开法老去求耶和华。
\VS{31}耶和华就照{\PN{摩西}}的话行,叫成群的苍蝇离开法老和他的臣仆并他的百姓,一个也没有留下。
\VS{32}这一次法老又硬着心,不容百姓去。

\par }\Chap{9}{\SH 畜疫之灾
\par }{\PP \VerseOne{1}耶和华吩咐{\PN{摩西}}说:「你进去见法老,对他说:『耶和华—{\PN{希伯来}}人的 神这样说:容我的百姓去,好事奉我。
\VS{2}你若不肯容他们去,仍旧强留他们,
\VS{3}耶和华的手加在你田间的牲畜上,就是在马、驴、骆驼、牛群、羊群上,必有重重的瘟疫。
\VS{4}耶和华要分别{\PN{以色列}}的牲畜和{\PN{埃及}}的牲畜,凡属{\PN{以色列}}人的,一样都不死。』」
\VS{5}耶和华就定了时候,说:「明天耶和华必在此地行这事。」
\VS{6}第二天,耶和华就行这事。{\PN{埃及}}的牲畜{\ADD{几乎}}都死了,只是{\PN{以色列}}人的牲畜,一个都没有死。
\VS{7}法老打发人去{\ADD{看}},谁知{\PN{以色列}}人的牲畜连一个都没有死。法老的心却是固执,不容百姓去。
\par }{\SH 疮灾
\par }{\PP \VS{8}耶和华吩咐{\PN{摩西}}、{\PN{亚伦}}说:「你们取几捧炉灰,{\PN{摩西}}要在法老面前向天扬起来。
\VS{9}这灰要在{\PN{埃及}}全地变作尘土,在人身上和牲畜身上成了起泡的疮。」
\VS{10}{\PN{摩西}}、{\PN{亚伦}}取了炉灰,站在法老面前。{\PN{摩西}}向天扬起来,就在人身上和牲畜身上成了起泡的疮。
\VS{11}行法术的在{\PN{摩西}}面前站立不住,因为在他们身上和一切{\PN{埃及}}人身上都有这疮。
\VS{12}耶和华使法老的心刚硬,不听他们,正如耶和华对{\PN{摩西}}所说的。
\par }{\SH 雹灾
\par }{\PP \VS{13}耶和华对{\PN{摩西}}说:「你清早起来,站在法老面前,对他说:『耶和华—{\PN{希伯来}}人的 神这样说:容我的百姓去,好事奉我。
\VS{14}因为这一次我要叫一切的灾殃临到你和你臣仆并你百姓的身上,叫你知道在普天下没有像我的。
\VS{15}我若伸手用瘟疫攻击你和你的百姓,你早就从地上除灭了。
\VS{16}其实,我叫你存立,是特要向你显我的大能,并要使我的名传遍天下。
\VS{17}你还向我的百姓自高,不容他们去吗?
\VS{18}到明天约在这时候,我必叫重大的冰雹降下,自从{\PN{埃及}}开国以来,没有这样的冰雹。
\VS{19}现在你要打发人把你的牲畜和你田间一切所有的催进来;凡在田间不收回家的,无论是人是牲畜,冰雹必降在他们身上,他们就必死。』」
\VS{20}法老的臣仆中,惧怕耶和华这话的,便叫他的奴仆和牲畜跑进家来。
\VS{21}但那不把耶和华这话放在心上的,就将他的奴仆和牲畜留在田里。
\par }{\PP \VS{22}耶和华对{\PN{摩西}}说:「你向天伸杖,使{\PN{埃及}}遍地的人身上和牲畜身上,并田间各样菜蔬上,都有冰雹。」
\VS{23}{\PN{摩西}}向天伸杖,耶和华就打雷下雹,有火闪到地上;耶和华下雹在{\PN{埃及}}地上。
\VS{24}那时,雹与火搀杂,甚是厉害,自从{\PN{埃及}}成国以来,遍地没有这样的。
\VS{25}在{\PN{埃及}}遍地,雹击打了田间所有的人和牲畜,并一切的菜蔬,又打坏田间一切的树木。
\VS{26}惟独{\PN{以色列}}人所住的{\PN{歌珊}}地没有冰雹。
\par }{\PP \VS{27}法老打发人召{\PN{摩西}}、{\PN{亚伦}}来,对他们说:「这一次我犯了罪了。耶和华是公义的;我和我的百姓是邪恶的。
\VS{28}这雷轰和冰雹已经够了。请你们求耶和华,我就容你们去,不再留住你们。」
\VS{29}{\PN{摩西}}对他说:「我一出城,就要向耶和华举手{\ADD{祷告}};雷必止住,也不再有冰雹,叫你知道全地都是属耶和华的。
\VS{30}至于你和你的臣仆,我知道你们还是不惧怕耶和华 神。」(
\VS{31}那时,麻和大麦被{\ADD{雹}}击打;因为大麦已经吐穗,麻也开了花。
\VS{32}只是小麦和粗麦没有被击打,因为还没有长成。)
\VS{33}{\PN{摩西}}离了法老出城,向耶和华举手{\ADD{祷告}};雷和雹就止住,雨也不再浇在地上了。
\VS{34}法老见雨和雹与雷止住,就越发犯罪;他和他的臣仆都硬着心。
\VS{35}法老的心刚硬,不容{\PN{以色列}}人去,正如耶和华借着{\PN{摩西}}所说的。

\par }\Chap{10}{\SH 蝗灾
\par }{\PP \VerseOne{1}耶和华对{\PN{摩西}}说:「你进去见法老。我使他和他臣仆的心刚硬,为要在他们中间显我这些神迹,
\VS{2}并要叫你将我向{\PN{埃及}}人所做的事,和在他们中间所行的神迹,传于你儿子和你孙子的耳中,好叫你们知道我是耶和华。」
\par }{\PP \VS{3}{\PN{摩西}}、{\PN{亚伦}}就进去见法老,对他说:「耶和华—{\PN{希伯来}}人的 神这样说:『你在我面前不肯自卑要到几时呢?容我的百姓去,好事奉我。
\VS{4}你若不肯容我的百姓去,明天我要使蝗虫进入你的境内,
\VS{5}遮满地面,甚至看不见地,并且吃那冰雹所剩的和田间所长的一切树木。
\VS{6}你的宫殿和你众臣仆的房屋,并一切{\PN{埃及}}人的房屋,都要被蝗虫占满了;自从你祖宗和你祖宗的祖宗在世以来,直到今日,没有见过这样的{\ADD{灾}}。』」{\PN{摩西}}就转身离开法老出去。
\par }{\PP \VS{7}法老的臣仆对法老说:「这人为我们的网罗要到几时呢?容这些人去事奉耶和华—他们的 神吧!{\PN{埃及}}已经败坏了,你还不知道吗?」
\VS{8}于是{\PN{摩西}}、{\PN{亚伦}}被召回来见法老;法老对他们说:「你们去事奉耶和华—你们的 神;但那要去的是谁呢?」
\VS{9}{\PN{摩西}}说:「我们要和我们老的少的、儿子女儿同去,且把羊群牛群一同带去,因为我们务要向耶和华守节。」
\VS{10}法老对他们说:「我容你们和你们{\ADD{妇人}}孩子去的时候,耶和华与你们同在吧!你们要谨慎;因为有祸在你们眼前\FTNT{}{{\FR 10:10: }或译:你们存着恶意},
\VS{11}不可都去!你们这壮年人去事奉耶和华吧,因为这是你们所求的。」于是把他们从法老面前撵出去。
\par }{\PP \VS{12}耶和华对{\PN{摩西}}说:「你向{\PN{埃及}}地伸杖,使蝗虫到{\PN{埃及}}地上来,吃地上一切的菜蔬,就是冰雹所剩的。」
\VS{13}{\PN{摩西}}就向{\PN{埃及}}地伸杖,那一昼一夜,耶和华使东风刮在{\PN{埃及}}地上;到了早晨,东风把蝗虫刮了来。
\VS{14}蝗虫上来,落在{\PN{埃及}}的四境,甚是厉害;以前没有这样的,以后也必没有。
\VS{15}因为这蝗虫遮满地面,甚至地都黑暗了,又吃地上一切的菜蔬和冰雹所剩树上的果子。{\PN{埃及}}遍地,无论是树木,是田间的菜蔬,连一点青的也没有留下。
\VS{16}于是法老急忙召了{\PN{摩西}}、{\PN{亚伦}}来,说:「我得罪耶和华—你们的 神,又得罪了你们。
\VS{17}现在求你,只这一次,饶恕我的罪,求耶和华—你们的 神使我脱离这一次的死亡。」
\VS{18}{\PN{摩西}}就离开法老去求耶和华。
\VS{19}耶和华转了极大的西风,把蝗虫刮起,吹入{\PN{红海}};在{\PN{埃及}}的四境连一个也没有留下。
\VS{20}但耶和华使法老的心刚硬,不容{\PN{以色列}}人去。
\par }{\SH 黑暗之灾
\par }{\PP \VS{21}耶和华对{\PN{摩西}}说:「你向天伸杖,使{\PN{埃及}}地黑暗;这黑暗似乎摸得着。」
\VS{22}{\PN{摩西}}向天伸杖,{\PN{埃及}}遍地就乌黑了三天。
\VS{23}三天之久,人不能相见,谁也不敢起来离开本处;惟有{\PN{以色列}}人家中都有亮光。
\VS{24}法老就召{\PN{摩西}}来,说:「你们去事奉耶和华;只是你们的羊群牛群要留下;你们的{\ADD{妇人}}孩子可以和你们同去。」
\VS{25}{\PN{摩西}}说:「你总要把祭物和燔祭牲交给我们,使我们可以祭祀耶和华—我们的 神。
\VS{26}我们的牲畜也要带去,连一蹄也不留下;因为我们要从其中取出来,事奉耶和华—我们的 神。我们未到那里,还不知道用什么事奉耶和华。」
\VS{27}但耶和华使法老的心刚硬,不肯容他们去。
\VS{28}法老对{\PN{摩西}}说:「你离开我去吧,你要小心,不要再见我的面!因为你见我面的那日你就必死!」
\VS{29}{\PN{摩西}}说:「你说得好!我必不再见你的面了。」

\par }\Chap{11}{\SH 所有头胎的都死
\par }{\PP \VerseOne{1}耶和华对{\PN{摩西}}说:「我再使一样的灾殃临到法老和{\PN{埃及}},然后他必容你们离开这地。他容你们去的时候,总要催逼你们都从这地出去。
\VS{2}你要传于百姓的耳中,叫他们男女各人向邻舍要金器银器。」
\VS{3}耶和华叫百姓在{\PN{埃及}}人眼前蒙恩,并且{\PN{摩西}}在{\PN{埃及}}地、法老臣仆,和百姓的眼中看为极大。
\par }{\PP \VS{4}{\PN{摩西}}说:「耶和华这样说:『约到半夜,我必出去巡行{\PN{埃及}}遍地。
\VS{5}凡在{\PN{埃及}}地,从坐宝座的法老直到磨子后的婢女所有的长子,以及一切头生的牲畜,都必死。
\VS{6}{\PN{埃及}}遍地必有大哀号;从前没有这样的,后来也必没有。
\VS{7}至于{\PN{以色列}}中,无论是人是牲畜,连狗也不敢向他们摇舌,好叫你们知道耶和华是将{\PN{埃及}}人和{\PN{以色列}}人分别出来。』
\VS{8}你这一切臣仆都要俯伏来见我,说:『求你和跟从你的百姓都出去』,然后我要出去。」于是,{\PN{摩西}}气忿忿地离开法老,出去了。
\VS{9}耶和华对{\PN{摩西}}说:「法老必不听你 们,使我的奇事在{\PN{埃及}}地多起来。」
\par }{\PP \VS{10}{\PN{摩西}}、{\PN{亚伦}}在法老面前行了这一切奇事;耶和华使法老的心刚硬,不容{\PN{以色列}}人出离他的地。

\par }\Chap{12}{\SH 逾越节
\par }{\PP \VerseOne{1}耶和华在{\PN{埃及}}地晓谕{\PN{摩西}}、{\PN{亚伦}}说:
\VS{2}「你们要以本月为正月,为一年之首。
\VS{3}你们吩咐{\PN{以色列}}全会众说:本月初十日,各人要按着父家取羊羔,一家一只。
\VS{4}若是一家的人太少,吃不了一只羊羔,本人就要和他隔壁的邻舍共取一只。你们预备羊羔,要按着人数和饭量计算。
\VS{5}要无残疾、一岁的公羊羔,你们或从绵羊里取,或从山羊里取,都可以。
\VS{6}要留到本月十四日,在黄昏的时候,{\PN{以色列}}全会众把羊羔宰了。
\VS{7}各家要取点血,涂在吃羊羔的房屋左右的门框上和门楣上。
\VS{8}当夜要吃羊羔的肉;用火烤了,与无酵饼和苦菜同吃。
\VS{9}不可吃生的,断不可吃水煮的,要带着头、腿、五脏,用火烤了吃。
\VS{10}不可剩下一点留到早晨;若留到早晨,要用火烧了。
\VS{11}你们吃羊羔当腰间束带,脚上穿鞋,手中拿杖,赶紧地吃;这是耶和华的逾越节。
\VS{12}因为那夜我要巡行{\PN{埃及}}地,把{\PN{埃及}}地一切头生的,无论是人是牲畜,都击杀了,又要败坏{\PN{埃及}}一切的神。我是耶和华。
\VS{13}这血要在你们所住的房屋上作记号;我一见这血,就越过你们去。我击杀{\PN{埃及}}地{\ADD{头生}}的时候,灾殃必不临到你们身上灭你们。」
\par }{\SH 无酵节
\par }{\PP \VS{14}「你们要记念这日,守为耶和华的节,作为你们世世代代永远的定例。
\VS{15}你们要吃无酵饼七日。头一日要把酵从你们各家中除去;因为从头一日起,到第七日为止,凡吃有酵之饼的,必从{\PN{以色列}}中剪除。
\VS{16}头一日你们当有圣会,第七日也当有圣会。这两日之内,除了预备各人所要吃的以外,无论何工都不可做。
\VS{17}你们要守无酵节,因为我正当这日把你们的军队从{\PN{埃及}}地领出来。所以,你们要守这日,作为世世代代永远的定例。
\VS{18}从正月十四日晚上,直到二十一日晚上,你们要吃无酵饼。
\VS{19}在你们各家中,七日之内不可有酵;因为凡吃有酵之物的,无论是寄居的,是本地的,必从{\PN{以色列}}的会中剪除。
\VS{20}有酵的物,你们都不可吃;在你们一切住处要吃无酵饼。」
\par }{\SH 第一个逾越节
\par }{\PP \VS{21}于是,{\PN{摩西}}召了{\PN{以色列}}的众长老来,对他们说:「你们要按着家口取出羊羔,把这逾越节{\ADD{的羊羔}}宰了。
\VS{22}拿一把牛膝草,蘸盆里的血,打在门楣上和左右的门框上。你们谁也不可出自己的房门,直到早晨。
\VS{23}因为耶和华要巡行击杀{\PN{埃及}}人,他看见血在门楣上和左右的门框上,就必越过那门,不容灭命的进你们的房屋,击杀你们。
\VS{24}这例,你们要守着,作为你们和你们子孙永远的定例。
\VS{25}日后,你们到了耶和华按着所应许赐给你们的那地,就要守这礼。
\VS{26}你们的儿女问你们说:『行这礼是什么意思?』
\VS{27}你们就说:『这是献给耶和华逾越节的祭。当{\PN{以色列}}人在{\PN{埃及}}的时候,他击杀{\PN{埃及}}人,越过{\PN{以色列}}人的房屋,救了我们各家。』」于是百姓低头下拜。
\par }{\PP \VS{28}耶和华怎样吩咐{\PN{摩西}}、{\PN{亚伦}},{\PN{以色列}}人就怎样行。
\par }{\SH 头胎的被杀
\par }{\PP \VS{29}到了半夜,耶和华把{\PN{埃及}}地所有的长子,就是从坐宝座的法老,直到被掳囚在监里之人的长子,以及一切头生的牲畜,尽都杀了。
\VS{30}法老和一切臣仆,并{\PN{埃及}}众人,夜间都起来了。在{\PN{埃及}}有大哀号,无一家不死一个人的。
\VS{31}夜间,法老召了{\PN{摩西}}、{\PN{亚伦}}来,说:「起来!连你们带{\PN{以色列}}人,从我民中出去,依你们所说的,去事奉耶和华吧!
\VS{32}也依你们所说的,连羊群牛群带着走吧!并要为我祝福。」
\par }{\PP \VS{33}{\PN{埃及}}人催促百姓,打发他们快快出离那地,因为{\PN{埃及}}人说:「我们都要死了。」
\VS{34}百姓就拿着没有酵的生面,把抟面盆包在衣服中,扛在肩头上。
\VS{35}{\PN{以色列}}人照着{\PN{摩西}}的话行,向{\PN{埃及}}人要金器、银器,和衣裳。
\VS{36}耶和华叫百姓在{\PN{埃及}}人眼前蒙恩,以致{\PN{埃及}}人给他们所要的。他们就把{\PN{埃及}}人的{\ADD{财物}}夺去了。
\par }{\SH 以色列人出埃及
\par }{\PP \VS{37}{\PN{以色列}}人从{\PN{兰塞}}起行,往{\PN{疏割}}去;除了{\ADD{妇人}}孩子,步行的男人约有六十万。
\VS{38}又有许多闲杂人,并有羊群牛群,和他们一同上去。
\VS{39}他们用{\PN{埃及}}带出来的生面烤成无酵饼。这生面原没有发起;因为他们被催逼离开{\PN{埃及}},不能耽延,也没有为自己预备什么食物。
\par }{\PP \VS{40}{\PN{以色列}}人住在{\PN{埃及}}共有四百三十年。
\VS{41}正满了四百三十年的那一天,耶和华的军队都从{\PN{埃及}}地出来了。
\VS{42}这夜是耶和华的夜;因耶和华领他们出了{\PN{埃及}}地,所以当向耶和华谨守,是{\PN{以色列}}众人世世代代该谨守的。
\par }{\SH 逾越节的条例
\par }{\PP \VS{43}耶和华对{\PN{摩西}}、{\PN{亚伦}}说:「逾越节的例是这样:外邦人都不可吃这羊羔。
\VS{44}但各人用银子买的奴仆,既受了割礼就可以吃。
\VS{45}寄居的和雇工人都不可吃。
\VS{46}应当在一个房子里吃;不可把一点肉从房子里带到外头去。羊羔的骨头一根也不可折断。
\VS{47}{\PN{以色列}}全会众都要守这礼。
\VS{48}若有外人寄居在你们中间,愿向耶和华守逾越节,他所有的男子务要受割礼,然后才容他前来遵守,他也就像本地人一样;但未受割礼的,都不可吃这羊羔。
\VS{49}本地人和寄居在你们中间的外人同归一例。」
\par }{\PP \VS{50}耶和华怎样吩咐{\PN{摩西}}、{\PN{亚伦}},{\PN{以色列}}众人就怎样行了。
\VS{51}正当那日,耶和华将{\PN{以色列}}人按着他们的军队,从{\PN{埃及}}地领出来。

\par }\Chap{13}{\SH 奉献头胎的
\par }{\PP \VerseOne{1}耶和华晓谕{\PN{摩西}}说:
\VS{2}「{\PN{以色列}}中凡头生的,无论是人是牲畜,都是我的,要分别为圣归我。」
\par }{\SH 除酵节
\par }{\PP \VS{3}{\PN{摩西}}对百姓说:「你们要记念从{\PN{埃及}}为奴之家出来的这日,因为耶和华用大能的手将你们从这地方领出来。有酵的饼都不可吃。
\VS{4}亚笔月间的这日是你们出来的日子。
\VS{5}将来耶和华领你进{\PN{迦南}}人、{\PN{赫}}人、{\PN{亚摩利}}人、{\PN{希未}}人、{\PN{耶布斯}}人之地,就是他向你的祖宗起誓应许给你那流奶与蜜之地,那时你要在这月间守这礼。
\VS{6}你要吃无酵饼七日,到第七日要向耶和华守节。
\VS{7}这七日之久,要吃无酵饼;在你四境之内不可见有酵的饼,也不可见发酵的物。
\VS{8}当那日,你要告诉你的儿子说:『这是因耶和华在我出{\PN{埃及}}的时候为我所行的事。
\VS{9}这要在你手上作记号,在你额上作纪念,使耶和华的律法常在你口中,因为耶和华曾用大能的手将你从{\PN{埃及}}领出来。
\VS{10}所以你每年要按着日期守这例。』」
\par }{\SH 长子
\par }{\PP \VS{11}「将来,耶和华照他向你和你祖宗所起的誓将你领进{\PN{迦南}}人之地,把这地赐给你,
\VS{12}那时你要将一切头生的,并牲畜中头生的,归给耶和华;公的都要属耶和华。
\VS{13}凡头生的驴,你要用羊羔代赎;若不代赎,就要打折它的颈项。凡你儿子中头生的都要赎出来。
\VS{14}日后,你的儿子问你说:『这是什么意思?』你就说:『耶和华用大能的手将我们从{\PN{埃及}}为奴之家领出来。
\VS{15}那时法老几乎不容我们去,耶和华就把{\PN{埃及}}地所有头生的,无论是人是牲畜,都杀了。因此,我把一切头生的公{\ADD{牲畜}}献给耶和华为祭,但将头生的儿子都赎出来。
\VS{16}这要在你手上作记号,在你额上作经文,因为耶和华用大能的手将我们从{\PN{埃及}}领出来。』」
\par }{\SH 云柱和火柱
\par }{\PP \VS{17}法老容百姓去的时候,{\PN{非利士}}地的道路虽近, 神却不领他们从那里走;因为 神说:「恐怕百姓遇见打仗后悔,就回{\PN{埃及}}去。」
\VS{18}所以 神领百姓绕道而行,走{\PN{红海}}旷野的路。{\PN{以色列}}人出{\PN{埃及}}地,都带着兵器上去。
\VS{19}{\PN{摩西}}把{\PN{约瑟}}的骸骨一同带去;因为{\PN{约瑟}}曾叫{\PN{以色列}}人严严地起誓,{\ADD{对他们}}说:「 神必眷顾你们,你们要把我的骸骨从这里一同带上去。」
\VS{20}他们从{\PN{疏割}}起行,在旷野边的{\PN{以倘}}安营。
\VS{21}日间,耶和华在云柱中领他们的路;夜间,在火柱中光照他们,使他们日夜都可以行走。
\VS{22}日间云柱,夜间火柱,总不离开百姓的面前。

\par }\Chap{14}{\SH 过红海
\par }{\PP \VerseOne{1}耶和华晓谕{\PN{摩西}}说:
\VS{2}「你吩咐{\PN{以色列}}人转回,安营在{\PN{比·哈希录}}前,{\PN{密夺}}和海的中间,对着{\PN{巴力·}}
{\PN{洗分}},靠近海边安营。
\VS{3}法老必说:『{\PN{以色列}}人在地中绕迷了,旷野把他们困住了。』
\VS{4}我要使法老的心刚硬,他要追赶他们,我便在法老和他全军身上得荣耀;{\PN{埃及}}人就知道我是耶和华。」于是{\PN{以色列}}人这样行了。
\par }{\PP \VS{5}有人告诉{\PN{埃及}}王说:「百姓逃跑。」法老和他的臣仆就向百姓变心,说:「我们容{\PN{以色列}}人去,不再服事我们,这做的是什么事呢?」
\VS{6}法老就预备他的车辆,带领军兵同去,
\VS{7}并带着六百辆特选的车和{\PN{埃及}}所有的车,每辆都有车兵长。
\VS{8}耶和华使{\PN{埃及}}王法老的心刚硬,他就追赶{\PN{以色列}}人,因为{\PN{以色列}}人是昂然无惧地出{\PN{埃及}}。
\VS{9}{\PN{埃及}}人追赶他们,法老一切的马匹、车辆、马兵,与军兵就在海边上,靠近{\PN{比·哈希录}},对着{\PN{巴力·洗分}},在他们安营的地方追上了。
\par }{\PP \VS{10}法老临近的时候,{\PN{以色列}}人举目看见{\PN{埃及}}人赶来,就甚惧怕,向耶和华哀求。
\VS{11}他们对{\PN{摩西}}说:「难道在{\PN{埃及}}没有坟地,你把我们带来死在旷野吗?你为什么这样待我们,将我们从{\PN{埃及}}领出来呢?
\VS{12}我们在{\PN{埃及}}岂没有对你说过,不要搅扰我们,容我们服事{\PN{埃及}}人吗?因为服事{\PN{埃及}}人比死在旷野还好。」
\VS{13}{\PN{摩西}}对百姓说:「不要惧怕,只管站住!看耶和华今天向你们所要施行的救恩。因为,你们今天所看见的{\PN{埃及}}人必永远不再看见了。
\VS{14}耶和华必为你们争战;你们只管静默,不要作声。」
\VS{15}耶和华对{\PN{摩西}}说:「你为什么向我哀求呢?你吩咐{\PN{以色列}}人往前走。
\VS{16}你举手向海伸杖,把水分开。{\PN{以色列}}人要下海中走干地。
\VS{17}我要使{\PN{埃及}}人的心刚硬,他们就跟着下去。我要在法老和他的全军、车辆、马兵上得荣耀。
\VS{18}我在法老和他的车辆、马兵上得荣耀的时候,{\PN{埃及}}人就知道我是耶和华了。」
\par }{\PP \VS{19}在{\PN{以色列}}营前行走 神的使者,转到他们后边去;云柱也从他们前边转到他们后边立住。
\VS{20}在{\PN{埃及}}营和{\PN{以色列}}营中间有云柱,一边黑暗,一边发光,终夜两下不得相近。
\par }{\PP \VS{21}{\PN{摩西}}向海伸杖,耶和华便用大东风,使海水一夜退去,水便分开,海就成了干地。
\VS{22}{\PN{以色列}}人下海中走干地,水在他们的左右作了墙垣。
\VS{23}{\PN{埃及}}人追赶他们,法老一切的马匹、车辆,和马兵都跟着下到海中。
\VS{24}到了晨更的时候,耶和华从云火柱中向{\PN{埃及}}的军兵观看,使{\PN{埃及}}的军兵混乱了;
\VS{25}又使他们的车轮脱落,难以行走,以致{\PN{埃及}}人说:「我们从{\PN{以色列}}人面前逃跑吧!因耶和华为他们攻击我们了。」
\par }{\PP \VS{26}耶和华对{\PN{摩西}}说:「你向海伸杖,叫水仍合在{\PN{埃及}}人并他们的车辆、马兵身上。」
\VS{27}{\PN{摩西}}就向海伸杖,到了天一亮,海水仍旧复原。{\PN{埃及}}人避水逃跑的时候,耶和华把他们推翻在海中,
\VS{28}水就回流,淹没了车辆和马兵。那些跟着{\PN{以色列}}人下海法老的全军,连一个也没有剩下。
\VS{29}{\PN{以色列}}人却在海中走干地;水在他们的左右作了墙垣。
\par }{\PP \VS{30}当日,耶和华这样拯救{\PN{以色列}}人脱离{\PN{埃及}}人的手,{\PN{以色列}}人看见{\PN{埃及}}人的死尸都在海边了。
\VS{31}{\PN{以色列}}人看见耶和华向{\PN{埃及}}人所行的大事,就敬畏耶和华,又信服他和他的仆人{\PN{摩西}}。

\par }\Chap{15}{\SH 摩西的歌
\par }{\PP \VerseOne{1}那时,{\PN{摩西}}和{\PN{以色列}}人向耶和华唱歌说:
\par }{\Q 我要向耶和华歌唱,因他大大战胜,
\par }{\Q 将马和骑马的投在海中。
\par }{\Q \VS{2}耶和华是我的力量,我的诗歌,
\par }{\Q 也成了我的拯救。
\par }{\Q 这是我的 神,我要赞美他,
\par }{\Q 是我父亲的 神,我要尊崇他。
\par }{\Q \VS{3}耶和华是战士;
\par }{\Q 他的名是耶和华。
\par }{\BB \par }{\Q \VS{4}法老的车辆、军兵,耶和华已抛在海中;
\par }{\Q 他特选的军长都沉于{\PN{红海}}。
\par }{\Q \VS{5}深水淹没他们;
\par }{\Q 他们如同石头坠到深处。
\par }{\Q \VS{6}耶和华啊,你的右手施展能力,显出荣耀;
\par }{\Q 耶和华啊,你的右手摔碎仇敌。
\par }{\Q \VS{7}你大发威严,推翻那些起来攻击你的;
\par }{\Q 你发出烈怒{\ADD{如火}},烧灭他们像烧碎秸一样。
\par }{\Q \VS{8}你发鼻中的气,水便聚起成堆,
\par }{\Q 大水直立如垒,
\par }{\Q 海中的深水凝结。
\par }{\Q \VS{9}仇敌说:我要追赶,我要追上;
\par }{\Q 我要分掳物,我要在他们身上称我的心愿。
\par }{\Q 我要拔出刀来,亲手杀灭他们。
\par }{\Q \VS{10}你叫风一吹,海就把他们淹没;
\par }{\Q 他们如铅沉在大水之中。
\par }{\BB \par }{\Q \VS{11}耶和华啊,众神之中,谁能像你?
\par }{\Q 谁能像你—至圣至荣,
\par }{\Q 可颂可畏,施行奇事?
\par }{\Q \VS{12}你伸出右手,
\par }{\Q 地便吞灭他们。
\par }{\BB \par }{\Q \VS{13}你凭慈爱领了你所赎的百姓;
\par }{\Q 你凭能力引他们到了你的圣所。
\par }{\Q \VS{14}外邦人听见就发颤;
\par }{\Q 疼痛抓住{\PN{非利士}}的居民。
\par }{\Q \VS{15}那时,{\PN{以东}}的族长惊惶,
\par }{\Q {\PN{摩押}}的英雄被战兢抓住,
\par }{\Q {\PN{迦南}}的居民{\ADD{心}}都消化了。
\par }{\Q \VS{16}惊骇恐惧临到他们。
\par }{\Q 耶和华啊,因你膀臂的大能,
\par }{\Q 他们如石头寂然不动,
\par }{\Q 等候你的百姓过去,
\par }{\Q 等候你所赎的百姓过去。
\par }{\Q \VS{17}你要将他们领进去,栽于你产业的山上—
\par }{\Q 耶和华啊,就是你为自己所造的住处;
\par }{\Q 主啊,就是你手所建立的圣所。
\par }{\Q \VS{18}耶和华必作王,直到永永远远!
\par }{\BB \par }{\SH 米利暗的歌
\par }{\PP \VS{19}法老的马匹、车辆,和马兵下到海中,耶和华使海水回流,淹没他们;惟有{\PN{以色列}}人在海中走干地。
\VS{20}{\PN{亚伦}}的姊姊,女先知{\PN{米利暗}},手里拿着鼓;众妇女也跟她出去拿鼓跳舞。
\VS{21}{\PN{米利暗}}应声说:
\par }{\Q 你们要歌颂耶和华,因他大大战胜,
\par }{\Q 将马和骑马的投在海中。
\par }{\SH 苦水
\par }{\PP \VS{22}{\PN{摩西}}领{\PN{以色列}}人从{\PN{红海}}往前行,到了{\PN{书珥}}的旷野,在旷野走了三天,找不着水。
\VS{23}到了{\PN{玛拉}},不能喝那里的水;因为水苦,所以那地名叫{\PN{玛拉}}。
\VS{24}百姓就向{\PN{摩西}}发怨言,说:「我们喝什么呢?」
\VS{25}{\PN{摩西}}呼求耶和华,耶和华指示他一棵树。他把树丢在水里,水就变甜了。
\par }{\PP 耶和华在那里为他们定了律例、典章,在那里试验他们;
\VS{26}又说:「你若留意听耶和华—你 神的话,又行我眼中看为正的事,留心听我的诫命,守我一切的律例,我就不将所加与{\PN{埃及}}人的疾病加在你身上,因为我—耶和华是医治你的。」
\par }{\PP \VS{27}他们到了{\PN{以琳}},在那里有十二股水泉,七十棵棕树;他们就在那里的水边安营。

\par }\Chap{16}{\SH 吗哪和鹌鹑
\par }{\PP \VerseOne{1}{\PN{以色列}}全会众从{\PN{以琳}}起行,在出{\PN{埃及}}后第二个月十五日到了{\PN{以琳}}和{\PN{西奈}}中间、{\PN{汛}}的旷野。
\VS{2}{\PN{以色列}}全会众在旷野向{\PN{摩西}}、{\PN{亚伦}}发怨言,
\VS{3}说:「巴不得我们早死在{\PN{埃及}}地、耶和华的手下;那时我们坐在肉锅旁边,吃得饱足。你们将我们领出来,到这旷野,是要叫这全会众都饿死啊!」
\par }{\PP \VS{4}耶和华对{\PN{摩西}}说:「我要将粮食从天降给你们。百姓可以出去,每天收每天的分,我好试验他们遵不遵我的法度。
\VS{5}到第六天,他们要把所收进来的预备好了,比每天所收的多一倍。」
\VS{6}{\PN{摩西}}、{\PN{亚伦}}对{\PN{以色列}}众人说:「到了晚上,你们要知道是耶和华将你们从{\PN{埃及}}地领出来的。
\VS{7}早晨,你们要看见耶和华的荣耀,因为耶和华听见你们向他所发的怨言了。我们算什么,你们竟向我们发怨言呢?」
\VS{8}{\PN{摩西}}又说:「耶和华晚上必给你们肉吃,早晨必给你们食物得饱;因为你们向耶和华发的怨言,他都听见了。我们算什么,你们的怨言不是向我们发的,乃是向耶和华发的。」
\par }{\PP \VS{9}{\PN{摩西}}对{\PN{亚伦}}说:「你告诉{\PN{以色列}}全会众说:『你们就近耶和华面前,因为他已经听见你们的怨言了。』」
\VS{10}{\PN{亚伦}}正对{\PN{以色列}}全会众说话的时候,他们向旷野观看,不料,耶和华的荣光在云中显现。
\VS{11}耶和华晓谕{\PN{摩西}}说:
\VS{12}「我已经听见{\PN{以色列}}人的怨言。你告诉他们说:『到黄昏的时候,你们要吃肉,早晨必有食物得饱,你们就知道我是耶和华—你们的 神。』」
\par }{\PP \VS{13}到了晚上,有鹌鹑飞来,遮满了营;早晨在营四围的地上有露水。
\VS{14}露水上升之后,不料,野地面上有如白霜的小圆物。
\VS{15}{\PN{以色列}}人看见,不知道是什么,就彼此对问说:「这是什么呢?」{\PN{摩西}}对他们说:「这就是耶和华给你们吃的食物。
\VS{16}耶和华所吩咐的是这样:你们要按着各人的饭量,为帐棚里的人,按着人数收起来,各拿一俄梅珥。」
\VS{17}{\PN{以色列}}人就这样行;有多收的,有少收的。
\VS{18}及至用俄梅珥量一量,多收的也没有余,少收的也没有缺;各人按着自己的饭量收取。
\VS{19}{\PN{摩西}}对他们说:「所收的,不许什么人留到早晨。」
\VS{20}然而他们不听{\PN{摩西}}的话,内中有留到早晨的,就生虫变臭了;{\PN{摩西}}便向他们发怒。
\VS{21}他们每日早晨,按着各人的饭量收取,日头一发热,就消化了。
\par }{\PP \VS{22}到第六天,他们收了双倍的食物,每人两俄梅珥。会众的官长来告诉{\PN{摩西}};
\VS{23}{\PN{摩西}}对他们说:「耶和华这样说:『明天是圣安息日,是向耶和华守的圣安息日。你们要烤的就烤了,要煮的就煮了,所剩下的都留到早晨。』」
\VS{24}他们就照{\PN{摩西}}的吩咐留到早晨,也不臭,里头也没有虫子。
\VS{25}{\PN{摩西}}说:「你们今天吃这个吧!因为今天是向耶和华守的安息日;你们在田野必找不着了。
\VS{26}六天可以收取,第七天乃是安息日,那一天必没有了。」
\VS{27}第七天,百姓中有人出去收,什么也找不着。
\VS{28}耶和华对{\PN{摩西}}说:「你们不肯守我的诫命和律法,要到几时呢?
\VS{29}你们看!耶和华既将安息日赐给你们,所以第六天他赐给你们两天的食物,第七天各人要住在自己的地方,不许什么人出去。」
\VS{30}于是百姓第七天安息了。
\par }{\PP \VS{31}这食物,{\PN{以色列}}家叫吗哪;样子像芫荽子,颜色是白的,滋味如同搀蜜的薄饼。
\VS{32}{\PN{摩西}}说:「耶和华所吩咐的是这样:『要将一满俄梅珥吗哪留到世世代代,使后人可以看见我当日将你们领出{\PN{埃及}}地,在旷野所给你们吃的食物。』」
\VS{33}{\PN{摩西}}对{\PN{亚伦}}说:「你拿一个罐子,盛一满俄梅珥吗哪,存在耶和华面前,要留到世世代代。」
\VS{34}耶和华怎么吩咐{\PN{摩西}},{\PN{亚伦}}就怎么行,把吗哪放在法{\ADD{柜}}前存留。
\VS{35}{\PN{以色列}}人吃吗哪共四十年,直到进了有人居住之地,就是{\PN{迦南}}的境界。(
\VS{36}俄梅珥就是伊法十分之一。)

\par }\Chap{17}{\SH 磐石出水
\par }{\R (民20·1—13)
\par }{\PP \VerseOne{1}{\PN{以色列}}全会众都遵耶和华的吩咐,按着站口从{\PN{汛}}的旷野往前行,在{\PN{利非订}}安营。百姓没有水喝,
\VS{2}所以与{\PN{摩西}}争闹,说:「给我们水喝吧!」{\PN{摩西}}对他们说:「你们为什么与我争闹?为什么试探耶和华呢?」
\VS{3}百姓在那里甚渴,要喝水,就向{\PN{摩西}}发怨言,说:「你为什么将我们从{\PN{埃及}}领出来,使我们和我们的儿女并牲畜都渴死呢?」
\VS{4}{\PN{摩西}}就呼求耶和华说:「我向这百姓怎样行呢?他们几乎要拿石头打死我。」
\VS{5}耶和华对{\PN{摩西}}说:「你手里拿着你先前击打河水的杖,带领{\PN{以色列}}的几个长老,从百姓面前走过去。
\VS{6}我必在{\PN{何烈}}的磐石那里,站在你面前。你要击打磐石,从磐石里必有水流出来,使百姓可以喝。」{\PN{摩西}}就在{\PN{以色列}}的长老眼前这样行了。
\VS{7}他给那地方起名叫{\PN{玛撒}}\FTNT{}{{\FR 17:7: }就是试探的意思},又叫{\PN{米利巴}}\FTNT{}{{\FR 17:7: }就是争闹的意思};因{\PN{以色列}}人争闹,又因他们试探耶和华,说:「耶和华是在我们中间不是?」
\par }{\SH 跟亚玛力人争战
\par }{\PP \VS{8}那时,{\PN{亚玛力}}人来在{\PN{利非订}},和{\PN{以色列}}人争战。
\VS{9}{\PN{摩西}}对{\PN{约书亚}}说:「你为我们选出人来,出去和{\PN{亚玛力}}人争战。明天我手里要拿着 神的杖,站在山顶上。」
\VS{10}于是{\PN{约书亚}}照着{\PN{摩西}}对他所说的话行,和{\PN{亚玛力}}人争战。{\PN{摩西}}、{\PN{亚伦}},与{\PN{户珥}}都上了山顶。
\VS{11}{\PN{摩西}}何时举手,{\PN{以色列}}人就得胜,何时垂手,{\PN{亚玛力}}人就得胜。
\VS{12}但{\PN{摩西}}的手发沉,他们就搬石头来,放在他以下,他就坐在上面。{\PN{亚伦}}与{\PN{户珥}}扶着他的手,一个在这边,一个在那边,他的手就稳住,直到日落的时候。
\VS{13}{\PN{约书亚}}用刀杀了{\PN{亚玛力}}王和他的百姓。
\par }{\PP \VS{14}耶和华对{\PN{摩西}}说:「我要将{\PN{亚玛力}}的名号从天下全然涂抹了;你要将这话写在书上作纪念,又念给{\PN{约书亚}}听。」
\VS{15}{\PN{摩西}}筑了一座坛,起名叫「耶和华尼西」\FTNT{}{{\FR 17:15: }就是耶和华是我旌旗的意思},
\VS{16}又说:「耶和华已经起了誓,必世世代代和{\PN{亚玛力}}人争战。」

\par }\Chap{18}{\SH 叶忒罗访问摩西
\par }{\PP \VerseOne{1}{\PN{摩西}}的岳父,{\PN{米甸}}祭司{\PN{叶忒罗}},听见 神为{\PN{摩西}}和 神的百姓{\PN{以色列}}所行的一切事,就是耶和华将{\PN{以色列}}从{\PN{埃及}}领出来的事,
\VS{2}便带着{\PN{摩西}}的妻子{\PN{西坡拉}},就是{\PN{摩西}}从前打发回去的,
\VS{3}又带着{\PN{西坡拉}}的两个儿子,一个名叫{\PN{革舜}},因为{\PN{摩西}}说:「我在外邦作了寄居的」;
\VS{4}一个名叫{\PN{以利以谢}},因为{\ADD{他说}}:「我父亲的 神帮助了我,救我脱离法老的刀。」
\VS{5}{\PN{摩西}}的岳父{\PN{叶忒罗}}带着{\PN{摩西}}的妻子和两个儿子来到 神的山,就是{\PN{摩西}}在旷野安营的地方。
\VS{6}他对{\PN{摩西}}说:「我是你岳父{\PN{叶忒罗}},带着你的妻子和两个儿子来到你这里。」
\VS{7}{\PN{摩西}}迎接他的岳父,向他下拜,与他亲嘴,彼此问安,都进了帐棚。
\VS{8}{\PN{摩西}}将耶和华为{\PN{以色列}}的缘故向法老和{\PN{埃及}}人所行的一切事,以及路上所遭遇的一切艰难,并耶和华怎样搭救他们,都述说与他岳父听。
\VS{9}{\PN{叶忒罗}}因耶和华待{\PN{以色列}}的一切好处,就是拯救他们脱离{\PN{埃及}}人的手,便甚欢喜。
\par }{\PP \VS{10}{\PN{叶忒罗}}说:「耶和华是应当称颂的;他救了你们脱离{\PN{埃及}}人和法老的手,将这百姓从{\PN{埃及}}人的手下救出来。
\VS{11}我现今在{\PN{埃及}}人向这百姓发狂傲的事上得知,耶和华比万神都大。」
\VS{12}{\PN{摩西}}的岳父{\PN{叶忒罗}}把燔祭和{\ADD{平安}}祭献给 神。{\PN{亚伦}}和{\PN{以色列}}的众长老都来了,与{\PN{摩西}}的岳父在 神面前吃饭。
\par }{\SH 选立百姓的官长
\par }{\R (申1·9—18)
\par }{\PP \VS{13}第二天,{\PN{摩西}}坐着审判百姓,百姓从早到晚都站在{\PN{摩西}}的左右。
\VS{14}{\PN{摩西}}的岳父看见他向百姓所做的一切事,就说:「你向百姓做的是什么事呢?你为什么独自坐着,众百姓从早到晚都站在你的左右呢?」
\VS{15}{\PN{摩西}}对岳父说:「这是因百姓到我这里来求问 神。
\VS{16}他们有事的时候就到我这里来,我便在两造之间施行审判;我又叫他们知道 神的律例和法度。」
\VS{17}{\PN{摩西}}的岳父说:「你这做的不好。
\VS{18}你和这些百姓必都疲惫;因为这事太重,你独自一人办理不了。
\VS{19}现在你要听我的话。我为你出个主意,愿 神与你同在。你要替百姓到 神面前,将案件奏告 神;
\VS{20}又要将律例和法度教训他们,指示他们当行的道,当做的事;
\VS{21}并要从百姓中拣选有才能的人,就是敬畏 神、诚实无妄、恨不义之财的人,派他们作千夫长、百夫长、五十夫长、十夫长,管理百姓,
\VS{22}叫他们随时审判百姓,大事都要呈到你这里,小事他们自己可以审判。这样,你就轻省些,他们也可以同当此任。
\VS{23}你若这样行, 神也这样吩咐你,你就能受得住,这百姓也都平平安安归回他们的住处。」
\par }{\PP \VS{24}于是,{\PN{摩西}}听从他岳父的话,按着他所说的去行。
\VS{25}{\PN{摩西}}从{\PN{以色列}}人中拣选了有才能的人,立他们为百姓的首领,作千夫长、百夫长、五十夫长、十夫长。
\VS{26}他们随时审判百姓,有难断的案件就呈到{\PN{摩西}}那里,但各样小事他们自己审判。
\par }{\PP \VS{27}此后,{\PN{摩西}}让他的岳父去,他就往本地去了。

\par }\Chap{19}{\SH 以色列人在西奈山
\par }{\PP \VerseOne{1}{\PN{以色列}}人出{\PN{埃及}}地以后,满了三个月的那一天,就来到{\PN{西奈}}的旷野。
\VS{2}他们离了{\PN{利非订}},来到{\PN{西奈}}的旷野,就在那里的山下安营。
\VS{3}{\PN{摩西}}到 神那里,耶和华从山上呼唤他说:「你要这样告诉{\PN{雅各}}家,晓谕{\PN{以色列}}人说:
\VS{4}『我向{\PN{埃及}}人所行的事,你们都看见了,且看见我如鹰将你们背在翅膀上,带来归我。
\VS{5}如今你们若实在听从我的话,遵守我的约,就要在万民中作属我的子民,因为全地都是我的。
\VS{6}你们要归我作祭司的国度,为圣洁的国民。』这些话你要告诉{\PN{以色列}}人。」
\par }{\PP \VS{7}{\PN{摩西}}去召了民间的长老来,将耶和华所吩咐他的话都在他们面前陈明。
\VS{8}百姓都同声回答说:「凡耶和华所说的,我们都要遵行。」{\PN{摩西}}就将百姓的话回复耶和华。
\VS{9}耶和华对{\PN{摩西}}说:「我要在密云中临到你那里,叫百姓在我与你说话的时候可以听见,也可以永远信你了。」于是,{\PN{摩西}}将百姓的话奏告耶和华。
\par }{\PP \VS{10}耶和华又对{\PN{摩西}}说:「你往百姓那里去,叫他们今天明天自洁,又叫他们洗衣服。
\VS{11}到第三天要预备好了,因为第三天耶和华要在众百姓眼前降临在{\PN{西奈山}}上。
\VS{12}你要在山的四围给百姓定界限,说:『你们当谨慎,不可上山去,也不可摸山的边界;凡摸这山的,必要治死他。
\VS{13}不可用手摸他,必用石头打死,或用{\ADD{箭}}射透;无论是人是牲畜,都不得活。到角声拖长的时候,他们才可到山根来。』」
\VS{14}{\PN{摩西}}下山往百姓那里去,叫他们自洁,他们就洗衣服。
\VS{15}他对百姓说:「到第三天要预备好了。不可亲近女人。」
\par }{\PP \VS{16}到了第三天早晨,在山上有雷轰、闪电,和密云,并且角声甚大,营中的百姓尽都发颤。
\VS{17}{\PN{摩西}}率领百姓出营迎接 神,都站在山下。
\VS{18}{\PN{西奈}}全山冒烟,因为耶和华在火中降于山上。山的烟气上腾,如烧窑一般,遍山大大地震动。
\VS{19}角声渐渐地高而又高,{\PN{摩西}}就说话, 神有声音答应他。
\VS{20}耶和华降临在{\PN{西奈山}}顶上,耶和华召{\PN{摩西}}上山顶,{\PN{摩西}}就上去。
\VS{21}耶和华对{\PN{摩西}}说:「你下去嘱咐百姓,不可闯过来到我面前观看,恐怕他们有多人死亡;
\VS{22}又叫亲近我的祭司自洁,恐怕我忽然出来{\ADD{击杀}}他们。」
\VS{23}{\PN{摩西}}对耶和华说:「百姓不能上{\PN{西奈山}},因为你已经嘱咐我们说:『要在山的四围定界限,叫山成圣。』」
\VS{24}耶和华对他说:「下去吧,你要和{\PN{亚伦}}一同上来;只是祭司和百姓不可闯过来上到我面前,恐怕我忽然出来{\ADD{击杀}}他们。」
\VS{25}于是{\PN{摩西}}下到百姓那里告诉他们。

\par }\Chap{20}{\SH 十诫
\par }{\R (申5·1—21)
\par }{\PP \VerseOne{1}神吩咐这一切的话说:
\VS{2}「我是耶和华—你的 神,曾将你从{\PN{埃及}}地为奴之家领出来。
\par }{\PP \VS{3}「除了我以外,你不可有别的神。
\par }{\PP \VS{4}「不可为自己雕刻偶像,也不可做什么形象仿佛上天、下地,和地底下、水中的{\ADD{百物}}。
\VS{5}不可跪拜那些像,也不可事奉它,因为我耶和华—你的 神是忌邪的 神。恨我的,我必追讨他的罪,自父及子,直到三四代;
\VS{6}爱我、守我诫命的,我必向他们发慈爱,直到千代。
\par }{\PP \VS{7}「不可妄称耶和华—你 神的名;因为妄称耶和华名的,耶和华必不以他为无罪。
\par }{\PP \VS{8}「当记念安息日,守为圣日。
\VS{9}六日要劳碌做你一切的工,
\VS{10}但第七日是向耶和华—你 神当守的安息日。{\ADD{这一日}}你和你的儿女、仆婢、牲畜,并你城里寄居的客旅,无论何工都不可做;
\VS{11}因为六日之内,耶和华造天、地、海,和其中的万物,第七日便安息,所以耶和华赐福与安息日,定为圣日。
\par }{\PP \VS{12}「当孝敬父母,使你的日子在耶和华—你 神所赐你的地上得以长久。
\par }{\PP \VS{13}「不可杀人。
\par }{\PP \VS{14}「不可奸淫。
\par }{\PP \VS{15}「不可偷盗。
\par }{\PP \VS{16}「不可作假见证陷害人。
\par }{\PP \VS{17}「不可贪恋人的房屋;也不可贪恋人的妻子、仆婢、牛驴,并他一切所有的。」
\par }{\SH 众民战惧
\par }{\R (申5·22—23)
\par }{\PP \VS{18}众百姓见雷轰、闪电、角声、山上冒烟,就都发颤,远远地站立,
\VS{19}对{\PN{摩西}}说:「求你和我们说话,我们必听;不要 神和我们说话,恐怕我们死亡。」
\VS{20}{\PN{摩西}}对百姓说:「不要惧怕;因为 神降临是要试验你们,叫你们时常敬畏他,不致犯罪。」
\VS{21}于是百姓远远地站立;{\PN{摩西}}就挨近 神所在的幽暗之中。
\par }{\SH 祭坛的条例
\par }{\PP \VS{22}耶和华对{\PN{摩西}}说:「你要向{\PN{以色列}}人这样说:『你们自己看见我从天上和你们说话了。
\VS{23}你们不可做什么{\ADD{神像}}与我相配,不可为自己做金银的神像。
\VS{24}你要为我筑土坛,在上面以牛羊献为燔祭和平安祭。凡记下我名的地方,我必到那里赐福给你。
\VS{25}你若为我筑一座石坛,不可用凿成的石头,因你在上头一动家具,就把坛污秽了。
\VS{26}你上我的坛,不可用台阶,免得露出你的下体来。』」

\par }\Chap{21}{\SH 对待奴仆的条例
\par }{\R (申15·12—18)
\par }{\PP \VerseOne{1}「你在百姓面前所要立的典章是这样:
\VS{2}你若买{\PN{希伯来}}人作奴仆,他必服事你六年;第七年他可以自由,白白地出去。
\VS{3}他若孤身来就可以孤身去;他若有妻,他的妻就可以同他出去。
\VS{4}他主人若给他妻子,妻子给他生了儿子或女儿,妻子和儿女要归主人,他要独自出去。
\VS{5}倘或奴仆明说:『我爱我的主人和我的妻子儿女,不愿意自由出去。』
\VS{6}他的主人就要带他到审判官\FTNT{}{{\FR 21:6: }或译: 神;下同}那里,又要带他到门前,靠近门框,用锥子穿他的耳朵,他就永远服事主人。
\par }{\PP \VS{7}「人若卖女儿作婢女,婢女不可像男仆那样出去。
\VS{8}主人选定她归自己,若不喜欢她,就要许她赎身;主人既然用诡诈待她,就没有权柄卖给外邦人。
\VS{9}主人若选定她给自己的儿子,就当待她如同女儿。
\VS{10}若另娶一个,那女子的吃食、衣服,并好合的事,仍不可减少。
\VS{11}若不向她行这三样,她就可以不用钱赎,白白地出去。」
\par }{\SH 惩罚暴行的条例
\par }{\PP \VS{12}「打人以致打死的,必要把他治死。
\VS{13}人若不是埋伏着{\ADD{杀人}},乃是 神交在他手中,我就设下一个地方,他可以往那里逃跑。
\VS{14}人若任意用诡计杀了他的邻舍,就是逃到我的坛那里,也当捉去把他治死。
\par }{\PP \VS{15}「打父母的,必要把他治死。
\par }{\PP \VS{16}「拐带人口,或是把人卖了,或是留在他手下,必要把他治死。
\par }{\PP \VS{17}「咒骂父母的,必要把他治死。
\par }{\PP \VS{18}「人若彼此相争,这个用石头或是拳头打那个,尚且不至于死,不过躺卧在床,
\VS{19}若再能起来扶杖而出,那打他的可算无罪;但要将他耽误的工夫用钱赔补,并要将他全然医好。
\par }{\PP \VS{20}「人若用棍子打奴仆或婢女,立时死在他的手下,他必要受刑。
\VS{21}若过一两天才死,就可以不受刑,因为是用钱买的。
\par }{\PP \VS{22}「人若彼此争斗,伤害有孕的妇人,甚至坠胎,随后却无别害,那伤害她的,总要按妇人的丈夫所要的,照审判官所断的,受罚。
\VS{23}若有别害,就要以命偿命,
\VS{24}以眼还眼,以牙还牙,以手还手,以脚还脚,
\VS{25}以烙还烙,以伤还伤,以打还打。
\par }{\PP \VS{26}「人若打坏了他奴仆或是婢女的一只眼,就要因他的眼放他去得以自由。
\VS{27}若打掉了他奴仆或是婢女的一个牙,就要因他的牙放他去得以自由。」
\par }{\SH 物主的责任
\par }{\PP \VS{28}「牛若触死男人或是女人,总要用石头打死那牛,却不可吃它的肉;牛的主人可算无罪。
\VS{29}倘若那牛素来是触人的,有人报告了牛主,他竟不把牛拴着,以致把男人或是女人触死,就要用石头打死那牛,牛主也必治死。
\VS{30}若罚他赎{\ADD{命}}的价银,他必照所罚的赎他的命。
\VS{31}牛无论触了人的儿子或是女儿,必照这例办理。
\VS{32}牛若触了奴仆或是婢女,必将银子三十舍客勒给他们的主人,也要用石头把牛打死。
\par }{\PP \VS{33}「人若敞着井口,或挖井不遮盖,有牛或驴掉在里头,
\VS{34}井主要拿钱赔还本主人,死牲畜要归自己。
\par }{\PP \VS{35}「这人的牛若伤了那人的牛,以至于死,他们要卖了活牛,平分价值,也要平分死牛。
\VS{36}人若知道这牛素来是触人的,主人竟不把牛拴着,他必要以牛还牛,死牛要归自己。」

\par }\Chap{22}{\SH 赔偿的条例
\par }{\PP \VerseOne{1}「人若偷牛或羊,无论是宰了,是卖了,他就要以五牛赔一牛,四羊赔一羊。
\VS{2}人若遇见贼挖窟窿,把贼打了,以至于死,就不能为他有流血的罪。
\VS{3}若太阳已经出来,就为他有流血的罪。贼若被拿,总要赔还。若他一无所有,就要被卖,顶他所偷的物。
\VS{4}若他所偷的,或牛,或驴,或羊,仍在他手下存活,他就要加倍赔还。
\par }{\PP \VS{5}「人若在田间或在葡萄园里放牲畜,任凭牲畜上别人的田里去吃,就必拿自己田间上好的和葡萄园上好的赔还。
\par }{\PP \VS{6}「若点火焚烧荆棘,以致将别人堆积的禾捆,站着的禾稼,或是田园,都烧尽了,那点火的必要赔还。
\par }{\PP \VS{7}「人若将银钱或家具交付邻舍看守,这物从那人的家被偷去,若把贼找到了,贼要加倍赔还;
\VS{8}若找不到贼,那家主必就近审判官,要{\ADD{看看}}他拿了原主的物件没有。
\par }{\PP \VS{9}「两个人的案件,无论是为什么过犯,或是为牛,为驴,为羊,为衣裳,或是为什么失掉之物,有一人说:『这是我的』,两造就要将案件禀告审判官,审判官定谁有罪,谁就要加倍赔还。
\par }{\PP \VS{10}「人若将驴,或牛,或羊,或别的牲畜,交付邻舍看守,牲畜或死,或受伤,或被赶去,无人看见,
\VS{11}那看守的人要凭着耶和华起誓,手里未曾拿邻舍的物,本主就要罢休,看守的人不必赔还。
\VS{12}牲畜若从看守的那里被偷去,他就要赔还本主;
\VS{13}若被{\ADD{野兽}}撕碎,看守的要带来当作证据,所撕的不必赔还。
\par }{\PP \VS{14}「人若向邻舍借什么,所借的或受伤,或死,本主没有同在一处,借的人总要赔还;
\VS{15}若本主同在一处,他就不必赔还;若是雇的,也不必赔还,本是为雇价来的。」
\par }{\SH 道德上和宗教上的条例
\par }{\PP \VS{16}「人若引诱没有受聘的处女,与她行淫,他总要交出聘礼,娶她为妻。
\VS{17}若女子的父亲决不肯将女子给他,他就要按处女的聘礼,交出钱来。
\par }{\PP \VS{18}「行邪术的女人,不可容她存活。
\par }{\PP \VS{19}「凡与兽淫合的,总要把他治死。
\par }{\PP \VS{20}「祭祀别神,不单单祭祀耶和华的,那人必要灭绝。
\par }{\PP \VS{21}「不可亏负寄居的,也不可欺压他,因为你们在{\PN{埃及}}地也作过寄居的。
\VS{22}不可苦待寡妇和孤儿;
\VS{23}若是苦待他们一点,他们向我一哀求,我总要听他们的哀声,
\VS{24}并要发烈怒,用刀杀你们,使你们的妻子为寡妇,儿女为孤儿。
\par }{\PP \VS{25}「我民中有贫穷人与你同住,你若借钱给他,不可如放债的向他取利。
\VS{26}你即或拿邻舍的衣服作当头,必在日落以先归还他;
\VS{27}因他只有这一件当盖头,是他盖身的衣服,{\ADD{若是没有}},他拿什么睡觉呢?他哀求我,我就应允,因为我是有恩惠的。
\par }{\PP \VS{28}「不可毁谤 神;也不可毁谤你百姓的官长。
\par }{\PP \VS{29}「你要从你庄稼中的{\ADD{谷}}和{\ADD{酒}}榨中滴出来的酒拿来献上,不可迟延。
\par }{\PP 「你要将头生的儿子归给我。
\VS{30}你牛羊{\ADD{头生的}},也要这样;七天当跟着母,第八天要归给我。
\par }{\PP \VS{31}「你们要在我面前为圣洁的人。因此,田间被野兽撕裂牲畜的肉,你们不可吃,要丢给狗吃。」

\par }\Chap{23}{\SH 正义和公道
\par }{\PP \VerseOne{1}「不可随伙布散谣言;不可与恶人连手妄作见证。
\VS{2}不可随众行恶;不可在争讼的事上随众偏行,作见证屈枉{\ADD{正直}};
\VS{3}也不可在争讼的事上偏护穷人。
\par }{\PP \VS{4}「若遇见你仇敌的牛或驴失迷了路,总要牵回来交给他。
\VS{5}若看见恨你人的驴压卧在重驮之下,不可走开,务要和驴主一同抬开{\ADD{重驮}}。
\par }{\PP \VS{6}「不可在穷人争讼的事上屈枉正直。
\VS{7}当远离虚假的事。不可杀无辜和有义的人,因我必不以恶人为义。
\VS{8}不可受贿赂;因为贿赂能叫明眼人变瞎了,又能颠倒义人的话。
\par }{\PP \VS{9}「不可欺压寄居的;因为你们在{\PN{埃及}}地作过寄居的,知道寄居的心。」
\par }{\SH 安息年和安息日
\par }{\PP \VS{10}「六年你要耕种田地,收藏土产,
\VS{11}只是第七年要叫地歇息,不耕不种,使你民中的穷人有吃的;他们所剩下的,野兽可以吃。你的葡萄园和橄榄园也要照样办理。
\par }{\PP \VS{12}「六日你要做工,第七日要安息,使牛、驴可以歇息,并使你婢女的儿子和寄居的都可以舒畅。
\par }{\PP \VS{13}「凡我对你们说的话,你们要谨守。别神的名,你不可提,也不可从你口中传说。」
\par }{\SH 三大节期
\par }{\R (出34·18—26;申16·1—17)
\par }{\PP \VS{14}「一年三次,你要向我守节。
\VS{15}你要守除酵节,照我所吩咐你的,在亚笔月内所定的日期,吃无酵饼七天。谁也不可空手朝见我,因为你是这月出了{\PN{埃及}}。
\VS{16}又{\ADD{要守}}收割节,所收的是你田间所种、劳碌得来初熟之物。并在年底收藏,{\ADD{要守}}收藏节。
\VS{17}一切的男丁要一年三次朝见主耶和华。
\par }{\PP \VS{18}「不可将我祭牲的血和有酵的饼一同献上;也不可将我节上{\ADD{祭牲}}的脂油留到早晨。
\par }{\PP \VS{19}「地里首先初熟之物要送到耶和华—你 神的殿。
\par }{\PP 「不可用山羊羔母的奶煮山羊羔。」
\par }{\SH 应许和指示
\par }{\PP \VS{20}「看哪,我差遣使者在你前面,在路上保护你,领你到我所预备的地方去。
\VS{21}他是奉我名来的;你们要在他面前谨慎,听从他的话,不可惹\FTNT{}{{\FR 23:21: }或译:违背}他,因为他必不赦免你们的过犯。
\par }{\PP \VS{22}「你若实在听从他的话,照着我一切所说的去行,我就向你的仇敌作仇敌,向你的敌人作敌人。
\par }{\PP \VS{23}「我的使者要在你前面行,领你到{\PN{亚摩利}}人、{\PN{赫}}人、{\PN{比利洗}}人、{\PN{迦南}}人、{\PN{希未}}人、{\PN{耶布斯}}人那里去,我必将他们剪除。
\VS{24}你不可跪拜他们的神,不可事奉他,也不可效法他们的行为,却要把神像尽行拆毁,打碎他们的柱像。
\VS{25}你们要事奉耶和华—你们的 神,他必赐福与你的粮与你的水,也必从你们中间除去疾病。
\VS{26}你境内必没有坠胎的,不生产的。我要使你满了你年日的数目。
\VS{27}凡你所到的地方,我要使那里的众民在你面前惊骇,扰乱,又要使你一切仇敌转背{\ADD{逃跑}}。
\VS{28}我要打发黄蜂飞在你前面,把{\PN{希未}}人、{\PN{迦南}}人、{\PN{赫}}人撵出去。
\VS{29}我不在一年之内将他们从你面前撵出去,恐怕地成为荒凉,野地的兽多起来害你。
\VS{30}我要渐渐地将他们从你面前撵出去,等到你的人数加多,承受那地为业。
\VS{31}我要定你的境界,从{\PN{红海}}直到{\PN{非利士海}},又从旷野直到大河。我要将那地的居民交在你手中,你要将他们从你面前撵出去。
\VS{32}不可和他们并他们的神立约。
\VS{33}他们不可住在你的地上,恐怕他们使你得罪我。你若事奉他们的神,这必成为你的网罗。」

\par }\Chap{24}{\SH 立约的凭据
\par }{\PP \VerseOne{1}耶和华对{\PN{摩西}}说:「你和{\PN{亚伦}}、{\PN{拿答}}、{\PN{亚比户}},并{\PN{以色列}}长老中的七十人,都要上到我这里来,远远地下拜。
\VS{2}惟独你可以亲近耶和华;他们却不可亲近;百姓也不可和你一同上来。」
\par }{\PP \VS{3}{\PN{摩西}}下山,将耶和华的命令典章都述说与百姓听。众百姓齐声说:「耶和华所吩咐的,我们都必遵行。」
\VS{4}{\PN{摩西}}将耶和华的命令都写上。清早起来,在山下筑一座坛,按{\PN{以色列}}十二支派立十二根柱子,
\VS{5}又打发{\PN{以色列}}人中的少年人去献燔祭,又向耶和华献牛为平安祭。
\VS{6}{\PN{摩西}}将血一半盛在盆中,一半洒在坛上;
\VS{7}又将约书念给百姓听。他们说:「耶和华所吩咐的,我们都必遵行。」
\VS{8}{\PN{摩西}}将血洒在百姓身上,说:「你看!这是立约的血,是耶和华按这一切话与你们立约{\ADD{的凭据}}。」
\par }{\PP \VS{9}{\PN{摩西}}、{\PN{亚伦}}、{\PN{拿答}}、{\PN{亚比户}},并{\PN{以色列}}长老中的七十人,都上了山。
\VS{10}他们看见{\PN{以色列}}的 神,他脚下仿佛有平铺的蓝宝石,如同天色明净。
\VS{11}他的手不加害在{\PN{以色列}}的尊者身上。他们观看 神;他们又吃又喝。
\par }{\SH 摩西在西奈山上
\par }{\PP \VS{12}耶和华对{\PN{摩西}}说:「你上山到我这里来,住在这里,我要将石版并我所写的律法和诫命赐给你,使你可以教训百姓。」
\VS{13}{\PN{摩西}}和他的帮手{\PN{约书亚}}起来,上了 神的山。
\VS{14}{\PN{摩西}}对长老说:「你们在这里等着,等到我们再回来,有{\PN{亚伦}}、{\PN{户珥}}与你们同在。凡有争讼的,都可以就近他们去。」
\par }{\PP \VS{15}{\PN{摩西}}上山,有云彩把山遮盖。
\VS{16}耶和华的荣耀停于{\PN{西奈山}};云彩遮盖山六天,第七天他从云中召{\PN{摩西}}。
\VS{17}耶和华的荣耀在山顶上,在{\PN{以色列}}人眼前,形状如烈火。
\VS{18}{\PN{摩西}}进入云中上山,在山上四十昼夜。

\par }\Chap{25}{\SH 奉献圣所
\par }{\R (出35·4—9)
\par }{\PP \VerseOne{1}耶和华晓谕{\PN{摩西}}说:
\VS{2}「你告诉{\PN{以色列}}人当为我送礼物来;凡甘心乐意的,你们就可以收下归我。
\VS{3}所要收的礼物:就是金、银、铜,
\VS{4}蓝色、紫色、朱红色{\ADD{线}},细麻,山羊{\ADD{毛}},
\VS{5}染红的公羊皮,海狗皮,皂荚木,
\VS{6}点灯的油并做膏油和香的香料,
\VS{7}红玛瑙与别样的宝石,可以镶嵌在以弗得和胸牌上。
\VS{8}又当为我造圣所,使我可以住在他们中间。
\VS{9}制造帐幕和其中的一切器具都要照我所指示你的样式。」
\par }{\SH 约柜
\par }{\R (出37·1—9)
\par }{\PP \VS{10}「要用皂荚木做一柜,长二肘半,宽一肘半,高一肘半。
\VS{11}要里外包上精金,四围镶上金牙边。
\VS{12}也要铸四个金环,安在柜的四脚上;这边两环,那边两环。
\VS{13}要用皂荚木做两根杠,用金包裹。
\VS{14}要把杠穿在柜旁的环内,以便抬柜。
\VS{15}这杠要常在柜的环内,不可抽出来。
\VS{16}必将我所要赐给你的法{\ADD{版}}放在柜里。
\VS{17}要用精金做施恩座\FTNT{}{{\FR 25:17: }施恩:或译蔽罪;下同},长二肘半,宽一肘半。
\VS{18}要用金子锤出两个基路伯来,安在施恩座的两头。
\VS{19}这头做一个基路伯,那头做一个基路伯,二基路伯要接连一块,在施恩座的两头。
\VS{20}二基路伯要高张翅膀,遮掩施恩座。基路伯要脸对脸,朝着施恩座。
\VS{21}要将施恩座安在柜的上边,又将我所要赐给你的法{\ADD{版}}放在柜里。
\VS{22}我要在那里与你相会,又要从法柜施恩座上二基路伯中间,和你说我所要吩咐你传给{\PN{以色列}}人的一切事。」
\par }{\SH 陈设饼的桌子
\par }{\R (出37·10—16)
\par }{\PP \VS{23}「要用皂荚木做一张桌子,长二肘,宽一肘,高一肘半。
\VS{24}要包上精金,四围镶上金牙边。
\VS{25}桌子的四围各做一掌宽的横梁,横梁上镶着金牙边。
\VS{26}要做四个金环,安在桌子的四角上,就是桌子四脚上的四角。
\VS{27}安环子的地方要挨近横梁,可以穿杠抬桌子。
\VS{28}要用皂荚木做两根杠,用金包裹,以便抬桌子。
\VS{29}要做桌子上的盘子、调羹,并奠{\ADD{酒}}的爵和瓶;这都要用精金制作。
\VS{30}又要在桌子上,在我面前,常摆陈设饼。」
\par }{\SH 灯台
\par }{\R (出37·17—24)
\par }{\PP \VS{31}「要用精金做一个灯台。灯台的座和干与杯、球、花,都要接连一块锤出来。
\VS{32}灯台两旁要杈出六个枝子:这旁三个,那旁三个。
\VS{33}这旁每枝上有三个杯,形状像杏花,有球,有花;那旁每枝上也有三个杯,形状像杏花,有球,有花。从灯台杈出来的六个枝子都是如此。
\VS{34}灯台上有四个杯,形状像杏花,有球,有花。
\VS{35}灯台每两个枝子以下有球与枝子接连一块。灯台出的六个枝子都是如此。
\VS{36}球和枝子要接连一块,都是一块精金锤出来的。
\VS{37}要做灯台的七个灯盏。祭司要点这灯,使灯光对照。
\VS{38}灯台的蜡剪和蜡花盘也是要精金的。
\VS{39}做灯台和这一切的器具要用精金一他连得。
\VS{40}要谨慎做这些物件,都要照着在山上指示你的样式。」

\par }\Chap{26}{\SH 会幕
\par }{\R (出36·8—38)
\par }{\PP \VerseOne{1}「你要用十幅幔子做帐幕。这些幔子要用捻的细麻和蓝色、紫色、朱红色{\ADD{线}}制造,并用巧匠的手工绣上基路伯。
\VS{2}每幅幔子要长二十八肘,宽四肘,幔子都要一样的尺寸。
\VS{3}这五幅幔子要幅幅相连;那五幅幔子也要幅幅相连。
\VS{4}在这相连的幔子末幅边上要做蓝色的钮扣;在那相连的幔子末幅边上也要照样做。
\VS{5}要在这相连的幔子上做五十个钮扣;在那相连的幔子上也做五十个钮扣,都要两两相对。
\VS{6}又要做五十个金钩,用钩使幔子相连,这才成了一个帐幕。
\par }{\PP \VS{7}「你要用山羊{\ADD{毛}}织十一幅幔子,作为帐幕以上的罩棚。
\VS{8}每幅幔子要长三十肘,宽四肘;十一幅幔子都要一样的尺寸。
\VS{9}要把五幅幔子连成一幅,又把六幅幔子连成一幅,这第六幅幔子要在罩棚的前面折上去。
\VS{10}在这相连的幔子末幅边上要做五十个钮扣;在那相连的幔子{\ADD{末幅}}边上也做五十个钮扣。
\VS{11}又要做五十个铜钩,钩在钮扣中,使罩棚连成一个。
\VS{12}罩棚的幔子所余那垂下来的半幅幔子,要垂在帐幕的后头。
\VS{13}罩棚的幔子所余长的,这边一肘,那边一肘,要垂在帐幕的两旁,遮盖帐幕。
\VS{14}又要用染红的公羊皮做罩棚的盖;再用海狗皮做一层罩棚上的顶盖。
\par }{\PP \VS{15}「你要用皂荚木做帐幕的竖板。
\VS{16}每块要长十肘,宽一肘半;
\VS{17}每块必有两榫相对。帐幕一切的板都要这样做。
\VS{18}帐幕的南面要做板二十块。
\VS{19}在这二十块板底下要做四十个带卯的银座,两卯接这块板上的两榫,两卯接那块板上的两榫。
\VS{20}帐幕第二面,就是北面,也要做板二十块
\VS{21}和带卯的银座四十个;这板底下有两卯,那板底下也有两卯。
\VS{22}帐幕的后面,就是西面,要做板六块。
\VS{23}帐幕后面的拐角要做板两块。
\VS{24}板的下半截要双的,上半截要整的,直顶到第一个环子;两块都要这样做两个拐角。
\VS{25}必有八块板和十六个带卯的银座;这板底下有两卯,那板底下也有两卯。
\par }{\PP \VS{26}「你要用皂荚木作闩:为帐幕这面的板作五闩,
\VS{27}为帐幕那面的板做五闩,又为帐幕后面的板做五闩。
\VS{28}板腰间的中闩要从这一头通到那一头。
\VS{29}板要用金子包裹,又要做板上的金环套闩;闩也要用金子包裹。
\VS{30}要照着在山上指示你的样式立起帐幕。
\par }{\PP \VS{31}「你要用蓝色、紫色、朱红色{\ADD{线}},和捻的细麻织幔子,以巧匠的手工绣上基路伯。
\VS{32}要把幔子挂在四根包金的皂荚木柱子上,柱子上当有金钩,柱子安在四个带卯的银座上。
\VS{33}要使幔子垂在钩子下,把法柜抬进幔子内;这幔子要将圣所和至圣所隔开。
\VS{34}又要把施恩座安在至圣所内的法柜上,
\VS{35}把桌子安在幔子外帐幕的北面;把灯台安在帐幕的南面,彼此相对。
\par }{\PP \VS{36}「你要拿蓝色、紫色、朱红色{\ADD{线}},和捻的细麻,用绣花的手工织帐幕的门帘。
\VS{37}要用皂荚木为帘子做五根柱子,用金子包裹。柱子上当有金钩;又要为柱子用铜铸造五个带卯的座。」

\par }\Chap{27}{\SH 祭坛
\par }{\R (出38·1—7)
\par }{\PP \VerseOne{1}「你要用皂荚木做坛。这坛要四方的,长五肘,宽五肘,高三肘。
\VS{2}要在坛的四拐角上做四个角,与坛接连一块,用铜把坛包裹。
\VS{3}要做盆,收去坛上的灰,又做铲子、盘子、肉锸子、火鼎;坛上一切的器具都用铜做。
\VS{4}要为坛做一个铜网,在网的四角上做四个铜环,
\VS{5}把网安在坛四面的围腰板以下,使网从下达到坛的半腰。
\VS{6}又要用皂荚木为坛做杠,用铜包裹。
\VS{7}这杠要穿在坛两旁的环子内,用以抬坛。
\VS{8}要用板做坛,坛是空的,都照着在山上指示你的样式做。」
\par }{\SH 会幕的周围
\par }{\R (出38·9—20)
\par }{\PP \VS{9}「你要做帐幕的院子。院子的南面要用捻的细麻做帷子,长一百肘。
\VS{10}帷子的柱子要二十根,带卯的铜座二十个。柱子上的钩子和杆子都要用银子做。
\VS{11}北面也当有帷子,长一百肘,帷子的柱子二十根,带卯的铜座二十个。柱子上的钩子和杆子都要用银子做。
\VS{12}院子的西面当有帷子,宽五十肘,帷子的柱子十根,带卯的座十个。
\VS{13}院子的东面要宽五十肘。
\VS{14}{\ADD{门}}这边的帷子要十五肘,帷子的柱子三根,带卯的座三个。
\VS{15}{\ADD{门}}那边的帷子也要十五肘,帷子的柱子三根,带卯的座三个。
\VS{16}院子的门当有帘子,长二十肘,要拿蓝色、紫色、朱红色{\ADD{线}},和捻的细麻,用绣花的手工织成,柱子四根,带卯的座四个。
\VS{17}院子四围一切的柱子都要用银杆连络,柱子上的钩子要用银做,带卯的座要用铜做。
\VS{18}院子要长一百肘,宽五十肘,高五肘,帷子要用捻的细麻做,带卯的座要用铜做。
\VS{19}帐幕各样用处的器具,并帐幕一切的橛子,和院子里一切的橛子,都要用铜做。」
\par }{\SH 燃灯的条例
\par }{\R (利24·1—4)
\par }{\PP \VS{20}「你要吩咐{\PN{以色列}}人,把那为点灯捣成的清橄榄油拿来给你,使灯常常点着。
\VS{21}在会幕中法{\ADD{柜}}前的幔外,{\PN{亚伦}}和他的儿子,从晚上到早晨,要在耶和华面前经理这灯。这要作{\PN{以色列}}人世世代代永远的定例。」

\par }\Chap{28}{\SH 祭司的圣服
\par }{\R (出39·1—7)
\par }{\PP \VerseOne{1}「你要从{\PN{以色列}}人中,使你的哥哥{\PN{亚伦}}和他的儿子{\PN{拿答}}、{\PN{亚比户}}、{\PN{以利亚撒}}、{\PN{以他玛}}一同就近你,给我供祭司的职分。
\VS{2}你要给你哥哥{\PN{亚伦}}做圣衣为荣耀,为华美。
\VS{3}又要吩咐一切心中有智慧的,就是我用智慧的灵所充满的,给{\PN{亚伦}}做衣服,使他分别为圣,可以给我供祭司的职分。
\VS{4}所要做的就是胸牌、以弗得、外袍、杂色的内袍、冠冕、腰带,使你哥哥{\PN{亚伦}}和他儿子{\ADD{穿}}这圣服,可以给我供祭司的职分。
\VS{5}要用金{\ADD{线}}和蓝色、紫色、朱红色{\ADD{线}},并细麻去做。
\par }{\PP \VS{6}「他们要拿金{\ADD{线}}和蓝色、紫色、朱红色{\ADD{线}},并捻的细麻,用巧匠的手工做以弗得。
\VS{7}以弗得当有两条肩带,接上两头,使它相连。
\VS{8}其上巧工织的带子,要和以弗得一样的做法,用以束上,与以弗得接连一块,要用金{\ADD{线}}和蓝色、紫色、朱红色{\ADD{线}},并捻的细麻做成。
\VS{9}要取两块红玛瑙,在上面刻{\PN{以色列}}儿子的名字:
\VS{10}六个名字在这块宝石上,六个名字在那块宝石上,都照他们生来的次序。
\VS{11}要用刻宝石的手工,仿佛刻图书,按着{\PN{以色列}}儿子的名字,刻这两块宝石,要镶在金槽上。
\VS{12}要将这两块宝石安在以弗得的两条肩带上,为{\PN{以色列}}人做纪念石。{\PN{亚伦}}要在两肩上担他们的名字,在耶和华面前作为纪念。
\VS{13}要用金子做二槽,
\VS{14}又拿精金,用拧工仿佛拧绳子,做两条链子,把这拧成的链子搭在二槽上。」
\par }{\SH 胸牌
\par }{\R (出39·8—21)
\par }{\PP \VS{15}「你要用巧匠的手工做一个决断的胸牌。要和以弗得一样的做法:用金{\ADD{线}}和蓝色、紫色、朱红色{\ADD{线}},并捻的细麻做成。
\VS{16}这胸牌要四方的,叠为两层,长一虎口,宽一虎口。
\VS{17}要在上面镶宝石四行:第一行是红宝石、红璧玺、红玉;
\VS{18}第二行是绿宝石、蓝宝石、金钢石;
\VS{19}第三行是紫玛瑙、白玛瑙、紫晶;
\VS{20}第四行是水苍玉、红玛瑙、碧玉。这都要镶在金槽中。
\VS{21}这些宝石都要按着{\PN{以色列}}十二个儿子的名字,仿佛刻图书,刻十二个支派的名字。
\VS{22}要在胸牌上用精金拧成如绳的链子。
\VS{23}在胸牌上也要做两个金环,安在胸牌的两头。
\VS{24}要把那两条拧成的金链子,穿过胸牌两头的环子。
\VS{25}又要把链子的那两头接在两槽上,安在以弗得前面肩带上。
\VS{26}要做两个金环,安在胸牌的两头,在以弗得里面的边上。
\VS{27}又要做两个金环,安在以弗得前面两条肩带的下边,挨近相接之处,在以弗得巧工织的带子以上。
\VS{28}要用蓝细带子把胸牌的环子与以弗得的环子系住,使胸牌贴在以弗得巧工织的带子上,不可与以弗得离缝。
\VS{29}{\PN{亚伦}}进圣所的时候,要将决断胸牌,就是刻着{\PN{以色列}}儿子名字的,带在胸前,在耶和华面前常作纪念。
\VS{30}又要将乌陵和土明放在决断的胸牌里;{\PN{亚伦}}进到耶和华面前的时候,要带在胸前,在耶和华面前常将{\PN{以色列}}人的决断牌带在胸前。」
\par }{\SH 祭司的其他圣服
\par }{\R (出39·22—31)
\par }{\PP \VS{31}「你要做以弗得的外袍,颜色全是蓝的。
\VS{32}袍上要为头留一领口,口的周围织出领边来,仿佛铠甲的领口,免得破裂。
\VS{33}袍子周围底边上要用蓝色、紫色、朱红色{\ADD{线}}做石榴。在袍子周围的石榴中间要有金铃铛:
\VS{34}一个金铃铛一个石榴,一个金铃铛一个石榴,在袍子周围的底边上。
\VS{35}{\PN{亚伦}}供职的时候要穿这袍子。他进圣所到耶和华面前,以及出来的时候,袍上的响声必被听见,使他不至于死亡。
\par }{\PP \VS{36}「你要用精金做一面牌,在上面按刻图书之法刻着『归耶和华为圣』。
\VS{37}要用一条蓝细带子将牌系在冠冕的前面。
\VS{38}这牌必在{\PN{亚伦}}的额上,{\PN{亚伦}}要担当干犯圣物{\ADD{条例}}的罪孽;这圣物是{\PN{以色列}}人在一切的圣礼物上所分别为圣的。这牌要常在他的额上,使他们可以在耶和华面前蒙悦纳。
\VS{39}要用杂色细麻{\ADD{线}}织内袍,用细麻布做冠冕,又用绣花的手工做腰带。
\par }{\PP \VS{40}「你要为{\PN{亚伦}}的儿子做内袍、腰带、裹头巾,为荣耀,为华美。
\VS{41}要把这些给你的哥哥{\PN{亚伦}}和他的儿子穿戴,又要膏他们,将他们分别为圣,好给我供祭司的职分。
\VS{42}要给他们做细麻布裤子,遮掩下体;裤子当从腰达到大腿。
\VS{43}{\PN{亚伦}}和他儿子进入会幕,或就近坛,在圣所供职的时候必穿上,免得担罪而死。这要为{\PN{亚伦}}和他的后裔作永远的定例。」

\par }\Chap{29}{\SH 立亚伦和他子孙作祭司的条例
\par }{\R (利8·1—36)
\par }{\PP \VerseOne{1}「你使{\PN{亚伦}}和他儿子成圣,给我供祭司的职分,要如此行:取一只公牛犊,两只无残疾的公绵羊,
\VS{2}无酵饼和调油的无酵饼,与抹油的无酵薄饼;这都要用细麦面做成。
\VS{3}这饼要装在一个筐子里,连筐子带来,又把公牛和两只公绵羊牵来。
\VS{4}要使{\PN{亚伦}}和他儿子到会幕门口来,用水洗身。
\VS{5}要给{\PN{亚伦}}穿上内袍和以弗得的外袍,并以弗得,又带上胸牌,束上以弗得巧工织的带子。
\VS{6}把冠冕戴在他头上,将圣冠加在冠冕上,
\VS{7}就把膏油倒在他头上膏他。
\VS{8}要叫他的儿子来,给他们穿上内袍。
\VS{9}给{\PN{亚伦}}和他儿子束上腰带,包上裹头巾,他们就凭永远的定例得了祭司的职任。又要将{\PN{亚伦}}和他儿子分别为圣。
\par }{\PP \VS{10}「你要把公牛牵到会幕前,{\PN{亚伦}}和他儿子要按手在公牛的头上。
\VS{11}你要在耶和华面前,在会幕门口,宰这公牛。
\VS{12}要取些公牛的血,用指头抹在坛的四角上,把血都倒在坛脚那里。
\VS{13}要把一切盖脏的脂油与肝上的网子,并两个腰子和腰子上的脂油,都烧在坛上。
\VS{14}只是公牛的皮、肉、粪都要用火烧在营外。这牛是赎罪祭。
\par }{\PP \VS{15}「你要牵一只公绵羊来,{\PN{亚伦}}和他儿子要按手在这羊的头上。
\VS{16}要宰这羊,把血洒在坛的周围。
\VS{17}要把羊切成块子,洗净五脏和腿,连块子带头,都放在一处。
\VS{18}要把全羊烧在坛上,是给耶和华献的燔祭,是献给耶和华为馨香的火祭。
\par }{\PP \VS{19}「你要将那一只公绵羊牵来,{\PN{亚伦}}和他儿子要按手在羊的头上。
\VS{20}你要宰这羊,取点血抹在{\PN{亚伦}}的右耳垂上和他儿子的右耳垂上,又抹在他们右手的大拇指上和右脚的大拇指上;并要把血洒在坛的四围。
\VS{21}你要取点膏油和坛上的血,弹在{\PN{亚伦}}和他的衣服上,并他儿子和他儿子的衣服上,他们和他们的衣服就一同成圣。
\par }{\PP \VS{22}「你要取这羊的脂油和肥尾巴,并盖脏的脂油与肝上的网子,两个腰子和腰子上的脂油并右腿(这是承接圣职所献的羊)。
\VS{23}再从耶和华面前装无酵饼的筐子中取一个饼,一个调油的饼和一个薄饼,
\VS{24}都放在{\PN{亚伦}}的手上和他儿子的手上,作为摇祭,在耶和华面前摇一摇。
\VS{25}要从他们手中接过来,烧在耶和华面前坛上的燔祭上,是献给耶和华为馨香的火祭。
\par }{\PP \VS{26}「你要取{\PN{亚伦}}承接圣职所献公羊的胸,作为摇祭,在耶和华面前摇一摇,这就可以作你的分。
\VS{27}那摇祭的胸和举祭的腿,就是承接圣职所摇的、所举的,是归{\PN{亚伦}}和他儿子的。这些你都要成为圣,
\VS{28}作{\PN{亚伦}}和他子孙从{\PN{以色列}}人中永远所得的分,因为是举祭。这要从{\PN{以色列}}人的平安祭中,作为献给耶和华的举祭。
\par }{\PP \VS{29}「{\PN{亚伦}}的圣衣要留给他的子孙,可以穿着受膏,又穿着承接圣职。
\VS{30}他的子孙接续他当祭司的,每逢进会幕在圣所供职的时候,要穿七天。
\par }{\PP \VS{31}「你要将承接圣职所献公羊的肉煮在圣处。
\VS{32}{\PN{亚伦}}和他儿子要在会幕门口吃这羊的肉和筐内的饼。
\VS{33}他们吃那些赎罪之物,好承接圣职,使他们成圣;只是外人不可吃,因为这是圣物。
\VS{34}那承接圣职所献的肉或饼,若有一点留到早晨,就要用火烧了,不可吃这物,因为是圣物。
\par }{\PP \VS{35}「你要这样照我一切所吩咐的,向{\PN{亚伦}}和他儿子行承接圣职的礼七天。
\VS{36}每天要献公牛一只为赎罪祭。你洁净坛的时候,坛就洁净了;且要用膏抹坛,使坛成圣。
\VS{37}要洁净坛七天,使坛成圣,坛就成为至圣。凡挨着坛的都成为圣。」
\par }{\SH 每天当献的祭
\par }{\R (民28·1—8)
\par }{\PP \VS{38}「你每天所要献在坛上的就是两只一岁的羊羔;
\VS{39}早晨要献这一只,黄昏的时候要献那一只。
\VS{40}和这一只羊羔同献的,要用细面{\ADD{伊法}}十分之一与捣成的油一欣四分之一调和,又用酒一欣四分之一作为奠祭。
\VS{41}那一只羊羔要在黄昏的时候献上,照着早晨的素祭和奠祭的礼办理,作为献给耶和华馨香的火祭。
\VS{42}这要在耶和华面前、会幕门口,作你们世世代代常献的燔祭。我要在那里与你们相会,和你们说话。
\VS{43}我要在那里与{\PN{以色列}}人相会,{\ADD{会幕}}就要因我的荣耀成为圣。
\VS{44}我要使会幕和坛成圣,也要使{\PN{亚伦}}和他的儿子成圣,给我供祭司的职分。
\VS{45}我要住在{\PN{以色列}}人中间,作他们的 神。
\VS{46}他们必知道我是耶和华—他们的 神,是将他们从{\PN{埃及}}地领出来的,为要住在他们中间。我是耶和华—他们的 神。」

\par }\Chap{30}{\SH 香坛
\par }{\R (出37·25—28)
\par }{\PP \VerseOne{1}「你要用皂荚木做一座烧香的坛。
\VS{2}这坛要四方的,长一肘,宽一肘,高二肘;坛的四角要与坛接连一块。
\VS{3}要用精金把坛的上面与坛的四围,并坛的四角,包裹;又要在坛的四围镶上金牙边。
\VS{4}要做两个金环安在牙子边以下,在坛的两旁,两根横撑上,作为穿杠的用处,以便抬坛。
\VS{5}要用皂荚木做杠,用金包裹。
\VS{6}要把坛放在法柜前的幔子外,对着法{\ADD{柜}}上的施恩座,就是我要与你相会的地方。
\VS{7}{\PN{亚伦}}在坛上要烧馨香料做的香;每早晨他收拾灯的时候,要烧这香。
\VS{8}黄昏点灯的时候,他要在耶和华面前烧这香,作为世世代代常烧的香。
\VS{9}在这坛上不可奉上异样的香,不可献燔祭、素祭,也不可浇上奠祭。
\VS{10}{\PN{亚伦}}一年一次要在坛的角上行赎罪之礼。他一年一次要用赎罪祭牲的血在坛上行赎罪之礼,作为世世代代的定例。这坛在耶和华面前为至圣。」
\par }{\SH 会幕的捐献
\par }{\PP \VS{11}耶和华晓谕{\PN{摩西}}说:
\VS{12}「你要按{\PN{以色列}}人被数的,计算总数,你数的时候,他们各人要为自己的生命把赎价奉给耶和华,免得数的时候在他们中间有灾殃。
\VS{13}凡过去归那些被数之人的,每人要按圣所的平,拿银子半舍客勒;这半舍客勒是奉给耶和华的礼物(一舍客勒是二十季拉)。
\VS{14}凡过去归那些被数的人,从二十岁以外的,要将这礼物奉给耶和华。
\VS{15}他们为赎生命将礼物奉给耶和华,富足的不可多出,贫穷的也不可少出,各人要出半舍客勒。
\VS{16}你要从{\PN{以色列}}人收这赎罪银,作为会幕的使用,可以在耶和华面前为{\PN{以色列}}人作纪念,赎生命。」
\par }{\SH 铜盆
\par }{\PP \VS{17}耶和华晓谕{\PN{摩西}}说:
\VS{18}「你要用铜做洗濯盆和盆座,以便洗濯。要将盆放在会幕和坛的中间,在盆里盛水。
\VS{19}{\PN{亚伦}}和他的儿子要在这盆里洗手洗脚。
\VS{20}他们进会幕,或是就近坛前供职给耶和华献火祭的时候,必用水洗濯,免得死亡。
\VS{21}他们洗手洗脚就免得死亡。这要作{\PN{亚伦}}和他后裔世世代代永远的定例。」
\par }{\SH 圣膏
\par }{\PP \VS{22}耶和华晓谕{\PN{摩西}}说:
\VS{23}「你要取上品的香料,就是流质的没药五百{\ADD{舍客勒}},香肉桂一半,就是二百五十舍客勒,菖蒲二百五十舍客勒,
\VS{24}桂皮五百舍客勒,都按着圣所的平,又取橄榄油一欣,
\VS{25}按做香之法调和做成圣膏油。
\VS{26}要用这膏油抹会幕和法柜,
\VS{27}桌子与桌子的一切器具,灯台和灯台的器具,并香坛、
\VS{28}燔祭坛,和坛的一切器具,洗濯盆和盆座。
\VS{29}要使这些物成为圣,好成为至圣;凡挨着的都成为圣。
\VS{30}要膏{\PN{亚伦}}和他的儿子,使他们成为圣,可以给我供祭司的职分。
\VS{31}你要对{\PN{以色列}}人说:『这油,我要世世代代以为圣膏油。
\VS{32}不可倒在别人的身上,也不可按这调和之法做与此相似的。这膏油是圣的,你们也要以为圣。
\VS{33}凡调和与此相似的,或将这膏膏在别人身上的,这人要从民中剪除。』」
\par }{\SH 香的做法
\par }{\PP \VS{34}耶和华吩咐{\PN{摩西}}说:「你要取馨香的香料,就是拿他弗、施喜列、喜利比拿;这馨香的香料和净乳香各样要一般大的分量。
\VS{35}你要用这些加上盐,按做香之法做成清净圣洁的香。
\VS{36}这香要取点捣得极细,放在会幕内、法{\ADD{柜}}前,我要在那里与你相会。你们要以这香为至圣。
\VS{37}你们不可按这调和之法为自己做香;要以这香为圣,归耶和华。
\VS{38}凡做香和这香一样,为要闻香味的,这人要从民中剪除。」

\par }\Chap{31}{\SH 会幕的技工
\par }{\R (出35·30—36·1)
\par }{\PP \VerseOne{1}耶和华晓谕{\PN{摩西}}说:
\VS{2}「看哪,{\PN{犹大}}支派中,{\PN{户珥}}的孙子、{\PN{乌利}}的儿子{\PN{比撒列}},我已经提他的名召他。
\VS{3}我也以我的灵充满了他,使他有智慧,有聪明,有知识,能做各样的工,
\VS{4}能想出巧工,用金、银、铜制造各物,
\VS{5}又能刻宝石,可以镶嵌,能雕刻木头,能做各样的工。
\VS{6}我分派{\PN{但}}支派中、{\PN{亚希撒抹}}的儿子{\PN{亚何利亚伯}}与他同工。凡心里有智慧的,我更使他们有智慧,能做我一切所吩咐的,
\VS{7}就是会幕和法柜,并其上的施恩座,与{\ADD{会}}幕中一切的器具,
\VS{8}桌子和桌子的器具,精{\ADD{金}}的灯台和灯台的一切器具并香坛,
\VS{9}燔祭坛和坛的一切器具,并洗濯盆与盆座,
\VS{10}精工做的礼服,和祭司{\PN{亚伦}}并他儿子用以供祭司职分的圣衣,
\VS{11}膏油和为圣所用馨香的香料。他们都要照我一切所吩咐的去做。」
\par }{\SH 安息日
\par }{\PP \VS{12}耶和华晓谕{\PN{摩西}}说:
\VS{13}「你要吩咐{\PN{以色列}}人说:『你们务要守我的安息日;因为这是你我之间世世代代的证据,使你们知道我—耶和华是叫你们成为圣的。
\VS{14}所以你们要守安息日,以为圣日。凡干犯这日的,必要把他治死;凡在这日做工的,必从民中剪除。
\VS{15}六日要做工,但第七日是安息圣日,是向耶和华守为圣的。凡在安息日做工的,必要把他治死。』
\VS{16}故此,{\PN{以色列}}人要世世代代守安息日为永远的约。
\VS{17}这是我和{\PN{以色列}}人永远的证据;因为六日之内耶和华造天地,第七日便安息舒畅。」
\par }{\PP \VS{18}耶和华在{\PN{西奈山}}和{\PN{摩西}}说完了话,就把两块法版交给他,是 神用指头写的石版。

\par }\Chap{32}{\SH 金牛犊
\par }{\R (申9·6—29)
\par }{\PP \VerseOne{1}百姓见{\PN{摩西}}迟延不下山,就大家聚集到{\PN{亚伦}}那里,对他说:「起来!为我们做神{\ADD{像}},可以在我们前面引路;因为领我们出{\PN{埃及}}地的那个{\PN{摩西}},我们不知道他遭了什么事。」
\VS{2}{\PN{亚伦}}对他们说:「你们去摘下你们妻子、儿女耳上的金环,拿来给我。」
\VS{3}百姓就都摘下他们耳上的金环,拿来给{\PN{亚伦}}。
\VS{4}{\PN{亚伦}}从他们手里接过来,铸了一只牛犊,用雕刻的器具做成。他们就说:「{\PN{以色列}}啊,这是领你出{\PN{埃及}}地的神。」
\VS{5}{\PN{亚伦}}看见,就在牛犊面前筑坛,且宣告说:「明日要向耶和华守节。」
\VS{6}次日清早,百姓起来献燔祭和平安祭,就坐下吃喝,起来玩耍。
\par }{\PP \VS{7}耶和华吩咐{\PN{摩西}}说:「下去吧,因为你的百姓,就是你从{\PN{埃及}}地领出来的,已经败坏了。
\VS{8}他们快快偏离了我所吩咐的道,为自己铸了一只牛犊,向它下拜献祭,说:『{\PN{以色列}}啊,这就是领你出{\PN{埃及}}地的神。』」
\VS{9}耶和华对{\PN{摩西}}说:「我看这百姓真是硬着颈项的百姓。
\VS{10}你且由着我,我要向他们发烈怒,将他们灭绝,使你的后裔成为大国。」
\par }{\PP \VS{11}{\PN{摩西}}便恳求耶和华—他的 神说:「耶和华啊,你为什么向你的百姓发烈怒呢?这百姓是你用大力和大能的手从{\PN{埃及}}地领出来的。
\VS{12}为什么使{\PN{埃及}}人议论说『他领他们出去,是要降祸与他们,把他们杀在山中,将他们从地上除灭』?求你转意,不发你的烈怒,后悔,不降祸与你的百姓。
\VS{13}求你记念你的仆人{\PN{亚伯拉罕}}、{\PN{以撒}}、{\PN{以色列}}。你曾指着自己起誓说:『我必使你们的后裔像天上的星那样多,并且我所应许的这全地,必给你们的后裔,他们要永远承受为业。』」
\VS{14}于是耶和华后悔,不把所说的祸降与他的百姓。
\par }{\PP \VS{15}{\PN{摩西}}转身下山,手里拿着两块法版。这版是两面写的,这面那面都有字,
\VS{16}是 神的工作,字是 神写的,刻在版上。
\VS{17}{\PN{约书亚}}一听见百姓呼喊的声音,就对{\PN{摩西}}说:「在营里有争战的声音。」
\VS{18}{\PN{摩西}}说:「这不是人打胜仗的声音,也不是人打败仗的声音;我所听见的乃是人歌唱的声音。」
\VS{19}{\PN{摩西}}挨近营前就看见牛犊,又看见人跳舞,便发烈怒,把两块版扔在山下摔碎了,
\VS{20}又将他们所铸的牛犊用火焚烧,磨得粉碎,撒在水面上,叫{\PN{以色列}}人喝。
\par }{\PP \VS{21}{\PN{摩西}}对{\PN{亚伦}}说:「这百姓向你做了什么?你竟使他们陷在大罪里!」
\VS{22}{\PN{亚伦}}说:「求我主不要发烈怒。这百姓{\ADD{专于}}作恶,是你知道的。
\VS{23}他们对我说:『你为我们做神{\ADD{像}},可以在我们前面引路;因为领我们出{\PN{埃及}}地的那个{\PN{摩西}},我们不知道他遭了什么事。』
\VS{24}我对他们说:『凡有金环的可以摘下来』,他们就给了我。我把金环扔在火中,这牛犊便出来了。」
\par }{\PP \VS{25}{\PN{摩西}}见百姓放肆({\PN{亚伦}}纵容他们,使他们在仇敌中间被讥刺),
\VS{26}就站在营门中,说:「凡属耶和华的,都{\ADD{要到}}我这里来!」于是{\PN{利未}}的子孙都到他那里聚集。
\VS{27}他对他们说:「耶和华—{\PN{以色列}}的 神这样说:『你们各人把刀跨在腰间,在营中往来,从这门到那门,各人杀他的弟兄与同伴并邻舍。』」
\VS{28}{\PN{利未}}的子孙照{\PN{摩西}}的话行了。那一天百姓中被杀的约有三千。
\VS{29}{\PN{摩西}}说:「今天你们要自洁,归耶和华为圣,各人攻击他的儿子和弟兄,使耶和华赐福与你们。」
\par }{\PP \VS{30}到了第二天,{\PN{摩西}}对百姓说:「你们犯了大罪。我如今要上耶和华那里去,或者可以为你们赎罪。」
\VS{31}{\PN{摩西}}回到耶和华那里,说:「唉!这百姓犯了大罪,为自己做了金像。
\VS{32}倘或你肯赦免他们的罪......不然,求你从你所写的册上涂抹我的名。」
\VS{33}耶和华对{\PN{摩西}}说:「谁得罪我,我就从我的册上涂抹谁的名。
\VS{34}现在你去领这百姓,往我所告诉你的{\ADD{地方}}去,我的使者必在你前面引路;只是到我追讨的日子,我必追讨他们的罪。」
\par }{\PP \VS{35}耶和华杀百姓的缘故是因他们同{\PN{亚伦}}做了牛犊。

\par }\Chap{33}{\SH 耶和华吩咐以色列人离开西奈山
\par }{\PP \VerseOne{1}耶和华吩咐{\PN{摩西}}说:「我曾起誓应许{\PN{亚伯拉罕}}、{\PN{以撒}}、{\PN{雅各}}说:『要将{\PN{迦南}}地赐给你的后裔。』现在你和你从{\PN{埃及}}地所领出来的百姓,要从这里往那地去。
\VS{2}我要差遣使者在你前面,撵出{\PN{迦南}}人、{\PN{亚摩利}}人、{\PN{赫}}人、{\PN{比利洗}}人、{\PN{希未}}人、{\PN{耶布斯}}人,
\VS{3}领你到那流奶与蜜之地。我自己不同你们上去;因为你们是硬着颈项的百姓,恐怕我在路上把你们灭绝。」
\par }{\PP \VS{4}百姓听见这凶信就悲哀,也没有人佩戴妆饰。
\VS{5}耶和华对{\PN{摩西}}说:「你告诉{\PN{以色列}}人说:『{\ADD{耶和华说}}:你们是硬着颈项的百姓,我若一霎时临到你们中间,必灭绝你们。现在你们要把身上的妆饰摘下来,使我可以知道怎样待你们。』」
\VS{6}{\PN{以色列}}人从住{\PN{何烈山}}以后,就把身上的妆饰摘得干净。
\par }{\SH 耶和华的会幕
\par }{\PP \VS{7}{\PN{摩西}}素常将帐棚支搭在营外,离营却远,他称这帐棚为会幕。凡求问耶和华的,就到营外的会幕那里去。
\VS{8}当{\PN{摩西}}出营到{\ADD{会}}幕去的时候,百姓就都起来,各人站在自己帐棚的门口,望着{\PN{摩西}},直等到他进了{\ADD{会}}幕。
\VS{9}{\PN{摩西}}进{\ADD{会}}幕的时候,云柱降下来,立在{\ADD{会}}幕的门前,{\ADD{耶和华}}便与{\PN{摩西}}说话。
\VS{10}众百姓看见云柱立在{\ADD{会}}幕门前,就都起来,各人在自己帐棚的门口下拜。
\VS{11}耶和华与{\PN{摩西}}面对面说话,好像人与朋友说话一般。{\PN{摩西}}转到营里去,惟有他的帮手—一个少年人{\PN{嫩}}的儿子{\PN{约书亚}}不离开{\ADD{会}}幕。
\par }{\SH 耶和华应许跟百姓同在
\par }{\PP \VS{12}{\PN{摩西}}对耶和华说:「你吩咐我说:『将这百姓领上去』,却没有叫我知道你要打发谁与我同去,只说:『我按你的名认识你,你在我眼前也蒙了恩。』
\VS{13}我如今若在你眼前蒙恩,求你将你的道指示我,使我可以认识你,好在你眼前蒙恩。求你想到这民是你的民。」
\VS{14}耶和华说:「我必亲自{\ADD{和你同}}去,使你得安息。」
\VS{15}{\PN{摩西}}说:「你若不亲自{\ADD{和我同}}去,就不要把我们从这里领上去。
\VS{16}人在何事上得以知道我和你的百姓在你眼前蒙恩呢?岂不是因你与我们同去、使我和你的百姓与地上的万民有分别吗?」
\par }{\PP \VS{17}耶和华对{\PN{摩西}}说:「你这所求的我也要行;因为你在我眼前蒙了恩,并且我按你的名认识你。」
\VS{18}{\PN{摩西}}说:「求你显出你的荣耀给我看。」
\VS{19}耶和华说:「我要显我一切的恩慈,在你面前经过,宣告我的名。我要恩待谁就恩待谁;要怜悯谁就怜悯谁」;
\VS{20}又说:「你不能看见我的面,因为人见我的面不能存活。」
\VS{21}耶和华说:「看哪,在我这里有地方,你要站在磐石上。
\VS{22}我的荣耀经过的时候,我必将你放在磐石穴中,用我的手遮掩你,等我过去,
\VS{23}然后我要将我的手收回,你就得见我的背,却不得见我的面。」

\par }\Chap{34}{\SH 复造法版
\par }{\R (申10·1—5)
\par }{\PP \VerseOne{1}耶和华吩咐{\PN{摩西}}说:「你要凿出两块石版,和先前你摔碎的那版一样;其上的字我要写在这版上。
\VS{2}明日早晨,你要预备好了,上{\PN{西奈山}},在山顶上站在我面前。
\VS{3}谁也不可和你一同上去,遍山都不可有人,在山根也不可叫羊群牛群吃草。」
\VS{4}{\PN{摩西}}就凿出两块石版,和先前的一样。清晨起来,照耶和华所吩咐的上{\PN{西奈山}}去,手里拿着两块石版。
\VS{5}耶和华在云中降临,和{\PN{摩西}}一同站在那里,宣告耶和华的名。
\VS{6}耶和华在他面前宣告{\ADD{说}}:「耶和华,耶和华,是有怜悯有恩典的 神,不轻易发怒,并有丰盛的慈爱和诚实,
\VS{7}为千万人存留慈爱,赦免罪孽、过犯,和罪恶,万不以{\ADD{有罪的}}为无罪,必追讨他的罪,自父及子,直到三、四代。」
\VS{8}{\PN{摩西}}急忙伏地下拜,
\VS{9}说:「主啊,我若在你眼前蒙恩,求你在我们中间同行,因为这是硬着颈项的百姓。又求你赦免我们的罪孽和罪恶,以我们为你的产业。」
\par }{\SH 重新立约
\par }{\R (出23·14—19;申7·1—5;16·1—17)
\par }{\PP \VS{10}耶和华说:「我要立约,要在百姓面前行奇妙的事,是在遍地万国中所未曾行的。在你四围的外邦人就要看见耶和华的作为,因我向你所行的是可畏惧的事。
\par }{\PP \VS{11}「我今天所吩咐你的,你要谨守。我要从你面前撵出{\PN{亚摩利}}人、{\PN{迦南}}人、{\PN{赫}}人、{\PN{比利洗}}人、{\PN{希未}}人、{\PN{耶布斯}}人。
\VS{12}你要谨慎,不可与你所去那地的居民立约,恐怕成为你们中间的网罗;
\VS{13}却要拆毁他们的祭坛,打碎他们的柱像,砍下他们的木偶。
\VS{14}不可敬拜别神;因为耶和华是忌邪的 神,名为忌邪者。
\VS{15}只怕你与那地的居民立约,百姓随从他们的神,就行邪淫,祭祀他们的神,有人叫你,你便吃他的祭物,
\VS{16}又为你的儿子娶他们的女儿为妻,他们的女儿随从她们的神,就行邪淫,使你的儿子也随从她们的神行邪淫。
\par }{\PP \VS{17}「不可为自己铸造神{\ADD{像}}。
\par }{\PP \VS{18}「你要守除酵节,照我所吩咐你的,在亚笔月内所定的日期吃无酵饼七天,因为你是这亚笔月内出了{\PN{埃及}}。
\VS{19}凡头生的都是我的;一切牲畜头生的,无论是牛是羊,公的都是我的。
\VS{20}头生的驴要用羊羔代赎,若不代赎就要打折它的颈项。凡头生的儿子都要赎出来。谁也不可空手朝见我。
\par }{\PP \VS{21}「你六日要做工,第七日要安息,虽在耕种收割的时候也要安息。
\VS{22}在收割初熟麦子的时候要守七七节;又在年底要守收藏节。
\VS{23}你们一切男丁要一年三次朝见主耶和华—{\PN{以色列}}的 神。
\VS{24}我要从你面前赶出外邦人,扩张你的境界。你一年三次上去朝见耶和华—你 神的时候,必没有人贪慕你的地土。
\par }{\PP \VS{25}「你不可将我祭物的血和有酵的饼一同献上。逾越节的祭物也不可留到早晨。
\VS{26}地里首先初熟之物要送到耶和华—你 神的殿。不可用山羊羔母的奶煮山羊羔。」
\par }{\PP \VS{27}耶和华吩咐{\PN{摩西}}说:「你要将这些话写上,因为我是按这话与你和{\PN{以色列}}人立约。」
\VS{28}{\PN{摩西}}在耶和华那里四十昼夜,也不吃饭也不喝水。耶和华将这约的话,就是十条诫,写在两块版上。
\par }{\SH 摩西下西奈山
\par }{\PP \VS{29}{\PN{摩西}}手里拿着两块法版下{\PN{西奈山}}的时候,不知道自己的面皮因耶和华和他说话就发了光。
\VS{30}{\PN{亚伦}}和{\PN{以色列}}众人看见{\PN{摩西}}的面皮发光就怕挨近他。
\VS{31}{\PN{摩西}}叫他们来;于是{\PN{亚伦}}和会众的官长都到他那里去,{\PN{摩西}}就与他们说话。
\VS{32}随后{\PN{以色列}}众人都近前来,他就把耶和华在{\PN{西奈山}}与他所说的一切话都吩咐他们。
\VS{33}{\PN{摩西}}与他们说完了话就用帕子蒙上脸。
\VS{34}但{\PN{摩西}}进到耶和华面前与他说话就揭去帕子,及至出来的时候便将耶和华所吩咐的告诉{\PN{以色列}}人。
\VS{35}{\PN{以色列}}人看见{\PN{摩西}}的面皮发光。{\PN{摩西}}又用帕子蒙上脸,等到他进去与耶和华说话{\ADD{就揭去帕子}}。

\par }\Chap{35}{\SH 安息日的条例
\par }{\PP \VerseOne{1}{\PN{摩西}}招聚{\PN{以色列}}全会众,对他们说:「这是耶和华所吩咐的话,叫你们照着行:
\VS{2}六日要做工,第七日乃为圣日,当向耶和华守为安息圣日。凡这日之内做工的,必把他治死。
\VS{3}当安息日,不可在你们一切的住处生火。」
\par }{\SH 为会幕奉献
\par }{\R (出25·1—9)
\par }{\PP \VS{4}{\PN{摩西}}对{\PN{以色列}}全会众说:「耶和华所吩咐的是这样:
\VS{5}你们中间要拿礼物献给耶和华,凡乐意献的可以拿耶和华的礼物来,就是金、银、铜,
\VS{6}蓝色、紫色、朱红色{\ADD{线}},细麻,山羊{\ADD{毛}},
\VS{7}染红的公羊皮,海狗皮,皂荚木,
\VS{8}点灯的油,并做膏油和香的香料,
\VS{9}红玛瑙与别样的宝石,可以镶嵌在以弗得和胸牌上。」
\par }{\SH 会幕的器物
\par }{\R (出39·32—43)
\par }{\PP \VS{10}「你们中间凡心里有智慧的都要来做耶和华一切所吩咐的:
\VS{11}就是帐幕和帐幕的罩棚,并帐幕的盖、钩子、板、闩、柱子、带卯的座,
\VS{12}柜和柜的杠,施恩座和遮掩{\ADD{柜}}的幔子,
\VS{13}桌子和桌子的杠与桌子的一切器具,并陈设饼,
\VS{14}灯台和灯台的器具,灯盏并点灯的油,
\VS{15}香坛和坛的杠,膏油和馨香的香料,并帐幕门口的帘子,
\VS{16}燔祭坛和坛的铜网,坛的杠并坛的一切器具,洗濯盆和盆座,
\VS{17}院子的帷子和帷子的柱子,带卯的座和院子的门帘,
\VS{18}帐幕的橛子并院子的橛子,和这两处的绳子,
\VS{19}精工做的礼服和祭司{\PN{亚伦}}并他儿子在圣所用以供祭司职分的圣衣。」
\par }{\SH 百姓奉献礼物
\par }{\PP \VS{20}{\PN{以色列}}全会众从{\PN{摩西}}面前退去。
\VS{21}凡心里受感和甘心乐意的都拿耶和华的礼物来,用以做会幕和其中一切的使用,又用以做圣衣。
\VS{22}凡心里乐意献礼物的,连男带女,各将金器,就是胸前针、耳环\FTNT{}{{\FR 35:22: }或译:鼻环}、打印的戒指,和手钏带来献给耶和华。
\VS{23}凡有蓝色、紫色、朱红色{\ADD{线}},细麻,山羊{\ADD{毛}},染红的公羊皮,海狗皮的,都拿了来;
\VS{24}凡献银子和铜给耶和华为礼物的都拿了来;凡有皂荚木可做什么使用的也拿了来。
\VS{25}凡心中有智慧的妇女亲手纺线,把所纺的蓝色、紫色、朱红色{\ADD{线}},和细麻都拿了来。
\VS{26}凡有智慧、心里受感的妇女就纺山羊{\ADD{毛}}。
\VS{27}众官长把红玛瑙和别样的宝石,可以镶嵌在以弗得与胸牌上的,都拿了来;
\VS{28}又拿香料做香,拿油点灯,做膏油。
\VS{29}{\PN{以色列}}人,无论男女,凡甘心乐意献礼物给耶和华的,都将礼物拿来,做耶和华借{\PN{摩西}}所吩咐的一切工。
\par }{\SH 造会幕的技工
\par }{\R (出31·1—11)
\par }{\PP \VS{30}{\PN{摩西}}对{\PN{以色列}}人说:「{\PN{犹大}}支派中,{\PN{户珥}}的孙子、{\PN{乌利}}的儿子{\PN{比撒列}},耶和华已经提他的名召他,
\VS{31}又以 神的灵充满了他,使他有智慧、聪明、知识,能做各样的工,
\VS{32}能想出巧工,用金、银、铜制造各物,
\VS{33}又能刻宝石,可以镶嵌,能雕刻木头,能做各样的巧工。
\VS{34}耶和华又使他,和{\PN{但}}支派中{\PN{亚希撒抹}}的儿子{\PN{亚何利亚伯}},心里灵明,能教导人。
\VS{35}耶和华使他们的心满有智慧,能做各样的工,无论是雕刻的工,巧匠的工,用蓝色、紫色、朱红色{\ADD{线}},和细麻、绣花的工,并机匠的工,他们都能做,也能想出奇巧的工。

\par }\Chap{36}{\PP \VerseOne{1}「{\PN{比撒列}}和{\PN{亚何利亚伯}},并一切心里有智慧的,就是蒙耶和华赐智慧聪明、叫他知道做圣所各样使用之工的,都要照耶和华所吩咐的做工。」
\par }{\SH 百姓奉献礼物
\par }{\PP \VS{2}凡耶和华赐他心里有智慧、而且受感前来做这工的,{\PN{摩西}}把他们和{\PN{比撒列}}并{\PN{亚何利亚伯}}一同召来。
\VS{3}这些人就从{\PN{摩西}}收了{\PN{以色列}}人为做圣所并圣所使用之工所拿来的礼物。百姓每早晨还把甘心献的礼物拿来。
\VS{4}凡做圣所一切工的智慧人各都离开他所做的工,
\VS{5}来对{\PN{摩西}}说:「百姓为耶和华吩咐使用之工所拿来的,富富有余。」
\VS{6}{\PN{摩西}}传命,他们就在全营中宣告说:「无论男女,不必再为圣所拿什么礼物来。」这样才拦住百姓不再拿礼物来。
\VS{7}因为他们所有的材料够做一切当做的物,而且有余。
\par }{\SH 造耶和华的会幕
\par }{\R (出26·1—37)
\par }{\PP \VS{8}他们中间,凡心里有智慧做工的,用十幅幔子做帐幕。这幔子是{\PN{比撒列}}用捻的细麻和蓝色、紫色、朱红色{\ADD{线}}制造的,并用巧匠的手工绣上基路伯。
\VS{9}每幅幔子长二十八肘,宽四肘,都是一样的尺寸。
\VS{10}他使这五幅幔子幅幅相连,又使那五幅幔子幅幅相连;
\VS{11}在这相连的幔子末幅边上做蓝色的钮扣,在那相连的幔子末幅边上也照样做;
\VS{12}在这相连的幔子上做五十个钮扣,在那相连的幔子上也做五十个钮扣,都是两两相对;
\VS{13}又做五十个金钩,使幔子相连。这才成了一个帐幕。
\par }{\PP \VS{14}他用山羊{\ADD{毛}}织十一幅幔子,作为帐幕以上的罩棚。
\VS{15}每幅幔子长三十肘,宽四肘;十一幅幔子都是一样的尺寸。
\VS{16}他把五幅幔子连成一幅,又把六幅幔子连成一幅;
\VS{17}在这相连的幔子末幅边上做五十个钮扣,在那相连的幔子{\ADD{末幅}}边上也做五十个钮扣;
\VS{18}又做五十个铜钩,使罩棚连成一个;
\VS{19}并用染红的公羊皮做罩棚的盖,再用海狗皮做一层罩棚上的顶盖。
\par }{\PP \VS{20}他用皂荚木做帐幕的竖板。
\VS{21}每块长十肘,宽一肘半;
\VS{22}每块有两榫相对。帐幕一切的板都是这样做。
\VS{23}帐幕的南面做板二十块。
\VS{24}在这二十块板底下又做四十个带卯的银座:两卯接这块板上的两榫,两卯接那块板上的两榫。
\VS{25}帐幕的第二面,就是北面,也做板二十块
\VS{26}和带卯的银座四十个:这板底下有两卯,那板底下也有两卯。
\VS{27}帐幕的后面,就是西面,做板六块。
\VS{28}帐幕后面的拐角做板两块。
\VS{29}板的下半截是双的,上半截是整的,直到第一个环子;在帐幕的两个拐角上都是这样做。
\VS{30}有八块板和十六个带卯的银座,每块板底下有两卯。
\par }{\PP \VS{31}他用皂荚木做闩:为帐幕这面的板做五闩,
\VS{32}为帐幕那面的板做五闩,又为帐幕后面的板做五闩,
\VS{33}使板腰间的中闩从这一头通到那一头。
\VS{34}用金子将板包裹,又做板上的金环套闩;闩也用金子包裹。
\par }{\PP \VS{35}他用蓝色、紫色、朱红色{\ADD{线}},和捻的细麻织幔子,以巧匠的手工绣上基路伯。
\VS{36}为幔子做四根皂荚木柱子,用金包裹,柱子上有金钩,又为柱子铸了四个带卯的银座。
\VS{37}拿蓝色、紫色、朱红色{\ADD{线}},和捻的细麻,用绣花的手工织帐幕的门帘;
\VS{38}又为帘子做五根柱子和柱子上的钩子,用金子把柱顶和柱子上的杆子包裹。柱子有五个带卯的座,是铜的。

\par }\Chap{37}{\SH 造约柜
\par }{\R (出25·10—22)
\par }{\PP \VerseOne{1}{\PN{比撒列}}用皂荚木做柜,长二肘半,宽一肘半,高一肘半。
\VS{2}里外包上精金,四围镶上金牙边,
\VS{3}又铸四个金环,安在柜的四脚上:这边两环,那边两环。
\VS{4}用皂荚木做两根杠,用金包裹。
\VS{5}把杠穿在柜旁的环内,以便抬柜。
\VS{6}用精金做施恩座,长二肘半,宽一肘半。
\VS{7}用金子锤出两个基路伯来,安在施恩座的两头,
\VS{8}这头做一个基路伯,那头做一个基路伯,二基路伯接连一块,在施恩座的两头。
\VS{9}二基路伯高张翅膀,遮掩施恩座;基路伯是脸对脸,朝着施恩座。
\par }{\SH 造陈设饼的桌子
\par }{\R (出25·23—30)
\par }{\PP \VS{10}他用皂荚木做一张桌子,长二肘,宽一肘,高一肘半,
\VS{11}又包上精金,四围镶上金牙边。
\VS{12}桌子的四围各做一掌宽的横梁,横梁上镶着金牙边,
\VS{13}又铸了四个金环,安在桌子四脚的四角上。
\VS{14}安环子的地方是挨近横梁,可以穿杠抬桌子。
\VS{15}他用皂荚木做两根杠,用金包裹,以便抬桌子;
\VS{16}又用精金做桌子上的器皿,就是盘子、调羹,并奠{\ADD{酒}}的瓶和爵。
\par }{\SH 造灯台
\par }{\R (出25·31—40)
\par }{\PP \VS{17}他用精金做一个灯台;这灯台的座和干,与杯、球、花,都是接连一块锤出来的。
\VS{18}灯台两旁杈出六个枝子:这旁三个,那旁三个。
\VS{19}这旁每枝上有三个杯,形状像杏花,有球有花;那旁每枝上也有三个杯,形状像杏花,有球有花。从灯台杈出来的六个枝子都是如此。
\VS{20}灯台上有四个杯,形状像杏花,有球有花。
\VS{21}灯台每两个枝子以下有球,与枝子接连一块;灯台杈出的六个枝子都是如此。
\VS{22}球和枝子是接连一块,都是一块精金锤出来的。
\VS{23}用精金做灯台的七个灯盏,并灯台的蜡剪和蜡花盘。
\VS{24}他用精金一他连得做灯台和灯台的一切器具。
\par }{\SH 造香坛
\par }{\R (出30·1—5)
\par }{\PP \VS{25}他用皂荚木做香坛,是四方的,长一肘,宽一肘,高二肘,坛的四角与坛接连一块;
\VS{26}又用精金把坛的上面与坛的四面并坛的四角包裹,又在坛的四围镶上金牙边。
\VS{27}做两个金环,安在牙子边以下,在坛的两旁、两根横撑上,作为穿杠的用处,以便抬坛。
\VS{28}用皂荚木做杠,用金包裹。
\par }{\SH 造圣油和圣膏
\par }{\R (出30·22—38)
\par }{\PP \VS{29}又按做香之法做圣膏油和馨香料的净香。

\par }\Chap{38}{\SH 造燔祭坛
\par }{\R (出27·1—8)
\par }{\PP \VerseOne{1}他用皂荚木做燔祭坛,是四方的,长五肘,宽五肘,高三肘,
\VS{2}在坛的四拐角上做四个角,与坛接连一块,用铜把坛包裹。
\VS{3}他做坛上的盆、铲子、盘子、肉锸子、火鼎;这一切器具都是用铜做的。
\VS{4}又为坛做一个铜网,安在坛四面的围腰板以下,从下达到坛的半腰。
\VS{5}为铜网的四角铸四个环子,作为穿杠的用处。
\VS{6}用皂荚木做杠,用铜包裹,
\VS{7}把杠穿在坛两旁的环子内,用以抬坛,并用板做坛;坛是空的。
\par }{\SH 造铜盆
\par }{\R (出30·18)
\par }{\PP \VS{8}他用铜做洗濯盆和盆座,是用会幕门前伺候的妇人之镜子做的。
\par }{\SH 耶和华会幕的周围
\par }{\R (出27·9—19)
\par }{\PP \VS{9}他做{\ADD{帐幕}}的院子。院子的南面用捻的细麻做帷子,{\ADD{宽}}一百肘。
\VS{10}帷子的柱子二十根,带卯的铜座二十个;柱子上的钩子和杆子都是用银子做的。
\VS{11}北面也{\ADD{有帷子}},{\ADD{宽}}一百肘。帷子的柱子二十根,带卯的铜座二十个;柱子上的钩子和杆子都是用银子做的。
\VS{12}{\ADD{院子的}}西面有帷子,{\ADD{宽}}五十肘。帷子的柱子十根,带卯的座十个;柱子的钩子和杆子都是用银子做的。
\VS{13}{\ADD{院子的}}东面,{\ADD{宽}}五十肘。
\VS{14-15}门这边的帷子十五肘,那边也是一样。帷子的柱子三根,带卯的座三个。在门的左右各有帷子十五肘,帷子的柱子三根,带卯的座三个。
\VS{16}院子四面的帷子都是用捻的细麻做的。
\VS{17}柱子带卯的座是铜的,柱子上的钩子和杆子是银的,柱顶是用银子包的。院子一切的柱子都是用银杆连络的。
\VS{18}院子的门帘是以绣花的手工,用蓝色、紫色、朱红色{\ADD{线}},和捻的细麻织的,宽二十肘,高五肘,与院子的帷子相配。
\VS{19}帷子的柱子四根,带卯的铜座四个;柱子上的钩子和杆子是银的;柱顶是用银子包的。
\VS{20}帐幕一切的橛子和院子四围的橛子都是铜的。
\par }{\SH 耶和华会幕所用的金属
\par }{\PP \VS{21}这是法{\ADD{柜}}的帐幕中{\PN{利未}}人所用{\ADD{物件}}的总数,是照{\PN{摩西}}的吩咐,经祭司{\PN{亚伦}}的儿子{\PN{以他玛}}的手数点的。
\VS{22}凡耶和华所吩咐{\PN{摩西}}的都是{\PN{犹大}}支派{\PN{户珥}}的孙子、{\PN{乌利}}的儿子{\PN{比撒列}}做的。
\VS{23}与他同工的有{\PN{但}}支派中{\PN{亚希撒抹}}的儿子{\PN{亚何利亚伯}};他是雕刻匠,又是巧匠,又能用蓝色、紫色、朱红色{\ADD{线}},和细麻绣花。
\par }{\PP \VS{24}为圣所一切工作使用所献的金子,按圣所的平,有二十九他连得并七百三十舍客勒。
\VS{25}会中被数的人所出的银子,按圣所的平,有一百他连得并一千七百七十五舍客勒。
\VS{26}凡过去归那些被数之人的,从二十岁以外,有六十万零三千五百五十人。按圣所的平,每人出银半舍客勒,就是一比加。
\VS{27}用那一百他连得银子铸造圣所带卯的座和幔子{\ADD{柱子}}带卯的座;一百他连得共一百带卯的座,每带卯的座用一他连得。
\VS{28}用那一千七百七十五{\ADD{舍客勒银子}}做柱子上的钩子,包裹柱顶并柱子上的杆子。
\VS{29}所献的铜有七十他连得并二千四百舍客勒。
\VS{30}用这铜做会幕门带卯的座和铜坛,并坛上的铜网和坛的一切器具,
\VS{31}并院子四围带卯的座和院门带卯的座,与帐幕一切的橛子和院子四围所有的橛子。

\par }\Chap{39}{\SH 为祭司制圣服
\par }{\R (出28·1—14)
\par }{\PP \VerseOne{1}{\PN{比撒列}}用蓝色、紫色、朱红色{\ADD{线}}做精致的衣服,在圣所用以供职,又为{\PN{亚伦}}做圣衣,是照耶和华所吩咐{\PN{摩西}}的。
\par }{\PP \VS{2}他用金{\ADD{线}}和蓝色、紫色、朱红色{\ADD{线}},并捻的细麻做以弗得;
\VS{3}把金子锤成薄片,剪出线来,与蓝色、紫色、朱红色{\ADD{线}},用巧匠的手工一同绣上。
\VS{4}又为以弗得做两条相连的肩带,接连在以弗得的两头。
\VS{5}其上巧工织的带子和以弗得一样的做法,用以束上,与以弗得接连一块,是用金{\ADD{线}}和蓝色、紫色、朱红色{\ADD{线}},并捻的细麻做的,是照耶和华所吩咐{\PN{摩西}}的。
\par }{\PP \VS{6}又琢出两块红玛瑙,镶在金槽上,仿佛刻图书,按着{\PN{以色列}}儿子的名字雕刻;
\VS{7}将这两块宝石安在以弗得的两条肩带上,为{\PN{以色列}}人做纪念石,是照耶和华所吩咐{\PN{摩西}}的。
\par }{\SH 制胸牌
\par }{\R (出28·15—30)
\par }{\PP \VS{8}他用巧匠的手工做胸牌,和以弗得一样的做法,用金{\ADD{线}}与蓝色、紫色、朱红色{\ADD{线}},并捻的细麻做的。
\VS{9}胸牌是四方的,叠为两层;这两层长一虎口,宽一虎口,
\VS{10}上面镶着宝石四行:第一行是红宝石、红璧玺、红玉;
\VS{11}第二行是绿宝石、蓝宝石、金钢石;
\VS{12}第三行是紫玛瑙、白玛瑙、紫晶;
\VS{13}第四行是水苍玉、红玛瑙、碧玉。这都镶在金槽中。
\VS{14}这些宝石都是按着{\PN{以色列}}十二个儿子的名字,仿佛刻图书,刻十二个支派的名字。
\VS{15}在胸牌上,用精金拧成如绳子的链子。
\VS{16}又做两个金槽和两个金环,安在胸牌的两头。
\VS{17}把那两条拧成的金链子穿过胸牌两头的环子,
\VS{18}又把链子的那两头接在两槽上,安在以弗得前面肩带上。
\VS{19}做两个金环,安在胸牌的两头,在以弗得里面的边上,
\VS{20}又做两个金环,安在以弗得前面两条肩带的下边,挨近相接之处,在以弗得巧工织的带子以上。
\VS{21}用一条蓝细带子把胸牌的环子和以弗得的环子系住,使胸牌贴在以弗得巧工织的带子上,不可与以弗得离缝,是照耶和华所吩咐{\PN{摩西}}的。
\par }{\SH 祭司其他的圣服
\par }{\R (出28·31—43)
\par }{\PP \VS{22}他用织工做以弗得的外袍,颜色全是蓝的。
\VS{23}袍上留一领口,口的周围织出领边来,仿佛铠甲的领口,免得破裂。
\VS{24}在袍子底边上,用蓝色、紫色、朱红色{\ADD{线}},并捻的{\ADD{细麻}}做石榴,
\VS{25}又用精金做铃铛,把铃铛钉在袍子周围底边上的石榴中间:
\VS{26}一个铃铛一个石榴,一个铃铛一个石榴,在袍子周围底边上用以供职,是照耶和华所吩咐{\PN{摩西}}的。
\par }{\PP \VS{27}他用织成的细麻布为{\PN{亚伦}}和他的儿子做内袍,
\VS{28}并用细麻布做冠冕和华美的裹头巾,用捻的细麻布做裤子,
\VS{29}又用蓝色、紫色、朱红色{\ADD{线}},并捻的细麻,以绣花的手工做腰带,是照耶和华所吩咐{\PN{摩西}}的。
\VS{30}他用精金做圣冠上的牌,在上面按刻图书之法,刻着「归耶和华为圣」。
\VS{31}又用一条蓝细带子将牌系在冠冕上,是照耶和华所吩咐{\PN{摩西}}的。
\par }{\SH 会幕完工
\par }{\R (出35·10—19)
\par }{\PP \VS{32}帐幕,就是会幕,一切的工就这样做完了。凡耶和华所吩咐{\PN{摩西}}的,{\PN{以色列}}人都照样做了。
\VS{33}他们送到{\PN{摩西}}那里。帐幕和帐幕的一切器具,就是钩子、板、闩、柱子、带卯的座,
\VS{34}染红公羊皮的盖、海狗皮的顶盖,和遮掩{\ADD{柜}}的幔子,
\VS{35}法柜和柜的杠并施恩座,
\VS{36}桌子和桌子的一切器具并陈设饼,
\VS{37}精{\ADD{金}}的灯台和摆列的灯盏,与灯台的一切器具,并点灯的油,
\VS{38}金坛、膏油、馨香的香料、会幕的门帘,
\VS{39}铜坛和坛上的铜网,坛的杠并坛的一切器具,洗濯盆和盆座,
\VS{40}院子的帷子和柱子,并带卯的座,院子的门帘、绳子、橛子,并帐幕和会幕中一切使用的器具,
\VS{41}精工做的礼服,和祭司{\PN{亚伦}}并他儿子在圣所用以供祭司职分的圣衣。
\VS{42}这一切工作都是{\PN{以色列}}人照耶和华所吩咐{\PN{摩西}}做的。
\VS{43}耶和华怎样吩咐的,他们就怎样做了。{\PN{摩西}}看见一切的工都做成了,就给他们祝福。

\par }\Chap{40}{\SH 耶和华会幕的建立和奉献
\par }{\PP \VerseOne{1}耶和华晓谕{\PN{摩西}}说:
\VS{2}「正月初一日,你要立起帐幕,
\VS{3}把法柜安放在里面,用幔子将柜遮掩。
\VS{4}把桌子搬进去,摆设上面的物。把灯台搬进去,点其上的灯。
\VS{5}把{\ADD{烧}}香的金坛安在法柜前,挂上帐幕的门帘。
\VS{6}把燔祭坛安在帐幕门前。
\VS{7}把洗濯盆安在会幕和坛的中间,在盆里盛水。
\VS{8}又在四围立院帷,把院子的门帘挂上。
\VS{9}用膏油把帐幕和其中所有的都抹上,使帐幕和一切器具成圣,就都成圣。
\VS{10}又要抹燔祭坛和一切器具,使坛成圣,就都成为至圣。
\VS{11}要抹洗濯盆和盆座,使盆成圣。
\VS{12}要使{\PN{亚伦}}和他儿子到会幕门口来,用水洗身。
\VS{13}要给{\PN{亚伦}}穿上圣衣,又膏他,使他成圣,可以给我供祭司的职分;
\VS{14}又要使他儿子来,给他们穿上内袍。
\VS{15}怎样膏他们的父亲,也要照样膏他们,使他们给我供祭司的职分。他们世世代代凡受膏的,就永远当祭司的职任。」
\par }{\PP \VS{16}{\PN{摩西}}这样行,都是照耶和华所吩咐他的。
\VS{17}第二年正月初一日,帐幕就立起来。
\VS{18}{\PN{摩西}}立起帐幕,安上带卯的座,立上板,穿上闩,立起柱子。
\VS{19}在帐幕以上搭罩棚,把罩棚的顶盖盖在其上,是照耶和华所吩咐他的。
\VS{20}又把法{\ADD{版}}放在柜里,把杠穿在柜的两旁,把施恩座安在柜上。
\VS{21}把柜抬进帐幕,挂上遮掩{\ADD{柜}}的幔子,把法柜遮掩了,是照耶和华所吩咐他的。
\VS{22}又把桌子安在会幕内,在帐幕北边,在幔子外。
\VS{23}在桌子上将饼陈设在耶和华面前,是照耶和华所吩咐他的。
\VS{24}又把灯台安在会幕内,在帐幕南边,与桌子相对,
\VS{25}在耶和华面前点灯,是照耶和华所吩咐他的。
\VS{26}把金坛安在会幕内的幔子前,
\VS{27}在坛上烧了馨香料做的香,是照耶和华所吩咐他的。
\VS{28}又挂上帐幕的门帘。
\VS{29}在会幕的帐幕门前,安设燔祭坛,把燔祭和素祭献在其上,是照耶和华所吩咐他的。
\VS{30}把洗濯盆安在会幕和坛的中间,盆中盛水,以便洗濯。
\VS{31}{\PN{摩西}}和{\PN{亚伦}}并{\PN{亚伦}}的儿子在这盆里洗手洗脚。
\VS{32}他们进会幕或就近坛的时候,便都洗濯,是照耶和华所吩咐他的。
\VS{33}在帐幕和坛的四围立了院帷,把院子的门帘挂上。这样,{\PN{摩西}}就完了工。
\par }{\SH 云彩覆盖会幕
\par }{\R (民9·15—23)
\par }{\PP \VS{34}当时,云彩遮盖会幕,耶和华的荣光就充满了帐幕。
\VS{35}{\PN{摩西}}不能进会幕;因为云彩停在其上,并且耶和华的荣光充满了帐幕。
\VS{36}每逢云彩从帐幕收上去,{\PN{以色列}}人就起程前往;
\VS{37}云彩若不收上去,他们就不起程,直等到云彩收上去。
\VS{38}日间,耶和华的云彩是在帐幕以上;夜间,云中有火,在{\PN{以色列}}全家的眼前。在他们所行的路上都是这样。
\par }