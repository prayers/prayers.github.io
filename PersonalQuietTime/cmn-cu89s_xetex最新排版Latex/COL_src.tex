\NormalFont\ShortTitle{歌罗西书}
{\MT 歌罗西书

\par }\ChapOne{1}{\SH 问候
\par }{\PP \VerseOne{1}奉 神旨意,作基督耶稣使徒的{\PN{保罗}}和兄弟{\PN{提摩太}}
\VS{2}写信给{\PN{歌罗西}}的圣徒,在基督里有忠心的弟兄。愿恩惠、平安从 神我们的父归与你们!
\par }{\SH 为信徒感谢 神
\par }{\PP \VS{3}我们感谢 神我们主耶稣基督的父,常常为你们祷告;
\VS{4}因听见你们在基督耶稣里的信心,并向众圣徒的爱心,
\VS{5}是为那给你们存在天上的盼望;这盼望就是你们从前在福音真理的道上所听见的。
\VS{6}这福音传到你们那里,也传到普天之下,并且结果,增长,如同在你们中间,自从你们听见福音,真知道 神恩惠的日子一样。
\VS{7}正如你们从我们所亲爱、一同作仆人的{\PN{以巴弗}}所学的。他为我们\FTNT{}{{\FR 1:7: }有古卷:你们}作了基督忠心的执事,
\VS{8}也把你们因{\ADD{圣}}灵所存的爱心告诉了我们。
\par }{\SH 基督的位格和工作
\par }{\PP \VS{9}因此,我们自从听见的日子,也就为你们不住地祷告祈求,愿你们在一切属灵的智慧悟性上,满心知道 神的旨意;
\VS{10}好叫你们行事为人对得起主,凡事蒙他喜悦,在一切善事上结果子,渐渐地多知道 神;
\VS{11}照他荣耀的权能,得以在各样的力上加力,好叫你们凡事欢欢喜喜地忍耐宽容;
\VS{12}又感谢父,叫我们能与众圣徒在光明中同得基业。
\VS{13}他救了我们脱离黑暗的权势,把我们迁到他爱子的国里;
\VS{14}我们在爱子里得蒙救赎,罪过得以赦免。
\VS{15}爱子是那不能看见之 神的像,是首生的,在一切被造的以先。
\VS{16}因为万有都是靠他造的,无论是天上的,地上的;能看见的,不能看见的;或是有位的,主治的,执政的,掌权的;一概都是借着他造的,又是为他造的。
\VS{17}他在万有之先;万有也靠他而立。
\VS{18}他也是教会全体之首。他是元始,是从死里首先复生的,使他可以在凡事上居首位。
\VS{19}因为{\ADD{父}}喜欢叫一切的丰盛在他里面居住。
\VS{20}既然借着他在十字架上所流的血成就了和平,便借着他叫万有—无论是地上的、天上的—都与自己和好了。
\par }{\PP \VS{21}你们从前与 神隔绝,因着恶行,心里{\ADD{与他}}为敌。
\VS{22}但如今他借着基督的肉身受死,叫你们与自己和好,都成了圣洁,没有瑕疵,无可责备,把你们引到自己面前。
\VS{23}只要你们在所信的道上恒心,根基稳固,坚定不移,不至被引动失去\FTNT{}{{\FR 1:23: }原文是离开}福音的盼望。这福音就是你们所听过的,也是传与普天下万人听的\FTNT{}{{\FR 1:23: }原文是凡受造的},我—{\PN{保罗}}也作了这福音的执事。
\par }{\SH 保罗为教会的工作
\par }{\PP \VS{24}现在我为你们受苦,倒觉欢乐;并且为基督的身体,就是为教会,要在我肉身上补满基督患难的缺欠。
\VS{25}我照 神为你们所赐我的职分作了教会的执事,要把 神的道理传得全备,
\VS{26}这道理就是历世历代所隐藏的奥秘;但如今向他的圣徒显明了。
\VS{27}神愿意叫他们知道,这奥秘在外邦人中有何等丰盛的荣耀,就是基督在你们心里成了{\ADD{有}}荣耀的盼望。
\VS{28}我们传扬他,是用诸般的智慧,劝戒各人,教导各人,要把各人在基督里完完全全地引到 神面前。
\VS{29}我也为此劳苦,照着他在我里面运用的大能尽心竭力。

\par }\Chap{2}{\PP \VerseOne{1}我愿意你们晓得,我为你们和{\PN{老底嘉}}人,并一切没有与我亲自见面的人,是何等地尽心竭力;
\VS{2}要叫他们的心得安慰,因爱心互相联络,以致丰丰足足在悟性中有充足的信心,使他们真知 神的奥秘,就是基督;
\VS{3}所积蓄的一切智慧知识,都在他里面藏着。
\VS{4}我说这话,免得有人用花言巧语迷惑你们。
\VS{5}我身子虽与你们相离,心却与你们同在,见你们循规蹈矩,信基督的心也坚固,我就欢喜了。
\par }{\SH 基督里生命的丰盛
\par }{\PP \VS{6}你们既然接受了主基督耶稣,就当遵他而行,
\VS{7}在他里面生根建造,信心坚固,正如你们所领的教训,感谢的心也更增长了。
\VS{8}你们要谨慎,恐怕有人用他的理学和虚空的妄言,不照着基督,乃照人间的遗传和世上的小学就把你们掳去。
\VS{9}因为 神本性一切的丰盛都有形有体地居住在基督里面,
\VS{10}你们在他里面也得了丰盛。他是各样执政掌权者的元首。
\VS{11}你们在他里面也受了不是人手所行的割礼,乃是基督使你们脱去肉体{\ADD{情欲}}的割礼。
\VS{12}你们既受洗与他一同埋葬,也就在此与他一同复活,都因信那叫他从死里复活 神的功用。
\VS{13}你们从前在过犯和未受割礼的肉体中死了, 神赦免了你们\FTNT{}{{\FR 2:13: }或译:我们}一切过犯,便叫你们与基督一同活过来;
\VS{14}又涂抹了在律例上所写、攻击我们、有碍于我们的字据,把它撤去,钉在十字架上。
\VS{15}既将一切执政的、掌权的掳来,明显给众人看,就仗着十字架夸胜。
\par }{\PP \VS{16}所以,不拘在饮食上,或节期、月朔、安息日都不可让人论断你们。
\VS{17}这些原是后事的影儿;那形体却是基督。
\VS{18}不可让人因着故意谦虚和敬拜天使,就夺去你们的奖赏。这等人拘泥在所见过的\FTNT{}{{\FR 2:18: }有古卷:这等人窥察所没有见过的},随着自己的欲心,无故地自高自大,
\VS{19}不持定元首。全身既然靠着他,筋节得以相助联络,就因 神大得长进。
\par }{\SH 基督里的新生活
\par }{\PP \VS{20-21}你们若是与基督同死,脱离了世上的小学,为什么仍像在世俗中活着、服从那「不可拿、不可尝、不可摸」等类的规条呢?
\VS{22}这都是照人所吩咐、所教导的。说到这一切,正用的时候就都败坏了。
\VS{23}这些规条使人徒有智慧之名,用私意崇拜,自表谦卑,苦待己身,其实在克制肉体的情欲上是毫无功效。

\par }\Chap{3}{\PP \VerseOne{1}所以,你们若真与基督一同复活,就当求在上面的事;那里有基督坐在 神的右边。
\VS{2}你们要思念上面的事,不要思念地上的事。
\VS{3}因为你们已经死了,你们的生命与基督一同藏在 神里面。
\VS{4}基督是我们的生命,他显现的时候,你们也要与他一同显现在荣耀里。
\par }{\PP \VS{5}所以,要治死你们在地上的肢体,就如淫乱、污秽、邪情、恶欲,和贪婪(贪婪就与拜偶像一样)。
\VS{6}因这些事, 神的忿怒必临到那悖逆之子。
\VS{7}当你们在这些事中活着的时候,也曾这样行过。
\VS{8}但现在你们要弃绝这一切的事,以及恼恨、忿怒、恶毒\FTNT{}{{\FR 3:8: }或译:阴毒}、毁谤,并口中污秽的言语。
\VS{9}不要彼此说谎;因你们已经脱去旧人和旧人的行为,
\VS{10}穿上了新人。这新人在知识上渐渐更新,正如造他主的形象。
\VS{11}在此并不分{\PN{希腊}}人、{\PN{犹太}}人,受割礼的、未受割礼的,化外人、{\PN{西古提}}人,为奴的、自主的,惟有基督是包括一切,又住在各人之内。
\par }{\PP \VS{12}所以,你们既是 神的选民,圣洁蒙爱的人,就要存\FTNT{}{{\FR 3:12: }原文是穿;下同}怜悯、恩慈、谦虚、温柔、忍耐的心。
\VS{13}倘若这人与那人有嫌隙,总要彼此包容,彼此饶恕;主怎样饶恕了你们,你们也要怎样饶恕人。
\VS{14}在这一切之外,要存着爱心,爱心就是联络全德的。
\VS{15}又要叫基督的平安在你们心里作主;你们也为此蒙召,归为一体;且要存感谢的心。
\VS{16}当用各样的智慧,把基督的道理丰丰富富地存在心里\FTNT{}{{\FR 3:16: }或译:当把基督的道理丰丰富富地存在心里,以各样的智慧},用诗章、颂词、灵歌,彼此教导,互相劝戒,心被恩感,歌颂 神。
\VS{17}无论做什么,或说话或行事,都要奉主耶稣的名,借着他感谢父 神。
\par }{\SH 新生活的本分
\par }{\PP \VS{18}你们作妻子的,当顺服自己的丈夫,这在主里面是相宜的。
\VS{19}你们作丈夫的,要爱你们的妻子,不可苦待她们。
\par }{\PP \VS{20}你们作儿女的,要凡事听从父母,因为这是主所喜悦的。
\VS{21}你们作父亲的,不要惹儿女的气,恐怕他们失了志气。
\par }{\PP \VS{22}你们作仆人的,要凡事听从你们肉身的主人,不要只在眼前事奉,像是讨人喜欢的,总要存心诚实敬畏主。
\VS{23}无论做什么,都要从心里做,像是给主做的,不是给人做的,
\VS{24}因你们知道从主那里必得着基业为赏赐;你们所事奉的乃是主基督。
\VS{25}那行不义的必受不义的报应;{\ADD{主}}并不偏待人。

\par }\Chap{4}{\PP \VerseOne{1}你们作主人的,要公公平平地待仆人,因为知道你们也有一位主在天上。
\par }{\SH 劝导
\par }{\PP \VS{2}你们要恒切祷告,在此警醒感恩。
\VS{3}也要为我们祷告,求 神给我们开传道的门,能以讲基督的奥秘(我为此被捆锁),
\VS{4}叫我按着所该说的话将这奥秘发明出来。
\VS{5}你们要爱惜光阴,用智慧与外人交往。
\VS{6}你们的言语要常常带着和气,{\ADD{好像}}用盐调和,就可知道该怎样回答各人。
\par }{\SH 最后的问候
\par }{\PP \VS{7}有我亲爱的兄弟{\PN{推基古}}要将我一切的事都告诉你们。他是忠心的执事,和我一同作主的仆人。
\VS{8}我特意打发他到你们那里去,好叫你们知道我们的光景,又叫他安慰你们的心。
\VS{9}我又打发一位亲爱忠心的兄弟{\PN{阿尼西谋}}同去;他也是你们那里的人。他们要把这里一切的事都告诉你们。
\par }{\PP \VS{10}与我一同坐监的{\PN{亚里达古}}问你们安。{\PN{巴拿巴}}的表弟{\PN{马可}}也问你们安。(说到这{\PN{马可}},你们已经受了吩咐;他若到了你们那里,你们就接待他。)
\VS{11}{\PN{耶数}}又称为{\PN{犹士都}},也问你们安。奉割礼的人中,只有这三个人是为 神的国与我一同做工的,也是叫我心里得安慰的。
\VS{12}有你们那里的人,作基督耶稣仆人的{\PN{以巴弗}}问你们安。他在祷告之间,常为你们竭力地祈求,愿你们在 神一切的旨意上得以完全,信心充足,能站立得稳。
\VS{13}他为你们和{\PN{老底嘉}}并{\PN{希拉坡里}}的弟兄多多地劳苦,这是我可以给他作见证的。
\VS{14}所亲爱的医生{\PN{路加}}和{\PN{底马}}问你们安。
\VS{15}请问{\PN{老底嘉}}的弟兄和{\PN{宁法}},并她家里的教会安。
\VS{16}你们念了这书信,便交给{\PN{老底嘉}}的教会,叫他们也念;你们也要念从{\PN{老底嘉}}来的书信。
\VS{17}要对{\PN{亚基布}}说:「务要谨慎,尽你从主所受的职分。」
\par }{\PP \VS{18}我—{\PN{保罗}}亲笔问你们安。你们要记念我的捆锁。愿恩惠{\ADD{常}}与你们同在!
\par }