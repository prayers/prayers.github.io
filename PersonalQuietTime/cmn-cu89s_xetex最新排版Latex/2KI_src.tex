\NormalFont\ShortTitle{列王纪下}
{\MT 列王纪下

\par }\ChapOne{1}{\SH 以利亚和亚哈谢王
\par }{\PP \VerseOne{1}{\PN{亚哈}}死后,{\PN{摩押}}背叛{\PN{以色列}}。
\VS{2}{\PN{亚哈谢}}在{\PN{撒马利亚}},一日从楼上的栏杆里掉下来,就病了;于是差遣使者说:「你们去问{\PN{以革伦}}的神{\PN{巴力·西卜}},我这病能好不能好。」
\VS{3}但耶和华的使者对{\PN{提斯比}}人{\PN{以利亚}}说:「你起来,去迎着{\PN{撒马利亚}}王的使者,对他们说:『你们去问{\PN{以革伦}}神{\PN{巴力·西卜}},岂因{\PN{以色列}}中没有 神吗?』
\VS{4}所以耶和华如此说:『你必不下你所上的床,必定要死!』」{\PN{以利亚}}就去了。
\par }{\PP \VS{5}使者回来见王,王问他们说:「你们为什么回来呢?」
\VS{6}使者回答说:「有一个人迎着我们来,对我们说:『你们回去见差你们来的王,对他说:耶和华如此说,你差人去问{\PN{以革伦}}神{\PN{巴力·西卜}},岂因{\PN{以色列}}中没有 神吗?所以你必不下所上的床,必定要死。』」
\VS{7}王问他们说:「迎着你们来告诉你们这话的,是怎样的人?」
\VS{8}回答说:「他身穿毛衣,腰束皮带。」王说:「这必是{\PN{提斯比}}人{\PN{以利亚}}。」
\par }{\PP \VS{9}于是,王差遣五十夫长,带领五十人去见{\PN{以利亚}},他就上到{\PN{以利亚}}那里;{\PN{以利亚}}正坐在山顶上。五十夫长对他说:「神人哪,王吩咐你下来!」
\VS{10}{\PN{以利亚}}回答说:「我若是神人,愿火从天上降下来,烧灭你和你那五十人!」于是有火从天上降下来,烧灭五十夫长和他那五十人。
\par }{\PP \VS{11}王第二次差遣一个五十夫长,带领五十人去见{\PN{以利亚}}。五十夫长对{\PN{以利亚}}说:「神人哪,王吩咐你快快下来!」
\VS{12}{\PN{以利亚}}回答说:「我若是神人,愿火从天上降下来,烧灭你和你那五十人!」于是 神的火从天上降下来,烧灭五十夫长和他那五十人。
\par }{\PP \VS{13}王第三次差遣一个五十夫长,带领五十人去。这五十夫长上去,双膝跪在{\PN{以利亚}}面前,哀求他说:「神人哪,愿我的性命和你这五十个仆人的性命在你眼前看为宝贵!
\VS{14}已经有火从天上降下来,烧灭前两次来的五十夫长和他们各自带的五十人;现在愿我的性命在你眼前看为宝贵!」
\VS{15}耶和华的使者对{\PN{以利亚}}说:「你同着他下去,不要怕他!」{\PN{以利亚}}就起来,同着他下去见王,
\VS{16}对王说:「耶和华如此说:『你差人去问{\PN{以革伦}}神{\PN{巴力·西卜}},岂因{\PN{以色列}}中没有 神可以求问吗?所以你必不下所上的床,必定要死!』」
\par }{\SH 亚哈谢死后约兰继位
\par }{\PP \VS{17}{\PN{亚哈谢}}果然死了,正如耶和华借{\PN{以利亚}}所说的话。因他没有儿子,{\ADD{他兄弟}}{\PN{约兰}}接续他作王,正在{\PN{犹大}}王{\PN{约沙法}}的儿子{\PN{约兰}}第二年。
\VS{18}{\PN{亚哈谢}}其余所行的事都写在{\PN{以色列}}诸王记上。

\par }\Chap{2}{\SH 以利亚被接升天
\par }{\PP \VerseOne{1}耶和华要用旋风接{\PN{以利亚}}升天的时候,{\PN{以利亚}}与{\PN{以利沙}}从{\PN{吉甲}}前往。
\VS{2}{\PN{以利亚}}对{\PN{以利沙}}说:「耶和华差我往{\PN{伯特利}}去,你可以在这里等候。」{\PN{以利沙}}说:「我指着永生的耶和华,又敢在你面前起誓,我必不离开你。」于是二人下到{\PN{伯特利}}。
\VS{3}住{\PN{伯特利}}的先知门徒出来见{\PN{以利沙}},对他说:「耶和华今日要接你的师傅离开你,你知道不知道?」他说:「我知道,你们不要作声。」
\par }{\PP \VS{4}{\PN{以利亚}}对{\PN{以利沙}}说:「耶和华差遣我往{\PN{耶利哥}}去,你可以在这里等候。」{\PN{以利沙}}说:「我指着永生的耶和华,又敢在你面前起誓,我必不离开你。」于是二人到了{\PN{耶利哥}}。
\VS{5}住{\PN{耶利哥}}的先知门徒就近{\PN{以利沙}},对他说:「耶和华今日要接你的师傅离开你,你知道不知道?」他说:「我知道,你们不要作声。」
\par }{\PP \VS{6}{\PN{以利亚}}对{\PN{以利沙}}说:「耶和华差遣我往{\PN{约旦河}}去,你可以在这里等候。」{\PN{以利沙}}说:「我指着永生的耶和华,又敢在你面前起誓,我必不离开你。」于是二人一同前往。
\VS{7}有先知门徒去了五十人,远远地站在他们对面;二人在{\PN{约旦河}}边站住。
\VS{8}{\PN{以利亚}}将自己的外衣卷起来,用以打水,水就左右分开,二人走干地而过。
\par }{\PP \VS{9}过去之后,{\PN{以利亚}}对{\PN{以利沙}}说:「我未曾被接去离开你,你要我为你做什么,只管求我。」{\PN{以利沙}}说:「愿{\ADD{感动}}你的灵加倍地感动我。」
\VS{10}{\PN{以利亚}}说:「你所求的难得。{\ADD{虽然如此}},我被接去离开你的时候,你若看见我,就必得着;不然,必得不着了。」
\par }{\SH 以利亚乘旋风升天
\par }{\PP \VS{11}他们正走着说话,忽有火车火马将二人隔开,{\PN{以利亚}}就乘旋风升天去了。
\VS{12}{\PN{以利沙}}看见,就呼叫说:「我父啊!我父啊!{\PN{以色列}}的战车马兵啊!」以后不再见他了。于是{\PN{以利沙}}把自己的衣服撕为两片。
\par }{\PP \VS{13}他拾起{\PN{以利亚}}身上掉下来的外衣,回去站在{\PN{约旦河}}边。
\VS{14}他用{\PN{以利亚}}身上掉下来的外衣打水,说:「耶和华—{\PN{以利亚}}的 神在哪里呢?」打水之后,水也左右分开,{\PN{以利沙}}就过来了。
\par }{\PP \VS{15}住{\PN{耶利哥}}的先知门徒从对面看见他,就说:「{\ADD{感动}}{\PN{以利亚}}的灵感动{\PN{以利沙}}了。」他们就来迎接他,在他面前俯伏于地,
\VS{16}对他说:「仆人们这里有五十个壮士,求你容他们去寻找你师傅,或者耶和华的灵将他提起来,投在某山某谷。」{\PN{以利沙}}说:「你们不必打发人去。」
\VS{17}他们再三催促他,他难以推辞,就说:「你们打发人去吧!」他们便打发五十人去,寻找了三天,也没有找着。
\VS{18}{\PN{以利沙}}仍然在{\PN{耶利哥}}等候他们回到他那里;他对他们说:「我岂没有告诉你们不必去吗?」
\par }{\SH 以利沙行的神迹
\par }{\PP \VS{19}{\PN{耶利哥}}城的人对{\PN{以利沙}}说:「这城的地势美好,我主看见了;只是水恶劣,土产不熟而落。」
\VS{20}{\PN{以利沙}}说:「你们拿一个新瓶来,装盐给我」;他们就拿来给他。
\VS{21}他出到水源,将盐倒在水中,说:「耶和华如此说:『我治好了这水,从此必不再使人死,也不再使地土不生产。』」
\VS{22}于是那水治好了,直到今日,正如{\PN{以利沙}}所说的。
\par }{\PP \VS{23}{\PN{以利沙}}从那里上{\PN{伯特利}}去,正上去的时候,有些童子从城里出来,戏笑他说:「秃头的上去吧!秃头的上去吧!」
\VS{24}他回头看见,就奉耶和华的名咒诅他们。于是有两只母熊从林中出来,撕裂他们中间四十二个童子。
\VS{25}{\PN{以利沙}}从{\PN{伯特利}}上{\PN{迦密山}},又从{\PN{迦密山}}回到{\PN{撒马利亚}}。

\par }\Chap{3}{\SH 以色列与摩押争战
\par }{\PP \VerseOne{1}{\PN{犹大}}王{\PN{约沙法}}十八年,{\PN{亚哈}}的儿子{\PN{约兰}}在{\PN{撒马利亚}}登基作了{\PN{以色列}}王十二年。
\VS{2}他行耶和华眼中看为恶的事,但不至像他父母所行的,因为除掉他父所造{\PN{巴力}}的柱像。
\VS{3}然而,他贴近{\PN{尼八}}的儿子{\PN{耶罗波安}}使{\PN{以色列}}人陷在罪里的那罪,总不离开。
\par }{\PP \VS{4}{\PN{摩押}}王{\PN{米沙}}牧养许多羊,{\ADD{每年}}将十万羊羔的毛和十万公绵羊的毛给{\PN{以色列}}王进贡。
\VS{5}{\PN{亚哈}}死后,{\PN{摩押}}王背叛{\PN{以色列}}王。
\VS{6}那时{\PN{约兰}}王出{\PN{撒马利亚}},数点{\PN{以色列}}众人。
\VS{7}前行的时候,差人去见{\PN{犹大}}王{\PN{约沙法}},说:「{\PN{摩押}}王背叛我,你肯同我去攻打{\PN{摩押}}吗?」他说:「我肯上去,你我不分彼此,我的民与你的民一样,我的马与你的马一样。」
\VS{8}{\PN{约兰}}说:「我们从哪条路上去呢?」回答说:「从{\PN{以东}}旷野的路上去。」
\par }{\PP \VS{9}于是,{\PN{以色列}}王和{\PN{犹大}}王,并{\PN{以东}}王,都一同去绕行七日的路程;军队和所带的牲畜没有水喝。
\VS{10}{\PN{以色列}}王说:「哀哉!耶和华招聚我们这三王,乃要交在{\PN{摩押}}人的手里。」
\VS{11}{\PN{约沙法}}说:「这里不是有耶和华的先知吗?我们可以托他求问耶和华。」{\PN{以色列}}王的一个臣子回答说:「这里有{\PN{沙法}}的儿子{\PN{以利沙}},就是从前服事{\PN{以利亚}}的\FTNT{}{{\FR 3:11: }原文是倒水在以利亚手上的}。」
\VS{12}{\PN{约沙法}}说:「他必有耶和华的话。」于是{\PN{以色列}}王和{\PN{约沙法}},并{\PN{以东}}王都下去见他。
\par }{\PP \VS{13}{\PN{以利沙}}对{\PN{以色列}}王说:「我与你何干?去问你父亲的先知和你母亲的先知吧!」{\PN{以色列}}王对他说:「不要这样说,耶和华招聚我们这三王,乃要交在{\PN{摩押}}人的手里。」
\VS{14}{\PN{以利沙}}说:「我指着所事奉永生的万军耶和华起誓,我若不看{\PN{犹大}}王{\PN{约沙法}}的情面,必不理你,不顾你。
\VS{15}现在你们给我找一个弹琴的来。」弹琴的时候,耶和华的灵\FTNT{}{{\FR 3:15: }原文是手}就降在{\PN{以利沙}}身上。
\VS{16}他便说:「耶和华如此说:『你们要在这谷中满处挖沟;
\VS{17}因为耶和华如此说:你们虽不见风,不见雨,这谷必满了水,使你们和牲畜有水喝。』
\VS{18}在耶和华眼中这还算为小事,他也必将{\PN{摩押}}人交在你们手中。
\VS{19}你们必攻破一切坚城美邑,砍伐各种佳树,塞住一切水泉,用石头糟踏一切美田。」
\VS{20}次日早晨,约在献祭的时候,有水从{\PN{以东}}而来,遍地就满了水。
\par }{\SH 摩押人败遁
\par }{\PP \VS{21}{\PN{摩押}}众人听见这三王上来要与他们争战,凡能顶盔贯甲的,无论老少,尽都聚集站在边界上。
\VS{22}次日早晨,日光照在水上,{\PN{摩押}}人起来,看见对面水红如血,
\VS{23}就说:「这是血啊!必是三王互相击杀,俱都灭亡。{\PN{摩押}}人哪,我们现在去抢夺财物吧!」
\VS{24}{\PN{摩押}}人到了{\PN{以色列}}营,{\PN{以色列}}人就起来攻打他们,以致他们在{\PN{以色列}}人面前逃跑。{\PN{以色列}}人往前追杀{\PN{摩押}}人,直杀入{\PN{摩押}}的境内,
\VS{25}拆毁{\PN{摩押}}的城邑,各人抛石填满一切美田,塞住一切水泉,砍伐各种佳树,只剩下{\PN{吉珥·哈列设}}的石{\ADD{墙}};甩石的兵在四围攻打那城。
\VS{26}{\PN{摩押}}王见阵势甚大,难以对敌,就率领七百拿刀的兵,要冲过阵去到{\PN{以东}}王那里,却是不能;
\VS{27}便将那应当接续他作王的长子,在城上献为燔祭。{\PN{以色列}}人遭遇{\ADD{耶和华的}}大怒\FTNT{}{{\FR 3:27: }或译:招人痛恨},于是三王离开{\PN{摩押}}王,各回本国去了。

\par }\Chap{4}{\SH 以利沙帮助一个穷寡妇
\par }{\PP \VerseOne{1}有一个先知门徒的妻哀求{\PN{以利沙}}说:「你仆人—我丈夫死了,他敬畏耶和华是你所知道的。现在有债主来,要取我两个儿子作奴仆。」
\VS{2}{\PN{以利沙}}问她说:「我可以为你做什么呢?你告诉我,你家里有什么?」她说:「婢女家中除了一瓶油之外,没有什么。」
\VS{3}{\PN{以利沙}}说:「你去,向你众邻舍借空器皿,不要少借;
\VS{4}回到家里,关上门,你和你儿子在里面将油倒在所有的器皿里,倒满了的放在一边。」
\VS{5}于是,妇人离开{\PN{以利沙}}去了,关上门,自己和儿子在里面;儿子把{\ADD{器皿}}拿来,她就倒油。
\VS{6}器皿都满了,她对儿子说:「再给我拿器皿来。」儿子说:「再没有器皿了。」油就止住了。
\VS{7}妇人去告诉神人,神人说:「你去卖油还债,所剩的你和你儿子可以靠着度日。」
\par }{\SH 书念妇人接待以利沙
\par }{\PP \VS{8}一日,{\PN{以利沙}}走到{\PN{书念}},在那里有一个大户的妇人强留他吃饭。此后,{\PN{以利沙}}每从那里经过就进去吃饭。
\VS{9}妇人对丈夫说:「我看出那常从我们这里经过的是圣洁的神人。
\VS{10}我们可以为他在墙上盖一间小楼,在其中安放床榻、桌子、椅子、灯台,他来到我们这里,就可以住在其间。」
\par }{\PP \VS{11}一日,{\PN{以利沙}}来到那里,就进了那楼躺卧。
\VS{12}{\PN{以利沙}}吩咐仆人{\PN{基哈西}}说:「你叫这{\PN{书念}}妇人来。」他就把妇人叫了来,妇人站在{\PN{以利沙}}面前。
\VS{13}{\PN{以利沙}}吩咐仆人说:「你对她说:你既为我们费了许多心思,可以为你做什么呢?你向王或元帅有所求的没有?」她回答说:「我在我本乡安居无事。」
\VS{14}{\PN{以利沙}}{\ADD{对仆人}}说:「究竟当为她做什么呢?」{\PN{基哈西}}说:「她没有儿子,她丈夫也老了。」
\VS{15}{\PN{以利沙}}说:「再叫她来。」于是叫了她来,她就站在门口。
\VS{16}{\PN{以利沙}}说:「明年到这时候,你必抱一个儿子。」她说:「神人,我主啊,不要那样欺哄婢女。」
\VS{17}妇人果然怀孕,到了那时候,生了一个儿子,正如{\PN{以利沙}}所说的。
\par }{\PP \VS{18}孩子渐渐长大,一日到他父亲和收割的人那里,
\VS{19}他对父亲说:「我的头啊,我的头啊!」他父亲对仆人说:「把他抱到他母亲那里。」
\VS{20}仆人抱去,交给他母亲;孩子坐在母亲的膝上,到晌午就死了。
\VS{21}他母亲抱他上了楼,将他放在神人的床上,关上{\ADD{门}}出来,
\VS{22}呼叫她丈夫说:「你叫一个仆人给我牵一匹驴来,我要快快地去见神人,就回来。」
\VS{23}丈夫说:「今日不是月朔,也不是安息日,你为何要去见他呢?」妇人说:「平安无事。」
\VS{24}于是备上驴,对仆人说:「你快快赶着走,我若不吩咐你,就不要迟慢。」
\VS{25}妇人就往{\PN{迦密山}}去见神人。
\par }{\PP 神人远远地看见她,对仆人{\PN{基哈西}}说:「看哪,{\PN{书念}}的妇人来了!
\VS{26}你跑去迎接她,问她说:你平安吗?你丈夫平安吗?孩子平安吗?」她说:「平安。」
\VS{27}妇人上了山,到神人那里,就抱住神人的脚。{\PN{基哈西}}前来要推开她,神人说:「由她吧!因为她心里愁苦,耶和华向我隐瞒,没有指示我。」
\VS{28}妇人说:「我何尝向我主求过儿子呢?我岂没有说过,不要欺哄我吗?」
\VS{29}{\PN{以利沙}}吩咐{\PN{基哈西}}说:「你束上腰,手拿我的杖前去;若遇见人,不要向他问安;人若向你问安,也不要回答;要把我的杖放在孩子脸上。」
\VS{30}孩子的母亲说:「我指着永生的耶和华,又敢在你面前起誓,我必不离开你。」于是{\PN{以利沙}}起身,随着她去了。
\VS{31}{\PN{基哈西}}先去,把杖放在孩子脸上,却没有声音,也没有动静。{\PN{基哈西}}就迎着{\PN{以利沙}}回来,告诉他说:「孩子还没有醒过来。」
\par }{\PP \VS{32}{\PN{以利沙}}来到,进了屋子,看见孩子死了,放在自己的床上。
\VS{33}他就关上门,只有自己和孩子在里面,他便祈祷耶和华,
\VS{34}上床伏在孩子身上,口对口,眼对眼,手对手;既伏在孩子身上,孩子的身体就渐渐温和了。
\VS{35}然后他下来,在屋里来往走了一趟,又上去伏在孩子身上,孩子打了七个喷嚏,就睁开眼睛了。
\VS{36}{\PN{以利沙}}叫{\PN{基哈西}}说:「你叫这{\PN{书念}}妇人来」;于是叫了她来。{\PN{以利沙}}说:「将你儿子抱起来。」
\VS{37}妇人就进来,在{\PN{以利沙}}脚前俯伏于地,抱起她儿子出去了。
\par }{\SH 两件神迹
\par }{\PP \VS{38}{\PN{以利沙}}又来到{\PN{吉甲}},那地正有饥荒。先知门徒坐在他面前,他吩咐仆人说:「你将大锅放在火上,给先知门徒熬汤。」
\VS{39}有一个人去到田野掐菜,遇见一棵野瓜藤,就摘了一兜野瓜回来,切了搁在熬汤的锅中,因为他们不知道是什么东西;
\VS{40}倒出来给众人吃,吃的时候,都喊叫说:「神人哪,锅中有致死的毒物!」所以众人不能吃了。
\VS{41}{\PN{以利沙}}说:「拿点面来」,就把面撒在锅中,说:「倒出来,给众人吃吧!」锅中就没有毒了。
\par }{\PP \VS{42}有一个人从{\PN{巴力·沙利沙}}来,带着初熟大麦做的饼二十个,并新穗子,装在口袋里送给神人。神人说:「把这些给众人吃。」
\VS{43}仆人说:「这一点岂可摆给一百人吃呢?」{\PN{以利沙}}说:「你只管给众人吃吧!因为耶和华如此说,众人必吃了,还剩下。」
\VS{44}仆人就摆在众人面前,他们吃了,果然还剩下,正如耶和华所说的。

\par }\Chap{5}{\SH 乃缦得医治
\par }{\PP \VerseOne{1}{\PN{亚兰}}王的元帅{\PN{乃缦}}在他主人面前为尊为大,因耶和华曾借他使{\PN{亚兰}}人得胜;他又是大能的勇士,只是长了大麻风。
\VS{2}先前{\PN{亚兰}}人成群地出去,从{\PN{以色列}}国掳了一个小女子,这女子就服事{\PN{乃缦}}的妻。
\VS{3}她对主母说:「巴不得我主人去见{\PN{撒马利亚}}的先知,必能治好他的大麻风。」
\VS{4}{\PN{乃缦}}进去,告诉他主人说,{\PN{以色列}}国的女子如此如此说。
\VS{5}{\PN{亚兰}}王说:「你可以去,我也达信于{\PN{以色列}}王。」于是{\PN{乃缦}}带银子十他连得,金子六千{\ADD{舍客勒}},衣裳十套,就去了;
\VS{6}且带信给{\PN{以色列}}王,信上说:「我打发臣仆{\PN{乃缦}}去见你,你接到这信,就要治好他的大麻风。」
\VS{7}{\PN{以色列}}王看了信就撕裂衣服,说:「我岂是 神,能使人死使人活呢?这人竟打发人来,叫我治好他的大麻风。你们看一看,这人何以寻隙攻击我呢?」
\par }{\PP \VS{8}神人{\PN{以利沙}}听见{\PN{以色列}}王撕裂衣服,就打发人去见王,说:「你为什么撕了衣服呢?可使那人到我这里来,他就知道{\PN{以色列}}中有先知了。」
\VS{9}于是,{\PN{乃缦}}带着车马到了{\PN{以利沙}}的家,站在门前。
\VS{10}{\PN{以利沙}}打发一个使者,对{\PN{乃缦}}说:「你去在{\PN{约旦河}}中沐浴七回,你的肉就必复原,而得洁净。」
\VS{11}{\PN{乃缦}}却发怒走了,说:「我想他必定出来见我,站着求告耶和华—他 神的名,在患处以上摇手,治好这大麻风。
\VS{12}{\PN{大马士革}}的河{\PN{亚罢拿}}和{\PN{法珥法}}岂不比{\PN{以色列}}的一切水更好吗?我在那里沐浴不得洁净吗?」于是气忿忿地转身去了。
\VS{13}他的仆人进前来,对他说:「我父啊,先知若吩咐你做一件大事,你岂不做吗?何况说你去沐浴而得洁净呢?」
\VS{14}于是{\PN{乃缦}}下去,照着神人的话,在{\PN{约旦河}}里沐浴七回;他的肉复原,好像小孩子的肉,他就洁净了。
\par }{\PP \VS{15}{\PN{乃缦}}带着一切跟随他的人,回到神人那里,站在他面前,说:「如今我知道,除了{\PN{以色列}}之外,普天下没有 神。现在求你收点仆人的礼物。」
\VS{16}{\PN{以利沙}}说:「我指着所事奉永生的耶和华起誓,我必不受。」{\PN{乃缦}}再三地求他,他却不受。
\VS{17}{\PN{乃缦}}说:「你若不肯受,请将两骡子驮的土赐给仆人。从今以后,仆人必不再将燔祭或{\ADD{平安}}祭献与别神,只献给耶和华。
\VS{18}惟有一件事,愿耶和华饶恕你仆人:我主人进{\PN{临门}}庙叩拜的时候,我用手搀他在{\PN{临门}}庙,我也屈身。我在{\PN{临门}}庙屈身的这事,愿耶和华饶恕我。」
\VS{19}{\PN{以利沙}}对他说:「你可以平平安安地回去!」
\par }{\PP {\PN{乃缦}}就离开他去了;走了不远,
\VS{20}神人{\PN{以利沙}}的仆人{\PN{基哈西}}{\ADD{心里}}说:「我主人不愿从这{\PN{亚兰}}人{\PN{乃缦}}手里受他带来的礼物,我指着永生的耶和华起誓,我必跑去追上他,向他要些。」
\VS{21}于是{\PN{基哈西}}追赶{\PN{乃缦}}。{\PN{乃缦}}看见有人追赶,就急忙下车迎着他,说:「都平安吗?」
\VS{22}说:「都平安。我主人打发我来说:『刚才有两个少年人,是先知门徒,从{\PN{以法莲}}山地来见我,请你赐他们一他连得银子,两套衣裳。』」
\VS{23}{\PN{乃缦}}说:「请受二他连得」;再三地请受,便将二他连得银子装在两个口袋里,又将两套衣裳交给两个仆人;他们就在{\PN{基哈西}}前头抬着走。
\VS{24}到了山冈,{\PN{基哈西}}从他们手中接过来,放在屋里,打发他们回去。
\VS{25}{\PN{基哈西}}进去,站在他主人面前。{\PN{以利沙}}问他说:「{\PN{基哈西}}你从哪里来?」回答说:「仆人没有往哪里去。」
\VS{26}{\PN{以利沙}}对他说:「那人下车转回迎你的时候,我的心岂没有去呢?这岂是受银子、衣裳、{\ADD{买}}橄榄园、葡萄园、牛羊、仆婢的时候呢?
\VS{27}因此,{\PN{乃缦}}的大麻风必沾染你和你的后裔,直到永远。」{\PN{基哈西}}从{\PN{以利沙}}面前退出去,就长了大麻风,像雪{\ADD{那样白}}。

\par }\Chap{6}{\SH 找回斧子的头
\par }{\PP \VerseOne{1}先知门徒对{\PN{以利沙}}说:「看哪,我们同你所住的地方过于窄小,
\VS{2}求你容我们往{\PN{约旦河}}去,各人从那里取一根木料建造房屋居住。」他说:「你们去吧!」
\VS{3}有一人说:「求你与仆人同去。」回答说:「我可以去。」
\VS{4}于是{\PN{以利沙}}与他们同去。到了{\PN{约旦河}},就砍伐树木。
\VS{5}有一人砍树的时候,斧头掉在水里,他就呼叫说:「哀哉!我主啊,这斧子是借的。」
\VS{6}神人问说:「掉在哪里了?」他将那地方指给{\PN{以利沙}}看。{\PN{以利沙}}砍了一根木头,抛在水里,斧头就漂上来了。
\VS{7}{\PN{以利沙}}说:「拿起来吧!」那人就伸手拿起来了。
\par }{\SH 亚兰军败于以色列人
\par }{\PP \VS{8}{\PN{亚兰}}王与{\PN{以色列}}人争战,和他的臣仆商议说:「我要在某处某处安营。」
\VS{9}神人打发人去见{\PN{以色列}}王,说:「你要谨慎,不要从某处经过,因为{\PN{亚兰}}人从那里下来了。」
\VS{10}{\PN{以色列}}王差人去{\ADD{窥探}}神人所告诉所警戒他去的地方,就防备未受其害,不止一两次。
\par }{\PP \VS{11}{\PN{亚兰}}王因这事心里惊疑,召了臣仆来,对他们说:「我们这里有谁帮助{\PN{以色列}}王,你们不指给我吗?」
\VS{12}有一个臣仆说:「我主,我王!无人帮助他,只有{\PN{以色列}}中的先知{\PN{以利沙}},将王在卧房所说的话告诉{\PN{以色列}}王了。」
\VS{13}王说:「你们去探他在哪里,我好打发人去捉拿他。」有人告诉王说:「他在{\PN{多坍}}。」
\VS{14}王就打发车马和大军往那里去,夜间到了,围困那城。
\par }{\PP \VS{15}神人的仆人清早起来出去,看见车马军兵围困了城。仆人对神人说:「哀哉!我主啊,我们怎样行才好呢?」
\VS{16}神人说:「不要惧怕!与我们同在的比与他们同在的更多。」
\VS{17}{\PN{以利沙}}祷告说:「耶和华啊,求你开这少年人的眼目,使他能看见。」耶和华开他的眼目,他就看见满山有火车火马围绕{\PN{以利沙}}。
\VS{18}敌人下到{\PN{以利沙}}那里,{\PN{以利沙}}祷告耶和华说:「求你使这些人的眼目昏迷。」耶和华就照{\PN{以利沙}}的话,使他们的眼目昏迷。
\VS{19}{\PN{以利沙}}对他们说:「这不是那道,也不是那城;你们跟我去,我必领你们到所寻找的人那里。」于是领他们到了{\PN{撒马利亚}}。
\par }{\PP \VS{20}他们进了{\PN{撒马利亚}},{\PN{以利沙}}{\ADD{祷告}}说:「耶和华啊,求你开这些人的眼目,使他们能看见。」耶和华开他们的眼目,他们就看见了,不料,是在{\PN{撒马利亚}}的城中。
\VS{21}{\PN{以色列}}王见了他们,就问{\PN{以利沙}}说:「我父啊,我可以击杀他们吗?」
\VS{22}回答说:「不可击杀他们!就是你用刀用弓掳来的,岂可击杀他们吗\FTNT{}{{\FR 6:22: }或译:也不可击杀,何况这些人呢}?当在他们面前设摆饮食,使他们吃喝回到他们的主人那里。」
\VS{23}王就为他们预备了许多食物;他们吃喝完了,打发他们回到他们主人那里。从此,{\PN{亚兰}}军不再犯{\PN{以色列}}境了。
\par }{\SH 围困撒马利亚
\par }{\PP \VS{24}此后,{\PN{亚兰}}王{\PN{便·哈达}}聚集他的全军,上来围困{\PN{撒马利亚}}。
\VS{25}于是{\PN{撒马利亚}}被围困,有饥荒,甚至一个驴头值银八十{\ADD{舍客勒}},二升鸽子粪值银五{\ADD{舍客勒}}。
\VS{26}一日,{\PN{以色列}}王在城上经过,有一个妇人向他呼叫说:「我主,我王啊!求你帮助。」
\VS{27}王说:「耶和华不帮助你,我从何处帮助你?是从禾场,是从酒榨呢?」
\VS{28}王问妇人说:「你有什么苦处?」她回答说:「这妇人对我说:『将你的儿子取来,我们今日可以吃,明日可以吃我的儿子。』
\VS{29}我们就煮了我的儿子吃了。次日我对她说:『要将你的儿子取来,我们可以吃。』她却将她的儿子藏起来了。」
\VS{30}王听见妇人的话,就撕裂衣服;(王在城上经过)百姓看见王贴身穿着麻衣。
\VS{31}王说:「我今日若容{\PN{沙法}}的儿子{\PN{以利沙}}的头仍在他项上,愿 神重重地降罚与我!」
\par }{\PP \VS{32}那时,{\PN{以利沙}}正坐在家中,长老也与他同坐。{\ADD{王}}打发一个伺候他的人去;他还没有到,{\PN{以利沙}}对长老说:「你们看这凶手之子,打发人来斩我的头;你们看着使者来到,就关上门,用门将他推出去。在他后头不是有他主人脚步的响声吗?」
\VS{33}正说话的时候,使者来到,{\ADD{王也到了}},说:「这灾祸是从耶和华那里来的,我何必再仰望耶和华呢?」

\par }\Chap{7}{\PP \VerseOne{1}{\PN{以利沙}}说:「你们要听耶和华的话,耶和华如此说:明日约到这时候,在{\PN{撒马利亚}}城门口,一细亚细面要{\ADD{卖}}银一舍客勒,二细亚大麦也要{\ADD{卖}}银一舍客勒。」
\VS{2}有一个搀扶王的军长对神人说:「即便耶和华使天开了窗户,也不能有这事。」{\PN{以利沙}}说:「你必亲眼看见,却不得吃。」
\par }{\SH 亚兰军弃营逃命
\par }{\PP \VS{3}在城门那里有四个长大麻风的人,他们彼此说:「我们为何坐在这里等死呢?
\VS{4}我们若说,进城去吧!城里有饥荒,必死在那里;若在这里坐着不动,也必是死。来吧,我们去投降{\PN{亚兰}}人的军队,他们若留我们的活命,就活着;若杀我们,就死了吧!」
\VS{5}黄昏的时候,他们起来往{\PN{亚兰}}人的营盘去;到了营边,不见一人在那里。
\VS{6}因为主使{\PN{亚兰}}人的军队听见车马的声音,是大军的声音;他们就彼此说:「这必是{\PN{以色列}}王贿买{\PN{赫}}人的诸王和{\PN{埃及}}人的诸王来攻击我们。」
\VS{7}所以,在黄昏的时候他们起来逃跑,撇下帐棚、马、驴,营盘照旧,只顾逃命。
\VS{8}那些长大麻风的到了营边,进了帐棚,吃了喝了,且从其中拿出金银和衣服来,去收藏了;回来又进了一座帐棚,从其中拿出财物来去收藏了。
\par }{\PP \VS{9}那时,他们彼此说:「我们所做的不好!今日是有好信息的日子,我们竟不作声!若等到天亮,罪必临到我们。来吧,我们与王家报信去!」
\VS{10}他们就去叫守城门的,告诉他们说:「我们到了{\PN{亚兰}}人的营,不见一人在那里,也无人声,只有拴着的马和驴,帐棚都照旧。」
\VS{11}守城门的叫了众守门的人来,他们就进去与王家报信。
\VS{12}王夜间起来,对臣仆说:「我告诉你们{\PN{亚兰}}人向我们如何行。他们知道我们饥饿,所以离营,埋伏在田野,说:『{\PN{以色列}}人出城的时候,我们就活捉他们,得以进城。』」
\VS{13}有一个臣仆对王说:「我们不如用城里剩下之马中的五匹马(马和城里剩下的{\PN{以色列}}人都是一样,快要灭绝),打发人去窥探。」
\VS{14}于是取了两辆车和马,王差人去追寻{\PN{亚兰}}军,说:「你们去窥探窥探。」
\VS{15}他们就追寻到{\PN{约旦河}},看见满道上都是{\PN{亚兰}}人急跑时丢弃的衣服器具,使者就回来报告王。
\par }{\PP \VS{16}众人就出去,掳掠{\PN{亚兰}}人的营盘。于是一细亚细面{\ADD{卖}}银一舍客勒,二细亚大麦也{\ADD{卖}}银一舍客勒,正如耶和华所说的。
\VS{17}王派搀扶他的那军长在城门口弹压,众人在那里将他践踏,他就死了,正如神人在王下来见他的时候所说的。
\VS{18}神人曾对王说:「明日约到这时候,在{\PN{撒马利亚}}城门口,二细亚大麦要{\ADD{卖}}银一舍客勒,一细亚细面也要{\ADD{卖}}银一舍客勒。」
\VS{19}那军长对神人说:「即便耶和华使天开了窗户,也不能有这事。」神人说:「你必亲眼看见,却不得吃。」
\VS{20}这话果然应验在他身上;因为众人在城门口将他践踏,他就死了。

\par }\Chap{8}{\SH 书念的妇人回原住地
\par }{\PP \VerseOne{1}{\PN{以利沙}}曾对所救活之子的那妇人说:「你和你的全家要起身往你可住的地方去住,因为耶和华命饥荒降在这地七年。」
\VS{2}妇人就起身,照神人的话带着全家往{\PN{非利士}}地去,住了七年。
\VS{3}七年完了,那妇人从{\PN{非利士}}地回来,就出去为自己的房屋田地哀告王。
\VS{4}那时王正与神人的仆人{\PN{基哈西}}说:「请你将{\PN{以利沙}}所行的一切大事告诉我。」
\VS{5}{\PN{基哈西}}告诉王{\PN{以利沙}}如何使死人复活,恰巧{\PN{以利沙}}所救活、她儿子的那妇人为自己的房屋田地来哀告王。{\PN{基哈西}}说:「我主我王,这就是那妇人,这是她的儿子,就是{\PN{以利沙}}所救活的。」
\VS{6}王问那妇人,她就把那事告诉王。于是王为她派一个太监,说:「凡属这妇人的都还给她,自从她离开本地直到今日,她田地的出产也都还给她。」
\par }{\SH 以利沙和便·哈达王
\par }{\PP \VS{7}{\PN{以利沙}}来到{\PN{大马士革}},{\PN{亚兰}}王{\PN{便·哈达}}正患病。有人告诉王说:「神人来到这里了。」
\VS{8}王就吩咐{\PN{哈薛}}说:「你带着礼物去见神人,托他求问耶和华,我这病能好不能好?」
\VS{9}于是{\PN{哈薛}}用四十个骆驼,驮着{\PN{大马士革}}的各样美物为礼物,去见{\PN{以利沙}}。到了他那里,站在他面前,说:「你儿子{\PN{亚兰}}王{\PN{便·哈达}}打发我来见你,他问说:『我这病能好不能好?』」
\VS{10}{\PN{以利沙}}对{\PN{哈薛}}说:「你回去告诉他说,这病必能好;但耶和华指示我,他必要死。」
\VS{11}神人定睛看着{\PN{哈薛}},甚至他惭愧。神人就哭了;
\VS{12}{\PN{哈薛}}说:「我主为什么哭?」回答说:「因为我知道你必苦害{\PN{以色列}}人,用火焚烧他们的保障,用刀杀死他们的壮丁,摔死他们的婴孩,剖开他们的孕妇。」
\VS{13}{\PN{哈薛}}说:「你仆人算什么,不过是一条狗,焉能行这大事呢?」{\PN{以利沙}}回答说:「耶和华指示我,你必作{\PN{亚兰}}王。」
\VS{14}{\PN{哈薛}}离开{\PN{以利沙}},回去见他的主人。主人问他说:「{\PN{以利沙}}对你说什么?」回答说:「他告诉我你必能好。」
\VS{15}次日,{\PN{哈薛}}拿被窝浸在水中,蒙住王的脸,王就死了。于是{\PN{哈薛}}篡了他的位。
\par }{\SH 犹大王约兰
\par }{\R (代下21·1—20)
\par }{\PP \VS{16}{\PN{以色列}}王{\PN{亚哈}}的儿子{\PN{约兰}}第五年,{\PN{犹大}}王{\PN{约沙法}}还在位的时候,{\PN{约沙法}}的儿子{\PN{约兰}}登基作了{\PN{犹大}}王。
\VS{17}{\PN{约兰}}登基的时候年三十二岁,在{\PN{耶路撒冷}}作王八年。
\VS{18}他行{\PN{以色列}}诸王所行的,与{\PN{亚哈}}家一样;因为他娶了{\PN{亚哈}}的女儿为妻,行耶和华眼中看为恶的事。
\VS{19}耶和华却因他仆人{\PN{大卫}}的缘故,仍不肯灭绝{\PN{犹大}},照他所应许{\PN{大卫}}的话,永远赐灯光与他的子孙。
\par }{\PP \VS{20}{\PN{约兰}}年间,{\PN{以东}}人背叛{\PN{犹大}},脱离他的权下,自己立王。
\VS{21}{\PN{约兰}}率领所有的战车往{\PN{撒益}}去,夜间起来,攻打围困他的{\PN{以东}}人和车{\ADD{兵}}长;{\PN{犹大}}兵就逃跑,各回各家去了。
\VS{22}这样,{\PN{以东}}人背叛{\PN{犹大}},脱离他的权下,直到今日。那时{\PN{立拿}}人也背叛了。
\par }{\PP \VS{23}{\PN{约兰}}其余的事,凡他所行的,都写在{\PN{犹大}}列王记上。
\VS{24}{\PN{约兰}}与他列祖同睡,葬在{\PN{大卫城}}他列祖的坟地里。他儿子{\PN{亚哈谢}}接续他作王。
\par }{\SH 犹大王亚哈谢
\par }{\R (代下22·1—6)
\par }{\PP \VS{25}{\PN{以色列}}王{\PN{亚哈}}的儿子{\PN{约兰}}十二年,{\PN{犹大}}王{\PN{约兰}}的儿子{\PN{亚哈谢}}登基。
\VS{26}他登基的时候年二十二岁,在{\PN{耶路撒冷}}作王一年。他母亲名叫{\PN{亚她利雅}},是{\PN{以色列}}王{\PN{暗利}}的孙女。
\VS{27}{\PN{亚哈谢}}效法{\PN{亚哈}}家行耶和华眼中看为恶的事,与{\PN{亚哈}}家一样,因为他是{\PN{亚哈}}家的女婿。
\par }{\PP \VS{28}他与{\PN{亚哈}}的儿子{\PN{约兰}}同往{\PN{基列}}的{\PN{拉末}}去,与{\PN{亚兰}}王{\PN{哈薛}}争战。{\PN{亚兰}}人打伤了{\PN{约兰}},
\VS{29}{\PN{约兰}}王回到{\PN{耶斯列}},医治在{\PN{拉末}}与{\PN{亚兰}}王{\PN{哈薛}}打仗的时候所受的伤。{\PN{犹大}}王{\PN{约兰}}的儿子{\PN{亚哈谢}}因为{\PN{亚哈}}的儿子{\PN{约兰}}病了,就下到{\PN{耶斯列}}看望他。

\par }\Chap{9}{\SH 耶户被膏立为以色列王
\par }{\PP \VerseOne{1}先知{\PN{以利沙}}叫了一个先知门徒来,吩咐他说:「你束上腰,手拿这瓶膏油往{\PN{基列}}的{\PN{拉末}}去。
\VS{2}到了那里,要寻找{\PN{宁示}}的孙子、{\PN{约沙法}}的儿子{\PN{耶户}},使他从同僚中起来,带他进严密的屋子,
\VS{3}将瓶里的膏油倒在他头上,说:『耶和华如此说:我膏你作{\PN{以色列}}王。』{\ADD{说完了}},就开门逃跑,不要迟延。」
\par }{\PP \VS{4}于是那少年先知往{\PN{基列}}的{\PN{拉末}}去了。
\VS{5}到了那里,看见众军长都坐着,就说:「将军哪,我有话对你说。」{\PN{耶户}}说:「我们众人里,你要对哪一个说呢?」回答说:「将军哪,我要对你说。」
\VS{6}{\PN{耶户}}就起来,进了屋子,少年人将膏油倒在他头上,对他说:「耶和华—{\PN{以色列}}的 神如此说:『我膏你作耶和华民{\PN{以色列}}的王。
\VS{7}你要击杀你主人{\PN{亚哈}}的全家,我好在{\PN{耶洗别}}身上伸我仆人众先知和耶和华一切仆人流血的冤。
\VS{8}{\PN{亚哈}}全家必都灭亡,凡属{\PN{亚哈}}的男丁,无论是困住的、自由的,我必从{\PN{以色列}}中剪除,
\VS{9}使{\PN{亚哈}}的家像{\PN{尼八}}儿子{\PN{耶罗波安}}的家,又像{\PN{亚希雅}}儿子{\PN{巴沙}}的家。
\VS{10}{\PN{耶洗别}}必在{\PN{耶斯列}}田里被狗所吃,无人葬埋。』」{\ADD{说完了}},少年人就开门逃跑了。
\par }{\PP \VS{11}{\PN{耶户}}出来,回到他主人的臣仆那里,有一人问他说:「平安吗?这狂妄的人来见你有什么事呢?」回答说:「你们认得那人,也知道他说什么。」
\VS{12}他们说:「这是假话,你据实地告诉我们。」回答说:「他如此如此对我说。他说:『耶和华如此说:我膏你作{\PN{以色列}}王。』」
\VS{13}他们就急忙各将自己的衣服铺在上层台阶,使{\PN{耶户}}坐在其上;他们吹角,说:「{\PN{耶户}}作王了!」
\par }{\SH 以色列王约兰被杀
\par }{\PP \VS{14}这样,{\PN{宁示}}的孙子、{\PN{约沙法}}的儿子{\PN{耶户}}背叛{\PN{约兰}}。先是{\PN{约兰}}和{\PN{以色列}}众人因为{\PN{亚兰}}王{\PN{哈薛}}的缘故,把守{\PN{基列}}的{\PN{拉末}};
\VS{15}但{\PN{约兰}}王回到{\PN{耶斯列}},医治与{\PN{亚兰}}王{\PN{哈薛}}打仗所受的伤。{\PN{耶户}}说:「若合你们的意思,就不容人逃出城往{\PN{耶斯列}}报信去。」
\VS{16}于是{\PN{耶户}}坐车往{\PN{耶斯列}}去,因为{\PN{约兰}}病卧在那里。{\PN{犹大}}王{\PN{亚哈谢}}已经下去看望他。
\par }{\PP \VS{17}有一个守望的人站在{\PN{耶斯列}}的楼上,看见{\PN{耶户}}带着一群人来,就说:「我看见一群人。」{\PN{约兰}}说:「打发一个骑马的去迎接他们,问说:平安不平安?」
\VS{18}骑马的就去迎接{\PN{耶户}},说:「王问说,平安不平安?」耶户说:「平安不平安与你何干?你转在我后头吧!」守望的人又说:「使者到了他们那里,却不回来。」
\VS{19}王又打发一个骑马的去。这人到了他们那里,说:「王问说,平安不平安?」{\PN{耶户}}说:「平安不平安与你何干?你转在我后头吧!」
\VS{20}守望的人又说:「他到了他们那里,也不回来;车赶得甚猛,像{\PN{宁示}}的孙子{\PN{耶户}}的赶法。」
\par }{\PP \VS{21}{\PN{约兰}}吩咐说:「套车!」人就给他套车。{\PN{以色列}}王{\PN{约兰}}和{\PN{犹大}}王{\PN{亚哈谢}}各坐自己的车出去迎接{\PN{耶户}},在{\PN{耶斯列}}人{\PN{拿伯}}的田那里遇见他。
\VS{22}{\PN{约兰}}见{\PN{耶户}}就说:「{\PN{耶户}}啊,平安吗?」{\PN{耶户}}说:「你母亲{\PN{耶洗别}}的淫行邪术这样多,焉能平安呢?」
\VS{23}{\PN{约兰}}就转车逃跑,对{\PN{亚哈谢}}说:「{\PN{亚哈谢}}啊,反了!」
\VS{24}{\PN{耶户}}开满了弓,射中{\PN{约兰}}的脊背,箭从心窝穿出,{\PN{约兰}}就仆倒在车上。
\VS{25}{\PN{耶户}}对他的军长{\PN{毕甲}}说:「你把他抛在{\PN{耶斯列}}人{\PN{拿伯}}的田间。你当追想,你我一同坐车跟随他父{\PN{亚哈}}的时候,耶和华对{\PN{亚哈}}所说的预言,
\VS{26}说:『我昨日看见{\PN{拿伯}}的血和他众子的血,我必在这块田上报应你。』这是耶和华说的,现在你要照着耶和华的话,把他抛在这田间。」
\par }{\SH 犹大王亚哈谢被杀
\par }{\PP \VS{27}{\PN{犹大}}王{\PN{亚哈谢}}见这光景,就从园亭之路逃跑。{\PN{耶户}}追赶他,说:「把这人也杀在车上。」到了靠近{\PN{以伯莲}}{\PN{姑珥}}的坡上{\ADD{击伤了他}}。他逃到{\PN{米吉多}},就死在那里。
\VS{28}他的臣仆用车将他的尸首送到{\PN{耶路撒冷}},葬在{\PN{大卫城}}他自己的坟墓里,与他列祖同葬。
\par }{\PP \VS{29}{\PN{亚哈谢}}登基作{\PN{犹大}}王的时候,是在{\PN{亚哈}}的儿子{\PN{约兰}}第十一年。
\par }{\SH 耶洗别王后被杀
\par }{\PP \VS{30}{\PN{耶户}}到了{\PN{耶斯列}};{\PN{耶洗别}}听见就擦粉、梳头,从窗户里往外观看。
\VS{31}{\PN{耶户}}进门的时候,{\PN{耶洗别}}说:「杀主人的{\PN{心利}}啊,平安吗?」
\VS{32}{\PN{耶户}}抬头向窗户观看,说:「谁顺从我?」有两三个太监从窗户往外看他。
\VS{33}{\PN{耶户}}说:「把她扔下来!」他们就把她扔下来。她的血溅在墙上和马上;于是把她践踏了。
\VS{34}{\PN{耶户}}进去,吃了喝了,吩咐说:「你们把这被咒诅的妇人葬埋了,因为她是王的女儿。」
\VS{35}他们就去葬埋她,只寻得她的头骨和脚,并手掌。
\VS{36}他们回去告诉{\PN{耶户}},{\PN{耶户}}说:「这正应验耶和华借他仆人{\PN{提斯比}}人{\PN{以利亚}}所说的话,说:『在{\PN{耶斯列}}田间,狗必吃{\PN{耶洗别}}的肉;
\VS{37}{\PN{耶洗别}}的尸首必在{\PN{耶斯列}}田间如同粪土,甚至人不能说这是{\PN{耶洗别}}。』」

\par }\Chap{10}{\SH 耶户写信給耶斯列首领
\par }{\PP \VerseOne{1}{\PN{亚哈}}有七十个儿子在{\PN{撒马利亚}}。{\PN{耶户}}写信送到{\PN{撒马利亚}},通知{\PN{耶斯列}}的首领,就是长老和教养{\PN{亚哈}}{\ADD{众子}}的人,说:
\VS{2}「你们那里既有你们主人的众子和车马、器械、坚固城,
\VS{3}接了这信,就可以在你们主人的众子中选择一个贤能合宜的,使他坐他父亲的位,你们也可以为你们主人的家争战。」
\VS{4}他们却甚惧怕,彼此说:「二王在他面前尚且站立不住,我们怎能站得住呢?」
\VS{5}家宰、邑宰,和长老,并教养{\ADD{众子}}的人,打发人去见{\PN{耶户}},说:「我们是你的仆人,凡你所吩咐我们的都必遵行,我们不立谁作王,你看怎样好就怎样行。」
\par }{\SH 亚哈众子被杀
\par }{\PP \VS{6}{\PN{耶户}}又给他们写信说:「你们若归顺我,听从我的话,明日这时候,要将你们主人众子的首级带到{\PN{耶斯列}}来见我。」那时王的儿子七十人都住在教养他们那城中的尊贵人家里。
\VS{7}信一到,他们就把王的七十个儿子杀了,将首级装在筐里,送到在{\PN{耶斯列}}的{\PN{耶户}}那里。
\VS{8}有使者来告诉{\PN{耶户}}说:「他们将王众子的首级送来了。」{\PN{耶户}}说:「将首级在城门口堆作两堆,搁到明日。」
\VS{9}次日早晨,{\PN{耶户}}出来,站着对众民说:「你们都是公义的,我背叛我主人,将他杀了;这些人却是谁杀的呢?
\VS{10}由此可知,耶和华指着{\PN{亚哈}}家所说的话一句没有落空,因为耶和华借他仆人{\PN{以利亚}}所说的话都成就了。」
\VS{11}凡{\PN{亚哈}}家在{\PN{耶斯列}}所剩下的人和他的大臣、密友、祭司,{\PN{耶户}}尽都杀了,没有留下一个。
\par }{\SH 亚哈谢的兄弟被杀
\par }{\PP \VS{12}{\PN{耶户}}起身往{\PN{撒马利亚}}去。在路上、牧人剪羊毛之处,
\VS{13}遇见{\PN{犹大}}王{\PN{亚哈谢}}的弟兄,问他们说:「你们是谁?」回答说:「我们是{\PN{亚哈谢}}的弟兄,现在下去要问王和太后的众子安。」
\VS{14}{\PN{耶户}}吩咐说:「活捉他们!」跟从的人就活捉了他们,将他们杀在剪羊毛之处的坑边,共四十二人,没有留下一个。
\par }{\SH 亚哈家剩下的人被杀
\par }{\PP \VS{15}{\PN{耶户}}从那里前行,恰遇{\PN{利甲}}的儿子{\PN{约拿达}}来迎接他,{\PN{耶户}}问他安,对他说:「你诚心待我像我诚心待你吗?」{\PN{约拿达}}回答说:「是。」{\PN{耶户}}说:「若是这样,你向我伸手」,他就伸手;{\PN{耶户}}拉他上车。
\VS{16}{\PN{耶户}}说:「你和我同去,看我为耶和华怎样热心」;于是请他坐在车上,
\VS{17}到了{\PN{撒马利亚}},就把{\PN{撒马利亚}}的{\PN{亚哈}}家剩下的人都杀了,直到灭尽,正如耶和华对{\PN{以利亚}}所说的。
\par }{\SH 拜巴力的人被杀
\par }{\PP \VS{18}{\PN{耶户}}招聚众民,对他们说:「{\PN{亚哈}}事奉{\PN{巴力}}还冷淡,{\PN{耶户}}却更热心。
\VS{19}现在我要给{\PN{巴力}}献大祭。应当叫{\PN{巴力}}的众先知和一切拜{\PN{巴力}}的人,并{\PN{巴力}}的众祭司,都到我这里来,不可缺少一个;凡不来的必不得活。」{\PN{耶户}}这样行,是用诡计要杀尽拜{\PN{巴力}}的人。
\VS{20}{\PN{耶户}}说:「要为{\PN{巴力}}宣告严肃会!」于是宣告了。
\VS{21}{\PN{耶户}}差人走遍{\PN{以色列}}地;凡拜{\PN{巴力}}的人都来齐了,没有一个不来的。他们进了{\PN{巴力}}庙,{\PN{巴力}}庙中从前边直到后边都满了人。
\VS{22}{\PN{耶户}}吩咐掌管礼服的人说:「拿出礼服来,给一切拜{\PN{巴力}}的人穿。」他就拿出礼服来给了他们。
\VS{23}{\PN{耶户}}和{\PN{利甲}}的儿子{\PN{约拿达}}进了{\PN{巴力}}庙,对拜{\PN{巴力}}的人说:「你们察看察看,在你们这里不可有耶和华的仆人,只可容留拜{\PN{巴力}}的人。」
\VS{24}{\PN{耶户}}和{\PN{约拿达}}进去,献{\ADD{平安}}祭和燔祭。
\par }{\PP {\PN{耶户}}先安派八十人在庙外,吩咐说:「我将这些人交在你们手中,若有一人脱逃,{\ADD{谁放的}}必叫他偿命!」
\VS{25}{\PN{耶户}}献完了燔祭,就{\ADD{出来}}吩咐护卫兵和众军长说:「你们进去杀他们,不容一人出来!」护卫兵和军长就用刀杀他们,将尸首抛出去,便到{\PN{巴力}}庙的城去了,
\VS{26}将{\PN{巴力}}庙中的柱像都拿出来烧了;
\VS{27}毁坏了{\PN{巴力}}柱像,拆毁了{\PN{巴力}}庙作为厕所,直到今日。
\par }{\PP \VS{28}这样,{\PN{耶户}}在{\PN{以色列}}中灭了{\PN{巴力}}。
\VS{29}只是{\PN{耶户}}不离开{\PN{尼八}}的儿子{\PN{耶罗波安}}使{\PN{以色列}}人陷在罪里的那罪,就是拜{\PN{伯特利}}和{\PN{但}}的金牛犊。
\VS{30}耶和华对{\PN{耶户}}说:「因你办好我眼中看为正的事,照我的心意待{\PN{亚哈}}家,你的子孙必接续你坐{\PN{以色列}}的国位,直到四代。」
\VS{31}只是{\PN{耶户}}不尽心遵守耶和华—{\PN{以色列}} 神的律法,不离开{\PN{耶罗波安}}使{\PN{以色列}}人陷在罪里的那罪。
\par }{\SH 耶户去世
\par }{\PP \VS{32}在那些日子,耶和华才割裂{\PN{以色列}}国,使{\PN{哈薛}}攻击{\PN{以色列}}的境界,
\VS{33}乃是{\PN{约旦河}}东、{\PN{基列}}全地,从靠近{\PN{亚嫩谷}}边的{\PN{亚罗珥}}起,就是{\PN{基列}}和{\PN{巴珊}}的{\PN{迦得}}人、{\PN{吕便}}人、{\PN{玛拿西}}人之地。
\VS{34}{\PN{耶户}}其余的事,凡他所行的和他的勇力都写在{\PN{以色列}}诸王记上。
\VS{35}{\PN{耶户}}与他列祖同睡,葬在{\PN{撒马利亚}};他儿子{\PN{约哈斯}}接续他作王。
\VS{36}{\PN{耶户}}在{\PN{撒马利亚}}作{\PN{以色列}}王二十八年。

\par }\Chap{11}{\SH 犹大的王后亚她利雅
\par }{\R (代下22·10—23·15)
\par }{\PP \VerseOne{1}{\PN{亚哈谢}}的母亲{\PN{亚她利雅}}见她儿子死了,就起来剿灭王室。
\VS{2}但{\PN{约兰}}王的女儿,{\PN{亚哈谢}}的妹子{\PN{约示巴}},将{\PN{亚哈谢}}的儿子{\PN{约阿施}}从那被杀的王子中偷出来,把他和他的乳母都{\ADD{藏}}在卧房里,躲避{\PN{亚她利雅}},免得被杀。
\VS{3}{\PN{约阿施}}和他的乳母藏在耶和华的殿里六年;{\PN{亚她利雅}}篡了国位。
\par }{\PP \VS{4}第七年,{\PN{耶何耶大}}打发人叫{\PN{迦利}}人\FTNT{}{{\FR 11:4: }或译:亲兵}和护卫兵的众百夫长来,领他们进了耶和华的殿,与他们立约,使他们在耶和华殿里起誓,又将王的儿子指给他们看,
\VS{5}吩咐他们说:「你们当这样行:凡安息日进班的三分之一要看守王宫,
\VS{6}三分之一要在{\PN{苏珥}}门,三分之一要在护卫兵院的后门。这样把守王宫,拦阻闲人。
\VS{7}你们安息日所有出班的三分之二要在耶和华的殿里护卫王;
\VS{8}各人手拿兵器,四围护卫王。凡擅入你们班次的必当治死,王出入的时候,你们当跟随他。」
\par }{\PP \VS{9}众百夫长就照着祭司{\PN{耶何耶大}}一切所吩咐的去行,各带所管安息日进班出班的人来见祭司{\PN{耶何耶大}}。
\VS{10}祭司便将耶和华殿里所藏{\PN{大卫}}王的枪和盾牌交给百夫长。
\VS{11}护卫兵手中各拿兵器,在坛和殿那里,从殿右直到殿左,站在王子的四围。
\VS{12}祭司领王子出来,给他戴上冠冕,将律法{\ADD{书交给他}},膏他作王;众人就拍掌说:「愿王万岁!」
\par }{\PP \VS{13}{\PN{亚她利雅}}听见护卫兵和民的声音,就到民那里,进耶和华的殿,
\VS{14}看见王照例站在柱旁,百夫长和吹号的人侍立在王左右,国中的众民欢乐吹号;{\PN{亚她利雅}}就撕裂衣服,喊叫说:「反了!反了!」
\VS{15}祭司{\PN{耶何耶大}}吩咐管辖军兵的百夫长说:「将她赶出班外,凡跟随她的必用刀杀死!」因为祭司说「不可在耶和华殿里杀她」,
\VS{16}众兵就闪开让她去;她从马路上王宫去,便在那里被杀。
\par }{\SH 耶何耶大的改革
\par }{\R (代下23·16—21)
\par }{\PP \VS{17}{\PN{耶何耶大}}使王和民与耶和华立约,作耶和华的民;又使王与民立约。
\VS{18}于是国民都到{\PN{巴力}}庙,拆毁了庙,打碎坛和像,又在坛前将{\PN{巴力}}的祭司{\PN{玛坦}}杀了。祭司{\PN{耶何耶大}}派官看守耶和华的殿,
\VS{19}又率领百夫长和{\PN{迦利}}人\FTNT{}{{\FR 11:19: }或译:亲兵}与护卫兵,以及国中的众民,请王从耶和华殿下来,由护卫兵的门进入王宫,他就坐了王位。
\VS{20}国民都欢乐,合城都安静。众人已将{\PN{亚她利雅}}在王宫那里用刀杀了。
\par }{\PP \VS{21}{\PN{约阿施}}登基的时候年方七岁。

\par }\Chap{12}{\SH 犹大王约阿施
\par }{\R (代下24·1—16)
\par }{\PP \VerseOne{1}{\PN{耶户}}第七年,{\PN{约阿施}}登基,在{\PN{耶路撒冷}}作王四十年。他母亲名叫{\PN{西比亚}},是{\PN{别是巴}}人。
\VS{2}{\PN{约阿施}}在祭司{\PN{耶何耶大}}教训他的时候,就行耶和华眼中看为正的事;
\VS{3}只是邱坛还没有废去,百姓仍在那里献祭烧香。
\par }{\PP \VS{4}{\PN{约阿施}}对众祭司说:「凡奉到耶和华殿分别为圣之物所值通用的银子,或各人当纳的身价,或乐意奉到耶和华殿的银子,
\VS{5}你们当从所认识的人收了来,修理殿的一切破坏之处。」
\VS{6}无奈到了{\PN{约阿施}}王二十三年,祭司仍未修理殿的破坏之处。
\VS{7}所以{\PN{约阿施}}王召了大祭司{\PN{耶何耶大}}和众祭司来,对他们说:「你们怎么不修理殿的破坏之处呢?从今以后,你们不要从所认识的人再收银子,要将所收的交出来,修理殿的破坏之处。」
\VS{8}众祭司答应不再收百姓的银子,也不修理殿的破坏之处。
\par }{\PP \VS{9}祭司{\PN{耶何耶大}}取了一个柜子,在柜盖上钻了一个窟窿,放于坛旁,在进耶和华殿的右边;守门的祭司将奉到耶和华殿的一切银子投在柜里。
\VS{10}他们见柜里的银子多了,便叫王的书记和大祭司上来,将耶和华殿里的银子数算包起来。
\VS{11}把所平的银子交给督工的,就是耶和华殿里办事的人;他们把银子转交修理耶和华殿的木匠和工人,
\VS{12}并瓦匠、石匠,又买木料和凿成的石头,修理耶和华殿的破坏之处,以及修理殿的各样使用。
\VS{13}但那奉到耶和华殿的银子,没有用以做耶和华殿里的银杯、蜡剪、碗、号,和别样的金银器皿,
\VS{14}乃将那银子交给督工的人修理耶和华的殿;
\VS{15}且将银子交给办事的人转交做工的人,不与他们算账,因为他们办事诚实。
\VS{16}惟有赎愆祭、赎罪祭的银子没有奉到耶和华的殿,都归祭司。
\par }{\PP \VS{17}那时,{\PN{亚兰}}王{\PN{哈薛}}上来攻打{\PN{迦}}
{\PN{特}},攻取了,就定意上来攻打{\PN{耶路撒冷}}。
\VS{18}{\PN{犹大}}王{\PN{约阿施}}将他列祖{\PN{犹大}}王{\PN{约沙法}}、{\PN{约兰}}、{\PN{亚哈谢}}所分别为圣的物和自己所分别为圣的物,并耶和华殿与王宫府库里所有的金子都送给{\PN{亚兰}}王{\PN{哈薛}};{\PN{哈薛}}就不上{\PN{耶路撒冷}}来了。
\par }{\PP \VS{19}{\PN{约阿施}}其余的事,凡他所行的都写在{\PN{犹大}}列王记上。
\par }{\PP \VS{20}{\PN{约阿施}}的臣仆起来背叛,在下{\PN{悉拉}}的{\PN{米罗}}宫那里将他杀了。
\VS{21}杀他的那臣仆就是{\PN{示米押}}的儿子{\PN{约撒甲}}和{\PN{朔默}}的儿子{\PN{约萨拔}}。众人将他葬在{\PN{大卫城}}他列祖的坟地里。他儿子{\PN{亚玛谢}}接续他作王。

\par }\Chap{13}{\SH 以色列王约哈斯
\par }{\PP \VerseOne{1}{\PN{犹大}}王{\PN{亚哈谢}}的儿子{\PN{约阿施}}二十三年,{\PN{耶户}}的儿子{\PN{约哈斯}}在{\PN{撒马利亚}}登基作{\PN{以色列}}王十七年。
\VS{2}{\PN{约哈斯}}行耶和华眼中看为恶的事,效法{\PN{尼八}}的儿子{\PN{耶罗波安}}使{\PN{以色列}}人陷在罪里的那罪,总不离开。
\VS{3}于是,耶和华的怒气向{\PN{以色列}}人发作,将他们屡次交在{\PN{亚兰}}王{\PN{哈薛}}和他儿子{\PN{便哈达}}的手里。
\VS{4}{\PN{约哈斯}}恳求耶和华,耶和华就应允他,因为见{\PN{以色列}}人所受{\PN{亚兰}}王的欺压。
\VS{5}耶和华赐给{\PN{以色列}}人一位拯救者,使他们脱离{\PN{亚兰}}人的手;于是{\PN{以色列}}人仍旧安居在家里。
\VS{6}然而,他们不离开{\PN{耶罗波安}}家使{\PN{以色列}}人陷在罪里的那罪,仍然去行,并且在{\PN{撒马利亚}}留下{\PN{亚舍拉}}。
\VS{7}{\PN{亚兰}}王灭绝{\PN{约哈斯}}的民,践踏他们如禾场上的尘沙,只给{\PN{约哈斯}}留下五十马兵,十辆战车,一万步兵。
\VS{8}{\PN{约哈斯}}其余的事,凡他所行的和他的勇力都写在{\PN{以色列}}诸王记上。
\VS{9}{\PN{约哈斯}}与他列祖同睡,葬在{\PN{撒马利亚}}。他儿子{\PN{约阿施}}接续他作王。
\par }{\SH 以色列王约阿施
\par }{\PP \VS{10}{\PN{犹大}}王{\PN{约阿施}}三十七年,{\PN{约哈斯}}的儿子{\PN{约阿施}}在{\PN{撒马利亚}}登基作{\PN{以色列}}王十六年。
\VS{11}他行耶和华眼中看为恶的事,不离开{\PN{尼八}}的儿子{\PN{耶罗波安}}使{\PN{以色列}}人陷在罪里的一切罪,仍然去行。
\VS{12}{\PN{约阿施}}其余的事,凡他所行的和他与{\PN{犹大}}王{\PN{亚玛谢}}争战的勇力,都写在{\PN{以色列}}诸王记上。
\VS{13}{\PN{约阿施}}与他列祖同睡,{\PN{耶罗波安}}坐了他的位。{\PN{约阿施}}与{\PN{以色列}}诸王一同葬在{\PN{撒马利亚}}。
\par }{\SH 以利沙去世
\par }{\PP \VS{14}{\PN{以利沙}}得了必死的病,{\PN{以色列}}王{\PN{约阿施}}下来看他,伏在他脸上哭泣,说:「我父啊!我父啊!{\PN{以色列}}的战车马兵啊!」
\VS{15}{\PN{以利沙}}对他说:「你取弓箭来。」王就取了弓箭来;
\VS{16}又对{\PN{以色列}}王说:「你用手拿弓。」王就用手拿{\ADD{弓}}。{\PN{以利沙}}按手在王的手上,
\VS{17}说:「你开朝东的窗户。」他就开了。{\PN{以利沙}}说:「射箭吧!」他就射箭。{\PN{以利沙}}说:「这是耶和华的得胜箭,就是战胜{\PN{亚兰}}人的箭;因为你必在{\PN{亚弗}}攻打{\PN{亚兰}}人,直到灭尽他们。」
\VS{18}{\PN{以利沙}}又说:「取几枝箭来。」他就取了来。{\PN{以利沙}}说:「打地吧!」他打了三次,便止住了。
\VS{19}神人向他发怒,说:「应当击打五六次,就能攻打{\PN{亚兰}}人直到灭尽;现在只能打败{\PN{亚兰}}人三次。」
\par }{\PP \VS{20}{\PN{以利沙}}死了,人将他葬埋。到了新年,有一群{\PN{摩押}}人犯境,
\VS{21}有人正葬死人,忽然看见一群人,就把死人抛在{\PN{以利沙}}的坟墓里,一碰着{\PN{以利沙}}的骸骨,死人就复活,站起来了。
\par }{\SH 以色列与亚兰争战
\par }{\PP \VS{22}{\PN{约哈斯}}年间,{\PN{亚兰}}王{\PN{哈薛}}屡次欺压{\PN{以色列}}人。
\VS{23}耶和华却因与{\PN{亚伯拉罕}}、{\PN{以撒}}、{\PN{雅各}}所立的约,仍施恩给{\PN{以色列}}人,怜恤他们,眷顾他们,不肯灭尽他们,尚未赶逐他们离开自己面前。
\par }{\PP \VS{24}{\PN{亚兰}}王{\PN{哈薛}}死了,他儿子{\PN{便·哈达}}接续他作王。
\VS{25}从前{\PN{哈薛}}和{\PN{约阿施}}的父亲{\PN{约哈斯}}争战,攻取了些城邑,现在{\PN{约哈斯}}的儿子{\PN{约阿施}}三次打败{\PN{哈薛}}的儿子{\PN{便哈达}},就收回了{\PN{以色列}}的城邑。

\par }\Chap{14}{\SH 犹大王亚玛谢
\par }{\R (代下25·1—24)
\par }{\PP \VerseOne{1}{\PN{以色列}}王{\PN{约哈斯}}的儿子{\PN{约阿施}}第二年,{\PN{犹大}}王{\PN{约阿施}}的儿子{\PN{亚玛谢}}登基。
\VS{2}他登基的时候年二十五岁,在{\PN{耶路撒冷}}作王二十九年。他母亲名叫{\PN{约耶但}},是{\PN{耶路撒冷}}人。
\VS{3}{\PN{亚玛谢}}行耶和华眼中看为正的事,但不如他祖{\PN{大卫}},乃效法他父{\PN{约阿施}}一切所行的;
\VS{4}只是邱坛还没有废去,百姓仍在那里献祭烧香。
\VS{5}国一坚定,就把杀他父王的臣仆杀了,
\VS{6}却没有治死杀王之人的儿子,是照{\PN{摩西}}律法书上耶和华所吩咐的说:「不可因子杀父,也不可因父杀子,各人要为本身的罪而死。」
\par }{\PP \VS{7}{\PN{亚玛谢}}在{\PN{盐谷}}杀了{\PN{以东}}人一万,又攻取了{\PN{西拉}},改名叫{\PN{约帖}},直到今日。
\par }{\PP \VS{8}那时,{\PN{亚玛谢}}差遣使者去见{\PN{耶户}}的孙子{\PN{约哈斯}}的儿子{\PN{以色列}}王{\PN{约阿}}
{\PN{施}},说:「你来,我们二人相见{\ADD{于战场}}。」
\VS{9}{\PN{以色列}}王{\PN{约阿施}}差遣使者去见{\PN{犹大}}王{\PN{亚玛谢}},说:「{\PN{黎巴嫩}}的蒺藜差遣使者去见{\PN{黎巴嫩}}的香柏树,说:将你的女儿给我儿子为妻。后来{\PN{黎巴嫩}}有一只野兽经过,把蒺藜践踏了。
\VS{10}你打败了{\PN{以东}}人就心高气傲,你以此为荣耀,在家里安居就罢了,为何要惹祸,使自己和{\PN{犹大}}国一同败亡呢?」
\par }{\PP \VS{11}{\PN{亚玛谢}}却不肯听这话。于是{\PN{以色列}}王{\PN{约阿施}}上来,在{\PN{犹大}}的{\PN{伯·示麦}}与{\PN{犹大}}王{\PN{亚玛谢}}相见{\ADD{于战场}}。
\VS{12}{\PN{犹大}}人败在{\PN{以色列}}人面前,各自逃回家里去了。
\VS{13}{\PN{以色列}}王{\PN{约阿施}}在{\PN{伯·示麦}}擒住{\PN{亚哈谢}}的孙子、{\PN{约阿施}}的儿子{\PN{犹大}}王{\PN{亚玛谢}},就来到{\PN{耶路撒冷}},拆毁{\PN{耶路撒冷}}的城墙,从{\PN{以法莲}}门直到{\PN{角门}}共四百肘,
\VS{14}又将耶和华殿里与王宫府库里所有的金银和器皿都拿了去,并带人去为质,就回{\PN{撒马利亚}}去了。
\par }{\PP \VS{15}{\PN{约阿施}}其余所行的事和他的勇力,并与{\PN{犹大}}王{\PN{亚玛谢}}争战的事,都写在{\PN{以色列}}诸王记上。
\VS{16}{\PN{约阿施}}与他列祖同睡,葬在{\PN{撒马利亚}},{\PN{以色列}}诸王的坟地里。他儿子{\PN{耶罗波安}}接续他作王。
\par }{\SH 犹大王亚玛谢去世
\par }{\R (代下25·25—28)
\par }{\PP \VS{17}{\PN{以色列}}王{\PN{约哈斯}}的儿子{\PN{约阿施}}死后,{\PN{犹大}}王{\PN{约阿施}}的儿子{\PN{亚玛谢}}又活了十五年。
\VS{18}{\PN{亚玛谢}}其余的事都写在{\PN{犹大}}列王记上。
\VS{19}{\PN{耶路撒冷}}有人背叛{\PN{亚玛谢}},他就逃到{\PN{拉吉}};叛党却打发人到{\PN{拉吉}}将他杀了。
\VS{20}人就用马将他的尸首驮到{\PN{耶路撒冷}},葬在{\PN{大卫城}}他列祖的坟地里。
\VS{21}{\PN{犹大}}众民立{\PN{亚玛谢}}的儿子{\PN{亚撒利雅}}\FTNT{}{{\FR 14:21: }又名乌西雅}接续他父作王,那时他年十六岁。
\VS{22}{\PN{亚玛谢}}与他列祖同睡之后,{\PN{亚撒利雅}}收回{\PN{以拉他}}仍归{\PN{犹大}},又重新修理。
\par }{\SH 以色列王耶罗波安二世
\par }{\PP \VS{23}{\PN{犹大}}王{\PN{约阿施}}的儿子{\PN{亚玛谢}}十五年,{\PN{以色列}}王{\PN{约阿施}}的儿子{\PN{耶罗波安}}在{\PN{撒马利亚}}登基,作王四十一年。
\VS{24}他行耶和华眼中看为恶的事,不离开{\PN{尼八}}的儿子{\PN{耶罗波安}}使{\PN{以色列}}人陷在罪里的一切罪。
\VS{25}他收回{\PN{以色列}}边界之地,从{\PN{哈马口}}直到{\PN{亚拉巴海}},正如耶和华—{\PN{以色列}}的 神借他仆人{\PN{迦特希弗}}人{\PN{亚米太}}的儿子先知{\PN{约拿}}所说的。
\VS{26}因为耶和华看见{\PN{以色列}}人甚是艰苦,无论困住的、自由的都没有了,也无人帮助{\PN{以色列}}人。
\VS{27}耶和华并没有说要将{\PN{以色列}}的名从天下涂抹,乃借{\PN{约阿施}}的儿子{\PN{耶罗波安}}拯救他们。
\par }{\PP \VS{28}{\PN{耶罗波安}}其余的事,凡他所行的和他的勇力,他怎样争战,怎样收回{\PN{大马士革}}和{\ADD{先前属}}{\PN{犹大}}的{\PN{哈马}}归{\PN{以色列}},都写在{\PN{以色列}}诸王记上。
\VS{29}{\PN{耶罗波安}}与他列祖{\PN{以色列}}诸王同睡。他儿子{\PN{撒迦利雅}}接续他作王。

\par }\Chap{15}{\SH 犹大王亚撒利雅
\par }{\R (代下26·1—23)
\par }{\PP \VerseOne{1}{\PN{以色列}}王{\PN{耶罗波安}}二十七年,{\PN{犹大}}王{\PN{亚玛谢}}的儿子{\PN{亚撒利雅}}登基,
\VS{2}他登基的时候年十六岁,在{\PN{耶路撒冷}}作王五十二年。他母亲名叫{\PN{耶可利雅}},是{\PN{耶路撒冷}}人。
\VS{3}{\PN{亚撒利雅}}行耶和华眼中看为正的事,效法他父亲{\PN{亚玛谢}}一切所行的;
\VS{4}只是邱坛还没有废去,百姓仍在那里献祭烧香。
\VS{5}耶和华降灾与王,使他长大麻风,直到死日,他就住在别的宫里。他的儿子{\PN{约坦}}管理家事,治理国民。
\VS{6}{\PN{亚撒利雅}}其余的事,凡他所行的都写在{\PN{犹大}}列王记上。
\VS{7}{\PN{亚撒利雅}}与他列祖同睡,葬在{\PN{大卫城}}他列祖的坟地里。他儿子{\PN{约坦}}接续他作王。
\par }{\SH 以色列王撒迦利雅
\par }{\PP \VS{8}{\PN{犹大}}王{\PN{亚撒利雅}}三十八年,{\PN{耶罗波安}}的儿子{\PN{撒迦利雅}}在{\PN{撒马利亚}}作{\PN{以色列}}王六个月。
\VS{9}他行耶和华眼中看为恶的事,效法他列祖所行的,不离开{\PN{尼八}}的儿子{\PN{耶罗波安}}使{\PN{以色列}}人陷在罪里的那罪。
\VS{10}{\PN{雅比}}的儿子{\PN{沙龙}}背叛他,在百姓面前击杀他,篡了他的位。
\par }{\PP \VS{11}{\PN{撒迦利雅}}其余的事都写在{\PN{以色列}}诸王记上。
\VS{12}这是从前耶和华应许{\PN{耶户}}说:「你的子孙必坐{\PN{以色列}}的国位直到四代。」这话果然应验了。
\par }{\SH 以色列王沙龙
\par }{\PP \VS{13}{\PN{犹大}}王{\PN{乌西雅}}\FTNT{}{{\FR 15:13: }就是亚撒利雅}三十九年,{\PN{雅比}}的儿子{\PN{沙龙}}登基在{\PN{撒马利亚}}作王一个月。
\VS{14}{\PN{迦底}}的儿子{\PN{米拿现}}从{\PN{得撒}}上{\PN{撒马利亚}},杀了{\PN{雅比}}的儿子{\PN{沙龙}},篡了他的位。
\VS{15}{\PN{沙龙}}其余的事和他背叛的情形都写在{\PN{以色列}}诸王记上。
\VS{16}那时{\PN{米拿现}}从{\PN{得撒}}起攻打{\PN{提斐萨}}和其四境,击杀城中一切的人,剖开其中所有的孕妇,都因他们没有给他开城。
\par }{\SH 以色列王米拿现
\par }{\PP \VS{17}{\PN{犹大}}王{\PN{亚撒利雅}}三十九年,{\PN{迦底}}的儿子{\PN{米拿现}}登基,在{\PN{撒马利亚}}作{\PN{以色列}}王十年。
\VS{18}他行耶和华眼中看为恶的事,终身不离开{\PN{尼八}}的儿子{\PN{耶罗波安}}使{\PN{以色列}}人陷在罪里的那罪。
\VS{19}{\PN{亚述}}王{\PN{普勒}}来攻击{\PN{以色列}}国,{\PN{米拿现}}给他一千他连得银子,请{\PN{普勒}}帮助他坚定国位。
\VS{20}{\PN{米拿现}}向{\PN{以色列}}一切大富户索要银子,使他们各出五十舍客勒,就给了{\PN{亚述}}王。于是{\PN{亚述}}王回去,不在国中停留。
\VS{21}{\PN{米拿现}}其余的事,凡他所行的都写在{\PN{以色列}}诸王记上。
\VS{22}{\PN{米拿现}}与他列祖同睡。他儿子{\PN{比加辖}}接续他作王。
\par }{\SH 以色列王比加辖
\par }{\PP \VS{23}{\PN{犹大}}王{\PN{亚撒利雅}}五十年,{\PN{米拿现}}的儿子{\PN{比加辖}}在{\PN{撒马利亚}}登基作{\PN{以色列}}王二年。
\VS{24}他行耶和华眼中看为恶的事,不离开{\PN{尼八}}的儿子{\PN{耶罗波安}}使{\PN{以色列}}人陷在罪里的那罪。
\VS{25}{\PN{比加辖}}的将军、{\PN{利玛利}}的儿子{\PN{比加}}背叛他,在{\PN{撒马利亚}}王宫里的卫所杀了他。{\PN{亚珥歌伯}}和{\PN{亚利耶}}并{\PN{基列}}的五十人帮助{\PN{比加}};{\PN{比加}}击杀他,篡了他的位。
\VS{26}{\PN{比加辖}}其余的事,凡他所行的都写在{\PN{以色列}}诸王记上。
\par }{\SH 以色列王比加
\par }{\PP \VS{27}{\PN{犹大}}王{\PN{亚撒利雅}}五十二年,{\PN{利玛利}}的儿子{\PN{比加}}在{\PN{撒马利亚}}登基作{\PN{以色列}}王二十年。
\VS{28}他行耶和华眼中看为恶的事,不离开{\PN{尼八}}的儿子{\PN{耶罗波安}}使{\PN{以色列}}人陷在罪里的那罪。
\par }{\PP \VS{29}{\PN{以色列}}王{\PN{比加}}年间,{\PN{亚述}}王{\PN{提革拉·毗列色}}来夺了{\PN{以云}}、{\PN{亚伯·伯·玛迦}}、{\PN{亚挪}}、{\PN{基低斯}}、{\PN{夏琐}}、{\PN{基列}}、{\PN{加利利}},和{\PN{拿弗他利}}全地,将这些地方的居民都掳到{\PN{亚述}}去了。
\VS{30}{\PN{乌西雅}}的儿子{\PN{约坦}}二十年,{\PN{以拉}}的儿子{\PN{何细亚}}背叛{\PN{利玛利}}的儿子{\PN{比加}},击杀他,篡了他的位。
\VS{31}{\PN{比加}}其余的事,凡他所行的都写在{\PN{以色列}}诸王记上。
\par }{\SH 犹大王约坦
\par }{\R (代下27·1—9)
\par }{\PP \VS{32}{\PN{以色列}}王{\PN{利玛利}}的儿子{\PN{比加}}第二年,{\PN{犹大}}王{\PN{乌西雅}}的儿子{\PN{约坦}}登基。
\VS{33}他登基的时候年二十五岁,在{\PN{耶路撒冷}}作王十六年。他母亲名叫{\PN{耶路沙}},是{\PN{撒督}}的女儿。
\VS{34}{\PN{约坦}}行耶和华眼中看为正的事,效法他父亲{\PN{乌西雅}}一切所行的;
\VS{35}只是邱坛还没有废去,百姓仍在那里献祭烧香。{\PN{约坦}}建立耶和华殿的上门。
\VS{36}{\PN{约坦}}其余的事,凡他所行的都写在{\PN{犹大}}列王记上。
\VS{37}在那些日子,耶和华才使{\PN{亚兰}}王{\PN{利汛}}和{\PN{利玛利}}的儿子{\PN{比加}}去攻击{\PN{犹大}}。
\VS{38}{\PN{约坦}}与他列祖同睡,葬在他祖{\PN{大卫城}}他列祖的坟地里。他儿子{\PN{亚哈斯}}接续他作王。

\par }\Chap{16}{\SH 犹大王亚哈斯
\par }{\R (代下28·1—27)
\par }{\PP \VerseOne{1}{\PN{利玛利}}的儿子{\PN{比加}}十七年,{\PN{犹大}}王{\PN{约坦}}的儿子{\PN{亚哈斯}}登基。
\VS{2}他登基的时候年二十岁,在{\PN{耶路撒冷}}作王十六年;不像他祖{\PN{大卫}}行耶和华—他 神眼中看为正的事,
\VS{3}却效法{\PN{以色列}}诸王所行的,又照着耶和华从{\PN{以色列}}人面前赶出的外邦人所行可憎的事,使他的儿子经火,
\VS{4}并在邱坛上、山冈上、各青翠树下献祭烧香。
\par }{\PP \VS{5}{\PN{亚兰}}王{\PN{利汛}}和{\PN{以色列}}王{\PN{利玛利}}的儿子{\PN{比加}}上来攻打{\PN{耶路撒冷}},围困{\PN{亚哈斯}},却不能胜他。
\VS{6}当时{\PN{亚兰}}王{\PN{利汛}}收回{\PN{以拉他}}归与{\PN{亚兰}},将{\PN{犹大}}人从{\PN{以拉他}}赶出去。{\PN{亚兰}}人\FTNT{}{{\FR 16:6: }有译以东人的}就来到{\PN{以拉他}},住在那里,直到今日。
\VS{7}{\PN{亚哈斯}}差遣使者去见{\PN{亚述}}王{\PN{提革拉·毗列色}},说:「我是你的仆人、你的儿子。现在{\PN{亚兰}}王和{\PN{以色列}}王攻击我,求你来救我脱离他们的手。」
\VS{8}{\PN{亚哈斯}}将耶和华殿里和王宫府库里所有的金银都送给{\PN{亚述}}王为礼物。
\VS{9}{\PN{亚述}}王应允了他,就上去攻打{\PN{大马士革}},将城攻取,杀了{\PN{利汛}},把{\ADD{居民}}掳到{\PN{吉珥}}。
\par }{\PP \VS{10}{\PN{亚哈斯}}王上{\PN{大马士革}}去迎接{\PN{亚述}}王{\PN{提革拉·毗列色}},在{\PN{大马士革}}看见一座坛,就照坛的规模样式作法{\ADD{画了图样}},送到祭司{\PN{乌利亚}}那里。
\VS{11}祭司{\PN{乌利亚}}照着{\PN{亚哈斯}}王从{\PN{大马士革}}送来的图样,在{\PN{亚哈斯}}王没有从{\PN{大马士革}}回来之先,建筑一座坛。
\VS{12}王从{\PN{大马士革}}回来看见坛,就近前来,在坛上献祭;
\VS{13}烧燔祭、素祭、浇奠祭,将平安祭牲的血洒在坛上,
\VS{14}又将耶和华面前的铜坛从耶和华殿和{\ADD{新}}坛的中间搬到{\ADD{新}}坛的北边。
\VS{15}{\PN{亚哈斯}}王吩咐祭司{\PN{乌利亚}}说:「早晨的燔祭、晚上的素祭,王的燔祭、素祭,国内众民的燔祭、素祭、奠祭都要烧在大坛上。燔祭牲和{\ADD{平安}}祭牲的血也要洒在这坛上,只是铜坛我要用以求问{\ADD{耶和华}}。」
\VS{16}祭司{\PN{乌利亚}}就照着{\PN{亚哈斯}}王所吩咐的行了。
\par }{\PP \VS{17}{\PN{亚哈斯}}王打掉盆座{\ADD{四面镶着}}的心子,把盆从座上挪下来,又将铜海从驮海的铜牛上搬下来,放在铺石地;
\VS{18}又因{\PN{亚述}}王的缘故,将耶和华殿为安息日所盖的廊子和王从外入殿的廊子挪移,围绕耶和华的殿。
\VS{19}{\PN{亚哈斯}}其余所行的事都写在{\PN{犹大}}列王记上。
\VS{20}{\PN{亚哈斯}}与他列祖同睡,葬在{\PN{大卫城}}他列祖的坟地里。他儿子{\PN{希西家}}接续他作王。

\par }\Chap{17}{\SH 以色列王何细亚
\par }{\PP \VerseOne{1}{\PN{犹大}}王{\PN{亚哈斯}}十二年,{\PN{以拉}}的儿子{\PN{何细亚}}在{\PN{撒马利亚}}登基作{\PN{以色列}}王九年。
\VS{2}他行耶和华眼中看为恶的事,只是不像在他以前的{\PN{以色列}}诸王。
\VS{3}{\PN{亚述}}王{\PN{撒缦以色}}上来攻击{\PN{何细亚}},{\PN{何细亚}}就服事他,给他进贡。
\VS{4}{\PN{何细亚}}背叛,差人去见{\PN{埃及}}王{\PN{梭}},不照往年所行的与{\PN{亚述}}王进贡。{\PN{亚述}}王知道了,就把他锁禁,囚在监里。
\VS{5}{\PN{亚述}}王上来攻击{\PN{以色列}}遍地,上到{\PN{撒马利亚}},围困三年。
\VS{6}{\PN{何细亚}}第九年{\PN{亚述}}王攻取了{\PN{撒马利亚}},将{\PN{以色列}}人掳到{\PN{亚述}},把他们安置在{\PN{哈腊}}与{\PN{歌散}}的{\PN{哈博河}}边,并{\PN{米底亚}}人的城邑。
\par }{\PP \VS{7}这是因{\PN{以色列}}人得罪那领他们出{\PN{埃及}}地、脱离{\PN{埃及}}王法老手的耶和华—他们的 神,去敬畏别神,
\VS{8}随从耶和华在他们面前所赶出外邦人的风俗和{\PN{以色列}}诸王所立的条规。
\VS{9}{\PN{以色列}}人暗中行不正的事,违背耶和华—他们的 神,在他们所有的城邑,从了望楼直到坚固城,建筑邱坛;
\VS{10}在各高冈上、各青翠树下立柱像和木偶;
\VS{11}在邱坛上烧香,效法耶和华在他们面前赶出的外邦人所行的,又行恶事惹动耶和华的怒气;
\VS{12}且事奉偶像,就是耶和华警戒他们不可行的。
\VS{13}但耶和华借众先知、先见劝戒{\PN{以色列}}人和{\PN{犹大}}人说:「当离开你们的恶行,谨守我的诫命律例,遵行我吩咐你们列祖,并借我仆人众先知所传给你们的律法。」
\VS{14}他们却不听从,竟硬着颈项,效法他们列祖,不信服耶和华—他们的 神,
\VS{15}厌弃他的律例和他与他们列祖所立的约,并劝戒他们的话,随从虚无{\ADD{的神}},自己成为虚妄,效法周围的外邦人,就是耶和华嘱咐他们不可效法的;
\VS{16}离弃耶和华—他们 神的一切诫命,为自己铸了两个牛犊的像,立了{\PN{亚舍拉}},敬拜天上的万象,事奉{\PN{巴力}},
\VS{17}又使他们的儿女经火,用占卜,行法术卖了自己,行耶和华眼中看为恶的事,惹动他的怒气。
\VS{18}所以耶和华向{\PN{以色列}}人大大发怒,从自己面前赶出他们,只剩下{\PN{犹大}}一个支派。
\par }{\PP \VS{19}{\PN{犹大}}人也不遵守耶和华—他们 神的诫命,随从{\PN{以色列}}人所立的条规。
\VS{20}耶和华就厌弃{\PN{以色列}}全族,使他们受苦,把他们交在抢夺他们的人手中,以致赶出他们离开自己面前,
\VS{21}将{\PN{以色列}}国从{\PN{大卫}}家夺回;他们就立{\PN{尼八}}的儿子{\PN{耶罗波安}}作王。{\PN{耶罗波安}}引诱{\PN{以色列}}人不随从耶和华,陷在大罪里。
\VS{22}{\PN{以色列}}人犯{\PN{耶罗波安}}所犯的一切罪,总不离开,
\VS{23}以致耶和华从自己面前赶出他们,正如借他仆人众先知所说的。这样,{\PN{以色列}}人从本地被掳到{\PN{亚述}},直到今日。
\par }{\SH 亚述人定居于以色列人中
\par }{\PP \VS{24}{\PN{亚述}}王从{\PN{巴比伦}}、{\PN{古他}}、{\PN{亚瓦}}、{\PN{哈马}},和{\PN{西法瓦音}}迁移人来,安置在{\PN{撒马利亚}}的城邑,代替{\PN{以色列}}人;他们就得了{\PN{撒马利亚}},住在其中。
\VS{25}他们才住那里的时候,不敬畏耶和华,所以耶和华叫狮子进入他们中间,咬死了些人。
\VS{26}有人告诉{\PN{亚述}}王说:「你所迁移安置在{\PN{撒马利亚}}各城的那些民,不知道那地之神的规矩,所以那神叫狮子进入他们中间,咬死他们。」
\VS{27}{\PN{亚述}}王就吩咐说:「叫所掳来的祭司回去一个,使他住在那里,将那地之神的规矩指教那些民。」
\VS{28}于是有一个从{\PN{撒马利亚}}掳去的祭司回来,住在{\PN{伯特利}},指教他们怎样敬畏耶和华。
\par }{\PP \VS{29}然而,各族之人在所住的城里各为自己制造神像,安置在{\PN{撒马利亚}}人所造有邱坛的殿中。
\VS{30}{\PN{巴比伦}}人造{\PN{疏割·比讷}}像;{\PN{古他}}人造{\PN{匿甲}}像;{\PN{哈马}}人造{\PN{亚示玛}}像;
\VS{31}{\PN{亚瓦}}人造{\PN{匿哈}}和{\PN{他珥他}}像;{\PN{西法瓦音}}人用火焚烧儿女,献给{\PN{西法瓦音}}的神{\PN{亚得米勒}}和{\PN{亚拿米勒}}。
\VS{32}他们惧怕耶和华,也从他们中间立邱坛的祭司,为他们在有邱坛的殿中献祭。
\VS{33}他们又惧怕耶和华,又事奉自己的神,从何邦迁移,就随何邦的风俗。
\par }{\PP \VS{34}他们直到如今仍照先前的风俗去行,不{\ADD{专心}}敬畏耶和华,不{\ADD{全}}守自己的规矩、典章,也不遵守耶和华吩咐{\PN{雅各}}后裔的律法、诫命。({\PN{雅各}},就是从前耶和华起名叫{\PN{以色列}}的。)
\VS{35}耶和华曾与他们立约,嘱咐他们说:「不可敬畏别神,不可跪拜事奉他,也不可向他献祭。
\VS{36}但那用大能和伸出来的膀臂领你们出{\PN{埃及}}地的耶和华,你们当敬畏,跪拜,向他献祭。
\VS{37}他给你们写的律例、典章、律法、诫命,你们应当永远谨守遵行,不可敬畏别神。
\VS{38}我—{\ADD{耶和华}}与你们所立的约你们不可忘记,也不可敬畏别神。
\VS{39}但要敬畏耶和华—你们的 神,他必救你们脱离一切仇敌的手。」
\VS{40}他们却不听从,仍照先前的风俗去行。
\VS{41}如此这些民又惧怕耶和华,又事奉他们的偶像。他们子子孙孙也都照样行,效法他们的祖宗,直到今日。

\par }\Chap{18}{\SH 犹大王希西家
\par }{\R (代下29·1—2;30·1)
\par }{\PP \VerseOne{1}{\PN{以色列}}王{\PN{以拉}}的儿子{\PN{何细亚}}第三年,{\PN{犹大}}王{\PN{亚哈斯}}的儿子{\PN{希西家}}登基。
\VS{2}他登基的时候年二十五岁,在{\PN{耶路撒冷}}作王二十九年。他母亲名叫{\PN{亚比}},是{\PN{撒迦利雅}}的女儿。
\VS{3}{\PN{希西家}}行耶和华眼中看为正的事,效法他祖{\PN{大卫}}一切所行的。
\VS{4}他废去邱坛,毁坏柱像,砍下木偶,打碎{\PN{摩西}}所造的铜蛇,因为到那时{\PN{以色列}}人仍向铜蛇烧香。{\PN{希西家}}叫铜蛇为铜块\FTNT{}{{\FR 18:4: }或译:人称铜蛇为铜像}。
\VS{5}{\PN{希西家}}倚靠耶和华—{\PN{以色列}}的 神,在他前后的{\PN{犹大}}列王中没有一个及他的。
\VS{6}因为他专靠耶和华,总不离开,谨守耶和华所吩咐{\PN{摩西}}的诫命。
\VS{7}耶和华与他同在,他无论往何处去尽都亨通。他背叛、不肯事奉{\PN{亚述}}王。
\VS{8}{\PN{希西家}}攻击{\PN{非利士}}人,直到{\PN{迦萨}},并{\PN{迦萨}}的四境,从了望楼到坚固城。
\par }{\PP \VS{9}{\PN{希西家}}王第四年,就是{\PN{以色列}}王{\PN{以拉}}的儿子{\PN{何细亚}}第七年,{\PN{亚述}}王{\PN{撒缦以色}}上来围困{\PN{撒马利亚}};
\VS{10}过了三年就攻取了城。{\PN{希西家}}第六年,{\PN{以色列}}王{\PN{何细亚}}第九年,{\PN{撒马利亚}}被攻取了。
\VS{11}{\PN{亚述}}王将{\PN{以色列}}人掳到{\PN{亚述}},把他们安置在{\PN{哈腊}}与{\PN{歌散}}的{\PN{哈博河}}边,并{\PN{米底亚}}人的城邑;
\VS{12}都因他们不听从耶和华—他们 神的话,违背他的约,就是耶和华仆人{\PN{摩西}}吩咐他们所当守的。
\par }{\SH 亚述人威胁耶路撒冷
\par }{\R (代下32·1—19;赛36·1—22)
\par }{\PP \VS{13}{\PN{希西家}}王十四年,{\PN{亚述}}王{\PN{西拿基立}}上来攻击{\PN{犹大}}的一切坚固城,将城攻取。
\VS{14}{\PN{犹大}}王{\PN{希西家}}差人往{\PN{拉吉}}去见{\PN{亚述}}王,说:「我有罪了,求你离开我;凡你罚我的,我必承当。」于是{\PN{亚述}}王罚{\PN{犹大}}王{\PN{希西家}}银子三百他连得,金子三十他连得。
\VS{15}{\PN{希西家}}就把耶和华殿里和王宫府库里所有的银子都给了{\ADD{他}}。
\VS{16}那时,{\PN{犹大}}王{\PN{希西家}}将耶和华殿门上的{\ADD{金子}}和他自己包在柱上的{\ADD{金子}}都刮下来,给了{\PN{亚述}}王。
\VS{17}{\PN{亚述}}王从{\PN{拉吉}}差遣他珥探、拉伯撒利,和拉伯沙基率领大军往{\PN{耶路撒冷}},到{\PN{希西家}}王那里去。他们上到{\PN{耶路撒冷}},就站在上池的水沟旁,在漂布地的大路上。
\VS{18}他们呼叫王的时候,就有{\PN{希勒家}}的儿子家宰{\PN{以利亚敬}},并书记{\PN{舍伯那}}和{\PN{亚萨}}的儿子史官{\PN{约亚}},出来见他们。
\par }{\PP \VS{19}拉伯沙基说:「你们去告诉{\PN{希西家}}说,{\PN{亚述}}大王如此说:『你所倚靠的有什么可仗赖的呢?
\VS{20}你说有打仗的计谋和能力,{\ADD{我看}}不过是虚话。你到底倚靠谁才背叛我呢?
\VS{21}看哪,你所倚靠的{\PN{埃及}}是那压伤的苇杖;人若靠这杖,就必刺透他的手。{\PN{埃及}}王法老向一切倚靠他的人也是这样。
\VS{22}你们若对我说:我们倚靠耶和华—我们的 神,{\PN{希西家}}岂不是将 神的邱坛和祭坛废去,且对{\PN{犹大}}和{\PN{耶路撒冷}}的人说:你们当在{\PN{耶路撒冷}}这坛前敬拜吗?
\VS{23}现在你把当头给我主{\PN{亚述}}王,我给你二千匹马,看你这一面骑马的人够不够。
\VS{24}若不然,怎能打败我主臣仆中最小的军长呢?你竟倚靠{\PN{埃及}}的战车马兵吗?
\VS{25}现在我上来攻击毁灭这地,岂没有耶和华的意思吗?耶和华吩咐我说:你上去攻击毁灭这地吧!』」
\par }{\PP \VS{26}{\PN{希勒家}}的儿子{\PN{以利亚敬}}和{\PN{舍伯那}},并{\PN{约亚}},对拉伯沙基说:「求你用{\PN{亚兰}}言语和仆人说话,因为我们懂得;不要用{\PN{犹大}}言语和我们说话,达到城上百姓的耳中。」
\VS{27}拉伯沙基说:「我主差遣我来,岂是单对你和你的主说这些话吗?不也是对这些坐在城上、要与你们一同吃自己粪、喝自己尿的人说吗?」
\VS{28}于是拉伯沙基站着,用{\PN{犹大}}言语大声喊着说:「你们当听{\PN{亚述}}大王的话!
\VS{29}王如此说:『你们不要被{\PN{希西家}}欺哄了;因他不能救你们脱离我的手。
\VS{30}也不要听{\PN{希西家}}使你们倚靠耶和华,说耶和华必要拯救我们,这城必不交在{\PN{亚述}}王的手中。』
\VS{31}不要听{\PN{希西家}}的话!因{\PN{亚述}}王如此说:『你们要与我和好,出来投降我,各人就可以吃自己葡萄树和无花果树的果子,喝自己井里的水。
\VS{32}等我来领你们到一个地方与你们本地一样,就是有五谷和新酒之地,有粮食和葡萄园之地,有橄榄树和蜂蜜之地,好使你们存活,不至于死。{\PN{希西家}}劝导你们,说耶和华必拯救我们;你们不要听他的话。
\VS{33}列国的神有哪一个救他本国脱离{\PN{亚述}}王的手呢?
\VS{34}{\PN{哈马}}、{\PN{亚珥拔}}的神在哪里呢?{\PN{西法瓦音}}、{\PN{希拿}}、{\PN{以瓦}}的神在哪里呢?他们曾救{\PN{撒马利亚}}脱离我的手吗?
\VS{35}这些国的神有谁曾救自己的国脱离我的手呢?难道耶和华能救{\PN{耶路撒冷}}脱离我的手吗?』」
\par }{\PP \VS{36}百姓静默不言,并不回答一句,因为王曾吩咐说:「不要回答他。」
\VS{37}当下,{\PN{希勒家}}的儿子家宰{\PN{以利亚敬}}和书记{\PN{舍伯那}},并{\PN{亚萨}}的儿子史官{\PN{约亚}},都撕裂衣服,来到{\PN{希西家}}那里,将拉伯沙基的话告诉了他。

\par }\Chap{19}{\SH 希西家王派人求问以赛亚
\par }{\R (赛37·1—7)
\par }{\PP \VerseOne{1}{\PN{希西家}}王听见,就撕裂衣服,披上麻布,进了耶和华的殿;
\VS{2}使家宰{\PN{以利亚敬}}和书记{\PN{舍伯那}},并祭司中的长老,都披上麻布,去见{\PN{亚摩斯}}的儿子先知{\PN{以赛亚}},
\VS{3}对他说:「{\PN{希西家}}如此说:『今日是急难、责罚、凌辱的日子,就如妇人将要生产婴孩,却没有力量生产。
\VS{4}或者耶和华—你的 神听见拉伯沙基的一切话,就是他主人{\PN{亚述}}王打发他来辱骂永生 神的话,耶和华—你的 神听见这话,就发斥责。故此,求你为余剩的民扬声祷告。』」
\VS{5}{\PN{希西家}}王的臣仆就去见{\PN{以赛亚}}。
\VS{6}{\PN{以赛亚}}对他们说:「要这样对你们的主人说,耶和华如此说:『你听见{\PN{亚述}}王的仆人亵渎我的话,不要惧怕。
\VS{7}我必惊动\FTNT{}{{\FR 19:7: }原文是使灵进入}他的心,他要听见风声就归回本地。我必使他在那里倒在刀下。』」
\par }{\SH 亚述人再度威胁
\par }{\R (赛37·8—20)
\par }{\PP \VS{8}拉伯沙基回去,正遇见{\PN{亚述}}王攻打{\PN{立拿}},原来他早听见{\PN{亚述}}王拔营离开{\PN{拉吉}}。
\VS{9}{\PN{亚述}}王听见人论{\PN{古实}}王{\PN{特哈加}}说:「他出来要与你争战。」于是{\PN{亚述}}王又打发使者去见{\PN{希西家}},吩咐他们说:
\VS{10}「你们对{\PN{犹大}}王{\PN{希西家}}如此说:『不要听你所倚靠的 神欺哄你,说{\PN{耶路撒冷}}必不交在{\PN{亚述}}王的手中。
\VS{11}你总听说{\PN{亚述}}诸王向列国所行的,乃是尽行灭绝,难道你还能得救吗?
\VS{12}我列祖所毁灭的,就是{\PN{歌散}}、{\PN{哈兰}}、{\PN{利色}},和属{\PN{提·拉撒}}的{\PN{伊甸}}人,这些国的神何曾拯救这些国呢?
\VS{13}{\PN{哈马}}的王、{\PN{亚珥拔}}的王、{\PN{西法瓦音}}城的王、{\PN{希拿}},和{\PN{以瓦}}的王都在哪里呢?』」
\par }{\PP \VS{14}{\PN{希西家}}从使者手里接过书信来,看完了,就上耶和华的殿,将书信在耶和华面前展开。
\VS{15}{\PN{希西家}}向耶和华祷告说:「坐在二基路伯上耶和华—{\PN{以色列}}的 神啊,你是天下万国的 神,你曾创造天地。
\VS{16}耶和华啊,求你侧耳而听!耶和华啊,求你睁眼而看!要听{\PN{西拿基立}}打发使者来辱骂永生 神的话。
\VS{17}耶和华啊,{\PN{亚述}}诸王果然使列国和列国之地变为荒凉,
\VS{18}将列国的神像都扔在火里;因为它本不是神,乃是人手所造的,是木头石头的,所以灭绝它。
\VS{19}耶和华—我们的 神啊,现在求你救我们脱离{\PN{亚述}}王的手,使天下万国都知道惟独你—耶和华是 神!」
\par }{\SH 以赛亚传给王的信息
\par }{\R (赛37·20—37)
\par }{\PP \VS{20}{\PN{亚摩斯}}的儿子{\PN{以赛亚}}就打发人去见{\PN{希西家}},说:「耶和华—{\PN{以色列}}的 神如此说:『你既然求我攻击{\PN{亚述}}王{\PN{西拿基立}},我已听见了。』
\VS{21}耶和华论他这样说:
\par }{\Q {\PN{锡安}}的处女藐视你,嗤笑你;
\par }{\Q {\PN{耶路撒冷}}的女子向你摇头。
\par }{\Q \VS{22}你辱骂谁?亵渎谁?
\par }{\Q 扬起声来,高举眼目攻击谁呢?
\par }{\Q 乃是攻击{\PN{以色列}}的圣者!
\par }{\Q \VS{23}你借你的使者辱骂主,
\par }{\Q 并说:我率领许多战车上山顶,
\par }{\Q 到{\PN{黎巴嫩}}极深之处;
\par }{\Q 我要砍伐其中高大的香柏树和佳美的松树;
\par }{\Q 我必上极高之处,进入肥田的树林。
\par }{\Q \VS{24}我已经在外邦挖井喝水;
\par }{\Q 我必用脚掌踏干{\PN{埃及}}的一切河。
\par }{\BB \par }{\Q \VS{25}{\ADD{耶和华说}},我早先所做的,
\par }{\Q 古时所立的,就是现在借你
\par }{\Q 使坚固城荒废,变为乱堆,
\par }{\Q 这事你岂没有听见吗?
\par }{\Q \VS{26}所以其中的居民力量甚小,
\par }{\Q 惊惶羞愧。
\par }{\Q 他们像野草,像青菜,
\par }{\Q 如房顶上的草,
\par }{\Q 又如未长成而枯干的禾稼。
\par }{\Q \VS{27}你坐下,你出去,你进来,
\par }{\Q 你向我发烈怒,我都知道。
\par }{\Q \VS{28}因你向我发烈怒,
\par }{\Q 又因你狂傲{\ADD{的话}}达到我耳中,
\par }{\Q 我就要用钩子钩上你的鼻子,
\par }{\Q 把嚼环放在你口里,
\par }{\Q 使你从你来的路转回去。
\par }{\PP \VS{29}「{\PN{以色列}}人哪,我赐你们一个证据:你们今年要吃自生的,明年也要吃自长的;至于后年,你们要耕种收割,栽植葡萄园,吃其中的果子。
\VS{30}{\PN{犹大}}家所逃脱余剩的,仍要往下扎根,向上结果。
\VS{31}必有余剩的民从{\PN{耶路撒冷}}而出;必有逃脱的人从{\PN{锡安山}}而来。耶和华的热心必成就这事。
\par }{\PP \VS{32}「所以,耶和华论{\PN{亚述}}王如此说:『他必不得来到这城,也不在这里射箭,不得拿盾牌到城前,也不筑垒攻城。
\VS{33}他从哪条路来,必从那条路回去,必不得来到这城。这是耶和华说的。
\VS{34}因我为自己的缘故,又为我仆人{\PN{大卫}}的缘故,必保护拯救这城。』」
\par }{\PP \VS{35}当夜,耶和华的使者出去,在{\PN{亚述}}营中杀了十八万五千人。清早有人起来,一看,都是死尸了。
\VS{36}{\PN{亚述}}王{\PN{西拿基立}}就拔营回去,住在{\PN{尼尼微}}。
\VS{37}一日在他的神{\PN{尼斯洛}}庙里叩拜,{\ADD{他儿子}}{\PN{亚得米勒}}和{\PN{沙利色}}用刀杀了他,就逃到{\PN{亚拉腊}}地。他儿子{\PN{以撒哈顿}}接续他作王。

\par }\Chap{20}{\SH 希西家病重康复
\par }{\R (赛38·1—8;21—22;代下32·24—26)
\par }{\PP \VerseOne{1}那时,{\PN{希西家}}病得要死。{\PN{亚摩斯}}的儿子先知{\PN{以赛亚}}去见他,对他说:「耶和华如此说:『你当留遗命与你的家,因为你必死,不能活了。』」
\VS{2}{\PN{希西家}}就转脸朝墙,祷告耶和华说:
\VS{3}「耶和华啊,求你记念我在你面前怎样存完全的心,按诚实行事,又做你眼中所看为善的。」{\PN{希西家}}就痛哭了。
\VS{4}{\PN{以赛亚}}出来,还没有到中院\FTNT{}{{\FR 20:4: }院:或译城},耶和华的话就临到他,说:
\VS{5}「你回去告诉我民的君{\PN{希西家}}说:耶和华—你祖{\PN{大卫}}的 神如此说:『我听见了你的祷告,看见了你的眼泪,我必医治你;到第三日,你必上到耶和华的殿。
\VS{6}我必加增你十五年的寿数,并且我要救你和这城脱离{\PN{亚述}}王的手。我为自己和我仆人{\PN{大卫}}的缘故,必保护这城。』」
\VS{7}{\PN{以赛亚}}说:「当取一块无花果饼来。」人就取了来,贴在疮上,王便痊愈了。
\par }{\PP \VS{8}{\PN{希西家}}问{\PN{以赛亚}}说:「耶和华必医治我,到第三日,我能上耶和华的殿,有什么兆头呢?」
\VS{9}{\PN{以赛亚}}说:「耶和华必成就他所说的。这是他给你的兆头:你要日影向前进十度呢?是要往后退十度呢?」
\VS{10}{\PN{希西家}}回答说:「日影向前进十度容易,我要日影往后退十度。」
\VS{11}先知{\PN{以赛亚}}求告耶和华,耶和华就使{\PN{亚哈斯}}的日晷向前进的日影,往后退了十度。
\par }{\SH 从巴比伦来的使者
\par }{\R (赛39·1—8)
\par }{\PP \VS{12}那时,{\PN{巴比伦}}王{\PN{巴拉但}}的儿子{\PN{米罗达·巴拉但}}听见{\PN{希西家}}病而痊愈,就送书信和礼物给他。
\VS{13}{\PN{希西家}}听从使者的话,就把他宝库的金子、银子、香料、贵重的膏油,和他武库的一切军器,并他所有的财宝,都给他们看。他家中和他全国之内,{\PN{希西家}}没有一样不给他们看的。
\VS{14}于是先知{\PN{以赛亚}}来见{\PN{希西家}}王,问他说:「这些人说什么?他们从哪里来见你?」{\PN{希西家}}说:「他们从远方的{\PN{巴比伦}}来。」
\VS{15}{\PN{以赛亚}}说:「他们在你家里看见了什么?」{\PN{希西家}}说:「凡我家中所有的,他们都看见了;我财宝中没有一样不给他们看的。」
\par }{\PP \VS{16}{\PN{以赛亚}}对{\PN{希西家}}说:「你要听耶和华的话,
\VS{17}日子必到,凡你家里所有的,并你列祖积蓄到如今的,都要被掳到{\PN{巴比伦}}去,不留下一样。这是耶和华说的。
\VS{18}并且从你本身所生的众子,其中必有被掳去在{\PN{巴比伦}}王宫里当太监的。」
\VS{19}{\PN{希西家}}对{\PN{以赛亚}}说:「你所说耶和华的话甚好!若在我的年日中有太平和稳固的景况,岂不是好吗?」
\par }{\SH 希西家逝世
\par }{\R (代下32·32—33)
\par }{\PP \VS{20}{\PN{希西家}}其余的事和他的勇力,他怎样挖池、挖沟、引水入城,都写在{\PN{犹大}}列王记上。
\VS{21}{\PN{希西家}}与他列祖同睡。他儿子{\PN{玛拿西}}接续他作王。

\par }\Chap{21}{\SH 犹大王玛拿西
\par }{\R (代下33·1—20)
\par }{\PP \VerseOne{1}{\PN{玛拿西}}登基的时候年十二岁,在{\PN{耶路撒冷}}作王五十五年。他母亲名叫{\PN{协西巴}}。
\VS{2}{\PN{玛拿西}}行耶和华眼中看为恶的事,效法耶和华在{\PN{以色列}}人面前赶出的外邦人所行可憎的事。
\VS{3}重新建筑他父{\PN{希西家}}所毁坏的邱坛,又为{\PN{巴力}}筑坛,做{\PN{亚舍拉}}像,效法{\PN{以色列}}王{\PN{亚哈}}所行的,且敬拜事奉天上的万象;
\VS{4}在耶和华殿宇中筑坛。耶和华曾指着这殿说:「我必立我的名在{\PN{耶路撒冷}}。」
\VS{5}他在耶和华殿的两院中为天上的万象筑坛,
\VS{6}并使他的儿子经火,又观兆,用法术,立交鬼的和行巫术的,多行耶和华眼中看为恶的事,惹动他的怒气;
\VS{7}又在殿内立雕刻的{\PN{亚舍拉}}像。耶和华曾对{\PN{大卫}}和他儿子{\PN{所罗门}}说:「我在{\PN{以色列}}众支派中所选择的{\PN{耶路撒冷}}和这殿,必立我的名,直到永远。
\VS{8}{\PN{以色列}}人若谨守遵行我一切所吩咐他们的和我仆人{\PN{摩西}}所吩咐他们的一切律法,我就不再使他们挪移离开我所赐给他们列祖之地。」
\VS{9}他们却不听从。{\PN{玛拿西}}引诱他们行恶,比耶和华在{\PN{以色列}}人面前所灭的列国更甚。
\par }{\PP \VS{10}耶和华借他仆人众先知说:
\VS{11}「因{\PN{犹大}}王{\PN{玛拿西}}行这些可憎的恶事比先前{\PN{亚摩利}}人所行的更甚,使{\PN{犹大}}人拜他的偶像,陷在罪里;
\VS{12}所以耶和华—{\PN{以色列}}的 神如此说:我必降祸与{\PN{耶路撒冷}}和{\PN{犹大}},叫一切听见的人无不耳鸣。
\VS{13}我必用量{\PN{撒马利亚}}的准绳和{\PN{亚哈}}家的线铊拉在{\PN{耶路撒冷}}上,必擦净{\PN{耶路撒冷}},如人擦盘,将盘倒扣。
\VS{14}我必弃掉所余剩的子民\FTNT{}{{\FR 21:14: }原文是产业},把他们交在仇敌手中,使他们成为一切仇敌掳掠之物;
\VS{15}是因他们自从列祖出{\PN{埃及}}直到如今,常行我眼中看为恶的事,惹动我的怒气。」
\par }{\PP \VS{16}{\PN{玛拿西}}行耶和华眼中看为恶的事,使{\PN{犹大}}人陷在罪里,又流许多无辜人的血,充满了{\PN{耶路撒冷}},从这边直到那边。
\par }{\PP \VS{17}{\PN{玛拿西}}其余的事,凡他所行的和他所犯的罪都写在{\PN{犹大}}列王记上。
\VS{18}{\PN{玛拿西}}与他列祖同睡,葬在自己宫院{\PN{乌撒}}的园内;他儿子{\PN{亚们}}接续他作王。
\par }{\SH 犹大王亚们
\par }{\R (代下33·21—25)
\par }{\PP \VS{19}{\PN{亚们}}登基的时候年二十二岁,在{\PN{耶路撒冷}}作王二年。他母亲名叫{\PN{米舒利密}},是{\PN{约提巴}}人{\PN{哈鲁斯}}的女儿。
\VS{20}{\PN{亚们}}行耶和华眼中看为恶的事,与他父亲{\PN{玛拿西}}所行的一样;
\VS{21}行他父亲一切所行的,敬奉他父亲所敬奉的偶像,
\VS{22}离弃耶和华—他列祖的 神,不遵行耶和华的道。
\VS{23}{\PN{亚们}}王的臣仆背叛他,在宫里杀了他。
\VS{24}但国民杀了那些背叛{\PN{亚们}}王的人,立他儿子{\PN{约西亚}}接续他作王。
\VS{25}{\PN{亚们}}其余所行的事都写在{\PN{犹大}}列王记上。
\VS{26}{\PN{亚们}}葬在{\PN{乌撒}}的园内自己的坟墓里。他儿子{\PN{约西亚}}接续他作王。

\par }\Chap{22}{\SH 犹大王约西亚
\par }{\R (代下34·1—2)
\par }{\PP \VerseOne{1}{\PN{约西亚}}登基的时候年八岁,在{\PN{耶路撒冷}}作王三十一年。他母亲名叫{\PN{耶底大}},是{\PN{波斯加}}人{\PN{亚大雅}}的女儿。
\VS{2}{\PN{约西亚}}行耶和华眼中看为正的事,行他祖{\PN{大卫}}一切所行的,不偏左右。
\par }{\SH 发现律法书
\par }{\R (代下34·8—28)
\par }{\PP \VS{3}{\PN{约西亚}}王十八年,王差遣{\PN{米书兰}}的孙子、{\PN{亚萨利}}的儿子—书记{\PN{沙番}}上耶和华殿去,吩咐他说:
\VS{4}「你去见大祭司{\PN{希勒家}},使他将奉到耶和华殿的银子,就是守门的从民中收聚的银子,数算数算,
\VS{5}交给耶和华殿里办事的人,使他们转交耶和华殿里做工的人,好修理殿的破坏之处,
\VS{6}就是转交木匠和工人,并瓦匠,又买木料和凿成的石头修理殿宇,
\VS{7}将银子交在办事的人手里,不与他们算帐,因为他们办事诚实。」
\par }{\PP \VS{8}大祭司{\PN{希勒家}}对书记{\PN{沙番}}说:「我在耶和华殿里得了律法书。」{\PN{希勒家}}将书递给{\PN{沙番}},{\PN{沙番}}就看了。
\VS{9}书记{\PN{沙番}}到王那里,回复王说:「你的仆人已将殿里的银子倒出数算,交给耶和华殿里办事的人了。」
\VS{10}书记{\PN{沙番}}又对王说:「祭司{\PN{希勒家}}递给我一卷书。」{\PN{沙番}}就在王面前读那书。
\par }{\PP \VS{11}王听见律法书上的话,便撕裂衣服,
\VS{12}吩咐祭司{\PN{希勒家}}与{\PN{沙番}}的儿子{\PN{亚希甘}}、{\PN{米该亚}}的儿子{\PN{亚革波}}、书记{\PN{沙番}}和王的臣仆{\PN{亚撒雅}},说:
\VS{13}「你们去为我、为民、为{\PN{犹大}}众人,以这书上的话求问耶和华;因为我们列祖没有听从这书上的言语,没有遵着书上所吩咐我们的去行,耶和华就向我们大发烈怒。」
\par }{\PP \VS{14}于是,祭司{\PN{希勒家}}和{\PN{亚希甘}}、{\PN{亚革波}}、{\PN{沙番}}、{\PN{亚撒雅}}都去见女先知{\PN{户勒大}}。{\PN{户勒大}}是掌管礼服{\PN{沙龙}}的妻;{\PN{沙龙}}是{\PN{哈珥哈斯}}的孙子、{\PN{特瓦}}的儿子。{\PN{户勒大}}住在{\PN{耶路撒冷}}第二区。他们请问于她。
\VS{15}她对他们说:「耶和华—{\PN{以色列}}的 神如此说:『你们可以回复那差遣你们来见我的人说,
\VS{16}耶和华如此说:我必照着{\PN{犹大}}王所读那书上的一切话,降祸与这地和其上的居民。
\VS{17}因为他们离弃我,向别神烧香,用他们手所做的惹我发怒,所以我的忿怒必向这地发作,总不止息。』
\VS{18}然而,差遣你们来求问耶和华的{\PN{犹大}}王,你们要这样回复他说:『耶和华—{\PN{以色列}}的 神如此说:至于你所听见的话,
\VS{19}就是听见我指着这地和其上的居民所说、要使这地变为荒场、民受咒诅的话,你便心里敬服,在我面前自卑,撕裂衣服,向我哭泣,因此我应允了你。这是我—耶和华说的。
\VS{20}我必使你平平安安地归到坟墓到你列祖那里;我要降与这地的一切灾祸,你也不至亲眼看见。』」他们就回复王去了。

\par }\Chap{23}{\SH 约西亚废除异教崇拜
\par }{\R (代下34·3—7;29—33)
\par }{\PP \VerseOne{1}王差遣人招聚{\PN{犹大}}和{\PN{耶路撒冷}}的众长老来。
\VS{2}王和{\PN{犹大}}众人与{\PN{耶路撒冷}}的居民,并祭司、先知,和所有的百姓,无论大小,都一同上到耶和华的殿;王就把耶和华殿里所得的约书念给他们听。
\VS{3}王站在柱旁,在耶和华面前立约,要尽心尽性地顺从耶和华,遵守他的诫命、法度、律例,成就这书上所记的约言。众民都服从这约。
\par }{\PP \VS{4}王吩咐大祭司{\PN{希勒家}}和副祭司,并把门的,将那为{\PN{巴力}}和{\PN{亚舍拉}},并天上万象所造的器皿,都从耶和华殿里搬出来,在{\PN{耶路撒冷}}外{\PN{汲沦溪}}旁的田间烧了,把灰拿到{\PN{伯特利}}去。
\VS{5}从前{\PN{犹大}}列王所立拜偶像的祭司,在{\PN{犹大}}城邑的邱坛和{\PN{耶路撒冷}}的周围烧香,现在王都废去,又废去向{\PN{巴力}}和日、月、行星\FTNT{}{{\FR 23:5: }或译:十二宫},并天上万象烧香的人;
\VS{6}又从耶和华殿里将{\PN{亚舍拉}}搬到{\PN{耶路撒冷}}外{\PN{汲沦溪}}边焚烧,打碎成灰,将灰撒在平民的坟上;
\VS{7}又拆毁耶和华殿里娈童的屋子,就是妇女为{\PN{亚舍拉}}织帐子的屋子,
\VS{8}并且从{\PN{犹大}}的城邑带众祭司来,污秽祭司烧香的邱坛,从{\PN{迦巴}}直到{\PN{别是巴}},又拆毁城门旁的邱坛,这邱坛在邑宰{\PN{约书亚}}门前,进城门的左边。
\VS{9}但是邱坛的祭司不登{\PN{耶路撒冷}}耶和华的坛,只在他们弟兄中间吃无酵饼。
\VS{10}又污秽{\PN{欣嫩子谷}}的{\PN{陀斐特}},不许人在那里使儿女经火献给{\PN{摩洛}};
\VS{11}又将{\PN{犹大}}列王在耶和华殿门旁、太监{\PN{拿单·米勒}}靠近游廊的屋子、向日头所献的马废去,且用火焚烧日车。
\VS{12}{\PN{犹大}}列王在{\PN{亚哈斯}}楼顶上所筑的坛和{\PN{玛拿西}}在耶和华殿两院中所筑的坛,王都拆毁打碎了,就把灰倒在{\PN{汲沦溪}}中。
\VS{13}从前{\PN{以色列}}王{\PN{所罗门}}在{\PN{耶路撒冷}}前、邪僻山右边为{\PN{西顿}}人可憎的神{\PN{亚斯她录}}、{\PN{摩押}}人可憎的神{\PN{基抹}}、{\PN{亚扪}}人可憎的神{\PN{米勒公}}所筑的邱坛,王都污秽了,
\VS{14}又打碎柱像,砍下木偶,将人的骨头充满了那地方。
\par }{\PP \VS{15}他将{\PN{伯特利}}的坛,就是叫{\PN{以色列}}人陷在罪里、{\PN{尼八}}的儿子{\PN{耶罗波安}}所筑的那坛,都拆毁焚烧,打碎成灰,并焚烧了{\PN{亚舍拉}}。
\VS{16}{\PN{约西亚}}回头,看见山上的坟墓,就打发人将坟墓里的骸骨取出来,烧在坛上,污秽了坛,正如从前神人宣传耶和华的话。
\VS{17}{\PN{约西亚}}问说:「我所看见的是什么碑?」那城里的人回答说:「先前有神人从{\PN{犹大}}来,预先说王现在向{\PN{伯特利}}坛所行的事,这就是他的墓碑。」
\VS{18}{\PN{约西亚}}说:「由他吧!不要挪移他的骸骨。」他们就不动他的骸骨,也不动从{\PN{撒马利亚}}来那先知的骸骨。
\VS{19}从前{\PN{以色列}}诸王在{\PN{撒马利亚}}的城邑建筑邱坛的殿,惹动{\ADD{耶和华}}的怒气,现在{\PN{约西亚}}都废去了,就如他在{\PN{伯特利}}所行的一般;
\VS{20}又将邱坛的祭司都杀在坛上,并在坛上烧人的骨头,就回{\PN{耶路撒冷}}去了。
\par }{\SH 约西亚守逾越节
\par }{\R (代下35·1—19)
\par }{\PP \VS{21}王吩咐众民说:「你们当照这约书上所写的,向耶和华—你们的 神守逾越节。」
\VS{22}自从士师治理{\PN{以色列}}人和{\PN{以色列}}王、{\PN{犹大}}王的时候,{\ADD{直到如今}},实在没有守过这样的逾越节;
\VS{23}只有{\PN{约西亚}}王十八年在{\PN{耶路撒冷}}向耶和华守这逾越节。
\par }{\SH 约西亚的其他改革
\par }{\PP \VS{24}凡{\PN{犹大}}国和{\PN{耶路撒冷}}所有交鬼的、行巫术的,与家中的神像和偶像,并一切可憎之物,{\PN{约西亚}}尽都除掉,成就了祭司{\PN{希勒家}}在耶和华殿里所得律法书上所写的话。
\VS{25}在{\PN{约西亚}}以前没有王像他尽心、尽性、尽力地归向耶和华,遵行{\PN{摩西}}的一切律法;在他以后也没有兴起一个王像他。
\par }{\PP \VS{26}然而,耶和华向{\PN{犹大}}所发猛烈的怒气仍不止息,是因{\PN{玛拿西}}诸事惹动他。
\VS{27}耶和华说:「我必将{\PN{犹大}}人从我面前赶出,如同赶出{\PN{以色列}}人一般;我必弃掉我从前所选择的这城—{\PN{耶路撒冷}}和我所说立我名的殿。」
\par }{\SH 约西亚逝世
\par }{\R (代下35·20—36·1)
\par }{\PP \VS{28}{\PN{约西亚}}其余的事,凡他所行的都写在{\PN{犹大}}列王记上。
\VS{29}{\PN{约西亚}}年间,{\PN{埃及}}王法老{\PN{尼哥}}上到{\PN{幼发拉底河}}攻击{\PN{亚述}}王;{\PN{约西亚}}王去抵挡他。{\PN{埃及}}王遇见{\PN{约西亚}}在{\PN{米吉多}},就杀了他。
\VS{30}他的臣仆用车将他的尸首从{\PN{米吉多}}送到{\PN{耶路撒冷}},葬在他自己的坟墓里。国民膏{\PN{约西亚}}的儿子{\PN{约哈斯}}接续他父亲作王。
\par }{\SH 犹大王约哈斯
\par }{\R (代下36·2—4)
\par }{\PP \VS{31}{\PN{约哈斯}}登基的时候年二十三岁,在{\PN{耶路撒冷}}作王三个月。他母亲名叫{\PN{哈慕她}},是{\PN{立拿}}人{\PN{耶利米}}的女儿。
\VS{32}{\PN{约哈斯}}行耶和华眼中看为恶的事,效法他列祖一切所行的。
\VS{33}法老{\PN{尼哥}}将{\PN{约哈斯}}锁禁在{\PN{哈马}}地的{\PN{利比拉}},不许他在{\PN{耶路撒冷}}作王,又罚{\ADD{
{\PN{犹大}}}}国银子一百他连得,金子一他连得。
\VS{34}法老{\PN{尼哥}}立{\PN{约西亚}}的儿子{\PN{以利亚敬}}接续他父亲{\PN{约西亚}}作王,给他改名叫{\PN{约雅敬}},却将{\PN{约哈斯}}带到{\PN{埃及}},他就死在那里。
\par }{\SH 犹大王约雅敬
\par }{\R (代下36·5—8)
\par }{\PP \VS{35}{\PN{约雅敬}}将金银给法老,遵着法老的命向国民征取金银,按着各人的力量派定,索要金银,好给法老{\PN{尼哥}}。
\VS{36}{\PN{约雅敬}}登基的时候年二十五岁,在{\PN{耶路撒冷}}作王十一年。他母亲名叫{\PN{西布大}},是{\PN{鲁玛}}人{\PN{毗大雅}}的女儿。
\VS{37}{\PN{约雅敬}}行耶和华眼中看为恶的事,效法他列祖一切所行的。

\par }\Chap{24}{\PP \VerseOne{1}{\PN{约雅敬}}年间,{\PN{巴比伦}}王{\PN{尼布甲尼撒}}上{\ADD{到
{\PN{犹大}}}};{\PN{约雅敬}}服事他三年,然后背叛他。
\VS{2}耶和华使{\PN{迦勒底}}军、{\PN{亚兰}}军、{\PN{摩押}}军,和{\PN{亚扪}}人的军来攻击{\PN{约雅敬}},毁灭{\PN{犹大}},正如耶和华借他仆人众先知所说的。
\VS{3}这祸临到{\PN{犹大}}人,诚然是耶和华所命的,要将他们从自己面前赶出,是因{\PN{玛拿西}}所犯的一切罪;
\VS{4}又因他流无辜人的血,充满了{\PN{耶路撒冷}};耶和华决不肯赦免。
\VS{5}{\PN{约雅敬}}其余的事,凡他所行的都写在{\PN{犹大}}列王记上。
\VS{6}{\PN{约雅敬}}与他列祖同睡。他儿子{\PN{约雅斤}}接续他作王。
\VS{7}{\PN{埃及}}王不再从他国中出来;因为{\PN{巴比伦}}王将{\PN{埃及}}王所管之地,从{\PN{埃及}}小河直到{\PN{幼发拉底河}}都夺去了。
\par }{\SH 犹大王约雅斤
\par }{\R (代下36·9—10)
\par }{\PP \VS{8}{\PN{约雅斤}}登基的时候年十八岁,在{\PN{耶路撒冷}}作王三个月。他母亲名叫{\PN{尼护施她}},是{\PN{耶路撒冷}}人{\PN{以利拿单}}的女儿。
\VS{9}{\PN{约雅斤}}行耶和华眼中看为恶的事,效法他父亲一切所行的。
\par }{\PP \VS{10}那时,{\PN{巴比伦}}王{\PN{尼布甲尼撒}}的军兵上到{\PN{耶路撒冷}},围困城。
\VS{11}当他军兵围困城的时候,{\PN{巴比伦}}王{\PN{尼布甲尼撒}}就亲自来了。
\VS{12}{\PN{犹大}}王{\PN{约雅斤}}和他母亲、臣仆、首领、太监一同出城,投降{\PN{巴比伦}}王;{\PN{巴比伦}}王便拿住他。那时是{\PN{巴比伦}}王第八年。
\VS{13}{\PN{巴比伦}}王将耶和华殿和王宫里的宝物都拿去了,将{\PN{以色列}}王{\PN{所罗门}}所造耶和华殿里的金器都毁坏了,正如耶和华所说的;
\VS{14}又将{\PN{耶路撒冷}}的众民和众首领,并所有大能的勇士,共一万人,连一切木匠、铁匠都掳了去;除了国中极贫穷的人以外,没有剩下的;
\VS{15}并将{\PN{约雅斤}}和王母、后妃、太监,与国中的大官,都从{\PN{耶路撒冷}}掳到{\PN{巴比伦}}去了;
\VS{16}又将一切勇士七千人和木匠、铁匠一千人,都是能上阵的勇士,全掳到{\PN{巴比伦}}去了。
\VS{17}{\PN{巴比伦}}王立{\ADD{
{\PN{约雅斤}}}}的叔叔{\PN{玛探雅}}代替他作王,给{\PN{玛探雅}}改名叫{\PN{西底家}}。
\par }{\SH 犹大王西底家
\par }{\R (代下36·11—12;耶52·1—3)
\par }{\PP \VS{18}{\PN{西底家}}登基的时候年二十一岁,在{\PN{耶路撒冷}}作王十一年。他母亲名叫{\PN{哈慕她}},是{\PN{立拿}}人{\PN{耶利米}}的女儿。
\VS{19}{\PN{西底家}}行耶和华眼中看为恶的事,是照{\PN{约雅敬}}一切所行的。
\VS{20}因此耶和华的怒气在{\PN{耶路撒冷}}和{\PN{犹大}}发作,以致将人民从自己面前赶出。

\par }\Chap{25}{\SH 耶路撒冷陷落
\par }{\R (代下36·13—21;耶52·3—11)
\par }{\PP \VerseOne{1}{\PN{西底家}}背叛{\PN{巴比伦}}王。他作王第九年十月初十日,{\PN{巴比伦}}王{\PN{尼布甲尼撒}}率领全军来攻击{\PN{耶路撒冷}},对城安营,四围筑垒攻城。
\VS{2}于是城被围困,直到{\PN{西底家}}王十一年。
\VS{3}四月初九日,城里有大饥荒,甚至百姓都没有粮食。
\VS{4}城被攻破,一切兵丁就在夜间从靠近王园两城中间的门{\ADD{逃跑}}。{\PN{迦勒底}}人正在四围攻城,{\ADD{王}}就向{\PN{亚拉巴}}逃走。
\VS{5}{\PN{迦勒底}}的军队追赶王,在{\PN{耶利哥}}的平原追上他;他的全军都离开他四散了。
\VS{6}{\PN{迦勒底}}人就拿住王,带他到在{\PN{利比拉}}的{\PN{巴比伦}}王那里审判他。
\VS{7}在{\PN{西底家}}眼前杀了他的众子,并且剜了{\PN{西底家}}的眼睛,用铜链锁着他,带到{\PN{巴比伦}}去。
\par }{\SH 圣殿被毁
\par }{\R (耶52·12—33)
\par }{\PP \VS{8}{\PN{巴比伦}}王{\PN{尼布甲尼撒}}十九年五月初七日,{\PN{巴比伦}}王的臣仆、护卫长{\PN{尼布撒拉旦}}来到{\PN{耶路撒冷}},
\VS{9}用火焚烧耶和华的殿和王宫,又焚烧{\PN{耶路撒冷}}的房屋,就是各大户家的房屋。
\VS{10}跟从护卫长{\PN{迦勒底}}的全军就拆毁{\PN{耶路撒冷}}四围的城墙。
\VS{11}那时护卫长{\PN{尼布撒拉旦}}将城里所剩下的百姓,并已经投降{\PN{巴比伦}}王的人,以及大众所剩下的人,都掳去了。
\VS{12}但护卫长留下些民中最穷的,使他们修理葡萄园,耕种田地。
\par }{\PP \VS{13}耶和华殿的铜柱,并耶和华殿的{\ADD{盆}}座和铜海,{\PN{迦勒底}}人都打碎了,将那铜运到{\PN{巴比伦}}去了,
\VS{14}又带去锅、铲子、蜡剪、调羹,并所用的一切铜器,
\VS{15}火鼎、碗,无论金的银的,护卫长也都带去了。
\VS{16}{\PN{所罗门}}为耶和华殿所造的两根{\ADD{铜}}柱、一个{\ADD{铜}}海,和几个{\ADD{盆}}座,这一切的铜,多得无法可称。
\VS{17}这一根柱子高十八肘,柱上有铜顶,高三肘;铜顶的周围有网子和石榴,都是铜的。那一根柱子,照此一样,也有网子。
\par }{\SH 犹大人被掳
\par }{\R (耶52·24—27)
\par }{\PP \VS{18}护卫长拿住大祭司{\PN{西莱雅}}、副祭司{\PN{西番亚}},和三个把门的,
\VS{19}又从城中拿住一个管理兵丁的官\FTNT{}{{\FR 25:19: }或译:太监},并在城里所遇常见王面的五个人和检点国民军长的书记,以及城里遇见的国民六十个人。
\VS{20}护卫长{\PN{尼布撒拉旦}}将这些人带到在{\PN{利比拉}}的{\PN{巴比伦}}王那里。
\VS{21}{\PN{巴比伦}}王就把他们击杀在{\PN{哈马}}地的{\PN{利比拉}}。这样,{\PN{犹大}}人被掳去离开本地。
\par }{\SH 立基大利作犹大省长
\par }{\R (耶40·7—9;41·1—3)
\par }{\PP \VS{22}至于{\PN{犹大}}国剩下的民,就是{\PN{巴比伦}}王{\PN{尼布甲尼撒}}所剩下的,{\PN{巴比伦}}王立了{\PN{沙番}}的孙子、{\PN{亚希甘}}的儿子{\PN{基大利}}作他们的省长。
\VS{23}众军长和属他们的人听见{\PN{巴比伦}}王立了{\PN{基大利}}作省长,于是军长{\PN{尼探雅}}的儿子{\PN{以实玛利}}、{\PN{加利亚}}的儿子{\PN{约哈难}}、{\PN{尼陀法}}人{\PN{单户篾}}的儿子{\PN{西莱雅}}、{\PN{玛迦}}人的儿子{\PN{雅撒尼亚}},和属他们的人都到{\PN{米斯巴}}见{\PN{基大利}}。
\VS{24}{\PN{基大利}}向他们和属他们的人起誓说:「你们不必惧怕{\PN{迦勒底}}臣仆,只管住在这地服事{\PN{巴比伦}}王,就可以得福。」
\VS{25}七月间,宗室{\PN{以利沙玛}}的孙子、{\PN{尼探雅}}的儿子{\PN{以实玛利}}带着十个人来,杀了{\PN{基大利}}和同他在{\PN{米斯巴}}的{\PN{犹大}}人与{\PN{迦勒底}}人。
\VS{26}于是众民,无论大小,连众军长;因为惧怕{\PN{迦勒底}}人,都起身往{\PN{埃及}}去了。
\par }{\SH 约雅斤被释
\par }{\R (耶52·31—34)
\par }{\PP \VS{27}{\PN{犹大}}王{\PN{约雅斤}}被掳后三十七年,{\PN{巴比伦}}王{\PN{以未·米罗达}}元年十二月二十七日,使{\PN{犹大}}王{\PN{约雅斤}}抬头,提他出监;
\VS{28}又对他说恩言,使他的位高过与他一同在{\PN{巴比伦}}众王的位,
\VS{29}给他脱了囚服。他终身常在{\ADD{
{\PN{巴比伦}}}}王面前吃饭。
\VS{30}王赐他所需用的食物,日日赐他一分,终身都是这样。
\par }