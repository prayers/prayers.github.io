\NormalFont\ShortTitle{撒迦利亚书}
{\MT 撒迦利亚书

\par }\ChapOne{1}{\SH 耶和华呼唤子民归向他
\par }{\PP \VerseOne{1}{\PN{大流士}}{\ADD{王}}第二年八月,耶和华的话临到{\PN{易多}}的孙子、{\PN{比利家}}的儿子先知{\PN{撒迦利亚}},说:
\VS{2}「耶和华曾向你们列祖大大发怒。
\VS{3}所以你要对{\PN{以色列}}人说,万军之耶和华如此说:你们要转向我,我就转向你们。这是万军之耶和华说的。
\VS{4}不要效法你们列祖。从前的先知呼叫他们说,万军之耶和华如此说:『你们要回头离开你们的恶道恶行。』他们却不听,也不顺从我。这是耶和华说的。
\VS{5}你们的列祖在哪里呢?那些先知能永远存活吗?
\VS{6}只是我的言语和律例,就是所吩咐我仆人众先知的,岂不临到你们列祖吗?他们就回头,说:『万军之耶和华定意按我们的行动作为向我们怎样行,他已照样行了。』」
\par }{\SH 马的异象
\par }{\PP \VS{7}{\PN{大流士}}第二年十一月,就是细罢特月二十四日,耶和华的话临到{\PN{易多}}的孙子、{\PN{比利家}}的儿子先知{\PN{撒迦利亚}},说:
\VS{8}「我夜间观看,见一人骑着红马,站在洼地番石榴树中间。在他身后又有红马、黄马,和白马。」
\VS{9}我对与我说话的天使说:「主啊,这是什么意思?」他说:「我要指示你这是什么意思。」
\VS{10}那站在番石榴树中间的人说:「这是奉耶和华差遣在遍地走来走去的。」
\VS{11}那些骑马的对站在番石榴树中间耶和华的使者说:「我们已在遍地走来走去,见全地都安息平静。」
\par }{\PP \VS{12}于是,耶和华的使者说:「万军之耶和华啊,你恼恨{\PN{耶路撒冷}}和{\PN{犹大}}的城邑已经七十年,你不施怜悯要到几时呢?」
\VS{13}耶和华就用美善的安慰话回答那与我说话的天使。
\VS{14}与我说话的天使对我说:「你要宣告说,万军之耶和华如此说:我为{\PN{耶路撒冷}}为{\PN{锡安}},心里极其火热。
\VS{15}我甚恼怒那安逸的列国,因我从前稍微恼怒{\ADD{我民}},他们就加害过分。
\VS{16}所以耶和华如此说:现今我回到{\PN{耶路撒冷}},仍施怜悯,我的殿必重建在其中,准绳必拉在{\PN{耶路撒冷}}之上。这是万军之耶和华说的。
\VS{17}你要再宣告说,万军之耶和华如此说:我的城邑必再丰盛发达。耶和华必再安慰{\PN{锡安}},拣选{\PN{耶路撒冷}}。」
\par }{\SH 角的异象
\par }{\PP \VS{18}我举目观看,见有四角。
\VS{19}我就问与我说话的天使说:「这是什么意思?」他回答说:「这是打散{\PN{犹大}}、{\PN{以色列}},和{\PN{耶路撒冷}}的角。」
\VS{20}耶和华又指四个匠人给我看。
\VS{21}我说:「他们来做什么呢?」他说:「这是打散{\PN{犹大}}的角,使人不敢抬头;但这些匠人来威吓列国,打掉他们的角,就是举起打散{\PN{犹大}}地的角。」

\par }\Chap{2}{\SH 准绳的异象
\par }{\PP \VerseOne{1}我又举目观看,见一人手拿准绳。
\VS{2}我说:「你往哪里去?」他对我说:「要去量{\PN{耶路撒冷}},看有多宽多长。」
\VS{3}与我说话的天使去的时候,又有一位天使迎着他来,
\VS{4}对他说:「你跑去告诉那少年人说,{\PN{耶路撒冷}}必有人居住,好像无城墙的乡村,因为人民和牲畜甚多。
\VS{5}耶和华说:我要作{\PN{耶路撒冷}}四围的火城,并要作其中的荣耀。」
\par }{\SH 呼唤被掳者归回
\par }{\PP \VS{6}耶和华说:「我从前分散你们在天的四方\FTNT{}{{\FR 2:6: }原文是犹如天的四风},现在你们要从北方之地逃回。这是耶和华说的。
\VS{7}与{\PN{巴比伦}}人同住的{\PN{锡安}}民哪,应当逃脱。
\VS{8}万军之耶和华说,在显出荣耀之后,差遣我去{\ADD{惩罚}}那掳掠你们的列国,摸你们的就是摸他眼中的瞳人。
\VS{9}看哪,我\FTNT{}{{\FR 2:9: }或译:他}要向他们抡手,他们就必作服事他们之人的掳物,你们便知道万军之耶和华差遣我了。
\VS{10}{\PN{锡安}}城啊,应当欢乐歌唱,因为我来要住在你中间。这是耶和华说的。」
\VS{11}那时,必有许多国归附耶和华,作他\FTNT{}{{\FR 2:11: }原文是我}的子民。他\FTNT{}{{\FR 2:11: }原文是我}要住在你中间,你就知道万军之耶和华差遣我到你那里去了。
\VS{12}耶和华必收回{\PN{犹大}}作他圣地的分,也必再拣选{\PN{耶路撒冷}}。
\par }{\PP \VS{13}凡有血气的都当在耶和华面前静默无声;因为他兴起,从圣所出来了。

\par }\Chap{3}{\SH 大祭司的异象
\par }{\PP \VerseOne{1}天使\FTNT{}{{\FR 3:1: }原文是他}又指给我看:大祭司{\PN{约书亚}}站在耶和华的使者面前;撒但也站在{\PN{约书亚}}的右边,与他作对。
\VS{2}耶和华向撒但说:「撒但哪,耶和华责备你!就是拣选{\PN{耶路撒冷}}的耶和华责备你!这不是从火中抽出来的一根柴吗?」
\VS{3}{\PN{约书亚}}穿着污秽的衣服站在使者面前。
\VS{4}使者吩咐站在面前的说:「你们要脱去他污秽的衣服」;又对{\PN{约书亚}}说:「我使你脱离罪孽,要给你穿上华美的衣服。」
\VS{5}我说:「要将洁净的冠冕戴在他头上。」他们就把洁净的冠冕戴在他头上,给他穿上{\ADD{华美的}}衣服,耶和华的使者在旁边站立。
\par }{\PP \VS{6}耶和华的使者告诫{\PN{约书亚}}说:
\VS{7}「万军之耶和华如此说:你若遵行我的道,谨守我的命令,你就可以管理我的家,看守我的院宇;我也要使你在这些站立的人中间来往。
\VS{8}大祭司{\PN{约书亚}}啊,你和坐在你面前的同伴都当听。(他们是作预兆的。)我必使我仆人{\ADD{
{\PN{大卫}}}}的苗裔发出。
\VS{9}看哪,我在{\PN{约书亚}}面前所立的石头,在一块石头上有七眼。万军之耶和华说:我要亲自雕刻这石头,并要在一日之间除掉这地的罪孽。
\VS{10}当那日,你们各人要请邻舍坐在葡萄树和无花果树下。这是万军之耶和华说的。」

\par }\Chap{4}{\SH 金灯台的异象
\par }{\PP \VerseOne{1}那与我说话的天使又来叫醒我,好像人睡觉被唤醒一样。
\VS{2}他问我说:「你看见了什么?」我说:「我看见了一个纯金的灯台,顶上有灯盏,灯台上有七盏灯,每盏有七个管子。
\VS{3}旁边有两棵橄榄树,一棵在灯盏的右边,一棵在灯盏的左边。」
\VS{4}我问与我说话的天使说:「主啊,这是什么意思?」
\VS{5}与我说话的天使回答我说:「你不知道这是什么意思吗?」我说:「主啊,我不知道。」
\par }{\SH  神给所罗巴伯的应许
\par }{\PP \VS{6}他对我说:「这是耶和华指示{\PN{所罗巴伯}}的。万军之耶和华说:不是倚靠势力,不是倚靠才能,乃是倚靠我的灵{\ADD{方能成事}}。
\VS{7}大山哪,你算什么呢?在{\PN{所罗巴伯}}面前,你必成为平地。他必搬出一块石头,安在{\ADD{殿}}顶上。人且大声欢呼说:『愿恩惠恩惠归与这殿\FTNT{}{{\FR 4:7: }殿:或译石}!』」
\VS{8}耶和华的话又临到我说:
\VS{9}「{\PN{所罗巴伯}}的手立了这殿的根基,他的手也必完成这工,你就知道万军之耶和华差遣我到你们这里来了。
\VS{10}谁藐视这日的事为小呢?这七{\ADD{眼}}乃是耶和华的眼睛,遍察全地,见{\PN{所罗巴伯}}手拿线铊就欢喜。」
\par }{\PP \VS{11}我又问天使说:「这灯台左右的两棵橄榄树是什么意思?」
\VS{12}我二次问他说:「这两根橄榄枝在两个流出金色{\ADD{油}}的金嘴旁边是什么意思?」
\VS{13}他对我说:「你不知道这是什么意思吗?」我说:「主啊,我不知道。」
\VS{14}他说:「这是两个受膏者站在普天下主的旁边。」

\par }\Chap{5}{\SH 飞卷的异象
\par }{\PP \VerseOne{1}我又举目观看,见有一飞行的书卷。
\VS{2}他问我说:「你看见什么?」我回答说:「我看见一飞行的书卷,长二十肘,宽十肘。」
\VS{3}他对我说:「这是发出行在遍地上的咒诅。凡偷窃的必按卷上这面的话除灭;凡起{\ADD{假}}誓的必按卷上那面的话除灭。
\VS{4}万军之耶和华说:我必使这书卷出去,进入偷窃人的家和指我名起假誓人的家,必常在他家里,连房屋带木石都毁灭了。」
\par }{\SH 量器中妇人的异象
\par }{\PP \VS{5}与我说话的天使出来,对我说:「你要举目观看,见所出来的是什么?」
\VS{6}我说:「这是什么呢?」他说:「这出来的是量器。」他又说:「这是恶人在遍地的形状。」
\VS{7}(我见有一片圆铅被举起来。)这坐在量器中的是个妇人。
\VS{8}天使说:「这是罪恶。」他就把妇人扔在量器中,将那片圆铅扔在量器的口上。
\VS{9}我又举目观看,见有两个妇人出来,在她们翅膀中有风,{\ADD{飞得甚快}},翅膀如同鹳鸟的翅膀。她们将量器抬起来,悬在天地中间。
\VS{10}我问与我说话的天使说:「她们要将量器抬到哪里去呢?」
\VS{11}他对我说:「要往{\PN{示拿}}地去,为它盖造房屋;等房屋齐备,就把它安置在自己的地方。」

\par }\Chap{6}{\SH 马车的异象
\par }{\PP \VerseOne{1}我又举目观看,见有四辆车从两山中间出来;那山是铜山。
\VS{2}第一辆车套着红马,第二辆车套着黑马。
\VS{3}第三辆车套着白马,第四辆车套着有斑点的壮马。
\VS{4}我就问与我说话的天使说:「主啊,这是什么意思?」
\VS{5}天使回答我说:「这是天的四风,是从普天下的主面前出来的。」
\VS{6}套着黑马的{\ADD{车}}往北方去,白马跟随在后;有斑点的马往南方去。
\VS{7}壮马出来,要在遍地走来走去。天使说:「你们只管在遍地走来走去。」它们就照样行了。
\VS{8}他又呼叫我说:「看哪,往北方去的已在北方安慰我的心。」
\par }{\SH 为约书亚加冕
\par }{\PP \VS{9}耶和华的话临到我说:
\VS{10}「你要从被掳之人中取{\PN{黑玳}}、{\PN{多比雅}}、{\PN{耶大雅}}的{\ADD{金银}}。这三人是从{\PN{巴比伦}}来到{\PN{西番雅}}的儿子{\PN{约西亚}}的家里。当日你要进他的家,
\VS{11}取这金银做冠冕,戴在{\PN{约撒答}}的儿子大祭司{\PN{约书亚}}的头上,
\VS{12}对他说,万军之耶和华如此说:看哪,那名称为{\PN{大卫}}苗裔的,他要在本处长起来,并要建造耶和华的殿。
\VS{13}他要建造耶和华的殿,并担负尊荣,坐在位上掌王权;又必在位上作祭司,使两职之间筹定和平。
\VS{14}这冠冕要归{\PN{希连}}\FTNT{}{{\FR 6:14: }就是黑玳}、{\PN{多比雅}}、{\PN{耶大雅}},和{\PN{西番雅}}的儿子{\PN{贤}}\FTNT{}{{\FR 6:14: }就是约西亚},放在耶和华的殿里为记念。」
\par }{\PP \VS{15}远方的人也要来建造耶和华的殿,你们就知道万军之耶和华差遣我到你们这里来。你们若留意听从耶和华—你们 神的话,{\ADD{这事}}必然成就。

\par }\Chap{7}{\SH 谴责不诚恳的禁食
\par }{\PP \VerseOne{1}{\PN{大流士}}王第四年九月,就是基斯流月初四日,耶和华的话临到{\PN{撒迦利亚}}。
\VS{2}那时{\PN{伯特利}}人已经打发{\PN{沙利色}}和{\PN{利坚米勒}},并跟从他们的人,去恳求耶和华的恩,
\VS{3}并问万军之耶和华殿中的祭司和先知说:「我历年以来,在五月间哭泣斋戒,现在还当这样行吗?」
\VS{4}万军之耶和华的话就临到我说:
\VS{5}「你要宣告国内的众民和祭司,说:『你们这七十年,在五月、七月禁食悲哀,岂是丝毫向我禁食吗?
\VS{6}你们吃喝,不是为自己吃,为自己喝吗?
\VS{7}当{\PN{耶路撒冷}}和四围的城邑有居民,正兴盛,南地高原有人居住的时候,耶和华借从前的先知所宣告的话,你们不{\ADD{当听}}吗?』」
\par }{\SH 被掳的原因是悖逆
\par }{\PP \VS{8}耶和华的话又临到{\PN{撒迦利亚}}说:
\VS{9}「万军之耶和华曾对{\ADD{你们的列祖}}如此说:『要按至理判断,各人以慈爱怜悯弟兄。
\VS{10}不可欺压寡妇、孤儿、寄居的,和贫穷人。谁都不可心里谋害弟兄。』」
\VS{11}他们却不肯听从,扭转肩头,塞耳不听,
\VS{12}使心硬如金钢石,不听律法和万军之耶和华用灵借从前的先知所说的话。故此,万军之耶和华大发烈怒。
\VS{13}万军之耶和华说:「我曾呼唤他们,他们不听;将来他们呼求我,我也不听!
\VS{14}我必以旋风吹散他们到素不认识的万国中。这样,他们的地就荒凉,甚至无人来往经过,因为他们使美好之地荒凉了。」

\par }\Chap{8}{\SH 耶和华应许复兴耶路撒冷
\par }{\PP \VerseOne{1}万军之耶和华的话临到我说:
\VS{2}「万军之耶和华如此说:我为{\PN{锡安}}心里极其火热,我为她火热,向{\ADD{她的仇敌}}发烈怒。
\VS{3}耶和华如此说:我现在回到{\PN{锡安}},要住在{\PN{耶路撒冷}}中。{\PN{耶路撒冷}}必称为诚实的城,万军之耶和华的山必称为圣山。
\VS{4}万军之耶和华如此说:将来必有年老的男女坐在{\PN{耶路撒冷}}街上,因为年纪老迈就手拿拐杖。
\VS{5}城中街上必满有男孩女孩玩耍。
\VS{6}万军之耶和华如此说:到那日,这事在余剩的民眼中看为希奇,在我眼中也看为希奇吗?这是万军之耶和华说的。
\VS{7}万军之耶和华如此说:我要从东方从西方救回我的民。
\VS{8}我要领他们来,使他们住在{\PN{耶路撒冷}}中。他们要作我的子民,我要作他们的 神,都凭诚实和公义。」
\par }{\PP \VS{9}万军之耶和华如此说:「当建造万军之耶和华的殿,立根基之日的先知所说的话,现在你们听见,应当手里强壮。
\VS{10}那日以先,人得不着雇价,牲畜也是如此;且因敌人的缘故,出入之人不得平安,乃因我使众人互相攻击。
\VS{11}但如今,我待这余剩的民必不像从前。这是万军之耶和华说的。
\VS{12}因为他们必平安撒种,葡萄树必结果子,地土必有出产,天也必降甘露。我要使这余剩的民享受这一切的福。
\VS{13}{\PN{犹大}}家和{\PN{以色列}}家啊,你们从前在列国中怎样成为可咒诅的;照样,我要拯救你们,使人称你们为有福的\FTNT{}{{\FR 8:13: }或译:使你们叫人得福}。你们不要惧怕,手要强壮。」
\par }{\PP \VS{14}万军之耶和华如此说:「你们列祖惹我发怒的时候,我怎样定意降祸,并不后悔。
\VS{15}现在我照样定意施恩与{\PN{耶路撒冷}}和{\PN{犹大}}家,你们不要惧怕。
\VS{16}你们所当行的是这样:各人与邻舍说话诚实,在城门口按至理判断,使人和睦。
\VS{17}谁都不可心里谋害邻舍,也不可喜爱起假誓,因为这些事都为我所恨恶。这是耶和华说的。」
\par }{\PP \VS{18}万军之耶和华的话临到我说:
\VS{19}「万军之耶和华如此说:四月、五月禁食的日子,七月、十月禁食的日子,必变为{\PN{犹大}}家欢喜快乐的日子和欢乐的节期;所以你们要喜爱诚实与和平。」
\par }{\PP \VS{20}万军之耶和华如此说:「将来必有列国的人和多城的居民来到。
\VS{21}这城的居民必到那城,说:『我们要快去恳求耶和华的恩,寻求万军之耶和华;我也要去。』
\VS{22}必有列邦的人和强国的民来到{\PN{耶路撒冷}}寻求万军之耶和华,恳求耶和华的恩。
\VS{23}万军之耶和华如此说:在那些日子,必有十个人从列国诸族\FTNT{}{{\FR 8:23: }原文是方言}中出来,拉住一个{\PN{犹大}}人的衣襟,说:『我们要与你们同去,因为我们听见 神与你们同在了。』」

\par }\Chap{9}{\SH 审判邻国
\par }{\PP \VerseOne{1}耶和华的默示应验在{\PN{哈得拉}}地{\PN{大马士革}}
\par }{\Q —世人和{\PN{以色列}}各支派的眼目都仰望耶和华—
\par }{\Q \VS{2}和靠近的{\PN{哈马}},并{\PN{泰尔}}、{\PN{西顿}};
\par }{\Q 因为这二城的人大有智慧。
\par }{\Q \VS{3}{\PN{泰尔}}为自己修筑保障,
\par }{\Q 积蓄银子如尘沙,
\par }{\Q 堆起精金如街上的泥土。
\par }{\Q \VS{4}主必赶出她,
\par }{\Q 打败她海上的权利;
\par }{\Q 她必被火烧灭。
\par }{\BB \par }{\Q \VS{5}{\PN{亚实基伦}}看见必惧怕;
\par }{\Q {\PN{迦萨}}看见甚痛苦;
\par }{\Q {\PN{以革伦}}因失了盼望蒙羞。
\par }{\Q {\PN{迦萨}}必不再有君王;
\par }{\Q {\PN{亚实基伦}}也不再有居民。
\par }{\Q \VS{6}私生子\FTNT{}{{\FR 9:6: }或译:外族人}必住在{\PN{亚实突}};
\par }{\Q 我必除灭{\PN{非利士}}人的骄傲。
\par }{\Q \VS{7}我必除去他口中带血之肉
\par }{\Q 和牙齿内可憎之物。
\par }{\Q 他必作为余剩的人归与我们的 神,
\par }{\Q 必在{\PN{犹大}}像族长;
\par }{\Q {\PN{以革伦}}人必如{\PN{耶布斯}}人。
\par }{\Q \VS{8}我必在我家的四围安营,
\par }{\Q 使敌军不得任意往来,
\par }{\Q 暴虐的人也不再经过,
\par }{\Q 因为我亲眼看顾{\ADD{我的家}}。
\par }{\SH 未来的君王
\par }{\Q \VS{9}{\PN{锡安}}的民哪,应当大大喜乐;
\par }{\Q {\PN{耶路撒冷}}的民哪,应当欢呼。
\par }{\Q 看哪,你的王来到你这里!
\par }{\Q 他是公义的,并且施行拯救,
\par }{\Q 谦谦和和地骑着驴,
\par }{\Q 就是骑着驴的驹子。
\par }{\Q \VS{10}我必除灭{\PN{以法莲}}的战车
\par }{\Q 和{\PN{耶路撒冷}}的战马;
\par }{\Q 争战的弓也必除灭。
\par }{\Q 他必向列国讲和平;
\par }{\Q 他的权柄必从这海管到那海,
\par }{\Q 从大河管到地极。
\par }{\SH  神子民的复兴
\par }{\Q \VS{11}{\PN{锡安}}哪,我因与你立约的血,
\par }{\Q 将你中间被掳而囚的人从无水的坑中释放出来。
\par }{\Q \VS{12}你们被囚而有指望的人都要转回保障。
\par }{\Q 我今日说明,我必加倍赐福给你们。
\par }{\Q \VS{13}我拿{\PN{犹大}}作上弦的弓;
\par }{\Q 我拿{\PN{以法莲}}为张弓的箭。
\par }{\Q {\PN{锡安}}哪,我要激发你的众子,
\par }{\Q 攻击{\PN{希腊}}\FTNT{}{{\FR 9:13: }原文是雅完}的众子,使你如勇士的刀。
\par }{\BB \par }{\Q \VS{14}耶和华必显现在他们以上;
\par }{\Q 他的箭必射出像闪电。
\par }{\Q 主耶和华必吹角,
\par }{\Q 乘南方的旋风而行。
\par }{\Q \VS{15}万军之耶和华必保护他们;
\par }{\Q 他们必吞灭{\ADD{仇敌}},践踏弹石。
\par }{\Q 他们必喝{\ADD{血}}呐喊,犹如饮酒;
\par }{\Q 他们必像盛满{\ADD{血}}的碗,
\par }{\Q 又像坛的四角满了{\ADD{血}}。
\par }{\BB \par }{\Q \VS{16}当那日,耶和华—他们的 神
\par }{\Q 必看他的民如群羊,拯救他们;
\par }{\Q 因为{\ADD{他们必像}}冠冕上的宝石,
\par }{\Q 高举在他的地以上\FTNT{}{{\FR 9:16: }或译:在他的地上发光辉}。
\par }{\Q \VS{17}他的恩慈何等大!
\par }{\Q 他的荣美何其盛!
\par }{\Q 五谷健壮少男;
\par }{\Q 新酒培养处女。

\par }\Chap{10}{\SH 耶和华应许解救
\par }{\Q \VerseOne{1}当春雨的时候,
\par }{\Q 你们要向发闪电的耶和华求雨。
\par }{\Q 他必为众人降下甘霖,
\par }{\Q 使田园生长菜蔬。
\par }{\Q \VS{2}因为,家神所言的是虚空;
\par }{\Q 卜士所见的是虚假;
\par }{\Q 做梦者所说的是假梦。
\par }{\Q 他们白白地安慰人,
\par }{\Q 所以众人如羊流离,
\par }{\Q 因无牧人就受苦。
\par }{\BB \par }{\Q \VS{3}我的怒气向牧人发作;
\par }{\Q 我必惩罚公山羊;
\par }{\Q 因我—万军之耶和华
\par }{\Q 眷顾自己的羊群,就是{\ADD{
{\PN{犹大}}}}家,
\par }{\Q 必使他们如骏马在阵上。
\par }{\Q \VS{4}房角石、钉子、争战的弓,
\par }{\Q 和一切掌权的都从他而出。
\par }{\Q \VS{5}他们必如勇士在阵上
\par }{\Q 将{\ADD{仇敌}}践踏在街上的泥土中。
\par }{\Q 他们必争战,因为耶和华与他们同在;
\par }{\Q 骑马的也必羞愧。
\par }{\BB \par }{\Q \VS{6}我要坚固{\PN{犹大}}家,拯救{\PN{约瑟}}家,
\par }{\Q 要领他们归回。
\par }{\Q 我要怜恤他们;
\par }{\Q 他们必像未曾弃绝的一样,
\par }{\Q 都因我是耶和华—他们的 神,
\par }{\Q 我必应允他们{\ADD{的祷告}}。
\par }{\Q \VS{7}{\PN{以法莲}}人必如勇士;
\par }{\Q 他们心中畅快如同喝酒;
\par }{\Q 他们的儿女必看见而快活;
\par }{\Q 他们的心必因耶和华喜乐。
\par }{\BB \par }{\Q \VS{8}我要发嘶声,聚集他们,
\par }{\Q 因我已经救赎他们。
\par }{\Q 他们的人数必加增,
\par }{\Q 如从前加增一样。
\par }{\Q \VS{9}我虽然\FTNT{}{{\FR 10:9: }或译:必}播散他们在列国中,
\par }{\Q 他们必在远方记念我。
\par }{\Q 他们与儿女都必存活,且得归回。
\par }{\Q \VS{10}我必再领他们出{\PN{埃及}}地,
\par }{\Q 招聚他们出{\PN{亚述}},
\par }{\Q 领他们到{\PN{基列}}和{\PN{黎巴嫩}};
\par }{\Q 这地尚且不够他们居住。
\par }{\Q \VS{11}耶和华必经过苦海,击打海浪,
\par }{\Q 使{\PN{尼罗河}}的深处都枯干。
\par }{\Q {\PN{亚述}}的骄傲必致卑微;
\par }{\Q {\PN{埃及}}的权柄必然灭没。
\par }{\Q \VS{12}我必使他们倚靠我,得以坚固;
\par }{\Q 一举一动必奉我的名。
\par }{\Q 这是耶和华说的。

\par }\Chap{11}{\SH 暴君的没落
\par }{\Q \VerseOne{1}{\PN{黎巴嫩}}哪,开开你的门,
\par }{\Q 任火烧灭你的香柏树。
\par }{\Q \VS{2}松树啊,应当哀号;
\par }{\Q 因为香柏树倾倒,佳美的树毁坏。
\par }{\Q {\PN{巴珊}}的橡树啊,应当哀号,
\par }{\Q 因为茂盛的树林已经倒了。
\par }{\Q \VS{3}听啊,有牧人哀号的声音,
\par }{\Q 因他们荣华{\ADD{的草场}}毁坏了。
\par }{\Q 有少壮狮子咆哮的声音,
\par }{\Q 因{\PN{约旦河}}旁的丛林荒废了。
\par }{\SH 两个牧人
\par }{\PP \VS{4}耶和华—我的 神如此说:「你—{\PN{撒迦利亚}}要牧养这将宰的群羊。
\VS{5}买他们的宰了他们,以自己为无罪;卖他们的说:『耶和华是应当称颂的,因我成为富足。』牧养他们的并不怜恤他们。
\VS{6}耶和华说:『我不再怜恤这地的居民,必将这民交给各人的邻舍和他们王的手中。他们必毁灭这地,我也不救这民脱离他们的手。』」
\par }{\PP \VS{7}于是,我牧养这将宰的群羊,就是群中最困苦的羊。我拿着两根杖,一根我称为「荣美」,一根我称为「联索」。这样,我牧养了群羊。
\VS{8}一月之内,我除灭三个牧人,因为我的心厌烦他们;他们的心也憎嫌我。
\VS{9}我就说:「我不牧养你们。要死的,由他死;要丧亡的,由他丧亡;余剩的,由他们彼此相食。」
\VS{10}我折断那称为「荣美」的杖,表明我废弃与万民所立的约。
\VS{11}当日就废弃了。这样,那些仰望我的困苦羊就知道所说的是耶和华的话。
\VS{12}我对他们说:「你们若以为美,就给我工价。不然,就罢了!」于是他们给了三十块钱作为我的工价。
\VS{13}耶和华吩咐我说:「要把众人所估定美好的价值丢给窑户。」我便将这三十块钱,在耶和华的殿中丢给窑户了。
\VS{14}我又折断称为「联索」的那根杖,表明我废弃{\PN{犹大}}与{\PN{以色列}}弟兄的情谊。
\par }{\PP \VS{15}耶和华又吩咐我说:「你再取愚昧牧人所用的器具,
\VS{16}因我要在这地兴起一个牧人。他不看顾丧亡的,不寻找分散的,不医治受伤的,也不牧养强壮的;却要吃肥{\ADD{羊}}的肉,撕裂它的蹄子。
\par }{\Q \VS{17}无用的牧人丢弃羊群有祸了!
\par }{\Q 刀必临到他的膀臂和右眼上。
\par }{\Q 他的膀臂必全然枯干;
\par }{\Q 他的右眼也必昏暗失明。」

\par }\Chap{12}{\SH 耶路撒冷将蒙拯救
\par }{\PP \VerseOne{1}耶和华论{\PN{以色列}}的默示。
\par }{\PP 铺张诸天、建立地基、造人里面之灵的耶和华说:
\VS{2}「我必使{\PN{耶路撒冷}}被围困的时候,向四围列国的民成为令人昏醉的杯;这默示也论到{\PN{犹大}}\FTNT{}{{\FR 12:2: }或译:犹大也是如此}。
\VS{3}那日,我必使{\PN{耶路撒冷}}向聚集攻击他的万民当作一块重石头;凡举起的必受重伤。
\VS{4}耶和华说:到那日,我必使一切马匹惊惶,使骑马的颠狂。我必看顾{\PN{犹大}}家,使列国的一切马匹瞎眼。
\VS{5}{\PN{犹大}}的族长必心里说:『{\PN{耶路撒冷}}的居民倚靠万军之耶和华—他们的 神,就作我们的能力。』
\par }{\PP \VS{6}「那日,我必使{\PN{犹大}}的族长如火盆在木柴中,又如火把在禾捆里;他们必左右烧灭四围列国的民。{\PN{耶路撒冷}}人必仍住本处,就是{\PN{耶路撒冷}}。
\par }{\PP \VS{7}「耶和华必先拯救{\PN{犹大}}的帐棚,免得{\PN{大卫}}家的荣耀和{\PN{耶路撒冷}}居民的荣耀胜过{\PN{犹大}}。
\VS{8}那日,耶和华必保护{\PN{耶路撒冷}}的居民。他们中间软弱的必如{\PN{大卫}};{\PN{大卫}}的家必如 神,如行在他们前面之耶和华的使者。
\VS{9}那日,我必定意灭绝来攻击{\PN{耶路撒冷}}各国的民。
\par }{\PP \VS{10}「我必将那施恩叫人恳求的灵,浇灌{\PN{大卫}}家和{\PN{耶路撒冷}}的居民。他们必仰望我\FTNT{}{{\FR 12:10: }或译:他;本节同},就是他们所扎的;必为我悲哀,如丧独生子,又为我愁苦,如丧长子。
\VS{11}那日,{\PN{耶路撒冷}}必有大大的悲哀,如{\PN{米吉多}}平原之{\PN{哈达临门}}的悲哀。
\VS{12}境内一家一家地都必悲哀。{\PN{大卫}}家,男的独在一处,女的独在一处。{\PN{拿单}}家,男的独在一处,女的独在一处。
\VS{13}{\PN{利未}}家,男的独在一处,女的独在一处。{\PN{示每}}家,男的独在一处,女的独在一处。
\VS{14}其余的各家,男的独在一处,女的独在一处。

\par }\Chap{13}{\PP \VerseOne{1}「那日,必给{\PN{大卫}}家和{\PN{耶路撒冷}}的居民开一个泉源,洗除罪恶与污秽。」
\par }{\PP \VS{2}万军之耶和华说:「那日,我必从地上除灭偶像的名,不再被人记念;也必使这地不再有{\ADD{假}}先知与污秽的灵。
\VS{3}若再有人说预言,生他的父母必对他说:『你不得存活,因为你托耶和华的名说假预言。』生他的父母在他说预言的时候,要将他刺透。
\VS{4}那日,凡作先知说预言的必因他所论的异象羞愧,不再穿毛衣哄骗人。
\VS{5}他必说:『我不是先知,我是耕地的;我从幼年作人的奴仆。』
\VS{6}必有人问他说:『你两臂中间是什么伤呢?』他必回答说:『这是我在亲友家中所受的伤。』」
\par }{\SH 牧人被杀
\par }{\Q \VS{7}万军之耶和华说:
\par }{\Q 刀剑哪,应当兴起,
\par }{\Q 攻击我的牧人和我的同伴。
\par }{\Q 击打牧人,羊就分散;
\par }{\Q 我必反手加在微小者的身上。
\par }{\Q \VS{8}耶和华说:这全地的人,
\par }{\Q 三分之二必剪除而死,
\par }{\Q 三分之一仍必存留。
\par }{\Q \VS{9}我要使这三分之一经火,
\par }{\Q 熬炼他们,如熬炼银子;
\par }{\Q 试炼他们,如试炼金子。
\par }{\Q 他们必求告我的名,
\par }{\Q 我必应允他们。
\par }{\Q 我要说:这是我的子民。
\par }{\Q 他们也要说:耶和华是我{\ADD{们}}的 神。

\par }\Chap{14}{\SH 耶路撒冷和列国
\par }{\PP \VerseOne{1}耶和华的日子临近,你的财物必被抢掠,在你中间分散。
\VS{2}因为我必聚集万国与{\PN{耶路撒冷}}争战,城必被攻取,房屋被抢夺,妇女被玷污,城中的民一半被掳去;剩下的民仍在城中,不致剪除。
\VS{3}那时,耶和华必出去与那些国争战,好像从前争战一样。
\VS{4}那日,他的脚必站在{\PN{耶路撒冷}}前面朝东的{\PN{橄榄山}}上。这山必从中间分裂,自东至西成为极大的谷。山的一半向北挪移,一半向南挪移。
\VS{5}你们要从我山的谷中逃跑,因为山谷必延到{\PN{亚萨}}。你们逃跑,必如{\PN{犹大}}王{\PN{乌西雅}}年间的人逃避大地震一样。耶和华—我的 神必降临,有一切圣者同来。
\par }{\PP \VS{6}那日,必没有光,三光必退缩。
\VS{7}那日,必是耶和华所知道的,不是白昼,也不是黑夜,到了晚上才有光明。
\par }{\PP \VS{8}那日,必有活水从{\PN{耶路撒冷}}出来,一半往东海流,一半往西海流;冬夏都是如此。
\par }{\PP \VS{9}耶和华必作全地的王。那日耶和华必为独一无二的,他的名也是独一无二的。
\VS{10}全地,从{\PN{迦巴}}直到{\PN{耶路撒冷}}南方的{\PN{临门}},要变为{\PN{亚拉巴}}。{\PN{耶路撒冷}}必仍居高位,就是从{\PN{便雅悯门}}到第一门之处,又到{\PN{角门}},并从{\PN{哈楠业楼}},直到王的酒榨。
\VS{11}人必住在其中,不再有咒诅。{\PN{耶路撒冷}}人必安然居住。
\par }{\PP \VS{12}耶和华用灾殃攻击那与{\PN{耶路撒冷}}争战的列国人,必是这样:他们两脚站立的时候,肉必消没,眼在眶中干瘪,舌在口中溃烂。
\VS{13}那日,耶和华必使他们大大扰乱。他们各人彼此揪住,举手攻击。
\VS{14}{\PN{犹大}}也必在{\PN{耶路撒冷}}争战。那时四围各国的财物,就是许多金银衣服,必被收聚。
\VS{15}那临到马匹、骡子、骆驼、驴,和营中一切牲畜的灾殃是与那灾殃一般。
\par }{\PP \VS{16}所有来攻击{\PN{耶路撒冷}}列国中剩下的人,必年年上来敬拜{\ADD{大}}君王—万军之耶和华,并守住棚节。
\VS{17}地上{\ADD{万}}族中,凡不上{\PN{耶路撒冷}}敬拜{\ADD{大}}君王—万军之耶和华的,必无雨降在他们的地上。
\VS{18}{\PN{埃及}}族若不上来,{\ADD{雨}}也不降在他们的地上;凡不上来守住棚节的列国人,耶和华也必用这灾攻击他们。
\VS{19}这就是{\PN{埃及}}的刑罚和那不上来守住棚节之列国的刑罚。
\par }{\PP \VS{20}当那日,马的铃铛上必有「归耶和华为圣」的这句话。耶和华殿内的锅必如祭坛前的碗一样。
\VS{21}凡{\PN{耶路撒冷}}和{\PN{犹大}}的锅都必归万军之耶和华为圣。凡献祭的都必来取这锅,煮{\ADD{肉}}在其中。当那日,在万军之耶和华的殿中必不再有{\PN{迦南}}人。
\par }