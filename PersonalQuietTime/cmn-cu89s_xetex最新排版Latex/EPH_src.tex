\NormalFont\ShortTitle{以弗所书}
{\MT 以弗所书

\par }\ChapOne{1}{\SH 问候
\par }{\PP \VerseOne{1}奉 神旨意,作基督耶稣使徒的{\PN{保罗}},写信给在{\PN{以弗所}}的圣徒,就是在基督耶稣里有忠心的人。
\VS{2}愿恩惠、平安从 神我们的父和主耶稣基督归与你们!
\par }{\SH 基督里的属灵福气
\par }{\PP \VS{3}愿颂赞归与我们主耶稣基督的父 神!他在基督里曾赐给我们天上各样属灵的福气:
\VS{4}就如 神从创立世界以前,在基督里拣选了我们,使我们在他面前成为圣洁,无有瑕疵;
\VS{5}又因爱我们,就按着自己的意旨所喜悦的,预定我们借着耶稣基督得儿子的名分,
\VS{6}使他荣耀的恩典得着称赞;这恩典是他在爱子里所赐给我们的。
\VS{7}我们借这爱子的血得蒙救赎,过犯得以赦免,乃是照他丰富的恩典。
\VS{8}这恩典是 神用诸般智慧聪明,充充足足赏给我们的;
\VS{9}都是照他自己所预定的美意,叫我们知道他旨意的奥秘,
\VS{10}要照所安排的,在日期满足的时候,使天上、地上、一切所有的都在基督里面同归于一。
\VS{11}我们也在他里面得\FTNT{}{{\FR 1:11: }得:或译成}了基业;这原是那位随己意行、做万事的,照着他旨意所预定的,
\VS{12}叫他的荣耀从我们这首先在基督里有盼望的人可以得着称赞。
\VS{13}你们既听见真理的道,就是那叫你们得救的福音,也{\ADD{信了}}基督,既然信他,就受了所应许的圣灵为印记。
\VS{14}这圣灵是我们得基业的凭据\FTNT{}{{\FR 1:14: }原文是质},直等到 {\ADD{神}}之民\FTNT{}{{\FR 1:14: }民:原文是产业}被赎,使他的荣耀得着称赞。
\par }{\SH 保罗的祷告
\par }{\PP \VS{15}因此,我既听见你们信从主耶稣,亲爱众圣徒,
\VS{16}就为你们不住地感谢 神。祷告的时候,常提到你们,
\VS{17}求我们主耶稣基督的 神,荣耀的父,将那赐人智慧和启示的灵赏给你们,使你们真知道他,
\VS{18}并且照明你们心中的眼睛,使你们知道他的恩召有何等指望,他在圣徒中得的基业有何等丰盛的荣耀;
\VS{19}并{\ADD{知道}}他向我们这信的人所显的能力是何等浩大,
\VS{20}就是照他在基督身上所运行的大能大力,使他从死里复活,叫他在天上坐在自己的右边,
\VS{21}远超过一切执政的、掌权的、有能的、主治的,和一切有名的;不但是今世的,连来世的也都超过了。
\VS{22}又将万有服在他的脚下,使他为教会作万有之首。
\VS{23}教会是他的身体,是那充满万有者所充满的。

\par }\Chap{2}{\SH 出死入生
\par }{\PP \VerseOne{1}你们死在过犯罪恶之中,{\ADD{他叫你们活过来}}。
\VS{2}那时,你们在其中行事为人,随从今世的风俗,顺服空中掌权者的首领,就是现今在悖逆之子心中运行的{\ADD{邪}}灵。
\VS{3}我们从前也都在他们中间,放纵肉体的私欲,随着肉体和心中所喜好的去行,本为可怒之子,和别人一样。
\VS{4}然而, 神既有丰富的怜悯,因他爱我们的大爱,
\VS{5}当我们死在过犯中的时候,便叫我们与基督一同活过来。你们得救是本乎恩。
\VS{6}他又叫我们与基督耶稣一同复活,一同坐在天上,
\VS{7}要将他极丰富的恩典,就是他在基督耶稣里向我们所施的恩慈,显明给后来的世代看。
\VS{8}你们得救是本乎恩,也因着信;这并不是出于自己,乃是 神所赐的;
\VS{9}也不是出于行为,免得有人自夸。
\VS{10}我们原是他的工作,在基督耶稣里造成的,为要叫我们行善,就是 神所预备叫我们行的。
\par }{\SH 在基督里合一
\par }{\PP \VS{11}所以你们应当记念:你们从前按肉体是外邦人,是称为没受割礼的;这名原是那些凭人手在肉身上称为受割礼之人所起的。
\VS{12}那时,你们与基督无关,在{\PN{以色列}}国民以外,在所应许的诸约上是局外人,并且{\ADD{活}}在世上没有指望,没有 神。
\VS{13}你们从前远离 {\ADD{神}}的人,如今却在基督耶稣里,靠着他的血,已经得亲近了。
\par }{\PP \VS{14}因他使我们和睦\FTNT{}{{\FR 2:14: }原文是因他是我们的和睦},将两下合而为一,拆毁了中间隔断的墙;
\VS{15}而且以自己的身体废掉冤仇,就是那{\ADD{记}}在律法上的规条,为要将两下借着自己造成一个新人,{\ADD{如此}}便成就了和睦。
\VS{16}既在十字架上灭了冤仇,便借这十字架使两下归为一体,与 神和好了,
\VS{17}并且来传和平的福音给你们远处的人,也给那近处的人。
\VS{18}因为我们两下借着他被一个{\ADD{圣}}灵所感,得以进到父面前。
\par }{\PP \VS{19}这样,你们不再作外人和客旅,是与圣徒同国,是 神家里的人了;
\VS{20}并且被建造在使徒和先知的根基上,有基督耶稣自己为房角石,
\VS{21}各\FTNT{}{{\FR 2:21: }或译:全}房靠他联络得合式,渐渐成为主的圣殿。
\VS{22}你们也靠他同被建造,成为 神借着圣灵居住的所在。

\par }\Chap{3}{\SH 保罗奉差向外邦人传福音
\par }{\PP \VerseOne{1}因此,我—{\PN{保罗}}为你们外邦人作了基督耶稣被囚的,{\ADD{替你们祈祷}}\FTNT{}{{\FR 3:1: }此句是照对十四节所加}。
\VS{2}谅必你们曾听见 神赐恩给我,将关切你们的职分托付我,
\VS{3}用启示使我知道{\ADD{福音的}}奥秘,正如我以前略略写过的。
\VS{4}你们念了,就能晓得我深知基督的奥秘。
\VS{5}这奥秘在以前的世代没有叫人知道,像如今借着圣灵启示他的圣使徒和先知一样。
\VS{6}{\ADD{这奥秘}}就是外邦人在基督耶稣里,借着福音,得以同为后嗣,同为一体,同蒙应许。
\VS{7}我作了这福音的执事,是照 神的恩赐,这恩赐是照他运行的大能赐给我的。
\VS{8}我本来比众圣徒中最小的还小,然而他还赐我这恩典,叫我把基督那测不透的丰富传给外邦人,
\VS{9}又使众人都明白,这历代以来隐藏在创造万物之 神里的奥秘是如何安排的,
\VS{10}为要借着教会使天上执政的、掌权的,现在得知 神百般的智慧。
\VS{11}这是照 神从万世以前,在我们主基督耶稣里所定的旨意。
\VS{12}我们因信耶稣,就在他里面放胆无惧,笃信不疑地来到 神面前。
\VS{13}所以,我求你们不要因我为你们所受的患难丧胆,这原是你们的荣耀。
\par }{\SH 明白基督的爱
\par }{\PP \VS{14}因此,我在父面前屈膝,(
\VS{15}天上地上的各\FTNT{}{{\FR 3:15: }或译:全}家,都是从他得名。)
\VS{16}求他按着他丰盛的荣耀,借着他的灵,叫你们心里的力量刚强起来,
\VS{17}使基督因你们的信,住在你们心里,叫你们的爱心有根有基,
\VS{18}能以和众圣徒一同明白基督的爱是何等长阔高深,
\VS{19}并知道这爱是过于人所能测度的,便叫 神一切所充满的,充满了你们。
\par }{\PP \VS{20}神能照着运行在我们心里的大力充充足足地成就一切,超过我们所求所想的。
\VS{21}但愿他在教会中,并在基督耶稣里,得着荣耀,直到世世代代,永永远远。阿们!

\par }\Chap{4}{\SH 身体的合一
\par }{\PP \VerseOne{1}我为主被囚的劝你们:既然蒙召,行事为人就当与蒙召的恩相称。
\VS{2}凡事谦虚、温柔、忍耐,用爱心互相宽容,
\VS{3}用和平彼此联络,竭力保守{\ADD{圣}}灵所赐合而为一的心。
\VS{4}身体只有一个,{\ADD{圣}}灵只有一个,正如你们蒙召同有一个指望。
\VS{5}一主,一信,一洗,
\VS{6}一 神,就是众人的父,超乎众人之上,贯乎众人之中,也住在众人之内。
\VS{7}我们各人蒙恩,都是照基督所量给各人的恩赐。
\VS{8}所以{\ADD{经上}}说:
\par }{\Q 他升上高天的时候,掳掠了仇敌,
\par }{\Q 将各样的恩赐赏给人。
\par }{\MM (
\VS{9}既说升上,岂不是先降在地下吗?
\VS{10}那降下的,就是远升诸天之上要充满万有的。)
\VS{11}他所赐的,有使徒,有先知,有传福音的,有牧师和教师,
\VS{12}为要成全圣徒,各尽其职,建立基督的身体,
\VS{13}直等到我们众人在真道上同归于一,认识 神的儿子,得以长大成人,满有基督长成的身量,
\VS{14}使我们不再作小孩子,中了人的诡计和欺骗的法术,被一切异教之风摇动,飘来飘去,就随从各样的异端;
\VS{15}惟用爱心说诚实话,凡事长进,连于元首基督,
\VS{16}全身都靠他联络得合式,百节各按各职,照着各体的功用彼此相助,便叫身体渐渐增长,在爱中建立自己。
\par }{\SH 旧人和新人
\par }{\PP \VS{17}所以我说,且在主里确实地说,你们行事不要再像外邦人存虚妄的心行事。
\VS{18}他们心地昏昧,与 神所赐的生命隔绝了,都因自己无知,心里刚硬;
\VS{19}良心既然丧尽,就放纵私欲,贪行种种的污秽。
\VS{20}你们学了基督,却不是这样。
\VS{21}如果你们听过他的道,领了他的教,学了他的真理,
\VS{22}就要脱去你们从前行为上的旧人,这旧人是因私欲的迷惑渐渐变坏的;
\VS{23}又要将你们的心志改换一新,
\VS{24}并且穿上新人;这新人是照着 神{\ADD{的形象}}造的,有真理的仁义和圣洁。
\par }{\SH 新生活的守则
\par }{\PP \VS{25}所以,你们要弃绝谎言,各人与邻舍说实话,因为我们是互相为肢体。
\VS{26}生气却不要犯罪;不可含怒到日落,
\VS{27}也不可给魔鬼留地步。
\VS{28}从前偷窃的,不要再偷;总要劳力,亲手做正经事,就可有{\ADD{余}}分给那缺少的人。
\VS{29}污秽的言语一句不可出口,只要随事说造就人的好话,叫听见的人得益处。
\VS{30}不要叫 神的圣灵担忧;你们原是受了他的印记,等候得赎的日子来到。
\VS{31}一切苦毒、恼恨、忿怒、嚷闹、毁谤,并一切的恶毒\FTNT{}{{\FR 4:31: }或译:阴毒},都当从你们中间除掉;
\VS{32}并要以恩慈相待,存怜悯的心,彼此饶恕,正如 神在基督里饶恕了你们一样。

\par }\Chap{5}{\PP \VerseOne{1}所以,你们该效法 神,好像蒙慈爱的儿女一样。
\VS{2}也要凭爱心行事,正如基督爱我们,为我们舍了自己,当作馨香的供物和祭物,献与 神。
\VS{3}至于淫乱并一切污秽,或是贪婪,在你们中间连提都不可,方合圣徒的体统。
\VS{4}淫词、妄语,和戏笑的话都不相宜;总要说感谢的话。
\VS{5}因为你们确实地知道,无论是淫乱的,是污秽的,是有贪心的,在基督和 神的国里都是无分的。有贪心的,就与拜偶像的一样。
\par }{\SH 行事为人当像光明的子女
\par }{\PP \VS{6}不要被人虚浮的话欺哄;因这些事, 神的忿怒必临到那悖逆之子。
\VS{7}所以,你们不要与他们同伙。
\VS{8}从前你们是暗昧的,但如今在主里面是光明的,行事为人就当像光明的子女。
\VS{9}光明所结的果子就是一切良善、公义、诚实。
\VS{10}总要察验何为主所喜悦的事。
\VS{11}那暗昧无益的事,不要与人同行,倒要责备{\ADD{行这事的人}};
\VS{12}因为他们暗中所行的,就是提起来也是可耻的。
\VS{13}凡事受了责备,就被光显明出来,因为一切能显明的就是光。
\VS{14}所以{\ADD{主}}说:
\par }{\Q 你这睡着的人当醒过来,
\par }{\Q 从死里复活!
\par }{\Q 基督就要光照你了。
\par }{\PP \VS{15}你们要谨慎行事,不要像愚昧人,当像智慧人。
\VS{16}要爱惜光阴,因为现今的世代邪恶。
\VS{17}不要作糊涂人,要明白主的旨意如何。
\VS{18}不要醉酒,酒能使人放荡;乃要被{\ADD{圣}}灵充满。
\VS{19}当用诗章、颂词、灵歌彼此对说,口唱心和地赞美主。
\VS{20}凡事要奉我们主耶稣基督的名常常感谢父 神。
\VS{21}又当存敬畏基督的心,彼此顺服。
\par }{\SH 丈夫和妻子
\par }{\PP \VS{22}你们作妻子的,{\ADD{当顺服}}自己的丈夫,如同{\ADD{顺服}}主。
\VS{23}因为丈夫是妻子的头,如同基督是教会的头;他又是{\ADD{教会}}全体的救主。
\VS{24}教会怎样顺服基督,妻子也要怎样凡事顺服丈夫。
\VS{25}你们作丈夫的,要爱你们的妻子,正如基督爱教会,为教会舍己。
\VS{26}要用水借着道把教会洗净,成为圣洁,
\VS{27}可以献给自己,作个荣耀的教会,毫无玷污、皱纹等类的病,乃是圣洁没有瑕疵的。
\VS{28}丈夫也当照样爱妻子,如同爱自己的身子;爱妻子便是爱自己了。
\VS{29}从来没有人恨恶自己的身子,总是保养顾惜,正像基督待教会一样,
\VS{30}因我们是他身上的肢体\FTNT{}{{\FR 5:30: }有古卷加:就是他的骨他的肉}。
\VS{31}为这个缘故,人要离开父母,与妻子连合,二人成为一体。
\VS{32}这是极大的奥秘,但我是指着基督和教会说的。
\VS{33}然而,你们各人都当爱妻子,如同爱自己一样。妻子也当敬重她的丈夫。

\par }\Chap{6}{\SH 儿女和父母
\par }{\PP \VerseOne{1}你们作儿女的,要在主里听从父母,这是理所当然的。
\VS{2-3}「要孝敬父母,使你得福,在世长寿。」这是第一条带应许的诫命。
\VS{4}你们作父亲的,不要惹儿女的气,只要照着主的教训和警戒养育他们。
\par }{\SH 仆人和主人
\par }{\PP \VS{5}你们作仆人的,要惧怕战兢,用诚实的心听从你们肉身的主人,好像听从基督一般。
\VS{6}不要只在眼前事奉,像是讨人喜欢的,要像基督的仆人,从心里遵行 神的旨意。
\VS{7}甘心事奉,好像服事主,不像服事人。
\VS{8}因为晓得各人所行的善事,不论是为奴的,是自主的,都必按所行的得主的赏赐。
\VS{9}你们作主人的,待仆人也是一理,不要威吓他们。因为知道,他们和你们同有一位主在天上;他并不偏待人。
\par }{\SH 与邪恶争战
\par }{\PP \VS{10}我还有末了的话:你们要靠着主,倚赖他的大能大力作刚强的人。
\VS{11}要穿戴 神{\ADD{所赐}}的全副军装,就能抵挡魔鬼的诡计。
\VS{12}因我们并不是与属血气的争战\FTNT{}{{\FR 6:12: }原文是摔跤;下同},乃是与那些执政的、掌权的、管辖这幽暗世界的,以及天空属灵气的恶魔争战。
\VS{13}所以,要拿起 神{\ADD{所赐}}的全副军装,好在磨难的日子抵挡{\ADD{仇敌}},并且成就了一切,还能站立得住。
\VS{14}所以要站稳了,用真理当作带子束腰,用公义当作护心镜遮胸,
\VS{15}又用平安的福音当作预备走路的鞋穿在脚上。
\VS{16}此外,又拿着信德当作盾牌,可以灭尽那恶者一切的火箭;
\VS{17}并戴上救恩的头盔,拿着{\ADD{圣}}灵的宝剑,就是 神的道;
\VS{18}靠着{\ADD{圣}}灵,随时多方祷告祈求;并要在此警醒不倦,为众圣徒祈求,
\VS{19}也为我祈求,使我得着口才,能以放胆开口讲明福音的奥秘,
\VS{20}(我为这福音的奥秘作了带锁链的使者,)并使我照着当尽的本分放胆讲论。
\par }{\SH 祝福
\par }{\PP \VS{21}今有所亲爱、忠心事奉主的兄弟{\PN{推基古}},他要把我的事情,并我的景况如何全告诉你们,叫你们知道。
\VS{22}我特意打发他到你们那里去,好叫你们知道我们的光景,又叫他安慰你们的心。
\par }{\PP \VS{23}愿平安、仁爱、信心从父 神和主耶稣基督归与弟兄们!
\VS{24}并愿所有诚心爱我们主耶稣基督的人都蒙恩惠!
\par }