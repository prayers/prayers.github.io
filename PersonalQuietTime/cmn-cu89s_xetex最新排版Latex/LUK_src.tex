\NormalFont\ShortTitle{路加福音}
{\MT 路加福音

\par }\ChapOne{1}{\SH 写给提阿非罗
\par }{\PP \VerseOne{1-2}{\PN{提阿非罗}}大人哪,有好些人提笔作书,述说在我们中间所成就的事,是照传道的人从起初亲眼看见又传给我们的。
\VS{3}这些事我既从起头都详细考察了,就定意要按着次序写给你,
\VS{4}使你知道所学之道都是确实的。
\par }{\SH 预言施洗约翰的诞生
\par }{\PP \VS{5}当{\PN{犹太}}王{\PN{希律}}的时候,{\PN{亚比雅}}班里有一个祭司,名叫{\PN{撒迦利亚}};他妻子是{\PN{亚伦}}的后人,名叫{\PN{伊利莎白}}。
\VS{6}他们二人在 神面前都是义人,遵行主的一切诫命礼仪,没有可指摘的,
\VS{7}只是没有孩子;因为{\PN{伊利莎白}}不生育,两个人又年纪老迈了。
\VS{8}{\PN{撒迦利亚}}按班次在 神面前供祭司的职分,
\VS{9}照祭司的规矩掣签,得进主殿烧香。
\VS{10}烧香的时候,众百姓在外面祷告。
\VS{11}有主的使者站在香坛的右边,向他显现。
\VS{12}{\PN{撒迦利亚}}看见,就惊慌害怕。
\VS{13}天使对他说:「{\PN{撒迦利亚}},不要害怕,因为你的祈祷已经被听见了。你的妻子{\PN{伊利莎白}}要给你生一个儿子,你要给他起名叫{\PN{约翰}}。
\VS{14}你必欢喜快乐;有许多人因他出世,也必喜乐。
\VS{15}他在主面前将要为大,淡酒浓酒都不喝,从母腹里就被圣灵充满了。
\VS{16}他要使许多{\PN{以色列}}人回转,归于主—他们的 神。
\VS{17}他必有{\PN{以利亚}}的心志能力,行在主的前面,叫为父的心转向儿女,叫悖逆的人转从义人的智慧,又为主预备合用的百姓。」
\VS{18}{\PN{撒迦利亚}}对天使说:「我凭着什么可知道这事呢?我已经老了,我的妻子也年纪老迈了。」
\VS{19}天使回答说:「我是站在 神面前的{\PN{加百列}},奉差而来对你说话,将这好信息报给你。
\VS{20}到了时候,这话必然应验;只因你不信,你必哑巴,不能说话,直到这事成就的日子。」
\VS{21}百姓等候{\PN{撒迦利亚}},诧异他许久在殿里。
\VS{22}及至他出来,不能和他们说话,他们就知道他在殿里见了异象;因为他直向他们打手式,竟成了哑巴。
\VS{23}他供职的日子已满,就回家去了。
\VS{24}这些日子以后,他的妻子{\PN{伊利莎白}}怀了孕,就隐藏了五个月,
\VS{25}说:「主在眷顾我的日子,这样看待我,要把我在人间的羞耻除掉。」
\par }{\SH 预言耶稣的降生
\par }{\PP \VS{26}到了第六个月,天使{\PN{加百列}}奉 神的差遣往{\PN{加利利}}的一座城去(这城名叫{\PN{拿撒勒}}),
\VS{27}到一个童女那里,是已经许配{\PN{大卫}}家的一个人,名叫{\PN{约瑟}}。童女的名字叫{\PN{马利亚}};
\VS{28}天使进去,对她说:「蒙大恩的女子,我问你安,主和你同在了!」
\VS{29}{\PN{马利亚}}因这话就很惊慌,又反复思想这样问安是什么意思。
\VS{30}天使对她说:「{\PN{马利亚}},不要怕!你在 神面前已经蒙恩了。
\VS{31}你要怀孕生子,可以给他起名叫耶稣。
\VS{32}他要为大,称为至高者的儿子;主 神要把他祖{\PN{大卫}}的位给他。
\VS{33}他要作{\PN{雅各}}家的王,直到永远;他的国也没有穷尽。」
\VS{34}{\PN{马利亚}}对天使说:「我没有出嫁,怎么有这事呢?」
\VS{35}天使回答说:「圣灵要临到你身上,至高者的能力要荫庇你,因此所要生的圣者必称为 神的儿子\FTNT{}{{\FR 1:35: }或译:所要生的,必称为圣,称为 神的儿子}。
\VS{36}况且你的亲戚{\PN{伊利莎白}},在年老的时候也怀了男胎,就是那素来称为不生育的,现在有孕六个月了。
\VS{37}因为,出于 神的话,没有一句不带能力的。」
\VS{38}{\PN{马利亚}}说:「我是主的使女,情愿照你的话成就在我身上。」天使就离开她去了。
\par }{\SH 马利亚访问伊利莎白
\par }{\PP \VS{39}那时候,{\PN{马利亚}}起身,急忙往山地里去,来到{\PN{犹大}}的一座城;
\VS{40}进了{\PN{撒迦利亚}}的家,问{\PN{伊利莎白}}安。
\VS{41}{\PN{伊利莎白}}一听{\PN{马利亚}}问安,所怀的胎就在腹里跳动。{\PN{伊利莎白}}且被圣灵充满,
\VS{42}高声喊着说:「你在妇女中是有福的!你所怀的胎也是有福的!
\VS{43}我主的母到我这里来,这是从哪里得的呢?
\VS{44}因为你问安的声音一入我耳,我腹里的胎就欢喜跳动。
\VS{45}这相信的女子是有福的!因为主对她所说的话都要应验。」
\par }{\SH 马利亚的尊主颂
\par }{\PP \VS{46}{\PN{马利亚}}说:
\par }{\Q 我心尊主为大;
\par }{\Q \VS{47}我灵以 神我的救主为乐;
\par }{\Q \VS{48}因为他顾念他使女的卑微;
\par }{\Q 从今以后,
\par }{\Q 万代要称我有福。
\par }{\Q \VS{49}那有权能的,为我成就了大事;
\par }{\Q 他的名为圣。
\par }{\Q \VS{50}他怜悯敬畏他的人,
\par }{\Q 直到世世代代。
\par }{\Q \VS{51}他用膀臂施展大能;
\par }{\Q 那狂傲的人正心里妄想就被他赶散了。
\par }{\Q \VS{52}他叫有权柄的失位,
\par }{\Q 叫卑贱的升高;
\par }{\Q \VS{53}叫饥饿的得饱美食,
\par }{\Q 叫富足的空手回去。
\par }{\Q \VS{54}他扶助了他的仆人{\PN{以色列}},
\par }{\Q \VS{55}为要记念{\PN{亚伯拉罕}}和他的后裔,
\par }{\Q 施怜悯直到永远,
\par }{\Q 正如从前对我们列祖所说的话。
\par }{\PP \VS{56}{\PN{马利亚}}和{\PN{伊利莎白}}同住,约有三个月,就回家去了。
\par }{\SH 施洗约翰的出生
\par }{\PP \VS{57}{\PN{伊利莎白}}的产期到了,就生了一个儿子。
\VS{58}邻里亲族听见主向她大施怜悯,就和她一同欢乐。
\VS{59}到了第八日,他们来要给孩子行割礼,并要照他父亲的名字叫他{\PN{撒迦利亚}}。
\VS{60}他母亲说:「不可!要叫他{\PN{约翰}}。」
\VS{61}他们说:「你亲族中没有叫这名字的。」
\VS{62}他们就向他父亲打手式,问他要叫这孩子什么名字。
\VS{63}他要了一块写字的板,就写上,说:「他的名字是{\PN{约翰}}。」他们便都希奇。
\VS{64}{\PN{撒迦利亚}}的口立时开了,舌头也{\ADD{舒展了}},就说出话来,称颂 神。
\VS{65}周围居住的人都惧怕;这一切的事就传遍了{\PN{犹太}}的山地。
\VS{66}凡听见的人都将这事放在心里,说:「这个孩子将来怎么样呢?因为有主与他同在。」
\par }{\SH 撒迦利亚的预言
\par }{\PP \VS{67}他父亲{\PN{撒迦利亚}}被圣灵充满了,就预言说:
\par }{\Q \VS{68}主—{\PN{以色列}}的 神是应当称颂的!
\par }{\Q 因他眷顾他的百姓,为他们施行救赎,
\par }{\Q \VS{69}在他仆人{\PN{大卫}}家中,
\par }{\Q 为我们兴起了拯救的角,
\par }{\Q \VS{70}正如主借着从创世以来圣先知的口所说的话,
\par }{\Q \VS{71}拯救我们脱离仇敌
\par }{\Q 和一切恨我们之人的手,
\par }{\Q \VS{72}向我们列祖施怜悯,
\par }{\Q 记念他的圣约—
\par }{\Q \VS{73}就是他对我们祖宗{\PN{亚伯拉罕}}所起的誓—
\par }{\Q \VS{74}叫我们既从仇敌手中被救出来,
\par }{\Q \VS{75}就可以终身在他面前,
\par }{\Q 坦然无惧地用圣洁、公义事奉他。
\par }{\Q \VS{76}孩子啊!你要称为至高者的先知;
\par }{\Q 因为你要行在主的前面,
\par }{\Q 预备他的道路,
\par }{\Q \VS{77}叫他的百姓因罪得赦,
\par }{\Q 就知道救恩。
\par }{\Q \VS{78}因我们 神怜悯的心肠,
\par }{\Q 叫清晨的日光从高天临到我们,
\par }{\Q \VS{79}要照亮坐在黑暗中死荫里的人,
\par }{\Q 把我们的脚引到平安的路上。
\par }{\PP \VS{80}那孩子渐渐长大,心灵强健,住在旷野,直到他显明在{\PN{以色列}}人面前的日子。

\par }\Chap{2}{\SH 耶稣降生
\par }{\R (太1·18—25)
\par }{\PP \VerseOne{1}当那些日子,凯撒{\PN{奥古斯都}}有旨意下来,叫天下人民都报名上册。
\VS{2}这是{\PN{居里扭}}作{\PN{叙利亚}}巡抚的时候,头一次行报名上册的事。
\VS{3}众人各归各城,报名上册。
\VS{4}{\PN{约瑟}}也从{\PN{加利利}}的{\PN{拿撒勒}}城上{\PN{犹太}}去,到了{\PN{大卫}}的城,名叫{\PN{伯利恒}},因他本是{\PN{大卫}}一族一家的人,
\VS{5}要和他所聘之妻{\PN{马利亚}}一同报名上册。那时{\PN{马利亚}}的身孕已经重了。
\VS{6}他们在那里的时候,{\PN{马利亚}}的产期到了,
\VS{7}就生了头胎的儿子,用布包起来,放在马槽里,因为客店里没有地方。
\par }{\SH 天使报喜讯给牧羊的人
\par }{\PP \VS{8}在{\PN{伯利恒}}之野地里有牧羊的人,夜间按着更次看守羊群。
\VS{9}有主的使者站在他们旁边,主的荣光四面照着他们;牧羊的人就甚惧怕。
\VS{10}那天使对他们说:「不要惧怕!我报给你们大喜的信息,是关乎万民的;
\VS{11}因今天在{\PN{大卫}}的城里,为你们生了救主,就是主基督。
\VS{12}你们要看见一个婴孩,包着布,卧在{\ADD{马}}槽里,那就是记号了。」
\VS{13}忽然,有一大队天兵同那天使赞美 神说:
\par }{\Q \VS{14}在至高之处荣耀归与 神!
\par }{\Q 在地上平安归与他所喜悦的人\FTNT{}{{\FR 2:14: }有古卷:喜悦归与人}!
\par }{\PP \VS{15}众天使离开他们,升天去了。牧羊的人彼此说:「我们往{\PN{伯利恒}}去,看看所成的事,就是主所指示我们的。」
\VS{16}他们急忙去了,就寻见{\PN{马利亚}}和{\PN{约瑟}},又有那婴孩卧在{\ADD{马}}槽里;
\VS{17}既然看见,就把天使论这孩子的话传开了。
\VS{18}凡听见的,就诧异牧羊之人对他们所说的话。
\VS{19}{\PN{马利亚}}却把这一切的事存在心里,反复思想。
\VS{20}牧羊的人回去了,因所听见所看见的一切事,正如{\ADD{天使}}向他们所说的,就归荣耀与 神,赞美他。
\par }{\PP \VS{21}满了八天,就给孩子行割礼,与他起名叫耶稣;这就是没有成胎以前,天使所起的名。
\par }{\SH 在圣殿里献与主
\par }{\PP \VS{22}按{\PN{摩西}}律法满了洁净的日子,他们带着孩子上{\PN{耶路撒冷}}去,要把他献与主,(
\VS{23}正如主的律法上所记:「凡头生的男子必称圣归主。」)
\VS{24}又要照主的律法上所说,或用一对斑鸠,或用两只雏鸽献祭。
\par }{\PP \VS{25}在{\PN{耶路撒冷}}有一个人,名叫{\PN{西面}};这人又公义又虔诚,素常盼望{\PN{以色列}}的安慰{\ADD{者}}来到,又有圣灵在他身上。
\VS{26}他得了圣灵的启示,知道自己未死以前,必看见主所立的基督。
\VS{27}他受了{\ADD{圣}}灵的感动,进入圣殿,正遇见耶稣的父母抱着孩子进来,要照律法的规矩办理。
\VS{28}{\PN{西面}}就用手接过他来,称颂 神说:
\par }{\Q \VS{29}主啊!如今可以照你的话,
\par }{\Q 释放仆人安然去世;
\par }{\Q \VS{30}因为我的眼睛已经看见你的救恩—
\par }{\Q \VS{31}就是你在万民面前所预备的:
\par }{\Q \VS{32}是照亮外邦人的光,
\par }{\Q 又是你民{\PN{以色列}}的荣耀。
\par }{\PP \VS{33}孩子的父母因这论耶稣的话就希奇。
\VS{34-35}{\PN{西面}}给他们祝福,又对孩子的母亲{\PN{马利亚}}说:「这{\ADD{孩子}}被立,是要叫{\PN{以色列}}中许多人跌倒,许多人兴起;又要作毁谤的话柄,叫许多人心里的意念显露出来;你自己的心也要被刀刺透。」
\par }{\PP \VS{36}又有女先知,名叫{\PN{亚拿}},是{\PN{亚设}}支派{\PN{法内力}}的女儿,年纪已经老迈,从作童女出嫁的时候,同丈夫住了七年就寡居了,
\VS{37}现在已经八十四岁\FTNT{}{{\FR 2:37: }或译:就寡居了八十四年},并不离开圣殿,禁食祈求,昼夜事奉 神。
\VS{38}正当那时,她进前来称谢 神,将孩子的事对一切盼望{\PN{耶路撒冷}}得救赎的人讲说。
\par }{\SH 回拿撒勒
\par }{\PP \VS{39}{\PN{约瑟}}和{\PN{马利亚}}照主的律法办完了一切的事,就回{\PN{加利利}},到自己的城{\PN{拿撒勒}}去了。
\VS{40}孩子渐渐长大,强健起来,充满智慧,又有 神的恩在他身上。
\par }{\SH 孩童耶稣在圣殿里
\par }{\PP \VS{41}每年到逾越节,他父母就上{\PN{耶路撒冷}}去。
\VS{42}当他十二岁的时候,他们按着节期的规矩上去。
\VS{43}守满了节期,他们回去,孩童耶稣仍旧在{\PN{耶路撒冷}}。他的父母并不知道,
\VS{44}以为他在同行的人中间,走了一天的路程,就在亲族和熟识的人中找他,
\VS{45}既找不着,就回{\PN{耶路撒冷}}去找他。
\VS{46}过了三天,就遇见他在殿里,坐在教师中间,一面听,一面问。
\VS{47}凡听见他的,都希奇他的聪明和他的应对。
\VS{48}他父母看见就很希奇。他母亲对他说:「我儿!为什么向我们这样行呢?看哪,你父亲和我伤心来找你!」
\VS{49}耶稣说:「为什么找我呢?岂不知我应当以我父的事为念吗\FTNT{}{{\FR 2:49: }或译:岂不知我应当在我父的家里吗}?」
\VS{50}他所说的这话,他们不明白。
\VS{51}他就同他们下去,回到{\PN{拿撒勒}},并且顺从他们。他母亲把这一切的事都存在心里。
\VS{52}耶稣的智慧和身量\FTNT{}{{\FR 2:52: }或译:年纪},并 神和人喜爱他的心,都一齐增长。

\par }\Chap{3}{\SH 施洗约翰传道
\par }{\R (太3·1—12;可1·1—8;约1·19—28)
\par }{\PP \VerseOne{1}凯撒{\PN{提庇留}}在位第十五年,{\PN{本丢·彼拉多}}作{\PN{犹太}}巡抚,{\PN{希律}}作{\PN{加利利}}分封的王,他兄弟{\PN{腓力}}作{\PN{以土利亚}}和{\PN{特拉可尼}}地方分封的王,{\PN{吕撒聂}}作{\PN{亚比利尼}}分封的王,
\VS{2}{\PN{亚那}}和{\PN{该亚法}}作大祭司。那时,{\PN{撒迦利亚}}的儿子{\PN{约翰}}在旷野里, 神的话临到他。
\VS{3}他就来到{\PN{约旦河}}一带地方,宣讲悔改的洗礼,使罪得赦。
\VS{4}正如先知{\PN{以赛亚}}书上所记的话,说:
\par }{\Q 在旷野有人声喊着说:
\par }{\Q 预备主的道,
\par }{\Q 修直他的路!
\par }{\Q \VS{5}一切山洼都要填满;
\par }{\Q 大小山冈都要削平!
\par }{\Q 弯弯曲曲的地方要改为正直;
\par }{\Q 高高低低的道路要改为平坦!
\par }{\Q \VS{6}凡有血气的,都要见 神的救恩!
\par }{\PP \VS{7}{\PN{约翰}}对那出来要受他洗的众人说:「毒蛇的种类!谁指示你们逃避将来的忿怒呢?
\VS{8}你们要结出果子来,与悔改的心相称。不要自己心里说:『有{\PN{亚伯拉罕}}为我们的祖宗。』我告诉你们, 神能从这些石头中,给{\PN{亚伯拉罕}}兴起子孙来。
\VS{9}现在斧子已经放在树根上,凡不结好果子的树就砍下来,丢在火里。」
\VS{10}众人问他说:「这样,我们当做什么呢?」
\VS{11}{\PN{约翰}}回答说:「有两件衣裳的,就分给那没有的;有食物的,也当这样行。」
\VS{12}又有税吏来要受洗,问他说:「夫子,我们当做什么呢?」
\VS{13}{\PN{约翰}}说:「除了例定的数目,不要多取。」
\VS{14}又有兵丁问他说:「我们当做什么呢?」{\PN{约翰}}说:「不要以强暴待人,也不要讹诈人,自己有钱粮就当知足。」
\par }{\PP \VS{15}百姓指望{\ADD{基督来}}的时候,人都心里猜疑,或者{\PN{约翰}}是基督。
\VS{16}{\PN{约翰}}说:「我是用水给你们施洗,但有一位能力比我更大的要来,我就是给他解鞋带也不配。他要用圣灵与火给你们施洗。
\VS{17}他手里拿着簸箕,要扬净他的场,把麦子收在仓里,把糠用不灭的火烧尽了。」
\VS{18}{\PN{约翰}}又用许多别的话劝百姓,向他们传福音。
\VS{19}只是分封的王{\PN{希律}},因他兄弟之妻{\PN{希罗底}}的缘故,并因他所行的一切恶事,受了{\PN{约翰}}的责备;
\VS{20}又另外添了一件,就是把{\PN{约翰}}收在监里。
\par }{\SH 耶稣受洗
\par }{\R (太3·13—17;可1·9—11)
\par }{\PP \VS{21}众百姓都受了洗,耶稣也受了洗。正祷告的时候,天就开了,
\VS{22}圣灵降临在他身上,形状仿佛鸽子;又有声音从天上来,说:「你是我的爱子,我喜悦你。」
\par }{\SH 耶稣的家谱
\par }{\R (太1·1—17)
\par }{\PP \VS{23}耶稣开头{\ADD{传道}},年纪约有三十岁。依人看来,他是{\PN{约瑟}}的儿子;{\PN{约瑟}}是{\PN{希里}}的儿子;
\VS{24}{\PN{希里}}是{\PN{玛塔}}的儿子;{\PN{玛塔}}是{\PN{利未}}的儿子;{\PN{利未}}是{\PN{麦基}}的儿子;{\PN{麦基}}是{\PN{雅拿}}的儿子;{\PN{雅拿}}是{\PN{约瑟}}的儿子;
\VS{25}{\PN{约瑟}}是{\PN{玛他提亚}}的儿子;{\PN{玛他提亚}}是{\PN{亚摩斯}}的儿子;{\PN{亚摩斯}}是{\PN{拿鸿}}的儿子;{\PN{拿鸿}}是{\PN{以斯利}}的儿子;{\PN{以斯利}}是{\PN{拿该}}的儿子;
\VS{26}{\PN{拿该}}是{\PN{玛押}}的儿子;{\PN{玛押}}是{\PN{玛他提亚}}的儿子;{\PN{玛他提亚}}是{\PN{西美}}的儿子;{\PN{西美}}是{\PN{约瑟}}的儿子;{\PN{约瑟}}是{\PN{犹大}}的儿子;{\PN{犹大}}是{\PN{约亚拿}}的儿子;
\VS{27}{\PN{约亚拿}}是{\PN{利撒}}的儿子;{\PN{利撒}}是{\PN{所罗巴伯}}的儿子;{\PN{所罗巴伯}}是{\PN{撒拉铁}}的儿子;{\PN{撒拉铁}}是{\PN{尼利}}的儿子;{\PN{尼利}}是{\PN{麦基}}的儿子;
\VS{28}{\PN{麦基}}是{\PN{亚底}}的儿子;{\PN{亚底}}是{\PN{哥桑}}的儿子;{\PN{哥桑}}是{\PN{以摩当}}的儿子;{\PN{以摩当}}是{\PN{珥}}的儿子;{\PN{珥}}是{\PN{约细}}的儿子;
\VS{29}{\PN{约细}}是{\PN{以利以谢}}的儿子;{\PN{以利以谢}}是{\PN{约令}}的儿子;{\PN{约令}}是{\PN{玛塔}}的儿子;{\PN{玛塔}}是{\PN{利未}}的儿子;
\VS{30}{\PN{利未}}是{\PN{西缅}}的儿子;{\PN{西缅}}是{\PN{犹大}}的儿子;{\PN{犹大}}是{\PN{约瑟}}的儿子;{\PN{约瑟}}是{\PN{约南}}的儿子;{\PN{约南}}是{\PN{以利亚敬}}的儿子;
\VS{31}{\PN{以利亚敬}}是{\PN{米利亚}}的儿子;{\PN{米利亚}}是{\PN{买南}}的儿子;{\PN{买南}}是{\PN{玛达他}}的儿子;{\PN{玛达他}}是{\PN{拿单}}的儿子;{\PN{拿单}}是{\PN{大卫}}的儿子;
\VS{32}{\PN{大卫}}是{\PN{耶西}}的儿子;{\PN{耶西}}是{\PN{俄备得}}的儿子;{\PN{俄备得}}是{\PN{波阿斯}}的儿子;{\PN{波阿斯}}是{\PN{撒门}}的儿子;{\PN{撒门}}是{\PN{拿顺}}的儿子;
\VS{33}{\PN{拿顺}}是{\PN{亚米拿达}}的儿子;{\PN{亚米拿达}}是{\PN{亚兰}}的儿子;{\PN{亚兰}}是{\PN{希斯仑}}的儿子;{\PN{希斯仑}}是{\PN{法勒斯}}的儿子;{\PN{法勒斯}}是{\PN{犹大}}的儿子;
\VS{34}{\PN{犹大}}是{\PN{雅各}}的儿子;{\PN{雅各}}是{\PN{以撒}}的儿子;{\PN{以撒}}是{\PN{亚伯拉罕}}的儿子;{\PN{亚伯拉罕}}是{\PN{他拉}}的儿子;{\PN{他拉}}是{\PN{拿鹤}}的儿子;
\VS{35}{\PN{拿鹤}}是{\PN{西鹿}}的儿子;{\PN{西鹿}}是{\PN{拉吴}}的儿子;{\PN{拉吴}}是{\PN{法勒}}的儿子;{\PN{法勒}}是{\PN{希伯}}的儿子;{\PN{希伯}}是{\PN{沙拉}}的儿子;
\VS{36}{\PN{沙拉}}是{\PN{该南}}的儿子;{\PN{该南}}是{\PN{亚法撒}}的儿子;{\PN{亚法撒}}是{\PN{闪}}的儿子;{\PN{闪}}是{\PN{挪亚}}的儿子;{\PN{挪亚}}是{\PN{拉麦}}的儿子;
\VS{37}{\PN{拉麦}}是{\PN{玛土撒拉}}的儿子;{\PN{玛土撒拉}}是{\PN{以诺}}的儿子;{\PN{以诺}}是{\PN{雅列}}的儿子;{\PN{雅列}}是{\PN{玛勒列}}的儿子;{\PN{玛勒列}}是该南的儿子;{\PN{该南}}是{\PN{以挪士}}的儿子;
\VS{38}{\PN{以挪士}}是{\PN{塞特}}的儿子;{\PN{塞特}}是{\PN{亚当}}的儿子;{\PN{亚当}}是 神的儿子。

\par }\Chap{4}{\SH 耶稣受试探
\par }{\R (太4·1—11;可1·12—13)
\par }{\PP \VerseOne{1}耶稣被圣灵充满,从{\PN{约旦河}}回来,圣灵将他引到旷野,
\VS{2}四十天受魔鬼的试探。那些日子没有吃什么;日子满了,他就饿了。
\VS{3}魔鬼对他说:「你若是 神的儿子,可以吩咐这块石头变成食物。」
\VS{4}耶稣回答说:「{\ADD{经上}}记着说:『人活着不是单靠食物,{\ADD{乃是靠 神口里所出的一切话}}。』」
\VS{5}魔鬼又领他上了{\ADD{高山}},霎时间把天下的万国都指给他看,
\VS{6}对他说:「这一切权柄、荣华,我都要给你,因为这原是交付我的,我愿意给谁就给谁。
\VS{7}你若在我面前下拜,这都要归你。」
\VS{8}耶稣说:「{\ADD{经上}}记着说:
\par }{\Q 当拜主—你的 神,
\par }{\Q 单要事奉他。」
\par }{\PP \VS{9}魔鬼又领他到{\PN{耶路撒冷}}去,叫他站在殿顶\FTNT{}{{\FR 4:9: }顶:原文是翅}上,对他说:「你若是 神的儿子,可以从这里跳下去;
\VS{10}因为{\ADD{经上}}记着说:
\par }{\Q 主要为你吩咐他的使者保护你;
\par }{\Q \VS{11}他们要用手托着你,
\par }{\Q 免得你的脚碰在石头上。」
\par }{\PP \VS{12}耶稣对他说:「{\ADD{经上}}说:『不可试探主—你的 神。』」
\VS{13}魔鬼用完了各样的试探,就暂时离开耶稣。
\par }{\SH 开始在加利利传道
\par }{\R (太4·12—17;可1·14—15)
\par }{\PP \VS{14}耶稣满有圣灵的能力,回到{\PN{加利利}};他的名声就传遍了四方。
\VS{15}他在各会堂里教训人,众人都称赞他。
\par }{\SH 拿撒勒人厌弃耶稣
\par }{\R (太13·53—58;可6·1—6)
\par }{\PP \VS{16}耶稣来到{\PN{拿撒勒}},就是他长大的地方。在安息日,照他平常的规矩进了会堂,站起来要念{\ADD{圣经}}。
\VS{17}有人把先知{\PN{以赛亚}}的书交给他,他就打开,找到一处写着说:
\par }{\Q \VS{18}主的灵在我身上,
\par }{\Q 因为他用膏膏我,
\par }{\Q 叫我传福音给贫穷的人;
\par }{\Q 差遣我报告:
\par }{\Q 被掳的得释放,
\par }{\Q 瞎眼的得看见,
\par }{\Q 叫那受压制的得自由,
\par }{\Q \VS{19}报告 神悦纳人的禧年。
\par }{\PP \VS{20}于是把书卷起来,交还执事,就坐下。会堂里的人都定睛看他。
\VS{21}耶稣对他们说:「今天这经应验在你们耳中了。」
\VS{22}众人都称赞他,并希奇他口中所出的恩言;又说:「这不是{\PN{约瑟}}的儿子吗?」
\VS{23}耶稣对他们说:「你们必引这俗语向我说:『医生,你医治自己吧!我们听见你在{\PN{迦百农}}所行的事,也当行在你自己家乡里』」;
\VS{24}又说:「我实在告诉你们,没有先知在自己家乡被人悦纳的。
\VS{25}我对你们说实话,当{\PN{以利亚}}的时候,天闭塞了三年零六个月,遍地有大饥荒,那时,{\PN{以色列}}中有许多寡妇,
\VS{26}{\PN{以利亚}}并没有奉差往她们一个人那里去,只奉差往{\PN{西顿}}的{\PN{撒勒法}}一个寡妇那里去。
\VS{27}先知{\PN{以利沙}}的时候,{\PN{以色列}}中有许多长大麻风的,但内中除了{\PN{叙利亚}}国的{\PN{乃缦}},没有一个得洁净的。」
\VS{28}会堂里的人听见这话,都怒气满胸,
\VS{29}就起来撵他出城(他们的城造在山上);他们带他到山崖,要把他推下去。
\VS{30}他却从他们中间直行,过去了。
\par }{\SH 一个污鬼附身的人
\par }{\R (可1·21—28)
\par }{\PP \VS{31}耶稣下到{\PN{迦百农}},就是{\PN{加利利}}的一座城,在安息日教训众人。
\VS{32}他们很希奇他的教训,因为他的话里有权柄。
\VS{33}在会堂里有一个人,被污鬼的精气附着,大声喊叫说:
\VS{34}「唉!{\PN{拿撒勒}}的耶稣,我们与你有什么相干?你来灭我们吗?我知道你是谁,乃是 神的圣者。」
\VS{35}耶稣责备他说:「不要作声,从这人身上出来吧!」鬼把那人摔倒在众人中间,就出来了,却也没有害他。
\VS{36}众人都惊讶,彼此对问说:「这是什么道理呢?因为他用权柄能力吩咐污鬼,污鬼就出来。」
\VS{37}于是耶稣的名声传遍了周围地方。
\par }{\SH 治好许多病人
\par }{\R (太8·14—17;可1·29—34)
\par }{\PP \VS{38}耶稣出了会堂,进了{\PN{西门}}的家。{\PN{西门}}的岳母害热病甚重,有人为她求耶稣。
\VS{39}耶稣站在她旁边,斥责那热病,热就退了。她立刻起来服事他们。
\VS{40}日落的时候,凡有病人的,不论害什么病,都带到耶稣那里。耶稣按手在他们各人身上,医好他们。
\VS{41}又有鬼从好些人身上出来,喊着说:「你是 神的儿子。」耶稣斥责他们,不许他们说话,因为他们知道他是基督。
\par }{\SH 在加利利各会堂传道
\par }{\R (可1·35—39)
\par }{\PP \VS{42}天亮的时候,耶稣出来,走到旷野地方。众人去找他,到了他那里,要留住他,不要他离开他们。
\VS{43}但耶稣对他们说:「我也必须在别城传 神国的福音,因我奉差原是为此。」
\VS{44}于是耶稣在{\PN{加利利}}的各会堂传道。

\par }\Chap{5}{\SH 耶稣呼召头几个门徒
\par }{\R (太4·18—22;可1·16—20)
\par }{\PP \VerseOne{1}耶稣站在{\PN{革尼撒勒}}湖边,众人拥挤他,要听 神的道。
\VS{2}他见有两只船湾在湖边;打鱼的人却离开船洗网去了。
\VS{3}有一只船是{\PN{西门}}的,耶稣就上去,请他把船撑开,稍微离岸,就坐下,从船上教训众人。
\VS{4}讲完了,对{\PN{西门}}说:「把船开到水深之处,下网打鱼。」
\VS{5}{\PN{西门}}说:「夫子,我们整夜劳力,并没有打着什么。但依从你的话,我就下网。」
\VS{6}他们下了网,就圈住许多鱼,网险些裂开,
\VS{7}便招呼那只船上的同伴来帮助。他们就来,把鱼装满了两只船,甚至船要沉下去。
\VS{8}{\PN{西门·彼得}}看见,就俯伏在耶稣膝前,说:「主啊,离开我,我是个罪人!」
\VS{9}他和一切同在的人都惊讶这一网所打的鱼。
\VS{10}他的伙伴{\PN{西庇太}}的儿子{\PN{雅各}}、{\PN{约翰}},也是这样。耶稣对{\PN{西门}}说:「不要怕!从今以后,你要得人了。」
\VS{11}他们把两只船拢了岸,就撇下所有的,跟从了耶稣。
\par }{\SH 洁净长大麻风的人
\par }{\R (太8·1—4;可1·40—45)
\par }{\PP \VS{12}有一回,耶稣在一个城里,有人满身长了大麻风,看见他,就俯伏在地,求他说:「主若肯,必能叫我洁净了。」
\VS{13}耶稣伸手摸他,说:「我肯,你洁净了吧!」大麻风立刻就离了他的身。
\VS{14}耶稣嘱咐他:「你切不可告诉人,只要去把身体给祭司察看,又要为你得了洁净,照{\PN{摩西}}所吩咐的献上{\ADD{礼物}},对众人作证据。」
\VS{15}但耶稣的名声越发传扬出去。有极多的人聚集来听道,也指望医治他们的病。
\VS{16}耶稣却退到旷野去祷告。
\par }{\SH 治好瘫痪病人
\par }{\R (太9·1—8;可2·1—12)
\par }{\PP \VS{17}有一天,耶稣教训人,有法利赛人和教法师在旁边坐着;他们是从{\PN{加利利}}各乡村和{\PN{犹太}}并{\PN{耶路撒冷}}来的。主的能力与耶稣同在,使他能医治病人。
\VS{18}有人用褥子抬着一个瘫子,要抬进去放在耶稣面前,
\VS{19}却因人多,寻不出法子抬进去,就上了房顶,从瓦间把他连褥子缒到当中,正在耶稣面前。
\VS{20}耶稣见他们的信心,就对瘫子说:「你的罪赦了。」
\VS{21}文士和法利赛人就议论说:「这说僭妄话的是谁?除了 神以外,谁能赦罪呢?」
\VS{22}耶稣知道他们所议论的,就说:「你们心里议论的是什么呢?
\VS{23}或说『你的罪赦了』,或说『你起来行走』,哪一样容易呢?
\VS{24}但要叫你们知道,人子在地上有赦罪的权柄。」就对瘫子说:「我吩咐你,起来,拿你的褥子回家去吧!」
\VS{25}那人当众人面前立刻起来,拿着他所躺卧的{\ADD{褥子}}回家去,归荣耀与 神。
\VS{26}众人都惊奇,也归荣耀与 神,并且满心惧怕,说:「我们今日看见非常的事了。」
\par }{\SH 呼召利未
\par }{\R (太9·9—13;可2·13—17)
\par }{\PP \VS{27}这事以后,耶稣出去,看见一个税吏,名叫{\PN{利未}},坐在税关上,就对他说:「你跟从我来。」
\VS{28}他就撇下所有的,起来,跟从了耶稣。
\VS{29}{\PN{利未}}在自己家里为耶稣大摆筵席,有许多税吏和别人与他们一同坐席。
\VS{30}法利赛人和文士就向耶稣的门徒发怨言说:「你们为什么和税吏并罪人一同吃喝呢?」
\VS{31}耶稣对他们说:「无病的人用不着医生;有病的人才用得着。
\VS{32}我来本不是召义人悔改,乃是召罪人悔改。」
\par }{\SH 禁食的问题
\par }{\R (太9·14—17;可2·18—22)
\par }{\PP \VS{33}他们说:「{\PN{约翰}}的门徒屡次禁食祈祷,法利赛人的{\ADD{门徒}}也是这样;惟独你的门徒又吃又喝。」
\VS{34}耶稣对他们说:「新郎和陪伴之人同在的时候,岂能叫陪伴之人禁食呢?
\VS{35}但日子将到,新郎要离开他们,那日他们就要禁食了。」
\VS{36}耶稣又设一个比喻,对他们说:「没有人把新衣服撕下一块来补在旧衣服上;若是这样,就把新的撕破了,并且所撕下来的那块新的和旧的也不相称。
\VS{37}也没有人把新酒装在旧皮袋里;若是这样,新酒必将皮袋裂开,酒便漏出来,皮袋也就坏了。
\VS{38}但新酒必须装在新皮袋里。
\VS{39}没有人喝了陈酒又想喝新的;他总说陈的好。」

\par }\Chap{6}{\SH 安息日的问题
\par }{\R (太12·1—8;可2·23—28)
\par }{\PP \VerseOne{1}有一个安息日,耶稣从麦地经过。他的门徒掐了麦穗,用手搓着吃。
\VS{2}有几个法利赛人说:「你们为什么做安息日不可做的事呢?」
\VS{3}耶稣对他们说:「{\ADD{经上记着}}{\PN{大卫}}和跟从他的人饥饿之时所做的事,连这个你们也没有念过吗?
\VS{4}他怎么进了 神的殿,拿陈设饼吃,又给跟从的人吃?这饼除了祭司以外,别人都不可吃。」
\VS{5}又对他们说:「人子是安息日的主。」
\par }{\SH 治好枯干了一只手的人
\par }{\R (太12·9—14;可3·1—6)
\par }{\PP \VS{6}又有一个安息日,耶稣进了会堂教训人,在那里有一个人右手枯干了。
\VS{7}文士和法利赛人窥探耶稣,在安息日治病不治病,要得把柄去告他。
\VS{8}耶稣却知道他们的意念,就对那枯干一只手的人说:「起来!站在当中。」那人就起来,站着。
\VS{9}耶稣对他们说:「我问你们,在安息日行善行恶,救命害命,哪样是可以的呢?」
\VS{10}他就周围看着他们众人,对那人说:「伸出手来!」他把手一伸,手就复了原。
\VS{11}他们就满心大怒,彼此商议怎样处治耶稣。
\par }{\SH 挑选十二门徒
\par }{\R (太10·1—4;可3·13—19)
\par }{\PP \VS{12}那时,耶稣出去,上山祷告,整夜祷告 神;
\VS{13}到了天亮,叫他的门徒来,就从他们中间挑选十二个人,称他们为使徒。
\VS{14}{\ADD{这十二个人}}有{\PN{西门}}(耶稣又给他起名叫{\PN{彼得}}),还有他兄弟{\PN{安得烈}},又有{\PN{雅各}}和{\PN{约翰}},{\PN{腓力}}和{\PN{巴多罗买}},
\VS{15}{\PN{马太}}和{\PN{多马}},{\PN{亚勒腓}}的儿子{\PN{雅各}}和奋锐党的{\PN{西门}},
\VS{16}{\PN{雅各}}的儿子\FTNT{}{{\FR 6:16: }或译:兄弟}{\PN{犹大}},和卖主的{\PN{加略}}人{\PN{犹大}}。
\par }{\SH 向群众传道
\par }{\R (太4·23—25)
\par }{\PP \VS{17}耶稣和他们下了山,站在一块平地上;同站的有许多门徒,又有许多百姓,从{\PN{犹太}}全地和{\PN{耶路撒冷}},并{\PN{泰尔}}、{\PN{西顿}}的海边来,都要听他{\ADD{讲道}},又指望医治他们的病;
\VS{18}还有被污鬼缠磨的,也得了医治。
\VS{19}众人都想要摸他;因为有能力从他身上发出来,医好了他们。
\par }{\SH 论福和祸
\par }{\R (太5·1—12)
\par }{\PP \VS{20}耶稣举目看着门徒,说:
\par }{\Q 你们贫穷的人有福了!
\par }{\Q 因为 神的国是你们的。
\par }{\Q \VS{21}你们饥饿的人有福了!
\par }{\Q 因为你们将要饱足。
\par }{\Q 你们哀哭的人有福了!
\par }{\Q 因为你们将要喜笑。
\par }{\PP \VS{22}「人为人子恨恶你们,拒绝你们,辱骂你们,弃掉你们的名,以为是恶,你们就有福了!
\VS{23}当那日,你们要{\ADD{欢喜}}跳跃,因为你们在天上的赏赐是大的。他们的祖宗待先知也是这样。
\par }{\Q \VS{24}但你们富足的人有祸了!
\par }{\Q 因为你们受过你们的安慰。
\par }{\Q \VS{25}你们饱足的人有祸了!
\par }{\Q 因为你们将要饥饿。
\par }{\Q 你们喜笑的人有祸了!
\par }{\Q 因为你们将要哀恸哭泣。
\par }{\PP \VS{26}「人都说你们好的时候,你们就有祸了!因为他们的祖宗待假先知也是这样。」
\par }{\SH 论爱仇敌
\par }{\R (太5·38—48;7·12)
\par }{\PP \VS{27}「只是我告诉你们这听道的人,你们的仇敌,要爱他!恨你们的,要待他好!
\VS{28}咒诅你们的,要为他祝福!凌辱你们的,要为他祷告!
\VS{29}有人打你这边的脸,连那边的脸也由他打。有人夺你的外衣,连里衣也由他拿去。
\VS{30}凡求你的,就给他。有人夺你的东西去,不用再要回来。
\VS{31}你们愿意人怎样待你们,你们也要怎样待人。
\VS{32}你们若单爱那爱你们的人,有什么可酬谢的呢?就是罪人也爱那爱他们的人。
\VS{33}你们若善待那善待你们的人,有什么可酬谢的呢?就是罪人也是这样行。
\VS{34}你们若借给人,指望从他收回,有什么可酬谢的呢?就是罪人也借给罪人,要如数收回。
\VS{35}你们倒要爱仇敌,也要善待他们,并要借给人不指望偿还,你们的赏赐就必大了,你们也必作至高者的儿子,因为他恩待那忘恩的和作恶的。
\VS{36}你们要慈悲,像你们的父慈悲一样。」
\par }{\SH 不要论断人
\par }{\R (太7·1—5)
\par }{\PP \VS{37}「你们不要论断人,就不被论断;你们不要定人的罪,就不被定罪;你们要饶恕人,就必蒙饶恕\FTNT{}{{\FR 6:37: }饶恕:原文是释放};
\VS{38}你们要给人,就必有给你们的,并且用十足的升斗,连摇带按,上尖下流地倒在你们怀里;因为你们用什么量器量给人,也必用什么量器量给你们。」
\VS{39}耶稣又用比喻对他们说:「瞎子岂能领瞎子,两个人不是都要掉在坑里吗?
\VS{40}学生不能高过先生;凡学成了的不过和先生一样。
\VS{41}为什么看见你弟兄眼中有刺,却不想自己眼中有梁木呢?
\VS{42}你不见自己眼中有梁木,怎能对你弟兄说:『容我去掉你眼中的刺』呢?你这假冒为善的人!先去掉自己眼中的梁木,然后才能看得清楚,去掉你弟兄眼中的刺。」
\par }{\SH 树和果子
\par }{\R (太7·16—20;12·33—35)
\par }{\PP \VS{43}「因为,没有好树结坏果子,也没有坏树结好果子。
\VS{44}凡树木看果子,就可以认出它来。人不是从荆棘上摘无花果,也不是从蒺藜里摘葡萄。
\VS{45}善人从他心里所存的善就发出善来;恶人从他{\ADD{心里}}所存的恶就发出恶来;因为心里所充满的,口里就说出来。」
\par }{\SH 两种盖房子的人
\par }{\R (太7·24—27)
\par }{\PP \VS{46}「你们为什么称呼我『主啊,主啊』却不遵我的话行呢?
\VS{47}凡到我这里来,听见我的话就去行的,我要告诉你们他像什么人:
\VS{48}他像一个人盖房子,深深地挖地,把根基安在磐石上;到发大水的时候,水冲那房子,房子总不能摇动,因为根基立在磐石上\FTNT{}{{\FR 6:48: }有古卷:因为盖造得好}。
\VS{49}惟有听见不去行的,就像一个人在土地上盖房子,没有根基;水一冲,随即倒塌了,并且那房子坏的很大。」

\par }\Chap{7}{\SH 治好百夫长的仆人
\par }{\R (太8·5—13)
\par }{\PP \VerseOne{1}耶稣对百姓讲完了这一切的话,就进了{\PN{迦百农}}。
\VS{2}有一个百夫长所宝贵的仆人害病,快要死了。
\VS{3}百夫长风闻耶稣的事,就托{\PN{犹太}}人的几个长老去求耶稣来救他的仆人。
\VS{4}他们到了耶稣那里,就切切地求他说:「你给他行这事是他所配得的;
\VS{5}因为他爱我们的百姓,给我们建造会堂。」
\VS{6}耶稣就和他们同去。离那家不远,百夫长托几个朋友去见耶稣,对他说:「主啊!不要劳动;因你到我舍下,我不敢当。
\VS{7}我也自以为不配去见你,只要你说一句话,我的仆人就必好了。
\VS{8}因为我在人的权下,也有兵在我以下,对这个说:『去!』他就去;对那个说:『来!』他就来;对我的仆人说:『你做这事!』他就去做。」
\VS{9}耶稣听见这话,就希奇他,转身对跟随的众人说:「我告诉你们,这么大的信心,就是在{\PN{以色列}}中,我也没有遇见过。」
\VS{10}那托来的人回到百夫长家里,看见仆人已经好了。
\par }{\SH 使拿因城寡妇之子复活
\par }{\PP \VS{11}过了不多时\FTNT{}{{\FR 7:11: }有古卷:次日},耶稣往一座城去,这城名叫{\PN{拿因}},他的门徒和极多的人与他同行。
\VS{12}将近城门,有一个死人被抬出来。这人是他母亲独生的儿子;他母亲又是寡妇。有城里的许多人同着寡妇送殡。
\VS{13}主看见那寡妇,就怜悯她,对她说:「不要哭!」
\VS{14}于是进前按着杠,抬的人就站住了。耶稣说:「少年人,我吩咐你,起来!」
\VS{15}那死人就坐起,并且说话。耶稣便把他交给他母亲。
\VS{16}众人都惊奇,归荣耀与 神,说:「有大先知在我们中间兴起来了!」又说:「 神眷顾了他的百姓!」
\VS{17}他这事的风声就传遍了{\PN{犹太}}和周围地方。
\par }{\SH 施洗约翰的门徒来见耶稣
\par }{\R (太11·2—19)
\par }{\PP \VS{18}{\PN{约翰}}的门徒把这些事都告诉{\PN{约翰}}。
\VS{19}他便叫了两个门徒来,打发他们到主那里去,说:「那将要来的是你吗?还是我们等候别人呢?」
\VS{20}那两个人来到耶稣那里,说:「施洗的{\PN{约翰}}打发我们来问你:『那将要来的是你吗?还是我们等候别人呢?』」
\VS{21}正当那时候,耶稣治好了许多有疾病的,受灾患的,被恶鬼附着的,又开恩叫好些瞎子能看见。
\VS{22}耶稣回答说:「你们去,把所看见所听见的事告诉{\PN{约翰}},就是瞎子看见,瘸子行走,长大麻风的洁净,聋子听见,死人复活,穷人有福音传给他们。
\VS{23}凡不因我跌倒的,就有福了!」
\VS{24}{\PN{约翰}}所差来的人既走了,耶稣就对众人讲论{\PN{约翰}}说:「你们从前出去到旷野,是要看什么呢?要看风吹动的芦苇吗?
\VS{25}你们出去,到底是要看什么?要看穿细软衣服的人吗?那穿华丽衣服、宴乐度日的人是在王宫里。
\VS{26}你们出去,究竟是要看什么?要看先知吗?我告诉你们,是的,他比先知大多了。
\VS{27}{\ADD{经上}}记着说:『我要差遣我的使者在你前面预备道路』,所说的就是这个人。
\VS{28}我告诉你们,凡妇人所生的,没有一个大过{\PN{约翰}}的;然而 神国里最小的比他还大。」
\VS{29}众百姓和税吏既受过{\PN{约翰}}的洗,听见这话,就以 神为义;
\VS{30}但法利赛人和律法师没有受过{\PN{约翰}}的洗,竟为自己废弃了 神 的旨意。\FTNT{}{{\FR 7:30: }二十九三十两节或译:众百姓和税吏听见了约翰的话,就受了他的洗,便以  神为义;但法利赛人和律法师不受约翰的洗,竟为自己废弃了 神的旨意。}
\par }{\PP \VS{31}{\ADD{主又说}}:「这样,我可用什么比这世代的人呢?他们好像什么呢?
\VS{32}好像孩童坐在街市上,彼此呼叫说:
\par }{\Q 我们向你们吹笛,
\par }{\Q 你们不跳舞;
\par }{\Q 我们向你们举哀,
\par }{\Q 你们不啼哭。
\par }{\MM \VS{33}施洗的{\PN{约翰}}来,不吃饼,不喝酒,你们说他是被鬼附着的。
\VS{34}人子来,也吃也喝,你们说他是贪食好酒的人,是税吏和罪人的朋友。
\VS{35}但智慧之子都以智慧为是。」
\par }{\SH 有罪的女人得蒙赦免
\par }{\PP \VS{36}有一个法利赛人请耶稣和他吃饭;耶稣就到法利赛人家里去坐席。
\VS{37}那城里有一个女人,是个罪人,知道耶稣在法利赛人家里坐席,就拿着盛香膏的玉瓶,
\VS{38}站在耶稣背后,挨着他的脚哭,眼泪湿了耶稣的脚,就用自己的头发擦干,又用嘴连连亲他的脚,把香膏抹上。
\VS{39}请耶稣的法利赛人看见这事,心里说:「这人若是先知,必知道摸他的是谁,是个怎样的女人;乃是个罪人。」
\VS{40}耶稣对他说:「{\PN{西门}}!我有句话要对你说。」{\PN{西门}}说:「夫子,请说。」
\VS{41}{\ADD{耶稣说}}:「一个债主有两个人欠他的债;一个欠五十两银子,一个欠五两银子;
\VS{42}因为他们无力偿还,债主就开恩免了他们两个人的债。这两个人哪一个更爱他呢?」
\VS{43}{\PN{西门}}回答说:「我想是那多得恩免的人。」耶稣说:「你断的不错。」
\VS{44}于是转过来向着那女人,便对{\PN{西门}}说:「你看见这女人吗?我进了你的家,你没有给我水洗脚;但这女人用眼泪湿了我的脚,用头发擦干。
\VS{45}你没有与我亲嘴;但这女人从我进来的时候就不住地用嘴亲我的脚。
\VS{46}你没有用油抹我的头;但这女人用香膏抹我的脚。
\VS{47}所以我告诉你,她许多的罪都赦免了,因为她的爱多;但那赦免少的,他的爱就少。」
\VS{48}于是对那女人说:「你的罪赦免了。」
\VS{49}同席的人心里说:「这是什么人,竟赦免人的罪呢?」
\VS{50}耶稣对那女人说:「你的信救了你;平平安安回去吧!」

\par }\Chap{8}{\SH 跟从耶稣的妇女们
\par }{\PP \VerseOne{1}过了不多日,耶稣周游各城各乡传道,宣讲 神国的福音。和他同去的有十二个门徒,
\VS{2}还有被恶鬼所附、被疾病所累、已经治好的几个妇女,内中有称为{\PN{抹大拉}}的{\PN{马利亚}}(曾有七个鬼从她身上赶出来),
\VS{3}又有{\PN{希律}}的家宰{\PN{苦撒}}的妻子{\PN{约亚拿}},并{\PN{苏撒拿}},和好些别的妇女,都是用自己的财物供给耶稣和门徒。
\par }{\SH 撒种的比喻
\par }{\R (太13·1—9;可4·1—9)
\par }{\PP \VS{4}当许多人聚集、又有人从各城里出来见耶稣的时候,耶稣就用比喻说:
\VS{5}「有一个撒种的出去撒种。撒的时候,有落在路旁的,被人践踏,天上的飞鸟又来吃尽了。
\VS{6}有落在磐石上的,一出来就枯干了,因为得不着滋润。
\VS{7}有落在荆棘里的,荆棘一同生长,把它挤住了。
\VS{8}又有落在好土里的,生长起来,结实百倍。」耶稣说了这些话,就大声说:「有耳可听的,就应当听!」
\par }{\SH 用比喻的目的
\par }{\R (太13·10—17;可4·10—12)
\par }{\PP \VS{9}门徒问耶稣说:「这比喻是什么意思呢?」
\VS{10}他说:「 神国的奥秘只叫你们知道;至于别人,就用比喻,叫他们看也看不见,听也听不明。」
\par }{\SH 解明撒种的比喻
\par }{\R (太13·18—23;可4·13—20)
\par }{\PP \VS{11}「这比喻乃是这样:种子就是 神的道。
\VS{12}那些在路旁的,就是人听了道,随后魔鬼来,从他们心里把道夺去,恐怕他们信了得救。
\VS{13}那些在磐石上的,就是人听道,欢喜领受,但心中没有根,不过暂时相信,及至遇见试炼就退后了。
\VS{14}那落在荆棘里的,就是人听了道,走开以后,被今生的思虑、钱财、宴乐挤住了,便结不出成熟的子粒来。
\VS{15}{\ADD{那落}}在好土里的,就是人听了道,持守在诚实善良的心里,并且忍耐着结实。」
\par }{\SH 器皿下的灯
\par }{\R (可4·21—25)
\par }{\PP \VS{16}「没有人点灯用器皿盖上,或放在床底下,乃是放在灯台上,叫进来的人看见亮光。
\VS{17}因为掩藏的事没有不显出来的;隐瞒的事没有不露出来被人知道的。
\VS{18}所以,你们应当小心怎样听;因为凡有的,还要加给他;凡没有的,连他自以为有的,也要夺去。」
\par }{\SH 耶稣的母亲和兄弟们
\par }{\R (太12·46—50;可3·31—35)
\par }{\PP \VS{19}耶稣的母亲和他弟兄来了,因为人多,不得到他跟前。
\VS{20}有人告诉他说:「你母亲和你弟兄站在外边,要见你。」
\VS{21}耶稣回答说:「听了 神之道而遵行的人就是我的母亲,我的弟兄了。」
\par }{\SH 平静风浪
\par }{\R (太8·23—27;可4·35—41)
\par }{\PP \VS{22}有一天,耶稣和门徒上了船,对门徒说:「我们可以渡到湖那边去。」他们就开了船。
\VS{23}正行的时候,耶稣睡着了。湖上忽然起了暴风,船将满了{\ADD{水}},甚是危险。
\VS{24}门徒来叫醒了他,说:「夫子!夫子!我们丧命啦!」耶稣醒了,斥责那狂风大浪;风浪就止住,平静了。
\VS{25}耶稣对他们说:「你们的信心在哪里呢?」他们又惧怕又希奇,彼此说:「这到底是谁?他吩咐风和水,连风和水也听从他了。」
\par }{\SH 治好格拉森被鬼附的人
\par }{\R (太8·28—34;可5·1—20)
\par }{\PP \VS{26}他们到了{\PN{格拉森}}\FTNT{}{{\FR 8:26: }有古卷:加大拉}人的地方,就是{\PN{加利利}}的对面。
\VS{27}耶稣上了岸,就有城里一个被鬼附着的人迎面而来。这个人许久不穿衣服,不住房子,只住在坟茔里。
\VS{28}他见了耶稣,就俯伏在他面前,大声喊叫,说:「至高 神的儿子耶稣,我与你有什么相干?求你不要叫我受苦!」
\VS{29}是因耶稣曾吩咐污鬼从那人身上出来。原来这鬼屡次抓住他;他常被人看守,又被铁链和脚镣捆锁,他竟把锁链挣断,被鬼赶到旷野去。
\VS{30}耶稣问他说:「你名叫什么?」他说:「我名叫『群』」;这是因为附着他的鬼多。
\VS{31}鬼就央求耶稣,不要吩咐他们到无底坑里去。
\par }{\PP \VS{32}那里有一大群猪在山上吃食。鬼央求耶稣,准他们进入猪里去。耶稣准了他们,
\VS{33}鬼就从那人出来,进入猪里去。于是那群猪闯下山崖,投在湖里淹死了。
\VS{34}放猪的看见这事就逃跑了,去告诉城里和乡下的人。
\VS{35}众人出来要看是什么事;到了耶稣那里,看见鬼所离开的那人,坐在耶稣脚前,穿着衣服,心里明白过来,他们就害怕。
\VS{36}看见这事的便将被鬼附着的人怎么得救告诉他们。
\VS{37}{\PN{格拉森}}四围的人,因为害怕得很,都求耶稣离开他们;耶稣就上船回去了。
\VS{38}鬼所离开的那人恳求和耶稣同在;耶稣却打发他回去,说:
\VS{39}「你回家去,传说 神为你做了何等大的事。」他就去,满城里传扬耶稣为他做了何等大的事。
\par }{\SH 睚鲁的女儿和血漏的女人
\par }{\R (太9·18—26;可5·21—43)
\par }{\PP \VS{40}耶稣回来的时候,众人迎接他,因为他们都等候他。
\VS{41}有一个管会堂的,名叫{\PN{睚鲁}},来俯伏在耶稣脚前,求耶稣到他家里去;
\VS{42}因他有一个独生女儿,约有十二岁,快要死了。
\par }{\PP 耶稣去的时候,众人拥挤他。
\VS{43}有一个女人,患了十二年的血漏,在医生手里花尽了她一切养生的,并没有一人能医好她。
\VS{44}她来到耶稣背后,摸他的衣裳 子,血漏立刻就止住了。
\VS{45}耶稣说:「摸我的是谁?」众人都不承认。{\PN{彼得}}和同行的人都说:「夫子,众人拥拥挤挤紧靠着你。\FTNT{}{{\FR 8:45: }有古卷加:你还问摸我的是谁吗?}」
\VS{46}耶稣说:「总有人摸我,因我觉得有能力从我身上出去。」
\VS{47}那女人知道不能隐藏,就战战兢兢地来俯伏在耶稣脚前,把摸他的缘故和怎样立刻得好了,当着众人都说出来。
\VS{48}耶稣对她说:「女儿,你的信救了你;平平安安地去吧!」
\par }{\PP \VS{49}还说话的时候,有人从管会堂的家里来,说:「你的女儿死了,不要劳动夫子。」
\VS{50}耶稣听见就对他说:「不要怕,只要信!你的女儿就必得救。」
\VS{51}耶稣到了他的家,除了{\PN{彼得}}、{\PN{约翰}}、{\PN{雅各}},和女儿的父母,不许别人同他进去。
\VS{52}众人都为这女儿哀哭捶胸。耶稣说:「不要哭!她不是死了,是睡着了。」
\VS{53}他们晓得女儿已经死了,就嗤笑耶稣。
\VS{54}耶稣拉着她的手,呼叫说:「女儿,起来吧!」
\VS{55}她的灵魂便回来,她就立刻起来了。耶稣吩咐给她东西吃。
\VS{56}她的父母惊奇得很;耶稣嘱咐他们,不要把所做的事告诉人。

\par }\Chap{9}{\SH 耶稣差遣十二个门徒
\par }{\R (太10·5—15;可6·7—13)
\par }{\PP \VerseOne{1}耶稣叫齐了十二个门徒,给他们能力、权柄,制伏一切的鬼,医治各样的病,
\VS{2}又差遣他们去宣传 神国{\ADD{的道}},医治病人,
\VS{3}对他们说:「行路的时候,不要带拐杖和口袋,不要带食物和银子,也不要带两件褂子。
\VS{4}无论进哪一家,就住在那里,也从那里起行。
\VS{5}凡不接待你们的,你们离开那城的时候,要把脚上的尘土跺下去,见证他们的不是。」
\VS{6}门徒就出去,走遍各乡宣传福音,到处治病。
\par }{\SH 希律的困惑
\par }{\R (太14·1—12;可6·14—29)
\par }{\PP \VS{7}分封的王{\PN{希律}}听见耶稣所做的一切事,就游移不定;因为有人说:「是{\PN{约翰}}从死里复活」;
\VS{8}又有人说:「是{\PN{以利亚}}显现」;还有人说:「是古时的一个先知又活了。」
\VS{9}{\PN{希律}}说:「{\PN{约翰}}我已经斩了,这却是什么人?我竟听见他这样的事呢?」就想要见他。
\par }{\SH 耶稣给五千人吃饱
\par }{\R (太14·13—21;可6·30—44;约6·1—14)
\par }{\PP \VS{10}使徒回来,将所做的事告诉耶稣,耶稣就带他们暗暗地离开那里,往一座城去;那城名叫{\PN{伯赛大}}。
\VS{11}但众人知道了,就跟着他去;耶稣便接待他们,对他们讲论 神国{\ADD{的道}},医治那些需医的人。
\VS{12}日头快要平西,十二个门徒来对他说:「请叫众人散开,他们好往四面乡村里去借宿找吃的,因为我们这里是野地。」
\VS{13}耶稣说:「你们给他们吃吧!」门徒说:「我们不过有五个饼,两条鱼,若不去为这许多人买食物就不够。」
\VS{14}那时,人数约有五千。耶稣对门徒说:「叫他们一排一排地坐下,每排大约五十个人。」
\VS{15}门徒就如此行,叫众人都坐下。
\VS{16}耶稣拿着这五个饼,两条鱼,望着天祝福,擘开,递给门徒,摆在众人面前。
\VS{17}他们就吃,并且都吃饱了;把剩下的零碎收拾起来,装满了十二篮子。
\par }{\SH 彼得认耶稣为基督
\par }{\R (太16·13—19;可8·27—29)
\par }{\PP \VS{18}耶稣自己祷告的时候,门徒也同他在那里。耶稣问他们说:「众人说我是谁?」
\VS{19}他们说:「有人说是施洗的{\PN{约翰}};有人说是{\PN{以利亚}};还有人说是古时的一个先知又活了。」
\VS{20}耶稣说:「你们说我是谁?」{\PN{彼得}}回答说:「是 神所立的基督。」
\par }{\SH 耶稣预言受难和复活
\par }{\R (太16·20—28;可8·30—9·1)
\par }{\PP \VS{21}耶稣切切地嘱咐他们,不可将这事告诉人,
\VS{22}又说:「人子必须受许多的苦,被长老、祭司长,和文士弃绝,并且被杀,第三日复活。」
\VS{23}耶稣又对众人说:「若有人要跟从我,就当舍己,天天背起他的十字架来跟从我。
\VS{24}因为,凡要救自己生命\FTNT{}{{\FR 9:24: }生命:或译灵魂;下同}的,必丧掉生命;凡为我丧掉生命的,必救了生命。
\VS{25}人若赚得全世界,却丧了自己,赔上自己,有什么益处呢?
\VS{26}凡把我和我的道当作可耻的,人子在自己的荣耀里,并天父与圣天使的荣耀里降临的时候,也要把那人当作可耻的。
\VS{27}我实在告诉你们,站在这里的,有人在没尝死味以前,必看见 神的国。」
\par }{\SH 耶稣改变形象
\par }{\R (太17·1—8;可9·2—8)
\par }{\PP \VS{28}说了这话以后约有八天,耶稣带着{\PN{彼得}}、{\PN{约翰}}、{\PN{雅各}}上山去祷告。
\VS{29}正祷告的时候,他的面貌就改变了,衣服洁白放光。
\VS{30}忽然有{\PN{摩西}}、{\PN{以利亚}}两个人同耶稣说话;
\VS{31}他们在荣光里显现,谈论耶稣去世的事,就是他在{\PN{耶路撒冷}}将要成的事。
\VS{32}{\PN{彼得}}和他的同伴都打盹,既清醒了,就看见耶稣的荣光,并同他站着的那两个人。
\VS{33}二人正要和耶稣分离的时候,{\PN{彼得}}对耶稣说:「夫子,我们在这里真好!可以搭三座棚,一座为你,一座为{\PN{摩西}},一座为{\PN{以利亚}}。」他却不知道所说的是什么。
\VS{34}说这话的时候,有一朵云彩来遮盖他们;他们进入云彩里就惧怕。
\VS{35}有声音从云彩里出来,说:「这是我的儿子,我所拣选的\FTNT{}{{\FR 9:35: }有古卷:这是我的爱子},你们要听他。」
\VS{36}声音住了,只见耶稣一人在那里。当那些日子,门徒不提所看见的事,一样也不告诉人。
\par }{\SH 治好被污鬼附身的孩子
\par }{\R (太17·14—18;可9·14—27)
\par }{\PP \VS{37}第二天,他们下了山,就有许多人迎见耶稣。
\VS{38}其中有一人喊叫说:「夫子!求你看顾我的儿子,因为他是我的独生子。
\VS{39}他被鬼抓住就忽然喊叫;鬼又叫他抽风,口中流沫,并且重重地伤害他,难以离开他。
\VS{40}我求过你的门徒,把鬼赶出去,他们却是不能。」
\VS{41}耶稣说:「嗳!这又不信又悖谬的世代啊,我在你们这里,忍耐你们要到几时呢?将你的儿子带到这里来吧!」
\VS{42}正来的时候,鬼把他摔倒,叫他重重地抽风。耶稣就斥责那污鬼,把孩子治好了,交给他父亲。
\VS{43}众人都诧异 神的大能\FTNT{}{{\FR 9:43: }大能:或译威荣}。
\par }{\SH 耶稣第二次预言他的死
\par }{\R (太17·22—23;可9·30—32)
\par }{\PP 耶稣所做的一切事,众人正希奇的时候,耶稣对门徒说:
\VS{44}「你们要把这些话存在耳中,因为人子将要被交在人手里。」
\VS{45}他们不明白这话,意思乃是隐藏的,叫他们不能明白,他们也不敢问这话的意思。
\par }{\SH 谁最伟大
\par }{\R (太18·1—5;可9·33—37)
\par }{\PP \VS{46}门徒中间起了议论,谁将为大。
\VS{47}耶稣看出他们心中的议论,就领一个小孩子来,叫他站在自己旁边,
\VS{48}对他们说:「凡为我名接待这小孩子的,就是接待我;凡接待我的,就是接待那差我来的。你们中间最小的,他便为大。」
\par }{\SH 不敌挡你们就是帮助你们
\par }{\R (可9·38—40)
\par }{\PP \VS{49}{\PN{约翰}}说:「夫子,我们看见一个人奉你的名赶鬼,我们就禁止他,因为他不与我们一同跟从{\ADD{你}}。」
\VS{50}耶稣说:「不要禁止他;因为不敌挡你们的,就是帮助你们的。」
\par }{\SH 不接待主的村庄
\par }{\PP \VS{51}耶稣被接上升的日子将到,他就定意向{\PN{耶路撒冷}}去,
\VS{52}便打发使者在他前头走。他们到了{\PN{撒马利亚}}的一个村庄,要为他预备。
\VS{53}那里的人不接待他,因他面向{\PN{耶路撒冷}}去。
\VS{54}他的门徒{\PN{雅各}}、{\PN{约翰}}看见了,就说:「主啊,你要我们吩咐火从天上降下来烧灭他们,像{\PN{以利亚}}所做的\FTNT{}{{\FR 9:54: }有古卷没有像以利亚所做的这几个字}吗?」
\VS{55}耶稣转身责备两个门徒,说:「你们的心如何,你们并不知道。
\VS{56}人子来不是要灭人的性命\FTNT{}{{\FR 9:56: }性命:或译灵魂;下同},是要救人的性命。」说着就往别的村庄去了\FTNT{}{{\FR 9:56: }有古卷只有五十五节首句,五十六节末句}。
\par }{\SH 要跟从耶稣的人
\par }{\R (太8·19—22)
\par }{\PP \VS{57}他们走路的时候,有一人对耶稣说:「你无论往哪里去,我要跟从你。」
\VS{58}耶稣说:「狐狸有洞,天空的飞鸟有窝,只是人子没有枕头的地方。」
\VS{59}又对一个人说:「跟从我来!」那人说:「主,容我先回去埋葬我的父亲。」
\VS{60}耶稣说:「任凭死人埋葬他们的死人,你只管去传扬 神国{\ADD{的道}}。」
\VS{61}又有一人说:「主,我要跟从你,但容我先去辞别我家里的人。」
\VS{62}耶稣说:「手扶着犁向后看的,不配进 神的国。」

\par }\Chap{10}{\SH 主差遣七十人
\par }{\PP \VerseOne{1}这事以后,主又设立七十个人,差遣他们两个两个地在他前面,往自己所要到的各城各地方去,
\VS{2}就对他们说:「要收的庄稼多,做工的人少。所以,你们当求庄稼的主打发工人出去收他的庄稼。
\VS{3}你们去吧!我差你们出去,如同羊羔进入狼群。
\VS{4}不要带钱囊,不要带口袋,不要带鞋;在路上也不要问人的安。
\VS{5}无论进哪一家,先要说:『愿这一家平安。』
\VS{6}那里若有当得平安的人\FTNT{}{{\FR 10:6: }当得平安的人:原文是平安之子},你们所求的平安就必临到那家;不然,就归与你们了。
\VS{7}你们要住在那家,吃喝他们所供给的,因为工人得工价是应当的;不要从这家搬到那家。
\VS{8}无论进哪一城,人若接待你们,给你们摆上什么,你们就吃什么。
\VS{9}要医治那城里的病人,对他们说:『 神的国临近你们了。』
\VS{10}无论进哪一城,人若不接待你们,你们就到街上去,
\VS{11}说:『就是你们城里的尘土黏在我们的脚上,我们也当着你们擦去。虽然如此,你们该知道 神的国临近了。』
\VS{12}我告诉你们,当审判的日子,{\PN{所多玛}}所受的,比那城还容易受呢!」
\par }{\SH 不悔改的城有祸了
\par }{\R (太11·20—24)
\par }{\PP \VS{13}「{\PN{哥拉汛}}哪,你有祸了!{\PN{伯赛大}}啊,你有祸了!因为在你们中间所行的异能若行在{\PN{泰尔}}、{\PN{西顿}},他们早已披麻蒙灰,坐在地上悔改了。
\VS{14}当审判的日子,{\PN{泰尔}}、{\PN{西顿}}所受的,比你们还容易受呢!
\VS{15}{\PN{迦百农}}啊,你已经升到天上\FTNT{}{{\FR 10:15: }或译:你将要升到天上吗},将来必推下阴间。」
\VS{16}{\ADD{又对门徒说}}:「听从你们的就是听从我;弃绝你们的就是弃绝我;弃绝我的就是弃绝那差我来的。」
\par }{\SH 七十个人回来
\par }{\PP \VS{17}那七十个人欢欢喜喜地回来,说:「主啊!因你的名,就是鬼也服了我们。」
\VS{18}耶稣对他们说:「我曾看见撒但从天上坠落,像闪电一样。
\VS{19}我已经给你们权柄可以践踏蛇和蝎子,又胜过仇敌一切的能力,断没有什么能害你们。
\VS{20}然而,不要因鬼服了你们就欢喜,要因你们的名记录在天上欢喜。」
\par }{\SH 耶稣的欢乐
\par }{\R (太11·25—27;13·16—17)
\par }{\PP \VS{21}正当那时,耶稣被圣灵感动就欢乐,说:「父啊,天地的主,我感谢你!因为你将这些事向聪明通达人就藏起来,向婴孩就显出来。父啊!是的,因为你的美意本是如此。
\VS{22}一切所有的都是我父交付我的;除了父,没有人知道子是谁;除了子和子所愿意指示的,没有人知道父是谁。」
\VS{23}耶稣转身暗暗地对门徒说:「看见你们所看见的,那眼睛就有福了。
\VS{24}我告诉你们,从前有许多先知和君王要看你们所看的,却没有看见,要听你们所听的,却没有听见。」
\par }{\SH 好撒马利亚人
\par }{\PP \VS{25}有一个律法师起来试探耶稣, 说:「夫子!我该做什么才可以承受永生?」
\VS{26}耶稣对他说:「律法上写的是什么?你念的是怎样呢?」
\VS{27}他回答说:「你要尽心、尽性、尽力、尽意爱主—你的 神;又要爱邻舍如同自己。」
\VS{28}耶稣说:「你回答的是;你这样行,就必得{\ADD{永}}生。」
\VS{29}那人要显明自己有理,就对耶稣说:「谁是我的邻舍呢?」
\VS{30}耶稣回答说:「有一个人从{\PN{耶路撒冷}}下{\PN{耶利哥}}去,落在强盗手中。他们剥去他的衣裳,把他打个半死,就丢下他走了。
\VS{31}偶然有一个祭司从这条路下来,看见他就从那边过去了。
\VS{32}又有一个{\PN{利未}}人来到这地方,看见他,也照样从那边过去了。
\VS{33}惟有一个{\PN{撒马利亚}}人行路来到那里,看见他就动了慈心,
\VS{34}上前用油和酒倒在他的伤处,包裹好了,扶他骑上自己的牲口,带到店里去照应他。
\VS{35}第二天拿出二钱银子来,交给店主,说:『你且照应他;此外所费用的,我回来必还你。』
\VS{36}你想,这三个人哪一个是落在强盗手中的邻舍呢?」
\VS{37}他说:「是怜悯他的。」耶稣说:「你去照样行吧。」
\par }{\SH 耶稣探望马大和马利亚
\par }{\PP \VS{38}他们走路的时候,耶稣进了一个村庄。有一个女人,名叫{\PN{马大}},接他到自己家里。
\VS{39}她有一个妹子,名叫{\PN{马利亚}},在耶稣脚前坐着听他的道。
\VS{40}{\PN{马大}}伺候的事多,心里忙乱,就进前来,说:「主啊,我的妹子留下我一个人伺候,你不在意吗?请吩咐她来帮助我。」
\VS{41}耶稣回答说:「{\PN{马大}}!{\PN{马大}}!你为许多的事思虑烦扰,
\VS{42}但是不可少的只有一件;{\PN{马利亚}}已经选择那上好的福分,是不能夺去的。」

\par }\Chap{11}{\SH 祷告的教训
\par }{\R (太6·9—13;7·7—11)
\par }{\PP \VerseOne{1}耶稣在一个地方祷告;祷告完了,有个门徒对他说:「求主教导我们祷告,像{\PN{约翰}}教导他的门徒。」
\VS{2}耶稣说:「你们祷告的时候,要说:
\par }{\Q 我们在天上的父\FTNT{}{{\FR 11:2: }有古卷:父啊}:
\par }{\Q 愿人都尊你的名为圣。
\par }{\Q 愿你的国降临;
\par }{\Q 愿你的旨意行在地上,
\par }{\Q 如同行在天上\FTNT{}{{\FR 11:2: }有古卷没有愿你的旨意…}。
\par }{\Q \VS{3}我们日用的饮食,
\par }{\Q 天天赐给我们。
\par }{\Q \VS{4}赦免我们的罪,
\par }{\Q 因为我们也赦免凡亏欠我们的人。
\par }{\Q 不叫我们遇见试探;
\par }{\Q 救我们脱离凶恶\FTNT{}{{\FR 11:4: }有古卷没有末句}。」
\par }{\PP \VS{5}耶稣又说:「你们中间谁有一个朋友半夜到他那里去,说:『朋友!请借给我三个饼;
\VS{6}因为我有一个朋友行路,来到我这里,我没有什么给他摆上。』
\VS{7}那人在里面回答说:『不要搅扰我,门已经关闭,孩子们也同我在床上了,我不能起来给你。』
\VS{8}我告诉你们,虽不因他是朋友起来给他,但因他情词迫切地直求,就必起来照他所需用的给他。
\VS{9}我又告诉你们,你们祈求,就给你们;寻找,就寻见;叩门,就给你们开门。
\VS{10}因为,凡祈求的,就得着;寻找的,就寻见;叩门的,就给他开门。
\VS{11}你们中间作父亲的,谁有儿子求饼,反给他石头呢?求鱼,反拿蛇当鱼给他呢?
\VS{12}求{\ADD{鸡}}蛋,反给他蝎子呢?
\VS{13}你们虽然不好,尚且知道拿好东西给儿女;何况天父,岂不更将圣灵给求他的人吗?」
\par }{\SH 耶稣和别西卜
\par }{\R (太12·22—30;可3·20—27)
\par }{\PP \VS{14}耶稣赶出一个叫人哑巴的鬼;鬼出去了,哑巴就说出话来;众人都希奇。
\VS{15}内中却有人说:「他是靠着鬼王别西卜赶鬼。」
\VS{16}又有人试探耶稣,向他求从天上来的神迹。
\VS{17}他晓得他们的意念,便对他们说:「凡一国自相纷争,就成为荒场;凡一家自相纷争,就必败落。
\VS{18}若撒但自相纷争,他的国怎能站得住呢?因为你们说我是靠着别西卜赶鬼。
\VS{19}我若靠着别西卜赶鬼,你们的子弟赶鬼又靠着谁呢?这样,他们就要断定你们的是非。
\VS{20}我若靠着 神的能力赶鬼,这就是 神的国临到你们了。
\VS{21}壮士披挂整齐,看守自己的住宅,他所有的都平安无事;
\VS{22}但有一个比他更壮的来,胜过他,就夺去他所倚靠的盔甲兵器,又分了他的赃。
\VS{23}不与我相合的,就是敌我的;不同我收聚的,就是分散的。」
\par }{\SH 污鬼回来
\par }{\R (太12·43—45)
\par }{\PP \VS{24}「污鬼离了人身,就在无水之地过来过去,寻求安歇{\ADD{之处}};既寻不着,便说:『我要回到我所出来的屋里去。』
\VS{25}到了,就看见{\ADD{里面}}打扫干净,修饰好了,
\VS{26}便去另带了七个比自己更恶的鬼来,都进去住在那里。那人末后的景况比先前更不好了。」
\par }{\SH 真正的福
\par }{\PP \VS{27}耶稣正说这话的时候,众人中间有一个女人大声说:「怀你胎的和乳养你的有福了!」
\VS{28}耶稣说:「是,却还不如听 神之道而遵守的人有福。」
\par }{\SH 求神迹的受责备
\par }{\R (太12·38—42)
\par }{\PP \VS{29}当众人聚集的时候,耶稣开讲说:「这世代是一个邪恶的世代。他们求看神迹,除了{\PN{约拿}}的神迹以外,再没有神迹给他们看。
\VS{30}{\PN{约拿}}怎样为{\PN{尼尼微}}人成了神迹,人子也要照样为这世代的人成了神迹。
\VS{31}当审判的时候,南方的女王要起来定这世代的罪;因为她从地极而来,要听{\PN{所罗门}}的智慧话。看哪,在这里有一人比{\PN{所罗门}}更大。
\VS{32}当审判的时候,{\PN{尼尼微}}人要起来定这世代的罪,因为{\PN{尼尼微}}人听了{\PN{约拿}}所传的就悔改了。看哪,在这里有一人比{\PN{约拿}}更大。」
\par }{\SH 论心里的光
\par }{\R (太5·15;6·22—23)
\par }{\PP \VS{33}「没有人点灯放在地窨子里,或是斗底下,总是放在灯台上,使进来的人得见亮光。
\VS{34}你眼睛就是身上的灯。你的眼睛若了亮,全身就光明;眼睛若昏花,全身就黑暗。
\VS{35}所以,你要省察,恐怕你里头的光或者黑暗了。
\VS{36}若是你全身光明,毫无黑暗,就必全然光明,如同灯的明光照亮你。」
\par }{\SH 谴责法利赛人和文士
\par }{\R (太23·1—36;可12·38—40)
\par }{\PP \VS{37}说话的时候,有一个法利赛人请耶稣同他吃饭,耶稣就进去坐席。
\VS{38}这法利赛人看见耶稣饭前不洗{\ADD{手}}便诧异。
\VS{39}主对他说:「如今你们法利赛人洗净杯盘的外面,你们里面却满了勒索和邪恶。
\VS{40}无知的人哪,造外面的,不也造里面吗?
\VS{41}只要把里面的施舍给人,凡物于你们就都洁净了。
\par }{\PP \VS{42}「你们法利赛人有祸了!因为你们将薄荷、芸香并各样菜蔬献上十分之一,那公义和爱 神的事反倒不行了。这原是你们当行的;那也是不可不行的。
\VS{43}你们法利赛人有祸了!因为你们喜爱会堂里的首位,又喜爱人在街市上问你们的安。
\VS{44}你们有祸了!因为你们如同不显露的坟墓,走在上面的人并不知道。」
\par }{\PP \VS{45}律法师中有一个回答耶稣说:「夫子!你这样说也把我们糟蹋了。」
\VS{46}耶稣说:「你们律法师也有祸了!因为你们把难担的担子放在人身上,自己一个指头却不肯动。
\VS{47}你们有祸了!因为你们修造先知的坟墓,那先知正是你们的祖宗所杀的。
\VS{48}可见你们祖宗所做的事,你们又证明又喜欢;因为他们杀了先知,你们修造{\ADD{先知的坟墓}}。
\VS{49}所以 神用智慧\FTNT{}{{\FR 11:49: }用智慧:或译的智者}曾说:『我要差遣先知和使徒到他们那里去,有的他们要杀害,有的他们要逼迫』,
\VS{50}使创世以来所流众先知血的罪都要问在这世代的人身上,
\VS{51}就是从{\PN{亚伯}}的血起,直到被杀在坛和殿中间{\PN{撒迦利亚}}的血为止。我实在告诉你们,这都要问在这世代的人身上。
\VS{52}你们律法师有祸了!因为你们把知识的钥匙夺了去,自己不进去,正要进去的人你们也阻挡他们。」
\VS{53}耶稣从那里出来,文士和法利赛人就极力地催逼他,引动他多说话,
\VS{54}私下窥听,要拿他的话柄。

\par }\Chap{12}{\SH 防备假冒为善
\par }{\R (太10·26—27)
\par }{\PP \VerseOne{1}这时,有几万人聚集,甚至彼此践踏。耶稣开讲,先对门徒说:「你们要防备法利赛人的酵,就是假冒为善。
\VS{2}掩盖的事没有不露出来的;隐藏的事没有不被人知道的。
\VS{3}因此,你们在暗中所说的,将要在明处被人听见;在内室附耳所说的,将要在房上被人宣扬。」
\par }{\SH 该怕的是谁
\par }{\R (太10·28—31)
\par }{\PP \VS{4}「我的朋友,我对你们说,那杀身体以后不能再做什么的,不要怕他们。
\VS{5}我要指示你们当怕的是谁:当怕那杀了以后又有权柄丢在地狱里的。我实在告诉你们,正要怕他。
\VS{6}五个麻雀不是卖二分银子吗?但在 神面前,一个也不忘记;
\VS{7}就是你们的头发,也都被数过了。不要惧怕,你们比许多麻雀还贵重!」
\par }{\SH 在人的面前承认基督
\par }{\R (太10·32—33;12·32;10·19—20)
\par }{\PP \VS{8}「我又告诉你们,凡在人面前认我的,人子在 神的使者面前也必认他;
\VS{9}在人面前不认我的,人子在 神的使者面前也必不认他。
\VS{10}凡说话干犯人子的,还可得赦免;惟独亵渎圣灵的,总不得赦免。
\VS{11}人带你们到会堂,并官府和有权柄的人面前,不要思虑怎么分诉,说什么话;
\VS{12}因为正在那时候,圣灵要指教你们当说的话。」
\par }{\SH 无知财主的比喻
\par }{\PP \VS{13}众人中有一个人对耶稣说:「夫子!请你吩咐我的兄长和我分开家业。」
\VS{14}耶稣说:「你这个人!谁立我作你们断事的官,给你们分家业呢?」
\VS{15}于是对众人说:「你们要谨慎自守,免去一切的贪心,因为人的生命不在乎家道丰富。」
\VS{16}就用比喻对他们说:「有一个财主田产丰盛;
\VS{17}自己心里思想说:『我的出产没有地方收藏,怎么办呢?』
\VS{18}又说:『我要这么办:要把我的仓房拆了,另盖更大的,在那里好收藏我一切的粮食和财物,
\VS{19}然后要对我的灵魂说:灵魂哪,你有许多财物积存,可作多年的费用,只管安安逸逸地吃喝快乐吧!』
\VS{20}神却对他说:『无知的人哪,今夜必要你的灵魂;你所预备的要归谁呢?』
\VS{21}凡为自己积财,在 神面前却不富足的,也是这样。」
\par }{\SH 不要忧虑
\par }{\R (太6·25—34,19—21)
\par }{\PP \VS{22}耶稣又对门徒说:「所以我告诉你们,不要为生命忧虑吃什么,为身体忧虑穿什么;
\VS{23}因为生命胜于饮食,身体胜于衣裳。
\VS{24}你想乌鸦,也不种也不收,又没有仓又没有库, 神尚且养活它。你们比飞鸟是何等地贵重呢!
\VS{25}你们哪一个能用思虑使寿数多加一刻呢\FTNT{}{{\FR 12:25: }或译:使身量多加一肘呢}?
\VS{26}这最小的事,你们尚且不能做,为什么还忧虑其余的事呢?
\VS{27}你想百合花怎么长起来;它也不劳苦,也不纺线。然而我告诉你们,就是{\PN{所罗门}}极荣华的时候,他所穿戴的,还不如这花一朵呢!
\VS{28}你们这小信的人哪,野地里的草今天还在,明天就丢在炉里, 神还给它这样的妆饰,何况你们呢!
\VS{29}你们不要求吃什么,喝什么,也不要挂心;
\VS{30}这都是外邦人所求的。你们必须用这些东西,你们的父是知道的。
\VS{31}你们只要求他的国,这些东西就必加给你们了。
\VS{32}你们这小群,不要惧怕,因为你们的父乐意把国赐给你们。
\VS{33}你们要变卖所有的周济人,为自己预备永不坏的钱囊,用不尽的财宝在天上,就是贼不能近、虫不能蛀的地方。
\VS{34}因为,你们的财宝在哪里,你们的心也在那里。」
\par }{\SH 警醒的仆人
\par }{\R (太24·45—51)
\par }{\PP \VS{35}「你们腰里要束上带,灯也要点着,
\VS{36}自己好像{\ADD{仆}}人等候主人从婚姻的筵席上回来。他来到,叩门,就立刻给他开门。
\VS{37}主人来了,看见仆人警醒,那仆人就有福了。我实在告诉你们,主人必叫他们坐席,自己束上带,进前伺候他们。
\VS{38}或是二更天来,或是三更天来,看见仆人这样,那仆人就有福了。
\VS{39}家主若知道贼什么时候来,就必警醒,不容贼挖透房屋,这是你们所知道的。
\VS{40}你们也要预备;因为你们想不到的时候,人子就来了。」
\par }{\PP \VS{41}{\PN{彼得}}说:「主啊,这比喻是为我们说的呢?还是为众人呢?」
\VS{42}主说:「谁是那忠心有见识的管家,主人派他管理家里的人,按时分粮给他们呢?
\VS{43}主人来到,看见仆人这样行,那仆人就有福了。
\VS{44}我实在告诉你们,主人要派他管理一切所有的。
\VS{45}那仆人若心里说:『我的主人必来得迟』,就动手打仆人和使女,并且吃喝醉酒;
\VS{46}在他想不到的日子,不知道的时辰,那仆人的主人要来,重重地处治他\FTNT{}{{\FR 12:46: }或译:把他腰斩了},定他和不忠心的人同罪。
\VS{47}仆人知道主人的意思,却不预备,又不顺他的意思行,那仆人必多受责打;
\VS{48}惟有那不知道的,做了当受责打的事,必少受责打;因为多给谁,就向谁多取;多托谁,就向谁多要。」
\par }{\SH 分裂的原因
\par }{\R (太10·34—36)
\par }{\PP \VS{49}「我来要把火丢在地上,倘若已经着起来,不也是我所愿意的吗?
\VS{50}我有当受的洗还没有成就,我是何等地迫切呢?
\VS{51}你们以为我来,是叫地上太平吗?我告诉你们,不是,乃是叫人纷争。
\VS{52}从今以后,一家五个人将要纷争:三个人和两个人相争,两个人和三个人相争;
\par }{\Q \VS{53}父亲和儿子相争,
\par }{\Q 儿子和父亲相争;
\par }{\Q 母亲和女儿相争,
\par }{\Q 女儿和母亲相争;
\par }{\Q 婆婆和媳妇相争,
\par }{\Q 媳妇和婆婆相争。」
\par }{\SH 分辨时候
\par }{\R (太16·2—3)
\par }{\PP \VS{54}耶稣又对众人说:「你们看见西边起了云彩,就说:『要下一阵雨』;果然就有。
\VS{55}起了南风,就说:『将要燥热』;也就有了。
\VS{56}假冒为善的人哪,你们知道分辨天地的气色,怎么不知道分辨这时候呢?」
\par }{\SH 同对头和解
\par }{\R (太5·25—26)
\par }{\PP \VS{57}「你们又为何不自己审量什么是合理的呢?
\VS{58}你同{\ADD{告}}你的对头去见官,还在路上,务要尽力地和他了结;恐怕他拉你到官面前,官交付差役,差役把你下在监里。
\VS{59}我告诉你,若有半文钱没有还清,你断不能从那里出来。」

\par }\Chap{13}{\SH 悔改或灭亡
\par }{\PP \VerseOne{1}正当那时,有人将{\PN{彼拉多}}使{\PN{加利利}}人的血搀杂在他们祭物中的事告诉耶稣。
\VS{2}耶稣说:「你们以为这些{\PN{加利利}}人比众{\PN{加利利}}人更有罪,所以受这害吗?
\VS{3}我告诉你们,不是的!你们若不悔改,都要如此灭亡!
\VS{4}从前{\PN{西罗亚}}楼倒塌了,压死十八个人;你们以为那些人比一切住在{\PN{耶路撒冷}}的人更有罪吗?
\VS{5}我告诉你们,不是的!你们若不悔改,都要如此灭亡!」
\par }{\SH 不结实的无花果树
\par }{\PP \VS{6}于是用比喻说:「一个人有一棵无花果树栽在葡萄园里。他来到树前找果子,却找不着。
\VS{7}就对管园的说:『看哪,我这三年来到这无花果树前找果子,竟找不着。把它砍了吧,何必白占地土呢!』
\VS{8}管园的说:『主啊,今年且留着,等我周围掘开土,加上粪;
\VS{9}以后若结果子便罢,不然再把它砍了。』」
\par }{\SH 安息日治好驼背的女人
\par }{\PP \VS{10}安息日,耶稣在会堂里教训人。
\VS{11}有一个女人被鬼附着,病了十八年,腰弯得一点直不起来。
\VS{12}耶稣看见,便叫过她来,对她说:「女人,你脱离这病了!」
\VS{13}于是用两只手按着她;她立刻直起腰来,就归荣耀与 神。
\VS{14}管会堂的因为耶稣在安息日治病,就气忿忿地对众人说:「有六日应当做工;那六日之内可以来求医,在安息日却不可。」
\VS{15}主说:「假冒为善的人哪,难道你们各人在安息日不解开槽上的牛、驴,牵去饮吗?
\VS{16}况且这女人本是{\PN{亚伯拉罕}}的后裔,被撒但捆绑了这十八年,不当在安息日解开她的绑吗?」
\VS{17}耶稣说这话,他的敌人都惭愧了;众人因他所行一切荣耀的事,就都欢喜了。
\par }{\SH 芥菜种和面酵的比喻
\par }{\R (太13·31—33;可4·30—32)
\par }{\PP \VS{18}耶稣说:「 神的国好像什么?我拿什么来比较呢?
\VS{19}好像一粒芥菜种,有人拿去种在园子里,长大成树,天上的飞鸟宿在它的枝上。」
\VS{20}又说:「我拿什么来比 神的国呢?
\VS{21}好比面酵,有妇人拿来藏在三斗面里,直等全团都发起来。」
\par }{\SH 当进窄门
\par }{\R (太7·13—14,21—23)
\par }{\PP \VS{22}耶稣往{\PN{耶路撒冷}}去,在所经过的各城各乡教训人。
\VS{23}有一个人问他说:「主啊,得救的人少吗?」
\VS{24}耶稣对众人说:「你们要努力进窄门。我告诉你们,将来有许多人想要进去,却是不能。
\VS{25}及至家主起来关了门,你们站在外面叩门,说:『主啊,给我们开门!』他就回答说:『我不认识你们,不晓得你们是哪里来的!』
\VS{26}那时,你们要说:『我们在你面前吃过喝过,你也在我们的街上教训过人。』
\VS{27}他要说:『我告诉你们,我不晓得你们是哪里来的。你们这一切作恶的人,离开我去吧!』
\VS{28}你们要看见{\PN{亚伯拉罕}}、{\PN{以撒}}、{\PN{雅各}},和众先知都在 神的国里,你们却被赶到外面,在那里必要哀哭切齿了。
\VS{29}从东、从西、从南、从北将有人来,在 神的国里坐席。
\VS{30}只是有在后的,将要在前;有在前的,将要在后。」
\par }{\SH 为耶路撒冷哀哭
\par }{\R (太23·37—39)
\par }{\PP \VS{31}正当那时,有几个法利赛人来对耶稣说:「离开这里去吧,因为{\PN{希律}}想要杀你。」
\VS{32}耶稣说:「你们去告诉那个狐狸说:『今天、明天我赶鬼治病,第三天我的事就成全了。』
\VS{33}虽然这样,今天、明天、后天,我必须前行,因为先知在{\PN{耶路撒冷}}之外丧命是不能的。
\VS{34}{\PN{耶路撒冷}}啊!{\PN{耶路撒冷}}啊!你常杀害先知,又用石头打死那奉差遣到你这里来的人。我多次愿意聚集你的儿女,好像母鸡把小鸡聚集在翅膀底下;只是你们不愿意。
\VS{35}看哪,你们的家成为荒场留给你们。我告诉你们,{\ADD{从今以后}}你们不得再见我,直等到你们说:『奉主名来的是应当称颂的。』」

\par }\Chap{14}{\SH 在安息日治好臌胀的人
\par }{\PP \VerseOne{1}安息日,耶稣到一个法利赛人的首领家里去吃饭,他们就窥探他。
\VS{2}在他面前有一个患水臌的人。
\VS{3}耶稣对律法师和法利赛人说:「安息日治病可以不可以?」
\VS{4}他们却不言语。耶稣就治好那人,叫他走了;
\VS{5}便对他们说:「你们中间谁有驴或有牛,在安息日掉在井里,不立时拉它上来呢?」
\VS{6}他们不能对答这话。
\par }{\SH 给客人和主人的教训
\par }{\PP \VS{7}耶稣见所请的客拣择首位,就用比喻对他们说:
\VS{8}「你被人请去赴婚姻的筵席,不要坐在首位上,恐怕有比你尊贵的客被他请来;
\VS{9}那请你们的人前来对你说:『让座给这一位吧!』你就羞羞惭惭地退到末位上去了。
\VS{10}你被请的时候,就去坐在末位上,好叫那请你的人来对你说:『朋友,请上座。』那时,你在同席的人面前就有光彩了。
\VS{11}因为,凡自高的,必降为卑;自卑的,必升为高。」
\VS{12}耶稣又对请他的人说:「你摆设午饭或晚饭,不要请你的朋友、弟兄、亲属,和富足的邻舍,恐怕他们也请你,你就得了报答。
\VS{13}你摆设筵席,倒要请那贫穷的、残废的、瘸腿的、瞎眼的,你就有福了!
\VS{14}因为他们没有什么可报答你。到义人复活的时候,你要得着报答。」
\par }{\SH 大筵席的比喻
\par }{\R (太22·1—10)
\par }{\PP \VS{15}同席的有一人听见这话,就对耶稣说:「在 神国里吃饭的有福了!」
\VS{16}耶稣对他说:「有一人摆设大筵席,请了许多客。
\VS{17}到了坐席的时候,打发仆人去对所请的人说:『请来吧!样样都齐备了。』
\VS{18}众人一口同音地推辞。头一个说:『我买了一块地,必须去看看。请你准我辞了。』
\VS{19}又有一个说:『我买了五对牛,要去试一试。请你准我辞了。』
\VS{20}又有一个说:『我才娶了妻,所以不能去。』
\VS{21}那仆人回来,把这事都告诉了主人。家主就动怒,对仆人说:『快出去,到城里大街小巷,领那贫穷的、残废的、瞎眼的、瘸腿的来。』
\VS{22}仆人说:『主啊,你所吩咐的已经办了,还有空座。』
\VS{23}主人对仆人说:『你出去到路上和篱笆那里,勉强人进来,坐满我的屋子。
\VS{24}我告诉你们,先前所请的人没有一个得尝我的筵席。』」
\par }{\SH 作门徒的代价
\par }{\R (太10·37—38)
\par }{\PP \VS{25}有极多的人和耶稣同行。他转过来对他们说:
\VS{26}「人到我这里来,若不爱我胜过爱\FTNT{}{{\FR 14:26: }爱我胜过爱:原文是恨}自己的父 母、妻子、儿女、弟兄、姊妹,和自己的性命,就不能作我的门徒。
\VS{27}凡不背着自己十字架跟从我的,也不能作我的门徒。
\VS{28}你们哪一个要盖一座楼,不先坐下算计花费,能盖成不能呢?
\VS{29}恐怕安了地基,不能成功,看见的人都笑话他,说:
\VS{30}『这个人开了工,却不能完工。』
\VS{31}或是一个王出去和别的王打仗,岂不先坐下酌量,能用一万兵去敌那领二万兵来攻打他的吗?
\VS{32}若是不能,就趁敌人还远的时候,派使者去求和息的条款。
\VS{33}这样,你们无论什么人,若不撇下一切所有的,就不能作我的门徒。」
\par }{\SH 失味的盐
\par }{\R (太5·13;可9·50)
\par }{\PP \VS{34}「盐本是好的;盐若失了味,可用什么叫它再咸呢?
\VS{35}或用在田里,或堆在粪里,都不合式,只好丢在外面。有耳可听的,就应当听!」

\par }\Chap{15}{\SH 迷羊的比喻
\par }{\R (太18·12—14)
\par }{\PP \VerseOne{1}众税吏和罪人都挨近耶稣,要听他{\ADD{讲道}}。
\VS{2}法利赛人和文士私下议论说:「这个人接待罪人,又同他们吃饭。」
\VS{3}耶稣就用比喻说:
\VS{4}「你们中间谁有一百只羊失去一只,不把这九十九只撇在旷野、去找那失去的羊,直到找着呢?
\VS{5}找着了,就欢欢喜喜地扛在肩上,回到家里,
\VS{6}就请朋友邻舍来,对他们说:『我失去的羊已经找着了,你们和我一同欢喜吧!』
\VS{7}我告诉你们,一个罪人悔改,在天上也要这样为他欢喜,较比为九十九个不用悔改的义人欢喜更大。」
\par }{\SH 失钱的比喻
\par }{\PP \VS{8}「或是一个妇人有十块钱,若失落一块,岂不点上灯,打扫屋子,细细地找,直到找着吗?
\VS{9}找着了,就请朋友邻舍来,对他们说:『我失落的那块钱已经找着了,你们和我一同欢喜吧!』
\VS{10}我告诉你们,一个罪人悔改,在 神的使者面前也是这样为他欢喜。」
\par }{\SH 浪子的比喻
\par }{\PP \VS{11}耶稣又说:「一个人有两个儿子。
\VS{12}小儿子对父亲说:『父亲,请你把我应得的家业分给我。』他父亲就把产业分给他们。
\VS{13}过了不多几日,小儿子就把他一切所有的都收拾起来,往远方去了。在那里任意放荡,浪费资财。
\VS{14}既耗尽了一切所有的,又遇着那地方大遭饥荒,就穷苦起来。
\VS{15}于是去投靠那地方的一个人;那人打发他到田里去放猪。
\VS{16}他恨不得拿猪所吃的豆荚充饥,也没有人给他。
\VS{17}他醒悟过来,就说:『我父亲有多少的雇工,口粮有余,我倒在这里饿死吗?
\VS{18}我要起来,到我父亲那里去,向他说:父亲!我得罪了天,又得罪了你;
\VS{19}从今以后,我不配称为你的儿子,把我当作一个雇工吧!』
\VS{20}于是起来,往他父亲那里去。相离还远,他父亲看见,就动了慈心,跑去抱着他的颈项,连连与他亲嘴。
\VS{21}儿子说:『父亲!我得罪了天,又得罪了你;从今以后,我不配称为你的儿子。』
\VS{22}父亲却吩咐仆人说:『把那上好的袍子快拿出来给他穿;把戒指戴在他指头上;把鞋穿在他脚上;
\VS{23}把那肥牛犊牵来宰了,我们可以吃喝快乐;
\VS{24}因为我这个儿子是死而复活,失而又得的。』他们就快乐起来。
\VS{25}那时,大儿子正在田里。他回来,离家不远,听见作乐跳舞的声音,
\VS{26}便叫过一个仆人来,问是什么事。
\VS{27}仆人说:『你兄弟来了;你父亲因为得他无灾无病地回来,把肥牛犊宰了。』
\VS{28}大儿子却生气,不肯进去;他父亲就出来劝他。
\VS{29}他对父亲说:『我服事你这多年,从来没有违背过你的命,你并没有给我一只山羊羔,叫我和朋友一同快乐。
\VS{30}但你这个儿子和娼妓吞尽了你的产业,他一来了,你倒为他宰了肥牛犊。』
\VS{31}父亲对他说:『儿啊!你常和我同在,我一切所有的都是你的;
\VS{32}只是你这个兄弟是死而复活、失而又得的,所以我们理当欢喜快乐。』」

\par }\Chap{16}{\SH 不义的管家
\par }{\PP \VerseOne{1}耶稣又对门徒说:「有一个财主的管家,别人向他主人告他浪费主人的财物。
\VS{2}主人叫他来,对他说:『我听见你这事怎么样呢?把你所经管的交代明白,因你不能再作{\ADD{我的}}管家。』
\VS{3}那管家心里说:『主人辞我,不用我再作管家,我将来做什么?锄地呢?无力;讨饭呢?怕羞。
\VS{4}我知道怎么行,好叫人在我不作管家之后,接我到他们家里去。』
\VS{5}于是把欠他主人债的,一个一个地叫了来,问头一个说:『你欠我主人多少?』
\VS{6}他说:『一百篓\FTNT{}{{\FR 16:6: }每篓约五十斤}油。』管家说:『拿你的帐,快坐下,写五十。』
\VS{7}又问一个说:『你欠多少?』他说:『一百石麦子。』管家说:『拿你的帐,写八十。』
\VS{8}主人就夸奖这不义的管家做事聪明;因为今世之子,在世事之上,较比光明之子更加聪明。
\VS{9}我又告诉你们,要借着那不义的钱财结交朋友,到了钱财无用的时候,他们可以接你们到永存的帐幕里去。
\VS{10}人在最小的事上忠心,在大事上也忠心;在最小的事上不义,在大事上也不义。
\VS{11}倘若你们在不义的钱财上不忠心,谁还把那真实的{\ADD{钱财}}托付你们呢?
\VS{12}倘若你们在别人的东西上不忠心,谁还把你们自己的东西给你们呢?
\VS{13}一个仆人不能事奉两个主;不是恶这个爱那个,就是重这个轻那个。你们不能又事奉 神,又事奉玛门。」
\par }{\SH 律法和 神的国
\par }{\R (太5·31—32;11·12—13;可10·11—12)
\par }{\PP \VS{14}法利赛人是贪爱钱财的;他们听见这一切话,就嗤笑耶稣。
\VS{15}耶稣对他们说:「你们是在人面前自称为义的,你们的心, 神却知道;因为人所尊贵的,是 神看为可憎恶的。
\VS{16}律法和先知到{\PN{约翰}}为止,从此 神国的福音传开了,人人努力要进去。
\VS{17}天地废去较比律法的一点一画落空还容易。
\VS{18}凡休妻另娶的就是犯奸淫;娶被休之妻的也是犯奸淫。」
\par }{\SH 财主和拉撒路
\par }{\PP \VS{19}「有一个财主穿着紫色{\ADD{袍}}和细麻布衣服,天天奢华宴乐。
\VS{20}又有一个讨饭的,名叫{\PN{拉撒路}},浑身生疮,被人放在财主门口,
\VS{21}要得财主桌子上掉下来的{\ADD{零碎}}充饥,并且狗来舔他的疮。
\VS{22}后来那讨饭的死了,被天使带去放在{\PN{亚伯拉罕}}的怀里。财主也死了,并且埋葬了。
\VS{23}他在阴间受痛苦,举目远远地望见{\PN{亚伯拉罕}},又望见{\PN{拉撒路}}在他怀里,
\VS{24}就喊着说:『我祖{\PN{亚伯拉罕}}哪,可怜我吧!打发{\PN{拉撒路}}来,用指头尖蘸点水,凉凉我的舌头;因为我在这火焰里,极其痛苦。』
\VS{25}{\PN{亚伯拉罕}}说:『儿啊,你该回想你生前享过福,{\PN{拉撒路}}也受过苦;如今他在这里得安慰,你倒受痛苦。
\VS{26}不但这样,并且在你我之间,有深渊限定,以致人要从这边过到你们那边是不能的;要从那边过到我们这边也是不能的。』
\VS{27}财主说:『我祖啊!既是这样,求你打发{\PN{拉撒路}}到我父家去;
\VS{28}因为我还有五个弟兄,他可以对他们作见证,免得他们也来到这痛苦的地方。』
\VS{29}{\PN{亚伯拉罕}}说:『他们有{\PN{摩西}}和先知{\ADD{的话}}可以听从。』
\VS{30}他说:『我祖{\PN{亚伯拉罕}}哪,不是的,若有一个从死里{\ADD{复活的}},到他们那里去的,他们必要悔改。』
\VS{31}{\PN{亚伯拉罕}}说:『若不听从{\PN{摩西}}和先知{\ADD{的话}},就是有一个从死里复活的,他们也是不听劝。』」

\par }\Chap{17}{\SH 罪,信心,仆人的本分
\par }{\R (太18·6—7,21—22;可9·42)
\par }{\PP \VerseOne{1}耶稣又对门徒说:「绊倒人的事是免不了的;但那绊倒人的有祸了。
\VS{2}就是把磨石拴在这人的颈项上,丢在海里,还强如他把这小子里的一个绊倒了。
\VS{3}你们要谨慎!若是你的弟兄得罪{\ADD{你}},就劝戒他;他若懊悔,就饶恕他。
\VS{4}倘若他一天七次得罪你,又七次回转,说:『我懊悔了』,你总要饶恕他。」
\par }{\PP \VS{5}使徒对主说:「{\ADD{求主}}加增我们的信心。」
\VS{6}主说:「你们若有信心像一粒芥菜种,就是对这棵桑树说:『你要拔起根来,栽在海里』,它也必听从你们。
\VS{7}你们谁有仆人耕地或是放羊,从田里回来,就对他说:『你快来坐下吃饭』呢?
\VS{8}岂不对他说:『你给我预备晚饭,束上带子伺候我,等我吃喝完了,你才可以吃喝』吗?
\VS{9}仆人照所吩咐的去做,主人还谢谢他吗?
\VS{10}这样,你们做完了一切所吩咐的,只当说:『我们是无用的仆人,所做的本是我们应分做的。』」
\par }{\SH 治好十个长大麻风的
\par }{\PP \VS{11}耶稣往{\PN{耶路撒冷}}去,经过{\PN{撒马利亚}}和{\PN{加利利}}。
\VS{12}进入一个村子,有十个长大麻风的,迎面而来,远远地站着,
\VS{13}高声说:「耶稣,夫子,可怜我们吧!」
\VS{14}耶稣看见,就对他们说:「你们去把身体给祭司察看。」他们去的时候就洁净了。
\VS{15}内中有一个见自己已经好了,就回来大声归荣耀与 神,
\VS{16}又俯伏在耶稣脚前感谢他;这人是{\PN{撒马利亚}}人。
\VS{17}耶稣说:「洁净了的不是十个人吗?那九个在哪里呢?
\VS{18}除了这外族人,再没有别人回来归荣耀与 神吗?」
\VS{19}就对那人说:「起来,走吧!你的信救了你了。」
\par }{\SH  神国的来到
\par }{\R (太24·23—28,37—41)
\par }{\PP \VS{20}法利赛人问:「 神的国几时来到?」耶稣回答说:「 神的国来到不是眼所能见的。
\VS{21}人也不得说:『看哪,在这里!看哪,在那里!』因为 神的国就在你们心里\FTNT{}{{\FR 17:21: }心里:或译中间}。」
\VS{22}他又对门徒说:「日子将到,你们巴不得看见人子的一个日子,却不得看见。
\VS{23}人将要对你们说:『看哪,在那里!看哪,在这里!』你们不要出去,也不要跟随他们!
\VS{24}因为人子在他{\ADD{降临}}的日子,好像闪电从天这边一闪直照到天那边。
\VS{25}只是他必须先受许多苦,又被这世代弃绝。
\VS{26}{\PN{挪亚}}的日子怎样,人子的日子也要怎样。
\VS{27}那时候的人又吃又喝,又娶又嫁,到{\PN{挪亚}}进方舟的那日,洪水就来,把他们全都灭了。
\VS{28}又好像{\PN{罗得}}的日子;人又吃又喝,又买又卖,又耕种又盖造。
\VS{29}到{\PN{罗得}}出{\PN{所多玛}}的那日,就有火与硫磺从天上降下来,把他们全都灭了。
\VS{30}人子显现的日子也要这样。
\VS{31}当那日,人在房上,器具在屋里,不要下来拿;人在田里,也不要回家。
\VS{32}你们要回想{\PN{罗得}}的妻子。
\VS{33}凡想要保全生命的,必丧掉生命;凡丧掉生命的,必救活生命。
\VS{34}我对你们说,当那一夜,两个人在一个床上,要取去一个,撇下一个。
\VS{35}两个女人一同推磨,要取去一个,撇下一个。\FTNT{}{{\FR 17:35: }有古卷加:36两个人在田里,要取去一个,撇下一个。}」
\VS{37}门徒说:「主啊,在哪里{\ADD{有这事}}呢?」耶稣说:「尸首在哪里,鹰也必聚在那里。」

\par }\Chap{18}{\SH 寡妇和法官的比喻
\par }{\PP \VerseOne{1}耶稣设一个比喻,是要人常常祷告,不可灰心。
\VS{2}说:「某城里有一个官,不惧怕 神,也不尊重世人。
\VS{3}那城里有个寡妇,常到他那里,说:『我有一个对头,求你给我伸冤。』
\VS{4}他多日不准,后来心里说:『我虽不惧怕 神,也不尊重世人,
\VS{5}只因这寡妇烦扰我,我就给她伸冤吧,免得她常来缠磨我!』」
\VS{6}主说:「你们听这不义之官所说的话。
\VS{7}神的选民昼夜呼吁他,他纵然为他们忍了多时,岂不终久给他们伸冤吗?
\VS{8}我告诉你们,要快快地给他们伸冤了。然而,人子来的时候,遇得见世上有信德吗?」
\par }{\SH 法利赛人和税吏的祷告
\par }{\PP \VS{9}耶稣向那些仗着自己是义人,藐视别人的,设一个比喻,
\VS{10}说:「有两个人上殿里去祷告:一个是法利赛人,一个是税吏。
\VS{11}法利赛人站着,自言自语地祷告说:『 神啊,我感谢你,我不像别人勒索、不义、奸淫,也不像这个税吏。
\VS{12}我一个礼拜禁食两次,凡我所得的都捐上十分之一。』
\VS{13}那税吏远远地站着,连举目望天也不敢,只捶着胸说:『 神啊,开恩可怜我这个罪人!』
\VS{14}我告诉你们,这人回家去比那人倒算为义了;因为,凡自高的,必降为卑;自卑的,必升为高。」
\par }{\SH 耶稣为小孩祝福
\par }{\R (太19·13—15;可10·13—16)
\par }{\PP \VS{15}有人抱着自己的婴孩来见耶稣,要他摸他们;门徒看见就责备那些人。
\VS{16}耶稣却叫他们来,说:「让小孩子到我这里来,不要禁止他们,因为在 神国的正是这样的人。
\VS{17}我实在告诉你们,凡要承受 神国的,若不像小孩子,断不能进去。」
\par }{\SH 富足的官寻求永生之道
\par }{\R (太19·16—30;可10·17—31)
\par }{\PP \VS{18}有一个官问耶稣说:「良善的夫子,我该做什么事才可以承受永生?」
\VS{19}耶稣对他说:「你为什么称我是良善的?除了 神一位之外,再没有良善的。
\VS{20}诫命你是晓得的:『不可奸淫;不可杀人;不可偷盗;不可作假见证;当孝敬父母。』」
\VS{21}那人说:「这一切我从小都遵守了。」
\VS{22}耶稣听见了,就说:「你还缺少一件:要变卖你一切所有的,分给穷人,就必有财宝在天上;你还要来跟从我。」
\VS{23}他听见这话,就甚忧愁,因为他很富足。
\VS{24}耶稣看见他,就说:「有钱财的人进 神的国是何等的难哪!
\VS{25}骆驼穿过针的眼比财主进 神的国还容易呢!」
\VS{26}听见的人说:「这样,谁能得救呢?」
\VS{27}耶稣说:「在人所不能的事,在 神却能。」
\VS{28}{\PN{彼得}}说:「看哪,我们已经撇下自己所有的跟从你了。」
\VS{29}耶稣说:「我实在告诉你们,人为 神的国撇下房屋,或是妻子、弟兄、父母、儿女,
\VS{30}没有在今世不得百倍,在来世不得永生的。」
\par }{\SH 耶稣第三次预言受难和复活
\par }{\R (太20·17—19;可10·32—34)
\par }{\PP \VS{31}耶稣带着十二个门徒,对他们说:「看哪,我们上{\PN{耶路撒冷}}去,先知所写的一切事都要成就在人子身上。
\VS{32}他将要被交给外邦人;他们要戏弄他,凌辱他,吐唾沫在他脸上,
\VS{33}并要鞭打他,杀害他;第三日他要复活。」
\VS{34}这些事门徒一样也不懂得,意思乃是隐藏的;他们不晓得所说的是什么。
\par }{\SH 治好耶利哥的瞎子
\par }{\R (太20·29—34;可10·46—52)
\par }{\PP \VS{35}耶稣将近{\PN{耶利哥}}的时候,有一个瞎子坐在路旁讨饭。
\VS{36}听见许多人经过,就问是什么事。
\VS{37}他们告诉他,是{\PN{拿撒勒}}人耶稣经过。
\VS{38}他就呼叫说:「{\PN{大卫}}的子孙耶稣啊,可怜我吧!」
\VS{39}在前头走的人就责备他,不许他作声;他却越发喊叫说:「{\PN{大卫}}的子孙,可怜我吧!」
\VS{40}耶稣站住,吩咐把他领过来,到了跟前,就问他说:
\VS{41}「你要我为你做什么?」他说:「主啊,我要能看见。」
\VS{42}耶稣说:「你可以看见!你的信救了你了。」
\VS{43}瞎子立刻看见了,就跟随耶稣,{\ADD{一路}}归荣耀与 神。众人看见这事,也赞美 神。

\par }\Chap{19}{\SH 耶稣和撒该
\par }{\PP \VerseOne{1}耶稣进了{\PN{耶利哥}},正经过的时候,
\VS{2}有一个人名叫{\PN{撒该}},作税吏长,是个财主。
\VS{3}他要看看耶稣是怎样的人;只因人多,他的身量又矮,所以不得看见,
\VS{4}就跑到前头,爬上桑树,要看耶稣,因为耶稣必从那里经过。
\VS{5}耶稣到了那里,抬头一看,对他说:「{\PN{撒该}},快下来!今天我必住在你家里。」
\VS{6}他就急忙下来,欢欢喜喜地接待耶稣。
\VS{7}众人看见,都私下议论说:「他竟到罪人家里去住宿。」
\VS{8}{\PN{撒该}}站着对主说:「主啊,我把所有的一半给穷人;我若讹诈了谁,就还他四倍。」
\VS{9}耶稣说:「今天救恩到了这家,因为他也是{\PN{亚伯拉罕}}的子孙。
\VS{10}人子来,为要寻找、拯救失丧的人。」
\par }{\SH 十锭银子的比喻
\par }{\R (太25·14—30)
\par }{\PP \VS{11}众人正在听见这些话的时候,耶稣因为将近{\PN{耶路撒冷}},又因他们以为 神的国快要显出来,就另设一个比喻,说:
\VS{12}「有一个贵胄往远方去,要得国回来,
\VS{13}便叫了他的十个仆人来,交给他们十锭\FTNT{}{{\FR 19:13: }锭:原文作弥拿,一弥拿约银十两}银子,说:『你们去做生意,直等我回来。』
\VS{14}他本国的人却恨他,打发使者随后去,说:『我们不愿意这个人作我们的王。』
\VS{15}他既得国回来,就吩咐叫那领银子的仆人来,要知道他们做生意赚了多少。
\VS{16}头一个上来,说:『主啊,你的一锭银子已经赚了十锭。』
\VS{17}主人说:『好!良善的仆人,你既在最小的事上有忠心,可以有权柄管十座城。』
\VS{18}第二个来,说:『主啊,你的一锭银子已经赚了五锭。』
\VS{19}主人说:『你也可以管五座城。』
\VS{20}又有一个来说:『主啊,看哪,你的一锭银子在这里,我把它包在手巾里存着。
\VS{21}我原是怕你,因为你是严厉的人;没有放下的,还要去拿,没有种下的,还要去收。』
\VS{22}主人对他说:『你这恶仆,我要凭你的口定你的罪。你既知道我是严厉的人,没有放下的,还要去拿,没有种下的,还要去收,
\VS{23}为什么不把我的银子交给银行,等我来的时候,连本带利都可以要回来呢?』
\VS{24}就对旁边站着的人说:『夺过他这一锭来,给那有十锭的。』
\VS{25}他们说:『主啊,他已经有十锭了。』
\VS{26}{\ADD{主人说}}:『我告诉你们,凡有的,还要加给他;没有的,连他所有的也要夺过来。
\VS{27}至于我那些仇敌,不要我作他们王的,把他们拉来,在我面前杀了吧!』」
\par }{\SH 光荣地进耶路撒冷
\par }{\R (太21·1—11;可11·1—11;约12·12—19)
\par }{\PP \VS{28}耶稣说完了这话,就在前面走,上{\PN{耶路撒冷}}去。
\VS{29}将近{\PN{伯法其}}和{\PN{伯大尼}},在一座山名叫{\PN{橄榄山}}那里,就打发两个门徒,说:
\VS{30}「你们往对面村子里去,进去的时候,必看见一匹驴驹拴在那里,是从来没有人骑过的,可以解开牵来。
\VS{31}若有人问为什么解它,你们就说:『主要用它。』」
\VS{32}打发的人去了,所遇见的正如耶稣所说的。
\VS{33}他们解驴驹的时候,主人问他们说:「解驴驹做什么?」
\VS{34}他们说:「主要用它。」
\VS{35}他们牵到耶稣那里,把自己的衣服搭在上面,扶着耶稣骑上。
\VS{36}走的时候,众人把衣服铺在路上。
\VS{37}将近{\PN{耶路撒冷}},正下{\PN{橄榄山}}的时候,众门徒因所见过的一切异能,都欢乐起来,大声赞美 神,
\VS{38}说:
\par }{\Q 奉主名来的王是应当称颂的!
\par }{\Q 在天上有和平;
\par }{\Q 在至高之处有荣光。
\par }{\PP \VS{39}众人中有几个法利赛人对耶稣说:「夫子,责备你的门徒吧!」
\VS{40}耶稣说:「我告诉你们,若是他们闭口不说,这些石头必要呼叫起来。」
\par }{\PP \VS{41}耶稣快到{\PN{耶路撒冷}},看见城,就为它哀哭,
\VS{42}说:「巴不得你在这日子知道关系你平安的事;无奈这事现在是隐藏的,叫你的眼看不出来。
\VS{43}因为日子将到,你的仇敌必筑起土垒,周围环绕你,四面困住你,
\VS{44}并要扫灭你和你里头的儿女,连一块石头也不留在石头上,因你不知道眷顾你的时候。」
\par }{\SH 洁净圣殿
\par }{\R (太21·12—17;可11·15—19;约2·13—22)
\par }{\PP \VS{45}耶稣进了殿,赶出里头做买卖的人,
\VS{46}对他们说:「{\ADD{经上}}说:
\par }{\Q 我的殿必作祷告的殿,
\par }{\Q 你们倒使它成为贼窝了。」
\par }{\PP \VS{47}耶稣天天在殿里教训人。祭司长和文士与百姓的尊长都想要杀他,
\VS{48}但寻不出法子来,因为百姓都侧耳听他。

\par }\Chap{20}{\SH 质问耶稣的权柄
\par }{\R (太21·23—27;可11·27—33)
\par }{\PP \VerseOne{1}有一天,耶稣在殿里教训百姓,讲福音的时候,祭司长和文士并长老上前来,
\VS{2}问他说:「你告诉我们,你仗着什么权柄做这些事?给你这权柄的是谁呢?」
\VS{3}耶稣回答说:「我也要问你们一句话,你们且告诉我。
\VS{4}{\PN{约翰}}的洗礼是从天上来的?是从人间来的呢?」
\VS{5}他们彼此商议说:「我们若说『从天上来』,他必说:『你们为什么不信他呢?』
\VS{6}若说『从人间来』,百姓都要用石头打死我们,因为他们信{\PN{约翰}}是先知。」
\VS{7}于是回答说:「不知道是从哪里来的。」
\VS{8}耶稣说:「我也不告诉你们,我仗着什么权柄做这些事。」
\par }{\SH 凶恶园户的比喻
\par }{\R (太21·33—46;可12·1—12)
\par }{\PP \VS{9}耶稣就设比喻对百姓说:「有人栽了一个葡萄园,租给园户,就往外国去住了许久。
\VS{10}到了时候,打发一个仆人到园户那里去,叫他们把园中当纳的果子交给他;园户竟打了他,叫他空手回去。
\VS{11}又打发一个仆人去,他们也打了他,并且凌辱他,叫他空手回去。
\VS{12}又打发第三个仆人去,他们也打伤了他,把他推出去了。
\VS{13}园主说:『我怎么办呢?我要打发我的爱子去,或者他们尊敬他。』
\VS{14}不料,园户看见他,就彼此商量说:『这是承受产业的,我们杀他吧,使产业归于我们!』
\VS{15}于是把他推出葡萄园外,杀了。这样,葡萄园的主人要怎样处治他们呢?
\VS{16}他要来除灭这些园户,将葡萄园转给别人。」听见的人说:「这是万不可的!」
\VS{17}耶稣看着他们说:「{\ADD{经上}}记着:
\par }{\Q 匠人所弃的石头
\par }{\Q 已作了房角的头块石头。
\par }{\MM 这是什么意思呢?
\VS{18}凡掉在那石头上的,必要跌碎;那石头掉在谁的身上,就要把谁砸得稀烂。」
\par }{\SH 纳税给凯撒的问题
\par }{\R (太22·15—22;可12·13—17)
\par }{\PP \VS{19}文士和祭司长看出这比喻是指着他们说的,当时就想要下手拿他,只是惧怕百姓。
\VS{20}于是窥探耶稣,打发奸细装作好人,要在他的话上得把柄,好将他交在巡抚的政权之下。
\VS{21}奸细就问耶稣说:「夫子,我们晓得你所讲所传都是正道,也不取人的外貌,乃是诚诚实实传 神的道。
\VS{22}我们纳税给凯撒,可以不可以?」
\VS{23}耶稣看出他们的诡诈,就对他们说:
\VS{24}「拿一个银钱来给我看。这像和这号是谁的?」他们说:「是凯撒的。」
\VS{25}耶稣说:「这样,凯撒的物当归给凯撒, 神的物当归给 神。」
\VS{26}他们当着百姓,在这话上得不着把柄,又希奇他的应对,就闭口无言了。
\par }{\SH 复活的问题
\par }{\R (太22·23—33;可12·18—27)
\par }{\PP \VS{27}撒都该人常说没有复活的事。有几个来问耶稣说:
\VS{28}「夫子!{\PN{摩西}}为我们写着说:『人若有妻无子就死了,他兄弟当娶他的妻,为哥哥生子立后。』
\VS{29}有弟兄七人,第一个娶了妻,没有孩子死了;
\VS{30}第二个、第三个也娶过她;
\VS{31}那七个人都娶过她,没有留下孩子就死了。
\VS{32}后来妇人也死了。
\VS{33}这样,当复活的时候,她是哪一个的妻子呢?因为他们七个人都娶过她。」
\VS{34}耶稣说:「这世界的人有娶有嫁;
\VS{35}惟有算为配得那世界,与从死里复活的人也不娶也不嫁;
\VS{36}因为他们不能再死,和天使一样;既是复活的人,就为 神的儿子。
\VS{37}至于死人复活,{\PN{摩西}}在荆棘篇上,称主是{\PN{亚伯拉罕}}的 神,{\PN{以撒}}的 神,{\PN{雅各}}的 神,就指示明白了。
\VS{38}神原不是死人的 神,乃是活人的 神;因为在他那里\FTNT{}{{\FR 20:38: }那里:或译看来},人都是活的。」
\VS{39}有几个文士说:「夫子!你说得好。」
\VS{40}以后他们不敢再问他什么。
\par }{\SH 大卫子孙的问题
\par }{\R (太22·41—46;可12·35—37)
\par }{\PP \VS{41}耶稣对他们说:「人怎么说基督是{\PN{大卫}}的子孙呢?
\VS{42}诗篇上{\PN{大卫}}自己说:
\par }{\Q 主对我主说:
\par }{\Q 你坐在我的右边,
\par }{\Q \VS{43}等我使你仇敌作你的脚凳。
\par }{\PP \VS{44}{\PN{大卫}}既称他为主,他怎么又是{\PN{大卫}}的子孙呢?」
\par }{\SH 谴责文士
\par }{\R (太23·1—36;可12·38—40)
\par }{\PP \VS{45}众百姓听的时候,耶稣对门徒说:
\VS{46}「你们要防备文士。他们好穿长衣游行,喜爱人在街市上问他们安,又喜爱会堂里的高位,筵席上的首座;
\VS{47}他们侵吞寡妇的家产,假意作很长的祷告。这些人要受更重的刑罚!」

\par }\Chap{21}{\SH 寡妇的奉献
\par }{\R (可12·41—44)
\par }{\PP \VerseOne{1}耶稣抬头观看,见财主把捐项投在库里,
\VS{2}又见一个穷寡妇投了两个小钱,
\VS{3}就说:「我实在告诉你们,这穷寡妇所投的比众人还多;
\VS{4}因为众人都是自己有余,拿出来投在捐项里,但这寡妇是自己不足,把她一切养生的都投上了。」
\par }{\SH 预言圣殿被毁
\par }{\R (太24·1—2;可13·1—2)
\par }{\PP \VS{5}有人谈论圣殿是用美石和供物妆饰的;
\VS{6}耶稣就说:「论到你们所看见的这一切,将来日子到了,在这里没有一块石头留在石头上,不被拆毁了。」
\par }{\SH 预兆和逼迫
\par }{\R (太24·3—14;可13·3—13)
\par }{\PP \VS{7}他们问他说:「夫子!什么时候有这事呢?这事将到的时候有什么预兆呢?」
\VS{8}耶稣说:「你们要谨慎,不要受迷惑;因为将来有好些人冒我的名来,说:『我是{\ADD{基督}}』,又说:『时候近了』,你们不要跟从他们!
\VS{9}你们听见打仗和扰乱的事,不要惊惶;因为这些事必须先有,只是末期不能立时就到。」
\VS{10}当时,耶稣对他们说:「民要攻打民,国要攻打国;
\VS{11}地要大大震动,多处必有饥荒、瘟疫,又有可怕的异象和大神迹从天上显现。
\VS{12}但这一切的事以先,人要下手拿住你们,逼迫你们,把你们交给会堂,并且收在监里,又为我的名拉你们到君王诸侯面前。
\VS{13}但这些事终必为你们的见证。
\VS{14}所以,你们当立定心意,不要预先思想怎样分诉;
\VS{15}因为我必赐你们口才、智慧,是你们一切敌人所敌不住、驳不倒的。
\VS{16}连你们的父母、弟兄、亲族、朋友也要把你们交官;你们也有被他们害死的。
\VS{17}你们要为我的名被众人恨恶,
\VS{18}然而,你们连一根头发也必不损坏。
\VS{19}你们常存忍耐,就必保全灵魂\FTNT{}{{\FR 21:19: }或译:必得生命}。」
\par }{\SH 预言耶路撒冷被毁
\par }{\R (太24·15—21;可13·14—19)
\par }{\PP \VS{20}「你们看见{\PN{耶路撒冷}}被兵围困,就可知道它成荒场的日子近了。
\VS{21}那时,在{\PN{犹太}}的应当逃到山上;在城里的应当出来;在乡下的不要进城;
\VS{22}因为这是报应的日子,使{\ADD{经上}}所写的都得应验。
\VS{23}当那些日子,怀孕的和奶孩子的有祸了!因为将有大灾难降在这地方,也有震怒临到这百姓。
\VS{24}他们要倒在刀下,又被掳到各国去。{\PN{耶路撒冷}}要被外邦人践踏,直到外邦人的日期满了。」
\par }{\SH 人子的降临
\par }{\R (太24·29—31;可13·24—27)
\par }{\PP \VS{25}「日、月、星辰要显出异兆,地上的邦国也有困苦;因海中波浪的响声,就慌慌不定。
\VS{26}天势都要震动,人想起那将要临到世界的事,就都吓得魂不附体。
\VS{27}那时,他们要看见人子有能力,有大荣耀驾云降临。
\VS{28}一有这些事,你们就当挺身昂首,因为你们得赎的日子近了。」
\par }{\SH 从无花果树学功课
\par }{\R (太24·32—35;可13·28—31)
\par }{\PP \VS{29-30}耶稣又设比喻对他们说:「你们看无花果树和各样的树;它发芽的时候,你们一看见,自然晓得夏天近了。
\VS{31}这样,你们看见这些事渐渐地成就,也该晓得 神的国近了。
\VS{32}我实在告诉你们,这世代还没有过去,这些事都要成就。
\VS{33}天地要废去,我的话却不能废去。」
\par }{\SH 劝告门徒警醒
\par }{\PP \VS{34}「你们要谨慎,恐怕因贪食、醉酒,并今生的思虑累住你们的心,那日子就如同网罗忽然临到你们;
\VS{35}因为那日子要{\ADD{这样}}临到全地上一切居住的人。
\VS{36}你们要时时警醒,常常祈求,使你们能逃避这一切要来的事,得以站立在人子面前。」
\VS{37}耶稣每日在殿里教训人,每夜出城在一座山,名叫{\PN{橄榄山}}住宿。
\VS{38}众百姓清早上{\ADD{圣}}殿,到耶稣那里,要听他讲道。

\par }\Chap{22}{\SH 杀害耶稣的阴谋
\par }{\R (太26·1—5;可14·1—2;约11·45—53)
\par }{\PP \VerseOne{1}除酵节(又名逾越{\ADD{节}})近了。
\VS{2}祭司长和文士想法子怎么才能杀害耶稣,是因他们惧怕百姓。
\VS{3}这时,撒但入了那称为{\PN{加略}}人{\PN{犹大}}的心;他本是十二门徒里的一个。
\VS{4}他去和祭司长并守殿官商量,怎么可以把耶稣交给他们。
\VS{5}他们欢喜,就约定给他银子。
\VS{6}他应允了,就找机会,要趁众人不在跟前的时候把耶稣交给他们。
\par }{\SH 预备逾越节的筵席
\par }{\R (太26·17—25;可14·12—21;约13·21—30)
\par }{\PP \VS{7}除酵{\ADD{节}},须宰逾越{\ADD{羊羔}}的那一天到了。
\VS{8}耶稣打发{\PN{彼得}}、{\PN{约翰}},说:「你们去为我们预备逾越{\ADD{节的筵席}},好叫我们吃。」
\VS{9}他们问他说:「要我们在哪里预备?」
\VS{10}耶稣说:「你们进了城,必有人拿着一瓶水迎面而来,你们就跟着他,到他所进的房子里去,
\VS{11}对那家的主人说:『夫子说:客房在哪里?我与门徒好在那里吃逾越{\ADD{节的筵席}}。』
\VS{12}他必指给你们摆设整齐的一间大楼,你们就在那里预备。」
\VS{13}他们去了,所遇见的正如耶稣所说的;他们就预备了逾越{\ADD{节的筵席}}。
\par }{\SH 设立主的晚餐
\par }{\R (太26·26—30;可14·22—26;林前11·23—25)
\par }{\PP \VS{14}时候到了,耶稣坐席,使徒也和他同坐。
\VS{15}耶稣对他们说:「我很愿意在受害以先和你们吃这逾越{\ADD{节的筵席}}。
\VS{16}我告诉你们,我不{\ADD{再}}吃{\ADD{这筵席}},直到成就在 神的国里。」
\VS{17}耶稣接过杯来,祝谢了,说:「你们拿这个,大家分着喝。
\VS{18}我告诉你们,从今以后,我不再喝这葡萄汁,直等 神的国来到。」
\VS{19}又拿起饼来,祝谢了,就擘开,递给他们,说:「这是我的身体,为你们舍的,你们也应当如此行,为的是记念我。」
\VS{20}饭后也照样拿起杯来,说:「这杯是用我血所立的新约,是为你们流出来的。
\VS{21}看哪,那卖我之人的手与我一同在桌子上。
\VS{22}人子固然要照所预定的去世,但卖人子的人有祸了!」
\VS{23}他们就彼此对问,是哪一个要做这事。
\par }{\SH 争论谁为大
\par }{\PP \VS{24}门徒起了争论,他们中间哪一个可算为大。
\VS{25}耶稣说:「外邦人有君王为主治理他们,那掌权管他们的称为恩主。
\VS{26}但你们不可这样;你们里头为大的,倒要像年幼的,为首领的,倒要像服事人的。
\VS{27}是谁为大?是坐席的呢?是服事人的呢?不是坐席的大吗?然而,我在你们中间如同服事人的。
\VS{28}我在磨炼之中,常和我同在的就是你们。
\VS{29}我将国赐给你们,正如我父赐给我一样,
\VS{30}叫你们在我国里,坐在我的席上吃喝,并且坐在宝座上,审判{\PN{以色列}}十二个支派。」
\par }{\SH 预言彼得不认主
\par }{\R (太26·31—35;可14·27—31;约13·36—38)
\par }{\PP \VS{31}{\ADD{主又说}}:「{\PN{西门}}!{\PN{西门}}!撒但想要得着你们,好筛你们像筛麦子一样;
\VS{32}但我已经为你祈求,叫你不至于失了信心。你回头以后,要坚固你的弟兄。」
\VS{33}{\PN{彼得}}说:「主啊,我就是同你下监,同你受死,也是甘心!」
\VS{34}耶稣说:「{\PN{彼得}},我告诉你,今日鸡还没有叫,你要三次说不认得我。」
\par }{\SH 钱囊、口袋、刀
\par }{\PP \VS{35}耶稣又对他们说:「我差你们出去的时候,没有钱囊,没有口袋,没有鞋,你们缺少什么没有?」他们说:「没有。」
\VS{36}耶稣说:「但如今有钱囊的可以带着,有口袋的也可以带着,没有刀的要卖衣服买刀。
\VS{37}我告诉你们,{\ADD{经上}}写着说:『他被列在罪犯之中。』这话必应验在我身上,因为那关系我的事必然成就。」
\VS{38}他们说:「主啊,请看!这里有两把刀。」耶稣说:「够了。」
\par }{\SH 在橄榄山上祷告
\par }{\R (太26·36—46;可14·32—42)
\par }{\PP \VS{39}耶稣出来,照常往{\PN{橄榄山}}去,门徒也跟随他。
\VS{40}到了那地方,就对他们说:「你们要祷告,免得入了迷惑。」
\VS{41}于是离开他们约有扔一块石头那么远,跪下祷告,
\VS{42}说:「父啊!你若愿意,就把这杯撤去;然而,不要成就我的意思,只要成就你的意思。」
\VS{43}有一位天使从天上显现,加添他的力量。
\VS{44}耶稣极其伤痛,祷告更加恳切,汗珠如大血点滴在地上。
\VS{45}祷告完了,就起来,到门徒那里,见他们因为忧愁都睡着了,
\VS{46}就对他们说:「你们为什么睡觉呢?起来祷告,免得入了迷惑!」
\par }{\SH 耶稣被捕
\par }{\R (太26·47—56;可14·43—50;约18·3—11)
\par }{\PP \VS{47}说话之间,来了许多人。那十二个门徒里名叫{\PN{犹大}}的,走在前头,就近耶稣,要与他亲嘴。
\VS{48}耶稣对他说:「{\PN{犹大}}!你用亲嘴{\ADD{的暗号}}卖人子吗?」
\VS{49}左右的人见光景不好,就说:「主啊!我们拿刀砍可以不可以?」
\VS{50}内中有一个人把大祭司的仆人砍了一刀,削掉了他的右耳。
\VS{51}耶稣说:「到了这个地步,由他们吧!」就摸那人的耳朵,把他治好了。
\VS{52}耶稣对那些来拿他的祭司长和守殿官并长老说:「你们带着刀棒出来拿我,如同拿强盗吗?
\VS{53}我天天同你们在殿里,你们不下手拿我。现在却是你们的时候,黑暗掌权了。」
\par }{\SH 彼得三次不认主
\par }{\R (太26·57—58,69—75;可14·53—54,66—72;约18·12—18,25—27)
\par }{\PP \VS{54}他们拿住耶稣,把他带到大祭司的宅里。{\PN{彼得}}远远地跟着。
\VS{55}他们在院子里生了火,一同坐着;{\PN{彼得}}也坐在他们中间。
\VS{56}有一个使女看见{\PN{彼得}}坐在{\ADD{火}}光里,就定睛看他,说:「这个人素来也是同那人一伙的。」
\VS{57}{\PN{彼得}}却不承认,说:「女子,我不认得他。」
\VS{58}过了不多的时候,又有一个人看见他,说:「你也是他们一党的。」{\PN{彼得}}说:「你这个人!我不是。」
\VS{59}约过了一小时,又有一个人极力地说:「他实在是同那人一伙的,因为他也是{\PN{加利利}}人。」
\VS{60}{\PN{彼得}}说:「你这个人!我不晓得你说的是什么!」正说话之间,鸡就叫了。
\VS{61}主转过身来看{\PN{彼得}},{\PN{彼得}}便想起主对他所说的话:「今日鸡叫以先,你要三次不认我。」
\VS{62}他就出去痛哭。
\par }{\SH 戏弄鞭打耶稣
\par }{\R (太26·67—68;可14·65)
\par }{\PP \VS{63}看守耶稣的人戏弄他,打他,
\VS{64}又蒙着他的眼,问他说:「你是先知,告诉我们打你的是谁?」
\VS{65}他们还用许多别的话辱骂他。
\par }{\SH 耶稣在公会里受审
\par }{\R (太26·59—66;可14·55—64;约18·19—24)
\par }{\PP \VS{66}天一亮,民间的众长老连祭司长带文士都聚会,把耶稣带到他们的公会里,
\VS{67}说:「你若是基督,就告诉我们。」耶稣说:「我若告诉你们,你们也不信;
\VS{68}我若问你们,你们也不回答。
\VS{69}从今以后,人子要坐在 神权能的右边。」
\VS{70}他们都说:「这样,你是 神的儿子吗?」耶稣说:「你们所说的是。」
\VS{71}他们说:「何必再用见证呢?他亲口所说的,我们都亲自听见了。」

\par }\Chap{23}{\SH 耶稣在彼拉多面前受审
\par }{\R (太27·1—2,11—14;可15·1—5;约18·28—38)
\par }{\PP \VerseOne{1}众人都起来,把耶稣解到{\PN{彼拉多}}面前,
\VS{2}就告他说:「我们见这人诱惑国民,禁止纳税给凯撒,并说自己是基督,是王。」
\VS{3}{\PN{彼拉多}}问耶稣说:「你是{\PN{犹太}}人的王吗?」耶稣回答说:「你说的是。」
\VS{4}{\PN{彼拉多}}对祭司长和众人说:「我查不出这人有什么罪来。」
\VS{5}但他们越发极力地说:「他煽惑百姓,在{\PN{犹太}}遍地传道,从{\PN{加利利}}起,直到这里了。」
\par }{\SH 希律藐视耶稣
\par }{\PP \VS{6}{\PN{彼拉多}}一听见,就问:「这人是{\PN{加利利}}人吗?」
\VS{7}既晓得耶稣属{\PN{希律}}所管,就把他送到{\PN{希律}}那里去。那时{\PN{希律}}正在{\PN{耶路撒冷}}。
\VS{8}{\PN{希律}}看见耶稣,就很欢喜;因为听见过他的事,久已想要见他,并且指望看他行一件神迹,
\VS{9}于是问他许多的话,耶稣却一言不答。
\VS{10}祭司长和文士都站着,极力地告他。
\VS{11}{\PN{希律}}和他的兵丁就藐视耶稣,戏弄他,给他穿上华丽衣服,把他送回{\PN{彼拉多}}那里去。
\VS{12}从前{\PN{希律}}和{\PN{彼拉多}}彼此有仇,在那一天就成了朋友。
\par }{\SH 耶稣被判死刑
\par }{\R (太27·15—26;可15·6—15;约18·39—19·16)
\par }{\PP \VS{13}{\PN{彼拉多}}传齐了祭司长和官府并百姓,
\VS{14}就对他们说:「你们解这人到我这里,说他是诱惑百姓的。看哪,我也曾将你们告他的事,在你们面前审问他,并没有查出他什么罪来;
\VS{15}就是{\PN{希律}}也是如此,所以把他送回来。可见他没有做什么该死的事。
\VS{16}故此,我要责打他,把他释放了。」\FTNT{}{{\FR 23:16: }有古卷加:17 每逢这节期,巡抚必须释放一个囚犯给他们。}
\VS{18}众人却一齐喊着说:「除掉这个人!释放{\PN{巴拉巴}}给我们!」
\VS{19}这{\PN{巴拉巴}}是因在城里作乱杀人,下在监里的。
\VS{20}{\PN{彼拉多}}愿意释放耶稣,就又劝解他们。
\VS{21}无奈他们喊着说:「钉他十字架!钉他十字架!」
\VS{22}{\PN{彼拉多}}第三次对他们说:「为什么呢?这人做了什么恶事呢?我并没有查出他什么该死的罪来。所以,我要责打他,把他释放了。」
\VS{23}他们大声催逼{\PN{彼拉多}},求他把耶稣钉在十字架上。他们的声音就得了胜。
\VS{24}{\PN{彼拉多}}这才照他们所求的定案,
\VS{25}把他们所求的那作乱杀人、下在监里的释放了,把耶稣交给他们,任凭他们的意思行。
\par }{\SH 耶稣被钉十字架
\par }{\R (太27·32—44;可15·21—32;约19·17—27)
\par }{\PP \VS{26}带耶稣去的时候,有一个{\PN{古利奈}}人{\PN{西门}},从乡下来;他们就抓住他,把十字架搁在他身上,叫他背着跟随耶稣。
\VS{27}有许多百姓跟随耶稣,内中有好些妇女;妇女们为他号咷痛哭。
\VS{28}耶稣转身对她们说:「{\PN{耶路撒冷}}的女子,不要为我哭,当为自己和自己的儿女哭。
\VS{29}因为日子要到,人必说:『不生育的,和未曾怀胎的,未曾乳养婴孩的,有福了!』
\par }{\PP \VS{30}那时,人要向大山说:
\par }{\Q 倒在我们身上!
\par }{\Q 向小山说:
\par }{\Q 遮盖我们!
\par }{\PP \VS{31}「这些事既行在有汁水的树上,那枯干的树将来怎么样呢?」
\VS{32}又有两个犯人,和耶稣一同带来处死。
\VS{33}到了一个地方,名叫「髑髅地」,就在那里把耶稣钉在十字架上,又钉了两个犯人:一个在左边,一个在右边。
\VS{34}当下耶稣说:「父啊!赦免他们;因为他们所做的,他们不晓得。」{\ADD{兵丁}}就拈阄分他的衣服。
\VS{35}百姓站在那里观看。官府也嗤笑他,说:「他救了别人;他若是基督, 神所拣选的,可以救自己吧!」
\VS{36}兵丁也戏弄他,上前拿醋送给他喝,
\VS{37}说:「你若是{\PN{犹太}}人的王,可以救自己吧!」
\VS{38}在耶稣以上有一个牌子\FTNT{}{{\FR 23:38: }有古卷加:用希腊、罗马、希伯来的文字}写着:「这是{\PN{犹太}}人的王。」
\par }{\PP \VS{39}那同钉的两个犯人有一个讥笑他,说:「你不是基督吗?可以救自己和我们吧!」
\VS{40}那一个就应声责备他,说:「你既是一样受刑的,还不怕 神吗?
\VS{41}我们是应该的,因我们所受的与我们所做的相称,但这个人没有做过一件不好的事。」
\VS{42}就说:「耶稣啊,你得国降临的时候,求你记念我!」
\VS{43}耶稣对他说:「我实在告诉你,今日你要同我在乐园里了。」
\par }{\SH 耶稣的死
\par }{\R (太27·45—56;可15·33—41;约19·28—30)
\par }{\PP \VS{44}那时约有午正,遍地都黑暗了,直到申初,
\VS{45}日头变黑了;殿里的幔子从当中裂为两半。
\VS{46}耶稣大声喊着说:「父啊!我将我的灵魂交在你手里。」说了这话,气就断了。
\VS{47}百夫长看见所成的事,就归荣耀与 神,说:「这真是个义人!」
\VS{48}聚集观看的众人见了这所成的事都捶着胸回去了。
\VS{49}还有一切与耶稣熟识的人,和从{\PN{加利利}}跟着他来的妇女们,都远远地站着看这些事。
\par }{\SH 耶稣的安葬
\par }{\R (太27·57—61;可15·42—47;约19·38—42)
\par }{\PP \VS{50}有一个人名叫{\PN{约瑟}},是个议士,为人善良公义;
\VS{51}众人所谋所为,他并没有附从。他本是{\PN{犹太}}、{\PN{亚利马太}}城里素常盼望 神国的人。
\VS{52}这人去见{\PN{彼拉多}},求耶稣的身体,
\VS{53}就取下来,用细麻布裹好,安放在石头凿成的坟墓里;那里头从来没有葬过人。
\VS{54}那日是预备日,安息日也快到了。
\VS{55}那些从{\PN{加利利}}和耶稣同来的妇女跟在后面,看见了坟墓和他的身体怎样安放。
\VS{56}她们就回去,预备了香料香膏。她们在安息日,便遵着诫命安息了。

\par }\Chap{24}{\SH 耶稣复活
\par }{\R (太28·1—10;可16·1—8;约20·1—10)
\par }{\PP \VerseOne{1}七日的头一日,黎明的时候,那些妇女带着所预备的香料来到坟墓前,
\VS{2}看见石头已经从坟墓滚开了,
\VS{3}她们就进去,只是不见主耶稣的身体。
\VS{4}正在猜疑之间,忽然有两个人站在旁边,衣服放光。
\VS{5}妇女们惊怕,将脸伏地。那两个人就对她们说:「为什么在死人中找活人呢?
\VS{6}他不在这里,已经复活了。当记念他还在{\PN{加利利}}的时候怎样告诉你们,
\VS{7}说:『人子必须被交在罪人手里,钉在十字架上,第三日复活。』」
\VS{8}她们就想起耶稣的话来,
\VS{9}便从坟墓那里回去,把这一切的事告诉十一个使徒和其余的人。
\VS{10}那告诉使徒的就是{\PN{抹大拉}}的{\PN{马利亚}}和{\PN{约亚拿}},并{\PN{雅各}}的母亲{\PN{马利亚}},还有与她们在一处的妇女。
\VS{11}她们这些话,使徒以为是胡言,就不相信。
\VS{12}{\PN{彼得}}起来,跑到坟墓前,低头往里看,见细麻布独在一处,就回去了,心里希奇所成的事。
\par }{\SH 在以马忤斯的路上
\par }{\R (可16·12—13)
\par }{\PP \VS{13}正当那日,门徒中有两个人往一个村子去;这村子名叫{\PN{以马忤斯}},离{\PN{耶路撒冷}}约有二十五里。
\VS{14}他们彼此谈论所遇见的这一切事。
\VS{15}正谈论相问的时候,耶稣亲自就近他们,和他们同行;
\VS{16}只是他们的眼睛迷糊了,不认识他。
\VS{17}耶稣对他们说:「你们走路彼此谈论的是什么事呢?」他们就站住,脸上带着愁容。
\VS{18}二人中有一个名叫{\PN{革流巴}}的回答说:「你在{\PN{耶路撒冷}}作客,还不知道这几天在那里所出的事吗?」
\VS{19}耶稣说:「什么事呢?」他们说:「就是{\PN{拿撒勒}}人耶稣的事。他是个先知,在 神和众百姓面前,说话行事都有大能。
\VS{20}祭司长和我们的官府竟把他解去,定了死罪,钉在十字架上。
\VS{21}但我们素来所盼望、要赎{\PN{以色列}}民的就是他!不但如此,而且这事成就,现在已经三天了。
\VS{22}再者,我们中间有几个妇女使我们惊奇;她们清早到了坟墓那里,
\VS{23}不见他的身体,就回来告诉我们,说看见了天使显现,说他活了。
\VS{24}又有我们的几个人往坟墓那里去,所遇见的正如妇女们所说的,只是没有看见他。」
\VS{25}耶稣对他们说:「无知的人哪,先知所说的一切话,你们的心信得太迟钝了。
\VS{26}基督这样受害,又进入他的荣耀,岂不是应当的吗?」
\VS{27}于是从{\PN{摩西}}和众先知起,凡经上所指着自己的话都给他们讲解明白了。
\par }{\PP \VS{28}将近他们所去的村子,耶稣好像还要往前行,
\VS{29}他们却强留他,说:「时候晚了,日头已经平西了,请你同我们住下吧!」耶稣就进去,要同他们住下。
\VS{30}到了坐席的时候,耶稣拿起饼来,祝谢了,擘开,递给他们。
\VS{31}他们的眼睛明亮了,这才认出他来。忽然耶稣不见了。
\VS{32}他们彼此说:「在路上,他和我们说话,给我们讲解圣经的时候,我们的心岂不是火热的吗?」
\VS{33}他们就立时起身,回{\PN{耶路撒冷}}去,正遇见十一个使徒和他们的同人聚集在一处,
\VS{34}说:「主果然复活,已经现给{\PN{西门}}看了。」
\VS{35}两个人就把路上{\ADD{所遇见}},和擘饼的时候怎么被他们认出来的事,都述说了一遍。
\par }{\SH 向门徒显现
\par }{\R (太28·16—20;可16·14—18;约20·19—23;徒1·6—8)
\par }{\PP \VS{36}正说这话的时候,耶稣亲自站在他们当中,说:「愿你们平安!」
\VS{37}他们却惊慌害怕,以为所看见的是魂。
\VS{38}耶稣说:「你们为什么愁烦?为什么心里起疑念呢?
\VS{39}你们看我的手,我的脚,就{\ADD{知道}}实在是我了。摸我看看!魂无骨无肉,你们看,我是有的。」
\VS{40}说了这话,就把手和脚给他们看。
\VS{41}他们正喜得不敢信,并且希奇;耶稣就说:「你们这里有什么吃的没有?」
\VS{42}他们便给他一片烧鱼。\FTNT{}{{\FR 24:42: }有古卷加:和一块蜜房。}
\VS{43}他接过来,在他们面前吃了。
\par }{\PP \VS{44}耶稣对他们说:「这就是我从前与你们同在之时所告诉你们的话说:{\PN{摩西}}的律法、先知的书,和诗篇上所记的,凡指着我的话都必须应验。」
\VS{45}于是耶稣开他们的心窍,使他们能明白圣经,
\VS{46}又对他们说:「照{\ADD{经上}}所写的,基督必受害,第三日从死里复活,
\VS{47}并且人要奉他的名传悔改、赦罪的道,从{\PN{耶路撒冷}}起直传到万邦。
\VS{48}你们就是这些事的见证。
\VS{49}我要将我父所应许的降在你们身上,你们要在城里等候,直到你们领受从上头来的能力。」
\par }{\SH 耶稣升天
\par }{\R (可16·19—20;徒1·9—11)
\par }{\PP \VS{50}耶稣领他们到{\PN{伯大尼}}的对面,就举手给他们祝福。
\VS{51}正祝福的时候,他就离开他们,被带到天上去了。
\VS{52}他们就拜他,大大地欢喜,回{\PN{耶路撒冷}}去,
\VS{53}常在殿里称颂 神。
\par }