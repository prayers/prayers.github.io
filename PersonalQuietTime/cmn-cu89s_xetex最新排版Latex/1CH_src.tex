\NormalFont\ShortTitle{历代志上}
{\MT 历代志上

\par }\ChapOne{1}{\SH 从亚当到亚伯拉罕
\par }{\R (创5·1—32;10·1—32;11·10—26)
\par }{\PP \VerseOne{1}{\PN{亚当}}{\ADD{生}}{\PN{塞特}};{\PN{塞特}}{\ADD{生}}{\PN{以挪士}};
\VS{2}{\PN{以挪士}}{\ADD{生}}{\PN{该南}};{\PN{该南}}{\ADD{生}}{\PN{玛勒列}};{\PN{玛勒列}}{\ADD{生}}{\PN{雅列}};
\VS{3}{\PN{雅列}}{\ADD{生}}{\PN{以诺}};{\PN{以诺}}{\ADD{生}}{\PN{玛土撒拉}};{\PN{玛土撒拉}}{\ADD{生}}{\PN{拉麦}};
\VS{4}{\PN{拉麦}}{\ADD{生}}{\PN{挪亚}};{\PN{挪亚}}{\ADD{生}}{\PN{闪}}、{\PN{含}}、{\PN{雅弗}}。
\par }{\PP \VS{5}{\PN{雅弗}}的儿子是{\PN{歌篾}}、{\PN{玛各}}、{\PN{玛代}}、{\PN{雅完}}、{\PN{土巴}}、{\PN{米设}}、{\PN{提拉}}。
\VS{6}{\PN{歌篾}}的儿子是{\PN{亚实基拿}}、{\PN{低法}}\FTNT{}{{\FR 1:6: }在创世记十章三节是利法}、{\PN{陀迦玛}}。
\VS{7}{\PN{雅完}}的儿子是{\PN{以利沙}}、{\PN{他施}}、{\PN{基提}}、{\PN{多单}}\FTNT{}{{\FR 1:7: }有作罗单的}。
\par }{\PP \VS{8}{\PN{含}}的儿子是{\PN{古实}}、{\PN{麦西}}、{\PN{弗}}、{\PN{迦南}}。
\VS{9}{\PN{古实}}的儿子是{\PN{西巴}}、{\PN{哈腓拉}}、{\PN{撒弗他}}、{\PN{拉玛}}、{\PN{撒弗提迦}}。{\PN{拉玛}}的儿子是{\PN{示巴}}、{\PN{底但}}。
\VS{10}{\PN{古实}}生{\PN{宁录}};他为世上英雄之首。
\par }{\PP \VS{11}{\PN{麦西}}生{\PN{路低}}人、{\PN{亚拿米}}人、{\PN{利哈比}}人、{\PN{拿弗土希}}人、
\VS{12}{\PN{帕斯鲁细}}人、{\PN{迦斯路希}}人、{\PN{迦斐托}}人;从{\PN{迦斐托}}出来的有{\PN{非利士}}人。
\par }{\PP \VS{13}{\PN{迦南}}生长子{\PN{西顿}},又生{\PN{赫}}
\VS{14}和{\PN{耶布斯}}人、{\PN{亚摩利}}人、{\PN{革迦撒}}人、
\VS{15}{\PN{希未}}人、{\PN{亚基}}人、{\PN{西尼}}人、
\VS{16}{\PN{亚瓦底}}人、{\PN{洗玛利}}人,并{\PN{哈马}}人。
\par }{\PP \VS{17}{\PN{闪}}的儿子是{\PN{以拦}}、{\PN{亚述}}、{\PN{亚法}}
{\PN{撒}}、{\PN{路德}}、{\PN{亚兰}}、{\PN{乌斯}}、{\PN{户勒}}、{\PN{基帖}}、{\PN{米设}}\FTNT{}{{\FR 1:17: }在创世记十章二十三节是玛施}。
\VS{18}{\PN{亚法撒}}生{\PN{沙拉}};{\PN{沙拉}}生{\PN{希伯}}。
\VS{19}{\PN{希伯}}生了两个儿子:一个名叫{\PN{法勒}}\FTNT{}{{\FR 1:19: }就是分的意思},因为那时人就分地居住;{\PN{法勒}}的兄弟名叫{\PN{约坍}}。
\VS{20}{\PN{约坍}}生{\PN{亚摩答}}、{\PN{沙列}}、{\PN{哈萨玛非}}、{\PN{耶拉}}、
\VS{21}{\PN{哈多兰}}、{\PN{乌萨}}、{\PN{德拉}}、
\VS{22}{\PN{以巴录}}、{\PN{亚比玛利}}、{\PN{示巴}}、
\VS{23}{\PN{阿斐}}、{\PN{哈腓拉}}、{\PN{约巴}}。这都是{\PN{约坍}}的儿子。
\par }{\PP \VS{24}{\PN{闪}}{\ADD{生}}{\PN{亚法撒}};{\PN{亚法撒}}{\ADD{生}}{\PN{沙拉}};
\VS{25}{\PN{沙拉}}{\ADD{生}}{\PN{希伯}};{\PN{希伯}}{\ADD{生}}{\PN{法勒}};{\PN{法勒}}{\ADD{生}}{\PN{拉吴}};
\VS{26}{\PN{拉吴}}{\ADD{生}}{\PN{西鹿}};{\PN{西鹿}}{\ADD{生}}{\PN{拿鹤}};{\PN{拿鹤}}{\ADD{生}}{\PN{他拉}};
\VS{27}{\PN{他拉}}{\ADD{生}}{\PN{亚伯兰}},{\PN{亚伯兰}}就是{\PN{亚伯拉罕}}。
\par }{\SH 以实玛利的后裔
\par }{\R (创25·12—16)
\par }{\PP \VS{28}{\PN{亚伯拉罕}}的儿子是{\PN{以撒}}、{\PN{以实玛利}}。
\VS{29}{\PN{以实玛利}}的儿子记在下面:{\PN{以实玛利}}的长子是{\PN{尼拜约}},其次是{\PN{基达}}、{\PN{押德别}}、{\PN{米比衫}}、
\VS{30}{\PN{米施玛}}、{\PN{度玛}}、{\PN{玛撒}}、{\PN{哈达}}、{\PN{提玛}}、
\VS{31}{\PN{伊突}}、{\PN{拿非施}}、{\PN{基底玛}}。这都是{\PN{以实玛利}}的儿子。
\VS{32}{\PN{亚伯拉罕}}的妾{\PN{基土拉}}所生的儿子,就是{\PN{心兰}}、{\PN{约珊}}、{\PN{米但}}、{\PN{米甸}}、{\PN{伊施巴}}、{\PN{书亚}}。{\PN{约珊}}的儿子是{\PN{示巴}}、{\PN{底但}}。
\VS{33}{\PN{米甸}}的儿子是{\PN{以法}}、{\PN{以弗}}、{\PN{哈诺}}、{\PN{亚比大}}、{\PN{以勒大}}。这都是{\PN{基土拉}}的子孙。
\par }{\SH 以扫的后裔
\par }{\R (创36·1—19)
\par }{\PP \VS{34}{\PN{亚伯拉罕}}生{\PN{以撒}};{\PN{以撒}}的儿子是{\PN{以扫}}和{\PN{以色列}}。
\VS{35}{\PN{以扫}}的儿子是{\PN{以利法}}、{\PN{流珥}}、{\PN{耶乌施}}、{\PN{雅兰}}、{\PN{可拉}}。
\VS{36}{\PN{以利法}}的儿子是{\PN{提幔}}、{\PN{阿抹}}、{\PN{洗玻}}、{\PN{迦坦}}、{\PN{基纳斯}}、{\PN{亭纳}}、{\PN{亚玛力}}。
\VS{37}{\PN{流珥}}的儿子是{\PN{拿哈}}、{\PN{谢拉}}、{\PN{沙玛}}、{\PN{米撒}}。
\par }{\SH 以东地的原住民
\par }{\R (创36·20—30)
\par }{\PP \VS{38}{\PN{西珥}}的儿子是{\PN{罗坍}}、{\PN{朔巴}}、{\PN{祭便}}、{\PN{亚拿}}、{\PN{底顺}}、{\PN{以察}}、{\PN{底珊}}。
\VS{39}{\PN{罗坍}}的儿子是{\PN{何利}}、{\PN{荷幔}};{\PN{罗坍}}的妹子是{\PN{亭纳}}。
\VS{40}{\PN{朔巴}}的儿子是{\PN{亚勒文}}、{\PN{玛拿辖}}、{\PN{以巴录}}、{\PN{示非}}、{\PN{阿南}}。{\PN{祭便}}的儿子是{\PN{亚雅}}、{\PN{亚拿}}。
\VS{41}{\PN{亚拿}}的儿子是{\PN{底顺}}。{\PN{底顺}}的儿子是{\PN{哈默兰}}、{\PN{伊是班}}、{\PN{益兰}}、{\PN{基兰}}。
\VS{42}{\PN{以察}}的儿子是{\PN{辟罕}}、{\PN{撒番}}、{\PN{亚干}}。{\PN{底珊}}的儿子是{\PN{乌斯}}、{\PN{亚兰}}。
\par }{\SH 以东诸王
\par }{\R (创36·31—43)
\par }{\PP \VS{43}{\PN{以色列}}人未有君王治理之先,在{\PN{以东}}地作王的记在下面:有{\PN{比珥}}的儿子{\PN{比拉}},他的{\ADD{京}}城名叫{\PN{亭哈巴}}。
\VS{44}{\PN{比拉}}死了,{\PN{波斯拉}}人{\PN{谢拉}}的儿子{\PN{约巴}}接续他作王。
\VS{45}{\PN{约巴}}死了,{\PN{提幔}}地的人{\PN{户珊}}接续他作王。
\VS{46}{\PN{户珊}}死了,{\PN{比达}}的儿子{\PN{哈达}}接续他作王。这{\PN{哈达}}就是在{\PN{摩押}}地杀败{\PN{米甸}}人的,他的{\ADD{京}}城名叫{\PN{亚未得}}。
\VS{47}{\PN{哈达}}死了,{\PN{玛士利加}}人{\PN{桑拉}}接续他作王。
\VS{48}{\PN{桑拉}}死了,大河边的{\PN{利河伯}}人{\PN{扫罗}}接续他作王。
\VS{49}{\PN{扫罗}}死了,{\PN{亚革波}}的儿子{\PN{巴勒·哈南}}接续他作王。
\VS{50}{\PN{巴勒·哈南}}死了,{\PN{哈达}}接续他作王。他的{\ADD{京}}城名叫{\PN{巴伊}},他的妻子名叫{\PN{米希她别}},是{\PN{米·萨合}}的孙女,{\PN{玛特列}}的女儿。
\par }{\PP \VS{51}{\PN{哈达}}死了,{\PN{以东}}人的族长有{\PN{亭纳}}族长、{\PN{亚勒瓦}}族长、{\PN{耶帖}}族长、
\VS{52}{\PN{亚何利巴玛}}族长、{\PN{以拉}}族长、{\PN{比嫩}}族长、
\VS{53}{\PN{基纳斯}}族长、{\PN{提幔}}族长、{\PN{米比萨}}族长、
\VS{54}{\PN{玛基叠}}族长、{\PN{以兰}}族长。这都是{\PN{以东}}人的族长。

\par }\Chap{2}{\SH 犹大的后裔
\par }{\PP \VerseOne{1}{\PN{以色列}}的儿子是{\PN{吕便}}、{\PN{西缅}}、{\PN{利未}}、{\PN{犹大}}、{\PN{以萨迦}}、{\PN{西布伦}}、
\VS{2}{\PN{但}}、{\PN{约瑟}}、{\PN{便雅悯}}、{\PN{拿弗他利}}、{\PN{迦得}}、{\PN{亚设}}。
\VS{3}{\PN{犹大}}的儿子是{\PN{珥}}、{\PN{俄南}}、{\PN{示拉}},这三人是{\PN{迦南}}人{\PN{书亚}}女儿所生的。{\PN{犹大}}的长子{\PN{珥}}在耶和华眼中看为恶,耶和华就使他死了。
\VS{4}{\PN{犹大}}的儿妇{\PN{她玛}}给{\PN{犹大}}生{\PN{法勒斯}}和{\PN{谢拉}}。{\PN{犹大}}共有五个儿子。
\par }{\PP \VS{5}{\PN{法勒斯}}的儿子是{\PN{希斯
}}、{\PN{哈母勒}}。
\VS{6}{\PN{谢拉}}的儿子是{\PN{心利}}、{\PN{以探}}、{\PN{希幔}}、{\PN{甲各}}、{\PN{大拉}}\FTNT{}{{\FR 2:6: }就是达大},共五人。
\VS{7}{\PN{迦米}}的儿子是{\PN{亚干}},这{\PN{亚干}}在当灭的物上犯了罪,连累了{\PN{以色列}}人。
\VS{8}{\PN{以探}}的儿子是{\PN{亚撒利雅}}。
\par }{\SH 大卫王的家谱
\par }{\PP \VS{9}{\PN{希斯
}}所生的儿子是{\PN{耶拉篾}}、{\PN{兰}}、{\PN{基路拜}}。
\par }{\PP \VS{10}{\PN{兰}}生{\PN{亚米拿达}};{\PN{亚米拿达}}生{\PN{拿顺}}。{\PN{拿顺}}作{\PN{犹大}}人的首领。
\VS{11}{\PN{拿顺}}生{\PN{撒门}};{\PN{撒门}}生{\PN{波阿斯}};
\VS{12}{\PN{波阿斯}}生{\PN{俄备得}};{\PN{俄备得}}生{\PN{耶西}};
\VS{13}{\PN{耶西}}生长子{\PN{以利押}},次子{\PN{亚比拿达}},三子{\PN{示米亚}}\FTNT{}{{\FR 2:13: }示米亚就是沙玛,见撒母耳上十六章九节},
\VS{14}四子{\PN{拿坦业}},五子{\PN{拉代}},
\VS{15}六子{\PN{阿鲜}},七子{\PN{大卫}}。
\VS{16}他们的姊妹是{\PN{洗鲁雅}}和{\PN{亚比该}}。{\PN{洗鲁雅}}的儿子是{\PN{亚比筛}}、{\PN{约押}}、{\PN{亚撒黑}},共三人。
\VS{17}{\PN{亚比该}}生{\PN{亚玛撒}};{\PN{亚玛撒}}的父亲是{\PN{以实玛利}}人{\PN{益帖}}。
\par }{\SH 希斯 的后裔
\par }{\PP \VS{18}{\PN{希斯
}}的儿子{\PN{迦勒}}娶{\PN{阿苏巴}}和{\PN{耶略}}为妻,{\PN{阿苏巴}}的儿子是{\PN{耶设}}、{\PN{朔罢}}、{\PN{押墩}}。
\VS{19}{\PN{阿苏巴}}死了,{\PN{迦勒}}又娶{\PN{以法她}},生了{\PN{户珥}}。
\VS{20}{\PN{户珥}}生{\PN{乌利}};{\PN{乌利}}生{\PN{比撒列}}。
\par }{\PP \VS{21}{\PN{希斯
}}正六十岁娶了{\PN{基列}}父亲{\PN{玛吉}}的女儿,与她同房;{\PN{玛吉}}的女儿生了{\PN{西割}};
\VS{22}{\PN{西割}}生{\PN{睚珥}}。{\PN{睚珥}}在{\PN{基列}}地有二十三个城邑。
\VS{23}后来{\PN{基述}}人和{\PN{亚兰}}人夺了{\PN{睚珥}}的城邑,并{\PN{基纳}}和其乡村,共六十个。这都是{\PN{基列}}父亲{\PN{玛吉}}之子的。
\VS{24}{\PN{希斯
}}在{\PN{迦勒·以法他}}死后,他的妻{\PN{亚比雅}}给他生了{\PN{亚施户}};{\PN{亚施户}}是{\PN{提哥亚}}的父亲。
\par }{\SH 耶拉篾的后裔
\par }{\PP \VS{25}{\PN{希斯
}}的长子{\PN{耶拉篾}}生长子{\PN{兰}},又生{\PN{布拿}}、{\PN{阿连}}、{\PN{阿鲜}}、{\PN{亚希雅}}。
\VS{26}{\PN{耶拉篾}}又娶一妻名叫{\PN{亚她拉}},是{\PN{阿南}}的母亲。
\VS{27}{\PN{耶拉篾}}长子{\PN{兰}}的儿子是{\PN{玛斯}}、{\PN{雅悯}}、{\PN{以结}}。
\VS{28}{\PN{阿南}}的儿子是{\PN{沙买}}、{\PN{雅大}}。{\PN{沙买}}的儿子是{\PN{拿答}}、{\PN{亚比述}}。
\VS{29}{\PN{亚比述}}的妻名叫{\PN{亚比孩}},{\PN{亚比孩}}给他生了{\PN{亚办}}和{\PN{摩利}}。
\VS{30}{\PN{拿答}}的儿子是{\PN{西列}}、{\PN{亚遍}};{\PN{西列}}死了没有儿子。
\VS{31}{\PN{亚遍}}的儿子是{\PN{以示}};{\PN{以示}}的儿子是{\PN{示珊}};{\PN{示珊}}的儿子是{\PN{亚来}}。
\VS{32}{\PN{沙买}}兄弟{\PN{雅大}}的儿子是{\PN{益帖}}、{\PN{约拿单}};{\PN{益帖}}死了没有儿子。
\VS{33}{\PN{约拿单}}的儿子是{\PN{比勒}}、{\PN{撒萨}}。这都是{\PN{耶拉篾}}的子孙。
\VS{34}{\PN{示珊}}没有儿子,只有女儿。{\PN{示珊}}有一个仆人名叫{\PN{耶哈}},是{\PN{埃及}}人。
\VS{35}{\PN{示珊}}将女儿给了仆人{\PN{耶哈}}为妻,给他生了{\PN{亚太}}。
\VS{36}{\PN{亚太}}生{\PN{拿单}};{\PN{拿单}}生{\PN{撒拔}};
\VS{37}{\PN{撒拔}}生{\PN{以弗拉}};{\PN{以弗拉}}生{\PN{俄备得}};
\VS{38}{\PN{俄备得}}生{\PN{耶户}};{\PN{耶户}}生{\PN{亚撒利雅}};
\VS{39}{\PN{亚撒利雅}}生{\PN{希利斯}};{\PN{希利斯}}生{\PN{以利亚萨}};
\VS{40}{\PN{以利亚萨}}生{\PN{西斯买}};{\PN{西斯买}}生{\PN{沙龙}};
\VS{41}{\PN{沙龙}}生{\PN{耶加米雅}};{\PN{耶加米雅}}生{\PN{以利沙玛}}。
\par }{\SH 迦勒的其余后裔
\par }{\PP \VS{42}{\PN{耶拉篾}}兄弟{\PN{迦勒}}的长子{\PN{米沙}},是{\PN{西弗}}之祖{\PN{玛利沙}}的儿子,是{\PN{希伯
}}之祖。
\VS{43}{\PN{希伯
}}的儿子是{\PN{可拉}}、{\PN{他普亚}}、{\PN{利肯}}、{\PN{示玛}}。
\VS{44}{\PN{示玛}}生{\PN{拉含}},是{\PN{约干}}之祖。{\PN{利肯}}生{\PN{沙买}}。
\VS{45}{\PN{沙买}}的儿子是{\PN{玛云}};{\PN{玛云}}是{\PN{伯·夙}}之祖。
\VS{46}{\PN{迦勒}}的妾{\PN{以法}}生{\PN{哈兰}}、{\PN{摩撒}}、{\PN{迦谢}};{\PN{哈兰}}生{\PN{迦卸}}。(
\VS{47}{\PN{雅代}}的儿子是{\PN{利健}}、{\PN{约坦}}、{\PN{基珊}}、{\PN{毗力}}、{\PN{以法}}、{\PN{沙亚弗}}。)
\VS{48}{\PN{迦勒}}的妾{\PN{玛迦}}生{\PN{示别}}、{\PN{特哈拿}},
\VS{49}又生{\PN{麦玛拿}}之祖{\PN{沙亚弗}}、{\PN{抹比拿}}和{\PN{基比亚}}之祖{\PN{示法}}。{\PN{迦勒}}的女儿是{\PN{押撒}}。
\par }{\PP \VS{50}{\PN{迦勒}}的子孙就是{\PN{以法她}}的长子、{\PN{户珥}}的儿子,记在下面:{\PN{基列·耶琳}}之祖{\PN{朔巴}},
\VS{51}{\PN{伯利恒}}之祖{\PN{萨玛}},{\PN{伯迦得}}之祖{\PN{哈勒}}。
\VS{52}{\PN{基列·耶琳}}之祖{\PN{朔巴}}的子孙是{\PN{哈罗以}}和一半{\PN{米努·哈}}人\FTNT{}{{\FR 2:52: }米努·哈人就是玛拿哈人}。
\VS{53}{\PN{基列·耶琳}}的诸族是{\PN{以帖}}人、{\PN{布特}}人、{\PN{舒玛}}人、{\PN{密来}}人,又从这些族中生出{\PN{琐拉}}人和{\PN{以实陶}}人来。
\VS{54}{\PN{萨玛}}的子孙是{\PN{伯利恒}}人、{\PN{尼陀法}}人、{\PN{亚他绿·伯·约押}}人、一半{\PN{玛拿哈}}人、{\PN{琐利}}人,
\VS{55}和住{\PN{雅比斯}}众文士家的{\PN{特拉}}人、{\PN{示米押}}人、{\PN{苏甲}}人。这都是{\PN{基尼}}人{\PN{利甲}}家之祖{\PN{哈末}}所生的。

\par }\Chap{3}{\SH 大卫的儿女
\par }{\PP \VerseOne{1}{\PN{大卫}}在{\PN{希伯
}}所生的儿子记在下面:长子{\PN{暗嫩}}是{\PN{耶斯列}}人{\PN{亚希暖}}生的。次子{\PN{但以利}}是{\PN{迦密}}人{\PN{亚比该}}生的。
\VS{2}三子{\PN{押沙龙}}是{\PN{基述}}王{\PN{达买}}的女儿{\PN{玛迦}}生的。四子{\PN{亚多尼雅}}是{\PN{哈及}}生的。
\VS{3}五子{\PN{示法提雅}}是{\PN{亚比她}}生的。六子{\PN{以特念}}是{\PN{大卫}}的妻{\PN{以格拉}}生的。
\VS{4}这六人都是{\PN{大卫}}在{\PN{希伯
}}生的。{\PN{大卫}}在{\PN{希伯
}}作王七年零六个月,在{\PN{耶路撒冷}}作王三十三年。
\VS{5}{\PN{大卫}}在{\PN{耶路撒冷}}所生的儿子是{\PN{示米亚}}、{\PN{朔罢}}、{\PN{拿单}}、{\PN{所罗门}}。这四人是{\PN{亚米利}}的女儿{\PN{拔·书亚}}生的。
\VS{6}还有{\PN{益辖}}、{\PN{以利沙玛}}、{\PN{以利法列}}、
\VS{7}{\PN{挪迦}}、{\PN{尼斐}}、{\PN{雅非亚}}、
\VS{8}{\PN{以利沙玛}}、{\PN{以利雅大}}、{\PN{以利法列}},共九人。
\VS{9}这都是{\PN{大卫}}的儿子,还有他们的妹子{\PN{她玛}},妃嫔的儿子不在其内。
\par }{\SH 所罗门的后裔
\par }{\PP \VS{10}{\PN{所罗门}}的儿子是{\PN{罗波安}};{\PN{罗波安}}的儿子是{\PN{亚比雅}};{\PN{亚比雅}}的儿子是{\PN{亚撒}};{\PN{亚撒}}的儿子是{\PN{约沙法}};
\VS{11}{\PN{约沙法}}的儿子是{\PN{约兰}};{\PN{约兰}}的儿子是{\PN{亚哈谢}};{\PN{亚哈谢}}的儿子是{\PN{约阿施}};
\VS{12}{\PN{约阿施}}的儿子是{\PN{亚玛谢}};{\PN{亚玛谢}}的儿子是{\PN{亚撒利雅}};{\PN{亚撒利雅}}的儿子是{\PN{约坦}};
\VS{13}{\PN{约坦}}的儿子是{\PN{亚哈斯}};{\PN{亚哈斯}}的儿子是{\PN{希西家}};{\PN{希西家}}的儿子是{\PN{玛拿西}};
\VS{14}{\PN{玛拿西}}的儿子是{\PN{亚们}};{\PN{亚们}}的儿子是{\PN{约西亚}};
\VS{15}{\PN{约西亚}}的长子是{\PN{约哈难}},次子是{\PN{约雅敬}},三子是{\PN{西底家}},四子是{\PN{沙龙}}。
\VS{16}{\PN{约雅敬}}的儿子是{\PN{耶哥尼雅}}和{\PN{西底家}}。
\par }{\SH 耶哥尼雅的后裔
\par }{\PP \VS{17}{\PN{耶哥尼雅}}被掳。他的儿子是{\PN{撒拉铁}}、
\VS{18}{\PN{玛基兰}}、{\PN{毗大雅}}、{\PN{示拿萨}}、{\PN{耶加米}}、{\PN{何沙玛}}、{\PN{尼大比雅}}。
\VS{19}{\PN{毗大雅}}的儿子是{\PN{所罗巴伯}}、{\PN{示每}}。{\PN{所罗巴伯}}的儿子是{\PN{米书兰}}、{\PN{哈拿尼雅}},他们的妹子名叫{\PN{示罗密}}。
\VS{20}{\PN{米书兰}}的儿子是{\PN{哈舒巴}}、{\PN{阿黑}}、{\PN{比利家}}、{\PN{哈撒底}}、{\PN{于沙·希悉}},共五人。
\VS{21}{\PN{哈拿尼雅}}的儿子是{\PN{毗拉提}}、{\PN{耶筛亚}}。还有{\PN{利法雅}}的众子,{\PN{亚珥难}}的众子,{\PN{俄巴底亚}}的众子,{\PN{示迦尼}}的众子。
\VS{22}{\PN{示迦尼}}的儿子是{\PN{示玛雅}};{\PN{示玛雅}}的儿子是{\PN{哈突}}、{\PN{以甲}}、{\PN{巴利亚}}、{\PN{尼利雅}}、{\PN{沙法}},共六人。
\VS{23}{\PN{尼利雅}}的儿子是{\PN{以利约乃}}、{\PN{希西家}}、{\PN{亚斯利干}},共三人。
\VS{24}{\PN{以利约乃}}的儿子是{\PN{何大雅}}、{\PN{以利亚实}}、{\PN{毗莱雅}}、{\PN{阿谷}}、{\PN{约哈难}}、{\PN{第莱雅}}、{\PN{阿拿尼}},共七人。

\par }\Chap{4}{\SH 复记犹大的后裔
\par }{\PP \VerseOne{1}{\PN{犹大}}的儿子是{\PN{法勒斯}}、{\PN{希斯
}}、{\PN{迦米}}、{\PN{户珥}}、{\PN{朔巴}}。
\VS{2}{\PN{朔巴}}的儿子{\PN{利亚雅}}生{\PN{雅哈}};{\PN{雅哈}}生{\PN{亚户买}}和{\PN{拉哈}}。这是{\PN{琐拉}}人的诸族。
\VS{3}{\PN{以坦}}之祖的儿子是{\PN{耶斯列}}、{\PN{伊施玛}}、{\PN{伊得巴}};他们的妹子名叫{\PN{哈悉勒玻尼}}。
\VS{4}{\PN{基多}}之祖是{\PN{毗努伊勒}}。{\PN{户沙}}之祖是{\PN{以谢珥}}。这都是{\PN{伯利恒}}之祖{\PN{以法她}}的长子{\PN{户珥}}所生的。
\VS{5}{\PN{提哥亚}}之祖{\PN{亚施户}}有两个妻子:一名{\PN{希拉}},一名{\PN{拿拉}}。
\VS{6}{\PN{拿拉}}给{\PN{亚施户}}生{\PN{亚户撒}}、{\PN{希弗}}、{\PN{提米尼}}、{\PN{哈辖斯他利}}。这都是{\PN{拿拉}}的儿子。
\VS{7}{\PN{希拉}}的儿子是{\PN{洗列}}、{\PN{琐辖}}、{\PN{伊提南}}。
\VS{8}{\PN{哥斯}}生{\PN{亚诺}}、{\PN{琐比巴}},并{\PN{哈
}}儿子{\PN{亚哈黑}}的诸族。
\VS{9}{\PN{雅比斯}}比他众弟兄更尊贵,他母亲给他起名叫{\PN{雅比斯}},意思说:我生他甚是痛苦。
\VS{10}{\PN{雅比斯}}求告{\PN{以色列}}的 神说:「甚愿你赐福与我,扩张我的境界,常与我同在,保佑我不遭患难,不受艰苦。」 神就应允他所求的。
\par }{\SH 其他的家族
\par }{\PP \VS{11}{\PN{书哈}}的弟兄{\PN{基绿}}生{\PN{米黑}},{\PN{米黑}}是{\PN{伊施屯}}之祖。
\VS{12}{\PN{伊施屯}}生{\PN{伯拉巴}}、{\PN{巴西亚}},并{\PN{珥拿辖}}之祖{\PN{提欣拿}},这都是{\PN{利迦}}人。
\VS{13}{\PN{基纳斯}}的儿子是{\PN{俄陀聂}}、{\PN{西莱雅}}。{\PN{俄陀聂}}的儿子是{\PN{哈塔}}。
\VS{14}{\PN{悯挪太}}生{\PN{俄弗拉}};{\PN{西莱雅}}生{\PN{革·夏纳欣}}人之祖{\PN{约押}}。他们都是匠人。
\VS{15}{\PN{耶孚尼}}的儿子是{\PN{迦勒}};{\PN{迦勒}}的儿子是{\PN{以路}}、{\PN{以拉}}、{\PN{拿安}}。{\PN{以拉}}的儿子是{\PN{基纳斯}}。
\VS{16}{\PN{耶哈利勒}}的儿子是{\PN{西弗}}、{\PN{西法}}、{\PN{提利}}、{\PN{亚撒列}}。
\VS{17-18}{\PN{以斯拉}}的儿子是{\PN{益帖}}、{\PN{米列}}、{\PN{以弗}}、{\PN{雅伦}}。{\PN{米列}}娶法老女儿{\PN{比提雅}}为妻,生{\PN{米利暗}}、{\PN{沙买}},和{\PN{以实提摩}}之祖{\PN{益巴}}。{\PN{米列}}又娶{\PN{犹大}}女子为妻,生{\PN{基多}}之祖{\PN{雅列}},{\PN{梭哥}}之祖{\PN{希伯}},和{\PN{撒挪亚}}之祖{\PN{耶古铁}}。
\VS{19}{\PN{荷第雅}}的妻是{\PN{拿含}}的妹子,她所生的儿子是{\PN{迦米}}人{\PN{基伊拉}}和{\PN{玛迦}}人{\PN{以实提摩}}之祖。
\VS{20}{\PN{示门}}的儿子是{\PN{暗嫩}}、{\PN{林拿}}、{\PN{便·哈南}}、{\PN{提伦}}。
\par }{\PP {\PN{以示}}的儿子是{\PN{梭黑}}与{\PN{便·梭黑}}。
\par }{\SH 示拉的后裔
\par }{\PP \VS{21}{\PN{犹大}}的儿子是{\PN{示拉}};{\PN{示拉}}的儿子是{\PN{利迦}}之祖{\PN{珥}},{\PN{玛利沙}}之祖{\PN{拉大}},和属{\PN{亚实比}}族织细麻布的各家。
\VS{22}还有{\PN{约敬}}、{\PN{哥西巴}}人、{\PN{约阿施}}、{\PN{萨拉}},就是在{\PN{摩押}}地掌权的,又有{\PN{雅叔比利恒}}。这都是古时所记载的。
\VS{23}这些人都是窑匠,是{\PN{尼他应}}和{\PN{基低拉}}的居民;与王同处,为王做工。
\par }{\SH 西缅的后裔
\par }{\PP \VS{24}{\PN{西缅}}的儿子是{\PN{尼母利}}、{\PN{雅悯}}、{\PN{雅立}}、{\PN{谢拉}}、{\PN{扫罗}}。
\VS{25}{\PN{扫罗}}的儿子是{\PN{沙龙}};{\PN{沙龙}}的儿子是{\PN{米比衫}};{\PN{米比衫}}的儿子是{\PN{米施玛}};
\VS{26}{\PN{米施玛}}的儿子是{\PN{哈母利}};{\PN{哈母利}}的儿子是{\PN{撒刻}};{\PN{撒刻}}的儿子是{\PN{示每}}。
\VS{27}{\PN{示每}}有十六个儿子,六个女儿,他弟兄的儿女不多,他们各家不如{\PN{犹大}}族的人丁增多。
\VS{28}{\PN{西缅}}人住在{\PN{别是巴}}、{\PN{摩拉大}}、{\PN{哈萨·书亚}}、
\VS{29}{\PN{辟拉}}、{\PN{以森}}、{\PN{陀腊}}、
\VS{30}{\PN{彼土利}}、{\PN{何珥玛}}、{\PN{洗革拉}}、
\VS{31}{\PN{伯·玛嘉博}}、{\PN{哈萨·苏撒}}、{\PN{伯·比利}}、{\PN{沙拉音}},这些城邑直到{\PN{大卫}}作王的时候都是属{\PN{西缅}}人的。
\VS{32}他们的五个城邑是{\PN{以坦}}、{\PN{亚因}}、{\PN{临门}}、{\PN{陀健}}、{\PN{亚珊}};
\VS{33}还有属城的乡村,直到{\PN{巴力}}。这是他们的住处,他们都有家谱。
\par }{\PP \VS{34}还有{\PN{米所巴}}、{\PN{雅米勒}}、{\PN{亚玛谢}}的儿子{\PN{约沙}}、
\VS{35}{\PN{约珥}}、{\PN{约示比}}的儿子{\PN{耶户}};{\PN{约示比}}是{\PN{西莱雅}}的儿子;{\PN{西莱雅}}是{\PN{亚薛}}的儿子。
\VS{36}还有{\PN{以利约乃}}、{\PN{雅哥巴}}、{\PN{约朔海}}、{\PN{亚帅雅}}、{\PN{亚底业}}、{\PN{耶西篾}}、{\PN{比拿雅}}、
\VS{37}{\PN{示非}}的儿子{\PN{细撒}}。{\PN{示非}}是{\PN{亚龙}}的儿子;{\PN{亚龙}}是{\PN{耶大雅}}的儿子;{\PN{耶大雅}}是{\PN{申利}}的儿子;{\PN{申利}}是{\PN{示玛雅}}的儿子。
\VS{38}以上所记的人名都是作族长的,他们宗族的人数增多。
\VS{39}他们往平原东边{\PN{基多}}口去,寻找牧放羊群的草场,
\VS{40}寻得肥美的草场地,又宽阔又平静。从前住那里的是{\PN{含}}族的人。
\VS{41}以上录名的人,在{\PN{犹大}}王{\PN{希西家}}年间,来攻击{\PN{含}}族人的帐棚和那里所有的{\PN{米乌尼}}人,将他们灭尽,就住在他们的地方,直到今日,因为那里有草场可以牧放羊群。
\VS{42}这{\PN{西缅}}人中,有五百人上{\PN{西珥山}},率领他们的是{\PN{以示}}的儿子{\PN{毗拉提}}、{\PN{尼利雅}}、{\PN{利法雅}},和{\PN{乌薛}},
\VS{43}杀了逃脱剩下的{\PN{亚玛力}}人,就住在那里直到今日。

\par }\Chap{5}{\SH 吕便的后裔
\par }{\PP \VerseOne{1}{\PN{以色列}}的长子原是{\PN{吕便}};因他污秽了父亲的床,他长子的名分就归了{\PN{约瑟}}。只是按家谱他不算长子。
\VS{2}{\PN{犹大}}胜过一切弟兄,君王也是从他而出;长子的名分却归{\PN{约瑟}}。
\VS{3}{\PN{以色列}}长子{\PN{吕便}}的儿子是{\PN{哈诺}}、{\PN{法路}}、{\PN{希斯伦}}、{\PN{迦米}}。
\VS{4}{\PN{约珥}}的儿子是{\PN{示玛雅}};{\PN{示玛雅}}的儿子是{\PN{歌革}};{\PN{歌革}}的儿子是{\PN{示每}};
\VS{5}{\PN{示每}}的儿子是{\PN{米迦}};{\PN{米迦}}的儿子是{\PN{利亚雅}};{\PN{利亚雅}}的儿子是{\PN{巴力}};
\VS{6}{\PN{巴力}}的儿子是{\PN{备拉}}。这{\PN{备拉}}作{\PN{吕便}}支派的首领,被{\PN{亚述}}王{\PN{提革拉·毗列色}}掳去。
\VS{7}他的弟兄照着宗族,按着家谱作族长的是{\PN{耶利}}、{\PN{撒迦利雅}}、{\PN{比拉}}。
\VS{8}{\PN{比拉}}是{\PN{亚撒}}的儿子;{\PN{亚撒}}是{\PN{示玛}}的儿子;{\PN{示玛}}是{\PN{约珥}}的儿子;{\PN{约珥}}所住的地方是从{\PN{亚罗珥}}直到{\PN{尼波}}和{\PN{巴力·免}},
\VS{9}又向东延到{\PN{幼发拉底河}}这边的旷野,因为他们在{\PN{基列}}地牲畜增多。
\VS{10}{\PN{扫罗}}年间,他们与{\PN{夏甲}}人争战,{\PN{夏甲}}人倒在他们手下,他们就在{\PN{基列}}东边的全地,住在{\PN{夏甲}}人的帐棚里。
\par }{\SH 迦得的后裔
\par }{\PP \VS{11}{\PN{迦得}}的子孙在{\PN{吕便}}对面,住在{\PN{巴珊}}地,延到{\PN{撒迦}}。
\VS{12}他们中间有作族长的{\PN{约珥}},有作副族长的{\PN{沙番}},还有{\PN{雅乃}}和住在{\PN{巴珊}}的{\PN{沙法}}。
\VS{13}他们族弟兄是{\PN{米迦勒}}、{\PN{米书兰}}、{\PN{示巴}}、{\PN{约赖}}、{\PN{雅干}}、{\PN{细亚}}、{\PN{希伯}},共七人。
\VS{14}这都是{\PN{亚比孩}}的儿子。{\PN{亚比孩}}是{\PN{户利}}的儿子;{\PN{户利}}是{\PN{耶罗亚}}的儿子;{\PN{耶罗亚}}是{\PN{基列}}的儿子;{\PN{基列}}是{\PN{米迦勒}}的儿子;{\PN{米迦勒}}是{\PN{耶示筛}}的儿子;{\PN{耶示筛}}是{\PN{耶哈多}}的儿子;{\PN{耶哈多}}是{\PN{布斯}}的儿子。
\VS{15}还有{\PN{古尼}}的孙子、{\PN{押比叠}}的儿子{\PN{亚希}}。这都是作族长的。
\VS{16}他们住在{\PN{基列}}与{\PN{巴珊}}和{\PN{巴珊}}的乡村,并{\PN{沙
}}的郊野,直到四围的交界。
\VS{17}这些人在{\PN{犹大}}王{\PN{约坦}}并在{\PN{以色列}}王{\PN{耶罗波安}}年间,都载入家谱。
\par }{\SH 吕便迦得玛拿西族的军队
\par }{\PP \VS{18}{\PN{吕便}}人、{\PN{迦得}}人,和{\PN{玛拿西}}半支派的人,能拿盾牌和刀剑、拉弓射箭、出征善战的勇士共有四万四千七百六十名。
\VS{19}他们与{\PN{夏甲}}人、{\PN{伊突}}人、{\PN{拿非施}}人、{\PN{挪答}}人争战。
\VS{20}他们得了 {\ADD{神的}}帮助,{\PN{夏甲}}人和跟随{\PN{夏甲}}的人都交在他们手中;因为他们在阵上呼求 神,倚赖 神, 神就应允他们。
\VS{21}他们掳掠了{\PN{夏甲}}人的牲畜,有骆驼五万,羊二十五万,驴二千;又有人十万。
\VS{22}敌人被杀仆倒的甚多,因为这争战是出乎 神。他们就住在敌人的地上,直到被掳的时候。
\par }{\SH 玛拿西半支派
\par }{\PP \VS{23}{\PN{玛拿西}}半支派的人住在那地。从{\PN{巴珊}}延到{\PN{巴力·黑们}}、{\PN{示尼珥}}与{\PN{黑门山}}。
\VS{24}他们的族长是{\PN{以弗}}、{\PN{以示}}、{\PN{以列}}、{\PN{亚斯列}}、{\PN{耶利米}}、{\PN{何达威雅}}、{\PN{雅叠}},都是大能的勇士,是有名的人,也是作族长的。
\par }{\SH 二支派半人叛逆 神
\par }{\PP \VS{25}他们得罪了他们列祖的 神,随从那地之民的神行邪淫;这民就是 神在他们面前所除灭的。
\VS{26}故此,{\PN{以色列}}的 神激动{\PN{亚述}}王{\PN{普勒}}和{\PN{亚述}}王{\PN{提革拉·毗列色}}的心,他们就把{\PN{吕便}}人、{\PN{迦得}}人、{\PN{玛拿西}}半支派的人掳到{\PN{哈腊}}、{\PN{哈博}}、{\PN{哈拉}}与{\PN{歌散河}}边,直到今日还在那里。

\par }\Chap{6}{\SH 利未的后裔
\par }{\PP \VerseOne{1}{\PN{利未}}的儿子是{\PN{革顺}}、{\PN{哥辖}}、{\PN{米拉利}}。
\VS{2}{\PN{哥辖}}的儿子是{\PN{暗兰}}、{\PN{以斯哈}}、{\PN{希伯伦}}、{\PN{乌薛}}。
\VS{3}{\PN{暗兰}}的儿子是{\PN{亚伦}}、{\PN{摩西}},还有女儿{\PN{米利暗}}。{\PN{亚伦}}的儿子是{\PN{拿答}}、{\PN{亚比户}}、{\PN{以利亚撒}}、{\PN{以他玛}}。
\par }{\PP \VS{4}{\PN{以利亚撒}}生{\PN{非尼哈}};{\PN{非尼哈}}生{\PN{亚比书}};
\VS{5}{\PN{亚比书}}生{\PN{布基}};{\PN{布基}}生{\PN{乌西}};
\VS{6}{\PN{乌西}}生{\PN{西拉希雅}};{\PN{西拉希雅}}生{\PN{米拉约}};
\VS{7}{\PN{米拉约}}生{\PN{亚玛利雅}};{\PN{亚玛利雅}}生{\PN{亚希突}};
\VS{8}{\PN{亚希突}}生{\PN{撒督}};{\PN{撒督}}生{\PN{亚希玛斯}};
\VS{9}{\PN{亚希玛斯}}生{\PN{亚撒利雅}};{\PN{亚撒利雅}}生{\PN{约哈难}};
\VS{10}{\PN{约哈难}}生{\PN{亚撒利雅}}(这{\PN{亚撒利雅}}在{\PN{所罗门}}于{\PN{耶路撒冷}}所建造的殿中,供祭司的职分);
\VS{11}{\PN{亚撒利雅}}生{\PN{亚玛利雅}};{\PN{亚玛利雅}}生{\PN{亚希突}};
\VS{12}{\PN{亚希突}}生{\PN{撒督}};{\PN{撒督}}生{\PN{沙龙}};
\VS{13}{\PN{沙龙}}生{\PN{希勒家}};{\PN{希勒家}}生{\PN{亚撒利雅}};
\VS{14}{\PN{亚撒利雅}}生{\PN{西莱雅}};{\PN{西莱雅}}生{\PN{约萨答}}。
\VS{15}当耶和华借{\PN{尼布甲尼撒}}的手掳掠{\PN{犹大}}和{\PN{耶路撒冷}}人的时候,这{\PN{约萨答}}也被{\ADD{掳去}}。
\par }{\SH 利未的其余后裔
\par }{\PP \VS{16}{\PN{利未}}的儿子是{\PN{革顺}}、{\PN{哥辖}}、{\PN{米拉利}}。
\VS{17}{\PN{革顺}}的儿子名叫{\PN{立尼}}、{\PN{示每}}。
\VS{18}{\PN{哥辖}}的儿子是{\PN{暗兰}}、{\PN{以斯哈}}、{\PN{希伯伦}}、{\PN{乌薛}}。
\VS{19}{\PN{米拉利}}的儿子是{\PN{抹利}}、{\PN{母示}}。这是按着{\PN{利未}}人宗族分的各家。
\VS{20}{\PN{革顺}}的儿子是{\PN{立尼}};{\PN{立尼}}的儿子是{\PN{雅哈}};{\PN{雅哈}}的儿子是{\PN{薪玛}};
\VS{21}{\PN{薪玛}}的儿子是{\PN{约亚}};{\PN{约亚}}的儿子是{\PN{易多}};{\PN{易多}}的儿子是{\PN{谢拉}};{\PN{谢拉}}的儿子是{\PN{耶特赖}}。
\VS{22}{\PN{哥辖}}的儿子是{\PN{亚米拿达}};{\PN{亚米拿达}}的儿子是{\PN{可拉}};{\PN{可拉}}的儿子是{\PN{亚惜}};
\VS{23}{\PN{亚惜}}的儿子是{\PN{以利加拿}};{\PN{以利加拿}}的儿子是{\PN{以比雅撒}};{\PN{以比雅撒}}的儿子是{\PN{亚惜}};
\VS{24}{\PN{亚惜}}的儿子是{\PN{他哈}};{\PN{他哈}}的儿子是{\PN{乌列}};{\PN{乌列}}的儿子是{\PN{乌西雅}};{\PN{乌西雅}}的儿子是{\PN{少罗}}。
\VS{25}{\PN{以利加拿}}的儿子是{\PN{亚玛赛}}和{\PN{亚希摩}}。
\VS{26}{\PN{亚希摩}}的儿子是{\PN{以利加拿}};{\PN{以利加拿}}的儿子是{\PN{琐菲}};{\PN{琐菲}}的儿子是{\PN{拿哈}};
\VS{27}{\PN{拿哈}}的儿子是{\PN{以利押}};{\PN{以利押}}的儿子是{\PN{耶罗罕}};{\PN{耶罗罕}}的儿子是{\PN{以利加拿}};{\PN{以利加拿}}的儿子是{\PN{撒母耳}}。
\VS{28}{\PN{撒母耳}}的长子是{\PN{约珥}},次子是{\PN{亚比亚}}。
\VS{29}{\PN{米拉利}}的儿子是{\PN{抹利}};{\PN{抹利}}的儿子是{\PN{立尼}};{\PN{立尼}}的儿子是{\PN{示每}};{\PN{示每}}的儿子是{\PN{乌撒}};
\VS{30}{\PN{乌撒}}的儿子是{\PN{示米亚}};{\PN{示米亚}}的儿子是{\PN{哈基雅}};{\PN{哈基雅}}的儿子是{\PN{亚帅雅}}。
\par }{\SH 圣殿中的歌者
\par }{\PP \VS{31}约柜安设之后,{\PN{大卫}}派人在耶和华殿中管理歌唱的事。
\VS{32}他们就在会幕前当歌唱的差,及至{\PN{所罗门}}在{\PN{耶路撒冷}}建造了耶和华的殿,他们便按着班次供职。
\VS{33}供职的人和他们的子孙记在下面:
\par }{\PP {\PN{哥辖}}的子孙中有歌唱的{\PN{希幔}}。{\PN{希幔}}是{\PN{约珥}}的儿子;{\PN{约珥}}是{\PN{撒母耳}}的儿子;
\VS{34}{\PN{撒母耳}}是{\PN{以利加拿}}的儿子;{\PN{以利加拿}}是{\PN{耶罗罕}}的儿子;{\PN{耶罗罕}}是{\PN{以列}}的儿子;{\PN{以列}}是{\PN{陀亚}}的儿子;
\VS{35}{\PN{陀亚}}是{\PN{苏弗}}的儿子;{\PN{苏弗}}是{\PN{以利加拿}}的儿子;{\PN{以利加拿}}是{\PN{玛哈}}的儿子;{\PN{玛哈}}是{\PN{亚玛赛}}的儿子;
\VS{36}{\PN{亚玛赛}}是{\PN{以利加拿}}的儿子;{\PN{以利加拿}}是{\PN{约珥}}的儿子;{\PN{约珥}}是{\PN{亚撒利雅}}的儿子;{\PN{亚撒利雅}}是{\PN{西番雅}}的儿子;
\VS{37}{\PN{西番雅}}是{\PN{他哈}}的儿子;{\PN{他哈}}是{\PN{亚惜}}的儿子;{\PN{亚惜}}是{\PN{以比雅撒}}的儿子;{\PN{以比雅撒}}是{\PN{可拉}}的儿子;
\VS{38}{\PN{可拉}}是{\PN{以斯哈}}的儿子;{\PN{以斯哈}}是{\PN{哥辖}}的儿子;{\PN{哥辖}}是{\PN{利未}}的儿子;{\PN{利未}}是{\PN{以色列}}的儿子。
\VS{39}{\PN{希幔}}的族兄{\PN{亚萨}}是{\PN{比利家}}的儿子,{\PN{亚萨}}在{\PN{希幔}}右边供职。{\PN{比利家}}是{\PN{示米亚}}的儿子;
\VS{40}{\PN{示米亚}}是{\PN{米迦勒}}的儿子;{\PN{米迦勒}}是{\PN{巴西雅}}的儿子;{\PN{巴西雅}}是{\PN{玛基雅}}的儿子;
\VS{41}{\PN{玛基雅}}是{\PN{伊特尼}}的儿子;{\PN{伊特尼}}是{\PN{谢拉}}的儿子;{\PN{谢拉}}是{\PN{亚大雅}}的儿子;
\VS{42}{\PN{亚大雅}}是{\PN{以探}}的儿子;{\PN{以探}}是{\PN{薪玛}}的儿子;{\PN{薪玛}}是{\PN{示每}}的儿子;
\VS{43}{\PN{示每}}是{\PN{雅哈}}的儿子;{\PN{雅哈}}是{\PN{革顺}}的儿子。{\PN{革顺}}是{\PN{利未}}的儿子。
\VS{44}他们的族弟兄{\PN{米拉利}}的子孙,在他们左边供职的有{\PN{以探}}。{\PN{以探}}是{\PN{基示}}的儿子;{\PN{基示}}是{\PN{亚伯底}}的儿子;{\PN{亚伯底}}是{\PN{玛鹿}}的儿子;
\VS{45}{\PN{玛鹿}}是{\PN{哈沙比雅}}的儿子;{\PN{哈沙比雅}}是{\PN{亚玛谢}}的儿子;{\PN{亚玛谢}}是{\PN{希勒家}}的儿子;
\VS{46}{\PN{希勒家}}是{\PN{暗西}}的儿子;{\PN{暗西}}是{\PN{巴尼}}的儿子;{\PN{巴尼}}是{\PN{沙麦}}的儿子;
\VS{47}{\PN{沙麦}}是{\PN{末力}}的儿子;{\PN{末力}}是{\PN{母示}}的儿子;{\PN{母示}}是{\PN{米拉利}}的儿子;{\PN{米拉利}}是{\PN{利未}}的儿子。
\VS{48}他们的族弟兄{\PN{利未}}人也被派办 神殿中的一切事。
\par }{\SH 亚伦的后裔
\par }{\PP \VS{49}{\PN{亚伦}}和他的子孙在燔祭坛和香坛上献祭烧香,又在至圣所办理一切的事,为{\PN{以色列}}人赎罪,是照 神仆人{\PN{摩西}}所吩咐的。
\VS{50}{\PN{亚伦}}的儿子是{\PN{以利亚撒}};{\PN{以利亚撒}}的儿子是{\PN{非尼哈}};{\PN{非尼哈}}的儿子是{\PN{亚比书}};
\VS{51}{\PN{亚比书}}的儿子是{\PN{布基}};{\PN{布基}}的儿子是{\PN{乌西}};{\PN{乌西}}的儿子是{\PN{西拉希雅}};
\VS{52}{\PN{西拉希雅}}的儿子是{\PN{米拉约}};{\PN{米拉约}}的儿子是{\PN{亚玛利雅}};{\PN{亚玛利雅}}的儿子是{\PN{亚希突}};
\VS{53}{\PN{亚希突}}的儿子是{\PN{撒督}};{\PN{撒督}}的儿子是{\PN{亚希玛斯}}。
\par }{\SH 利未支派的居地
\par }{\PP \VS{54}他们的住处按着境内的营寨,记在下面:{\PN{哥辖}}族{\PN{亚伦}}的子孙{\ADD{先}}拈阄得地,
\VS{55}在{\PN{犹大}}地中得了{\PN{希伯
}}和四围的郊野;
\VS{56}只是属城的田地和村庄都为{\PN{耶孚尼}}的儿子{\PN{迦勒}}所得。
\VS{57}{\PN{亚伦}}的子孙得了逃城{\PN{希伯
}},又得了{\PN{立拿}}与其郊野,{\PN{雅提珥}}、{\PN{以实提莫}}与其郊野;
\VS{58}{\PN{希
}}与其郊野,{\PN{底璧}}与其郊野,
\VS{59}{\PN{亚珊}}与其郊野,{\PN{伯·示麦}}与其郊野。
\VS{60}在{\PN{便雅悯}}支派的地中,得了{\PN{迦巴}}与其郊野,{\PN{阿勒篾}}与其郊野,{\PN{亚拿突}}与其郊野。他们诸家所得的城共十三座。
\par }{\PP \VS{61}{\PN{哥辖}}族其余的人又拈阄,在{\PN{玛拿西}}半支派的地中得了十座城。
\VS{62}{\PN{革顺}}族按着宗族,在{\PN{以萨迦}}支派的地中,{\PN{亚设}}支派的地中,{\PN{拿弗他利}}支派的地中,{\PN{巴珊}}内{\PN{玛拿西}}支派的地中,得了十三座城。
\VS{63}{\PN{米拉利}}族按着宗族拈阄,在{\PN{吕便}}支派的地中,{\PN{迦得}}支派的地中,{\PN{西布伦}}支派的地中,得了十二座城。
\VS{64}{\PN{以色列}}人将这些城与其郊野给了{\PN{利未}}人。
\VS{65}这以上录名的城,在{\PN{犹大}}、{\PN{西缅}}、{\PN{便雅悯}}三支派的地中,{\PN{以色列}}人拈阄给了他们。
\par }{\PP \VS{66}{\PN{哥辖}}族中有几家在{\PN{以法莲}}支派的地中也得了城邑,
\VS{67}在{\PN{以法莲}}山地得了逃城{\PN{示剑}}与其郊野,又得了{\PN{基色}}与其郊野,
\VS{68}{\PN{约缅}}与其郊野,{\PN{伯·和
}}与其郊野,
\VS{69}{\PN{亚雅
}}与其郊野,{\PN{迦特·临门}}与其郊野。
\VS{70}{\PN{哥辖}}族其余的人在{\PN{玛拿西}}半支派的地中得了{\PN{亚乃}}与其郊野,{\PN{比连}}与其郊野。
\par }{\PP \VS{71}{\PN{革顺}}族在{\PN{玛拿西}}半支派的地中得了{\PN{巴珊}}的{\PN{哥兰}}与其郊野,{\PN{亚斯他录}}与其郊野;
\VS{72}又在{\PN{以萨迦}}支派的地中得了{\PN{基低斯}}与其郊野,{\PN{大比拉}}与其郊野,
\VS{73}{\PN{拉末}}与其郊野,{\PN{亚年}}与其郊野;
\VS{74}在{\PN{亚设}}支派的地中得了{\PN{玛沙}}与其郊野,{\PN{押顿}}与其郊野,
\VS{75}{\PN{户割}}与其郊野,{\PN{利合}}与其郊野;
\VS{76}在{\PN{拿弗他利}}支派的地中得了{\PN{加利利}}的{\PN{基低斯}}与其郊野,{\PN{哈们}}与其郊野,{\PN{基列亭}}与其郊野。
\VS{77}还有{\PN{米拉利}}族的人在{\PN{西布伦}}支派的地中得了{\PN{临摩挪}}与其郊野,{\PN{他泊}}与其郊野;
\VS{78}又在{\PN{耶利哥}}的{\PN{约旦河}}东,在{\PN{吕便}}支派的地中得了旷野的{\PN{比悉}}与其郊野,{\PN{雅哈撒}}与其郊野,
\VS{79}{\PN{基底莫}}与其郊野,{\PN{米法押}}与其郊野;
\VS{80}又在{\PN{迦得}}支派的地中得了{\PN{基列}}的{\PN{拉末}}与其郊野,{\PN{玛哈念}}与其郊野,
\VS{81}{\PN{希实本}}与其郊野,{\PN{雅谢}}与其郊野。

\par }\Chap{7}{\SH 以萨迦的后裔
\par }{\PP \VerseOne{1}{\PN{以萨迦}}的儿子是{\PN{陀拉}}、{\PN{普瓦}}、{\PN{雅述}}\FTNT{}{{\FR 7:1: }在创世记第四十六章十三节是约伯}、{\PN{伸
}},共四人。
\VS{2}{\PN{陀拉}}的儿子是{\PN{乌西}}、{\PN{利法雅}}、{\PN{耶勒}}、{\PN{雅买}}、{\PN{易伯散}}、{\PN{示母利}},都是{\PN{陀拉}}的族长,是大能的勇士。到{\PN{大卫}}年间,他们的人数共有二万二千六百名。
\VS{3}{\PN{乌西}}的儿子是{\PN{伊斯拉希}};{\PN{伊斯拉希}}的儿子是{\PN{米迦勒}}、{\PN{俄巴底亚}}、{\PN{约珥}}、{\PN{伊示雅}},共五人,都是族长。
\VS{4}他们所率领的,按着宗族出战的军队,共有三万六千人,因为他们的妻和子众多。
\VS{5}他们的族弟兄在{\PN{以萨迦}}各族中都是大能的勇士,按着家谱计算共有八万七千人。
\par }{\SH 便雅悯和但的后裔
\par }{\PP \VS{6}{\PN{便雅悯}}{\ADD{的儿子}}是{\PN{比拉}}、{\PN{比结}}、{\PN{耶叠}},共三人。
\VS{7}{\PN{比拉}}的儿子是{\PN{以斯本}}、{\PN{乌西}}、{\PN{乌薛}}、{\PN{耶利摩}}、{\PN{以利}},共五人,都是族长,是大能的勇士。按着家谱计算,他们的子孙共有二万二千零三十四人。
\VS{8}{\PN{比结}}的儿子是{\PN{细米拉}}、{\PN{约阿施}}、{\PN{以利以谢}}、{\PN{以利约乃}}、{\PN{暗利}}、{\PN{耶利摩}}、{\PN{亚比雅}}、{\PN{亚拿突}}、{\PN{亚拉篾}}。这都是{\PN{比结}}的儿子。
\VS{9}他们都是族长,是大能的勇士。按着家谱计算,他们的子孙共有二万零二百人。
\VS{10}{\PN{耶叠}}的儿子是{\PN{比勒罕}};{\PN{比勒罕}}的儿子是{\PN{耶乌施}}、{\PN{便雅悯}}、{\PN{以忽}}、{\PN{基拿拿}}、{\PN{细坦}}、{\PN{他施}}、{\PN{亚希沙哈}}。
\VS{11}这都是{\PN{耶叠}}的儿子,都是族长,是大能的勇士;他们的子孙能上阵打仗的,共有一万七千二百人。
\VS{12}还有{\PN{以珥}}的儿子{\PN{书品}}、{\PN{户品}},并{\PN{亚黑}}的儿子{\PN{户伸}}。
\par }{\SH 拿弗他利的后裔
\par }{\PP \VS{13}{\PN{拿弗他利}}的儿子是{\PN{雅薛}}、{\PN{沽尼}}、{\PN{耶色}}、{\PN{沙龙}}。这都是{\PN{辟拉}}的子孙。
\par }{\SH 玛拿西的后裔
\par }{\PP \VS{14}{\PN{玛拿西}}的儿子{\PN{亚斯列}}是他妾{\PN{亚兰}}人所生的,又生了{\PN{基列}}之父{\PN{玛吉}}。
\VS{15}{\PN{玛吉}}娶的妻是{\PN{户品}}、{\PN{书品}}的妹子,名叫{\PN{玛迦}}。{\PN{玛拿西}}的次子名叫{\PN{西罗非哈}};{\PN{西罗非哈}}但有几个女儿。
\VS{16}{\PN{玛吉}}的妻{\PN{玛迦}}生了一个儿子,起名叫{\PN{毗利施}}。{\PN{毗利施}}的兄弟名叫{\PN{示利施}};{\PN{示利施}}的儿子是{\PN{乌兰}}和{\PN{利金}}。
\VS{17}{\PN{乌兰}}的儿子是{\PN{比但}}。这都是{\PN{基列}}的子孙。{\PN{基列}}是{\PN{玛吉}}的儿子,{\PN{玛吉}}是{\PN{玛拿西}}的儿子。
\VS{18}{\PN{基列}}的妹子{\PN{哈摩利吉}}生了{\PN{伊施荷}}、{\PN{亚比以谢}}、{\PN{玛拉}}。(
\VS{19}{\PN{示米大}}的儿子是{\PN{亚现}}、{\PN{示剑}}、{\PN{利克希}}、{\PN{阿尼安}}。)
\par }{\SH 以法莲的后裔
\par }{\PP \VS{20}{\PN{以法莲}}的儿子是{\PN{书提拉}};{\PN{书提拉}}的儿子是{\PN{比列}};{\PN{比列}}的儿子是{\PN{他哈}};{\PN{他哈}}的儿子是{\PN{以拉大}};{\PN{以拉大}}的儿子是{\PN{他哈}};
\VS{21}{\PN{他哈}}的儿子是{\PN{撒拔}};{\PN{撒拔}}的儿子是{\PN{书提拉}}。{\PN{以法莲}}又生{\PN{以谢}}、{\PN{以列}};这二人因为下去夺取{\PN{迦特}}人的牲畜,被本地的{\PN{迦特}}人杀了。
\VS{22}他们的父亲{\PN{以法莲}}为他们悲哀了多日,他的弟兄都来安慰他。
\VS{23}{\PN{以法莲}}与妻同房,他妻就怀孕生了一子,{\PN{以法莲}}因为家里遭祸,就给这儿子起名叫{\PN{比利亚}}。
\VS{24}他的女儿名叫{\PN{舍伊拉}},就是建筑上{\PN{伯·和
}}、下{\PN{伯·和
}}与{\PN{乌羡·舍伊拉}}的。
\VS{25}{\PN{比利阿}}的儿子是{\PN{利法}}和{\PN{利悉}}。{\PN{利悉}}的儿子是{\PN{他拉}};{\PN{他拉}}的儿子是{\PN{他罕}};
\VS{26}{\PN{他罕}}的儿子是{\PN{拉但}};{\PN{拉但}}的儿子是{\PN{亚米忽}};{\PN{亚米忽}}的儿子是{\PN{以利沙玛}};
\VS{27}{\PN{以利沙玛}}的儿子是{\PN{嫩}};{\PN{嫩}}的儿子是{\PN{约书亚}}。
\VS{28}{\PN{以法莲}}人的地业和住处是{\PN{伯特利}}与其村庄;东边{\PN{拿兰}},西边{\PN{基色}}与其村庄;{\PN{示剑}}与其村庄,直到{\PN{迦萨}}与其村庄;
\VS{29}还有靠近{\PN{玛拿西}}人的境界,{\PN{伯·善}}与其村庄;{\PN{他纳}}与其村庄;{\PN{米吉多}}与其村庄;{\PN{多珥}}与其村庄。{\PN{以色列}}儿子{\PN{约瑟}}的子孙住在这些地方。
\par }{\SH 亚设的后裔
\par }{\PP \VS{30}{\PN{亚设}}的儿子是{\PN{音拿}}、{\PN{亦施瓦}}、{\PN{亦施韦}}、{\PN{比利亚}},还有他们的妹子{\PN{西}}
{\PN{拉}}。
\VS{31}{\PN{比利亚}}的儿子是{\PN{希别}}、{\PN{玛结}};{\PN{玛结}}是{\PN{比撒威}}的父亲。
\VS{32}{\PN{希别}}生{\PN{雅弗勒}}、{\PN{朔默}}、{\PN{何坦}},和他们的妹子{\PN{书雅}}。
\VS{33}{\PN{雅弗勒}}的儿子是{\PN{巴萨}}、{\PN{宾哈}}、{\PN{亚施法}}。这都是{\PN{雅弗勒}}的儿子。
\VS{34}{\PN{朔默}}的儿子是{\PN{亚希}}、{\PN{罗迦}}、{\PN{耶户巴}}、{\PN{亚兰}}。
\VS{35}{\PN{朔默}}兄弟{\PN{希连}}的儿子是{\PN{琐法}}、{\PN{音那}}、{\PN{示利斯}}、{\PN{亚抹}}。
\VS{36}{\PN{琐法}}的儿子是{\PN{书亚}}、{\PN{哈尼弗}}、{\PN{书阿勒}}、{\PN{比利}}、{\PN{音拉}}、
\VS{37}{\PN{比悉}}、{\PN{河得}}、{\PN{珊玛}}、{\PN{施沙}}、{\PN{益兰}}、{\PN{比拉}}。
\VS{38}{\PN{益帖}}的儿子是{\PN{耶孚尼}}、{\PN{毗斯巴}}、{\PN{亚拉}}。
\VS{39}{\PN{乌拉}}的儿子是{\PN{亚拉}}、{\PN{汉尼业}}、{\PN{利写}}。
\VS{40}这都是{\PN{亚设}}的子孙,都是族长,是精壮大能的勇士,也是首领中的头目,按着家谱计算,他们的子孙能出战的共有二万六千人。

\par }\Chap{8}{\SH 便雅悯的后裔
\par }{\PP \VerseOne{1}{\PN{便雅悯}}的长子{\PN{比拉}},次子{\PN{亚实别}},三子{\PN{亚哈拉}},
\VS{2}四子{\PN{挪哈}},五子{\PN{拉法}}。
\VS{3}{\PN{比拉}}的儿子是{\PN{亚大}}、{\PN{基拉}}、{\PN{亚比忽}}、
\VS{4}{\PN{亚比书}}、{\PN{乃幔}}、{\PN{亚何亚}}、
\VS{5}{\PN{基拉}}、{\PN{示孚汛}}、{\PN{户兰}}。
\VS{6}{\PN{以忽}}的儿子作{\PN{迦巴}}居民的族长,被掳到{\PN{玛拿辖}};
\VS{7}{\PN{以忽}}的儿子{\PN{乃幔}}、{\PN{亚希亚}}、{\PN{基拉}}也被掳去。{\PN{基拉}}生{\PN{乌撒}}、{\PN{亚希忽}}。
\VS{8}{\PN{沙哈连}}休他二妻{\PN{户伸}}和{\PN{巴拉}}之后,在{\PN{摩押}}地生了儿子。
\VS{9}他与妻{\PN{贺得}}同房,生了{\PN{约巴}}、{\PN{洗比雅}}、{\PN{米沙}}、{\PN{玛拉干}}、
\VS{10}{\PN{耶乌斯}}、{\PN{沙迦}}、{\PN{米玛}}。他这些儿子都是族长。
\VS{11}他的妻{\PN{户伸}}给他生的儿子有{\PN{亚比突}}、{\PN{以利巴力}}。
\VS{12}{\PN{以利巴力}}的儿子是{\PN{希伯}}、{\PN{米珊}}、{\PN{沙麦}}。{\PN{沙麦}}建立{\PN{阿挪}}和{\PN{罗德}}二城与其村庄。
\par }{\SH 在亚雅 的便雅悯人
\par }{\PP \VS{13}又有{\PN{比利亚}}和{\PN{示玛}}是{\PN{亚雅
}}居民的族长,是驱逐{\PN{迦特}}人的。
\VS{14}{\PN{亚希约}}、{\PN{沙煞}}、{\PN{耶利末}}、
\VS{15}{\PN{西巴第雅}}、{\PN{亚拉得}}、{\PN{亚得}}、
\VS{16}{\PN{米迦勒}}、{\PN{伊施巴}}、{\PN{约哈}}都是{\PN{比利亚}}的儿子。
\par }{\SH 在耶路撒冷的便雅悯人
\par }{\PP \VS{17}{\PN{西巴第雅}}、{\PN{米书兰}}、{\PN{希西基}}、{\PN{希伯}}、
\VS{18}{\PN{伊施米莱}}、{\PN{伊斯利亚}}、{\PN{约巴}}都是{\PN{以利巴力}}的儿子。
\VS{19}{\PN{雅金}}、{\PN{细基}}
{\PN{利}}、{\PN{撒底}}、
\VS{20}{\PN{以利乃}}、{\PN{洗勒太}}、{\PN{以列}}、
\VS{21}{\PN{亚大雅}}、{\PN{比拉雅}}、{\PN{申拉}}都是{\PN{示每}}的儿子。
\VS{22}{\PN{伊施班}}、{\PN{希伯}}、{\PN{以列}}、
\VS{23}{\PN{亚伯顿}}、{\PN{细基利}}、{\PN{哈难}}、
\VS{24}{\PN{哈拿尼雅}}、{\PN{以拦}}、{\PN{安陀提雅}}、
\VS{25}{\PN{伊弗底雅}}、{\PN{毗努伊勒}}都是{\PN{沙煞}}的儿子。
\VS{26}{\PN{珊示莱}}、{\PN{示哈利}}、{\PN{亚他利雅}}、
\VS{27}{\PN{雅利西}}、{\PN{以利亚}}、{\PN{细基利}}都是{\PN{耶罗罕}}的儿子。
\VS{28}这些人都是著名的族长,住在{\PN{耶路撒冷}}。
\par }{\SH 在基遍和耶路撒冷的便雅悯人
\par }{\PP \VS{29}在{\PN{基遍}}住的有{\PN{基遍}}的父亲{\PN{耶利}}。他的妻名叫{\PN{玛迦}};
\VS{30}他长子是{\PN{亚伯顿}}。他又生{\PN{苏珥}}、{\PN{基士}}、{\PN{巴力}}、{\PN{拿答}}、
\VS{31}{\PN{基多}}、{\PN{亚希约}}、{\PN{撒迦}}、{\PN{米基罗}}。
\VS{32}{\PN{米基罗}}生{\PN{示米暗}}。这些人和他们的弟兄在{\PN{耶路撒冷}}对面居住。
\par }{\SH 扫罗王的家族
\par }{\PP \VS{33}{\PN{尼珥}}生{\PN{基士}};{\PN{基士}}生{\PN{扫罗}};{\PN{扫罗}}生{\PN{约拿单}}、{\PN{麦基舒亚}}、{\PN{亚比拿达}}、{\PN{伊施·巴力}}。
\VS{34}{\PN{约拿单}}的儿子是{\PN{米力·巴力}}\FTNT{}{{\FR 8:34: }在撒母耳下四章四节是米非波设};{\PN{米力·巴力}}生{\PN{米迦}}。
\VS{35}{\PN{米迦}}的儿子是{\PN{毗敦}}、{\PN{米勒}}、{\PN{他利亚}}、{\PN{亚哈斯}};
\VS{36}{\PN{亚哈斯}}生{\PN{耶何阿达}};{\PN{耶何阿达}}生{\PN{亚拉篾}}、{\PN{亚斯玛威}}、{\PN{心利}};{\PN{心利}}生{\PN{摩撒}};
\VS{37}{\PN{摩撒}}生{\PN{比尼亚}};{\PN{比尼亚}}的儿子是{\PN{拉法}};{\PN{拉法}}的儿子是{\PN{以利亚萨}};{\PN{以利亚萨}}的儿子是{\PN{亚悉}}。
\VS{38}{\PN{亚悉}}有六个儿子,他们的名字是{\PN{亚斯利干}}、{\PN{波基路}}、{\PN{以实玛利}}、{\PN{示亚利雅}}、{\PN{俄巴底雅}}、{\PN{哈难}}。这都是{\PN{亚悉}}的儿子。
\VS{39}{\PN{亚悉}}兄弟{\PN{以设}}的长子是{\PN{乌兰}},次子{\PN{耶乌施}},三子是{\PN{以利法列}}。
\VS{40}{\PN{乌兰}}的儿子都是大能的勇士,是弓箭手,他们有许多的子孙,共一百五十名,都是{\PN{便雅悯}}人。

\par }\Chap{9}{\SH 被掳归回的人
\par }{\PP \VerseOne{1}{\PN{以色列}}人都按家谱计算,写在{\PN{以色列}}诸王记上。{\PN{犹大}}人因犯罪就被掳到{\PN{巴比伦}}。
\VS{2}先从{\PN{巴比伦}}回来,住在自己地业城邑中的有{\PN{以色列}}人、祭司、{\PN{利未}}人、尼提宁{\ADD{的首领}}。
\VS{3}住在{\PN{耶路撒冷}}的有{\PN{犹大}}人、{\PN{便雅悯}}人、{\PN{以法莲}}人、{\PN{玛拿西}}人。
\VS{4}{\PN{犹大}}儿子{\PN{法勒斯}}的子孙中有{\PN{乌太}}。{\PN{乌太}}是{\PN{亚米忽}}的儿子;{\PN{亚米忽}}是{\PN{暗利}}的儿子;{\PN{暗利}}是{\PN{音利}}的儿子;{\PN{音利}}是{\PN{巴尼}}的儿子。
\VS{5}{\PN{示罗}}的子孙中有长子{\PN{亚帅雅}}和他的众子。
\VS{6}{\PN{谢拉}}的子孙中有{\PN{耶乌利}}和他的弟兄,共六百九十人。
\VS{7}{\PN{便雅悯}}人中有{\PN{哈西努}}的曾孙、{\PN{何达威雅}}的孙子、{\PN{米书兰}}的儿子{\PN{撒路}},
\VS{8}又有{\PN{耶罗罕}}的儿子{\PN{伊比尼雅}},{\PN{米基立}}的孙子、{\PN{乌西}}的儿子{\PN{以拉}},{\PN{伊比尼雅}}的曾孙、{\PN{流珥}}的孙子、{\PN{示法提雅}}的儿子{\PN{米书兰}},
\VS{9}和他们的族弟兄,按着家谱{\ADD{计算}}共有九百五十六名。这些人都是他们的族长。
\par }{\SH 住在耶路撒冷的祭司
\par }{\PP \VS{10}祭司中有{\PN{耶大雅}}、{\PN{耶何雅立}}、{\PN{雅斤}},
\VS{11}还有管理 神殿{\PN{希勒家}}的儿子{\PN{亚萨利雅}}。{\PN{希勒家}}是{\PN{米书兰}}的儿子;{\PN{米书兰}}是{\PN{撒督}}的儿子;{\PN{撒督}}是{\PN{米拉约}}的儿子;{\PN{米拉约}}是{\PN{亚希突}}的儿子。
\VS{12}有{\PN{玛基雅}}的曾孙、{\PN{巴施户珥}}的孙子、{\PN{耶罗罕}}的儿子{\PN{亚大雅}},又有{\PN{亚第业}}的儿子{\PN{玛赛}};{\PN{亚第业}}是{\PN{雅希细拉}}的儿子;{\PN{雅希细拉}}是{\PN{米书兰}}的儿子;{\PN{米书兰}}是{\PN{米实利密}}的儿子;{\PN{米实利密}}是{\PN{音麦}}的儿子。
\VS{13}他们和众弟兄都是族长,共有一千七百六十人,是善于做 神殿使用之工的。
\par }{\SH 住在耶路撒冷的利未人
\par }{\PP \VS{14}{\PN{利未}}人{\PN{米拉利}}的子孙中,有{\PN{哈沙比雅}}的曾孙、{\PN{押利甘}}的孙子、{\PN{哈述}}的儿子{\PN{示玛雅}}。
\VS{15}有{\PN{拔巴甲}}、{\PN{黑勒施}}、{\PN{迦拉}},并{\PN{亚萨}}的曾孙、{\PN{细基利}}的孙子、{\PN{米迦}}的儿子{\PN{玛探雅}},
\VS{16}又有{\PN{耶杜顿}}的曾孙、{\PN{迦拉}}的孙子、{\PN{示玛雅}}的儿子{\PN{俄巴底}},还有{\PN{以利加拿}}的孙子、{\PN{亚撒}}的儿子{\PN{比利家}}。他们都住在{\PN{尼陀法}}人的村庄。
\par }{\SH 住在耶路撒冷的圣殿守卫
\par }{\PP \VS{17}守门的是{\PN{沙龙}}、{\PN{亚谷}}、{\PN{达们}}、{\PN{亚希幔}},和他们的弟兄;{\PN{沙龙}}为长。
\VS{18}从前这些人看守朝东的王门,如今是{\PN{利未}}营中守门的。
\VS{19}{\PN{可拉}}的曾孙、{\PN{以比雅撒}}的孙子、{\PN{可利}}的儿子{\PN{沙龙}},和他的族弟兄{\PN{可拉}}人都管理使用之工,并守会幕的门。他们的祖宗曾管理耶和华的营盘,又把守营门。
\VS{20}从前{\PN{以利亚撒}}的儿子{\PN{非尼哈}}管理他们,耶和华也与他同在。
\VS{21}{\PN{米施利米雅}}的儿子{\PN{撒迦利雅}}是看守会幕之门的。
\VS{22}被选守门的人共有二百一十二名。他们在自己的村庄,按着家谱计算,是{\PN{大卫}}和先见{\PN{撒母耳}}所派当这紧要职任的。
\VS{23}他们和他们的子孙按着班次看守耶和华殿的门,就是会幕的门。
\VS{24}在东西南北,四方都有守门的。
\VS{25}他们的族弟兄住在村庄,每七日来与他们换班。
\VS{26}这四个门领都是{\PN{利未}}人,各有紧要的职任,看守 神殿的仓库。
\VS{27}他们住在 神殿的四围,是因委托他们守殿,要每日早晨开门。
\par }{\SH 其余的利未人
\par }{\PP \VS{28}{\PN{利未}}人中有管理使用器皿的,按着数目拿出拿入;
\VS{29}又有人管理器具和圣所的器皿,并细面、酒、油、乳香、香料。
\VS{30}祭司中有人用香料做膏油。
\VS{31}{\PN{利未}}人{\PN{玛他提雅}}是{\PN{可拉}}族{\PN{沙龙}}的长子,他紧要的职任是管理盘中烤的物。
\VS{32}他们族弟兄{\PN{哥辖}}子孙中,有管理陈设饼的,每安息日预备摆列。
\par }{\PP \VS{33}歌唱的有{\PN{利未}}人的族长,住在{\ADD{属殿的}}房屋,昼夜供职,不做别样的工。
\VS{34}以上都是{\PN{利未}}人著名的族长,住在{\PN{耶路撒冷}}。
\par }{\SH 扫罗王的祖先和后代
\par }{\R (代上8·29—38)
\par }{\PP \VS{35}在{\PN{基遍}}住的有{\PN{基遍}}的父亲{\PN{耶利}}。他的妻名叫{\PN{玛迦}};
\VS{36}他长子是{\PN{亚伯顿}}。他又生{\PN{苏珥}}、{\PN{基士}}、{\PN{巴力}}、{\PN{尼珥}}、{\PN{拿答}}、
\VS{37}{\PN{基多}}、{\PN{亚希约}}、{\PN{撒迦利雅}}、{\PN{米基罗}}。
\VS{38}{\PN{米基罗}}生{\PN{示米暗}}。这些人和他们的弟兄在{\PN{耶路撒冷}}对面居住。
\VS{39}{\PN{尼珥}}生{\PN{基士}};{\PN{基士}}生{\PN{扫罗}};{\PN{扫罗}}生{\PN{约拿单}}、{\PN{麦基舒亚}}、{\PN{亚比拿达}}、{\PN{伊施·巴力}}。
\VS{40}{\PN{约拿单}}的儿子是{\PN{米力·巴力}}\FTNT{}{{\FR 9:40: }就是米非波设};{\PN{米力·巴力}}生{\PN{米迦}}。
\VS{41}{\PN{米迦}}的儿子是{\PN{毗敦}}、{\PN{米勒}}、{\PN{他利亚}}、{\PN{亚哈斯}}。
\VS{42}{\PN{亚哈斯}}生{\PN{雅拉}};{\PN{雅拉}}生{\PN{亚拉篾}}、{\PN{亚斯玛威}}、{\PN{心利}};{\PN{心利}}生{\PN{摩撒}};
\VS{43}{\PN{摩撒}}生{\PN{比尼亚}};{\PN{比尼亚}}生{\PN{利法雅}};{\PN{利法雅}}的儿子是{\PN{以利亚萨}};{\PN{以利亚萨}}的儿子是{\PN{亚悉}}。
\VS{44}{\PN{亚悉}}有六个儿子,他们的名字是{\PN{亚斯利干}}、{\PN{波基路}}、{\PN{以实玛利}}、{\PN{示亚利雅}}、{\PN{俄巴底雅}}、{\PN{哈难}}。这都是{\PN{亚悉}}的儿子。

\par }\Chap{10}{\SH 扫罗逝世
\par }{\R (撒上31·1—13)
\par }{\PP \VerseOne{1}{\PN{非利士}}人与{\PN{以色列}}人争战,{\PN{以色列}}人在{\PN{非利士}}人面前逃跑,在{\PN{基利波山}}有被杀仆倒的。
\VS{2}{\PN{非利士}}人紧追{\PN{扫罗}}和他儿子们,就杀了{\PN{扫罗}}的儿子{\PN{约拿单}}、{\PN{亚比拿达}}、{\PN{麦基舒亚}}。
\VS{3}势派甚大,{\PN{扫罗}}被弓箭手追上,射伤甚重,
\VS{4}就吩咐拿他兵器的人说:「你拔出刀来,将我刺死,免得那些未受割礼的人来凌辱我。」但拿兵器的人甚惧怕,不肯刺他;
\VS{5}{\PN{扫罗}}就自己伏在刀上死了。拿兵器的人见{\PN{扫罗}}已死,也伏在刀上死了。
\VS{6}这样,{\PN{扫罗}}和他三个儿子,并他的全家都一同死亡。
\VS{7}住平原的{\PN{以色列}}众人见{\PN{以色列}}军兵逃跑,{\PN{扫罗}}和他儿子都死了,也就弃城逃跑,{\PN{非利士}}人便来住在其中。
\par }{\PP \VS{8}次日,{\PN{非利士}}人来剥那被杀之人的衣服,看见{\PN{扫罗}}和他儿子仆倒在{\PN{基利波山}},
\VS{9}就剥了他的军装,割下他的首级,打发人到\FTNT{}{{\FR 10:9: }到:或译送到}{\PN{非利士}}地的四境报信与他们的偶像和众民,
\VS{10}又将{\PN{扫罗}}的军装放在他们神的庙里,将他的首级钉在{\PN{大衮}}庙中。
\VS{11}{\PN{基列·雅比}}人听见{\PN{非利士}}人向{\PN{扫罗}}所行的一切事,
\VS{12}他们中间所有的勇士就起身前去,将{\PN{扫罗}}和他儿子的尸身送到{\PN{雅比}},将他们的尸骨葬在{\PN{雅比}}的橡树下,就禁食七日。
\par }{\PP \VS{13}这样,{\PN{扫罗}}死了。因为他干犯耶和华,没有遵守耶和华的命;又因他求问交鬼的妇人,
\VS{14}没有求问耶和华,所以耶和华使他被杀,把国归于{\PN{耶西}}的儿子{\PN{大卫}}。

\par }\Chap{11}{\SH 大卫作以色列和犹大的王
\par }{\R (撒下5·1—10)
\par }{\PP \VerseOne{1}{\PN{以色列}}众人聚集到{\PN{希伯
}}见{\PN{大卫}},说:「我们原是你的骨肉。
\VS{2}从前{\PN{扫罗}}作王的时候,率领{\PN{以色列}}人出入的是你;耶和华—你的 神也曾应许你说:『你必牧养我的民{\PN{以色列}},作{\PN{以色列}}的君。』」
\VS{3}于是{\PN{以色列}}的长老都来到{\PN{希伯
}}见{\PN{大卫}}王。{\PN{大卫}}在{\PN{希伯
}}耶和华面前与他们立约,他们就膏{\PN{大卫}}作{\PN{以色列}}的王,是照耶和华借{\PN{撒母耳}}所说的话。
\par }{\PP \VS{4}{\PN{大卫}}和{\PN{以色列}}众人到了{\PN{耶路撒冷}},就是{\PN{耶布斯}};那时{\PN{耶布斯}}人住在那里。
\VS{5}{\PN{耶布斯}}人对{\PN{大卫}}说:「你决不能进这地方。」然而{\PN{大卫}}攻取{\PN{锡安}}的保障,就是{\PN{大卫}}的城。
\VS{6}{\PN{大卫}}说:「谁先攻打{\PN{耶布斯}}人,必作首领元帅。」{\PN{洗鲁雅}}的儿子{\PN{约押}}先上去,就作了元帅。
\VS{7}{\PN{大卫}}住在保障里,所以那保障叫作{\PN{大卫城}}。
\VS{8}{\PN{大卫}}又从{\PN{米罗}}起,四围建筑城墙,其余的是{\PN{约押}}修理。
\VS{9}{\PN{大卫}}日见强盛,因为万军之耶和华与他同在。
\par }{\SH 大卫的有名将领
\par }{\R (撒下23·8—39)
\par }{\PP \VS{10}以下记录跟随{\PN{大卫}}勇士的首领,就是奋勇帮助他得国、照着耶和华吩咐{\PN{以色列}}人的话、与{\PN{以色列}}人一同立他作王的。
\VS{11}{\PN{大卫}}勇士的数目记在下面:{\PN{哈革摩尼}}的儿子{\PN{雅朔班}},他是军长的统领,一时举枪杀了三百人。
\par }{\PP \VS{12}其次是{\PN{亚合}}人{\PN{朵多}}的儿子{\PN{以利亚撒}},他是三个勇士里的一个。
\VS{13}他从前与{\PN{大卫}}在{\PN{巴斯·达闵}},{\PN{非利士}}人聚集要打仗。那里有一块长满大麦的田,众民就在{\PN{非利士}}人面前逃跑;
\VS{14}这勇士便站在那田间击杀{\PN{非利士}}人,救护了那田。耶和华使{\PN{以色列}}人大获全胜。
\par }{\PP \VS{15}三十个勇士中的三个人下到磐石那里,进了{\PN{亚杜兰洞}}见{\PN{大卫}};{\PN{非利士}}的军队在{\PN{利乏音谷}}安营。
\VS{16}那时{\PN{大卫}}在山寨,{\PN{非利士}}人的防营在{\PN{伯利恒}}。
\VS{17}{\PN{大卫}}渴想,说:「甚愿有人将{\PN{伯利恒}}城门旁井里的水打来给我喝!」
\VS{18}这三个勇士就闯过{\PN{非利士}}人的营盘,从{\PN{伯利恒}}城门旁的井里打水,拿来奉给{\PN{大卫}}。他却不肯喝,将水奠在耶和华面前,
\VS{19}说:「我的 神啊,这三个人冒死去打水,这水好像他们的血一般,我断不敢喝!」如此,{\PN{大卫}}不肯喝。这是三个勇士所做的事。
\par }{\PP \VS{20}{\PN{约押}}的兄弟{\PN{亚比筛}}是这三个勇士的首领;他举枪杀了三百人,就在三个勇士里得了名。
\VS{21}他在这三个勇士里是最尊贵的,所以作他们的首领;只是不及前三个勇士。
\par }{\PP \VS{22}有{\PN{甲薛}}勇士{\PN{耶何耶大}}的儿子{\PN{比拿雅}}行过大能的事:他杀了{\PN{摩押}}人{\PN{亚利伊勒}}的两个{\ADD{儿子}},又在下雪的时候下坑里去杀了一个狮子,
\VS{23}又杀了一个{\PN{埃及}}人。{\PN{埃及}}人身高五肘,手里拿着枪,枪杆粗如织布的机轴;{\PN{比拿雅}}只拿着棍子下去,从{\PN{埃及}}人手里夺过枪来,用那枪将他刺死。
\VS{24}这是{\PN{耶何耶大}}的儿子{\PN{比拿雅}}所行的事,就在三个勇士里得了名。
\VS{25}他比那三十个勇士都尊贵,只是不及前三个勇士。{\PN{大卫}}立他作护卫长。
\par }{\PP \VS{26}军中的勇士有{\PN{约押}}的兄弟{\PN{亚撒黑}},{\PN{伯利恒}}人{\PN{朵多}}的儿子{\PN{伊勒哈难}},
\VS{27}{\PN{哈律}}人{\PN{沙玛}},{\PN{比伦}}人{\PN{希利斯}},
\VS{28}{\PN{提哥亚}}人{\PN{益吉}}的儿子{\PN{以拉}},{\PN{亚拿突}}人{\PN{亚比以谢}},
\VS{29}{\PN{户沙}}人{\PN{西比该}},{\PN{亚合}}人{\PN{以来}},
\VS{30}{\PN{尼陀法}}人{\PN{玛哈莱}},{\PN{尼陀法}}人{\PN{巴拿}}的儿子{\PN{希立}},
\VS{31}{\PN{便雅悯}}族{\PN{基比亚}}人{\PN{利拜}}的儿子{\PN{以太}},{\PN{比拉顿}}人{\PN{比拿雅}},
\VS{32}{\PN{迦实溪}}人{\PN{户莱}},{\PN{亚拉巴}}人{\PN{亚比}},
\VS{33}{\PN{巴路米}}人{\PN{押斯玛弗}},{\PN{沙本}}人{\PN{以利雅哈巴}},
\VS{34}{\PN{基孙}}人{\PN{哈深}}的众子,{\PN{哈拉}}人{\PN{沙基}}的儿子{\PN{约拿单}},
\VS{35}{\PN{哈拉}}人{\PN{沙甲}}的儿子{\PN{亚希暗}},{\PN{吾珥}}的儿子{\PN{以利法勒}},
\VS{36}{\PN{米基拉}}人{\PN{希弗}},{\PN{比伦}}人{\PN{亚希雅}},
\VS{37}{\PN{迦密}}人{\PN{希斯罗}},{\PN{伊斯拜}}的儿子{\PN{拿莱}},
\VS{38}{\PN{拿单}}的兄弟{\PN{约珥}},{\PN{哈基利}}的儿子{\PN{弥伯哈}},
\VS{39}{\PN{亚扪}}人{\PN{洗勒}},{\PN{比录}}人{\PN{拿哈莱}}({\PN{拿哈莱}}是给{\PN{洗鲁雅}}的儿子{\PN{约押}}拿兵器的),
\VS{40}{\PN{以帖}}人{\PN{以拉}},{\PN{以帖}}人{\PN{迦立}},
\VS{41}{\PN{赫}}人{\PN{乌利亚}},{\PN{亚莱}}的儿子{\PN{撒拔}},
\VS{42}{\PN{吕便}}人{\PN{示撒}}的儿子{\PN{亚第拿}}(他是{\PN{吕便}}支派中的一个族长,率领三十人),
\VS{43}{\PN{玛迦}}的儿子{\PN{哈难}},{\PN{弥特尼}}人{\PN{约沙法}},
\VS{44}{\PN{亚施他拉}}人{\PN{乌西亚}},{\PN{亚罗珥}}人{\PN{何坦}}的儿子{\PN{沙玛}}、{\PN{耶利}},
\VS{45}{\PN{提洗}}人{\PN{申利}}的儿子{\PN{耶叠}}和他的兄弟{\PN{约哈}},
\VS{46}{\PN{玛哈未}}人{\PN{以利业}},{\PN{伊利拿安}}的儿子{\PN{耶利拜}}、{\PN{约沙未雅}},{\PN{摩押}}人{\PN{伊特玛}}、
\VS{47}{\PN{以利业}}、{\PN{俄备得}},并{\PN{米琐八}}人{\PN{雅西业}}。

\par }\Chap{12}{\SH 早期跟随大卫的便雅悯人
\par }{\PP \VerseOne{1}{\PN{大卫}}因怕{\PN{基士}}的儿子{\PN{扫罗}},躲在{\PN{洗革拉}}的时候,有勇士到他那里帮助他打仗。
\VS{2}他们善于拉弓,能用左右两手甩石射箭,都是{\PN{便雅悯}}人{\PN{扫罗}}的族弟兄。
\VS{3}为首的是{\PN{亚希以谢}},其次是{\PN{约阿施}},都是{\PN{基比亚}}人{\PN{示玛}}的儿子。还有{\PN{亚斯玛威}}的儿子{\PN{耶薛}}和{\PN{毗力}},又有{\PN{比拉迦}},并{\PN{亚拿突}}人{\PN{耶户}},
\VS{4}{\PN{基遍}}人{\PN{以实买雅}}(他在三十人中是勇士,管理他们),且有{\PN{耶利米}}、{\PN{雅哈悉}}、{\PN{约哈难}},和{\PN{基得拉}}人{\PN{约撒拔}}、
\VS{5}{\PN{伊利乌赛}}、{\PN{耶利摩}}、{\PN{比亚利雅}}、{\PN{示玛利雅}},{\PN{哈律弗}}人{\PN{示法提雅}},
\VS{6}{\PN{可拉}}人{\PN{以利加拿}}、{\PN{耶西亚}}、{\PN{亚萨列}}、{\PN{约以谢}}、{\PN{雅朔班}},
\VS{7}{\PN{基多}}人{\PN{耶罗罕}}的儿子{\PN{犹拉}}和{\PN{西巴第雅}}。
\par }{\SH 跟随大卫的迦得人
\par }{\PP \VS{8}{\PN{迦得}}支派中有人到旷野的山寨投奔{\PN{大卫}},都是大能的勇士,能拿盾牌和枪的战士。他们的面貌好像狮子,快跑如同山上的鹿。
\VS{9}第一{\PN{以薛}},第二{\PN{俄巴底雅}},第三{\PN{以利押}},
\VS{10}第四{\PN{弥施玛拿}},第五{\PN{耶利米}},
\VS{11}第六{\PN{亚太}},第七{\PN{以利业}},
\VS{12}第八{\PN{约哈难}},第九{\PN{以利萨巴}},
\VS{13}第十{\PN{耶利米}},第十一{\PN{末巴奈}}。
\VS{14}这都是{\PN{迦得}}人中的军长,至小的能抵一百人,至大的能抵一千人。
\VS{15}正月,{\PN{约旦河}}水涨过两岸的时候,他们过河,使一切住平原的人东奔西逃。
\par }{\SH 跟随大卫的便雅悯人和犹大人
\par }{\PP \VS{16}又有{\PN{便雅悯}}和{\PN{犹大}}人到山寨{\PN{大卫}}那里。
\VS{17}{\PN{大卫}}出去迎接他们,对他们说:「你们若是和和平平地来帮助我,我心就与你们相契;你们若是将我这无罪的人卖在敌人手里,愿我们列祖的 神察看责罚。」
\VS{18}那时 {\ADD{神}}的灵感动那三十个勇士的首领{\PN{亚玛撒}},{\ADD{他就说}}:
\par }{\Q {\PN{大卫}}啊,我们是归于你的!
\par }{\Q {\PN{耶西}}的儿子啊,我们是帮助你的!
\par }{\Q 愿你平平安安,
\par }{\Q 愿帮助你的也都平安!
\par }{\Q 因为你的 神帮助你。
\par }{\MM {\PN{大卫}}就收留他们,立他们作军长。
\par }{\SH 跟随大卫的玛拿西人
\par }{\PP \VS{19}{\PN{大卫}}从前与{\PN{非利士}}人同去,要与{\PN{扫罗}}争战,有些{\PN{玛拿西}}人来投奔{\PN{大卫}},他们却没有帮助{\PN{非利士}}人;因为{\PN{非利士}}人的首领商议,打发他们回去,说:「恐怕{\PN{大卫}}拿我们的首级,归降他的主人{\PN{扫罗}}。」
\VS{20}{\PN{大卫}}往{\PN{洗革拉}}去的时候,有{\PN{玛拿西}}人的千夫长{\PN{押拿}}、{\PN{约撒拔}}、{\PN{耶叠}}、{\PN{米迦勒}}、{\PN{约撒拔}}、{\PN{以利户}}、{\PN{洗勒太}}都来投奔他。
\VS{21}这些人帮助{\PN{大卫}}攻击群贼;他们都是大能的勇士,且作军长。
\VS{22}那时天天有人来帮助{\PN{大卫}},以致成了大军,如 神的军一样。
\par }{\SH 大卫的军力
\par }{\PP \VS{23}预备打仗的兵来到{\PN{希伯
}}见{\PN{大卫}},要照着耶和华的话将{\PN{扫罗}}的国位归与{\PN{大卫}}。他们的数目如下:
\VS{24}{\PN{犹大}}支派,拿盾牌和枪预备打仗的有六千八百人。
\VS{25}{\PN{西缅}}支派,能上阵大能的勇士有七千一百人。
\VS{26}{\PN{利未}}支派有四千六百人。
\VS{27}{\PN{耶何耶大}}是{\PN{亚伦}}{\ADD{家}}的首领,跟从他的有三千七百人。
\VS{28}还有少年大能的勇士{\PN{撒督}},同着他的有族长二十二人。
\VS{29}{\PN{便雅悯}}支派,{\PN{扫罗}}的族弟兄也有三千人,他们向来大半归顺{\PN{扫罗}}家。
\VS{30}{\PN{以法莲}}支派大能的勇士,在本族著名的有二万零八百人。
\VS{31}{\PN{玛拿西}}半支派,册上有名的共一万八千人,都来立{\PN{大卫}}作王。
\VS{32}{\PN{以萨迦}}支派,有二百族长都通达时务,知道{\PN{以色列}}人所当行的;他们族弟兄都听从他们的命令。
\VS{33}{\PN{西布伦}}支派,能上阵用各样兵器打仗、行伍整齐、不生二心的有五万人。
\VS{34}{\PN{拿弗他利}}支派,有一千军长;跟从他们、拿盾牌和枪的有三万七千人。
\VS{35}{\PN{但}}支派,能摆阵的有二万八千六百人。
\VS{36}{\PN{亚设}}支派,能上阵打仗的有四万人。
\VS{37}{\PN{约旦河}}东的{\PN{吕便}}支派、{\PN{迦得}}支派、{\PN{玛拿西}}半支派,拿着各样兵器打仗的有十二万人。
\par }{\PP \VS{38}以上都是能守行伍的战士,他们都诚心来到{\PN{希伯
}},要立{\PN{大卫}}作{\PN{以色列}}的王。{\PN{以色列}}其余的人也都一心要立{\PN{大卫}}作王。
\VS{39}他们在那里三日,与{\PN{大卫}}一同吃喝,因为他们的族弟兄给他们预备了。
\VS{40}靠近他们的人以及{\PN{以萨迦}}、{\PN{西布伦}}、{\PN{拿弗他利}}人将许多面饼、无花果饼、干葡萄、酒、油,用驴、骆驼、骡子、牛驮来,又带了许多的牛和羊来,因为{\PN{以色列}}人甚是欢乐。

\par }\Chap{13}{\SH 约柜迁出基列·耶琳
\par }{\R (撒下6·1—11)
\par }{\PP \VerseOne{1}{\PN{大卫}}与千夫长、百夫长,就是一切首领商议。
\VS{2}{\PN{大卫}}对{\PN{以色列}}全会众说:「你们若以为美,见这事是出于耶和华—我们的 神,我们就差遣人走遍{\PN{以色列}}地,见我们未来的弟兄,又见住在有郊野之城的祭司{\PN{利未}}人,使他们都到这里来聚集。
\VS{3}我们要把 神的{\ADD{约}}柜运到我们这里来;因为在{\PN{扫罗}}年间,我们没有在{\ADD{约}}柜前求问 {\ADD{神}}。」
\VS{4}全会众都说可以如此行;这事在众民眼中都看为好。
\par }{\PP \VS{5}于是,{\PN{大卫}}将{\PN{以色列}}人从{\PN{埃及}}的{\PN{西曷河}}直到{\PN{哈马口}}都招聚了来,要从{\PN{基列·耶琳}}将 神的{\ADD{约}}柜运来。
\VS{6}{\PN{大卫}}率领{\PN{以色列}}众人上到{\PN{巴拉}},就是属{\PN{犹大}}的{\PN{基列·耶琳}},要从那里将{\ADD{约}}柜运来。这{\ADD{约}}柜就是坐在二基路伯上耶和华 神留名的{\ADD{约}}柜。
\VS{7}他们将 神的{\ADD{约}}柜从{\PN{亚比拿达}}的家里抬出来,放在新车上。{\PN{乌撒}}和{\PN{亚希约}}赶车。
\VS{8}{\PN{大卫}}和{\PN{以色列}}众人在 神前用琴、瑟、锣、鼓、号作乐,极力跳舞歌唱。
\par }{\PP \VS{9}到了{\PN{基顿}}\FTNT{}{{\FR 13:9: }在撒母耳下六章六节是拿艮}的禾场;因为牛失前蹄\FTNT{}{{\FR 13:9: }或译:惊跳},{\PN{乌撒}}就伸手扶住{\ADD{约}}柜。
\VS{10}耶和华向他发怒,因他伸手扶住{\ADD{约}}柜击杀他,他就死在 神面前。
\VS{11}{\PN{大卫}}因耶和华击杀\FTNT{}{{\FR 13:11: }原文是闯杀}{\PN{乌撒}},心里愁烦,就称那地方为{\PN{毗列斯·乌撒}},直到今日。
\VS{12}那日,{\PN{大卫}}惧怕 神,说:「 神的{\ADD{约}}柜怎可运到我这里来?」
\VS{13}于是{\PN{大卫}}不将{\ADD{约}}柜运进{\PN{大卫}}的城,却运到{\PN{迦特}}人{\PN{俄别·以东}}的家中。
\VS{14}神的{\ADD{约}}柜在{\PN{俄别·以东}}家中三个月,耶和华赐福给{\PN{俄别·以东}}的家和他一切所有的。

\par }\Chap{14}{\SH 大卫在耶路撒冷的活动
\par }{\R (撒下5·11—16)
\par }{\PP \VerseOne{1}{\PN{泰尔}}王{\PN{希兰}}将香柏木运到{\PN{大卫}}那里,又差遣使者和石匠、木匠给{\PN{大卫}}建造宫殿。
\VS{2}{\PN{大卫}}就知道耶和华坚立他作{\PN{以色列}}王,又为自己的民{\PN{以色列}},使他的国兴旺。
\par }{\PP \VS{3}{\PN{大卫}}在{\PN{耶路撒冷}}又立后妃,又生儿女。
\VS{4}在{\PN{耶路撒冷}}所生的众子是{\PN{沙母亚}}、{\PN{朔罢}}、{\PN{拿单}}、{\PN{所罗门}}、
\VS{5}{\PN{益辖}}、{\PN{以利书亚}}、{\PN{以法列}}、
\VS{6}{\PN{挪迦}}、{\PN{尼斐}}、{\PN{雅非亚}}、
\VS{7}{\PN{以利沙玛}}、{\PN{比利雅大}}、{\PN{以利法列}}。
\par }{\SH 战胜非利士人
\par }{\R (撒下5·17—25)
\par }{\PP \VS{8}{\PN{非利士}}人听见{\PN{大卫}}受膏作{\PN{以色列}}众人的王,{\PN{非利士}}众人就上来寻索{\PN{大卫}}。{\PN{大卫}}听见,就出去迎敌。
\VS{9}{\PN{非利士}}人来了,布散在{\PN{利乏音谷}}。
\VS{10}{\PN{大卫}}求问 神,说:「我可以上去攻打{\PN{非利士}}人吗?你将他们交在我手里吗?」耶和华说:「你可以上去,我必将他们交在你手里。」
\VS{11}{\PN{非利士}}人来到{\PN{巴力·毗拉心}},{\PN{大卫}}在那里杀败他们。{\PN{大卫}}说:「 神借我的手冲破敌人,如同水冲去一般」;因此称那地方为{\PN{巴力·毗拉心}}。
\VS{12}{\PN{非利士}}人将神像撇在那里,{\PN{大卫}}吩咐人用火焚烧了。
\par }{\PP \VS{13}{\PN{非利士}}人又布散在{\PN{利乏音谷}}。
\VS{14}{\PN{大卫}}又求问 神。 神说:「不要一直地上去,要转到他们后头,从桑林对面攻打他们。
\VS{15}你听见桑树梢上有脚步的声音,就要出战,因为 神已经在你前头去攻打{\PN{非利士}}人的军队。」
\VS{16}{\PN{大卫}}就遵着 神所吩咐的,攻打{\PN{非利士}}人的军队,从{\PN{基遍}}直到{\PN{基色}}。
\VS{17}于是{\PN{大卫}}的名传扬到列国,耶和华使列国都惧怕他。

\par }\Chap{15}{\SH 准备搬运约柜
\par }{\PP \VerseOne{1}{\PN{大卫}}在{\PN{大卫城}}为自己建造宫殿,又为 神的{\ADD{约}}柜预备地方,支搭帐幕。
\VS{2}那时{\PN{大卫}}说:「除了{\PN{利未}}人之外,无人可抬 神的{\ADD{约}}柜;因为耶和华拣选他们抬 神的{\ADD{约}}柜,且永远事奉他。」
\VS{3}{\PN{大卫}}招聚{\PN{以色列}}众人到{\PN{耶路撒冷}},要将耶和华的{\ADD{约}}柜抬到他所预备的地方。
\VS{4}{\PN{大卫}}又聚集{\PN{亚伦}}的子孙和{\PN{利未}}人。
\VS{5}{\PN{哥辖}}子孙中有族长{\PN{乌列}}和他的弟兄一百二十人。
\VS{6}{\PN{米拉利}}子孙中有族长{\PN{亚帅雅}}和他的弟兄二百二十人。
\VS{7}{\PN{革顺}}子孙中有族长{\PN{约珥}}和他的弟兄一百三十人。
\VS{8}{\PN{以利撒反}}子孙中有族长{\PN{示玛雅}}和他的弟兄二百人。
\VS{9}{\PN{希伯
}}子孙中有族长{\PN{以列}}和他的弟兄八十人。
\VS{10}{\PN{乌薛}}子孙中有族长{\PN{亚米拿达}}和他的弟兄一百一十二人。
\VS{11}{\PN{大卫}}将祭司{\PN{撒督}}和{\PN{亚比亚他}},并{\PN{利未}}人{\PN{乌列}}、{\PN{亚帅雅}}、{\PN{约珥}}、{\PN{示玛雅}}、{\PN{以列}}、{\PN{亚米拿达}}召来,
\VS{12}对他们说:「你们是{\PN{利未}}人的族长,你们和你们的弟兄应当自洁,好将耶和华—{\PN{以色列}} 神的{\ADD{约}}柜抬到我所预备的地方。
\VS{13}因你们先前没有{\ADD{抬这约柜}},按定例求问耶和华—我们的 神,所以他刑罚\FTNT{}{{\FR 15:13: }原文是闯杀}我们。」
\VS{14}于是祭司{\PN{利未}}人自洁,好将耶和华—{\PN{以色列}} 神的{\ADD{约}}柜抬上来。
\VS{15}{\PN{利未}}子孙就用杠,肩抬 神的{\ADD{约}}柜,是照耶和华借{\PN{摩西}}所吩咐的。
\par }{\PP \VS{16}{\PN{大卫}}吩咐{\PN{利未}}人的族长,派他们歌唱的弟兄用琴瑟和钹作乐,欢欢喜喜地大声歌颂。
\VS{17}于是{\PN{利未}}人派{\PN{约珥}}的儿子{\PN{希幔}}和他弟兄中{\PN{比利家}}的儿子{\PN{亚萨}},并他们族弟兄{\PN{米拉利}}子孙里{\PN{古沙雅}}的儿子{\PN{以探}}。
\VS{18}其次还有他们的弟兄{\PN{撒迦利雅}}、{\PN{便雅薛}}、{\PN{示米拉末}}、{\PN{耶歇}}、{\PN{乌尼}}、{\PN{以利押}}、{\PN{比拿雅}}、{\PN{玛西雅}}、{\PN{玛他提雅}}、{\PN{以利斐利户}}、{\PN{弥克尼雅}},并守门的{\PN{俄别·以东}}和{\PN{耶利}}。
\VS{19}这样,派歌唱的{\PN{希幔}}、{\PN{亚萨}}、{\PN{以探}}敲铜钹,大发响声;
\VS{20}派{\PN{撒迦利雅}}、{\PN{雅薛}}、{\PN{示米拉末}}、{\PN{耶歇}}、{\PN{乌尼}}、{\PN{以利押}}、{\PN{玛西雅}}、{\PN{比拿雅}}鼓瑟,调用女音;
\VS{21}又派{\PN{玛他提雅}}、{\PN{以利斐利户}}、{\PN{弥克尼雅}}、{\PN{俄别·以东}}、{\PN{耶利}}、{\PN{亚撒西雅}}领首弹琴,调用第八。
\VS{22}{\PN{利未}}人的族长{\PN{基拿尼雅}}是歌唱人的首领,又教训人歌唱,因为他精通此事。
\VS{23}{\PN{比利家}}、{\PN{以利加拿}}是{\ADD{约}}柜前守门的。
\VS{24}祭司{\PN{示巴尼}}、{\PN{约沙法}}、{\PN{拿坦业}}、{\PN{亚玛赛}}、{\PN{撒迦利雅}}、{\PN{比拿亚}}、{\PN{以利以谢}}在 神的{\ADD{约}}柜前吹号。{\PN{俄别·以东}}和{\PN{耶希亚}}也是{\ADD{约}}柜前守门的。
\par }{\SH 约柜迁入耶路撒冷
\par }{\R (撒下6·12—22)
\par }{\PP \VS{25}于是,{\PN{大卫}}和{\PN{以色列}}的长老,并千夫长都去从{\PN{俄别·以东}}的家欢欢喜喜地将耶和华的约柜抬上来。
\VS{26}神赐恩与抬耶和华约柜的{\PN{利未}}人,他们就献上七只公牛,七只公羊。
\VS{27}{\PN{大卫}}和抬{\ADD{约}}柜的{\PN{利未}}人,并歌唱人的首领{\PN{基拿尼雅}},以及歌唱的人,都穿着细麻布的外袍;{\PN{大卫}}另外穿着细麻布的以弗得。
\VS{28}这样,{\PN{以色列}}众人欢呼吹角、吹号、敲钹、鼓瑟、弹琴,大发响声,将耶和华的约柜抬上来。
\par }{\PP \VS{29}耶和华的约柜进了{\PN{大卫城}}的时候,{\PN{扫罗}}的女儿{\PN{米甲}}从窗户里观看,见{\PN{大卫}}王踊跃跳舞,心里就轻视他。

\par }\Chap{16}{\PP \VerseOne{1}众人将 神的{\ADD{约}}柜请进去,安放在{\PN{大卫}}所搭的帐幕里,就在 神面前献燔祭和平安祭。
\VS{2}{\PN{大卫}}献完了燔祭和平安祭,就奉耶和华的名给民祝福,
\VS{3}并且分给{\PN{以色列}}人,无论男女,每人一个饼,一块{\ADD{肉}},一个葡萄饼。
\par }{\PP \VS{4}{\PN{大卫}}派几个{\PN{利未}}人在耶和华的{\ADD{约}}柜前事奉,颂扬,称谢,赞美耶和华—{\PN{以色列}}的 神:
\VS{5}为首的是{\PN{亚萨}},其次是{\PN{撒迦利雅}}、{\PN{雅薛}}、{\PN{示米拉末}}、{\PN{耶歇}}、{\PN{玛他提雅}}、{\PN{以利押}}、{\PN{比拿雅}}、{\PN{俄别·以东}}、{\PN{耶利}},鼓瑟弹琴;惟有{\PN{亚萨}}敲钹,大发响声;
\VS{6}祭司{\PN{比拿雅}}和{\PN{雅哈悉}}常在 神的约柜前吹号。
\VS{7}那日,{\PN{大卫}}初次借{\PN{亚萨}}和他的弟兄以诗歌称颂耶和华,{\ADD{说}}:
\par }{\SH 颂赞之歌
\par }{\R (诗105·1—15;96·1—13;106·1,47—48)
\par }{\Q \VS{8}你们要称谢耶和华,求告他的名,
\par }{\Q 在万民中传扬他的作为!
\par }{\Q \VS{9}要向他唱诗、歌颂,
\par }{\Q 谈论他一切奇妙的作为。
\par }{\Q \VS{10}要以他的圣名夸耀;
\par }{\Q 寻求耶和华的人,心中应当欢喜。
\par }{\Q \VS{11}要寻求耶和华与他的能力,
\par }{\Q 时常寻求他的面。
\par }{\Q \VS{12-13}他仆人{\PN{以色列}}的后裔,
\par }{\Q 他所拣选{\PN{雅各}}的子孙哪,
\par }{\Q 你们要记念他奇妙的作为和他的奇事,
\par }{\Q 并他口中的判语。
\par }{\BB \par }{\Q \VS{14}他是耶和华—我们的 神,
\par }{\Q 全地都有他的判断。
\par }{\Q \VS{15}你们要记念他的约,直到永远;
\par }{\Q 他所吩咐的话,直到千代,
\par }{\Q \VS{16}就是与{\PN{亚伯拉罕}}所立的约,
\par }{\Q 向{\PN{以撒}}所起的誓。
\par }{\Q \VS{17}他又将这约向{\PN{雅各}}定为律例,
\par }{\Q 向{\PN{以色列}}定为永远的约,
\par }{\Q \VS{18}说:我必将{\PN{迦南}}地赐给你,
\par }{\Q 作你产业的分。
\par }{\BB \par }{\Q \VS{19}当时你们人丁有限,数目稀少,
\par }{\Q 并且在那地为寄居的;
\par }{\Q \VS{20}他们从这邦游到那邦,
\par }{\Q 从这国行到那国。
\par }{\Q \VS{21}耶和华不容什么人欺负他们,
\par }{\Q 为他们的缘故责备君王,
\par }{\Q \VS{22}说:不可难为我受膏的人,
\par }{\Q 也不可恶待我的先知!
\par }{\BB \par }{\Q \VS{23}全地都要向耶和华歌唱!
\par }{\Q 天天传扬他的救恩,
\par }{\Q \VS{24}在列邦中述说他的荣耀,
\par }{\Q 在万民中述说他的奇事。
\par }{\Q \VS{25}因耶和华为大,当受极大的赞美;
\par }{\Q 他在万神之上,当受敬畏。
\par }{\Q \VS{26}外邦的神都属虚无,
\par }{\Q 惟独耶和华创造诸天。
\par }{\Q \VS{27}有尊荣和威严在他面前,
\par }{\Q 有能力和喜乐在他圣所。
\par }{\BB \par }{\Q \VS{28}民中的万族啊,
\par }{\Q 你们要将荣耀能力归给耶和华,都归给耶和华!
\par }{\Q \VS{29}要将耶和华的名所当得的荣耀归给他,
\par }{\Q 拿供物来奉到他面前;
\par }{\Q 当以圣洁的\FTNT{}{{\FR 16:29: }的:或译为}妆饰敬拜耶和华。
\par }{\Q \VS{30}全地要在他面前战抖,
\par }{\Q 世界也坚定不得动摇。
\par }{\Q \VS{31}愿天欢喜,愿地快乐;
\par }{\Q 愿人在列邦中说:
\par }{\Q 耶和华作王了!
\par }{\Q \VS{32}愿海和其中所充满的澎湃;
\par }{\Q 愿田和其中所有的都欢乐。
\par }{\Q \VS{33}那时,林中的树木都要在耶和华面前欢呼,
\par }{\Q 因为他来要审判全地。
\par }{\Q \VS{34}应当称谢耶和华;
\par }{\Q 因他本为善,他的慈爱永远长存!
\par }{\BB \par }{\Q \VS{35}要说:拯救我们的 神啊,求你救我们,
\par }{\Q 聚集我们,使我们脱离外邦,
\par }{\Q 我们好称赞你的圣名,以赞美你为夸胜。
\par }{\Q \VS{36}耶和华—{\PN{以色列}}的 神,
\par }{\Q 从亘古直到永远,是应当称颂的!
\par }{\Q 众民都说:「阿们!」并且赞美耶和华。
\par }{\SH 在耶路撒冷和基遍的敬拜
\par }{\PP \VS{37}{\PN{大卫}}派{\PN{亚萨}}和他的弟兄在约柜前常常事奉耶和华,一日尽一日的职分;
\VS{38}又派{\PN{俄别·以东}}和他的弟兄六十八人,与{\PN{耶杜顿}}的儿子{\PN{俄别·以东}},并{\PN{何萨}}作守门的;
\VS{39-40}且派祭司{\PN{撒督}}和他弟兄众祭司在{\PN{基遍}}的邱坛、耶和华的帐幕前燔祭坛上,每日早晚,照着耶和华律法书上所吩咐{\PN{以色列}}人的,常给耶和华献燔祭。
\VS{41}与他们一同被派的有{\PN{希幔}}、{\PN{耶杜顿}},和其余被选名字录在册上的,称谢耶和华,因他的慈爱永远长存。
\VS{42}{\PN{希幔}}、{\PN{耶杜顿}}同着他们吹号、敲钹,大发响声,并用别的乐器随着歌颂 神。{\PN{耶杜顿}}的子孙作守门的。
\par }{\PP \VS{43}于是众民各归各家;{\PN{大卫}}也回去为家眷祝福。

\par }\Chap{17}{\SH 拿单传给大卫的信息
\par }{\R (撒下7·1—17)
\par }{\PP \VerseOne{1}{\PN{大卫}}住在自己宫中,对先知{\PN{拿单}}说:「看哪,我住在香柏木的宫中,耶和华的约柜反在幔子里。」
\VS{2}{\PN{拿单}}对{\PN{大卫}}说:「你可以照你的心意而行,因为 神与你同在。」
\par }{\PP \VS{3}当夜, 神的话临到{\PN{拿单}},说:
\VS{4}「你去告诉我仆人{\PN{大卫}},说耶和华如此说:『你不可建造殿宇给我居住。
\VS{5}自从我领{\PN{以色列}}人出{\PN{埃及}},直到今日,我未曾住过殿宇,乃从这会幕到那会幕,从这帐幕{\ADD{到那帐幕}}。
\VS{6}凡我同{\PN{以色列}}人所走的地方,我何曾向{\PN{以色列}}的一个士师,就是我吩咐牧养我民的说:你为何不给我建造香柏木的殿宇呢?』
\VS{7}现在你要告诉我仆人{\PN{大卫}},说万军之耶和华如此说:『我从羊圈中将你召来,叫你不再跟从羊群,立你作我民{\PN{以色列}}的君。
\VS{8}你无论往哪里去,我常与你同在,剪除你的一切仇敌;我必使你得大名,好像世上大大有名的人一样。
\VS{9}我必为我民{\PN{以色列}}选定一个地方,栽培他们,使他们住自己的地方,不再迁移;凶恶之子也不像从前扰害他们,
\VS{10}并不像我命士师治理我民{\PN{以色列}}的时候一样。我必治服你的一切仇敌,并且我—耶和华应许你,必为你建立家室。
\VS{11}你寿数满足归你列祖的时候,我必使你的后裔接续你的位,我也必坚定他的国。
\VS{12}他必为我建造殿宇;我必坚定他的国位直到永远。
\VS{13}我要作他的父,他要作我的子;并不使我的慈爱离开他,像离开在你以前的{\PN{扫罗}}一样。
\VS{14}我却要将他永远坚立在我家里和我国里;他的国位也必坚定,直到永远。』」
\par }{\PP \VS{15}{\PN{拿单}}就按这一切话,照这默示告诉{\PN{大卫}}。
\par }{\SH 感恩的祷告
\par }{\R (撒下7·18—29)
\par }{\PP \VS{16}于是{\PN{大卫}}王进去,坐在耶和华面前,说:「耶和华 神啊,我是谁,我的家算什么,你竟使我到这地步呢?
\VS{17}神啊,这在你眼中还看为小,又应许你仆人的家至于久远。耶和华 神啊,你看顾我好像看顾高贵的人。
\VS{18}你加于仆人的尊荣,我还有何言可说呢?因为你知道你的仆人。
\VS{19}耶和华啊,你行了这大事,并且显明出来,是因你仆人的缘故,也是照你的心意。
\VS{20}耶和华啊,照我们耳中听见,没有可比你的,除你以外再无 神。
\VS{21}世上有何民能比你的民{\PN{以色列}}呢?你 神从{\PN{埃及}}救赎他们作自己的子民,又在你赎出来的民面前行大而可畏的事,驱逐列邦人,显出你的大名。
\VS{22}你使{\PN{以色列}}人作你的子民,直到永远;你—耶和华也作他们的 神。
\VS{23}耶和华啊,你所应许仆人和仆人家的话,求你坚定,直到永远,照你所说的而行。
\VS{24}愿你的名永远坚立,被尊为大,说:『万军之耶和华是{\PN{以色列}}的 神,是治理{\PN{以色列}}的 神。』这样,你仆人{\PN{大卫}}的家必在你面前坚立。
\VS{25}我的 神啊,因你启示仆人说,我必为你建立家室,所以仆人大胆在你面前祈祷。
\VS{26}耶和华啊,惟有你是 神,你也应许将这福气赐给仆人。
\VS{27}现在你喜悦赐福与仆人的家,可以永存在你面前。耶和华啊,你已经赐福,还要赐福到永远。」

\par }\Chap{18}{\SH 在军事上的胜利
\par }{\R (撒下8·1—18)
\par }{\PP \VerseOne{1}此后,{\PN{大卫}}攻打{\PN{非利士}}人,把他们治服,从他们手下夺取了{\PN{迦特}}和属{\PN{迦特}}的村庄;
\VS{2}又攻打{\PN{摩押}},{\PN{摩押}}人就归服{\PN{大卫}},给他进贡。
\par }{\PP \VS{3}{\PN{琐巴}}王{\PN{哈大利谢}}\FTNT{}{{\FR 18:3: }在撒母耳下八章三节是哈大底谢}往
{\PN{幼发拉底河}}去,要坚定自己的国权,{\PN{大卫}}就攻打他,直到{\PN{哈马}},
\VS{4}夺了他的战车一千,马兵七千,步兵二万,将拉战车的马砍断蹄筋,但留下一百辆车的马。
\VS{5}{\PN{大马士革}}的{\PN{亚兰}}人来帮助{\PN{琐巴}}王{\PN{哈大利谢}},{\PN{大卫}}就杀了{\PN{亚兰}}人二万二千。
\VS{6}于是{\PN{大卫}}在{\PN{大马士革}}的{\PN{亚兰}}地设立{\ADD{防营}},{\PN{亚兰}}人就归服他,给他进贡。{\PN{大卫}}无论往哪里去,耶和华都使他得胜。
\VS{7}他夺了{\PN{哈大利谢}}臣仆所拿的金盾牌带到{\PN{耶路撒冷}}。
\VS{8}{\PN{大卫}}又从属{\PN{哈大利谢}}的{\PN{提巴}}\FTNT{}{{\FR 18:8: }或译:比他}和{\PN{均}}二城中夺取了许多的铜。后来{\PN{所罗门}}用此制造铜海、铜柱,和一切的铜器。
\par }{\PP \VS{9}{\PN{哈马}}王{\PN{陀乌}}听见{\PN{大卫}}杀败{\PN{琐巴}}王{\PN{哈大利谢}}的全军,
\VS{10}就打发他儿子{\PN{哈多兰}}去见{\PN{大卫}}王,问他的安,为他祝福,因为他杀败了{\PN{哈大利谢}}(原来{\PN{陀乌}}与{\PN{哈大利谢}}常常争战)。{\PN{哈多兰}}带了金银铜的各样器皿来。
\VS{11}{\PN{大卫}}王将这些器皿,并从各国夺来的金银,就是从{\PN{以东}}、{\PN{摩押}}、{\PN{亚扪}}、{\PN{非利士}}、{\PN{亚玛力}}人所夺来的,都分别为圣献给耶和华。
\par }{\PP \VS{12}{\PN{洗鲁雅}}的儿子{\PN{亚比筛}}在{\PN{盐谷}}击杀了{\PN{以东}}一万八千人。
\VS{13}{\PN{大卫}}在{\PN{以东}}地设立防营,{\PN{以东}}人就都归服他。{\PN{大卫}}无论往哪里去,耶和华都使他得胜。
\par }{\PP \VS{14}{\PN{大卫}}作{\PN{以色列}}众人的王,又向众民秉公行义。
\VS{15}{\PN{洗鲁雅}}的儿子{\PN{约押}}作元帅;{\PN{亚希律}}的儿子{\PN{约沙法}}作史官;
\VS{16}{\PN{亚希突}}的儿子{\PN{撒督}}和{\PN{亚比亚他}}的儿子{\PN{亚希米勒}}作祭司{\ADD{长}};{\PN{沙威沙}}作书记;
\VS{17}{\PN{耶何耶大}}的儿子{\PN{比拿雅}}统辖{\PN{基利提}}人和{\PN{比利提}}人。{\PN{大卫}}的众子都在王的左右作领袖。

\par }\Chap{19}{\SH 大卫击败亚扪人和亚兰人
\par }{\R (撒下10·1—19)
\par }{\PP \VerseOne{1}此后,{\PN{亚扪}}人的王{\PN{拿辖}}死了,他儿子接续他作王。
\VS{2}{\PN{大卫}}说:「我要照{\PN{哈嫩}}的父亲{\PN{拿辖}}厚待我的恩典厚待{\PN{哈嫩}}。」于是{\PN{大卫}}差遣使者为他丧父安慰他。{\PN{大卫}}的臣仆到了{\PN{亚扪}}人的境内见{\PN{哈嫩}},要安慰他,
\VS{3}但{\PN{亚扪}}人的首领对{\PN{哈嫩}}说:「{\PN{大卫}}差人来安慰你,你想他是尊敬你父亲吗?他的臣仆来见你不是为详察窥探、倾覆这地吗?」
\VS{4}{\PN{哈嫩}}便将{\PN{大卫}}臣仆{\ADD{的胡须}}剃去{\ADD{一半}},又割断他们下半截的衣服,使他们露出下体,打发他们回去。
\VS{5}有人将臣仆所遇的事告诉{\PN{大卫}},他就差人去迎接他们,因为他们甚觉羞耻;告诉他们说:「可以住在{\PN{耶利哥}},等到胡须长起再回来。」
\par }{\PP \VS{6}{\PN{亚扪}}人知道{\PN{大卫}}憎恶他们,{\PN{哈嫩}}和{\PN{亚扪}}人就打发人拿一千他连得银子,从{\PN{美索不达米亚}}、{\PN{亚兰}}、{\PN{玛迦}}、{\PN{琐巴}}雇战车和马兵,
\VS{7}于是雇了三万二千辆战车和{\PN{玛迦}}王并他的军兵;他们来安营在{\PN{米底巴}}前。{\PN{亚扪}}人也从他们的城里出来,聚集交战。
\VS{8}{\PN{大卫}}听见了,就差派{\PN{约押}}统带勇猛的全军出去。
\VS{9}{\PN{亚扪}}人出来在城门前摆阵,所来的诸王另在郊野摆阵。
\par }{\PP \VS{10}{\PN{约押}}看见敌人在他前后摆阵,就从{\PN{以色列}}军中挑选精兵,使他们对着{\PN{亚兰}}人摆阵;
\VS{11}其余的兵交与他兄弟{\PN{亚比筛}},对着{\PN{亚扪}}人摆阵。
\VS{12}{\PN{约押}}对{\PN{亚比筛}}说:「{\PN{亚兰}}人若强过我,你就来帮助我;{\PN{亚扪}}人若强过你,我就去帮助你。
\VS{13}我们都当刚强,为本国的民和 神的城邑作大丈夫,愿耶和华凭他的意旨而行。」
\VS{14}于是{\PN{约押}}和跟随他的人前进攻打{\PN{亚兰}}人;{\PN{亚兰}}人在{\PN{约押}}面前逃跑。
\VS{15}{\PN{亚扪}}人见{\PN{亚兰}}人逃跑,他们也在{\PN{约押}}的兄弟{\PN{亚比筛}}面前逃跑进城。{\PN{约押}}就回{\PN{耶路撒冷}}去了。
\par }{\PP \VS{16}{\PN{亚兰}}人见自己被{\PN{以色列}}人打败,就打发使者将大河那边的{\PN{亚兰}}人调来,{\PN{哈大利谢}}的将军{\PN{朔法}}率领他们。
\VS{17}有人告诉{\PN{大卫}},他就聚集{\PN{以色列}}众人过{\PN{约旦河}},来到{\PN{亚兰}}人那里,迎着他们摆阵。{\PN{大卫}}既摆阵攻击{\PN{亚兰}}人,{\PN{亚兰}}人就与他打仗。
\VS{18}{\PN{亚兰}}人在{\PN{以色列}}人面前逃跑。{\PN{大卫}}杀了{\PN{亚兰}}七千辆战车{\ADD{的人}},四万步兵,又杀了{\PN{亚兰}}的将军{\PN{朔法}}。
\VS{19}属{\PN{哈大利谢}}{\ADD{的诸王}}见自己被{\PN{以色列}}人打败,就与{\PN{大卫}}和好,归服他。于是{\PN{亚兰}}人不敢再帮助{\PN{亚扪}}人了。

\par }\Chap{20}{\SH 大卫攻取拉巴城
\par }{\R (撒下12·26—31)
\par }{\PP \VerseOne{1}过了一年,到列王出战的时候,{\PN{约押}}率领军兵毁坏{\PN{亚扪}}人的地,围攻{\PN{拉巴}};{\PN{大卫}}仍住在{\PN{耶路撒冷}}。{\PN{约押}}攻打{\PN{拉巴}},将城倾覆。
\VS{2}{\PN{大卫}}夺了{\PN{亚扪}}人之王所戴的金冠冕\FTNT{}{{\FR 20:2: }王:或译玛勒堪。玛勒堪即米勒公,是亚扪族之神名},其上的金子重一他连得,又嵌着宝石;人将这冠冕戴在{\PN{大卫}}头上。{\PN{大卫}}从城里夺了许多财物,
\VS{3}将城里的人拉出来,放在锯下,或铁耙下,或{\ADD{铁}}斧下\FTNT{}{{\FR 20:3: }或译:强他们用锯,或用打粮食的铁器,或用铁斧做工},{\PN{大卫}}待{\PN{亚扪}}各城的居民都是如此。其后{\PN{大卫}}和众军都回{\PN{耶路撒冷}}去了。
\par }{\SH 与非利士巨人争战
\par }{\R (撒下21·15—22)
\par }{\PP \VS{4}后来,{\PN{以色列}}人在{\PN{基色}}与{\PN{非利士}}人打仗。{\PN{户沙}}人{\PN{西比该}}杀了伟人的一个儿子{\PN{细派}},{\PN{非利士}}人就被制伏了。
\VS{5}又与{\PN{非利士}}人打仗。{\PN{睚珥}}的儿子{\PN{伊勒哈难}}杀了{\PN{迦特}}人{\PN{歌利亚}}的兄弟{\PN{拉哈米}};这人的枪杆粗如织布的机轴。
\VS{6}又在{\PN{迦特}}打仗。那里有一个身量高大的人,手脚都是六指,共有二十四个指头,他也是伟人的儿子。
\VS{7}这人向{\PN{以色列}}人骂阵,{\PN{大卫}}的哥哥{\PN{示米亚}}的儿子{\PN{约拿单}}就杀了他。
\VS{8}这三个人是{\PN{迦特}}伟人的儿子,都死在{\PN{大卫}}和他仆人的手下。

\par }\Chap{21}{\SH 大卫数点以色列民
\par }{\R (撒下24·1—25)
\par }{\PP \VerseOne{1}撒但起来攻击{\PN{以色列}}人,激动{\PN{大卫}}数点他们。
\VS{2}{\PN{大卫}}就吩咐{\PN{约押}}和民中的首领说:「你们去数点{\PN{以色列}}人,从{\PN{别是巴}}直到{\PN{但}},回来告诉我,我好知道他们的数目。」
\VS{3}{\PN{约押}}说:「愿耶和华使他的百姓比现在加增百倍。我主我王啊,他们不都是你的仆人吗?我主为何吩咐行这事,为何使{\PN{以色列}}人陷在罪里呢?」
\VS{4}但王的命令胜过{\PN{约押}}。{\PN{约押}}就出去,走遍{\PN{以色列}}地,回到{\PN{耶路撒冷}},
\VS{5}将百姓的总数奏告{\PN{大卫}}:{\PN{以色列}}人拿刀的有一百一十万;{\PN{犹大}}人拿刀的有四十七万。
\VS{6}惟有{\PN{利未}}人和{\PN{便雅悯}}人没有数在其中,因为{\PN{约押}}厌恶王的这命令。
\par }{\PP \VS{7}神不喜悦这数点百姓的事,便降灾给{\PN{以色列}}人。
\VS{8}{\PN{大卫}}祷告 神说:「我行这事大有罪了!现在求你除掉仆人的罪孽,因我所行的甚是愚昧。」
\VS{9}耶和华吩咐{\PN{大卫}}的先见{\PN{迦得}}说:
\VS{10}「你去告诉{\PN{大卫}}说,耶和华如此说:我有三样灾,随你选择一样,我好降与你。」
\par }{\PP \VS{11}于是,{\PN{迦得}}来见{\PN{大卫}},对他说:「耶和华如此说:『你可以随意选择:
\VS{12}或三年的饥荒;或败在你敌人面前,被敌人的刀追杀三个月;或在你国中有耶和华的刀,就是三日的瘟疫,耶和华的使者在{\PN{以色列}}的四境施行毁灭。』现在你要想一想,我好回复那差我来的。」
\VS{13}{\PN{大卫}}对{\PN{迦得}}说:「我甚为难。我愿落在耶和华的手里,因为他有丰盛的怜悯;我不愿落在人的手里。」
\par }{\PP \VS{14}于是,耶和华降瘟疫与{\PN{以色列}}人,{\PN{以色列}}人就死了七万。
\VS{15}神差遣使者去灭{\PN{耶路撒冷}},刚要灭的时候,耶和华看见后悔,就不降这灾了,吩咐灭城的天使说:「够了,住手吧!」那时,耶和华的使者站在{\PN{耶布斯}}人{\PN{阿珥楠}}的禾场那里。
\VS{16}{\PN{大卫}}举目,看见耶和华的使者站在天地间,手里有拔出来的刀,伸在{\PN{耶路撒冷}}以上。{\PN{大卫}}和长老都身穿麻衣,面伏于地。
\VS{17}{\PN{大卫}}祷告 神说:「吩咐数点百姓的不是我吗?我犯了罪,行了恶,但这群羊做了什么呢?愿耶和华—我 神的手攻击我和我的父家,不要攻击你的民,降瘟疫与他们。」
\par }{\PP \VS{18}耶和华的使者吩咐{\PN{迦得}}去告诉{\PN{大卫}},叫他上去,在{\PN{耶布斯}}人{\PN{阿珥楠}}的禾场上为耶和华筑一座坛;
\VS{19}{\PN{大卫}}就照着{\PN{迦得}}奉耶和华名所说的话上去了。
\VS{20}那时{\PN{阿珥楠}}正打麦子,回头看见天使,就和他四个儿子都藏起来了。
\VS{21}{\PN{大卫}}到了{\PN{阿珥楠}}那里,{\PN{阿珥楠}}看见{\PN{大卫}},就从禾场上出去,脸伏于地,向他下拜。
\VS{22}{\PN{大卫}}对{\PN{阿珥楠}}说:「你将这禾场与相连之地卖给我,我必给你足价,我好在其上为耶和华筑一座坛,使民间的瘟疫止住。」
\VS{23}{\PN{阿珥楠}}对{\PN{大卫}}说:「你可以用这禾场,愿我主我王照你所喜悦的去行。我也将牛给你作燔祭,把打粮的器具当柴烧,拿麦子作素祭。这些我都送给你。」
\VS{24}{\PN{大卫}}王对{\PN{阿珥楠}}说:「不然!我必要用足价向你买。我不用你的物献给耶和华,也不用白得之物献为燔祭。」
\VS{25}于是{\PN{大卫}}为那块地平了六百舍客勒金子给{\PN{阿珥楠}}。
\VS{26}{\PN{大卫}}在那里为耶和华筑了一座坛,献燔祭和平安祭,求告耶和华。耶和华就应允他,使火从天降在燔祭坛上。
\VS{27}耶和华吩咐使者,他就收刀入鞘。
\par }{\PP \VS{28}那时,{\PN{大卫}}见耶和华在{\PN{耶布斯}}人{\PN{阿珥楠}}的禾场上应允了他,就在那里献祭。
\VS{29}{\PN{摩西}}在旷野所造之耶和华的帐幕和燔祭坛都在{\PN{基遍}}的高处;
\VS{30}只是{\PN{大卫}}不敢前去求问 神,因为惧怕耶和华使者的刀。

\par }\Chap{22}{\PP \VerseOne{1}大卫说:「这就是耶和华 神的殿,为{\PN{以色列}}人献燔祭的坛。」
\par }{\SH 筹备建殿
\par }{\PP \VS{2}{\PN{大卫}}吩咐聚集住{\PN{以色列}}地的外邦人,从其中派石匠凿石头,要建造 神的殿。
\VS{3}{\PN{大卫}}预备许多铁做门上的钉子和钩子,又预备许多铜,多得无法可称;
\VS{4}又预备无数的香柏木,因为{\PN{西顿}}人和{\PN{泰尔}}人给{\PN{大卫}}运了许多香柏木来。
\VS{5}{\PN{大卫}}说:「我儿子{\PN{所罗门}}还年幼娇嫩,要为耶和华建造的殿宇必须高大辉煌,使名誉荣耀传遍万国;所以我要为殿预备材料。」于是,{\PN{大卫}}在未死之先预备的材料甚多。
\par }{\PP \VS{6}{\PN{大卫}}召了他儿子{\PN{所罗门}}来,嘱咐他给耶和华—{\PN{以色列}}的 神建造殿宇,
\VS{7}对{\PN{所罗门}}说:「我儿啊,我心里本想为耶和华—我 神的名建造殿宇,
\VS{8}只是耶和华的话临到我说:『你流了多人的血,打了多次大仗,你不可为我的名建造殿宇,因为你在我眼前使多人的血流在地上。
\VS{9}你要生一个儿子,他必作太平的人;我必使他安静,不被四围的仇敌扰乱。他的名要叫{\PN{所罗门}}\FTNT{}{{\FR 22:9: }就是太平的意思}。他在位的日子,我必使{\PN{以色列}}人平安康泰。
\VS{10}他必为我的名建造殿宇。他要作我的子;我要作他的父。他作{\PN{以色列}}王;我必坚定他的国位,直到永远。』
\VS{11}我儿啊,现今愿耶和华与你同在,使你亨通,照他指着你说的话,建造耶和华—你 神的殿。
\VS{12}但愿耶和华赐你聪明智慧,好治理{\PN{以色列}}国,遵行耶和华—你 神的律法。
\VS{13}你若谨守遵行耶和华借{\PN{摩西}}吩咐{\PN{以色列}}的律例典章,就得亨通。你当刚强壮胆,不要惧怕,也不要惊惶。
\VS{14}我在困难之中为耶和华的殿预备了金子十万他连得,银子一百万他连得,铜和铁多得无法可称;我也预备了木头、石头,你还可以增添。
\VS{15}你有许多匠人,就是石匠、木匠,和一切能做各样工的巧匠,
\VS{16}并有无数的金银铜铁。你当起来办事,愿耶和华与你同在。」
\par }{\PP \VS{17}{\PN{大卫}}又吩咐{\PN{以色列}}的众首领帮助他儿子{\PN{所罗门}},{\ADD{说}}:
\VS{18}「耶和华—你们的 神不是与你们同在吗?不是叫你们四围都平安吗?因他已将这地的居民交在我手中,这地就在耶和华与他百姓面前制伏了。
\VS{19}现在你们应当立定心意,寻求耶和华—你们的 神;也当起来建造耶和华 神的圣所,好将耶和华的约柜和供奉 神的圣器皿都搬进为耶和华名建造的殿里。」

\par }\Chap{23}{\PP \VerseOne{1}{\PN{大卫}}年纪老迈,日子满足,就立他儿子{\PN{所罗门}}作{\PN{以色列}}的王。
\par }{\SH 利未人的职务
\par }{\PP \VS{2}{\PN{大卫}}招聚{\PN{以色列}}的众首领和祭司{\PN{利未}}人。
\VS{3}{\PN{利未}}人从三十岁以外的都被数点,他们男丁的数目共有三万八千;
\VS{4}其中有二万四千人管理耶和华殿的事,有六千人作官长和士师,
\VS{5}有四千人作守门的,又有四千人用{\PN{大卫}}所做的乐器颂赞耶和华。
\VS{6}{\PN{大卫}}将{\PN{利未}}人{\PN{革顺}}、{\PN{哥辖}}、{\PN{米拉利}}的子孙分了班次。
\par }{\PP \VS{7}{\PN{革顺}}的子孙有{\PN{拉但}}和{\PN{示每}}。
\VS{8}{\PN{拉但}}的长子是{\PN{耶歇}},还有{\PN{细坦}}和{\PN{约珥}},共三人。
\VS{9}{\PN{示每}}的儿子是{\PN{示罗密}}、{\PN{哈薛}}、{\PN{哈兰}}三人。这是{\PN{拉但}}族的族长。
\VS{10}{\PN{示每}}的儿子是{\PN{雅哈}}、{\PN{细拿}}、{\PN{耶乌施}}、{\PN{比利亚}}共四人。
\VS{11}{\PN{雅哈}}是长子,{\PN{细撒}}是次子。但{\PN{耶乌施}}和{\PN{比利亚}}的子孙不多,所以算为一族。
\par }{\PP \VS{12}{\PN{哥辖}}的儿子是{\PN{暗兰}}、{\PN{以斯哈}}、{\PN{希伯伦}}、{\PN{乌薛}}共四人。
\VS{13}{\PN{暗兰}}的儿子是{\PN{亚伦}}、{\PN{摩西}}。{\PN{亚伦}}和他的子孙分出来,好分别至圣的物,在耶和华面前烧香、事奉他,奉他的名祝福,直到永远。
\VS{14}至于神人{\PN{摩西}},他的子孙名字记在{\PN{利未}}支派的册上。
\VS{15}{\PN{摩西}}的儿子是{\PN{革舜}}和{\PN{以利以谢}}。
\VS{16}{\PN{革舜}}的长子是{\PN{细布业}};
\VS{17}{\PN{以利以谢}}的儿子是{\PN{利哈比雅}}。{\PN{以利以谢}}没有别的儿子,但{\PN{利哈比雅}}的子孙甚多。
\VS{18}{\PN{以斯哈}}的长子是{\PN{示罗密}}。
\VS{19}{\PN{希伯伦}}的长子是{\PN{耶利雅}},次子是{\PN{亚玛利亚}},三子是{\PN{雅哈悉}},四子是{\PN{耶加面}}。
\VS{20}{\PN{乌薛}}的长子是{\PN{米迦}},次子是{\PN{耶西雅}}。
\par }{\PP \VS{21}{\PN{米拉利}}的儿子是{\PN{抹利}}、{\PN{母示}}。{\PN{抹利}}的儿子是{\PN{以利亚撒}}、{\PN{基士}}。
\VS{22}{\PN{以利亚撒}}死了,没有儿子,只有女儿,他们本族{\PN{基士}}的儿子娶了她们为妻。
\VS{23}{\PN{母示}}的儿子是{\PN{末力}}、{\PN{以得}}、{\PN{耶利摩}}共三人。
\par }{\PP \VS{24}以上{\PN{利未}}子孙作族长的,照着男丁的数目,从二十岁以外,都办耶和华殿的事务。
\VS{25}{\PN{大卫}}说:「耶和华—{\PN{以色列}}的 神已经使他的百姓平安,他永远住在{\PN{耶路撒冷}}。
\VS{26}{\PN{利未}}人不必再抬帐幕和其中所用的一切器皿了。」
\VS{27}照着{\PN{大卫}}临终所吩咐的,{\PN{利未}}人从二十岁以外的都被数点。
\VS{28}他们的职任是服事{\PN{亚伦}}的子孙,在耶和华的殿和院子,并屋中办事,洁净一切圣物,就是办 神殿的事务,
\VS{29}并管理陈设饼,素祭的细面,或无酵薄饼,或用盘烤,或用油调和的物,又管理各样的升斗尺度;
\VS{30}每日早晚,站立称谢赞美耶和华,
\VS{31}又在安息日、月朔,并节期,按数照例,将燔祭常常献给耶和华;
\VS{32}又看守会幕和圣所,并守{\ADD{耶和华}}吩咐他们弟兄{\PN{亚伦}}子孙的,办耶和华殿的事。

\par }\Chap{24}{\SH 祭司的职务
\par }{\PP \VerseOne{1}{\PN{亚伦}}子孙的班次记在下面:{\PN{亚伦}}的儿子是{\PN{拿答}}、{\PN{亚比户}}、{\PN{以利亚撒}}、{\PN{以他玛}}。
\VS{2}{\PN{拿答}}、{\PN{亚比户}}死在他们父亲之先,没有留下儿子;故此,{\PN{以利亚撒}}、{\PN{以他玛}}供祭司的职分。
\VS{3}{\PN{以利亚撒}}的子孙{\PN{撒督}}和{\PN{以他玛}}的子孙{\PN{亚希米勒}},同着{\PN{大卫}}将他们的族弟兄分成班次。
\VS{4}{\PN{以利亚撒}}子孙中为首的比{\PN{以他玛}}子孙中为首的更多,分班如下:{\PN{以利亚撒}}的子孙中有十六个族长,{\PN{以他玛}}的子孙中有八个族长;
\VS{5}都掣签分立,彼此一样。在圣所和 神面前作首领的有{\PN{以利亚撒}}的子孙,也有{\PN{以他玛}}的子孙。
\VS{6}作书记的{\PN{利未}}人{\PN{拿坦业}}的儿子{\PN{示玛雅}}在王和首领,与祭司{\PN{撒督}}、{\PN{亚比亚他}}的儿子{\PN{亚希米勒}},并祭司{\PN{利未}}人的族长面前记录他们的名字。在{\PN{以利亚撒}}的子孙中取一族,在{\PN{以他玛}}的子孙中取一族。
\par }{\PP \VS{7}掣签的时候,第一掣出来的是{\PN{耶何雅立}},第二是{\PN{耶大雅}},
\VS{8}第三是{\PN{哈琳}},第四是{\PN{梭琳}},
\VS{9}第五是{\PN{玛基雅}},第六是{\PN{米雅民}},
\VS{10}第七是{\PN{哈歌斯}},第八是{\PN{亚比雅}},
\VS{11}第九是{\PN{耶书亚}},第十是{\PN{示迦尼}},
\VS{12}第十一是{\PN{以利亚实}},第十二是{\PN{雅金}},
\VS{13}第十三是{\PN{胡巴}},第十四是{\PN{耶是比押}},
\VS{14}第十五是{\PN{璧迦}},第十六是{\PN{音麦}},
\VS{15}第十七是{\PN{希悉}},第十八是{\PN{哈辟悉}},
\VS{16}第十九是{\PN{毗他希雅}},第二十是{\PN{以西结}},
\VS{17}第二十一是{\PN{雅斤}},第二十二是{\PN{迦末}},
\VS{18}第二十三是{\PN{第来雅}},第二十四是{\PN{玛西亚}}。
\VS{19}这就是他们的班次,要照耶和华—{\PN{以色列}}的 神借他们祖宗{\PN{亚伦}}所吩咐的条例进入耶和华的殿办理事务。
\par }{\SH 利未人名单
\par }{\PP \VS{20}{\PN{利未}}其余的子孙如下:{\PN{暗兰}}的子孙里有{\PN{书巴业}};{\PN{书巴业}}的子孙里有{\PN{耶希底亚}}。
\VS{21}{\PN{利哈比雅}}的子孙里有长子{\PN{伊示雅}}。
\VS{22}{\PN{以斯哈}}的子孙里有{\PN{示罗摩}};{\PN{示罗摩}}的子孙里有{\PN{雅哈}}。
\VS{23}{\ADD{
{\PN{希伯伦}} 的}}子孙里有{\ADD{长子}}{\PN{耶利雅}},次子{\PN{亚玛利亚}},三子{\PN{雅哈悉}},四子{\PN{耶加面}}。
\VS{24}{\PN{乌薛}}的子孙里有{\PN{米迦}};{\PN{米迦}}的子孙里有{\PN{沙密}}。
\VS{25}{\PN{米迦}}的兄弟是{\PN{伊示雅}};{\PN{伊示雅}}的子孙里有{\PN{撒迦利雅}}。
\VS{26}{\PN{米拉利}}的儿子是{\PN{抹利}}、{\PN{母示}}、{\ADD{
{\PN{雅西雅}}}};{\PN{雅西雅}}的儿子有{\PN{比挪}};
\VS{27}{\PN{米拉利}}的子孙里有{\PN{雅西雅}}的儿子{\PN{比挪}}、{\PN{朔含}}、{\PN{撒刻}}、{\PN{伊比利}}。
\VS{28}{\PN{抹利}}的儿子是{\PN{以利亚撒}};{\PN{以利亚撒}}没有儿子。
\VS{29}{\PN{基士}}的子孙里有{\PN{耶拉篾}}。
\VS{30}{\PN{母示}}的儿子是{\PN{末力}}、{\PN{以得}}、{\PN{耶利摩}}。按着宗族这都是{\PN{利未}}的子孙。
\VS{31}他们在{\PN{大卫}}王和{\PN{撒督}},并{\PN{亚希米勒}}与祭司{\PN{利未}}人的族长面前掣签,正如他们弟兄{\PN{亚伦}}的子孙一般。各族的长者与兄弟没有分别。

\par }\Chap{25}{\SH 圣殿中的圣乐人员
\par }{\PP \VerseOne{1}{\PN{大卫}}和众首领分派{\PN{亚萨}}、{\PN{希幔}},并{\PN{耶杜顿}}的子孙弹琴、鼓瑟、敲钹、唱歌\FTNT{}{{\FR 25:1: }原文是说预言;本章同}。他们供职的人数记在下面:
\VS{2}{\PN{亚萨}}的儿子{\PN{撒刻}}、{\PN{约瑟}}、{\PN{尼探雅}}、{\PN{亚萨利拉}}都归{\PN{亚萨}}指教,遵王的旨意唱歌。
\VS{3}{\PN{耶杜顿}}的儿子{\PN{基大利}}、{\PN{西利}}、{\PN{耶筛亚}}、{\PN{哈沙比雅}}、{\PN{玛他提雅}}、{\PN{示每}}共六人,都归他们父亲{\PN{耶杜顿}}指教,弹琴,唱歌,称谢,颂赞耶和华。
\VS{4}{\PN{希幔}}的儿子{\PN{布基雅}}、{\PN{玛探雅}}、{\PN{乌薛}}、{\PN{细布业}}、{\PN{耶利摩}}、{\PN{哈拿尼雅}}、{\PN{哈拿尼}}、{\PN{以利亚他}}、{\PN{基大利提}}、{\PN{罗幔提·以谢}}、{\PN{约施比加沙}}、{\PN{玛罗提}}、{\PN{何提}}、{\PN{玛哈秀}};
\VS{5}这都是{\PN{希幔}}的儿子,吹角颂赞。{\PN{希幔}}奉 神之命作王的先见。 神赐给{\PN{希幔}}十四个儿子,三个女儿,
\VS{6}都归他们父亲指教,在耶和华的殿唱歌、敲钹、弹琴、鼓瑟,办 神殿的事务。{\PN{亚萨}}、{\PN{耶杜顿}}、{\PN{希幔}}都是王所命定的。
\VS{7}他们和他们的弟兄学习颂赞耶和华;善于歌唱的共有二百八十八人。
\VS{8}这些人无论大小,为师的、为徒的,都一同掣签分了班次。
\par }{\PP \VS{9}掣签的时候,第一掣出来的是{\PN{亚萨}}的儿子{\PN{约瑟}}。第二是{\PN{基大利}};他和他弟兄并儿子共十二人。
\VS{10}第三是{\PN{撒刻}};他和他儿子并弟兄共十二人。
\VS{11}第四是{\PN{伊洗利}};他和他儿子并弟兄共十二人。
\VS{12}第五是{\PN{尼探雅}};他和他儿子并弟兄共十二人。
\VS{13}第六是{\PN{布基雅}};他和他儿子并弟兄共十二人。
\VS{14}第七是{\PN{耶萨利拉}};他和他儿子并弟兄共十二人。
\VS{15}第八是{\PN{耶筛亚}};他和他儿子并弟兄共十二人。
\VS{16}第九是{\PN{玛探雅}};他和他儿子并弟兄共十二人。
\VS{17}第十是{\PN{示每}};他和他儿子并弟兄共十二人。
\VS{18}第十一是{\PN{亚萨烈}};他和他儿子并弟兄共十二人。
\VS{19}第十二是{\PN{哈沙比雅}};他和他儿子并弟兄共十二人。
\VS{20}第十三是{\PN{书巴业}};他和他儿子并弟兄共十二人。
\VS{21}第十四是{\PN{玛他提雅}};他和他儿子并弟兄共十二人。
\VS{22}第十五是{\PN{耶利摩}};他和他儿子并弟兄共十二人。
\VS{23}第十六是{\PN{哈拿尼雅}};他和他儿子并弟兄共十二人。
\VS{24}第十七是{\PN{约施比加沙}};他和他儿子并弟兄共十二人。
\VS{25}第十八是{\PN{哈拿尼}};他和他儿子并弟兄共十二人。
\VS{26}第十九是{\PN{玛罗提}};他和他儿子并弟兄共十二人。
\VS{27}第二十是{\PN{以利亚他}};他和他儿子并弟兄共十二人。
\VS{28}第二十一是{\PN{何提}};他和他儿子并弟兄共十二人。
\VS{29}第二十二是{\PN{基大利提}};他和他儿子并弟兄共十二人。
\VS{30}第二十三是{\PN{玛哈秀}};他和他儿子并弟兄共十二人。
\VS{31}第二十四是{\PN{罗幔提·以谢}};他和他儿子并弟兄共十二人。

\par }\Chap{26}{\SH 圣殿的卫队
\par }{\PP \VerseOne{1}守门的班次记在下面:{\PN{可拉}}族{\PN{亚萨}}的子孙中,有{\PN{可利}}的儿子{\PN{米施利米雅}}。
\VS{2}{\PN{米施利米雅}}的长子是{\PN{撒迦利亚}},次子是{\PN{耶叠}},三子是{\PN{西巴第雅}},四子是{\PN{耶提聂}},
\VS{3}五子是{\PN{以拦}},六子是{\PN{约哈难}},七子是{\PN{以利约乃}}。
\VS{4}{\PN{俄别·以东}}的长子是{\PN{示玛雅}},次子是{\PN{约萨拔}},三子是{\PN{约亚}},四子是{\PN{沙甲}},五子是{\PN{拿坦业}},
\VS{5}六子是{\PN{亚米利}},七子是{\PN{以萨迦}},八子是{\PN{毗乌利太}},因为 神赐福与{\PN{俄别·以东}}。
\VS{6}他的儿子{\PN{示玛雅}}有几个儿子,都是大能的壮士,掌管父亲的家。
\VS{7}{\PN{示玛雅}}的儿子是{\PN{俄得尼}}、{\PN{利法益}}、{\PN{俄备得}}、{\PN{以利萨巴}}。{\PN{以利萨巴}}的弟兄是壮士,还有{\PN{以利户}}和{\PN{西玛迦}}。
\VS{8}这都是{\PN{俄别·以东}}的子孙,他们和他们的儿子并弟兄,都是善于办事的壮士。{\PN{俄别·以东}}的子孙共六十二人。
\VS{9}{\PN{米施利米雅}}的儿子和弟兄都是壮士,共十八人。
\VS{10}{\PN{米拉利}}子孙{\PN{何萨}}有几个儿子:长子是{\PN{申利}},他原不是长子,是他父亲立他作长子。
\VS{11}次子是{\PN{希勒家}},三子是{\PN{底巴利雅}},四子是{\PN{撒迦利亚}}。{\PN{何萨}}的儿子并弟兄共十三人。
\par }{\PP \VS{12}这些人都是守门的班长,与他们的弟兄一同在耶和华殿里按班供职。
\VS{13}他们无论大小,都按着宗族掣签分守各门。
\VS{14}掣签守东门的是{\PN{示利米雅}};他的儿子{\PN{撒迦利亚}}是精明的谋士,掣签守北门。
\VS{15}{\PN{俄别·以东}}守南门,他的儿子守库房。
\VS{16}{\PN{书聘}}与{\PN{何萨}}守西门,在靠近{\PN{沙利基}}门、通着往上去的街道上,班与班相对。
\VS{17}每日东门有六个{\PN{利未}}人,北门有四个,南门有四个,库房有两个,又有两个轮班替换。
\VS{18}在西面街道上有四个,在游廊上有两个。
\VS{19}以上是{\PN{可拉}}子孙和{\PN{米拉利}}子孙守门的班次。
\par }{\SH 其他职守
\par }{\PP \VS{20}{\PN{利未}}子孙中有{\PN{亚希雅}}掌管 神殿的府库和圣物的府库。
\VS{21}{\PN{革顺}}族、{\PN{拉但}}子孙里,作族长的是{\PN{革顺}}族{\PN{拉但}}的子孙{\PN{耶希伊利}}。
\par }{\PP \VS{22}{\PN{耶希伊利}}的儿子{\PN{西坦}}和他兄弟{\PN{约珥}}掌管耶和华殿里的府库。
\VS{23}{\PN{暗兰}}族、{\PN{以斯哈}}族、{\PN{希伯伦}}族、{\PN{乌泄}}族{\ADD{也有职分}}。
\VS{24}{\PN{摩西}}的孙子、{\PN{革舜}}的儿子{\PN{细布业}}掌管府库。
\VS{25}还有他的弟兄{\PN{以利以谢}}。{\PN{以利以谢}}的儿子是{\PN{利哈比雅}};{\PN{利哈比雅}}的儿子是{\PN{耶筛亚}};{\PN{耶筛亚}}的儿子是{\PN{约兰}};{\PN{约兰}}的儿子是{\PN{细基利}};{\PN{细基利}}的儿子是{\PN{示罗密}}。
\VS{26}这{\PN{示罗密}}和他的弟兄掌管府库的圣物,就是{\PN{大卫}}王和众族长、千夫长、百夫长,并军长所分别为圣的物。
\VS{27}他们将争战时所夺的财物分别为圣,以备修造耶和华的殿。
\VS{28}先见{\PN{撒母耳}}、{\PN{基士}}的儿子{\PN{扫罗}}、{\PN{尼珥}}的儿子{\PN{押尼珥}}、{\PN{洗鲁雅}}的儿子{\PN{约押}}所分别为圣的物都归{\PN{示罗密}}和他的弟兄掌管。
\par }{\SH 其余利未人的职守
\par }{\PP \VS{29}{\PN{以斯哈}}族有{\PN{基拿尼雅}}和他众子作官长和士师,管理{\PN{以色列}}的外事。
\VS{30}{\PN{希伯伦}}族有{\PN{哈沙比雅}}和他弟兄一千七百人,都是壮士,在{\PN{约旦河}}西、{\PN{以色列}}地办理耶和华与王的事。
\VS{31}{\PN{希伯伦}}族中有{\PN{耶利雅}}作族长。{\PN{大卫}}作王第四十年,在{\PN{基列}}的{\PN{雅谢}},从这族中寻得大能的勇士。
\VS{32}{\PN{耶利雅}}的弟兄有二千七百人,都是壮士,且作族长;{\PN{大卫}}王派他们在{\PN{吕便}}支派、{\PN{迦得}}支派、{\PN{玛拿西}}半支派中办理 神和王的事。

\par }\Chap{27}{\SH 军事长官名单
\par }{\PP \VerseOne{1}{\PN{以色列}}人的族长、千夫长、百夫长,和官长都分定班次,每班是二万四千人,周年按月轮流,替换出入服事王。
\par }{\PP \VS{2}正月第一班的班长是{\PN{撒巴第业}}的儿子{\PN{雅朔班}};他班内有二万四千人。
\VS{3}他是{\PN{法勒斯}}的子孙,统管正月{\ADD{班}}的一切军长。
\VS{4}二月的班长是{\PN{亚哈希}}人{\PN{朵代}},还有副官{\PN{密基罗}};他班内有二万四千人。
\VS{5}三月第三班的班长\FTNT{}{{\FR 27:5: }原文是军长;下同}是祭司{\PN{耶何耶大}}的儿子{\PN{比拿雅}};他班内有二万四千人。
\VS{6}这{\PN{比拿雅}}是那三十人中的勇士,管理那三十人;他班内又有他儿子{\PN{暗米萨拔}}。
\VS{7}四月第四班的班长是{\PN{约押}}的兄弟{\PN{亚撒黑}}。接续他的是他儿子{\PN{西巴第雅}};他班内有二万四千人。
\VS{8}五月第五班的班长是{\PN{伊斯拉}}人{\PN{珊合}};他班内有二万四千人。
\VS{9}六月第六班的班长是{\PN{提哥亚}}人{\PN{益吉}}的儿子{\PN{以拉}};他班内有二万四千人。
\VS{10}七月第七班的班长是{\PN{以法莲}}族{\PN{比伦}}人{\PN{希利斯}};他班内有二万四千人。
\VS{11}八月第八班的班长是{\PN{谢拉}}族{\PN{户沙}}人{\PN{西比该}};他班内有二万四千人。
\VS{12}九月第九班的班长是{\PN{便雅悯}}族{\PN{亚拿突}}人{\PN{亚比以谢}};他班内有二万四千人。
\VS{13}十月第十班的班长是{\PN{谢拉}}族{\PN{尼陀法}}人{\PN{玛哈莱}};他班内有二万四千人。
\VS{14}十一月第十一班的班长是{\PN{以法莲}}族{\PN{比拉顿}}人{\PN{比拿雅}};他班内有二万四千人。
\VS{15}十二月第十二班的班长是{\PN{俄陀聂}}族{\PN{尼陀法}}人{\PN{黑玳}};他班内有二万四千人。
\par }{\SH 以色列各支派的行政长官
\par }{\PP \VS{16}管理{\PN{以色列}}众支派的记在下面:管{\PN{吕便}}人的是{\PN{细基利}}的儿子{\PN{以利以谢}};管{\PN{西缅}}人的是{\PN{玛迦}}的儿子{\PN{示法提雅}};
\VS{17}管{\PN{利未}}人的是{\PN{基母利}}的儿子{\PN{哈沙比雅}};管{\PN{亚伦}}子孙的是{\PN{撒督}};
\VS{18}管{\PN{犹大}}人的是{\PN{大卫}}的一个哥哥{\PN{以利户}};管{\PN{以萨迦}}人的是{\PN{米迦勒}}的儿子{\PN{暗利}};
\VS{19}管{\PN{西布伦}}人的是{\PN{俄巴第雅}}的儿子{\PN{伊施玛雅}};管{\PN{拿弗他利}}人的是{\PN{亚斯列}}的儿子{\PN{耶利摩}};
\VS{20}管{\PN{以法莲}}人的是{\PN{阿撒细雅}}的儿子{\PN{何细亚}};管{\PN{玛拿西}}半支派的是{\PN{毗大雅}}的儿子{\PN{约珥}};
\VS{21}管{\PN{基列}}地{\PN{玛拿西}}那半支派的是{\PN{撒迦利亚}}的儿子{\PN{易多}};管{\PN{便雅悯}}人的是{\PN{押尼珥}}的儿子{\PN{雅西业}};
\VS{22}管{\PN{但}}人的是{\PN{耶罗罕}}的儿子{\PN{亚萨列}}。以上是{\PN{以色列}}众支派的首领。
\VS{23}{\PN{以色列}}人二十岁以内的,{\PN{大卫}}没有记其数目;因耶和华曾应许说,必加增{\PN{以色列}}人如天上的星那样多。
\VS{24}{\PN{洗鲁雅}}的儿子{\PN{约押}}动手数点,当时{\ADD{耶和华}}的烈怒临到{\PN{以色列}}人;因此,没有点完,数目也没有写在{\PN{大卫}}王记上。
\par }{\SH 管理王产的官员
\par }{\PP \VS{25}掌管王府库的是{\PN{亚叠}}的儿子{\PN{押斯马威}}。掌管田野城邑村庄保障之仓库的是{\PN{乌西雅}}的儿子{\PN{约拿单}}。
\VS{26}掌管耕田种地的是{\PN{基绿}}的儿子{\PN{以斯利}}。
\VS{27}掌管葡萄园的是{\PN{拉玛}}人{\PN{示每}}。掌管葡萄园酒窖的是{\PN{实弗米}}人{\PN{撒巴底}}。
\VS{28}掌管高原橄榄树和桑树的是{\PN{基第利}}人{\PN{巴勒·哈南}}。掌管油库的是{\PN{约阿施}}。
\VS{29}掌管{\PN{沙
}}牧放牛群的是{\PN{沙
}}人{\PN{施提赉}}。掌管山谷牧养牛群的是{\PN{亚第赉}}的儿子{\PN{沙法}}。
\VS{30}掌管驼{\ADD{群}}的是{\PN{以实玛利}}人{\PN{阿比勒}}。掌管驴{\ADD{群}}的是{\PN{米
}}人{\PN{耶希底亚}}。掌管羊群的是{\PN{夏甲}}人{\PN{雅悉}}。
\VS{31}这都是给{\PN{大卫}}王掌管产业的。
\par }{\SH 大卫的谋士
\par }{\PP \VS{32}{\PN{大卫}}的叔叔{\PN{约拿单}}作谋士;这人有智慧,又作书记。{\PN{哈摩尼}}的儿子{\PN{耶歇}}作王众子的师傅。
\VS{33}{\PN{亚希多弗}}也作王的谋士。{\PN{亚基}}人{\PN{户筛}}作王的陪伴。
\VS{34}{\PN{亚希多弗}}之后,有{\PN{比拿雅}}的儿子{\PN{耶何耶大}}和{\PN{亚比亚他}}接续他{\ADD{作谋士}}。{\PN{约押}}作王的元帅。

\par }\Chap{28}{\SH 大卫对建殿的指示
\par }{\PP \VerseOne{1}{\PN{大卫}}招聚{\PN{以色列}}各支派的首领和轮班服事王的军长,与千夫长、百夫长,掌管王和王子产业牲畜的,并太监,以及大能的勇士,都到{\PN{耶路撒冷}}来。
\VS{2}{\PN{大卫}}王就站起来,说:「我的弟兄,我的百姓啊,你们当听我言,我心里本想建造殿宇,安放耶和华的约柜,作为我 神的脚凳;我已经预备建造的材料。
\VS{3}只是 神对我说:『你不可为我的名建造殿宇;因你是战士,流了人的血。』
\VS{4}然而,耶和华—{\PN{以色列}}的 神在我父的全家拣选我作{\PN{以色列}}的王,直到永远。因他拣选{\PN{犹大}}为首领;在{\PN{犹大}}支派中拣选我父家,在我父的众子里喜悦我,立我作{\PN{以色列}}众人的王。
\VS{5}耶和华赐我许多儿子,在我儿子中拣选{\PN{所罗门}}坐耶和华的国位,治理{\PN{以色列}}人。
\VS{6}耶和华对我说:『你儿子{\PN{所罗门}}必建造我的殿和院宇;因为我拣选他作我的子,我也必作他的父。
\VS{7}他若恒久遵行我的诫命典章如今日一样,我就必坚定他的国位,直到永远。』
\VS{8}现今在耶和华的会中,{\PN{以色列}}众人眼前所说的,我们的 神也听见了。你们应当寻求耶和华—你们 神的一切诫命,谨守遵行,如此你们可以承受这美地,遗留给你们的子孙,永远为业。
\par }{\PP \VS{9}「我儿{\PN{所罗门}}哪,你当认识耶和华—你父的 神,诚心乐意地事奉他;因为他鉴察众人的心,知道一切心思意念。你若寻求他,他必使你寻见;你若离弃他,他必永远丢弃你。
\VS{10}你当谨慎,因耶和华拣选你建造殿宇作为圣所。你当刚强去行。」
\par }{\PP \VS{11}{\PN{大卫}}将{\ADD{殿的}}游廊、旁屋、府库、楼房、内殿,和施恩所的样式指示他儿子{\PN{所罗门}},
\VS{12}又将被灵感动所得的样式,就是耶和华 神殿的院子、周围的房屋、殿的府库,和圣物府库的一切样式都指示他;
\VS{13}又指示他祭司和{\PN{利未}}人的班次与耶和华殿里各样的工作,并耶和华殿里一切器皿的样式,
\VS{14}以及各样应用金器的分两和各样应用银器的分两,
\VS{15}金灯台和金灯的分两,银灯台和银灯的分两,轻重各都合宜;
\VS{16}陈设饼金桌子的分两,银桌子的分两,
\VS{17}精金的肉叉子、盘子,和爵的分两,各金碗与各银碗的分两,
\VS{18}精金香坛的分两,并用金子做基路伯\FTNT{}{{\FR 28:18: }原文是用金子做车式的基路伯};基路伯张开{\ADD{翅膀}},遮掩耶和华的约柜。
\VS{19}{\ADD{
{\PN{大卫}}}}说:「这一切工作的样式都是耶和华用手划出来使我明白的。」
\par }{\PP \VS{20}{\PN{大卫}}又对他儿子{\PN{所罗门}}说:「你当刚强壮胆去行!不要惧怕,也不要惊惶。因为耶和华 神就是我的 神,与你同在;他必不撇下你,也不丢弃你,直到耶和华殿的工作都完毕了。
\VS{21}有祭司和{\PN{利未}}人的各班,为要办理 神殿各样的事,又有灵巧的人在各样的工作上乐意帮助你;并有众首领和众民一心听从你的命令。」

\par }\Chap{29}{\SH 为建殿而献的礼物
\par }{\PP \VerseOne{1}{\PN{大卫}}王对会众说:「我儿子{\PN{所罗门}}是 神特选的,还年幼娇嫩;这工程甚大,因这殿不是为人,乃是为耶和华 神建造的。
\VS{2}我为我 神的殿已经尽力,预备金子做金器,银子做银器,铜做铜器,铁做铁器,木做木器,还有红玛瑙可镶嵌的宝石,彩石和一切的宝石,并许多汉白玉。
\VS{3}且因我心中爱慕我 神的殿,就在预备建造圣殿的材料之外,又将我自己积蓄的金银献上,建造我 神的殿,
\VS{4}就是{\PN{俄斐}}金三千他连得、精炼的银子七千他连得,以贴殿墙。
\VS{5}金子做金器,银子做银器,并借匠人的手制造一切。今日有谁乐意将自己献给耶和华呢?」
\par }{\PP \VS{6}于是,众族长和{\PN{以色列}}各支派的首领、千夫长、百夫长,并监管王工的官长,都乐意献上。
\VS{7}他们为 神殿的使用献上金子五千他连得零一万达利克,银子一万他连得,铜一万八千他连得,铁十万他连得。
\VS{8}凡有{\ADD{宝}}石的都交给{\PN{革顺}}人{\PN{耶歇}},送入耶和华殿的府库。
\VS{9}因这些人诚心乐意献给耶和华,百姓就欢喜,{\PN{大卫}}王也大大欢喜。
\par }{\SH 大卫称颂耶和华
\par }{\PP \VS{10}所以,{\PN{大卫}}在会众面前称颂耶和华说:「耶和华—我们的父,{\PN{以色列}}的 神是应当称颂,直到永永远远的!
\VS{11}耶和华啊,尊大、能力、荣耀、强胜、威严都是你的;凡天上地下的都是你的;国度也是你的,并且你为至高,为万有之首。
\VS{12}丰富尊荣都从你而来,你也治理万物。在你手里有大能大力,使人尊大强盛都出于你。
\VS{13}我们的 神啊,现在我们称谢你,赞美你荣耀之名!
\par }{\PP \VS{14}「我算什么,我的民算什么,竟能如此乐意奉献?因为万物都从你而来,我们把从你而得的献给你。
\VS{15}我们在你面前是客旅,是寄居的,与我们列祖一样。我们在世的日子如影儿,不能长存\FTNT{}{{\FR 29:15: }或译:没有长存的指望}。
\VS{16}耶和华—我们的 神啊,我们预备这许多材料,要为你的圣名建造殿宇,都是从你而来,都是属你的。
\VS{17}我的 神啊,我知道你察验人心,喜悦正直;我以正直的心乐意献上这一切物。现在我喜欢见你的民在这里都乐意奉献与你。
\VS{18}耶和华—我们列祖{\PN{亚伯拉罕}}、{\PN{以撒}}、{\PN{以色列}}的 神啊,求你使你的民常存这样的心思意念,坚定他们的心归向你,
\VS{19}又求你赐我儿子{\PN{所罗门}}诚实的心,遵守你的命令、法度、律例,成就这一切的事,用我所预备的建造殿宇。」
\par }{\PP \VS{20}{\PN{大卫}}对全会众说:「你们应当称颂耶和华—你们的 神。」于是会众称颂耶和华—他们列祖的 神,低头拜耶和华与王。
\par }{\PP \VS{21}次日,他们向耶和华献{\ADD{平安}}祭和燔祭,就是献公牛一千只,公绵羊一千只,羊羔一千只,并同献的奠祭;又为{\PN{以色列}}众人献许多的祭。那日,他们在耶和华面前吃喝,大大欢乐。
\par }{\PP \VS{22}他们奉耶和华的命再膏{\PN{大卫}}的儿子{\PN{所罗门}}作王,又膏{\PN{撒督}}作祭司。
\VS{23}于是{\PN{所罗门}}坐在耶和华所赐的位上,接续他父亲{\PN{大卫}}作王,万事亨通;{\PN{以色列}}众人也都听从他。
\VS{24}众首领和勇士,并{\PN{大卫}}王的众子,都顺服{\PN{所罗门}}王。
\VS{25}耶和华使{\PN{所罗门}}在{\PN{以色列}}众人眼前甚为尊大,极其威严,胜过在他以前的{\PN{以色列}}王。
\par }{\SH 大卫政绩简述
\par }{\PP \VS{26}{\PN{耶西}}的儿子{\PN{大卫}}作{\PN{以色列}}众人的王,
\VS{27}作王共四十年:在{\PN{希伯
}}作王七年,在{\PN{耶路撒冷}}作王三十三年。
\VS{28}他年纪老迈,日子满足,享受丰富、尊荣,就死了。他儿子{\PN{所罗门}}接续他作王。
\VS{29}{\PN{大卫}}王始终的事都写在先见{\PN{撒母耳}}的书上和先知{\PN{拿单}}并先见{\PN{迦得}}的书上。
\VS{30}他的国事和他的勇力,以及他和{\PN{以色列}}并列国所经过的事都写在这书上。
\par }