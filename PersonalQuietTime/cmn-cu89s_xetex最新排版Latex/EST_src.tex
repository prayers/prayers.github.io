\NormalFont\ShortTitle{以斯帖记}
{\MT 以斯帖记

\par }\ChapOne{1}{\SH 亚哈随鲁王废瓦实提王后
\par }{\PP \VerseOne{1}{\PN{亚哈随鲁}}作王,从{\PN{印度}}直到{\PN{古实}},统管一百二十七省。
\VS{2}{\PN{亚哈随鲁}}王在{\PN{书珊}}城的宫登基;
\VS{3}在位第三年,为他一切首领臣仆设摆筵席,有{\PN{波斯}}和{\PN{米底亚}}的权贵,就是各省的贵胄与首领,在他面前。
\VS{4}他把他荣耀之国的丰富和他美好威严的尊贵给他们看了许多日,就是一百八十日。
\VS{5}这日子满了,又为所有住{\PN{书珊}}城的大小人民在御园的院子里设摆筵席七日。
\VS{6}有白色、绿色、蓝色的{\ADD{帐子}},用细麻绳、紫色绳从银环内系在白玉石柱上;有金银的床榻摆在红、白、黄、黑玉石的铺石地上。
\VS{7}用金器皿赐酒,器皿各有不同。御酒甚多,足显王的厚意。
\VS{8}喝酒有例,不准勉强人,因王吩咐宫里的一切臣宰,让人各随己意。
\VS{9}王后{\PN{瓦实提}}在{\PN{亚哈随鲁}}王的宫内也为妇女设摆筵席。
\par }{\PP \VS{10}第七日,{\PN{亚哈随鲁}}王饮酒,心中快乐,就吩咐在他面前侍立的七个太监{\PN{米户幔}}、{\PN{比斯他}}、{\PN{哈波拿}}、{\PN{比革他}}、{\PN{亚拔他}}、{\PN{西达}}、{\PN{甲迦}},
\VS{11}请王后{\PN{瓦实提}}头戴王后的冠冕到王面前,使各等臣民看她的美貌,因为她容貌甚美。
\VS{12}王后{\PN{瓦实提}}却不肯遵太监所传的王命而来,所以王甚发怒,心如火烧。
\par }{\PP \VS{13-14}那时,在王左右常见王面、国中坐高位的,有{\PN{波斯}}和{\PN{米底亚}}的七个大臣,就是{\PN{甲示拿}}、{\PN{示达}}、{\PN{押玛他}}、{\PN{他施斯}}、{\PN{米力}}、{\PN{玛西拿}}、{\PN{米母干}},都是达时务的明哲人。按王的常规,办事必先询问知例明法的人。王问他们说:
\VS{15}「王后{\PN{瓦实提}}不遵太监所传的王命,照例应当怎样办理呢?」
\VS{16}{\PN{米母干}}在王和众首领面前回答说:「王后{\PN{瓦实提}}这事,不但得罪王,并且有害于王各省的臣民;
\VS{17}因为王后这事必传到众妇人的耳中,说:『{\PN{亚哈随鲁}}王吩咐王后{\PN{瓦实提}}到王面前,她却不来』,她们就藐视自己的丈夫。
\VS{18}今日{\PN{波斯}}和{\PN{米底亚}}的众夫人听见王后这事,必向王的大臣{\ADD{照样}}行;从此必大开藐视和忿怒之端。
\VS{19}王若以为美,就降旨写在{\PN{波斯}}和{\PN{米底亚}}人的例中,永不更改,不准{\PN{瓦实提}}再到王面前,将她王后的位分赐给比她还好的人。
\VS{20}所降的旨意传遍通国(国度本来广大),所有的妇人,无论丈夫贵贱都必尊敬他。」
\VS{21}王和众首领都以{\PN{米母干}}的话为美,王就照这话去行,
\VS{22}发诏书,用各省的文字、各族的方言通知各省,使为丈夫的在家中作主,各说本地的方言。

\par }\Chap{2}{\SH 以斯帖被立为王后
\par }{\PP \VerseOne{1}这事以后,{\PN{亚哈随鲁}}王的忿怒止息,就想念{\PN{瓦实提}}和她所行的,并怎样降旨办她。
\VS{2}于是王的侍臣对王说:「不如为王寻找美貌的处女。
\VS{3}王可以派官在国中的各省招聚美貌的处女到{\PN{书珊}}城\FTNT{}{{\FR 2:3: }或译:宫}的女院,交给掌管女子的太监{\PN{希该}},给她们当用的香品。
\VS{4}王所喜爱的女子可以立为王后,代替{\PN{瓦实提}}。」王以这事为美,就如此行。
\par }{\PP \VS{5}{\PN{书珊}}城有一个{\PN{犹大}}人,名叫{\PN{末底改}},是{\PN{便雅悯}}人{\PN{基士}}的曾孙,{\PN{示每}}的孙子,{\PN{睚珥}}的儿子。
\VS{6}从前{\PN{巴比伦}}王{\PN{尼布甲尼撒}}将{\PN{犹大}}王{\PN{耶哥尼雅}}\FTNT{}{{\FR 2:6: }又名约雅斤}和百姓从{\PN{耶路撒冷}}掳去,{\PN{末底改}}也在其内。
\VS{7}{\PN{末底改}}抚养他叔叔的女儿{\PN{哈大沙}}(后名{\PN{以斯帖}}),因为她没有父母。这女子又容貌俊美;她父母死了,{\PN{末底改}}就收她为自己的女儿。
\VS{8}王的谕旨传出,就招聚许多女子到{\PN{书珊}}城,交给掌管女子的{\PN{希该}};{\PN{以斯帖}}也送入王宫,交付{\PN{希该}}。
\VS{9}{\PN{希该}}喜悦{\PN{以斯帖}},就恩待她,急忙给她需用的香品和她所当得的分,又派所当得的七个宫女服事她,使她和她的宫女搬入女院上好的房屋。
\VS{10}{\PN{以斯帖}}未曾将籍贯宗族告诉人,因为{\PN{末底改}}嘱咐她不可叫人知道。
\VS{11}{\PN{末底改}}天天在女院前边行走,要知道{\PN{以斯帖}}平安不平安,并后事如何。
\par }{\PP \VS{12}众女子照例先洁净身体十二个月:六个月用没药油,六个月用香料和洁身之物。满了日期,然后挨次进去见{\PN{亚哈随鲁}}王。
\VS{13}女子进去见王是这样:从女院到王宫的时候,凡她所要的都必给她。
\VS{14}晚上进去,次日回到女子第二院,交给掌管妃嫔的太监{\PN{沙甲}};除非王喜爱她,再提名召她,就不再进去见王。
\par }{\PP \VS{15}{\PN{末底改}}叔叔{\PN{亚比孩}}的女儿,就是{\PN{末底改}}收为自己女儿的{\PN{以斯帖}},按次序当进去见王的时候,除了掌管女子的太监{\PN{希该}}所派定给她的,她别无所求。凡看见{\PN{以斯帖}}的都喜悦她。
\VS{16}{\PN{亚哈随鲁}}王第七年十月,就是提别月,{\PN{以斯帖}}被引入宫见王。
\VS{17}王爱{\PN{以斯帖}}过于爱众女,她在王眼前蒙宠爱比众处女更甚。王就把王后的冠冕戴在她头上,立她为王后,代替{\PN{瓦实提}}。
\VS{18}王因{\PN{以斯帖}}的缘故给众首领和臣仆设摆大筵席,又豁免各省的{\ADD{租税}},并照王的厚意大颁赏赐。
\par }{\SH 末底改救王的命
\par }{\PP \VS{19}第二次招聚处女的时候,{\PN{末底改}}坐在朝门。
\VS{20}{\PN{以斯帖}}照着{\PN{末底改}}所嘱咐的,还没有将籍贯宗族告诉人;因为{\PN{以斯帖}}遵{\PN{末底改}}的命,如抚养她的时候一样。
\VS{21}当那时候,{\PN{末底改}}坐在朝门,王的太监中有两个守门的,{\PN{辟探}}和{\PN{提列}},恼恨{\PN{亚哈随鲁}}王,想要下手害他。
\VS{22}{\PN{末底改}}知道了,就告诉王后{\PN{以斯帖}}。{\PN{以斯帖}}奉{\PN{末底改}}的名,报告于王;
\VS{23}究察这事,果然是实,就把二人挂在木头上,将这事在王面前写于历史上。

\par }\Chap{3}{\SH 哈曼阴谋除灭犹大人
\par }{\PP \VerseOne{1}这事以后,{\PN{亚哈随鲁}}王抬举{\PN{亚甲}}族{\PN{哈米大他}}的儿子{\PN{哈曼}},使他高升,叫他的爵位超过与他同事的一切臣宰。
\VS{2}在朝门的一切臣仆都跪拜{\PN{哈曼}},因为王如此吩咐;惟独{\PN{末底改}}不跪不拜。
\VS{3}在朝门的臣仆问{\PN{末底改}}说:「你为何违背王的命令呢?」
\VS{4}他们天天劝他,他还是不听,他们就告诉{\PN{哈曼}},要看{\PN{末底改}}的事站得住站不住,因他已经告诉他们自己是{\PN{犹大}}人。
\VS{5}{\PN{哈曼}}见{\PN{末底改}}不跪不拜,他就怒气填胸。
\VS{6}他们已将{\PN{末底改}}的本族告诉{\PN{哈曼}};他以为下手害{\PN{末底改}}一人是小事,就要灭绝{\PN{亚哈随鲁}}王通国所有的{\PN{犹大}}人,就是{\PN{末底改}}的本族。
\par }{\PP \VS{7}{\PN{亚哈随鲁}}王十二年正月,就是尼散月,人在{\PN{哈曼}}面前,按日日月月掣普珥,就是掣签,要定何月何日为吉,择定了十二月,就是亚达月。
\VS{8}{\PN{哈曼}}对{\PN{亚哈随鲁}}王说:「有一种民散居在王国各省的民中;他们的律例与万民的律例不同,也不守王的律例,所以容留他们与王无益。
\VS{9}王若以为美,请下旨意灭绝他们;我就捐一万他连得银子交给掌管国帑的人,纳入王的府库。」
\VS{10}于是王从自己手上摘下戒指给{\PN{犹大}}人的仇敌—{\PN{亚甲}}族{\PN{哈米大他}}的儿子{\PN{哈曼}}。
\VS{11}王对{\PN{哈曼}}说:「这银子仍赐给你,这民也交给你,你可以随意待他们。」
\par }{\PP \VS{12}正月十三日,就召了王的书记来,照着{\PN{哈曼}}一切所吩咐的,用各省的文字、各族的方言,奉{\PN{亚哈随鲁}}王的名写旨意,传与总督和各省的省长,并各族的首领;又用王的戒指盖印,
\VS{13}交给驿卒传到王的各省,吩咐将{\PN{犹大}}人,无论老少妇女孩子,在一日之间,十二月,就是亚达月十三日,全然剪除,杀戮灭绝,并夺他们的财为掠物。
\VS{14}抄录这旨意,颁行各省,宣告各族,使他们预备等候那日。
\VS{15}驿卒奉王命急忙起行,旨意也传遍{\PN{书珊}}城。王同{\PN{哈曼}}坐下饮{\ADD{酒}},{\PN{书珊}}城的民却都慌乱。

\par }\Chap{4}{\SH 末底改求以斯帖帮忙
\par }{\PP \VerseOne{1}{\PN{末底改}}知道所做的这一切事,就撕裂衣服,穿麻衣,蒙灰尘,在城中行走,痛哭哀号。
\VS{2}到了朝门前{\ADD{停住脚步}},因为穿麻衣的不可进朝门。
\VS{3}王的谕旨所到的各省各处,{\PN{犹大}}人大大悲哀,禁食哭泣哀号,穿麻衣躺在灰中的甚多。
\par }{\PP \VS{4}王后{\PN{以斯帖}}的宫女和太监来把这事告诉{\PN{以斯帖}},她甚是忧愁,就送衣服给{\PN{末底改}}穿,要他脱下麻衣,他却不受。
\VS{5}{\PN{以斯帖}}就把王所派伺候她的一个太监,名叫{\PN{哈他革}}召来,吩咐他去见{\PN{末底改}},要知道这是什么事,是什么缘故。
\VS{6}于是{\PN{哈他革}}出到朝门前的宽阔处见{\PN{末底改}}。
\VS{7}{\PN{末底改}}将自己所遇的事,并{\PN{哈曼}}为灭绝{\PN{犹大}}人应许捐入王库的银数都告诉了他;
\VS{8}又将所抄写传遍{\PN{书珊}}城要灭绝{\PN{犹大}}人的旨意交给{\PN{哈他革}},要给{\PN{以斯帖}}看,又要给她说明,并嘱咐她进去见王,为本族的人在王面前恳切祈求。
\VS{9}{\PN{哈他革}}回来,将{\PN{末底改}}的话告诉{\PN{以斯帖}};
\VS{10}{\PN{以斯帖}}就吩咐{\PN{哈他革}}去见{\PN{末底改}},{\ADD{说}}:
\VS{11}「王的一切臣仆和各省的人民都知道有一个定例:若不蒙召,擅入内院见王的,无论男女必被治死;除非王向他伸出金杖,不得存活。现在我没有蒙召进去见王已经三十日了。」
\VS{12}人就把{\PN{以斯帖}}这话告诉{\PN{末底改}}。
\VS{13}{\PN{末底改}}托人回复{\PN{以斯帖}}说:「你莫想在王宫里强过一切{\PN{犹大}}人,得免这祸。
\VS{14}此时你若闭口不言,{\PN{犹大}}人必从别处得解脱,蒙拯救;你和你父家必致灭亡。焉知你得了王后的位分不是为现今的机会吗?」
\VS{15}{\PN{以斯帖}}就吩咐人回报{\PN{末底改}}说:
\VS{16}「你当去招聚{\PN{书珊}}城所有的{\PN{犹大}}人,为我禁食三昼三夜,不吃不喝;我和我的宫女也要这样禁食。然后我违例进去见王,我若死就死吧!」
\VS{17}于是{\PN{末底改}}照{\PN{以斯帖}}一切所吩咐的去行。

\par }\Chap{5}{\SH 以斯帖为王和哈曼设筵席
\par }{\PP \VerseOne{1}第三日,{\PN{以斯帖}}穿上朝服,进王宫的内院,对殿站立。王在殿里坐在宝座上,对着殿门。
\VS{2}王见王后{\PN{以斯帖}}站在院内,就施恩于她,向她伸出手中的金杖;{\PN{以斯帖}}便向前摸杖头。
\VS{3}王对她说:「王后{\PN{以斯帖}}啊,你要什么?你求什么,就是国的一半也必赐给你。」
\VS{4}{\PN{以斯帖}}说:「王若以为美,就请王带着{\PN{哈曼}}今日赴我所预备的筵席。」
\VS{5}王说:「叫{\PN{哈曼}}速速照{\PN{以斯帖}}的话去行。」于是王带着{\PN{哈曼}}赴{\PN{以斯帖}}所预备的筵席。
\VS{6}在酒席筵前,王又问{\PN{以斯帖}}说:「你要什么,我必赐给你;你求什么,就是国的一半也必{\ADD{为你}}成就。」
\VS{7}{\PN{以斯帖}}回答说:「我有所要,我有所求。
\VS{8}我若在王眼前蒙恩,王若愿意赐我所要的,准我所求的,就请王带着{\PN{哈曼}}再赴我所要预备的筵席。明日我必照王所问的说明。」
\par }{\SH 哈曼阴谋杀末底改
\par }{\PP \VS{9}那日{\PN{哈曼}}心中快乐,欢欢喜喜地出来;但见{\PN{末底改}}在朝门不站起来,连身也不动,就满心恼怒{\PN{末底改}}。
\VS{10}{\PN{哈曼}}暂且忍耐回家,叫人请他朋友和他妻子{\PN{细利斯}}来。
\VS{11}{\PN{哈曼}}将他富厚的荣耀、众多的儿女,和王抬举他使他超乎首领臣仆之上,都述说给他们听。
\VS{12}{\PN{哈曼}}又说:「王后{\PN{以斯帖}}预备筵席,除了我之外不许别人随王赴席。明日王后又请我随王赴席;
\VS{13}只是我见{\PN{犹大}}人{\PN{末底改}}坐在朝门,虽有这一切{\ADD{荣耀}},也与我无益。」
\VS{14}他的妻{\PN{细利斯}}和他一切的朋友对他说:「不如立一个五丈高的木架,明早求王将{\PN{末底改}}挂在其上,然后你可以欢欢喜喜地随王赴席。」{\PN{哈曼}}以这话为美,就叫人做了木架。

\par }\Chap{6}{\SH 王赐末底改荣誉
\par }{\PP \VerseOne{1}那夜王睡不着觉,就吩咐人取历史来,念给他听。
\VS{2}正遇见书上写着{\ADD{说}}:王的太监中有两个守门的,{\PN{辟探}}和{\PN{提列}},想要下手害{\PN{亚哈随鲁}}王,{\PN{末底改}}将这事告诉{\ADD{王后}}。
\VS{3}王说:「{\PN{末底改}}行了这事,赐他什么尊荣爵位没有?」伺候王的臣仆回答说:「没有赐他什么。」
\VS{4}王说:「谁在院子里?」(那时{\PN{哈曼}}正进王宫的外院,要求王将{\PN{末底改}}挂在他所预备的木架上。)
\VS{5}臣仆说:「{\PN{哈曼}}站在院内。」王说:「叫他进来。」
\VS{6}{\PN{哈曼}}就进去。王问他说:「王所喜悦尊荣的人,当如何待他呢?」{\PN{哈曼}}心里说:「王所喜悦尊荣的,不是我是谁呢?」
\VS{7}{\PN{哈曼}}就回答说:「王所喜悦尊荣的,
\VS{8}当将王常穿的朝服和戴冠的御马,
\VS{9}都交给王极尊贵的一个大臣,命他将衣服给王所喜悦尊荣的人穿上,使他骑上马,走遍城里的街市,在他面前宣告说:王所喜悦尊荣的人,就如此待他。」
\VS{10}王对{\PN{哈曼}}说:「你速速将这衣服和马,照你所说的,向坐在朝门的{\PN{犹大}}人{\PN{末底改}}去行。凡你所说的,一样不可缺。」
\VS{11}于是{\PN{哈曼}}将朝服给{\PN{末底改}}穿上,使他骑上马,走遍城里的街市,在他面前宣告说:「王所喜悦尊荣的人,就如此待他。」
\par }{\PP \VS{12}{\PN{末底改}}仍回到朝门,{\PN{哈曼}}却忧忧闷闷地蒙着头,急忙回家去了,
\VS{13}将所遇的一切事详细说给他的妻{\PN{细利斯}}和他的众朋友听。他的智慧人和他的妻{\PN{细利斯}}对他说:「你在{\PN{末底改}}面前始而败落,他如果是{\PN{犹大}}人,你必不能胜他,终必在他面前败落。」
\par }{\PP \VS{14}他们还与{\PN{哈曼}}说话的时候,王的太监来催{\PN{哈曼}}快去赴{\PN{以斯帖}}所预备的筵席。

\par }\Chap{7}{\SH 哈曼被处死
\par }{\PP \VerseOne{1}王带着{\PN{哈曼}}来赴王后{\PN{以斯帖}}的筵席。
\VS{2}这第二次在酒席筵前,王又问{\PN{以斯帖}}说:「王后{\PN{以斯帖}}啊,你要什么,我必赐给你;你求什么,就是国的一半也必{\ADD{为你}}成就。」
\VS{3}王后{\PN{以斯帖}}回答说:「我若在王眼前蒙恩,王若以为美,我所愿的,是愿王将我的性命赐给我;我所求的,是求王将我的本族赐给我。
\VS{4}因我和我的本族被卖了,要剪除杀戮灭绝我们。我们若被卖为奴为婢,我也闭口不言;但王的损失,敌人万不能补足。」
\VS{5}{\PN{亚哈随鲁}}王问王后{\PN{以斯帖}}说:「擅敢起意如此行的是谁?这人在哪里呢?」
\VS{6}{\PN{以斯帖}}说:「仇人敌人就是这恶人{\PN{哈曼}}!」
\par }{\PP {\PN{哈曼}}在王和王后面前就甚惊惶。
\VS{7}王便大怒,起来离开酒席往御园去了。{\PN{哈曼}}见王定意要加罪与他,就起来,求王后{\PN{以斯帖}}救命。
\VS{8}王从御园回到酒席之处,见{\PN{哈曼}}伏在{\PN{以斯帖}}所靠的榻上;王说:「他竟敢在宫内、在我面前凌辱王后吗?」这话一出王口,人就蒙了{\PN{哈曼}}的脸。
\VS{9}伺候王的一个太监名叫{\PN{哈波拿}},说:「{\PN{哈曼}}为那救王有功的{\PN{末底改}}做了五丈高的木架,现今立在{\PN{哈曼}}家里。」王说:「把{\PN{哈曼}}挂在其上。」
\VS{10}于是人将{\PN{哈曼}}挂在他为{\PN{末底改}}所预备的木架上。王的忿怒这才止息。

\par }\Chap{8}{\SH 犹大人的反击
\par }{\PP \VerseOne{1}当日,{\PN{亚哈随鲁}}王把{\PN{犹大}}人仇敌{\PN{哈曼}}的家产赐给王后{\PN{以斯帖}}。{\PN{末底改}}也来到王面前,因为{\PN{以斯帖}}已经告诉王,{\PN{末底改}}是她的亲属。
\VS{2}王摘下自己的戒指,就是从{\PN{哈曼}}追回的,给了{\PN{末底改}}。{\PN{以斯帖}}派{\PN{末底改}}管理{\PN{哈曼}}的家产。
\par }{\PP \VS{3}{\PN{以斯帖}}又俯伏在王脚前,流泪哀告,求他除掉{\PN{亚甲}}族{\PN{哈曼}}害{\PN{犹大}}人的恶谋。
\VS{4}王向{\PN{以斯帖}}伸出金杖;{\PN{以斯帖}}就起来,站在王前,
\VS{5}说:「{\PN{亚甲}}族{\PN{哈米大他}}的儿子{\PN{哈曼}}设谋传旨,要杀灭在王各省的{\PN{犹大}}人。现今王若愿意,我若在王眼前蒙恩,王若以为美,若喜悦我,请王另下旨意,废除{\PN{哈曼}}所传的那旨意。
\VS{6}我何忍见我本族的人受害?何忍见我同宗的人被灭呢?」
\VS{7}{\PN{亚哈随鲁}}王对王后{\PN{以斯帖}}和{\PN{犹大}}人{\PN{末底改}}说:「因{\PN{哈曼}}要下手害{\PN{犹大}}人,我已将他的家产赐给{\PN{以斯帖}},人也将{\PN{哈曼}}挂在木架上。
\VS{8}现在你们可以随意奉王的名写谕旨给{\PN{犹大}}人,用王的戒指盖印;因为奉王名所写、用王戒指盖印的谕旨,人都不能废除。」
\par }{\PP \VS{9}三月,就是西弯月二十三日,将王的书记召来,按着{\PN{末底改}}所吩咐的,用各省的文字、各族的方言,并{\PN{犹大}}人的文字方言写谕旨,传给那从{\PN{印度}}直到{\PN{古实}}一百二十七省的{\PN{犹大}}人和总督省长首领。
\VS{10}{\PN{末底改}}奉{\PN{亚哈随鲁}}王的名写谕旨,用王的戒指盖印,交给骑御马圈快马的驿卒,传{\ADD{到各处}}。
\VS{11-12}谕旨中,王准各省各城的{\PN{犹大}}人在一日之间,十二月,就是亚达月十三日,聚集保护性命,剪除杀戮灭绝那要攻击{\PN{犹大}}人的一切仇敌和他们的妻子儿女,夺取他们的财为掠物。
\VS{13}抄录这谕旨,颁行各省,宣告各族,使{\PN{犹大}}人预备等候那日,在仇敌身上报仇。
\VS{14}于是骑快马的驿卒被王命催促,急忙起行;谕旨也传遍{\PN{书珊}}城。
\par }{\PP \VS{15}{\PN{末底改}}穿着蓝色白色的朝服,头戴大金冠冕,又穿紫色细麻布的外袍,从王面前出来;{\PN{书珊}}城的人民都欢呼快乐。
\VS{16}{\PN{犹大}}人有光荣,欢喜快乐而得尊贵。
\VS{17}王的谕旨所到的各省各城,{\PN{犹大}}人都欢喜快乐,设摆筵宴,以那日为吉日。那国的人民,有许多因惧怕{\PN{犹大}}人,就入了{\PN{犹大}}籍。

\par }\Chap{9}{\SH 犹大人除灭他们的仇敌
\par }{\PP \VerseOne{1}十二月,乃亚达月十三日,王的谕旨将要举行,就是{\PN{犹大}}人的仇敌盼望辖制他们的日子,{\PN{犹大}}人反倒辖制恨他们的人。
\VS{2}{\PN{犹大}}人在{\PN{亚哈随鲁}}王各省的城里聚集,下手击杀那要害他们的人。无人能敌挡他们,因为各族都惧怕他们。
\VS{3}各省的首领、总督、省长,和办理王事的人,因惧怕{\PN{末底改}},就都帮助{\PN{犹大}}人。
\VS{4}{\PN{末底改}}在朝中为大,名声传遍各省,日渐昌盛。
\VS{5}{\PN{犹大}}人用刀击杀一切仇敌,任意杀灭恨他们的人。
\VS{6}在{\PN{书珊}}城,{\PN{犹大}}人杀灭了五百人;
\VS{7}又杀{\PN{巴珊大他}}、{\PN{达分}}、{\PN{亚斯帕他}}、
\VS{8}{\PN{破拉他}}、{\PN{亚大利雅}}、{\PN{亚利大他}}、
\VS{9}{\PN{帕玛斯他}}、{\PN{亚利赛}}、{\PN{亚利代}}、{\PN{瓦耶撒他}};
\VS{10}这十人都是{\PN{哈米大他}}的孙子、{\PN{犹大}}人仇敌{\PN{哈曼}}的儿子。{\PN{犹大}}人却没有下手夺取财物。
\par }{\PP \VS{11}当日,将{\PN{书珊}}城被杀的人数呈在王前。
\VS{12}王对王后{\PN{以斯帖}}说:「{\PN{犹大}}人在{\PN{书珊}}城杀灭了五百人,又杀了{\PN{哈曼}}的十个儿子,在王的各省不知如何呢?现在你要什么,我必赐给你;你还求什么,也必{\ADD{为你}}成就。」
\VS{13}{\PN{以斯帖}}说:「王若以为美,求你准{\PN{书珊}}的{\PN{犹大}}人,明日也照今日的旨意行,并将{\PN{哈曼}}十个儿子的尸首挂在木架上。」
\VS{14}王便允准如此行。旨意传在{\PN{书珊}},人就把{\PN{哈曼}}十个儿子的尸首挂起来了。
\VS{15}亚达月十四日,{\PN{书珊}}的{\PN{犹大}}人又聚集在{\PN{书珊}},杀了三百人,却没有下手夺取财物。
\par }{\PP \VS{16}在王各省其余的{\PN{犹大}}人也都聚集保护性命,杀了恨他们的人七万五千,却没有下手夺取财物。这样,就脱离仇敌,得享平安。
\VS{17}亚达月十三日,{\ADD{行了这事}};十四日安息,以这日为设筵欢乐的日子。
\VS{18}但{\PN{书珊}}的{\PN{犹大}}人,这十三日、十四日聚集{\ADD{杀戮仇敌}};十五日安息,以这日为设筵欢乐的日子。
\VS{19}所以住无城墙乡村的{\PN{犹大}}人,如今都以亚达月十四日为设筵欢乐的吉日,彼此馈送礼物。
\par }{\SH 普珥日
\par }{\PP \VS{20}{\PN{末底改}}记录这事,写信与{\PN{亚哈随鲁}}王各省远近所有的{\PN{犹大}}人,
\VS{21}嘱咐他们每年守亚达月十四、十五两日,
\VS{22}以这月的两日为{\PN{犹大}}人脱离仇敌得平安、转忧为喜、转悲为乐的吉日。在这两日设筵欢乐,彼此馈送礼物,周济穷人。
\par }{\PP \VS{23}于是,{\PN{犹大}}人按着{\PN{末底改}}所写与他们的信,应承照初次所守的守为永例;
\VS{24}是因{\PN{犹大}}人的仇敌{\PN{亚甲}}族{\PN{哈米大他}}的儿子{\PN{哈曼}}设谋杀害{\PN{犹大}}人,掣普珥,就是掣签,为要杀尽灭绝他们;
\VS{25}这事报告于王,王便降旨使{\PN{哈曼}}谋害{\PN{犹大}}人的恶事归到他自己的头上,并吩咐把他和他的众子都挂在木架上。
\VS{26}照着普珥的名字,{\PN{犹大}}人就称这两日为「普珥日」。他们因这信上的话,又因所看见所遇见的事,
\VS{27}就应承自己与后裔,并归附他们的人,每年按时必守这两日,永远不废。
\VS{28}各省各城、家家户户、世世代代纪念遵守这两日,使这「普珥日」在{\PN{犹大}}人中不可废掉,在他们后裔中也不可忘记。
\par }{\PP \VS{29}{\PN{亚比孩}}的女儿—王后{\PN{以斯帖}}和{\PN{犹大}}人{\PN{末底改}}以全权写第二封信,坚嘱{\PN{犹大}}人守这「普珥日」,
\VS{30}用和平诚实话写信给{\PN{亚哈随鲁}}{\ADD{王}}国中一百二十七省所有的{\PN{犹大}}人,
\VS{31}劝他们按时守这「普珥日」,禁食呼求,是照{\PN{犹大}}人{\PN{末底改}}和王后{\PN{以斯帖}}所嘱咐的,也照{\PN{犹大}}人为自己与后裔所应承的。
\VS{32}{\PN{以斯帖}}命定守「普珥日」,这事也记录在书上。

\par }\Chap{10}{\SH 亚哈随鲁和末底改的功绩
\par }{\PP \VerseOne{1}{\PN{亚哈随鲁}}王使旱地和海岛的人民都进贡。
\VS{2}他以权柄能力所行的,并他抬举{\PN{末底改}}使他高升的事,岂不都写在{\PN{米底亚}}和{\PN{波斯}}王的历史上吗?
\VS{3}{\PN{犹大}}人{\PN{末底改}}作{\PN{亚哈随鲁}}王的宰相,在{\PN{犹大}}人中为大,得他众弟兄的喜悦,为本族的人求好处,向他们说和平的话。
\par }