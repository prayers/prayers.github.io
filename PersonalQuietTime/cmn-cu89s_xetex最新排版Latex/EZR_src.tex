\NormalFont\ShortTitle{以斯拉记}
{\MT 以斯拉记

\par }\ChapOne{1}{\SH 塞鲁士下令犹太人返国
\par }{\PP \VerseOne{1}{\PN{波斯}}王{\PN{塞鲁士}}元年,耶和华为要应验借{\PN{耶利米}}口所说的话,就激动{\PN{波斯}}王{\PN{塞鲁士}}的心,使他下诏通告全国说:
\par }{\PP \VS{2}「{\PN{波斯}}王{\PN{塞鲁士}}如此说:『耶和华天上的 神已将天下万国赐给我,又嘱咐我在{\PN{犹大}}的{\PN{耶路撒冷}}为他建造殿宇。
\VS{3}在你们中间凡作他子民的,可以上{\PN{犹大}}的{\PN{耶路撒冷}},在{\PN{耶路撒冷}}重建耶和华—{\PN{以色列}} 神的殿(只有他是 神)。愿 神与这人同在。
\VS{4}凡剩下的人,无论寄居何处,那地的人要用金银、财物、牲畜帮助他,另外也要为{\PN{耶路撒冷}} 神的殿甘心献上礼物。』」
\par }{\PP \VS{5}于是,{\PN{犹大}}和{\PN{便雅悯}}的族长、祭司、{\PN{利未}}人,就是一切被 神激动他心的人,都起来要上{\PN{耶路撒冷}}去建造耶和华的殿。
\VS{6}他们四围的人就拿银器、金子、财物、牲畜、珍宝帮助他们\FTNT{}{{\FR 1:6: }原文是坚固他们的手},另外还有甘心献的礼物。
\VS{7}{\PN{塞鲁士}}王也将耶和华殿的器皿拿出来,这器皿是{\PN{尼布甲尼撒}}从{\PN{耶路撒冷}}掠来、放在自己神之庙中的。
\VS{8}{\PN{波斯}}王{\PN{塞鲁士}}派库官{\PN{米提利达}}将这器皿拿出来,按数交给{\PN{犹大}}的首领{\PN{设巴萨}}。
\VS{9}器皿的数目记在下面:金盘三十个,银盘一千个,刀二十九把,
\VS{10}金碗三十个,银碗之次的四百一十个,别样的器皿一千件。
\VS{11}金银器皿共有五千四百件。被掳的人从{\PN{巴比伦}}上{\PN{耶路撒冷}}的时候,{\PN{设巴萨}}将这一切都带上来。

\par }\Chap{2}{\SH 被掳归回者的名单
\par }{\R (尼7·4—73)
\par }{\PP \VerseOne{1}{\PN{巴比伦}}王{\PN{尼布甲尼撒}}从前掳到{\PN{巴比伦}}之{\PN{犹大}}省的人,现在他们的子孙从被掳到之地回{\PN{耶路撒冷}}和{\PN{犹大}},各归本城。
\VS{2}他们是同着{\PN{所罗巴伯}}、{\PN{耶书亚}}、{\PN{尼希米}}、{\PN{西莱雅}}、{\PN{利来雅}}、{\PN{末底改}}、{\PN{必珊}}、{\PN{米斯拔}}、{\PN{比革瓦伊}}、{\PN{利宏}}、{\PN{巴拿}}回来的。
\par }{\PP \VS{3}{\PN{以色列}}人民的数目记在下面:{\PN{巴录}}的子孙二千一百七十二名;
\VS{4}{\PN{示法提雅}}的子孙三百七十二名;
\VS{5}{\PN{亚拉}}的子孙七百七十五名;
\VS{6}{\PN{巴哈·摩押}}的后裔,就是{\PN{耶书亚}}和{\PN{约押}}的子孙二千八百一十二名;
\VS{7}{\PN{以拦}}的子孙一千二百五十四名;
\VS{8}{\PN{萨土}}的子孙九百四十五名;
\VS{9}{\PN{萨改}}的子孙七百六十名;
\VS{10}{\PN{巴尼}}的子孙六百四十二名;
\VS{11}{\PN{比拜}}的子孙六百二十三名;
\VS{12}{\PN{押甲}}的子孙一千二百二十二名;
\VS{13}{\PN{亚多尼干}}的子孙六百六十六名;
\VS{14}{\PN{比革瓦伊}}的子孙二千零五十六名;
\VS{15}{\PN{亚丁}}的子孙四百五十四名;
\VS{16}{\PN{亚特}}的后裔,就是{\PN{希西家}}的子孙九十八名;
\VS{17}{\PN{比赛}}的子孙三百二十三名;
\VS{18}{\PN{约拉}}的子孙一百一十二名;
\VS{19}{\PN{哈顺}}的子孙二百二十三名;
\VS{20}{\PN{吉罢珥}}人九十五名;
\VS{21}{\PN{伯利恒}}人一百二十三名;
\VS{22}{\PN{尼陀法}}人五十六名;
\VS{23}{\PN{亚拿突}}人一百二十八名;
\VS{24}{\PN{亚斯玛弗}}人四十二名;
\VS{25}{\PN{基列·耶琳}}人、{\PN{基非拉}}人、{\PN{比录}}人共七百四十三名;
\VS{26}{\PN{拉玛}}人、{\PN{迦巴}}人共六百二十一名;
\VS{27}{\PN{默玛}}人一百二十二名;
\VS{28}{\PN{伯特利}}人、{\PN{艾}}人共二百二十三名;
\VS{29}{\PN{尼波}}人五十二名;
\VS{30}{\PN{末必}}人一百五十六名;
\VS{31}别的{\PN{以拦}}子孙一千二百五十四名;
\VS{32}{\PN{哈琳}}的子孙三百二十名;
\VS{33}{\PN{罗德}}人、{\PN{哈第}}人、{\PN{阿挪}}人共七百二十五名;
\VS{34}{\PN{耶利哥}}人三百四十五名;
\VS{35}{\PN{西拿}}人三千六百三十名。
\par }{\PP \VS{36}祭司:{\PN{耶书亚}}家{\PN{耶大雅}}的子孙九百七十三名;
\VS{37}{\PN{音麦}}的子孙一千零五十二名;
\VS{38}{\PN{巴施户珥}}的子孙一千二百四十七名;
\VS{39}{\PN{哈琳}}的子孙一千零一十七名。
\par }{\PP \VS{40}{\PN{利未}}人:{\PN{何达威雅}}的后裔,就是{\PN{耶书亚}}和{\PN{甲篾}}的子孙七十四名。
\VS{41}歌唱的:{\PN{亚萨}}的子孙一百二十八名。
\VS{42}守门的:{\PN{沙龙}}的子孙、{\PN{亚特}}的子孙、{\PN{达们}}的子孙、{\PN{亚谷}}的子孙、{\PN{哈底大}}的子孙、{\PN{朔拜}}的子孙,共一百三十九名。
\par }{\PP \VS{43}尼提宁\FTNT{}{{\FR 2:43: }就是殿役}:{\PN{西哈}}的子孙、{\PN{哈苏巴}}的子孙、{\PN{答巴俄}}的子孙、
\VS{44}{\PN{基绿}}的子孙、{\PN{西亚}}的子孙、{\PN{巴顿}}的子孙、
\VS{45}{\PN{利巴拿}}的子孙、{\PN{哈迦巴}}的子孙、{\PN{亚谷}}的子孙、
\VS{46}{\PN{哈甲}}的子孙、{\PN{萨买}}的子孙、{\PN{哈难}}的子孙、
\VS{47}{\PN{吉德}}的子孙、{\PN{迦哈}}的子孙、{\PN{利亚雅}}的子孙、
\VS{48}{\PN{利汛}}的子孙、{\PN{尼哥大}}的子孙、{\PN{迦散}}的子孙、
\VS{49}{\PN{乌撒}}的子孙、{\PN{巴西亚}}的子孙、{\PN{比赛}}的子孙、
\VS{50}{\PN{押拿}}的子孙、{\PN{米乌宁}}的子孙、{\PN{尼普心}}的子孙、
\VS{51}{\PN{巴卜}}的子孙、{\PN{哈古巴}}的子孙、{\PN{哈忽}}的子孙、
\VS{52}{\PN{巴洗律}}的子孙、{\PN{米希大}}的子孙、{\PN{哈沙}}的子孙、
\VS{53}{\PN{巴柯}}的子孙、{\PN{西西拉}}的子孙、{\PN{答玛}}的子孙、
\VS{54}{\PN{尼细亚}}的子孙、{\PN{哈提法}}的子孙。
\par }{\PP \VS{55}{\PN{所罗门}}仆人的后裔,就是{\PN{琐太}}的子孙、{\PN{琐斐列}}的子孙、{\PN{比路大}}的子孙、
\VS{56}{\PN{雅拉}}的子孙、{\PN{达昆}}的子孙、{\PN{吉德}}的子孙、
\VS{57}{\PN{示法提雅}}的子孙、{\PN{哈替}}的子孙、{\PN{玻黑列·哈斯巴音}}的子孙、{\PN{亚米}}的子孙。
\par }{\PP \VS{58}尼提宁和{\PN{所罗门}}仆人的后裔共三百九十二名。
\par }{\PP \VS{59}从{\PN{特·米拉}}、{\PN{特·哈萨}}、{\PN{基绿}}、{\PN{押但}}、{\PN{音麦}}上来的,不能指明他们的宗族谱系是{\PN{以色列}}人不是;
\VS{60}他们是{\PN{第来雅}}的子孙、{\PN{多比雅}}的子孙、{\PN{尼哥大}}的子孙,共六百五十二名。
\VS{61}祭司中,{\PN{哈巴雅}}的子孙、{\PN{哈哥斯}}的子孙、{\PN{巴西莱}}的子孙;因为他们的先祖娶了{\PN{基列}}人{\PN{巴西莱}}的女儿为妻,所以起名叫{\PN{巴西莱}}。
\VS{62}这三家的人在族谱之中寻查自己的谱系,却寻不着,因此算为不洁,不准供祭司的职任。
\VS{63}省长对他们说:「不可吃至圣的物,直到有用乌陵和土明决疑的祭司兴起来。」
\par }{\PP \VS{64}会众共有四万二千三百六十名。
\VS{65}此外,还有他们的仆婢七千三百三十七名,又有歌唱的男女二百名。
\VS{66}他们有马七百三十六匹,骡子二百四十五匹,
\VS{67}骆驼四百三十五只,驴六千七百二十匹。
\par }{\PP \VS{68}有些族长到了{\PN{耶路撒冷}}耶和华殿的地方,便为 神的殿甘心献上礼物,要重新建造。
\VS{69}他们量力捐入工程库的金子六万一千达利克,银子五千弥拿,并祭司的礼服一百件。
\par }{\PP \VS{70}于是祭司、{\PN{利未}}人、民中的一些人、歌唱的、守门的、尼提宁,并{\PN{以色列}}众人,各住在自己的城里。

\par }\Chap{3}{\SH 恢复敬拜 神的生活
\par }{\PP \VerseOne{1}到了七月,{\PN{以色列}}人住在各城;那时他们如同一人,聚集在{\PN{耶路撒冷}}。
\VS{2}{\PN{约萨达}}的儿子{\PN{耶书亚}}和他的弟兄众祭司,并{\PN{撒拉铁}}的儿子{\PN{所罗巴伯}}与他的弟兄,都起来建筑{\PN{以色列}} 神的坛,要照神人{\PN{摩西}}律法书上所写的,在坛上献燔祭。
\VS{3}他们在原有的根基上筑坛,因惧怕邻国的民,又在其上向耶和华早晚献燔祭,
\VS{4}又照律法书上所写的守住棚节,按数照例{\ADD{献}}每日所当献的燔祭;
\VS{5}其后献常献的燔祭,并在月朔与耶和华的一切圣节献祭,又向耶和华献各人的甘心祭。
\VS{6}从七月初一日起,他们就向耶和华献燔祭。但耶和华殿的根基尚未立定。
\VS{7}他们又将银子给石匠、木匠,把粮食、酒、油给{\PN{西顿}}人、{\PN{泰尔}}人,使他们将香柏树从{\PN{黎巴嫩}}运到海里,浮海运到{\PN{约帕}},是照{\PN{波斯}}王{\PN{塞鲁士}}所允准的。
\par }{\SH 开始重建圣殿
\par }{\PP \VS{8}百姓到了{\PN{耶路撒冷}} 神殿的地方。第二年二月,{\PN{撒拉铁}}的儿子{\PN{所罗巴伯}},{\PN{约萨达}}的儿子{\PN{耶书亚}}和其余的弟兄,就是祭司、{\PN{利未}}人,并一切被掳归回{\PN{耶路撒冷}}的人,都兴工建造;又派{\PN{利未}}人,从二十岁以外的,督理建造耶和华殿的工作。
\VS{9}于是{\PN{犹大}}\FTNT{}{{\FR 3:9: }在二章四十节作何达威雅}的后裔,就是{\PN{耶书亚}}和他的子孙与弟兄,{\PN{甲篾}}和他的子孙,{\PN{利未}}人{\PN{希拿达}}的子孙与弟兄,都一同起来,督理那在 神殿做工的人。
\par }{\PP \VS{10}匠人立耶和华殿根基的时候,祭司皆穿礼服吹号,{\PN{亚萨}}的子孙{\PN{利未}}人敲钹,照{\PN{以色列}}王{\PN{大卫}}所定的例,都站着赞美耶和华。
\VS{11}他们彼此唱和,赞美称谢耶和华{\ADD{说}}:
\par }{\Q 他本为善,
\par }{\Q 他向{\PN{以色列}}人永发慈爱。
\par }{\PP 他们赞美耶和华的时候,众民大声呼喊,因耶和华殿的根基已经立定。
\VS{12}然而有许多祭司、{\PN{利未}}人、族长,就是见过旧殿的老年人,现在亲眼看见立这殿的根基,便大声哭号,也有许多人大声欢呼,
\VS{13}甚至百姓不能分辨欢呼的声音和哭号的声音;因为众人大声呼喊,声音听到远处。

\par }\Chap{4}{\SH 建殿之工受阻
\par }{\PP \VerseOne{1}{\PN{犹大}}和{\PN{便雅悯}}的敌人听说被掳归回的人为耶和华—{\PN{以色列}}的 神建造殿宇,
\VS{2}就去见{\PN{所罗巴伯}}和{\PN{以色列}}的族长,对他们说:「请容我们与你们一同建造;因为我们寻求你们的 神,与你们一样。自从{\PN{亚述}}王{\PN{以撒哈顿}}带我们上这地以来,我们常祭祀 神。」
\VS{3}但{\PN{所罗巴伯}}、{\PN{耶书亚}},和其余{\PN{以色列}}的族长对他们说:「我们建造 神的殿与你们无干,我们自己为耶和华—{\PN{以色列}}的 神协力建造,是照{\PN{波斯}}王{\PN{塞鲁士}}所吩咐的。」
\par }{\PP \VS{4}那地的民,就在{\PN{犹大}}人建造的时候,使他们的手发软,扰乱他们;
\VS{5}从{\PN{波斯}}王{\PN{塞鲁士}}年间,直到{\PN{波斯}}王{\PN{大流士}}登基的时候,贿买谋士,要败坏他们的谋算。
\par }{\SH 阻扰重建耶路撒冷
\par }{\PP \VS{6}在{\PN{亚哈随鲁}}才登基的时候,上本控告{\PN{犹大}}和{\PN{耶路撒冷}}的居民。
\par }{\PP \VS{7}{\PN{亚达薛西}}年间,{\PN{比施兰}}、{\PN{米特利达}}、{\PN{他别}},和他们的同党上本奏告{\PN{波斯}}王{\PN{亚达薛西}}。本章是用{\PN{亚兰}}{\ADD{文字}},{\PN{亚兰}}{\ADD{方言}}。
\VS{8}省长{\PN{利宏}}、书记{\PN{伸帅}}要控告{\PN{耶路撒冷}}人,也上本奏告{\PN{亚达薛西}}王。
\VS{9}省长{\PN{利宏}}、书记{\PN{伸帅}},和同党的{\PN{底拿}}人、{\PN{亚法萨提迦}}人、{\PN{他毗拉}}人、{\PN{亚法撒}}人、{\PN{亚基卫}}人、{\PN{巴比伦}}人、{\PN{书珊迦}}人、{\PN{底亥}}人、{\PN{以拦}}人,
\VS{10}和尊大的{\PN{亚斯那巴}}所迁移、安置在{\PN{撒马利亚}}城,并大{\PN{河西}}一带地方的人等,
\VS{11}上奏{\PN{亚达薛西}}王说:「{\PN{河西}}的臣民云云:
\VS{12}王该知道,从王那里上到我们这里的{\PN{犹大}}人,已经到{\PN{耶路撒冷}}重建这反叛恶劣的城,筑立根基,建造城墙。
\VS{13}如今王该知道,他们若建造这城,城墙完毕就不再与王进贡,交课,纳税,终久王必受亏损。
\VS{14}我们既食御盐,不忍见王吃亏,因此奏告于王。
\VS{15}请王考察先王的实录,必在其上查知这城是反叛的城,与列王和各省有害;自古以来,其中常有悖逆的事,因此这城曾被拆毁。
\VS{16}我们谨奏王知,这城若再建造,城墙完毕,{\PN{河西}}之地王就无分了。」
\par }{\PP \VS{17}那时王谕复省长{\PN{利宏}}、书记{\PN{伸帅}},和他们的同党,就是住{\PN{撒马利亚}}并{\PN{河西}}一带地方的人,说:「愿你们平安云云。
\VS{18}你们所上的本,已经明读在我面前。
\VS{19}我已命人考查,得知此城古来果然背叛列王,其中常有反叛悖逆的事。
\VS{20}从前{\PN{耶路撒冷}}也有大君王统管{\PN{河西}}全{\ADD{地}},人就给他们进贡,交课,纳税。
\VS{21}现在你们要出告示命这些人停工,使这城不得建造,等我降旨。
\VS{22}你们当谨慎,不可迟延,为何容害加重,使王受亏损呢?」
\par }{\PP \VS{23}{\PN{亚达薛西}}王的上谕读在{\PN{利宏}}和书记{\PN{伸帅}},并他们的同党面前,他们就急忙往{\PN{耶路撒冷}}去见{\PN{犹大}}人,用势力强迫他们停工。
\par }{\SH 恢复重建圣殿工作
\par }{\PP \VS{24}于是,在{\PN{耶路撒冷}} 神殿的工程就停止了,直停到{\PN{波斯}}王{\PN{大流士}}第二年。

\par }\Chap{5}{\PP \VerseOne{1}那时,先知{\PN{哈该}}和{\PN{易多}}的孙子{\PN{撒迦利亚}}奉{\PN{以色列}} 神的名向{\PN{犹大}}和{\PN{耶路撒冷}}的{\PN{犹大}}人说劝勉的话。
\VS{2}于是{\PN{撒拉铁}}的儿子{\PN{所罗巴伯}}和{\PN{约萨达}}的儿子{\PN{耶书亚}}都起来动手建造{\PN{耶路撒冷}} 神的殿,有 神的先知在那里帮助他们。
\par }{\PP \VS{3}当时{\PN{河西}}的总督{\PN{达乃}}和{\PN{示他·波斯乃}},并他们的同党来问说:「谁降旨让你们建造这殿,修成这墙呢?」
\VS{4}我们便告诉他们建造这殿的人叫什么名字。
\VS{5}神的眼目看顾{\PN{犹大}}的长老,以致总督等没有叫他们停工,直到这事奏告{\PN{大流士}},得着他的回谕。
\par }{\PP \VS{6}{\PN{河西}}的总督{\PN{达乃}}和{\PN{示他·波斯乃}},并他们的同党,就是住{\PN{河西}}的{\PN{亚法萨迦}}人,上本奏告{\PN{大流士}}王。
\VS{7}本上写着说:「愿{\PN{大流士}}王诸事平安。
\VS{8}王该知道,我们往{\PN{犹大}}省去,到了至大 神的殿,这殿是用大石建造的。梁木插入墙内,工作甚速,他们手下亨通。
\VS{9}我们就问那些长老说:『谁降旨让你们建造这殿,修成这墙呢?』
\VS{10}又问他们的名字,要记录他们首领的名字,奏告于王。
\VS{11}他们回答说:『我们是天地之 神的仆人,重建前多年所建造的殿,就是{\PN{以色列}}的一位大君王建造修成的。
\VS{12}只因我们列祖惹天上的 神发怒, 神把他们交在{\PN{迦勒底}}人{\PN{巴比伦}}王{\PN{尼布甲尼撒}}的手中,他就拆毁这殿,又将百姓掳到{\PN{巴比伦}}。
\VS{13}然而{\PN{巴比伦}}王{\PN{塞鲁士}}元年,他降旨允准建造 神的这殿。
\VS{14}神殿中的金、银器皿,就是{\PN{尼布甲尼撒}}从{\PN{耶路撒冷}}的殿中掠去带到{\PN{巴比伦}}庙里的,{\PN{塞鲁士}}王从{\PN{巴比伦}}庙里取出来,交给派为省长的,名叫{\PN{设巴萨}},
\VS{15}对他说可以将这些器皿带去,放在{\PN{耶路撒冷}}的殿中,在原处建造 神的殿。
\VS{16}于是这{\PN{设巴萨}}来建立{\PN{耶路撒冷}} 神殿的根基。这殿从那时直到如今尚未造成。』
\VS{17}现在王若以为美,请察{\PN{巴比伦}}王的府库,看{\PN{塞鲁士}}王降旨允准在{\PN{耶路撒冷}}建造 神的殿没有,王的心意如何?请降旨晓谕我们。」

\par }\Chap{6}{\SH 发现塞鲁士王原有的诏令
\par }{\PP \VerseOne{1}于是{\PN{大流士}}王降旨,要寻察典籍库内,就是在{\PN{巴比伦}}藏宝物之处;
\VS{2}在{\PN{米底亚}}省{\PN{亚马他}}城的宫内寻得一卷,其中记着说:
\VS{3}「{\PN{塞鲁士}}王元年,他降旨论到{\PN{耶路撒冷}} 神的殿,要建造这殿为献祭之处,坚立殿的根基。殿高六十肘,宽六十肘,
\VS{4}用三层大石头,一层新木头,经费要出于王库;
\VS{5}并且 神殿的金银器皿,就是{\PN{尼布甲尼撒}}从{\PN{耶路撒冷}}的殿中掠到{\PN{巴比伦}}的,要归还带到{\PN{耶路撒冷}}的殿中,各按原处放在 神的殿里。」
\par }{\SH 大流士王下令继续建殿
\par }{\PP \VS{6}「现在{\PN{河西}}的总督{\PN{达乃}}和{\PN{示他·波斯乃}},并你们的同党,就是住{\PN{河西}}的{\PN{亚法萨迦}}人,你们当远离他们。
\VS{7}不要拦阻 神殿的工作,任凭{\PN{犹大}}人的省长和{\PN{犹大}}人的长老在原处建造 神的这殿。
\VS{8}我又降旨,吩咐你们向{\PN{犹大}}人的长老为建造 神的殿当怎样行,就是从{\PN{河西}}的款项中,急速拨取贡银作他们的经费,免得耽误工作。
\VS{9}他们与天上的 神献燔祭所需用的公牛犊、公绵羊、绵羊羔,并所用的麦子、盐、酒、油,都要照{\PN{耶路撒冷}}祭司的话,每日供给他们,不得有误;
\VS{10}好叫他们献馨香的祭给天上的 神,又为王和王众子的寿命祈祷。
\VS{11}我再降旨,无论谁更改这命令,必从他房屋中拆出一根梁来,把他举起,悬在其上,又使他的房屋成为粪堆。
\VS{12}若有王和民伸手更改这{\ADD{命令}},拆毁这殿,愿那使{\PN{耶路撒冷}}的殿作为他名居所的 神将他们灭绝。我—{\PN{大流士}}降这旨意,当速速遵行。」
\par }{\SH 圣殿奉献典礼
\par }{\PP \VS{13}于是,{\PN{河西}}总督{\PN{达乃}}和{\PN{示他·波斯乃}},并他们的同党,因{\PN{大流士}}王所发的命令,就急速遵行。
\VS{14}{\PN{犹大}}长老因先知{\PN{哈该}}和{\PN{易多}}的孙子{\PN{撒迦利亚}}所说劝勉的话就建造这殿,凡事亨通。他们遵着{\PN{以色列}} 神的命令和{\PN{波斯}}王{\PN{塞鲁士}}、{\PN{大流士}}、{\PN{亚达薛西}}的旨意,建造完毕。
\VS{15}{\PN{大流士}}王第六年,亚达月初三日,这殿修成了。
\par }{\PP \VS{16}{\PN{以色列}}的祭司和{\PN{利未}}人,并其余被掳归回的人都欢欢喜喜地行奉献 神殿的礼。
\VS{17}行奉献 神殿的礼就献公牛一百只,公绵羊二百只,绵羊羔四百只,又照{\PN{以色列}}支派的数目献公山羊十二只,为{\PN{以色列}}众人作赎罪祭;
\VS{18}且派祭司和{\PN{利未}}人按着班次在{\PN{耶路撒冷}}事奉 神,是照{\PN{摩西}}{\ADD{律法}}书上所写的。
\par }{\SH 守逾越节
\par }{\PP \VS{19}正月十四日,被掳归回的人守逾越节。
\VS{20}原来,祭司和{\PN{利未}}人一同自洁,无一人不洁净。{\PN{利未}}人为被掳归回的众人和他们的弟兄众祭司,并为自己宰逾越节{\ADD{的羊羔}}。
\VS{21}从掳到之地归回的{\PN{以色列}}人和一切除掉所染外邦人污秽、归附他们、要寻求耶和华—{\PN{以色列}} 神的人都吃{\ADD{这羊羔}},
\VS{22}欢欢喜喜地守除酵节七日;因为耶和华使他们欢喜,又使{\PN{亚述}}王的心转向他们,坚固他们的手,作{\PN{以色列}} 神殿的工程。

\par }\Chap{7}{\SH 以斯拉到达耶路撒冷
\par }{\PP \VerseOne{1}这事以后,{\PN{波斯}}王{\PN{亚达薛西}}年间,有个{\PN{以斯拉}},他是{\PN{西莱雅}}的儿子,{\PN{西莱雅}}是{\PN{亚撒利雅}}的儿子,{\PN{亚撒利雅}}是{\PN{希勒家}}的儿子,
\VS{2}{\PN{希勒家}}是{\PN{沙龙}}的儿子,{\PN{沙龙}}是{\PN{撒督}}的儿子,{\PN{撒督}}是{\PN{亚希突}}的儿子,
\VS{3}{\PN{亚希突}}是{\PN{亚玛利雅}}的儿子,{\PN{亚玛利雅}}是{\PN{亚撒利雅}}的儿子,{\PN{亚撒利雅}}是{\PN{米拉约}}的儿子,
\VS{4}{\PN{米拉约}}是{\PN{西拉希雅}}的儿子,{\PN{西拉希雅}}是{\PN{乌西}}的儿子,{\PN{乌西}}是{\PN{布基}}的儿子,
\VS{5}{\PN{布基}}是{\PN{亚比书}}的儿子,{\PN{亚比书}}是{\PN{非尼哈}}的儿子,{\PN{非尼哈}}是{\PN{以利亚撒}}的儿子,{\PN{以利亚撒}}是大祭司{\PN{亚伦}}的儿子。
\VS{6}这{\PN{以斯拉}}从{\PN{巴比伦}}上来,他是敏捷的文士,通达耶和华—{\PN{以色列}} 神所赐{\PN{摩西}}的律法书。王允准他一切所求的,是因耶和华—他 神的手帮助他。
\par }{\PP \VS{7}{\PN{亚达薛西}}王第七年,{\PN{以色列}}人、祭司、{\PN{利未}}人、歌唱的、守门的、尼提宁,有上{\PN{耶路撒冷}}的。
\VS{8}王第七年五月,{\PN{以斯拉}}到了{\PN{耶路撒冷}}。
\VS{9}正月初一日,他从{\PN{巴比伦}}起程;因他 神施恩的手帮助他,五月初一日就到了{\PN{耶路撒冷}}。
\VS{10}{\PN{以斯拉}}定志考究遵行耶和华的律法,又将律例典章教训{\PN{以色列}}人。
\par }{\SH 亚达薛西王给以斯拉的文件
\par }{\PP \VS{11}祭司{\PN{以斯拉}}是通达耶和华诫命和赐{\PN{以色列}}之律例的文士。{\PN{亚达薛西}}王赐给他们谕旨,上面写着说:
\VS{12}「诸王之王{\PN{亚达薛西}},达于祭司{\PN{以斯拉}}通达天上 神律法大德的文士,云云。
\VS{13}住在我国中的{\PN{以色列}}人、祭司、{\PN{利未}}人,凡甘心上{\PN{耶路撒冷}}去的,我降旨准他们与你同去。
\VS{14}王与七个谋士既然差你去,照你手中 神的律法书察问{\PN{犹大}}和{\PN{耶路撒冷}}的景况;
\VS{15}又带金银,就是王和谋士甘心献给住{\PN{耶路撒冷}}、{\PN{以色列}} 神的,
\VS{16}并带你在{\PN{巴比伦}}全省所得的金银,和百姓、祭司乐意献给{\PN{耶路撒冷}}—他们 神殿的礼物。
\VS{17}所以你当用这金银,急速买公牛、公绵羊、绵羊羔,和同献的素祭奠祭之物,献在{\PN{耶路撒冷}}—你们 神殿的坛上。
\VS{18}剩下的金银,你和你的弟兄看着怎样好,就怎样用,总要遵着你们 神的旨意。
\VS{19}所交给你 神殿中使用的器皿,你要交在{\PN{耶路撒冷}} 神面前。
\VS{20}你 神殿里若再有需用的经费,你可以从王的府库里支取。
\par }{\PP \VS{21}「我—{\PN{亚达薛西}}王又降旨与{\PN{河西}}的一切库官,说:『通达天上 神律法的文士祭司{\PN{以斯拉}},无论向你们要什么,你们要速速地备办,
\VS{22}就是银子直到一百他连得,麦子一百柯珥,酒一百罢特,油一百罢特,盐不计其数,也要给他。
\VS{23}凡天上之 神所吩咐的,当为天上 神的殿详细办理。为何使忿怒临到王和王众子的国呢?
\VS{24}我又晓谕你们,至于祭司、{\PN{利未}}人、歌唱的、守门的,和尼提宁,并在 神殿当差的人,不可叫他们进贡,交课,纳税。』
\par }{\PP \VS{25}「{\PN{以斯拉}}啊,要照着你 神赐你的智慧,将所有明白你 神律法的人立为士师、审判官,治理{\PN{河西}}的百姓,使他们教训一切不明白 {\ADD{神}}律法的人。
\VS{26}凡不遵行你 神律法和王命令的人就当速速定他的罪,或治死,或充军,或抄家,或囚禁。」
\par }{\SH 以斯拉称颂耶和华
\par }{\PP \VS{27}{\PN{以斯拉}}说:「耶和华—我们列祖的 神是应当称颂的!因他使王起这心意修饰{\PN{耶路撒冷}}耶和华的殿,
\VS{28}又在王和谋士,并大能的军长面前施恩于我。因耶和华—我 神的手帮助我,我就得以坚强,从{\PN{以色列}}中招聚首领,与我一同上来。」

\par }\Chap{8}{\SH 从被掳之地回来的人
\par }{\PP \VerseOne{1}当{\PN{亚达薛西}}王年间,同我从{\PN{巴比伦}}上来的人,他们的族长和他们的家谱记在下面:
\VS{2}属{\PN{非尼哈}}的子孙有{\PN{革顺}};属{\PN{以他玛}}的子孙有{\PN{但以理}};属{\PN{大卫}}的子孙有{\PN{哈突}};
\VS{3}属{\PN{巴录}}的后裔,就是{\PN{示迦尼}}的子孙有{\PN{撒迦利亚}},同着他,按家谱计算,男丁一百五十人;
\VS{4}属{\PN{巴哈·摩押}}的子孙有{\PN{西拉希雅}}的儿子{\PN{以利约乃}},同着他有男丁二百;
\VS{5}属{\PN{示迦尼}}的子孙有{\PN{雅哈悉}}的儿子,同着他有男丁三百;
\VS{6}属{\PN{亚丁}}的子孙有{\PN{约拿单}}的儿子{\PN{以别}},同着他有男丁五十;
\VS{7}属{\PN{以拦}}的子孙有{\PN{亚他利雅}}的儿子{\PN{耶筛亚}},同着他有男丁七十;
\VS{8}属{\PN{示法提雅}}的子孙有{\PN{米迦勒}}的儿子{\PN{西巴第雅}},同着他有男丁八十;
\VS{9}属{\PN{约押}}的子孙有{\PN{耶歇}}的儿子{\PN{俄巴底亚}},同着他有男丁二百一十八;
\VS{10}属{\PN{示罗密}}的子孙有{\PN{约细斐}}的儿子,同着他有男丁一百六十;
\VS{11}属{\PN{比拜}}的子孙有{\PN{比拜}}的儿子{\PN{撒迦利亚}},同着他有男丁二十八;
\VS{12}属{\PN{押甲}}的子孙有{\PN{哈加坦}}的儿子{\PN{约哈难}},同着他有男丁一百一十;
\VS{13}属{\PN{亚多尼干}}的子孙,就是末尾的,他们的名字是{\PN{以利法列}}、{\PN{耶利}}、{\PN{示玛雅}},同着他们有男丁六十;
\VS{14}属{\PN{比革瓦伊}}的子孙有{\PN{乌太}}和{\PN{撒布}},同着他们有男丁七十。
\par }{\SH 以斯拉为圣殿招集利未人
\par }{\PP \VS{15}我招聚这些人在流入{\PN{亚哈瓦}}的河边,我们在那里住了三日。我查看百姓和祭司,见没有{\PN{利未}}人在那里,
\VS{16}就召首领{\PN{以利以谢}}、{\PN{亚列}}、{\PN{示玛雅}}、{\PN{以利拿单}}、{\PN{雅立}}、{\PN{以利拿单}}、{\PN{拿单}}、{\PN{撒迦利亚}}、{\PN{米书兰}},又召教习{\PN{约雅立}}和{\PN{以利拿单}}。
\VS{17}我打发他们往{\PN{迦西斐雅}}地方去见那里的首领{\PN{易多}},又告诉他们当向{\PN{易多}}和他的弟兄尼提宁说什么话,叫他们为我们 神的殿带使用的人来。
\VS{18}蒙我们 神施恩的手帮助我们,他们在{\PN{以色列}}的曾孙、{\PN{利未}}的孙子、{\PN{抹利}}的后裔中带一个通达人来;还有{\PN{示利比}}和他的众子与弟兄共一十八人。
\VS{19}又有{\PN{哈沙比雅}},同着他有{\PN{米拉利}}的子孙{\PN{耶筛亚}},并他的众子和弟兄共二十人。
\VS{20}从前{\PN{大卫}}和众首领派尼提宁服事{\PN{利未}}人,现在从这尼提宁中也带了二百二十人来,都是按名指定的。
\par }{\SH 以斯拉领百姓禁食祷告
\par }{\PP \VS{21}那时,我在{\PN{亚哈瓦河}}边宣告禁食,为要在我们 神面前克苦己心,求他使我们和{\ADD{妇人}}孩子,并一切所有的,都得平坦的道路。
\VS{22}我求王拨步兵马兵帮助我们抵挡路上的仇敌,本以为羞耻;因我曾对王说:「我们 神施恩的手必帮助一切寻求他的;但他的能力和忿怒必攻击一切离弃他的。」
\VS{23}所以我们禁食祈求我们的 神,他就应允了我们。
\par }{\SH 献给圣殿的礼物
\par }{\PP \VS{24}我分派祭司长十二人,就是{\PN{示利比}}、{\PN{哈沙比雅}},和他们的弟兄十人,
\VS{25}将王和谋士、军长,并在那里的{\PN{以色列}}众人为我们 神殿所献的金银和器皿,都秤了交给他们。
\VS{26}我秤了交在他们手中的银子有六百五十他连得;银器重一百他连得;金子一百他连得;
\VS{27}金碗二十个,重一千达利克;上等光铜的器皿两个,宝贵如金。
\VS{28}我对他们说:「你们归耶和华为圣,器皿也为圣;金银是甘心献给耶和华—你们列祖之 神的。
\VS{29}你们当警醒看守,直到你们在{\PN{耶路撒冷}}耶和华殿的库内,在祭司长和{\PN{利未}}族长,并{\PN{以色列}}的各族长面前过了秤。」
\VS{30}于是,祭司、{\PN{利未}}人按着分量接受金银和器皿,要带到{\PN{耶路撒冷}}我们 神的殿里。
\par }{\SH 返回耶路撒冷
\par }{\PP \VS{31}正月十二日,我们从{\PN{亚哈瓦河}}边起行,要往{\PN{耶路撒冷}}去。我们 神的手保佑我们,救我们脱离仇敌和路上埋伏之人的手。
\VS{32}我们到了{\PN{耶路撒冷}},在那里住了三日。
\VS{33}第四日,在我们 神的殿里把金银和器皿都秤了,交在祭司{\PN{乌利亚}}的儿子{\PN{米利末}}的手中。同着他有{\PN{非尼哈}}的儿子{\PN{以利亚撒}},还有{\PN{利未}}人{\PN{耶书亚}}的儿子{\PN{约撒拔}}和{\PN{宾内}}的儿子{\PN{挪亚底}}。
\VS{34}当时都点了数目,按着分量写在册上。
\par }{\PP \VS{35}从掳到之地归回的人向{\PN{以色列}}的 神献燔祭,就是为{\PN{以色列}}众人献公牛十二只,公绵羊九十六只,绵羊羔七十七只,又献公山羊十二只作赎罪祭,这都是向耶和华焚献的。
\VS{36}他们将王的谕旨交给王所派的总督与{\PN{河西}}的省长,他们就帮助百姓,又供给 神殿里所需用的。

\par }\Chap{9}{\SH 谴责跟异族通婚
\par }{\PP \VerseOne{1}这事做完了,众首领来见我,说:「{\PN{以色列}}民和祭司并{\PN{利未}}人,没有离绝{\PN{迦南}}人、{\PN{赫}}人、{\PN{比利洗}}人、{\PN{耶布斯}}人、{\PN{亚扪}}人、{\PN{摩押}}人、{\PN{埃及}}人、{\PN{亚摩利}}人,仍效法这些国的民,行可憎的事。
\VS{2}因他们为自己和儿子娶了这些外邦女子为妻,以致圣洁的种类和这些国的民混杂;而且首领和官长在这事上为罪魁。」
\VS{3}我一听见这事,就撕裂衣服和外袍,拔了头发和胡须,惊惧忧闷而坐。
\VS{4}凡为{\PN{以色列}} 神言语战兢的,都因这被掳归回之人所犯的罪聚集到我这里来。我就惊惧忧闷而坐,直到献晚祭的时候。
\par }{\PP \VS{5}献晚祭的时候我起来,心中愁苦,穿着撕裂的衣袍,双膝跪下向耶和华—我的 神举手,
\VS{6}说:「我的 神啊,我抱愧蒙羞,不敢向我 神仰面;因为我们的罪孽灭顶,我们的罪恶滔天。
\VS{7}从我们列祖直到今日,我们的罪恶甚重;因我们的罪孽,我们和君王、祭司都交在外邦列王的手中,杀害、掳掠、抢夺、脸上蒙羞正如今日的光景。
\VS{8}现在耶和华—我们的 神暂且施恩与我们,给我们留些逃脱的人,使我们{\ADD{安稳}}如钉子钉在他的圣所,我们的 神好光照我们的眼目,使我们在受辖制之中稍微复兴。
\VS{9}我们是奴仆,然而在受辖制之中,我们的 神仍没有丢弃我们,在{\PN{波斯}}王眼前向我们施恩,叫我们复兴,能重建我们 神的殿,修其毁坏之处,使我们在{\PN{犹大}}和{\PN{耶路撒冷}}有墙垣。
\par }{\PP \VS{10}「我们的 神啊,既是如此,我们还有什么话可说呢?因为我们已经离弃你的命令,
\VS{11}就是你借你仆人众先知所吩咐的说:『你们要去得为业之地是污秽之地;因列国之民的污秽和可憎的事,叫全地从这边直到那边满了污秽。
\VS{12}所以不可将你们的女儿嫁他们的儿子,也不可为你们的儿子娶他们的女儿,永不可求他们的平安和他们的利益,这样你们就可以强盛,吃这地的美物,并遗留这地给你们的子孙永远为业。』
\VS{13}神啊,我们因自己的恶行和大罪,遭遇了这一切的事,并且你刑罚我们轻于我们罪所当得的,又给我们留下这些人。
\VS{14}我们岂可再违背你的命令,与这行可憎之事的民结亲呢?{\ADD{若这样行}},你岂不向我们发怒,将我们灭绝,以致没有一个剩下逃脱的人吗?
\VS{15}耶和华—{\PN{以色列}}的 神啊,因你是公义的,我们这剩下的人才得逃脱,正如今日的光景。看哪,我们在你面前有罪恶,因此无人在你面前站立得住。」

\par }\Chap{10}{\SH 终止与异族通婚的措施
\par }{\PP \VerseOne{1}{\PN{以斯拉}}祷告,认罪,哭泣,俯伏在 神殿前的时候,有{\PN{以色列}}中的男女孩童聚集到{\PN{以斯拉}}那里,成了大会,众民无不痛哭。
\VS{2}属{\PN{以拦}}的子孙、{\PN{耶歇}}的儿子{\PN{示迦尼}}对{\PN{以斯拉}}说:「我们在此地娶了外邦女子为妻,干犯了我们的 神,然而{\PN{以色列}}人还有指望。
\VS{3}现在当与我们的 神立约,休这一切的妻,离绝她们所生的,照着我主和那因 神命令战兢之人所议定的,按律法而行。
\VS{4}你起来,这是你当办的事,我们必帮助你,你当奋勉而行。」
\VS{5}{\PN{以斯拉}}便起来,使祭司长和{\PN{利未}}人,并{\PN{以色列}}众人起誓说,必照这话去行;他们就起了誓。
\par }{\PP \VS{6}{\PN{以斯拉}}从 神殿前起来,进入{\PN{以利亚实}}的儿子{\PN{约哈难}}的屋里,到了那里不吃饭,也不喝水;因为被掳归回之人所犯的罪,心里悲伤。
\VS{7}他们通告{\PN{犹大}}和{\PN{耶路撒冷}}被掳归回的人,叫他们在{\PN{耶路撒冷}}聚集。
\VS{8}凡不遵首领和长老所议定、三日之内不来的,就必抄他的家,使他离开被掳归回之人的会。
\par }{\PP \VS{9}于是,{\PN{犹大}}和{\PN{便雅悯}}众人,三日之内都聚集在{\PN{耶路撒冷}}。那日正是九月二十日,众人都坐在 神殿前的宽阔处;因这事,又因下大雨,就都战兢。
\VS{10}祭司{\PN{以斯拉}}站起来,对他们说:「你们有罪了;因你们娶了外邦的女子为妻,增添{\PN{以色列}}人的罪恶。
\VS{11}现在当向耶和华—你们列祖的 神认罪,遵行他的旨意,离绝这些国的民和外邦的女子。」
\VS{12}会众都大声回答说:「我们必照着你的话行,
\VS{13}只是百姓众多,又逢大雨的时令,我们不能站在外头,这也不是一两天办完的事,因我们在这事上犯了大罪;
\VS{14}不如为全会众派首领办理。凡我们城邑中娶外邦女子为妻的,当按所定的日期,同着本城的长老和士师而来,直到办完这事, 神的烈怒就转离我们了。」
\VS{15}惟有{\PN{亚撒黑}}的儿子{\PN{约拿单}},{\PN{特瓦}}的儿子{\PN{雅哈谢}}阻挡\FTNT{}{{\FR 10:15: }或译:总办}这事,并有{\PN{米书兰}}和{\PN{利未}}人{\PN{沙比太}}帮助他们。
\par }{\PP \VS{16}被掳归回的人如此而行。祭司{\PN{以斯拉}}和些族长按着宗族都指名见派;在十月初一日,一同在座查办这事,
\VS{17}到正月初一日,才查清娶外邦女子的人数。
\par }{\SH 娶外邦女子的人
\par }{\PP \VS{18}在祭司中查出娶外邦女子为妻的,就是{\PN{耶书亚}}的子孙{\PN{约萨达}}的儿子,和他弟兄{\PN{玛西雅}}、{\PN{以利以谢}}、{\PN{雅立}}、{\PN{基大利}};
\VS{19}他们便应许必休他们的妻。他们因有罪,就{\ADD{献}}群中的一只公绵羊赎罪。
\VS{20}{\PN{音麦}}的子孙中,有{\PN{哈拿尼}}、{\PN{西巴第雅}}。
\VS{21}{\PN{哈琳}}的子孙中,有{\PN{玛西雅}}、{\PN{以利雅}}、{\PN{示玛雅}}、{\PN{耶歇}}、{\PN{乌西雅}}。
\VS{22}{\PN{巴施户珥}}的子孙中,有{\PN{以利约乃}}、{\PN{玛西雅}}、{\PN{以实玛利}}、{\PN{拿坦业}}、{\PN{约撒拔}}、{\PN{以利亚撒}}。
\par }{\PP \VS{23}{\PN{利未}}人中,有{\PN{约撒拔}}、{\PN{示每}}、{\PN{基拉雅}}({\PN{基拉雅}}就是{\PN{基利他}}),还有{\PN{毗他希雅}}、{\PN{犹大}}、{\PN{以利以谢}}。
\VS{24}歌唱的人中有{\PN{以利亚实}}。守门的人中,有{\PN{沙龙}}、{\PN{提联}}、{\PN{乌利}}。
\par }{\PP \VS{25}{\PN{以色列}}人{\PN{巴录}}的子孙中,有{\PN{拉米}}、{\PN{耶西雅}}、{\PN{玛基雅}}、{\PN{米雅民}}、{\PN{以利亚撒}}、{\PN{玛基雅}}、{\PN{比拿雅}}。
\VS{26}{\PN{以拦}}的子孙中,有{\PN{玛他尼}}、{\PN{撒迦利亚}}、{\PN{耶歇}}、{\PN{押底}}、{\PN{耶利末}}、{\PN{以利雅}}。
\VS{27}{\PN{萨土}}的子孙中,有{\PN{以利约乃}}、{\PN{以利亚实}}、{\PN{玛他尼}}、{\PN{耶利末}}、{\PN{撒拔}}、{\PN{亚西撒}}。
\VS{28}{\PN{比拜}}的子孙中,有{\PN{约哈难}}、{\PN{哈拿尼雅}}、{\PN{萨拜}}、{\PN{亚勒}}。
\VS{29}{\PN{巴尼}}的子孙中,有{\PN{米书兰}}、{\PN{玛鹿}}、{\PN{亚大雅}}、{\PN{雅述}}、{\PN{示押}}、{\PN{耶利末}}。
\VS{30}{\PN{巴哈·摩押}}的子孙中,有{\PN{阿底拿}}、{\PN{基拉}}、{\PN{比拿雅}}、{\PN{玛西雅}}、{\PN{玛他尼}}、{\PN{比撒列}}、{\PN{宾内}}、{\PN{玛拿西}}。
\VS{31}{\PN{哈琳}}的子孙中,有{\PN{以利以谢}}、{\PN{伊示雅}}、{\PN{玛基雅}}、{\PN{示玛雅}}、{\PN{西缅}}、
\VS{32}{\PN{便雅悯}}、{\PN{玛鹿}}、{\PN{示玛利雅}}。
\VS{33}{\PN{哈顺}}的子孙中,有{\PN{玛特乃}}、{\PN{玛达他}}、{\PN{撒拔}}、{\PN{以利法列}}、{\PN{耶利买}}、{\PN{玛拿西}}、{\PN{示每}}。
\VS{34}{\PN{巴尼}}的子孙中,有{\PN{玛玳}}、{\PN{暗兰}}、{\PN{乌益}}、
\VS{35}{\PN{比拿雅}}、{\PN{比底雅}}、{\PN{基禄}}、
\VS{36}{\PN{瓦尼雅}}、{\PN{米利末}}、{\PN{以利亚实}}、
\VS{37}{\PN{玛他尼}}、{\PN{玛特乃}}、{\PN{雅扫}}、
\VS{38}{\PN{巴尼}}、{\PN{宾内}}、{\PN{示每}}、
\VS{39}{\PN{示利米雅}}、{\PN{拿单}}、{\PN{亚大雅}}、
\VS{40}{\PN{玛拿底拜}}、{\PN{沙赛}}、{\PN{沙赖}}、
\VS{41}{\PN{亚萨利}}、{\PN{示利米雅}}、{\PN{示玛利雅}}、
\VS{42}{\PN{沙龙}}、{\PN{亚玛利雅}}、{\PN{约瑟}}。
\VS{43}{\PN{尼波}}的子孙中,有{\PN{耶利}}、{\PN{玛他提雅}}、{\PN{撒拔}}、{\PN{西比拿}}、{\PN{雅玳}}、{\PN{约珥}}、{\PN{比拿雅}}。
\VS{44}这些人都娶了外邦女子为妻,其中也有生了儿女的。
\par }